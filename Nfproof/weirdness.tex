\documentclass[12pt]{article}

\usepackage{amssymb}

\title{Foundations turned upside down}

\author{Randall Holmes}

\begin{document}

\maketitle

We outline a possible but very bizarre view of foundations suggested by TZT.

In this view, mathematical objects are classes and elements are called sets.  The identity conditions for classes are that they have the same elements.  Any property of sets
(elements) determines a class.  This is very familiar.  The question is, what do the sets look like (as in any theory of this kind).

Our answer is that the sets look rather like the classes.  An element of a set is called a 2-set.  The identity criteria for sets are inherited from those of classes, since sets are classes.
Any property of 2-sets determines a set.

This is iterated through the concrete natural numbers.  The identity criteria for $n$-sets are inherited from those for classes. The elements of $n$-sets are called $n+1$-sets.
Any property of $n+1$-sets determines an $n$-set.

We observe that this is clearly consistent.  In a model of ordinary set theory with a nonstandard natural number $N$, let $X$ be a transitive set and let the domain of
classes be ${\cal P}^N(X)$.  The domain of $n$-sets is then interpreted as ${\cal P}^{N-n}(X)$, and the axioms so far stated will hold.

The fun bit is seeing how mathematics is done in this framework, and discovering that the framework sees itself this way internally.  That it sees itself this way has a weird consequence:  this theory can define a natural number which is clearly not one of the usual ones.

The theory we consider is a first-order theory with equality and membership.  General objects are called {\em classes\/}.

\begin{description}

\item[Definition scheme:]  We say ${\tt set\/}(x)$ ($x$ is a set) for $(\exists y:x \in y)$.  We define ${\tt set}_1(x)$ as ${\tt set}(x)$ and, when we have defined
${\tt set}_n(x)$ we define ${\tt set}_{n+1}(x)$ as $(\exists y: {\tt set}_n(y) \wedge x \in y)$.  We say ``$x$ is an $n$-set" for ${\tt set}_n(x)$.

\item[Axiom of extensionality:]  $$(\forall xy:x=y \leftrightarrow (\forall z:z \in x \leftrightarrow z \in y):$$  the criteria of identity for classes is that they have the same elements.

\item[Axiom scheme of comprehension:]  For each formula $\phi$ in which the variable $A$ is not free, we have the axiom $$(\exists A:(\forall x:x \in A \leftrightarrow {\tt set}(x) \wedge \phi)):$$  the witness to this axiom is unique by extensionality and we denote it as $\{x \in V_1:\phi\}$.  Of course $V_1$ denotes $\{x \in V_1:x=x\}$, the class of sets.  More generally, we let
$V_n$ denote $\{x \in V_1:{\tt set}_n(x)\}$, the class of $n$-sets.  This axiom is intended to say that each property of sets uniquely determines a class.

\item[Definitions:] We define $x \subseteq y$ as $(\forall z:z \in x \rightarrow z \in y)$.  We define $\emptyset$ as $\{x \in V_1:x \not\in x\}$.  We define $\{x\}$ as $\{y \in V_1:y=x\}$ (of course, this only has the intended reference if $x \in V_1$).
We define $x \cup y$ as $\{z \in V_1:z \in z \vee z \in y\}$.   We define $x \cap y$ as $\{z:z \in x \wedge z \in y)$.  We define $x \setminus y$ as $\{z \in V_1:z \in x \wedge z \not\in y\}$.  We define $\{x_1,x_2\}$ as $\{x_1\} \cup \{x_2\}$,
and $\{x_1,x_1,\ldots,x_n\}$ as $\{x_1\} \cup \{x_2,\ldots,x_n\}$.  We define $\bigcup A$ as $\{x \in V_1:(\exists a \in A:x \in a)\}$.  We define $\bigcap A$ as $\{x \in V_1:(\forall a \in A:x \in a)\}$.

\item[Definitions:]  For any expression $F(x_1,\ldots,x_n)$, we define $$\{F(x_1,\ldots,x_n)\in V_1:\phi\}$$ as $$\{y \in V_1:(\exists x_1\ldots x_n:y = F(x_1,\ldots,x_n) \wedge \phi)\}.$$  For any term $t$ and relation symbol $R$ we define $\{t\,R\,A:\phi\}$ as $$\{t \in V_1:t \,R\,A \wedge \phi\}.$$  For any term $t$, we may abbreviate $\{t \in V_1:\phi\}$ as simply $\{t :\phi\}$.

\item[Axiom of union:]  $(\forall A:\bigcup A \in V_1)$.  An equivalent formulation is $$\{\forall x:(\forall y \in x:y \in V_2)\rightarrow y \in V_1)\}:$$  any class all of whose elements are 2-sets is a set.

\item[Axiom scheme of uniformity:]  For any formula $\phi$ in the language of equality and membership, we define $\phi^+$ as the result of restricting all quantifiers in $\phi$ to $V_1$.  Note that
${\tt set}_n^+$ is equivalent to ${\tt set}_{n+1}$.  The axiom scheme of uniformity asserts that if $\phi$ is an axiom (appearing before or after this scheme), so is $\phi^+$.

\item[Definitions:]  We define 0 as $\{\emptyset\}$.  Note that this is the class of all sets with 0 elements.  We define $\sigma(n)$ as $\{y \cup \{z\}:y \in n \wedge z \in V_2 \setminus y\}$ for each $n \in V_1$.  Notice that if $n$ is the class of all sets with $n$ elements, $\sigma(n)$ is the class of all sets with $n+1$ elements.  We define 1 as $\sigma(0)$, 2 as $\sigma(1)$, and so forth.  We define $[n]_i$ as $n \cap V_{i+1}$, for each concrete natural number $i$:  note that $n_i \in V_i$ will always hold, by union and uniformity.  We define $\mathbb N$ as $$\{n \in V_1:(\forall I:0 \in I \wedge (\forall  x:x \in I \rightarrow [\sigma(x)]_1 \in I) \rightarrow n \in I)\}.$$  Notice that as we want a set of natural numbers, we need its elements to be sets of 2-sets with each finite cardinality:  so a typical element of $\mathbb N$ is not
(say) 3, the class of all sets with 3 elements, but $[3]_1$, the set of all 2-sets with 3 (3-set) elements.  We define $\mathbb F$ as $\bigcup N$, the set of all finite 2-sets.

\item[Axiom of infinity:]  $V_3 \not\in \mathbb F$.  This is the most convenient way to say that the universe is infinite:  the 2-set $V_3$ is not an element of any natural number.

\item[Axiom of choice:]  We say that a class $P$ is a partition iff $(\forall xy \in P:x \neq y \rightarrow x \cap y = \emptyset)$.  The axiom of choice asserts
that for every partition $P$, there is a set $C \subseteq \bigcup P$ such that for every $A \in P$, $A \cap C \in 1$.

\end{description}

This theory is exactly as strong as Zermelo set theory.  But it expresses a rather different view of the world.  In fact, it is very strange.  We first invite the reader to check that
indeed the models ${\cal P}^N(X)$ described above are models of this theory, with the membership of the nonstandard ${\cal P}^N(X)$ interpreting the membership of this theory,
and so the domain of $n$-sets being interpreted as ${\cal P}^{N-n}(X)$.

We set out to use the tools of the theory to define the structure its world appears to have from the outset.  We define ${\cal P}(x)$, the power set of $x$, as $\{y \in V_1:y \subseteq x\}$, for any $x \in V_2$.  We define $\mathbb L$ as the collection of all $V$ which belong to
each class $I$ such that $V_2 \in I$ and $(\forall V \in I:V = {\cal P}(\bigcup V) \rightarrow \bigcup V \in I)$.   Notice that for each $V_n$ ($n \geq 2$), we have $V_{n+1} = \bigcup V_n$ and
${\cal P}(V_{n+1}) = V_n$, so we might suppose that ${\mathbb L} = \{V_n:n \geq 2\}$.  But the truth is stranger than that.

Let $X = \bigcap L$, the class of all objects which belong to each element of $\mathbb L$.  Now consider $R = \{x \in X:x \not\in x\}$.  For standard reasons, this cannot be an element of $X$, so there must be $V \in \mathbb L$ such that $R \not\in V$.  We argue that $V = {\cal P}(\bigcup V)$ cannot hold.  If this were true, than $\bigcup V \in \mathbb L$ would hold, 
from which it would follow that $R \subseteq X \subseteq \bigcup V$ would hold, whence $R \in {\cal P}(\bigcup V) = V$ would hold, which is a contradiction.

We thus see that we have an element $V$ of $\mathbb L$ which cannot be one of the familiar $V_n$'s.  However, we can define the notation $V_n$ for all $n \in \mathbb N$:
consider the class of pairs which is the intersection of all classes which contain $\{[1]_2,V_3\}$ and for every $x$ contain $\{[\sigma(n)]_2,\bigcup x\}$ if they contain $\{n,x\}$ and $x = {\cal P}(\bigcup x)$.
The set of elements of such pairs which belong to $\mathbb L$ is clearly $\mathbb L \setminus \{V_2\}$.  We can thus define the notation $V_n$ for $n \in \mathbb N$:
for $n=1,2$ we define it as $V_1,V_2$ respectively, and for $n \geq 3$ $V_n$ is the unique $V$ such that $\{[n]_2,V\}$ belongs to the class described above.  But then there is
a natural number $N$ such that $R \not\in V_N$ and so $V_{N+1}$ is undefined.  This $N$, definable in our theory, is not one of the familiar natural numbers.

So any model of our theory in fact sees itself as of the form ${\cal P}^N(X)$, where $N$ is a nonstandard natural number and $X$ is an infinite transitive set which is not a power set.
We note that it is actually evident that $X= \bigcap \mathbb L$ coincides with the level $V$ which does not contain $R =\{x \in X:x \not\in x\}$, simply because the elements of
$\mathbb L$ are nested.  It is not necessarily the case that $R$ coincides with $X$:  we have not assumed foundation, and it should be evident that if there were a Quine atom
$x = \{x\}$, it would belong to $X \setminus R$.

There is more to be said about convenient methods of how to do mathematics in this framework.  But what I have said so far is enough to reveal that it is perfectly mad!  Yet it is serviceable.

Some hints at the further development.  The natural numbers as defined above are a proper class, which is not very convenient.  Of course $\{[n]_i:n \in \mathbb N\}$ is a representation of the natural numbers belonging to $V_{i-1}$ for each $i$:  the obvious implementation is $\mathbb N_{\tt fixed} =\{[n]_{N-3}:n \in \mathbb N\}$ where $N$ is the nonstandard index defined above.  We use $N-3$ because we want the level we are using to be a  double power set:  note that $[n]_i=n \cap V_{i+1}$ is defined only for $i \leq N-1$. Once we have given this formulation of $\mathbb N$, we can then define the Quine pair, and Quine pairs of objects in any concrete $V_n$ will belong to $V_n$.

More generally, we can define $|A|$ for any set $A$ as $[\{B:B \sim A\}]_i$ for the largest natural number $i$ for which this set is nonempty (this is a little reminiscent of the Scott definition of cardinals though our notion of rank is coarser).  Other definitions by isomorphism class can be handled similarly.


\end{document}