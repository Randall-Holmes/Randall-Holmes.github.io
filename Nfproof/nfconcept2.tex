\documentclass[12pt]{article}

\usepackage{amssymb}

\usepackage{comment}

\title{Another idea about tangled type theory}

\author{Randall Holmes}

\begin{document}

\maketitle

\tableofcontents

\newpage

\section{Development of relevant theories}

\subsection{The simple theory of types TST and TSTU}

We introduce a theory which we call the simple typed theory of sets or TST, a name favored by the school of Belgian logicians who studied NF ({\em th\'eorie simple de types}).  This is not the same as the simple type theory of Ramsey and it is most certainly not Russell's type theory  (see historical remarks below).

TST is a first order multi-sorted theory with sorts (types) indexed by the nonnegative integers.  The primitive predicates of TST are equality and membership.

The type of a variable $x$ is written ${\tt type}($`$x$'$)$:  this will be a nonnegative integer.   A countably infinite supply of variables of each type is supposed.  An atomic equality sentence `$x=y$' is well-formed iff ${\tt type}($`$x$'$)={\tt type}($`$y$'$)$.
An atomic membership sentence `$x \in y$' is well-formed iff ${\tt type}$`$(x$'$)+1 = {\tt type}($`$y$'$)$.

The axioms of TST are extensionality axioms and comprehension axioms.

The extensionality axioms are all the well-formed assertions of the shape $(\forall xy:x=y \leftrightarrow (\forall z:z \in x \leftrightarrow z\in y))$.  For this to be well typed, the variables
$x$ and $y$ must be of the same type, one type higher than the type of $z$.

The comprehension axioms are all the well-formed assertions of the shape $(\exists A:(\forall x:x \in A \leftrightarrow \phi))$, where $\phi$ is any formula in which $A$ does not occur free.

The witness to $(\exists A:(\forall x:x \in A \leftrightarrow \phi))$ is unique by extensionality, and we introduce the notation $\{x:\phi\}$ for this object.  Of course, $\{x:\phi\}$  is to be assigned type one higher than that of $x$;  in general, term constructions will have types as variables do.

The modification which gives TSTU (the simple type theory of sets with urelements) replaces the extensionality axioms with the formulas of the shape $$(\forall xyw:w \in x \rightarrow (x=y \leftrightarrow (\forall z:z \in x \leftrightarrow z\in y))),$$  allowing many objects with no elements (called atoms or urelements)  in each positive type.  A technically useful refinement adds a constant $\emptyset^i$ of each positive type $i$ with no elements:  we can then address the problem that $\{x^i:\phi\}$ is not uniquely defined when $\phi$ is uniformly false by defining $\{x^i:\phi\}$ as $\emptyset^{i+1}$ in this case.

\subsubsection{Typical ambiguity}

TST exhibits a symmetry which is important in the sequel.

Provide a bijection $(x \mapsto x^+)$ from variables to variables of positive type satisfying   ${\tt type}(x^+)$ = ${\tt type}(x)+1$.

If $\phi$ is a formula, define $\phi^+$ as the result of replacing every variable $x$ (free and bound) in $\phi$ with $x^+$.  It should be evident that if $\phi$ is well-formed, so is $\phi^+$,
and that if $\phi$ is a theorem, so is $\phi^+$ (the converse is not the case).  Further, if we define a mathematical object as a set abstract $\{x:\phi\}$ we have an analogous
object $\{x^+:\phi^+\}$ of the next higher type (this process can be iterated).

The axiom scheme asserting $\phi \leftrightarrow \phi^+$ for each closed formula $\phi$ is called the Ambiguity Scheme.   Notice that this is a stronger assertion than is warranted by the symmetry of proofs described above.

\subsubsection{Historical remarks}

TST is not the type theory of the {\em Principia Mathematica\/} of Russell and Whitehead, though a description of TST is a common careless description of Russell's theory of types.

Russell described something like TST informally in his 1904 {\em Principles of Mathematics\/}.  The obstruction to giving such an account in {\em Principia Mathematica\/} was that
Russell and Whitehead did not know how to describe ordered pairs as sets.  As a result, the system of {\em Principia Mathematica\/} has an elaborate system of  complex
types inhabited by $n$-ary relations with arguments of specified previously defined types, further complicated by predicativity restrictions (which are cancelled by an axiom of reducibility).
The simple theory of types of Ramsey eliminates the predicativity restrictions and the axiom of reducibility, but is still a theory with complex types inhabited by $n$-ary relations.

Russell noticed a phenomenon like the typical ambiguity of TST in the more complex system of {\em Principia Mathematica\/}, which he refers to as ``systematic ambiguity".

In 1914, Norbert Wiener gave a definition of the ordered pair as a set (not the one now in use) and seems to have recognized that the type theory of {\em Principia Mathematica\/} could be simplified to something like TST, but he did not give a formal description.  The theory we call TST was apparently first described by Tarski in 1930.

It is worth observing that the axioms of TST look exactly like those of ``naive set theory", the restriction preventing paradox being embodied in the restriction of the language by the type system.
For example, the Russell paradox is averted because one cannot have $\{x:x \not\in x\}$ because $x \in x$ (and so its negation $\neg x \in x$) cannot be a well-formed formula.

It was shown around 1950 that Zermelo set theory proves the consistency of TST with the axiom of infinity;  TST + Infinity has the same consistency strength as
Zermelo set theory with separation restricted to bounded formulas.


\newpage

\subsection{Some mathematics in TST;  the theories TST$_n$ and their natural models}

We briefly discuss some mathematics in TST.

We indicate how to define the natural numbers.  We use the definition of Frege ($n$ is the set of all sets with $n$ elements).  0 is $\{\emptyset\}$ (notice that we get a natural number 0 in each type $i+2$;  we will be deliberately ambiguous in this discussion, but we are aware that anything we define is actually not unique, but reduplicated in each type above the lowest one in which it can be defined).  For any set $A$ at all we define $\sigma(A)$ as $\{a \cup \{x\}:a \in A \wedge x \not\in a\}$.  This is definable for any $A$ of type $i+2$ ($a$ being of type $i+1$ and $x$ of type $i$).  Define 1 as $\sigma(0)$, 2 as $\sigma(1)$,  3 as $\sigma(2)$, and so forth.  Clearly we have successfully defined 3 as the set of all sets with three elements, without circularity.
But further, we can define $\mathbb N$ as $\{n:(\forall I:0 \in I \wedge (\forall x \in I:\sigma(x) \in I) \rightarrow n \in I\}$, that is, as the intersection of all inductive sets.
$\mathbb N$ is again a typically ambiguous notation:  there is an object defined in this way in each type $i+3$.

The collection of all finite sets can be defined as $\bigcup \mathbb N$.  The axiom of infinity can be stated as $V \not\in \bigcup \mathbb N$ (where $V= \{x:x=x\}$ is the typically ambiguous symbol for the type $i+1$ set of all type $i$ objects).  It is straightforward to show that the natural numbers in each type of a model of TST with Infinity are isomorphic in a way representable in the theory.

Ordered pairs can be defined following Kuratowski and a quite standard theory of functions and relations can be developed.  Cardinal and ordinal numbers can be defined as Frege or Russell would have defined them, as isomorphism classes of sets under equinumerousness and isomorphism classes of well-orderings under similarity.  

The Kuratowski pair $(x,y) = \{\{x\},\{x,y\}\}$ is of course two types higher than its projections, which must be of the same type.  There is an alternative definition (due to Quine) of an ordered pair
$\left< x,y\right>$ in TST + Infinity which is of the same type as its projections $x,y$.  This is a considerable technical convenience but we will not need to define it here.  Note for example that if we use the Kuratowski pair the cartesian product $A \times B$ is two types higher than $A,B$, so we cannot define $|A| \cdot |B|$ as $|A \times B|$ if we want multiplication of cardinals to be a sensible operation.  Let $\iota$ be the singleton operation and define $T(|A|)$ as $|\iota``A|$ (this is a very useful operation sending cardinals of a given type to cardinals in the next higher type which seem intuitively to be the same).  The definition of cardinal multiplication if we use the Kuratowski pair is then $|A| \cdot |B| =T^{-2}(|A\times B|)$.  If we use the Quine pair this becomes the usual definition $|A| \cdot |B| =|A\times B|$.  Use of the Quine pair simplifies matters in this case, but it should be noted that the T operation remains quite important (for example it provides the internally representable isomorphism between the systems of natural numbers in each sufficiently high type).

Note that the form of Cantor's Theorem in TST is not $|A| < |{\cal P}(A)|$, which would be ill-typed, but $|\iota``A|<|{\cal P}(A)|$:  a set has fewer unit subsets than subsets.  The exponential map $\exp(|A|) = 2^{|A|}$ is not defined as $|{\cal P}(A)||$, which would be one type too high, but as $T^{-1}(|{\cal P}(A))$, the cardinality of a set $X$ such that $|\iota``X| = |{\cal P}(A)|$;   notice that this is partial.  For example
$2^{|V|}$ is not defined (where $V=\{x:x=x\}$, an entire type), because there is no $X$ with $|\iota``X|=|{\cal P}(V)|$, because $|\iota``V|<|{\cal P}(V)| \leq |V|$, and of course there is no set larger than $V$ in its type.

For each natural number $n$, the theory TST$_n$ is defined as the subtheory of TST with vocabulary restricted to use variables only of types less than $n$ (TST with $n$ types).
In ordinary set theory TST and each theory TST$_n$ have natural models, in which type 0 is implemented as a set $X$ and each type $i$ in use is implemented as ${\cal P}^i(X)$.  It should be clear that each TST$_n$ has natural models in bounded Zermelo set theory, and TST has natural models in a modestly stronger fragment of ZFC.

Further, each TST$_n$ has natural models in TST itself, though some care must be exercised in defining them.  Let $X$ be a set.  Implement type $i$ for each $i<n$ as
$\iota^{(n-1)-i}``{\cal P}^i(X)$.  If $X$ is in type $j$, each of the types of this interpretation of TST$_n$ is a set in the same type $j+n-1$.  For any relation $R$, define
$R^{\iota}$ as $\{(\{x\},\{y\}):x R y\}$.  The membership relation of type $i-1$ in type $i$ in the interpretation described is the restriction of $\subseteq^{\iota^{(n-1)-i}}$ to
the product of the sets implementing type $i-1$ and type $i$.

Notice then that we can define truth for formulas in these natural models of TST$_n$ for each $n$ in TST, though not in a uniform way which would allow us to define truth for formulas
in TST in TST.

Further, both in ordinary set theory and in TST, observe that truth of sentences in models of TST$_n$ is completely determined by the cardinality of the set used as type 0.
since two natural models of TST or TST$_n$ with base types implemented by sets of the same cardinality are clearly isomorphic. 

\newpage

\subsection{New Foundations and NFU}

In 1937, Willard van Orman Quine proposed a set theory motivated by the typical ambiguity of TST described above.  The paper in which he did this was titled ``New foundations for mathematical logic", and the set theory it introduces is called ``New Foundations" or NF, after the title of the paper.

Quine's observation is that since any theorem $\phi$ of TST is accompanied by theorems $\phi^+, \phi^{++}, \phi^{+++}, \ldots$ and every defined object $\{x:\phi\}$ is accompanied by
$\{x^+:\phi^+\},\{x^{++}:\phi^{++}\},\{x^{+++}:\phi^{+++}\}$, so the picture of what we can prove and construct in TST looks rather like a hall of mirrors, we might reasonably suppose that the types are all the same.

The concrete implementation follows.  NF is the first order unsorted theory with equality and membership as primitive with an axiom of extensionality $(\forall xy:x=y \leftrightarrow (\forall z:z \in x \leftrightarrow z\in y))$ and an axiom of comprehension $(\exists A:(\forall x:x \in A \leftrightarrow \phi))$ for each formula $\phi$ in which $A$ is not free which can be obtained from a formula of TST by dropping all distinctions of type.  We give a precise formalization of this idea:  provide a bijective map $(x \mapsto x^*)$ from the countable supply of variables (of all types) of TST onto the countable supply of variables of the language of NF.  Where $\phi$ is a formula of the language of TST, let $\phi^*$ be the formula obtained by replacing every veriable $x$, free and bound,
in $\phi$ with $x^*$. For each formula $\phi$ of the language of TST in which $A$ is not free in $\phi^*$, an axiom of comprehension of NF asserts $(\exists A:(\forall x:x \in A \leftrightarrow \phi^*))$.

In the original paper, this is expressed in a way which avoids explicit dependence on the language of another theory.  Let $\phi$ be a formula of the language of
NF.  A function $\sigma$ is a stratification of $\phi$ if it is a (possibly partial) map from variables to non-negative integers such that for each atomic subformula
`$x=y$'  of $\phi$ we have $\sigma($`$x$'$)=\sigma($`$y$'$)$ and for each atomic subformula `$x \in y$' of $\phi$ we have $\sigma($`$x$'$)+1 = \sigma($`$y$'$)$.
A formula $\phi$ is said to be stratified iff there is a stratification of $\phi$.  Then for each stratified formula $\phi$ of the language of NF we have an axiom $(\exists A:(\forall x:x \in A \leftrightarrow \phi))$.  The stratified formulas are exactly the formulas $\phi^*$ up to renaming of variables.

NF has been dismissed as a ``syntactical trick" because of the way it is defined.  It might go some way toward dispelling this impression to note that the stratified comprehension scheme is equivalent to a finite collection of its instances, so the theory can be presented in a way which makes no reference to types at all.  This is a result of Hailperin, refined by others.  One obtains a finite axiomatization of NF by analogy with the method of finitely axiomating von Neumann-G\"odel-Bernays predicate class theory.  It should further be noted that the first thing one does with the finite axiomatization is prove stratified comprehension as a meta-theorem, in practice, but it remains significant that the theory can be axiomatized with no reference to types at all.

For each stratified formula $\phi$, there is a unique witness to $$(\exists A:(\forall x:x \in A \leftrightarrow \phi))$$ (uniqueness follows by extensionality) whch we denote by $\{x:\phi\}$.

Jensen in 1969 proposed the theory NFU which replaces the extensionality axiom of NF with $$(\forall xyw:w \in x \rightarrow (x=y \leftrightarrow (\forall z:z \in x \leftrightarrow z\in y))),$$  allowing many atoms or urelements.  One can reasonably add an elementless constant $\emptyset$, and define $\{x:\phi\}$ as $\emptyset$ when $\phi$ is false for all $x$.

Jensen showed that NFU is consistent and moreoever NFU + Infinity + Choice is consistent.  We will give an argument similar in spirit though not the same in detail for the consistency of NFU in the next section.

An important theorem of Specker (1962) is that NF is consistent if and only if TST + the Ambiguity Scheme is consistent.  His method of proof adapts to show that  NFU is consistent if and only if TSTU + the Ambiguity Scheme is consistent.  Jensen used this fact in his proof of the consistency of NFU.  We indicate a proof of Specker's result using concepts from this paper below.

In 1954, Specker had shown that NF disproves Choice, and so proves Infinity.  At this point if not before it was clear that there is a serious issue of showing that NF is consistent relative to some set theory in which we have confidence.  There is no evidence that NF is any stronger than TST + Infinity, the lower bound established by Specker's result.

Note that NF or NFU supports the implementation of mathematics in the same style as TST, but with the representations of mathematical concepts losing their ambiguous character.  The number 3 really is realized as the unique set of all sets with three elements, for example.  The universe is a set and sets make up a Boolean algebra.   Cardinal and ordinal numbers can be defined
in the manner of Russell and Whitehead.

The apparent vulnerability to the paradox of Cantor is an illusion.  Applying Cantor's theorem to the cardinality of the universe in NFU gives $|\iota``V| < |{\cal }(V)| \leq |V|$ (the last inequality would be an equation in NF), from which we conclude that there are fewer singletons of objects than objects in the universe.  The operation $(x \mapsto \{x\})$ is not a set function, and there is every reason to expect it not to be, as its definition is unstratified.  The resolution of the Burali-Forti paradox is also weird and wonderful in NF(U), but would take us too far afield.

\newpage

\subsection{Tangled type theory TTT and TTTU}

In 1995, this author described a reduction of the NF consistency problem to consistency of a typed theory,  motivated by reverse engineering from Jensen's method of proving the consistency of NFU.

Let $\lambda$ be a limit ordinal.  It can be $\omega$ but it does not have to be.  

In the theory TTT (tangled type theory) which we develop, each variable $x$ is supplied with a type ${\tt type}($`$x$'$) <\lambda$;  we are provided with countably many distinct variables of each type.

For any formula $\phi$ of the language of TST and any strictly increasing sequence $s$ in $\lambda$, let $\phi^s$ be the formula obtained by replacing each variable
of type $i$ with a variable of type $s(i)$.  To make this work rigorously, we suppose that we have a bijection from type $i$ variables of the language of TST to type $\alpha$ variables
of the language of TTT for each natural number $i$ and ordinal $\alpha<\lambda$.

TTT is then the first order theory with types indexed by the ordinals below $\lambda$ whose well formed atomic sentences `$x=y$' have ${\tt type}($`$x$'$) = {\tt type}($`$y$'$)$ and whose atomic sentences `$x \in y$' satisfy ${\tt type}($`$x$'$) < {\tt type}($`$y$'$)$, and whose axioms are the sentences $\phi^s$ for each axiom $\phi$ of TST and each strictly increasing sequence $s$ in $\lambda$.  TTTU has the same relation to TSTU (with the addition of constants $\emptyset^{\alpha,\beta}$ for each $\alpha<\beta<\lambda$  such that $(\forall {\bf x}_0^{\alpha} :{\bf x}_0^{\alpha}\not\in \emptyset^{\alpha,\beta})$ is an axiom).

It is important to notice how weird a theory TTT is.  This is not cumulative type theory.  Each type $\beta$ is being interpreted as a power set of {\em each\/} lower type $\alpha$.  Cantor's theorem in the metatheory makes it clear that most of these power set interpretations cannot be honest.

There is now a striking

\begin{description}

\item[Theorem (Holmes):]  TTT(U) is consistent iff NF(U) is consistent.

\item[Proof:]  Suppose NF(U) is consistent.  Let $(M,E)$ be a model of NF(U) (a set $M$ with a membership relation $E$).  Implement type $\alpha$ as $M \times \{\alpha\}$ for
each $\alpha<\lambda$.  Define $E_{\alpha,\beta}$ for $\alpha<\beta$ as $\{((x,\alpha),(y,\beta)):xEy\}$.  This gives a model of TTT(U).   Empty sets in TTTU present no essential additional difficulties.

Suppose TTT(U) is consistent, and so we can assume we are working with a fixed model of TTT(U).  Let $\Sigma$ be a finite set of sentences in the language of TST(U).  Let $n$ be the smallest type such that no type $n$ variable occurs in any sentence in $\Sigma$.  We define a partition of the $n$-element subsets of $\lambda$.  Each $A \in [\lambda]^n$ is put in a compartment
determined by the truth values of the sentences $\phi^s$ in our model of TTT(U), where $\phi \in \Sigma$ and ${\tt rng}(s \lceil \{0,\ldots,n-1\}) = A$.  By Ramsey's theorem, there is a homogeneous set $H \subseteq \lambda$ for this partition, which includes the range of a strictly increasing sequence $h$.  There is a complete extension of TST(U) which includes
$\phi$ iff the theory of our model of TTT(U) includes $\phi^h$.  This extension satisfies $\phi \leftrightarrow \phi^+$ for each $\phi \in \Sigma$.  But this implies by compactness that the full Ambiguity Scheme $\phi \leftrightarrow \phi^+$ is consistent with TST(U), and so that NF(U) is consistent by the 1962 result of Specker.

We note that we can give a treatment of the result of Specker (rather different from Specker's own) using TTT(U).  Note that it is easy to see that if we have a model of TST(U) augmented with a Hilbert symbol (a primitive term construction $(\epsilon x:\phi)$ (same type as $x$) with axiom scheme $\phi[(\epsilon x:\phi)/x] \leftrightarrow (\exists x:\phi)$) which cannot appear in instances of comprehension (the quantifiers are not defined in terms of the Hilbert symbol, because they do need to appear in instances of comprehension) and Ambiguity (for all formulas, including those which mention the Hilbert symbol) then we can readily get a model of NF, by constructing a term model using the Hilbert symbol in the natural way, then identifying all terms with their type-raised versions.  All statements in the resulting type-free theory can be decided by raising types far enough (the truth value of an atomic sentence $(\epsilon x:\phi) \,R\, (\epsilon y:\psi)$ in the model of NF is determined by raising the type of both sides until the formula is well-typed in TST and reading the truth value of the type raised version;  $R$ is either = or $\in$).  Now observe that a model of TTT(U) can readily be equipped with a Hilbert symbol if this creates no obligation to add instances of comprehension
containing the Hilbert symbol (use a well-ordering of the set implementing each type to interpret a Hilbert symbol  $(\epsilon x:\phi)$ in that type as the first $x$ such that $\phi$), and the argument above for consistency of TST(U) plus Ambiguity with the Hilbert symbol goes through.

\item[Theorem (essentially due to Jensen):]  NFU is consistent.

\item[Proof:]  It is enough to exhibit a model of TTTU.  Suppose $\lambda>\omega$.  Represent type $\alpha$ as $V_{\omega+\alpha} \times \{\alpha\}$ for each $\alpha<\lambda$ ($V_{\omega+\alpha}$ being a rank of the usual cumulative hierarchy).  Define $\in_{\alpha,\beta}$ for
$\alpha<\beta<\lambda$ as $$\{((x,\alpha),(y,\beta)):x \in V_{\omega+\alpha} \wedge y \in V_{\omega+\alpha+1} \wedge x \in y\}.$$  This gives a model of TTTU in which the membership of
type $\alpha$ in type $\beta$ interprets each $(y,\beta)$ with $y \in V_{\omega+\beta} \setminus V_{\omega+\alpha+1}$ as an urelement.

Our use of $V_{\omega+\alpha}$ enforces Infinity in the resulting models of NFU (note that we did not have to do this:  if we set $\lambda=\omega$ and interpret type $\alpha$ using $V_\alpha$ we prove the consistency of NFU with the negation of Infinity).  It should be clear that Choice holds in the models of NFU eventually obtained if it holds in the ambient set theory.

This shows in fact that mathematics in NFU is quite ordinary (with respect to stratified sentences), because mathematics in the models of TSTU embedded in the indicated model of TTTU is quite ordinary.  The notorious ways in which NF evades the paradoxes of Russell, Cantor and Burali-Forti can be examined in actual models and we can see how they work (since they work in NFU in the same way they work in NF).

\end{description}

Of course Jensen did not phrase his argument in terms of tangled type theory.  Our contribution here was to reverse engineer from Jensen's original argument for the consistency of NFU an argument for the consistency of NF itself, which requires additional input which we did not know how to supply (a proof of the consistency of TTT itself).  An intuitive way to say what is happening here is that Jensen noticed that it is possible to skip types in a certain sense in TSTU in a way which is not obviously possible in TST itself;  to suppose that TTT might be consistent is to suppose that such type skipping is also possible in TST.

\subsubsection{How internal type representations unfold in TTT}

We have seen above that TST can internally represent TST$_n$.   An attempt to represent types of TTT internally to TTT has stranger results.

In TST the strategy for representing type $i$ in type $n\geq i$  is to use the $n-i$-iterated singleton of any type $i$ object $x$ to represent $x$;  then membership of representations of type $i-1$ objects in type
$i$ objects is represented by the relation on $n-i$-iterated singletons induced by the subset relation and with domain restricted to $n-(i+1)$-fold singletons.  This is described more formally above.

In TTT the complication is that there are numerous ways to embed type $\alpha$ into type $\beta$ for $\alpha<\beta$ along the lines just suggested.    We define a generalized
iterated singleton operation:  where $A$ is a finite subset of $\lambda$, $\iota_A$ is an operation defined on objects of type ${\tt min}(A)$.  $\iota_{\{\alpha\}}(x)=x$.
If $A$ has $\alpha<\beta$ as its two smallest elements, $\iota_A(x)$ is  $\iota_{A_1}(\iota_{\alpha,\beta}(x))$, where $A_1$ is defined as $A \setminus \{{\tt min}(A)\}$ (a notation we will continue to use) and $\iota_{\alpha,\beta}(x)$ is the unique type $\beta$ object whose only type $\alpha$ element is $x$.

Now for any nonempty finite $A \subseteq \lambda$ with minimum $\alpha$ and maximum $\beta$. the range of $\iota_A$ is a set, and a representation of type $\alpha$ in
type $\beta$.  For simplicity we carry out further analysis in types $\beta, \beta+1,\beta+2\ldots$ though it could be done in more general increasing sequences.  Use the notation
$\tau_A$ for the range of $\iota_A$, for each set $A$ with $\beta$ as its maximum.  Each such set has a cardinal $|\tau_A|$ in type $\beta+2$.  It is a straightforward
argument in the version of TST with types taken from $A$ and a small finite number of types $\beta+i$ that $2^{|\tau_A|} = |\tau_{A_1}|$ for each $A$ with at least two elements.
The relevant theorem in TST is that $2^{|\iota^{n+1}``X|} = \iota^n``X$, relabelled with suitable types from $\lambda$.   We use the notation $\exp(\kappa)$ for $2^\kappa$ to support iteration.  Notice that for any $\tau_A$ we have $\exp^{|A|-1}(|\tau_A|) = |\tau_{\{\beta\}}|$, the cardinality of type $\beta$.  Now if $A$ and $A'$ have the same minimum $\alpha$ and maximum $\beta$ 
but are of different sizes, we see that $|\tau_A| \neq |\tau_{A'}|$, since one has its $|A|-1$-iterated exponential equal to $|\tau_{\{\beta\}}|$ and the other has its $|A'|-1$-iterated exponential equal to $|\tau_{\{\beta\}}|$.  This is odd because there is an obvious external bijection between the sets $\tau_A$ and $\tau_{A'}$:  we see that this external bijection cannot be realized as a set.  $\tau_A$ and $\tau_{A'}$ are representations of the same type, but this is not obvious from inside TTT.  We recall that we denote $A \setminus \{{\tt min}(A)\}$ by $A_1$;  we further denote $(A_i)_1$ as $A_{i+1}$.  Now suppose that $A$ and $B$ both have maximum $\beta$ and $A \setminus A_i = B \setminus B_i$, where $i<|A| \leq |B|$.
We observe that for any concrete sentence $\phi$  in the language of TST$_i$, the truth value of $\phi$ in natural models with base type of sizes $|\tau_A|$ and $|\tau_B|$ will be the same, because the truth values we read off are the truth values in the model of TTT of versions of $\phi$ in exactly the same types of the model (truth values of $\phi^s$ for
any $s$ having $A \setminus A_i = B\setminus B_i$ as the range of an initial segment).  This much information telling us that $\tau_{A_j}$ and $\tau_{B_j}$ for $j<i$ are representations of the same type  is visible to us internally, though the external isomorphism is not.  We can conclude that the full first-order theories of natural models of TST$_i$ with base types $|\tau_A|$ and $|\tau_B|$ are
the same as seen inside the model of TTT, if we assume that the natural numbers of our model of TTT are standard.

\newpage

\section{Construction of a model of tangled type theory}

\subsection{Introduction to the model section}

I believe I finally have a sensible and naturally motivated description of a model of tangled type theory.  I will try to combine the technical details with the motivation in this version.  The construction will be carried out in ZFAC  with a large set of atoms. 

\subsection{Cardinal parameters and related concepts}

Let $\lambda$ be a limit ordinal.  Lower case Greek letters other than $\kappa$ and $\lambda$  will refer to ordinals less than $\lambda$.

For any nonempty subset $A$ of $\lambda$, we define $A_1$ as $A \setminus \{{\tt min}(A)\}$.  We define $A_0$ as $A$ and define $A_{i+1}$ as $(A_i)_1$ if $A$ has more than $i+1$ elements. We at least attempt to reserve lightface  upper case Latin letters early in the alphabet for nonempty subsets of $\lambda$.

Let $\kappa>\lambda$ be a regular uncountable cardinal.  We refer to sets of cardinality $<\kappa$ as small and all other sets as large.

Let $\mu>\kappa$ be the cardinality of the set of atoms.  We stipulate that $\mu$ is a strong limit cardinal of cofinality at least $\kappa$.

\subsection{Atoms, litters, and local cardinals}

Let {\bf A} be the set of atoms (which is of cardinality $\mu$).

We provide a partition of the set ${\bf A}$ of atoms into sets of size $\kappa$ called litters.

A set of atoms with small symmetric difference from a litter is called a near-litter.  

For any near-litter $N$, we define $N^\circ$ as the litter with small symmetric difference from $N$.

For any near-litter $N$, we define its local cardinal $[N]$ as $\{N':N^{\circ} = N'^{\circ}\}$.

We may refer to the elements of $N \Delta N^{\circ}$ as the anomalies of $N$.

We impose a well-ordering on the litters and for each cardinal $\nu$ with $\kappa<\nu \leq \mu$  define ${\bf A}_\nu$ as the union of the litters with ordinal index $<\nu$ in this well-ordering, which will be of cardinality $\mu$.

\subsection{An initial overview of the system of levels}

We construct a system of levels indexed by the ordinals $<\lambda$ and parameterized by cardinals $\nu$ with $\kappa<\nu<\mu$.  The dependence on $\nu$ will often be left implicit.

Level 0 will be ${\bf A}_\nu$.  If we want to make the dependence on $\nu$ explicit, say level $0^{\nu}$ will be ${\bf A}_\nu$. 

The elements of level $1+\alpha^{\nu}$ (which is intended to serve as level $\alpha$ in a model of TTT) will be of the form $(1+\alpha,\beta,{\bf B})$, where $\beta<\alpha$
and ${\bf B}$ is a subset of level $\beta^{\nu}$.  We call such triples eligible triples.  Not all eligible triples will belong to level $1+\alpha^{\nu}$, which we will often call level $1+\alpha$ when $\nu$ is understood from context.

The intention is that in the model of tangled type theory, where an element of level $1+\alpha$ is understood to be an object of type $\alpha$, the type $1+\beta$ elements of
$(1+\alpha,1+\beta,{\bf B})$ in the sense of the model will be the elements of ${\bf B}$ in the sense of the ambient ZFAC.  The level $\gamma$ elements of a type $\alpha$ object in the interpretation of TTT for $\beta \neq \gamma <\alpha$ are determined in a special way described in the following subsections.

\subsection{Typed near-litters and local cardinals}

A typed (near-)litter is a triple of the form $(1+\alpha,0,N)$ where $N$ is a (near)-litter included in ${\bf A}_\nu$.  Such triples belong to level $1+\alpha$.

If $(1+\alpha,0,N)$ is a typed near-litter, we define $[(1+\alpha,0,N)]$, the typed local cardinal of $(1+\alpha,0,N)$, as $\{(1+\alpha,0,N'):N' \in [N]\}$.

\subsection{Structural maps}

We designate sets ${\tt rng}(\xi_{\beta,1+\alpha})$ of litters for each $\alpha, \beta <\lambda$ with $\beta \neq 1+\alpha$.  Sets of this kind with distinct indices are disjoint,
and the intersection of the union of each such set of litters with ${\bf A}_\nu$ is of size $\nu$.

Once level $\beta$ has been constructed, we define maps $\xi_{\beta,1+\alpha}$ which will be bijections from type $\beta$ to the set of typed litters included in the already defined set
${\tt rng}(\xi_{\beta,1+\alpha})\cap {\cal P}({\bf A}_\nu)$.  Of course, this requires that types $\beta,1+\alpha$ are of size $\nu$.  We stipulate (and will justify) that when
level $1+\alpha$ is constructed, we are able to establish that this level will be of size $\nu$ if $\nu$ exceeds a threshhold $\nu_{1+\alpha}<\mu$ to be described.  Of course we know that level 0 is of size $\nu$.  It should be noted that each map has an additional index $\nu$ and may be written $\xi_{\beta,1+\alpha,\nu}$ when this is needed for clarity.

\subsection{The other interpreted extensions of elements of levels are described.  The membership relation ${\bf E}$ of TTT is described, and extensionality is enforced.}

We can now describe the level $1+\gamma$ elements of $(1+\alpha,\beta,{\bf B})$ in the sense of the interpreted TTT for each $\beta \neq 1+\gamma <1+\alpha$.

The lavel $1+\gamma$ elements of $(1+\alpha,\beta,{\bf B})$ in the sense of the interpreted TTT will be the elements of the union of the typed local cardinals of the elements of $\xi_{\beta,1+\gamma}``{\bf B}$, that is, the triples $(1+\gamma,0,N)$ where $N$ is a near-litter and $N^\circ \in \xi_{\beta,1+\gamma}``{\bf B}$.

We can now state the definition of the relation {\bf E} on the union of the positive levels which is intended to interpret the membership relation of TTT.  

We need to define
$(1+\gamma,\delta,G) {\bf E} (1+\alpha,\beta,{\bf B})$.  

For this to hold we must have $1+\gamma<1+\alpha$.  

If $1+\gamma=\beta$, then $(1+\gamma,\delta,G) {\bf E} (1+\alpha,\beta,{\bf B})$ iff $(1+\gamma,\delta,G) \in {\bf B}$.  

If $\beta \neq 1+\gamma <1+\alpha$, then $(1+\gamma,\delta,G) {\bf E} (1+\alpha,\beta,{\bf B})$ iff $(1+\gamma,\delta,G) \in \bigcup_{b \in {\bf B}}[\xi_{\beta,1+\gamma}(b)]$.  Note that this implies
that $\delta=0$ and that $G$ is a near-litter with small symmetric difference from an element of ${\tt rng}(\xi_{\beta,1+\gamma})$.

In order to enforce extensionality, we need to impose the condition that in a triple $(1+\alpha,1+\beta,{\bf B})$, the set ${\bf B}$ cannot be a union of typed local cardinals of litters
in a fixed ${\tt rng}_{\gamma,1+\beta}$ for $1+\beta\neq \gamma<1+\alpha$, as otherwise there might be a triple $(1+\alpha,\gamma,G)$ with the same preimage in level $1+\beta$ under {\bf E}.


Note that this condition enforces the further condition that if ${\bf B}$ is empty in $(1+\alpha,\beta,{\bf B})$ then $\beta=0$.  This condition is sufficient to enforce extensionality in the interpretation of TTT:  it forces level elements which have the same preimage under {\bf E} in any given lower level to be equal.

\subsection{Allowable permutations}

We require that $(1+\alpha,\beta,\{b\})$ belong to level $1+\alpha$ for
each $b$ in level $\beta$.

We define a class of permutations of levels.

An $\alpha$-allowable permutation is a permutation of level $\alpha$ with additional properties we now describe.

A permutation $\pi$ of level 0 is 0-allowable iff $\pi``N$ is a near-litter for each near-litter $N$.

A permutation $\pi$ of level $1+\alpha$ is $1+\alpha$-allowable iff for each triple $(1+\alpha,\beta,{\bf B})$, $$\pi((1+\alpha,\beta,{\bf B})) = (1+\alpha,\beta,\pi_{\beta}``{\bf B}),$$ where $\pi_\beta$, defined implicitly by $\pi((1+\alpha,\beta,\{b\})) = (1+\alpha,\beta,\pi_\beta(b)\})$ is a $\beta$-allowable permutation, and a further side condition:  when $x$ in level $\beta$ and $y$ in level $1+\alpha$, we must have $x {\bf E} y \leftrightarrow \pi_{\beta}(x) {\bf E} \pi(y)$.

Note that $\pi$ is computable at every eligible triple $(1+\alpha,\beta,{\bf B})$ whether the triple actually belongs to level $1+\alpha$ or not.

We work out the consequences of the side condition.

$x {\bf E} y$ holds if one of two conditions holds:

\begin{enumerate}

\item  In the first case, $x=(1+\beta,\gamma,G)$, $y = (1+\alpha,1+\beta,{\bf B})$, and $x \in {\bf B}$.  In this case, $\pi(y) = (1+\alpha,1+\beta,\pi_\beta``{\bf B})$,
and we need no special stipulation to see that $\pi_{\beta}(x) \in \pi_{\beta}``{\bf B}$ so $\pi_\beta(x) {\bf E}  (1+\alpha,1+\beta,\pi_\beta``{\bf B}) = \pi(y)$:  the side condition is a consequence of the first part of the definition in this case.

\item  In the second case $x=(1+\beta,\gamma,G)$ and $y=(1+\alpha,\epsilon,H)$, and we have $x {\bf E} y$ equivalent to $(\exists h \in H:x \in [\xi_{\epsilon,1+\beta}(h)])$.
This tells us that $\gamma=0$ and $G$ is a near-litter with small symmetric difference from an element of $\xi_{\epsilon,1+\beta}``H$.  Stipulate that $H$ is a singleton set
$\{h\}$, so $G^{\circ} = \xi_{\epsilon,1+\beta}(h)$.   The side condition requires that $\pi_{1+\beta}(x) = (1+\beta,0,(\pi_{1+\beta})_0``G) {\bf E} \pi(y) = (1+\alpha,\epsilon,\{\pi_\epsilon(h)\})$, whence
$(\pi_{1+\beta})_0``G$ is a near-litter with small symmetric difference from an element of $\xi_{\epsilon,1+\beta}``\{\pi_\epsilon(h)\}$, that is, from $\xi_{\epsilon,1+\beta}(\pi_\epsilon(h))$.   We thus have the condition that  $(\pi_{1+\beta})_0``G$, which has small symmetric difference from $(\pi_{1+\beta})_0``\xi_{\epsilon,1+\beta}(h)$, also  has small symmetric difference from  $\xi_{\epsilon,1+\beta}(\pi_\epsilon(h))$, which should appear to be reasonable coherence condition.  Briefly, we can write  $((\pi_{1+\beta})_0``\xi_{\epsilon,1+\beta}(h))^{\circ}=\xi_{\epsilon,1+\beta}(\pi_\epsilon(h))$, where $1+\beta, \epsilon <1+\alpha$.  We leave it as an exercise to backtrack and show that this implies the side condition.

\end{enumerate}

\subsection{Notation for derived permutations}

We define a notation $\pi_A$ where $\pi$ is an $\alpha$-allowable permutation.  In each case $\pi_A$ will be a ${\tt min}(A)$-allowable permutation.

We provide as the base case that $\pi_{\{\alpha\}} = \pi$.

Where $A$ has more than one element, $\pi_A = (\pi_{A_1})_{{\tt min}(A)}$

We call these the derived permutations of $\pi$.

\subsection{Supports and the definition of positive levels}

An $\alpha$-support is a well-ordering $S$ of triples of the form $(x,A,\beta)$, where $A$ is a nonempty finite subset of $\lambda$,  ${\tt max}(A)=\alpha$, and $x$ is either an atom or a typed near-litter and belongs to level ${\tt min}(A)$, and $\beta\in A$.

The action of an $\alpha$-allowable permutation $\pi$ on $S$ replaces each domain element  $(x,A,\beta)$ with $(\pi_A(x),A,\beta)$:  the resulting support is written $\pi[S]$.

An object $X$ of level $\alpha$ has support $S$ iff for each $\alpha$-allowable permutation $\pi$ such that $\pi[S]=S$, we also have $\pi(X)=X$.

Level $1+\alpha$ is defined as the collection of all triples $(1+\alpha,\beta,{\bf B})$, where $\beta<1+\alpha$ and ${\bf B}$ is a subset of level $\beta$, which have a $1+\alpha$-support, for all values of the hidden
parameter $\nu$ for which the size of level $1+\alpha$ thus defined is actually $\nu$.  We will show that for a certain $\nu_{1+\alpha}<\mu$, for all $\nu \in [\nu_{1+\alpha},\mu]$, the definition succeeds.  Once the level is defined, define all maps $\xi_{1+\alpha,1+\gamma}$.

For any nonempty subset $A$ of $\lambda$, we define an $A$-support [of $X$] as a well-ordering $S$ of triples of the form $(z,C,\gamma)$  where $A$ is an upper segment of $C$ for each element $(z,C,\gamma)$ of the domain of $S$, and the
well-ordering obtained by replacing each $(z,C,\gamma)$ in the domain of $S$ with $(z,C \setminus A_1,\gamma)$ is
a ${\tt min}(A)$-support [of $X$].

\subsection{Strong supports defined}

An $\alpha$-support $S$ [of $X$] is a strong $\alpha$-support [of $X$] if it has the property that 

\begin{enumerate}

\item each $(z,C,\gamma)$ in the domain of $S$ for which $z$ is
an atom is preceded by $(({\tt min}(C_1),0,L),C_1,\gamma)$ where $L$ is the litter containing $z$,  if $\gamma \in C_1$.

\item and each $(z,C,\gamma)$ where $z$ is a typed near-litter belonging to the range of $\xi_{\epsilon,1+\delta}$
with $\epsilon, 1+\delta<\gamma$ is preceded in $S$, for some $\epsilon$-support $T$ of $\xi_{\epsilon,1+\delta}^{-1}(z)$, by all $(w,C_1 \cup D,\epsilon)$ such that $(w,D,\delta) \in T$ for some $\delta$.  [equivalently, is preceded in $S$ by a $C_1 \cup \{\epsilon\}$-support of $\xi_{\epsilon,1+\delta}^{-1}(z)$.]

\end{enumerate}

Any $\alpha$-support of an object $X$ can be transformed into an $\alpha$-support of $X$ in which all typed near-litters are typed litters by replacing each item $((1+\beta,0,N),C,\gamma)$ in the domain of the support with
$((1+\beta,0,N^\circ),C,\gamma)$ and all $(y,C \cup \{0\},\gamma)$ for $y \in N \Delta N^\circ$.

Any $\alpha$-support of an object $X$ in which all typed near-litters in the domain are typed litters can be extended and rearranged in order to a strong $\alpha$-support of $X$.  The basic move is to add just before each item which requires something to precede it the appropriate items, then if duplicates are introduced, eliminate all but the first of them.   Since only one item is added before an item whose first projection is an atom, and it does not have first projection an atom, and further, an item added before an item whose first projection is a typed litter must have smaller third projection, it will not be the case that an infinite descending sequence of items is ever added:  this process will converge at a well-ordering and therefore at a strong support.

The idea of the third component of an element $(z,C,\gamma)$ of the domain of a support $S$, which plays no role
in the definition of support of an object, is that it is a comment indicating that this is an element of an $A_j$-subsupport of $S$ of interest, where ${\tt min}(A_j)=\gamma$.

\subsection{Freedom of action of allowable permutations}

We say that an allowable permutation has $(x,A)$ as an exception, where $x$ is an atom belonging to litter $L$ and  $A$ has minimum 0, if and only if  either $\pi_A(x) \in \pi_A``L \setminus \pi_A``L^\circ$ or $\pi^{-1}_A(x) \in \pi_A^{-1}``L \setminus \pi^{-1}_A``L^\circ$.  Note that $\pi_A``L \setminus \pi_A``L^\circ$ and $\pi_A^{-1}``L \setminus \pi^{-1}_A``L^\circ$ are small sets.


We define an $\alpha$ local bijection as a system of maps $\pi^0_A$ where $A$ is a finite subset of $\lambda$ with maximum $\alpha$ and minimum 0, with the properties that each $\pi^0_A$ is injective and has domain equal to its range, and has domain a set of atoms with small intersection with each litter.

The Freedom of Action theorem asserts that given a $\alpha$-local bijection $(A \mapsto \pi^0_A)$  there is an $\alpha$-allowable permutation $\pi$ such that $\pi_A(x) = \pi^0_A(x)$ for every $A$ and every $x$ in the domain of $\pi^0_A$  and such that $(x,A)$ is an exception of $\pi$ only if $x$ is in the domain of $\pi^0_A$.

\subsection{Embedding the structure associated with a smaller value of $\nu$ into the structure associated with a larger value.}

We define maps $\chi_{\nu_1,\nu_2}$ which embed the system of levels associated with $\nu=\nu_1$
into the system of levels associated with $\nu=\nu_2>\nu_1$.

The restriction of $\chi_{\nu_1,\nu_2}$ to levels 0 is the identity embedding of ${\bf A}_{\nu_1}$ into ${\bf A}_{\nu_2}$.

The restriction of $\chi_{\nu_1,\nu_2}$ to level $1+\alpha^{\nu_1}$ acts as follows:  suppose that \newline$(1+\alpha,\beta,{\bf B})$ in level $1+\alpha$ has $1+\alpha$-support $S$.  We define $\chi_{\nu_1,\nu_2}[S]$ as the result of replacing each $(z,C,\gamma)$ in the domain of $S$ with $(\chi_{\nu_1,\nu_2}(z),C,\gamma)$.  We then define $\chi_{\nu_1,\nu_2}((1+\alpha,\beta,{\bf B}))$ as $(1+\alpha,\beta,{\bf B}')$, where ${\bf B}'$ is the set of all $\pi_\beta(\chi_{\nu_1,\nu_2}(b))$ where $b \in {\bf B}$ and $\pi$ is a $1+\alpha$-allowable permutation such that $\pi[\chi_{\nu_1,\nu_2}[S]] = \chi_{\nu_1,\nu_2}[S]$.

NOTE:  verify that this does not depend on choice of $S$.  verify that the side condition is respected.  Oddly, it seems that no coherence conditions for $\xi$ maps are needed.

Suppose that $(1+\alpha,\beta,{\bf B})$ has supports $S_1$ and $S_2$.   We need to show that ${\bf B}_1$, the set of all $\pi_\beta(\chi_{\nu_1,\nu_2}(b))$ where $b \in {\bf B}$ and $\pi$ is a $1+\alpha$-allowable permutation such that $\pi[\chi_{\nu_1,\nu_2}[S_1]] = \chi_{\nu_1,\nu_2}[S_1]$, is the same as ${\bf B}_2$, the set of all $\pi_\beta(\chi_{\nu_1,\nu_2}(b))$ where $b \in {\bf B}$ and $\pi$ is a $1+\alpha$-allowable permutation such that $\pi[\chi_{\nu_1,\nu_2}[S_2]] = \chi_{\nu_1,\nu_2}[S_2]$, in order to show that the definition of the $\chi$ map at positive levels does not depend on the choice of support.    It suffices to show that the intersection of two supports of an object is a support.













\end{document}