\documentclass[12pt]{article}

\title{A self contained account of a class of models of tangled type theory}

\author{Randall Holmes}

\date{starting, 7/13/2023\\
typos and mental failures fixed, 7/18/2023\\
revised during 7/24/2023 Zoom meeting}

\usepackage{amssymb}

\begin{document}

\maketitle

In this document, we give a self contained account of a structure which will turn out to be a model of tangled type theory, and therefore a witness to the consistency of Quine's New Foundations.
We will not discuss these theories until later in the narrative.  All of our business will be conducted in the usual set theory ZFC, and in fact in not very much of it, because New Foundations is not a very strong theory.

We are trying out the numbered paragraph format which Zermelo uses in his 1908 papers.

\begin{enumerate}

\item The construction has parameters which we introduce.

$\lambda$ is a limit ordinal.  Elements of $\lambda$ and a special object $-1$ will be our type indices (it simply doesn't matter what $-1$ is:  any set that isn't an ordinal will do).  The order on type indices is suggested by the choice of symbol for the additional type index:  $-1$ is the minimum in the order on type indices and the order on type indices extends the usual order on $\lambda$ (the natural order on ordinals $<\lambda$).

$\kappa$ is an uncountable regular cardinal greater than $\lambda$.  Sets of cardinality $<\kappa$ will be termed small and sets which are not small are called large.

$\mu$ is a strong limit cardinal $>\kappa$ of cofinality $\geq \kappa$.

These notations are fixed for the rest of the paper.

\item  Motivational notes:  we are letting parameters vary in size to support extensions of NF which are strong in various ways.  If one were aiming to put a cap on the consistency strength of NF (doing this precisely is not among the aims of this paper) note that $\lambda = \omega; \kappa = \omega_1; \mu = \beth_{\omega_1}$ works here.  We believe that NF is even weaker than these values of the parameters suggest, but we are not at pains here to show this.

\item We first build a system of {\em supertypes\/} indexed by the type indices, which will seen to be a model of nonextensional tangled type theory (we will explain what this means presently).

We write the supertype indexed by a type index $\iota$ as $\tau^*_\iota$.

Supertype $-1$ ($\tau^*_{-1}$) is unspecified at this point, except that it is a set of cardinality $\mu$.  We will describe it more precisely later, but its exact nature is unimportant at this stage.  Any choice of $\mu$ and a set of cardinality $\mu$ to serve as $\tau^*_{-1}$ can be taken to determine a system of supertypes at this point.

For $\alpha\in \lambda$ (a type index other than $-1$), we define $\tau^*_\alpha$ as $${\cal P} (\bigcup_{-1 \leq \iota<\alpha} \tau^*_\iota \cup \{\{\tau^*_\eta:-1\leq \eta <\alpha\}\}) \setminus {\cal P} (\bigcup_{-1 \leq \iota<\alpha}\tau^*_\iota):$$

an element of $\tau^*_\alpha$ is a union of subsets of the $\tau^*_\iota$'s for $-1\leq \iota <\alpha$ with the additional element $\{\tau_\eta:-1\leq \eta <\lambda\}$ added.

We denote $\{\tau^*_\eta:-1\leq \eta <\alpha\}$ by $\tau^+_{\alpha}$.

\item The axiom of foundation in the underlying set theory ZFC ensures that the supertypes are disjoint.  Notice that $\tau^+_{\alpha}$ has higher rank than any $\tau^*_\iota$ for
$-1\leq \iota <\alpha$ and so $\tau^+_{\alpha}$ cannot be an element of any element of $\tau^*_\iota$ for $-1\leq \iota<\alpha$, and so $\tau^*_\alpha$ (all of whose elements contain $\tau^+_{\alpha}$)  is disjoint from every
such $\tau^*_\iota$ (every type of smaller index).

None of the sets $\tau^+_{\alpha}$ are elements of any $\tau^*_\iota$:  no  $\tau^+_{\alpha}$ can be an element of $\tau^*_{-1}$ because $\tau^*_{-1} \in \tau^+_{\alpha}$ for every
$\alpha$.  For $\tau^+_{\alpha}$ to belong to $\tau_\beta$ ($\beta \neq -1)$ we would need $\tau^+_{\beta}$ to belong to each element of $\tau^+_\alpha$, including
$\tau^*_{-1}$, to which we have just seen that no $\tau^+_{\alpha}$ can belong.

\item For each $\alpha,\beta$ with $-1\leq\alpha<\beta<\lambda$, we define $x \in_{\alpha,\beta} y$ as holding iff $x \in \tau^*_\alpha \wedge y \in \tau^*_\beta \wedge x \in y$.

This structure is a model for nonextensional tangled type theory (TTT$^-$) which we now describe briefly for motivation.

TTT$^-$ is a theory with membership and equality as primitive relations and types indexed by the elements of $\lambda$ (type indices other than $-1$).

A formula $x^\alpha = y^\beta$ is meaningful iff $\alpha=\beta$.  A formula $x^\alpha \in y^\beta$ is meaningful iff $\alpha<\beta$.

The sole axiom scheme asserts that for any formula $\phi(x^\alpha)$ in the language and $\beta>\alpha$ there is $\{x^\alpha:\phi(x^\alpha)\}^\beta$ such that
$$(\forall z^\alpha:z^\alpha \in \{x^\alpha:\phi(x^\alpha)\}^\beta \leftrightarrow \phi(z^\alpha).$$

This theory is satisfied by the system of supertypes if we interpret $x^\alpha \in y^\beta$ as $x^\alpha \in_{\alpha,\beta} y^\beta$.

The theory is nonextensional:  there isn't a unique witness to serve as $\{x^\alpha:\phi(x^\alpha)\}^\beta$, though we can choose a canonical one, namely, the one whose intersection
with any $\tau_\iota$ with $\iota \neq \alpha$ is empty.

\item We further describe the theory TTT, in order to motivate the exertions we will go through in the rest of this paper.

TTT extends TTT$^-$ with the additional axiom scheme of extensionality, the collection of all well formed sentences of the form $$(\forall x^\beta y^\beta:(\forall z^\alpha:z^\alpha \in x^\beta \leftrightarrow z^\alpha \in y^\beta)\rightarrow x^\beta=y^\beta).$$

It is known that the consistency of TTT implies (in fact is equivalent to) the consistency of New Foundations.  We will discuss this later.

\item In the system of supertypes, each element of type positive $\alpha$ has an extension over each type $\beta\in \alpha$, namely, its intersection with type $\beta$.  These extensions can be mixed and matched freely:  there are many type $\alpha$ objects with any given extension over type $\beta$ (even in the case $\beta=0, \alpha=1$, as we can vary the intersection of
the type $\alpha$ object with type $-1$).

 In a model of TTT, each object of positive type is uniquely determined by each of its extensions individually.  This means that one extension of any particular object determines the others.
Our construction continues by exhibiting how this is done in our construction, starting with a presentation of more detail about type $-1$.

\item We will refer to the elements of type $-1$ as {\em atoms\/}.  They are not atoms in the sense of the metatheory (in fact, we will say something about their extensions as sets in a moment) but it is convenient to have a generic term for them (and in earlier constructions carried out in ZFA, analogous objects were atoms).

\item We now specify exactly what $\tau^*_{-1}$ is (in terms of the parameters $\kappa$ and $\mu$).

$\tau^*_{-1}= \{(\nu,\beta,\gamma,\alpha):\nu<\mu \wedge  \beta \in \lambda\cup \{-1\} \wedge \gamma \in \lambda \setminus \{\beta\}\wedge \alpha<\kappa\}$

\item For any suitable $\nu, \beta, \gamma$ we define $\Lambda_{\nu,\beta,\gamma}$ as $\{(\nu,\beta,\gamma,\alpha)\in \tau^*_{-1}:\alpha<\kappa\}$.  We regard this notation as defined
only if the resulting set is nonempty.  Such sets are called {\em litters} and the set of litters is a partition of $\tau^*_{-1}$ into sets of cardinality $\kappa$.

We define $X_{\beta,\gamma}$ as \{$\Lambda_{\nu,\beta,\gamma}:\nu<\mu\}$.  The use of this partition of the litters will be seen below.

\item A subset of $\tau^*_{-1}$ with small symmetric difference from a litter we call a {\em near-litter\/}.  For any near-litter $N$ we define $N^\circ$ as the uniquely determined litter $L$
such that $|N \Delta L|<\kappa$.  If $M$ and $N$ are litters, we write $M \sim N$ for $|M \Delta N| < \kappa$.  This is an equivalence relation on near-litters.

\item Our intention is to construct $\tau_\iota$ for each type index $\iota$ in such a way that $\tau_{-1} = \tau^*_{-1}$ (and we will henceforth abandon the latter notation, always writing $\tau_{-1}$)
and for each $\alpha \in \lambda$, $\tau_\alpha \subseteq \tau^*_\alpha$ and $|\tau_\alpha| = \mu$, and for each $-1 \leq \alpha < \beta <\lambda$ and $x \in \tau_\beta$,
$x \cap \tau^*_\alpha\subseteq \tau_\alpha$.   Further, for each $-1<\gamma<\beta<\lambda$ we
have for $x,y \in \tau_\beta$ that $x \cap \tau_\gamma = y \cap \tau_\gamma \rightarrow x=y$:  we have extensionality (in the strong form required to interpret TTT)  for the types indexed by ordinals.  There are of course further conditions to be unfolded as we proceed.

\item Our strategy will be to fix an $\alpha\in \lambda$ and hypothesize that the sets $\tau_\beta$ have already been constructed for each $\beta<\alpha$ (satisfying these conditions [and others yet to be stated]), and then describe how $\tau_\alpha$ is to be constructed [supposing at all points that earlier $\tau_\beta$'s were constructed in the same way].   We suppose that we have already specified
a well-ordering $\leq_\beta$ with order type $\mu$ of each type $\tau_\beta$ with $-1\leq \beta <\alpha$ (special conditions on the choice of these well-orderings will be given later).   

\item For any near-litter $N$ and $\gamma\neq -1$, we define $N_\gamma$ as the unique element $x$  of $\tau_\gamma$ with $x \cap \tau_{-1} = N$.  We stipulate that there is one (for any $N$ and for $\gamma<\alpha$).  More generally, if
$X \subseteq \tau_{-1}$, $X_\gamma$ is the unique element $x$ of $\tau_\gamma$ with $x \cap \tau_{-1} = X$, if there is one.  We do provide that $\emptyset_\gamma$ and $\{x\}_\gamma$ will exist
for $x \in \tau_{-1}$, $\gamma<\alpha$.

\item We define for each element $x$ of any $\tau_\beta$ the index $\iota_*(x)$
as the order type of the restriction of $\leq_\beta$ to $\{y \in \tau_\beta:y <_\beta x\}$.  Note that the domain of $\iota_*$ is the union of all the types!

\item We first indicate how extensionality is to be enforced.  

\item We construct, for each pair of ordinals $\beta,\gamma<\alpha$ with $\beta \neq \gamma\neq -1$ (note that $\beta$ can be $-1$),  an injection $f_{\beta,\gamma}$ from $\tau_\beta$ into $X_{\beta,\gamma} = \{\Lambda_{\nu,\beta,\gamma}:\nu<\mu\}$  (whose definition does not actually depend on $\alpha$:  it will be the same at every stage).

 $f_{\beta,\gamma}$ is an injection from $\tau_\beta$ into $X_{\beta,\gamma}$:  note that the ranges of distinct $f_{\beta,\gamma}$'s are disjoint.
When we define $f_{\beta,\gamma}(x)$, we presume that we have already defined it for $y <_\beta x$.
We define $f_{\beta,\gamma}(x)$ as $L\cap \tau_{-1}$, where $L$ is $<_\gamma$-first such that $L\cap \tau_{-1} \in X_{\beta,\gamma}$ and for every $N \sim L \cap \tau_{-1}$, $\iota_*(N_\gamma)>\iota_*(x)$ [and if $\beta=-1$, $\iota_*(N_\gamma)>\iota_*(\{x\}_0)$ (NOTE:  do I need this here?)], and for any $y<_\beta x$, $f_{\beta,\gamma}(y) \neq L\cap \tau_{-1}$.  That this can be done relies on the fact that the order type of each $\leq_\beta$ is $\mu$.

\item Let $-1<\beta\leq \alpha$.


Let $\tau_\beta^1$ be the set of elements of $\tau_\beta^*$ satisfying $x \cap \tau^*_\gamma \subseteq \tau_\gamma$ for each $\gamma<\beta$.

Let $\tau_\beta^2$ be the set of elements of $\tau_\beta^1$ which are ``weakly extensional" in a sense we now define.

 An extension of an element $x$ of $\tau_\beta^1$ is a set $x \cap \tau_\gamma$ for $-1 \leq \gamma <\beta$ [we call this extension for a particular value of $\gamma$ the $\gamma$-extension].  We say that an element $x$ of  $\tau_\beta^1$ is {\em weakly extensional\/}
iff it has an extension $x \cap \tau_\gamma$, called a distinguished extension, which has the property that if any extension of $x$ is empty or if $x \cap \tau_{-1}$ is nonempty, $\gamma = -1$, and that for any $\delta \in \beta \setminus \{-1,\gamma\}$ we have $$x \cap \tau_\delta = \{N_\delta:N^\circ \in f_{\gamma,\delta}``(x \cap \tau_\gamma)\}.$$

We pause to define a function implementing this.  For any nonempty subset $X$ of $\tau_\gamma$, we define $$A_\delta(X) =  \{N_\delta:N^\circ \in f_{\gamma,\delta}``(X)\}.$$

Note that this function $A_\delta$ can be taken to have the quite large domain $$\bigcup_{\gamma \in \beta \setminus \{\delta,-1\}} {\cal P}(\tau_\gamma) \setminus \{\emptyset\},$$ since we can determine given a set in the domain what the appropriate value of $\gamma$ is.

We can then state that a distinguished extension $x \cap \tau_\gamma$ of $x$ is characterized by the condition that for each $\delta$ not equal to $\gamma$ or $-1$,
$x \cap \tau_\delta = A_\delta(x \cap \tau_\gamma)$, and if $\gamma \neq -1$, $x \cap \tau_{-1}$ is empty.

Note that this allows us immediately to determine all extensions of objects $N_\gamma$ for $N$ a near-litter or $\{x\}_\gamma$ for $x$ an atom.

It is part of the hypotheses of the construction that $\tau_\beta \subseteq \tau^2_\beta$ for each ordinal $\beta$ less than $\alpha$:  elements of types already constructed are weakly extensional.

\item We show that no $x$ has more than one distinguished extension.  If the distinguished extension of $x$ is empty, all extensions of $x$ are empty, and we note further that 
the $-1$-extension is designated as the distinguished extension (of course all the extensions are the same set).  Note further that if the distinguished extension of $x$ is nonempty,
and $c$ is the element of this extension with minimal image under $\iota_*$, then every element of every other extension will have image under $\iota_*$ exceeding $\iota_*(c)$, because of the way the $f$ maps are constructed,  establishing that there is only one distinguished extension.

\item  That every element of a type in our system of types is weakly extensional will not enforce the extensionality condition we want.  Let $-1 \leq \gamma < \beta \leq \alpha$, and let
$x,y \in \tau^2_\beta$ with $x \cap \tau_\gamma = y \cap \tau_\gamma$.   

If $x \cap \tau_\gamma = y \cap \tau_\gamma$ is the empty set, then $x=y$ is immediate, because all the extensions of both sets are empty.  Note that if any extension
of $x$ or $y$ is empty, all are, so we can suppose hereinafter that all extensions of $x$ and $y$ are nonempty.

If the distinguished extensions of $x$ and $y$ are both the $\delta$-extension for some $\delta$ (which might or might not be $\gamma$), then again $x=y$ because we have a method of computation of all other extensions of $x$ and $y$ which will give $x \cap \tau_\epsilon = y \cap \tau_\epsilon$ for each appropriate $\epsilon$. 

If the distinguished extensions of $x$ and $y$ (supposed nonempty) are the $\delta$-extension of $x$ and the $\epsilon$-extension of $y$, with $\delta \neq \epsilon$, then any $z$ in the $\gamma$-extension of $x$ must be of the form $N_\gamma$ where $N^\circ$ is in the range of $f_{\delta,\gamma}$ and in the range of $f_{\epsilon,\gamma}$, and this is impossible, as the ranges of these maps are disjoint.

The possibility which cannot be excluded is that $x \cap \tau_\gamma = y \cap \tau_\gamma$ is the distinguished extension of one of $x,y$ and not of the other.

\item  Let $-1 < \beta \leq \alpha$.  We are working on defining the collection $\tau_\beta^3$ of {\em extensional\/} elements of $\tau^2_\beta$ (so to begin with,
an extensional element of $\tau^1_\beta$ is weakly extensional).

The maps $A_\delta$ defined above are injective.  The ranges of distinct maps $A_\delta$ are disjoint.  

Thus, we can define $A^{-1}(x)$ for $x \in \bigcup_{\gamma \in \beta\setminus \{-1\}} {\cal P} (\tau_\gamma) \setminus \{\emptyset\}$ as the unique $y$ such that $A_{\delta}(y)=x$ for some $\delta$ ($\delta$ of course being determined by $x$), if such a $y$ exists.  The map $A^{-1}$ is of course partial:  but for any $x$, if there is any such $y$ there is only one.

If $c$ is the element of $x$ with minimal image under $\iota_*$, the element d with minimal image under $\iota_*$ in any $A_\delta(x)$ will have $\iota_*(d) > \iota_*(c)$ because
of the way the $f$ maps are defined.  This implies that for any $x$, if $c$ is the element of $x$ with minimal image under $\iota_*$, and $A^{-1}(x)$ exists, then if
$d$ is the element of $A^{-1}(x)$ with minimal image under $\iota_*$, we have $\iota_*(d) < \iota_*(c)$.  This in turn implies that no set has infinitely many iterated images
under $A^{-1}$.

We then define $\tau^3_\beta$ as the collection of all elements $x$ of $\tau^2_\beta$ with the property that either the distinguished extension of $x$ is empty or the collection of iterated images of the distinguished extension of
$x$ under $A^{-1}$ (not including $x$) is of even cardinality.  Note that since every other extension of $x$ is an image of the distinguished extension under an $A_\delta$, they all have odd numbers of iterated images under $A^{-1}$.  

Now observe that in the case where two distinct elements $x,y$ of $\tau^2_\beta$ have the same $\gamma$-extension for a suitable $\gamma$, described above,
the common extension of $x$ and $y$ is the distinguished extension of one of them (wlog $x$) and not the distinguished extension of the other, and so the image under $A_\gamma$ of the distinguished extension of $y$.  This means that one of $x$ and $y$ is extensional, and the other is not, by considering the parities of the cardinalities of the sets of iterated images of the respective distinguished extensions under $A^{-1}$.

We further state that $\tau_\beta \subseteq \tau^3_\beta$ for $-1 < \beta < \alpha$ as a hypothesis of the construction:  all sets in types already constructed are extensional.

\item  We are going to attempt a different approach here.  Rather than defining the extensions we will include in our model as those symmetric under a class of permutations, we will
attempt to directly describe codes for the construction of these extensions (clearly a closely related approach, but we think it may have formal advantages).  A symmetry requirement will appear!

\item  A typed near-litter is an element $N_\gamma$ of $\tau_\gamma$ where $N$ is a near-litter.  A typed atom is an element $\{x\}_\gamma$ of $\tau_\gamma$ where
$x \in \tau_{-1}$.

A {\em support element\/} is a pair $(x,A)$ where $A$ is a finite subset of type indices with $x$ either an atom (type $-1$) [in which case we call it an atomic support element and ${\tt min}(A) =-1$] or a typed near-litter (of some type $>-1$ and $x \in \tau_{{\tt min}(A)}$ [in which case we call it a near-litter support element].

A {\em $\beta$-support\/} is a small set of support elements $(x,A)$ each of which has ${\tt max}(A)=\beta$, with the technical property that if it contains distinct $(x,A)$ and $(y,A)$ [with the same second component;  not all elements of a support need have the same second component], and $x$ and $y$ are both typed near-litters, they have disjoint $-1$-extensions.

For any $\beta$-support $S$ and $\gamma<\beta$, we define $S_\gamma$ as $$\{(x,A):{\tt max}(A)=\gamma \wedge (x,A\cup \{\beta\}) \in S\}.$$

\item  We attempt the recursive definition of a code for an element of our structure.

The executive summary of what a code for an element of $\tau_\beta$ is, if $\beta=-1$, a code for $x \in \tau_{-1}$ is $x$.  If $\beta>-1$, a code for an element $x$ of $\tau_\beta$ is a pair $(S,\Sigma)$, where $S$ is an $\alpha$-support and
$\Sigma$ is a set whose elements code precisely the elements of the distinguished extension $x \cap \tau_\gamma$ of $x$.

Notice that at this point we know how to unfold the element of the system of supertypes which is intended to be represented by any purported code.  There are additional conditions on an acceptable code, which we will state below.
The reason that we can make this claim is that (on the inductive hypothesis that it is true for $\gamma$ and that we have access to the $f$ maps) we can determine every element of every extension of $x$ -- using the inductive hypothesis to cover the distinguished extension, then the fact
that the set coded is intended to be weakly extensional to determine all the other extensions.   Note that the code for any element of the structure with nonempty extension over  $\tau_{-1}$ is a support paired with the actual extension, so we can extract the intended extension without paying
attention to the support.

We impose additional restrictions on the situation described above.  If $(b,B) \in \pi_1(z) \in \Sigma$, we already obviously require that ${\tt max}(B)=\gamma$.  We further require that $B$ is a not necessarily proper superset of $$S_\gamma = \{(x,A):{\tt max}(A) = \gamma \wedge (x,A \cup \{\beta\}) \in S\}$$

Finally, the set $\Sigma$ must be invariant under the action of ``substitutions" $\sigma$ acting on atomic support elements and fixing their second components and in addition fixing $S$ (speaking somewhat loosely here).  It remains to explain what we mean by this.

A $\beta$-substitution is a permutation $\sigma$ of atoms (if $\beta=-1$) or if $\beta >0$  a permutation $\sigma$ of atomic $\beta$-support elements $(x,A)$ with the following properties:

\begin{enumerate}

\item $\pi_2(\sigma((x,A)) = A$ for any atomic support element $(x,A)$  This very succinctly expresses the fact that a substitution $\sigma$ independently permutes, for each fixed $A$ with minimum $-1$, the set $$\{(x,A):x \in \tau_{-1}\}.$$

\item  For any $(N_\gamma,A)$, the set $\{\pi_1(\sigma((x,A\cup \{-1\})):x \in N\}$ is a near-litter.  We define the action of $\sigma$ on $(N_\gamma,A)$ as producing  $$\sigma[(N_\gamma,A)] = (\{\pi_1(\sigma((x,A\cup \{-1\})):x \in N\}_{{\tt min}(A)},A)$$.

Of course the action of $\sigma$ on an atomic support element coincides with the result of applying $\sigma$ as a function.

\item For any support $T$, we define $\sigma[T]$ as $\{\sigma[t]:t \in T\}$:  we already know how to compute the action of $\sigma$ on any element of a support.

\item  If $(S,\Sigma)$ is a $\beta$-code, and $\sigma$ is a $\beta$-substitution, we define $\sigma[(S,\Sigma)]$ as $(\sigma[S],\{\sigma_\gamma[c]:c \in \Sigma\})$, where the elements of $\Sigma$ are $\gamma$-codes and  $\sigma_{\gamma}$ is the $\gamma$-substitution satisfying $$\sigma_\gamma((x,A)) = (\pi_1(\sigma(x,A\cup \{\beta\})),A),$$ 
unless $\gamma=-1$, in which case $\sigma_{-1}(x) = \pi_1(\sigma(x,\{\beta,-1\}))$.

\item  Suppose that $(S,\Sigma)$ $\beta$-codes $x \in \tau_\beta$ and $\sigma$ is an arbitrary $\beta$-substitution;  then $\sigma[(S,\Sigma)]$ codes $\sigma[x]$ (definition of this notation;  the actions of substitutions on elements of the structure are what we call ``allowable permutations" in other presentations).  

We have a coherence relation with the $f$ maps restricting which permutations can be regarded as substitutions:  $\sigma[f_{\gamma,\delta}(x)]$ ($\gamma,\delta<\beta$), if defined, must  satisfy ``for some $N$, 
$\sigma[f_{\gamma,\delta}(x)_\delta] = N_\delta$ and $N^\circ = f_{\gamma,\delta}(\sigma[x])"$.  

Note that this implies that for any subset $x$ of $\tau_\beta$,
$$A_\delta(\{\sigma[x]:x \in X\}) = \{\sigma[y]:y \in A_\delta(X)\}:$$  the action of a substitution on the distinguished extension of an element of the structure is precisely parallel to its action on all other extensions of the structure element.  Further, this implies that the parity of the set of iterated images under $A_{-1}$ of a subset of $\tau_\beta$ is preserved by the action of the substitution (so the action of the substitution preserves extensionality).

\end{enumerate}

The so far unstated condition which makes a code acceptable is that $(S,\Sigma)$ is a code for an element of the type structure we are defining iff each element of $\Sigma$ is such a code, and each substitution $\sigma$ such that
$\sigma[s]=s$ for each element of $S$ also has $\sigma[(S,\Sigma)] = (S,\Sigma)$.  This exactly expresses the symmetry criterion for elements of our types $\tau_\beta$.

Note that if $(S,\Sigma)$ is an acceptable code, so is $\sigma[(S,\Sigma)]$:  this is because it is straightforward to see that (where elements of $\Sigma$ are $\gamma$-codes) the code $(\sigma[S],\{\sigma_\gamma[c]:c \in \Sigma\})$ is fixed by any substitution whose action  fixes $\sigma[S]$.

We assume as a hypothesis of the construction that each element of a type already constructed has an acceptable code and moreover has a designated acceptable code (hereinafter where we say ``code" we mean ``acceptable code").  We further suppose that the image under $\iota_*$ of the element with designated code $(S,\Sigma)$
is not less than the image under $\iota_*$  of each first projection of an element of $S$ in its proper type;  further, the image of a typed atom or near-litter in any type under $\iota_*$ is the same as the image of any other typed atom or near-litter with the same $-1$-extension, and any
near litter $N_\delta$ has $\iota_*(N^{\circ}_\delta) \leq \iota_*(N_\delta)$ and for any $x \in N$, $\iota_*(\{x\}_\delta) > \iota_*(N^\circ_\delta)$.  Further, $\iota_*(x) \leq \iota_*(y)$ iff $\iota_*(\{x\}_\delta) \leq \iota_*(\{y\}_\delta)$ for $x,y \in \tau_{-1}$.

\item We describe the selection of designated codes in more detail.

We choose a preliminary designated code for each codable element of the structure.

A support $S$ is said to be {\em strong\/} iff

\begin{enumerate}

\item for every $(N_\delta,A) \in S$, $N$ is a litter, and
\item  for every atomic support element $(x,A)$ in $S$ there is $$(N,A\setminus\{-1\}) \in S$$ with $x \in N$, and 

\item for every support element of the form $(f_{\delta,\epsilon}(x)_\epsilon,A) \in S$ for which $\delta$ is dominated by every element of $A$ except $\epsilon$ we also have for each $(y,C)$ in the preliminary designated support
of $x$ that $$(y,(B \setminus \{\epsilon\})\cup C)\in S.$$

\end{enumerate}  It should be evident that any support can be modified to one satisfying the first condition by replacing each element
$(N_\delta,A)$ with $(N^\circ_\delta,A)$ and the $(x,A\cup \{-1\})$ such that $x \in N \Delta N^\circ$:  modifying the first component of a code
in this way will preserve acceptability of the code, because any substitution whose action preserves the modified code also preserves the original code.

A code thus modified can be extended to a strong support satisfying the other two conditions simply by enforcing these closure conditions through $\omega$ steps.  The designated support of each object is obtained by extending the first projection
of the preliminary designated support to the smallest strong support including  it as a subset.

We refer to support elements of the form $(f_{\delta,\epsilon}(x)_\epsilon,A)$ for which $\delta$ is dominated by every element of $A$ except $\epsilon$ as {\em inflexible\/} support elements [because the coherence conditions restrict how substitutions can act on them], and refer to all other near-litter support elements as {\em flexible\/} support elements.

\item  We then complete the construction of $\tau_\alpha$ by defining it as the collection of codable elements of $\tau^3_\alpha$, and choosing a well-ordering $<_\alpha$ of $\tau_\alpha$ satisfying the stated conditions.

\item Specific elements of $\tau_\alpha$'s whose existence was postulated above need to be shown to be codable.  $\emptyset_\gamma$ is coded by $(\emptyset,\emptyset)$.  If $x$ is an atom, $\{x\}_\gamma$ is coded by $(\{(x,\{\gamma,-1\})\},\{x\})$.  If $N$ is a near-litter,
$N_\gamma$ is coded by $(\{(\{N_\gamma,\{\gamma\}),N\}$.

\item  We describe the construction of the well-orderings $<_\iota$ in detail.

We construct an order on the typed atoms and near-litters in $\tau_0$ alone, starting by specifying an arbitrary well-ordering of all litters.  At each of $\mu$ stages,
we take the next litter $L$ in the arbitrary order, add $L_0$ to $<_0$, followed by $\{x\}_0$ for each $x \in L$, followed by $N_0$ for each $N$ such that $N^{\circ}$
appears earlier in the order and $\{y\}_0$ appears earlier in the order for each $y \in N \Delta N^\circ$.  Notice that $<\mu$ objects are added at each stage, and that every typed near-litter will eventually be added because the cofinality of $\mu$ is at least $\kappa$.

The order $<_{-1}$
is induced by the order on typed atoms in $\tau_0$ just described.  In $<_0$ (whose full construction is included in the general construction below) the typed singletons and near-litters are placed in the even positions in $\mu$ in the order just described.

We describe how to construct all $<_\alpha$ for $\alpha\geq 0$.  We collect the extensional type $\alpha$ sets which are codable and designate a code for each one (axiom of choice) and convert the included support to a strong support as described above. We place the typed atoms and near-litters in even positions in $<_\alpha$ in the same positions at which the typed atoms and near-litters with the same $-1$-extensions are placed in
$<_0$ (we described this above).  We provide ourselves with an arbitrary well-ordering of the other sets in type $\alpha$ of order type $\mu$.  At each step, we go to the first unfilled position and choose the first set in the arbitrary ordering from $\tau_\alpha$ whose designated strong support does not include any support element whose first element [typed atoms standing in for atoms here]  is at a later position in the well-ordering of the type to which it belongs and place it there.  Every code is eventually placed, so the entire well ordering is filled (any given item will eventually be placeable because of the cofinality of $\mu$ being at least $\kappa$ and the fact that supports are small).

\item All that is needed to ensure that this works is the assurance that there are no more than $\mu$ codes, which ensures that there are exactly $\mu$ elements of each type.

\item  Note that a $\beta$-code is determined by a support $S$ and a set of orbits in the $\gamma$-codes (for some $\gamma<\beta$) over substitutions whose action fixes $S_\gamma$.

There are $\mu$ supports:  this is a consequence of the fact that there are $\mu$ near-litters, which is in turn a consequence of the fact that $\mu$ is of cofinality at least $\kappa$.

We claim that there are $<\mu$ orbits in the $\gamma$-codes under substitutions whose action fixes $S_\gamma$.  Note that if we can show this, there are $<\mu$ sets of such orbits because $\mu$ is strong limit  (for each choice of $\gamma$ and so for all choices of $\gamma$ because the number of choices for $\gamma$ is $<\lambda<\kappa<\mu$), and so $\leq \mu$ $\beta$-codes and $\leq \mu$ elements of $\tau_\beta$ (there are obviously at least $\mu$ elements
of $\tau_\beta$), so the claim would establish $|\tau_\beta| = \mu)$.

An orbit in the $\gamma$-codes over substitutions fixing $S_\gamma$ is determined by an orbit in the $\gamma$-supports over such substitutions, a $\delta<\gamma$, and a set of orbits in the $\delta$-codes under substitutions fixing $T_\delta$ for a fixed  $T$ in the orbit in $\gamma$-supports.
We can suppose inductively that there are $<\mu$ choices for the last set of orbits (since $\delta<\gamma$; we will check the base case below). 

Now we consider the number of orbits of a $\gamma$-support under the action of permutations fixing a given support.  The number of orbits of a $\gamma$ support will be less that or equal to the number of orbits of a well-ordering of a $\gamma$-support:  we consider the orbits over such well-orderings.
Each item in the well-ordered $\gamma$ support belongs to a class in which there are $<\mu$ orbits under the action of substitutions fixing any given support:  an atomic support element has $<\kappa$ orbits (orbits for singletons and near-litters in the (small) support and the complement of their union),
a near litter support element which is not an image under a relevant $f$ map has $<\kappa$ orbits for the same reason, and a near-litter support element which is an image under a relevant $f_{\delta,\epsilon}$ has orbits correlated with the $<\mu$ [by ind hyp since $\epsilon<\gamma$] orbits of the $\epsilon$-code for its preimage.  Now to choose an orbit for an entire well-ordering, we have at each step a number of choices equal to the number of orbits for the next item under permutations which fix the union of the original support we were working with and the supports of all previously chosen items in the well-ordering (all restricted to relevant types, important for the inductive hypothesis to remain applicable), thus,
by inductive hypothesis, $<\mu$ choices at each of $<\kappa$ stages, so $<\mu$ orbits for the entire well-ordering.  [This is similar in spirit to the approach taken in the previous paper, but much more abstract and thereby more succinct].

The orbits in the action of a substitution on 0-codes can be described directly.  Any support in a 0-code can be modified to consist of a collection of tagged atoms and near-litters not containing any of the atoms, so the orbit is simply determined by how many atoms
there are and how many near-litters there are, giving $<\kappa$ orbits (because supports are small).  The set of atoms which is the second component of the code will be a union of orbits in the atoms under the action of a permutation fixing the support elements, which are just
the individual elements of the support and the complement of their union, so there are $<2^{\kappa}<<\mu$ such sets, and $<\mu$ orbits in the 0-codes.  Similar considerations apply to counting orbits in the sets with nonempty $-1$-extensions in each type.  This handles the basis of the induction.

We have established that there are $<\mu$ codes and so that $\tau_{\beta}$ is of cardinality $\mu$ for each $\beta\leq \alpha$.

We are being very terse here, this can be laid out more expansively we are sure.

\item  At this point, the description of the structure is complete and its existence has been established.  We have not yet shown that it supports an interpretation of TTT.

\item Our criterion for acceptability of codes enforces a high degree of symmetry, assuming that substitutions act fairly freely on our structure.  We state a theorem about this.

A {\em $\beta$-partial substitution\/} is an injective map $\sigma$ from $\beta$-support items to $\beta$-support items with domain and range the same, satisfying $\pi_2(\sigma(x,A)) = A$, satisfying for each litter $L$ and each $A$ with minimum $-1$ that the set $\{(x,A): x \in L \wedge (x,A) \in {\tt dom}(\sigma)\}$ is small, and satisfying that each $(N_\delta,A)$
in the domain of $\sigma$ has $x$ is not an image under $f_{\gamma,\delta}$ for any $\gamma$ dominated by all elements of $A$ except $\delta$ (near-litter support elements satisfying this condition are said to be {\em flexible\/}; ones that do not are {\em inflexible}).

 We say that an atomic support element $(x,A)$ is an {\em exception\/} of a substitution $\sigma$ iff it satisfies the following condition:
let $L$ be the litter containing $x$; either $$\pi_1(\sigma(x,A)) \not\in \pi_1(\sigma[(L,A \setminus \{-1\})])^\circ$$ or $$\pi_1(\sigma^{-1}(x,A)) \not\in \pi_1(\sigma^{-1}[(L,A \setminus \{-1\})])^\circ.$$

The Freedom of Action theorem asserts that for each partial substitution there is a substitution which extends it and has no exceptions other than elements of its domain.



\end{enumerate}


\end{document}

