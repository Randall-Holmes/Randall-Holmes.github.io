\documentclass[12pt]{book}

\usepackage{bussproofs}
\usepackage{amssymb}
\usepackage{latexsym}
\usepackage{makeidx}
\usepackage{comment}

\makeindex

\title{Proof, Sets, and Logic}

\author{M. Randall Holmes}

\date{version of 12/31/2023}

\begin{document}

\maketitle

\newpage

\begin{centering}
For Jonathan\\
\end{centering}

\newpage

\tableofcontents

\newpage

\section{Introduction}

{\tiny

\subsection{Version Notes}

These are notes to myself as the editor of the document.  I will
highlight changes which actually affect material currently being
lectured (or past material), which will of course also be of interest
to current students.

\begin{description}

\item[12/31/2023:]  Added a new section 2.2 giving basics of the unsorted approach to type theory.  The appendix remains as a more thorough discussion of the same approach;  some reconciliation of notations between the new section 2.2 and the appendix may be wanted.  Discussion of technical issues about the untyped theory in later chapters may be desirable.

\item[5/15/2023:]  Careful readers noticed an error in the proof that all infinite sets have countable subsets.  8/3/2023 checked again, it is correct.

\item[9/14/2021:]  Added a subsection on arguments and added discussion of argument manipulations in Marcel to the basic discussion of propositional logic and quantifiers.

This strongly suggests that Marcel examples and exercises should appear in the proof section.

Marcel's treatment of equality?

Moved the unsorted preamble to the end.

\item[4/4/2021:]  added section 3.9.2 about unsorted type theory with proper classes...

\item[4/2/2021:]  C. Ryan-Smith pointed out some corrections needed in section 2.1.1.  Thank you!

\item[3/30/2021:]  minor edits

\item[3/28/2021:]  More edits, adding proof of the left distributive law in formal arithmetic.

\item[3/26/2021:]  complete rewrite and rearrangement of the suggestion of an unsorted foundation for the type theory of chapter 2.

\item[3/2/2021:]  (some of these dhanges already posted on the first):  new material on rules for reasoning about sets, and proof strategies for reasoning about sets, and a new homework set.

\item[2/11/201:]  some revised and additional solutions.  Allowed a dummy line $a=a$ at the beginning of a UG block (actually, reflexivity of equality already allows this line!)

\item[2/9/2021:]  logic homework updates

\item[2/4/2021:]  Added another problem on p. 42

\item[1/13/2021:]  Added a problem to the exercises on p. 42.

\item[1/11/2021:]  Just signalling the new term of Math 402/502 at this point.  Changes in the text will be signalled if made.

\item[3/24/2019:]  Corrected an error in the forcing section which I am reading and may soon revise.  Inserted the alternative presentation of type theory as an unsorted theory.

\item[12/6/2017:]  The proof of independence of CH by forcing is added;  I hadn't realized that I hadn't written it in the notes.

There is an indication of a considerable simplification of my forcing treatment, by restricting the construction of names, which appears to eliminate the atom problem without reintroducing the mutual recursion issue.

\item[12/5/2017:]  Notes on independence of choice from ZFA added.  Some exercises of an unsatisfactory nature are given:  some of them are things you might be able to do.

\item[12/4/17:]  Fixed stupid typo in an exercise.  Repaired definition of $D^{\infty}$.

\item[12/1/2017:]  Fixed error in the definition of $x_p$ in the forcing section.  Expanded on the discussion under one of the problems.

\item[11/29/2017:]  Notes on theories with proper classes added.

\item[11/16/2017:]  There are now exercise sets at the ends of sections 3.8 and 3.10.  I have no confidence in these, but one must ask something!

\item[11/15/2017:]  Removed the first approach to collapsing names and cleaned up typos and editing disasters found during lecture.   Also added headings for paragraphs so that it is easier to see the structure of the text.

\item[11/14/2017:]  Further cleanup of the second approach to collapsing names.  There were considerable difficulties (including a lot of cut and paste errors) now, one hopes, fixed.  It was in an awful state:  apologies to anyone who tried to read it.

\item[11/13/2017:]  I wrote out the alternative approach to collapsing names in full.  I think it is preferable, and with a little extra notation not noticeably harder to follow.  It has the advantage that we talk about equality of collapsed names rather than a further equivalence relation on collapsed names.  6:30 pm debugging the  proof.

\item[11/10/2017:]  I have the construction of the forcing model {\em correctly\/} described (the description I gave in lecture was incorrect).  I'm less charmed with this approach than I was when I thought it worked the way I described it in lecture, but it {\em does\/} work.  3:30 pm typo  10 am on the 11th, typo fixes and a footnote detailing an alternative approach.

\item[11/9/2017:]  Considerable tightening of the initial part of the forcing section, up to the definition of name closures and an indication of the reasons that the name closure construction works.

\item[11/7/2017:]  slight tweak while preparing forcing slides.

\item[11/3/2017:]  Notes on forcing added.  These are still rough, but ought to see the light of day...  Removed section 5.1, which was implemented as section 3.8. 4 pm more proofreading of section 3.10.

\item[11/2/2017:]  aiming to install the $L$ lectures.  The $L$ lectures (at least a first draft) are now installed.  Comments are seriously invited!

\item[11/1/2017:]  several minor edits

\item[10/24/2017:]  Added rough notes up to the global well-ordering on L.

\item[10/21/2017:]  Added the note that Separation follows from Replacement plus the existence of the empty set, in the section where Replacement is introduced.  I added a proof of the Mostowski Collapsing Lemma at the end of the same section.

\item[10/18/2017:]  Typo fixes (thank you, students).  Fixed up the language which provides for predicates picking out every element of $D$ in the logically regimented sets construction:  0 codes sethood and $\{x\}$ codes the predicate which picks out just $x$.

\item[10/16/2017:]  corrected a typo.

\item[10/13/2017:]  Enhancements to the section on logically regimented constructions inspired by today's lecture.   It was not nearly as much in error as I thought;  I added the stipulation that there is a predicate associated with each element of $D$, and I added the formal definition of the constructible universe.

\item[10/11/2017:]  Adding exercises.

\item[10/10/2017:]  Adding infinite sums and products of cardinals and K\"onig's theorem.

\item[10/6/2017:]  Added enhancements from the lecture of today.  The incorrect statement of the definition of cofinality in lecture is corrected; I claim insomnia!

\item[10/4/17:]  correction in proof of $\kappa+\kappa=\kappa$.  This and other improvements in corresponding sections in both chapter 2 and chapter 3.

\item[10/2/17:]  corrected embarrassing misstatement of Zorn's Lemma in recent material.  Thanks to student :-)

\item[10/1/2017:]  Adding material on transfinite arithmetic of cardinals and ordinals in chapter 3.  This required a couple of corrections to the chapter 2 material as well.

\item[9/29/2017:]  12pm:  Updated the discussion of the cumulative hierarchy with the definition of hierarchy along a well-ordering given in class.  I'm still working on defining the homework assignment.  3 pm:  a few more revisions based on today's lecture, and homework problems added in sections 3.5 and some parts of 3.6.

\item[9/28/2017:]  More work on logically regimented constructions.  I need to think about possibly reconciling the current section 3.8 and chapter 5 eventually.

\item[9/27/2017:]  drafting a section on representing sets associated with general logical formulas.  I'll be using this soon, but I think it will need more updates.

\item[9/25/2017:]  tightened up the text in all of chapter 3.  Added a new section on transfinite induction and recursion for a pending lecture.

\item[9/23:]  added lecture on Zorn's Lemma and the definition of cardinality.

\item[9/22:]  minor typo fix.  4:30 pm:  finished notes for the Wednesday lecture (added to section 3.4); notes for the Friday lecture will be coming shortly as a new section 3.5.

\item[9/19/17, 9/20/17:]  Fixed some typos interchanging $\oplus$ and $\otimes$ in the discussion of counting sets in the untyped set theory section.  Another typo of the same kind fixed on the 20th.

\item[9/16/2017, 6 pm, 7 pm:]  Most notes for the lecture of the 15th are written.  Exercises should appear sometime later this evening or tomorrow.  7 pm exercises on p. 238-9 ready.

\item[9/15/2017, 5 pm:]  I have completed the section on counting finite sets, with the discussion of the definitions of cardinal addition and multiplication.  I still need to write material covering the lecture on the 15th, and only after that will I post exercises.

\item[9/6/2017:]  Expansions and corrections motivated by the 9/6 lecture.

\item[9/5/2017:]  Working on section 3.3.2 on the natural numbers and counting elements of finite sets.

\item[9/1/2017:]  Added more text to section 3.3.1 on implementing the natural numbers in set theory, with exercises.  3 pm version suggested an alternative approach to one of the problems which does not work (as I found on trying to write it out);  the 3:30 version corrects this.

\item[8/31/2017:]  Adding notes about arithmetic from the 8/30 lecture.

\item[8/27/2017:]  Corrected some terminology in section 2.24 on category theory to standard form, in response to a student comment.

\item[8/25/2017:]  Changed from article format to book format.  Revisions mostly of section 3.2.1 inspired by class presentation of this section (fixing chapter 2 language there).  Added section 3.2.2 of exercises.

\item[8/24/2017:]  Very minor layout changes.
232
\item[8/23/2017:]  Revisions on first day of class.  Moved daggered subsections of chapter 3 to a separate section on bridges between untyped and typed set theory at the end.

\item[8/22/2017:]  Duplicating some of the development in the typed set theory chapter in the untyped set theory chapter to support Math 522 instruction, notably terminology related to relations and functions.

\item[7/10/2017:]  Added a discussion of bounded separation in the list of Zermelo axioms.  Started writing the section on logically motivated set constructions in chapter 5.

\item[7/6/2017:]  Various edits through the existing text.  Started working on the index.  Added some empty sections which will need to be filled in for my Math 522 intentions and some comments on already existing sections which are empty or partial which will need development for Math 522.  One point is that I'll be doing chapter 5 work in untyped set theory which I originally intended to do in type theory:  what I can do is present it in a way that it is clear how to do it in both approaches.

\item[7/5/2017:]  Added treatment of the Hilbert symbol and definite description operator in the first part, with footnotes on how to treat it in type theory if it is adopted.  6:30 pm working on refinements of the untyped set theory chapter.

\item[6/1/2017:]  I'm doing wildly speculative things in the category theory section.

\item[5/31/2017:]  Clearing the version notes at the end of the Spring 2017 class taught from these notes.  1:30 pm edits up to the beginning of the section on number systems.  I'm planning to review and extend the document this summer.  3:30 pm a few more edits here and there.

\end{description}}

\newpage

\subsection{Introductory Remarks}

This is being written as a textbook for Math 502, Logic and Set Theory, and Math 522, Advanced Set Theory,
at Boise State University, on the practical level.  \footnote{currently being used for Math 522 for the first time in Fall 2017, which will require addition of much new material!}

On the Platonic level, this is intended to communicate something about
proof, sets, and logic.  It is about the foundations of mathematics, a
subject which results when mathematicians examine the subject matter
and the practice of their own subject very carefully.

The ``proof'' part refers to an informal discussion of the practice of mathematical reasoning (not all {\em that\/} informal) which will serve as a
springboard for the ``logic'' component.  It also introduces formal
notation for logic.

The ``sets'' part refers to a careful development of mathematical
ontology (a fancy word for ``what are we talking about''?): familiar
mathematical concepts are analyzed in terms of the fundamental concepts 
of {\em set\/} and {\em ordered pair\/}.  This chapter gives us an opportunity to practice
the proof skills of which chapter 1 provides an overview.  A
distinctive feature of our development is that we first develop basic
concepts of set theory in a typed theory of sets, then make the
transition to the more usual untyped set theory in a separate chapter.

The ``logic'' part refers to a much more formal discussion of how we
prove things, which requires both the ``proof'' and ``sets''
components to work properly, and in which bits of language (sentences
and noun phrases) and proofs are actually mathematical objects.

All of this is supported by some software: the formal logic introduced
in chapter 4  (and one of the alternative set theories introduced in
chapter 6) are the logic of our sequent theorem prover Marcel\index{Marcel theorem prover}, to
which we will have occasion to refer, and which will be used for some
lab exercises.  We hope to find that experience with Marcel will
assist the learning of formal logic.

The final chapter on alternative set theories will probably not be
reached in the course (or in a first course, at any rate) but has some bearing on other ways we could get
from type theory to set theory and on the way set theory is
implemented in Marcel.

\newpage

\chapter{Proof}

In this chapter we discuss how we make ``formal proofs'' (really, as
we will see in the Logic chapter, rather {\em informal\/} proofs) in
English, augmented with formal notation.

Our framework is this.  We will identify basic logical structures of
statements.  Statements have two fundamental roles in proofs which
need to be carefully distinguished: there are statements which we are
trying to deduce from our current assumptions, which we will call
``goals'',\index{goal (statements)} and there are statements already assumed or deduced from
the current assumptions which we are allowed to use, which we will
call ``posits''.\index{posit(ed statements)}  The reason we call these last ``posits'' instead of
something like ``theorems'' or ``conclusions'' is that posits may be
consequences of statements which we have only assumed for the sake of
argument: a posit is not necessarily a theorem.  For each basic
logical structure, we will indicate strategies for deducing a goal of
that form (from the currently given posits) and strategies for using a
posit of that logical form to deduce further consequences.  Further,
we will supply formal notation for each of the basic logical
structures, and we will say something about the quite different
English forms which statements of the same underlying logical form may
take.

It is useful to note that my use of the word ``posit'' is eccentric;
this is not standard terminology.  We can adopt as a posit any current
assumption, any previously proved theorem, or anything which follows
logically from current assumptions and theorems.  We allow use of
``posit'' as a verb: when we adopt $A$ as a posit, we posit $A$ (to
posit is either to {\em assume\/} for the sake of argument or to
{\em deduce\/} from previous posits).

We are trying to say carefully ``deduce''\index{deduce (a statement) -- contrasted with ``prove"} rather than ``prove''\index{prove (a statement)} most
of the time: what we can {\em prove\/} is what we can deduce without
making any assumptions for the sake of argument.

\section{Basic Sentences}

Sentences in mathematical English (being sentences of natural language) have
subjects, verbs and objects.  Sentences in formal mathematical
language have similar characteristics.  A typical mathematical
sentence already familiar to you is $x < y$ (though we will see below
that we will usually call this particular (grammatical) sentence a
``formula''\index{formula (contrasted with ``sentence")} and not a ``sentence'' when we are being technical).  Here
$x$ and $y$ are noun phrases (the use of letters in mathematical
notation is most analogous to the use of {\em pronouns\/} in English,
except that for precision of reference mathematical language has a lot
more of them).  $<$ is the verb, in this case a transitive verb with
subject and object.  In the parlance of mathematical logic, a
transitive verb is called a ``binary predicate''.\index{binary predicate}\index{predicate, binary}  Another typical
kind of mathematical sentence is ``$x$ is prime''.  Here the verb
phrase ``is prime'' is viewed as an intransitive verb (we don't
distinguish between adjectives and intransitive verbs as English
does).  We can't think of examples of the use of intransitive verbs in
mathematical English, though we are sure that they do exist.  An
adjective or intransitive verb is a ``unary predicate''\index{unary predicate}\index{predicate, unary} in
mathematical logic.  Two commonly used words in mathematical logic
which have grammatical meanings are ``term''\index{term (noun phrase)} and ``formula''\index{formula (grammatical sentence)}: a
``term'' is a noun phrase (for the moment, the only terms we have are
variables, but more term constructions will be introduced as we go on)
and a ``formula'' is a sentence in the grammatical sense (``sentence''
in mathematical logic is usually reserved for formulas not containing
essential references to variables: so for example $x < y$ is a formula
and not (in the technical sense) a sentence, because its meaning
depends on the reference chosen for $x$ and $y$, while $2<3$ is a
formula and a sentence (no variables) and $(\exists x.x<2)$ is a
formula and a sentence (the $x$ is a dummy variable here)\index{sentence (constrasted with ``formula")}).  What we
call ``basic sentences'' (using terminology from grammar) in the title
of this section will really be called ``atomic formulas''\index{formula, atomic}\index{atomic formula} hereinafter.

The English word ``is'' is tricky.\index{is, uses of English word}  In addition to its purely formal
use in ``$x$ is prime'', converting an adjective to a verb phrase, it
is also used as a genuine transitive verb in formulas like ``$x$ is
the square of $y$'', written $x=y^2$ in mathematical language.  The =
of equality is a transitive verb (as far as we are concerned: it is
not treated the same by English grammar) and also part of our basic
logical machinery.\index{equality predicate}

The English word ``is'' may signal the presence of another binary
predicate.  A formula like ``$x$ is a real number'' may translate to
$x \in {\mathbb R}$, where $\in$ is the predicate of {\em
membership\/} and ${\mathbb R}$ is the name of the set of all real
numbers.  For that matter, the formula ``$x$ is prime'' could be read
$x \in {\mathbb P}$ where ${\mathbb P}$ is here supposed to be the set
of all prime numbers.\index{membership predicate}

In our formal language, we use lower case letters as variables\index{letters, uses of, reviewed}
(pronouns).  There will be much more on the care and feeding of
variables later on\index{variables}.  Some special names for specific objects will be
introduced as we go on (and in some contexts lower case letters
(usually from the beginning of the alphabet) may be understood as
names (constants)).  Capital letters will be used for predicates.
$P(x)$ (``$x$ is $P$'') is the form of the unary predicate formula.
$x \, R \, y$ is the form of the binary predicate formula.  Predicates
of higher arity could be considered but are not actually needed\index{predicates of arity higher than 2 not needed}\index{ternary predicates not needed}\footnote{The precise point here is that we do not require ternary predicates if we have a notion of ordered pair\index{ordered pair, used to eliminate higher arity predicates}, as $T(x,y,z)$ ($T$ a hypothetical ternary predicate) can be understood as abbreviating $T(x,\left<y,z\right>)$, and predicates with four or more arguments can be reduced to binary predicates similarly.}: a
ternary predicate formula might be written $P(x,y,z)$.  The specific
binary predicates of equality and membership are provided: $x=y$, $x
\in y$ are sample formulas.  Much more will be heard of these
predicates later.

We will have another use for capital letters, mostly if not entirely
in this Proof part: we will also use them as variables standing for
sentences.\index{variables, sentence, capital letters used for} We use variables $A$, $B$, $C$ for completely arbitrary
sentences (which may in fact have complex internal structure).  We use
variables $P$, $Q$, $R$ for propositions with no internal structure
(atomic formulas).  Once we get to the chapters on set theory we will
once again allow the use of capital letters as variables representing
objects (usually sets):  the grammar of our language will prevent confusion between capital letters used as terms
and capital letters used as unary or binary predicates.


\newpage

\section{Arguments}

An {\em argument\/} is a structure $P_1,\ldots,P_n \vdash C$, where the $P_i$'s and $C$ are sentences in our formal language, a pair consisting of a finite list of sentences $P_i$, called {\em premises\/}, and a single sentence $C$ called the {\em conclusion\/}.

A valid argument is an argument with the property that there is no global assignment of values to all variables appearing in the sentences which makes all the premises true and the conclusion false.

An argument may also be presented in the format

$$\begin{array}{c}

P_1 \\

P_2 \\

.\\
.\\
.\\

P_n \\ \hline

C

\end{array}$$

Valid arguments of particular importance as building blocks of reasoning (formalized or less formal) in mathematics may be introduced as {\em logical rules\/}.

The theorem proving system Marcel which we will occasionally reference here is designed to demonstrate validity of arguments, and relies on a set of basic maneuvers for reducing validity of an argument to validity of a simpler argument or small set of simpler arguments.

A more general related notion is that of a {\em sequent\/}:  a sequent is of the form $\Gamma \vdash \Delta$ where
$\Gamma$ and $\Delta$ are finite sets of sentences.  A sequent is valid if and only if any global assignment of values to variables which makes all sentences in $\Gamma$ true makes some sentence in $\Delta$ true.

It can be noted that an argument $A,P_2,\ldots,P_n \vdash A$ is always valid, that reordering premises will not affect validity or invalidity of an argument, and that asserting that $\vdash A$ is valid is simply asserting that $A$ is true no matter what values are assigned to variables.

\newpage

\section{Conjunction}

This brief section will review the mathematical uses of the simple
English word ``and''.\index{conjunction (and)}\index{and (conjunction)}  The use of ``and'' as a conjunction to link
sentences is what is considered here.  If $S$ is ``snow is white'' and
$G$ is ``grass is green'', we all know what ``snow is white and grass
is green'' means, and we formally write $S \wedge G$.

Certain English uses of ``and'' are excluded.\index{and, other uses of the English word}  The use of ``and'' to
link noun phrases as in ``John and Mary like chocolate'' is not
supported in mathematical language.  This use does have a close
connection to the logical ``and'': the sentence is equivalent to
``John likes chocolate and Mary likes chocolate''.  One should further
be warned that there is a further complex of uses of ``and'': ``John
and Mary went out together'' does not admit the logical analysis just
given, nor (probably) does ``John and Mary moved the half-ton safe''.
There is an example of the nonlogical use of ``and'' in mathematical
parlance: there is a strong tempation to say that the union of two
sets $a$ and $b$, $a \cup b$, consists of ``the elements of $a$ and
the elements of $b$''.\index{and, misleading use of in connection with union}  But $x \in a \cup b$ is true just in case $x
\in a$ {\em or\/} $x \in b$.  Another example of a use of ``and''
which is not a use of $\wedge$ is found in ``$x$ and $y$ are
relatively prime''.

We note and will use the common mathematical convention whereby $t
\,R\,u \,S\, v$ is read $t\,R\,u \wedge u \,S\,v$, as in common
expressions like $x=y=z$ or $2 < 3 \leq 4$.  This chaining can be
iterated:
$$t_0 \, R_1 \, t_1 \, R_1 \,t_2 \, \ldots \, t_{n-1} \, R_n \, t_n$$
can be read $$t_0 \,R_1 \, t_1 \wedge t_1 \,R_2 \, t_2 \wedge \ldots \wedge t_{n-1} \,R_n\,t_n.$$\index{chaining of relations, implicit conjunction via}

\begin{description}

\item[Proof Strategy:]

To deduce a goal of the form $A \wedge B$, first deduce the goal $A$,
then deduce the goal $B$.

This rule can be presented as a rule of inference $$\begin{array}{l} A \\ B \\ \hline A \wedge B \end{array}$$

We call this rule {\em conjunction introduction\/} (or just {\em conjunction\/}) if a name is needed.\index{conjunction (logic rule)}

If you have posited (assumed or deduced from current assumptions) $A
\wedge B$, then you may deduce $A$ and you may deduce $B$.

 This can be summarized in two rules of inference:

$$\begin{array}{l} A \wedge B \\ \hline A \end{array}$$

$$\begin{array}{l} A \wedge B \\ \hline B \end{array}$$

We call this rule {\em simplification\/} if a name is needed.\index{simplification (logic rule)}

\end{description}

The operation on propositions represented by $\wedge$ is called {\em
conjunction\/}: this is related to but should not be confused with the
grammatical use of ``conjunction'' for all members of the part of
speech to which ``and'' belongs.

An argument $A \wedge B,P_2,\ldots, P_n\vdash C$ is valid if and only if the argument $A,B,P_2,\ldots, P_n\vdash C$ is valid.

An argument $P_1,\ldots,P_n \vdash A \wedge B$ is valid if and only if both of the arguments $P_1,\ldots,P_n \vdash A$ and $P_1,\ldots,P_n \vdash B$ are valid.

These statements about manipulation of arguments correspond quite precisely to the way to use posited conjunctions and the way to prove a conjunction.

\section{Disjunction}

This subsection is about the English word ``or''.\index{disjunction(and/or)}\index{or (inclusive, disjunction)}

Again, we only consider ``or'' in its role as a conjunction linking
sentences; the use of ``or'' in English to construct noun phrases has
no analogue in our formal language.

When we say ``$A$ or $B$'' in mathematics, we mean that $A$ is true or
$B$ is true {\em or both\/}.  Here we draw a distinction between
senses of the word ``or'' which is also made formally by lawyers: our
mathematical ``or'' is the ``and/or''\index{and/or} of legal documents.  The
(presumably) exclusive or\index{or, exclusive} of ``You may have chocolate ice cream or you
may have vanilla ice cream'' is also a logical operation of some
interest but it is not yet introduced here.

We write ``$A$ or $B$'' as $A \vee B$, where $A$ and $B$ are sentences.


\begin{description}


\item[Proof Strategy:] To deduce a goal $A \vee B$, deduce $A$.  To
deduce a goal $A \vee B$, deduce $B$.  These are two different
strategies.  

This can also be presented as a rule of inference, which comes in two different versions.

$$\begin{array}{r} A \\ \hline A \vee B \end{array}$$

$$\begin{array}{r} B \\ \hline  A \vee B \\ \end{array}$$

The rule is called {\em addition\/} if a name is needed.\index{addition (logical rule)}

We will see below that two more powerful strategies
exist (generalizing these two): To deduce a goal $A \vee B$, assume
$\neg A$ (``not $A$'') and deduce $B$; To deduce a goal $A \vee B$,
assume $\neg B$ and deduce $A$.  We call both of these rules {\em disjunction introduction\/} (or {\em alternative elimination\/}).\footnote{It is a common error (or redundancy at least) to present proofs of a disjunction by alternative elimination in both forms, by a false analogy with the method of proof by cases.}\index{disjunction introduction (alternative elimination:  logical rule)}\index{alternative elimination (disjunction introduction:  logical rule)}

For a fuller discussion of this kind of proof strategy which involves
the introduction of an additional assumption, see the subsection on
implication below (and for more about negation see the section on
negation below).

To use a posit $A \vee B$ (assumed or deduced from the current
assumptions) to deduce a conclusion $G$, we use the strategy of {\em
proof by cases\/}: first deduce $G$ from the current assumptions with
$A$ replacing $A \vee B$, then deduce $G$ from the current assumptions
with $B$ replacing $A \vee B$ [both of these proofs are needed].\index{proof by cases (logical rule)}

\end{description}

The operation on propositions represented by $\vee$ is called {\em
disjunction\/}.

An argument $A \vee B,P_2,\ldots,P_n \vdash C$ is valid iff both the arguments $A,P_2,\ldots,P_n \vdash C$ and $B,P_2,\ldots,P_n \vdash C$ are valid.  This is another way to state the strategy of proof by cases.

An argument $P_1,\ldots,P_n \vdash A \vee B$ is valid iff the argument $P_1,\ldots,P_n,  \neg B^* \vdash A$ is valid,
where $\neg B^*$ is defined as $D$ if $B$ is a negation $\neg D$ and otherwise is defined as $\neg B$.  This expresses the derived proof strategy for disjunction called {\em alternative elimination\/}, which will be discussed further below.




\section{Implication}

The sentences ``if $A$, then $B$'', ``$B$ if $A$'', ``(that) $A$ (is
true) implies (that) $B$ (is true)'' all stand for the same logical
construction.\index{implication (if$\ldots$then$\ldots$)}\index{if$\ldots$then$\ldots$ (implication)}  Other, specifically mathematical forms of the same
construction are ``(that) $A$ (is true) is sufficient for $B$ (to be
true)'' and ``(that) $B$ (is true) is necessary for $A$ (to be
true)''.  We provide optional padding phrases in parentheses which are
needed in formal English because a proposition cannot grammatically
live in the place of a noun phrase in an English sentence.  Our formal
notation for any of these is $A \rightarrow B$.\index{implication, other English paraphrases of}\index{necessary}\index{sufficient}

Don't spend a lot of time worrying about ``necessary''
vs. ``sufficient'' for purposes of reading this text -- I only
occasionally use them.  But other writers use them more often; if you
are going to read a lot of mathematics you need to know this
vocabulary.

It is important to notice that unlike previous (and subsequent) constructions this one
is not symmetrical: ``if $A$, then $B$'' is not equivalent to ``if $B$,
then $A$''.

\begin{description}

\item[Proof Strategy:] To deduce a goal $A \rightarrow B$, assume $A$
(along with any other assumptions or previously deduced results
already given in the context) and deduce the goal $B$.  Once the goal
$B$ is proved, one withdraws the assumption that $A$ and all consequences deduced from it (it is local to
this part of the proof).  The same remarks apply to the negative
assumptions introduced in the rule of alternative elimination for proving
disjunctions indicated above.  

We call this rule {\em deduction\/}.\index{deduction (logical rule)}

An alternative strategy for proving $A
\rightarrow B$ (called ``indirect proof" or ``proving the contrapositive'') is justified in
the section on negation: assume $\neg B$ and adopt $\neg A$ as the new
goal.\index{indirect proof (logical rule)}\index{contrapositive, proving the (logic rule)}

A posit of the form $A \rightarrow B$ is used together with other
posits: if we have posited $A \rightarrow B$ and we have also posited
$A$, we can deduce $B$ (this rule has the classical name {\em modus
ponens\/}).  We will see below that we can use posits $A \rightarrow
B$ and $\neg B$ to deduce $\neg A$ as well (the rule of {\em modus
tollens}).\index{modus ponens (logic rule)}\index{modus tollens (logic rule)}

Another way to think of this: if we have a posit $A \rightarrow B$ we
can then introduce a new goal $A$, and once this goal is proved we can
deduce the further conclusion $B$.  [or, following the pattern of {\em
modus tollens\/}, we can introduce a new goal $\neg B$, and once this
goal is proved we can deduce $\neg A$].

\end{description}

The operation on propositions represented by $\rightarrow$ is called
{\em implication\/}.

The additional strategies indicated in this section and the section on
disjunction which involve negation ($\neg$) will be further discussed
in the section on negation below.

An argument $P_1,\ldots, P_n \vdash A \rightarrow B$ is valid exactly if $A,P_1,\ldots, P_n \vdash B$ is valid.  This expressed our basic strategy for proving an implication, the rule of deduction.

An argument $A \rightarrow B,P_2,\ldots,P_n \vdash C$ is valid iff $P_2,\ldots,P_n,\neg C^* \vdash A$ and $B,P_2,\ldots,P_n \vdash C$ are both valid.  This is the rule of modus ponens in disguise:  it is cast in an unfamiliar way because the argument strategies we present always act on a single premise, and the rule of modus ponens acts on two.


\section{Biconditional and Exclusive Or}

When we say ``$A$ if and only if $B$'', ``$A$ (being true) is
equivalent to $B$ (being true)'', ``$A$ exactly if $B$'', or similar
things we are saying that $A$ and $B$ are basically the same
statement.  Formal notations for this is $A \leftrightarrow B$.  We
have often used $\equiv$ for this operator elsewhere\footnote{which is an abuse, though others have used the symbol this way:  the usual meaning of $A \equiv B$ is that $A \leftrightarrow B$ is a tautology ($A$ and $B$ are logically equivalent\index{logical equivalence}\index{equivalence, logical})}, and the notation
of Marcel ({\tt ==}) is motivated by this alternative notation.  ``$A$
iff $B$'' is a learned abbreviation for ``$A$ if and only if $B$''
which is used in mathematical English.\index{biconditional}\index{iff}

\begin{description}

\item[Proof Strategy:]

To deduce a goal of the form $A \leftrightarrow B$, deduce $A \rightarrow B$ and
deduce $B \rightarrow A$.  Since there are at least two strategies for deducing these implications, there are a number of ways to structure the proof.

One can use a posit of the form $A \leftrightarrow B$ in a number of ways.
From posits $A \leftrightarrow B$ and $A$, we can deduce $B$; from posits $A
\leftrightarrow B$ and $B$ we can deduce $A$.  More powerfully, if we have
posits $A \leftrightarrow B$ and some complex $C[A]$, we can deduce $C[B]$
(simply replace occurrences of $A$ with $B$) or symmetrically from
posits $A \leftrightarrow B$ and $C[B]$ we can deduce $C[A]$\footnote{As a matter of pedagogy, we prefer that students not use the substitution rule for biconditionals in homework proofs in the Proof part of the book.  We will indicate specifically if we are allowing its use, or the use of specific kinds of substitution.}.\index{biconditional, proof strategies for}\index{substitution using biconditionals}\index{biconditional, substitutions using the}

\end{description}

The operation represented by $\leftrightarrow$ is called
the {\em biconditional\/}.

We note without pursuing the details at this point that $A \not\leftrightarrow
B$ (another commonly used notation is $A \oplus B$) is our notation
for the ``exclusive or'':  $A$ or $B$ is true but not both.\index{or, exclusive}

A common format for a theorem is to give a list of statements and
assert that all of them are equivalent.  A strategy for proving that
statements $A_1,\ldots,A_n$ are equivalent is to show that $A_i \rightarrow
A_{i+1\,{\tt mod}\, n}$ for each appropriate $i$ (showing that each
statement implies the next in a cycle).  In a theorem of this type
several linked cycles may be present.

We note that $(A \leftrightarrow B) \leftrightarrow C$ is equivalent to $A \leftrightarrow (B
\leftrightarrow C)$ but {\em not\/} equivalent to $(A \leftrightarrow B) \wedge (B
\leftrightarrow C)$ (there is an exercise about this later).\footnote{But $A \equiv B \equiv C$ does mean ``$A \leftrightarrow B$ is a tautology and $B \leftrightarrow C$ is a tautology", following the convention explained in  the conjunction section.}

An argument $A \leftrightarrow B,P_2,\ldots,P_n \vdash C$ is valid iff $$A \rightarrow B,B \rightarrow A,P_2,\ldots,P_n\vdash C$$ is valid.

An argument $P_1,\ldots,P_n \vdash A \leftrightarrow B$ is valid iff $P_1,\ldots,P_n,A \vdash B$ and $P_1,\ldots,P_n,B \vdash A$ are both valid.

These argument manipulations express quite precisely how to use and how to prove statements involving the biconditional, basically by reducing it to a conjunction of implications.

\section{Negation and Indirect Proof}

It is common to say that the logical operation of negation (the formal
notation is $\neg A$) means ``not $A$''.  But ``not $A$'' is not
necessarily an English sentence if $A$ is an English sentence.  A
locution that works is ``It is not the case that $A$'', but we do not
in fact usually say this in either everyday or mathematical English.

``Not snow is white'' is ungrammatical; ``It is not the case that snow
is white'' is pedantic; ``Snow isn't white'' is what we say.  If $R$
is a relation symbol, we will often introduce a new relation symbol
$\not{R}$ and let $x\, \not{R}\, y$ be synonymous with $\neg x \,R\,y$.
The use of $\neq$ and $\not\in$ should already be familiar to the
reader.

We do not as a rule negate complex sentences in English.  It is
possible to say ``It is not the case that both $A$ and $B$ are true''
but this is only a formal possibility: what we would really say is
``$A$ is false or $B$ is false''. It is
possible to say ``It is not the case that either  $A$ or $B$ is true''
but this is also only a formal possibility: what we would really say is
``$A$ is false and $B$ is false''.  The logical facts underlying these
locutions are the identities $$\neg(A \wedge B) \leftrightarrow (\neg
A \vee \neg B)$$ and
$$\neg(A \vee B) \leftrightarrow (\neg A \wedge \neg B),$$ which are known as
{\em de Morgan's laws\/}.  It is pure common sense that we do not need
to say ``It is not the case that it is not the case that $A$'', when
we can so easily say $A$ (the principle of double negation $\neg\neg A
\leftrightarrow A$).  $\neg(A \rightarrow B) \leftrightarrow A \wedge \neg B$ and
$\neg(A \leftrightarrow B) \leftrightarrow (A \not\leftrightarrow B)$ might require a little
thought.  The former is best approached via the equivalence of $A
\rightarrow B$ and $\neg A \vee B$ (which might itself require
thought); the result about the negation of $A \rightarrow B$ then
follows from de Morgan's laws and double negation.  Do please note
that we do not here authorize the use of these equivalences as proof
strategies (without proof): they are mentioned here only as part of
our discussion of the rhetoric of negation in mathematical English!\footnote{This is a special case of our generally not allowing the use of the substitution rule for biconditionals in homework proofs in this part of the book.}

We present a brief example from algebra.  To say $\neg(0 \leq x < 3)$ would be odd.  We analyze this step by step.
The chained relations hide a conjunction:  $\neg(0 \leq x \wedge x < 3)$.  De Morgan's law gives us $\neg 0 \leq x \vee \neg x < 3$.  Negating the binary predicates rather than the atomic formulas gives us $0 > x \vee x \geq 3$, and a further obvious (non-logical) transformation gives us $x < 0 \vee x \geq 3$.  Carrying out this kind of transformation reliably is expected of students in precalculus!

A statement of the form $A \wedge \neg A$ is called a {\em
contradiction\/}.  It is clear that such statements are always false.
It is a logical truth that $A \vee \neg A$ is always true (this is
called the law of {\em excluded middle\/}).

We introduce the notation $\perp$ for a fixed false statement, which we may call ``the absurd".

\begin{description}

\item[Proof Strategies:]

\begin{description}\item

 \item[1:] To deduce a goal of the form $\neg A$, add $A$
to your assumptions and deduce $\perp$, the absurd statement.
Notice that we will certainly withdraw the assumption $A$ and any posits deduced from it when this
proof is done!   We call this rule {\em negation introduction\/}.

\item[2:] From $A$ and $\neg A$, deduce $\perp$.  The only way to deduce the absurd is from a contradiction.  We call this rule {\em contradiction\/}.


\item[3:] From $\neg\neg A$, deduce $A$.  Otherwise, one can only use a negative
hypothesis if the current goal is $\perp$: if we have a posit
$\neg A$, use it by adopting $A$ as a goal (``for the sake of a
contradiction", so that $\perp$ can be deduced).   We call this rule {\em double negation elimination\/}.

\end{description}\end{description}

The first strategy above is {\em not\/} the notorious technique of
``proof by contradiction'': it is the direct strategy for proving a
negative sentence.  The strategy of proof by contradiction differs
from all our other techniques in being applicable to sentences of any
form:  it can be viewed as the strategy of last resort.

An argument of the form $P_1,\ldots,P_2 \vdash \neg A$ is valid iff $A,P_1,\ldots,P_2 \vdash \perp$ is valid.

In connection with $\perp$, we mention that a premise of the form $\neg \perp^*$ generated by any of our argument rules is simply omitted.

An argument of the form $\neg A,P_2,\ldots,P_n \vdash C$ is valid iff $P_2,\ldots,P_n,\neg C^* \vdash A$ is valid.

The first of these argument manipulations corresponds to negation introduction.  The second combines the contrapositive theorem proved below with double negation.  The definition of the construction $\neg C^*$ appeals to double negation.

\begin{description}

\item[Proof by Contradiction (reductio ad absurdum):] To deduce any goal $A$ at all, assume
$\neg A$ and reason to $\perp$ (by reasoning to a contradiction).  Notice that this is the same
as a direct proof of the goal $\neg\neg A$.  Our formal name for this rule is the classical {\em reductio ad absurdum\/}, since we have a rule above called ``contradiction".

\item[Principle of Double Negation:]  $\neg\neg P \leftrightarrow P$

\item[Proof:] Part 1 of the proof requires us to deduce $P$ given the
assumption $\neg\neg P$: this is given as a basic proof step above.
Part 2 requires us to deduce $\neg\neg P$ given the assumption $P$: to
do this, assume $\neg P$ and deduce $\perp$: but this is
immediate as we have already assumed $P$.  The proof is complete.

The derived rule ``from $P$, deduce $\neg\neg P$", may be called {\em double negation introduction\/}.

\end{description}

In later parts of the book we will not usually mention $\perp$, so the strategy for proving $\neg A$ will
generally be to deduce some contradiction $B \wedge \neg B$ from $A$ (from which the further deduction
of $\perp$ is immediate), and the strategy of proof by contradiction of $A$ will be to deduce some contradiction
$B \wedge \neg B$ from $\neg A$ (thus the name).

We prove that $P \rightarrow Q$ is equivalent to $\neg Q \rightarrow
\neg P$. 

\begin{description}

\item[Contrapositives Theorem:]  $(P \rightarrow Q) \leftrightarrow (\neg Q
\rightarrow \neg P)$

\item[Proof:]
This breaks into two subgoals: Goal 1 is to prove $(P \rightarrow Q)
\rightarrow (\neg Q \rightarrow \neg P)$ and Goal 2 is to prove $(\neg Q \rightarrow
\neg P) \rightarrow (P \rightarrow Q)$.

We prove Goal 1: $(P \rightarrow Q) \rightarrow (\neg Q \rightarrow \neg P)$.

This goal is an implication, so we assume for the sake of argument
that $P \rightarrow Q$:  our new goal is $\neg Q \rightarrow \neg P$.

The new goal is also an implication, so we assume $\neg Q$ and have our latest goal as $\neg P$.

To deduce $\neg P$ we need to assume $P$ and deduce $\perp$.
We duly assume $P$.  We have already assumed $P \rightarrow Q$, so
modus ponens allows us to conclude $Q$.  We have already assumed $\neg
Q$, so we can conclude $\perp$, which is the goal, which allows us to complete the deduction of our latest
goal $\neg P$, and so of the intermediate goal $\neg Q \rightarrow
\neg P$ and so of Goal 1.

Goal 2 remains to be proved: $(\neg Q \rightarrow \neg P) \rightarrow
(P \rightarrow Q)$.  To prove this we need to assume $(\neg Q
\rightarrow \neg P)$ and deduce an intermediate goal $P \rightarrow
Q$.  To deduce this goal, we need to assume $P$ and deduce a second
intermediate goal $Q$.  To prove $Q$, we assume $\neg Q$ and take as
our final intermediate goal $\perp$ (this is proof by
contradiction).  From $\neg Q$ and the earlier assumption $\neg Q
\rightarrow \neg P$ we can conclude $\neg P$ by modus ponens.  From
the earlier assumption $P$ and the recently proved $\neg P$ we
conclude $\perp$, completing the deductions
of all outstanding goals and the proof of the entire theorem.

\end{description}

\newpage

We present the proof of the Contrapositives Theorem a second time in a more exhaustive style with all lines carefully labelled and references to rules
used made explicit.  This is our first extended example of our proof techniques.

\begin{description}

\item[Contrapositives Theorem:]  $(P \rightarrow Q) \leftrightarrow (\neg Q
\rightarrow \neg P)$

\item[Proof:]

The statenment to be proved is a biconditional.  This dictates a proof plan in two parts.

\item[Part I:]

\begin{description}

\item[Assume (1):]  $P \rightarrow Q$

Each assumption gets a line number, because it is a posit and can be referenced in later applications of rules.

\item[Goal:]  $\neg Q \rightarrow \neg P$

A goal does not get a line number, but it plays an important role in proof planning.  In computer programming terms, it can be thought of as a comment.

\begin{description}

\item[Assume (2):]  $\neg Q$

\item[Goal:]  $\neg P$

\begin{description}

\item[Assume (3):]  $P$

\item[Goal:]  $\perp$

We have run out of ways to unpack our goals:  we need to look for a way to use our posits.  An opportunity presents itself!

\item[(4):]  $Q$  m.p. 1, 3 (m.p. abbreviates modus ponens)

\item[(5):]  $\perp$, contradiction 2,4

\end{description}

\item[ (6:] ) $\neg P$ negation introduction 3-5

In earlier versions of our logic style manual, we tended to omit these closing lines, assuming that it is clear when
goals have actually been met.  We have learned that students prefer closure!

\end{description}

\item [(7):] $\neg Q \rightarrow \neg P$  deduction 2-6

\end{description}

\item[(optional) (8):]  $(P \rightarrow Q) \rightarrow (\neg Q \rightarrow \neg P)$ deduction 1-7

We will present two ways of closing the entire argument, one using explicit references to the two implications
making up the biconditional, and one which uses references to the two blocks.

\newpage


\item[Part II:]

\begin{description}

\item[Assume (9):] $\neg Q \rightarrow \neg P$

The line number we use here is set of course after we see how long the first part is.  It could have been 8,
if we didn't use the last optional line.  It could also of course have been written at the beginning as something like
1b.  All that matter about line numbers is that different lines have different numbers, or at least that different lines
we are actually entitled to reference have different numbers.

\item[Goal:]  $P \rightarrow Q$

\begin{description}

\item[Assume (10):]  $P$

\item[Goal:]  $Q$

Now we are in a pickle.  The goal has no helpful structure and there is no obvious way to use the two posits in concert
(we cannot use {\em modus tollens\/} because in fact we are proving a theorem intended to justify {\em modus tollens\/}.)  When in doubt, use reductio ad absurdum!

\begin{description}

\item[Assume (11):] $\neg Q$ for the sake of a contradiction.

\item[Goal:]  $\perp$

\item[(12):]  $\neg P$  m.p. 9,11

\item[(13):]  $\perp$ contradiction 10,12

\end{description}

\item[(14):]  $Q$ reductio ad absurdum 11-13

\end{description}

\item[(15):]  $P \rightarrow Q$ deduction 10-14

\end{description}

\item[(optional)(16):]  $(\neg Q \rightarrow \neg P) \rightarrow (P \rightarrow Q)$  deduction 9-15

\item[(17):] $(P \rightarrow Q) \leftrightarrow (\neg Q \rightarrow \neg P)$ biconditional introduction 1-7, 9-15, {\bf or} biconditional introduction 8, 16.  Either style is acceptable; of course, if you use the first there is no reason to record line 8 or line 16.

\end{description}

Notice that we could replace the propositional letters $P$ and $Q$
with any statements $A$ and $B$, however complex, and the proof above
would still work: we have actually proved $(A \rightarrow B) \leftrightarrow
(\neg B \rightarrow \neg A)$.  This kind of generalization is the
subject of a subsection below.

This justifies proof strategies we have already signalled above.

\begin{description}

\item[Proof Strategy:]

To prove a statement $A \rightarrow B$, we can aim instead for the
equivalent $\neg B \rightarrow \neg A$: assume $\neg B$ and take $\neg
A$ as our new goal.  This is called {\em indirect proof\/}, or {\em proof by contrapositive\/}.

If we have posited both $A \rightarrow B$ and $\neg B$, then replacing
the implication with the equivalent $\neg B \rightarrow \neg A$ and
applying modus ponens allows us to conclude $\neg A$.  The rule ``From
$A \rightarrow B$ and $\neg B$, conclude $\neg A$'' is called {\em
modus tollens\/}, and we have justified it.

\end{description}

We prove another theorem which justifies some additional proof
strategies involving disjunction.

\begin{description}

\item[Theorem:]  $P \vee Q \leftrightarrow \neg P \rightarrow Q$

\item[Corollary:] $P \vee Q \leftrightarrow \neg Q \rightarrow P$.  This
follows from the theorem by equivalence of implications with their
contrapositives and double negation.

\item[Proof of Theorem:] For part 1 of the proof, we assume $P \vee Q$
and deduce Goal 1: $\neg P \rightarrow Q$.  The form of the posit
suggests a proof by cases.  

\begin{description}

\item[Case 1:]  We assume $P$.  We prove the
contrapositive of Goal 1: we assume $\neg Q$ and our goal is $\neg\neg
P$.  To prove $\neg\neg P$, we assume $\neg P$ and our goal is $\perp$, which is immediate as we have already posited $P$.
This completes the proof of case 1.

\item[Case 2:]  We assume $Q$.  To prove the goal $\neg P \rightarrow Q$, we
assume $\neg P$ and our new goal is $Q$.  But we have already posited
$Q$ so we are done.

\end{description}

For part 2 of the proof, we assume $\neg P \rightarrow Q$ and deduce
$P \vee Q$.  We prove the goal by contradiction: we assume $\neg(P
\vee Q)$ and take $\perp$ as our goal.  We do this by proving $P$ then proving $\neg P$.  Our first goal is $P$, which we prove by
contradiction: assume $\neg P$; by modus ponens $Q$ follows, from
which we can deduce $P \vee Q$, from which with our assumption
$\neg(P \vee Q)$ we can deduce $\perp$, completing the proof of $P$ by contradiction.  Our
second goal is $\neg P$: to prove this we assume $P$ and take $\perp$ as our goal; from the assumption $P$ we can deduce $P
\vee Q$ from which with our assumption
$\neg (P \vee Q)$ we can deduce $\perp$; this completes the proof of $\neg P$, which
completes the proof by contradiction of $P \vee Q$.

Since the implications in both directions have been proved, the proof
of the Theorem is complete.

\end{description}

\newpage

Again, we present the theorem in a more exhaustive (exhausting?) format.  This proof is actually rather different from the less formal proof given above; you could try formalizing the preceding proof as an exercise.

\begin{description}

\item[Theorem:]  $P \vee Q \leftrightarrow \neg P \rightarrow Q$

\item[Proof:]  The statement is a biconditional, and gets the usual proof plan for a biconditional.

\item[Part I:]

\begin{description}

\item[Assume (1):]  $P \vee Q$

\item[Goal:]  $\neg P \rightarrow Q$

\begin{description}

\item[Assume (2):]  $\neg P$

\item[Goal:] $Q$

\begin{description}

\item[Assume (3):]  Assume $\neg Q$ for the sake of a contradiction

\item [Goal:]  $\perp$

We start a proof by cases using line 1.

\begin{description}

\item[Case I:  assume (4a):]  $P$

\item[Goal:] $\perp$

\item[(5a):]  $\perp$ contradiction, 2,4a

\item[Case 2:  assume (4b):] $Q$

\item[Goal:]  $\perp$

\item[(5b):]  $\perp$ contradiction 3,4b

\end{description}

\item[(6):]  $\perp$ proof by cases, 1, 4a-5a, 4b-5b (proof by cases has the most complex line justifications of any of our rules!)

\end{description}

\item[(7):]  $Q$  reductio ad absurdum 3-6

\end{description}

\item[(8):]  $\neg P \rightarrow Q$ deduction 2-7

\end{description}

\newpage


\item[Part II:]

\begin{description}

\item[Assume (9):]  $\neg P \rightarrow Q$

\item[Goal:]  $P \vee Q$

We start by throwing up our hands in despair and trying reductio ad absurdum!

\begin{description}

\item[Assume (10):]  $\neg(P \vee Q)$ for the sake of a contradiction.

\item[Goal:]  $\perp$

We introduce a new goal $\neg P$ with the idea that if we can prove it, we can apply modus ponens with line 9 to prove $Q$.

\item[Goal:]  $\neg P$

\begin{description}

\item[Assume (11):]  $P$ 

\item[Goal:]  $\perp$

\item [(12):]   $P \vee Q$ addition, line 11

\item [(13):]  $\perp$  contradiction, 10, 12

\end{description}

\item[(14)]  $\neg P$ negation introduction 11-13

\item[(15):]  $Q$ m.p. 9,14

\item[(16):]  $P \vee Q$ addition 15

\item[(17):]  $\perp$ contradiction 10, 16

\end{description}

\item[(18):]  $P \vee Q$ reductio ad absurdum 10-17
\end{description}

\item[(19):]  $(P \vee Q) \leftrightarrow (\neg P \rightarrow Q)$ biconditional introduction 1-8, 9-18.

This is a hard proof because the only rules for disjunction we are allowed to use are disjunction and proof by cases.  Even excluded middle is proved using this theorem, and so cannot be assumed here.

\newpage

We prove the symmetric form $(A \vee B) \leftrightarrow (\neg B \rightarrow A)$.

\begin{description}

\item[Theorem:]  $(A \vee B) \leftrightarrow (\neg B \rightarrow A)$.

\item[Part I:]

\begin{description}


\item[Assume (1):]  $A \vee B$

\item[Goal:]  $\neg B \rightarrow A$

\begin{description}

\item[Assume(2):]  $\neg B$

\item[Goal:]  $B$

\item[(3):]  $(A \vee B) \leftrightarrow (\neg A \rightarrow B)$.  Previous theorem.

\item[(4):]  $\neg A \rightarrow B$  bimp (biconditional m.p.) 1,3

\item[(5):]  $(\neg A \rightarrow B) \leftrightarrow (\neg B \rightarrow \neg\neg A)$ contrapositives theorem, replacing $P$ with $\neg A$, $Q$ with $B$.

\item[(6):]  $\neg B \rightarrow \neg\neg A$  bimp 4,5

\item[(7):]  $\neg\neg A$ m.p. 2,6

\item[(8):] $A$ dne (double negation elimination) 7

\end{description}

\item[(9):]  $\neg B \rightarrow A$ deduction 2-8

\end{description}


\item[Part II:]

\begin{description}

\item[Assume (10):]  $\neg B \rightarrow A$

\item[Goal:]  $A \vee B$

\item[Goal:]  $\neg A \rightarrow B$ (so we can apply the previous theorem)

\begin{description}

\item[Assume(11):]  $\neg A$

\item[Goal:]  $B$

\item [(12):]  $\neg A \rightarrow \neg\neg B$ contrapositives theorem  10 (this time just using the theorem as a one step rule, ``deduce $\neg D \rightarrow \neg C$ from $C \rightarrow D$".)

\item [(13:)] $\neg\neg B$ m.p. 11, 12

\item[(14):]  $B$ dne 13

\end{description}

\item[(15):]  $\neg A \rightarrow B$ deduction 11-14

\item[(16):]  $A \vee B$ (the previous theorem, used as a one step rule, ``deduce $C \vee D$ from $\neg C \rightarrow D$")

\end{description}

\item[(17:)]  $(A \vee B) \leftrightarrow (\neg B \rightarrow A)$ biconditional introduction, 1-9, 10-16.

\end{description}

Note that in Part I of this proof we exhibit a very conservative way of appealing to a theorem already proved, and in Part II we illustrate a more liberal way of doing this.


\newpage

\end{description}

\newpage

The Theorem directly justifies the more general proof strategies for
disjunction involving negative hypotheses given above.

\begin{description}


\item[Proof Strategy:]

To deduce the goal $A \vee B$, assume $\neg A$ and deduce $B$: this is
valid because it is a proof of the equivalent implication $\neg A
\rightarrow B$.  Alternatively, assume $\neg B$ and deduce $A$: this
is a proof of the equivalent $\neg B \rightarrow A$.  These rules are called {\em disjunction introduction\/} (or {\em alternative elimination\/})

If we have posits $A \vee B$ and $\neg A$, we can draw the conclusion $B$, by
converting $A \vee B$ to the equivalent $\neg A \rightarrow B$ and
applying modus ponens.

Symmetrically, if we have posits $A \vee B$ and $\neg B$, we can deduce
$A$.

The latter two rules are called {\em disjunctive syllogism\/}.  We also view two other variants as instances of disjunctive syllogism:  ``given $A \vee \neg B$ and $B$, deduce $A$", and ``given $\neg A \vee B$ and $A$, deduce $B$".  Of course these follow from the other forms and applications of double negation introduction.

\end{description}

A classic theorem which we should  not neglect, often used as the basis for proofs by cases, is 

\begin{description}

\item[Theorem (excluded middle):]  $A \vee \neg A$

\item[Proof:]  To prove this by the alternative elimination strategy, assume $\neg \neg A$ and show that $A$ follows.  But it does follow, immediately,
by double negation elimination.

\end{description}

A perhaps shocking result is that anything at all follows from $\perp$, and so from any
contradiction.

\begin{description}

\item[Theorem:]  $\perp \rightarrow B$

\item[Proof:]  Assume $\perp$, and our goal becomes $B$.  We prove $B$ by contradiction,
that is, assume $\neg B$ and take $\perp$ as our new goal.  The new goal is already met by our initial assumption, so the proof is complete.\footnote{In a constructive logic, where double negation elimination is not allowed as a rule, deduction of any $B$ from $\perp$ would be a primitive rule, called {\em absurdity elimination\/}.}

\item[Theorem:]  $A \wedge \neg A \rightarrow B$

\item[Proof:] Assume $A \wedge \neg A$, and take $B$ as our new goal.
From $A \wedge \neg A$, we deduce $A$ and we deduce $\neg A$, and from these we deduce $\perp$.
$\perp \rightarrow B$ is true by the previous theorem, and $B$ follows by {\em modus ponens}.

\end{description}

The operation represented by $\neg$ is called {\em negation\/}.

\section{Generality and the Rule of Substitution}

A propositional letter $P$ reveals nothing about the structure of the
statement it denotes.  This means that any argument that shows that
certain hypotheses (possibly involving $P$) imply a certain conclusion
$A$ (possibly involving $P$) will remain valid if all occurrences of
the propositional letter $P$ in the entire context are replaced with
any fixed statement $B$ (which may be logically complex).

Denote the result of replacing $P$ with $B$ in $A$ by $A[B/P]$.
Extend this notation to sets $\Gamma$: $\Gamma[B/P] = \{A[B/P]\mid
A\in \Gamma\}$.

The rule of substitution for propositional logic can then be stated as

If we can deduce $A$ from a set of assumptions $\Gamma$, $P$ is a
propositional letter and $B$ is any proposition (possibly complex),
then we can deduce $A[B/P]$ from the assumptions $\Gamma[B/P]$.

Using the substitution notation, the strongest rules for the
biconditional can be stated as

``from $A\leftrightarrow B$ and $C[B/P]$, deduce $C[A/P]$.''

``from $A\leftrightarrow B$ and $C[A/P]$, deduce $C[B/P]$.''

\section{Note on Order of Operations}

The statements ``$A$ and either $B$ or $C$'' and ``Either $A$ and $B$,
or $C$'' (which can formally be written $A \wedge (B \vee C)$ and $(A
\wedge B) \vee C$) do not have the same meaning.  Making such grouping
distinctions in English is awkward; in our notation we have the
advantage of the mathematical device of parentheses.

To avoid having to write all parentheses in order to make the meaning
of a statement clear, we stipulate that just as multiplication is
carried out before addition when parentheses do not direct us to do
otherwise, we carry out $\neg$ first, then $\wedge$, then
$\vee$, then $\rightarrow$, then $\leftrightarrow$ or $\not\leftrightarrow$.  When a list of
operations at the same level are presented, we group to the right: $P
\rightarrow Q \rightarrow R$ means $P \rightarrow (Q \rightarrow R)$.
In fact, this only makes a difference for $\rightarrow$, as all the
other basic operations are associative (including $\leftrightarrow$ and
$\not\leftrightarrow$; check it out!).\footnote{The theorem proving software Marcel treats conjunction and disjunction as grouping to the left by default for reasons which will become clear to the observant when they actually carry out some proofs.  But this is a matter of convenience.}

There is a temptation to allow $A \leftrightarrow B \leftrightarrow C$ to mean $(A
\leftrightarrow B) \wedge (B \leftrightarrow C)$ by forbidding the omission of
parentheses in expressions $A \leftrightarrow (B \leftrightarrow C)$ and $(A \leftrightarrow B)
\leftrightarrow C$.  We resist this temptation.

\section{Reduction of Arguments in Propositional Logic}

We have given all the basic manipulations of arguments in propositional logic which Marcel supports
in remarks under each logical operation (and in the brief section on arguments).

We can at least informally argue that application of these rules to an argument formulated entirely in propositional letters and operations of propositional logic must terminate in a proof of validity or invalidity.

Take such an argument and expand all biconditionals $A \leftrightarrow B$ to the form $(A \rightarrow B) \wedge (B \rightarrow A)$ [and notice that the rules we give for biconditional are justified by this expanded form and the rules for conjunction and implication].

Observe that any application of a rule for the other operations reduces the number of logical operations appearing in each of the resulting arguments (there may be two arguments, but each of them will contain fewer operations), except that top-level negations of premises may be introduced.  We simplify matters by not counting top level negations of premises.  Note that
we can always move a negated premise into the conclusion and remove the top level operation of that premise by a rule application.

At any point, we have a collection of arguments to show valid.  Take the first one in our list with the largest number of logical operations other than top level negations of premises.  Apply a rule to it to reduce the number of such operations in one or two new arguments which will replace it in the list.  This may involve moving a negated premise into the conclusion then apply ing a rule.

This process must terminate in a number of arguments in which no operations appear except top level negations of premises:  that is, all formulas are either letters or negations of letters, and the conclusion is a letter.  If one of the premises is the same as the conclusion such an argument is valid.  If one of the premises is the negation of another premise, such an argument is valid.  If neither of these situations hold, it is possible to assign the value true to all premises and false to the conclusion, the argument is invalid, and the argument we started with was invalid.

This process is not efficient: in the worst case, each elimination of an operation doubles the number of arguments we have to consider, so we may in the worst case have exponential blowup.  You can explore this process by playing with Marcel.

We find it striking that Marcel uses exactly one rule to use posits of each logical form and one rule to deduce goals of each logical form:  thus Marcel has a single command to ``unpack" the goal, a single command to ``unpack" the first premise, and a command to bring a desired premise to the front for attention.  This supports our contention in teaching this material that attention to the logical form of posits and goals allows us to build large parts of proofs purely mechanically;  in this case, we show that a machine can do it.

\section{Quantifiers}

In this section, we go beyond propositional logic to what is variously
called first-order logic, predicate logic, or the logic of
quantifiers.  In any event, as in the propositional logic section, we
are not talking about a formal system, though we will introduce some
formal notations: we are talking about kinds of statement which appear
in informal mathematical argument in natural language, and introducing formal notations to represent such statements in aid of precision, and we hope in aid of clarity.

We denote an arbitrary complex statement, presumably involving the
variable $x$, by the notation $A[x]$.  We do not write $A(x)$ because
this is our notation for a unary predicate sentence in which $A$
stands for some definite unary predicate: a sentence of the form
$A(x)$ has the exact form of a predicate being asserted of $x$ while a
sentence of the form $A[x]$ could be any sentence that presumably
mentions $x$ (so $x=x$ is of the form $A[x]$ but not of the form
$A(x)$; a sentence like ${\tt Nat}(x)$ (meaning ``$x$ is a natural
number'') would be an example of the first form.  A related notation
is $A[t/x]$, the result of replacing the variable $x$ in the
proposition $A$ with the term $t$ (which may be a complex name rather
than simply a variable).  If we denote a formula $\cal A$ by the
notation $A[x]$ then for any term $t$ we use the notation $A[t]$ to
represent ${\cal A}[t/x]$.\footnote{It should be noted that this is a
subtle distinction I am drawing which is not universally made (the
exact notation here is specific to these notes); it is quite common to
write $P(x)$ for what I denote here as $P[x]$, and I have been known to write parentheses by mistake when teaching from this text.}

The two kinds of statement we consider can be written ``for all $x$,
$A[x]$'' (formulas with a {\em universal quantifier\/}) and ``for
some $x$, $A[x]$'', which is also often written ``there exists $x$
such that $A[x]$'' (which is why such formulas are said to have an
{\em existential quantifier\/}).  This language, although it is
acceptable mathematical English, is already semi-formalized.

Formulas (or sentences) with universal and existential quantifiers can
appear in a variety of forms.  The statement ``All men are mortal''
can be analyzed as ``for all $x$, if $x$ is a man then $x$ is
mortal'', and the statement ``Some men are immortal'' can be analyzed
as ``for some $x$, $x$ is a man and $x$ is immortal''.

The formal notation for ``for all $x$, $A[x]$" is $(\forall x.A[x])$
and for ``for some $x$, $A[x]$'' is $(\exists x.A[x])$.  The
parentheses in these notations are for us mandatory: this may seem
eccentric but we think there are good reasons for it.\footnote{Nowadays (2021) I tend to write $(\forall x:P[x])$ and $(\exists x:P[x])$, and this form may be seen in more recent edits of the text.}

Iteration of the same quantifier can be abbreviated.  We write
$(\forall xy.A[x,y])$ instead of $(\forall x.(\forall y.A[x,y]))$, and
similarly $(\exists xy.A[x,y])$ instead of $(\exists x.(\exists
y.A[x,y]))$, and notations like $(\forall xyz.A[x,y,z])$ are defined
similarly.

Quantifiers are sometimes (very often, in practice), {\em
restricted\/} to some domain.  Quantifiers restricted to a set have
special notation: $(\forall x\in S.A[x])$ can be read ``for all $x$ in
$S$, $A[x]$'' and is equivalent to $(\forall x.x\in S \rightarrow
A[x])$, while $(\exists x\in S.A[x])$ can be read ``for some $x$ in
$S$, $A[x]$'' and is equivalent to $(\exists x.x\in S \wedge A[x])$.
The same quantifier restricted to the same set can be iterated, as in
$(\forall xy\in S.A[x,y])$, meaning $(\forall x \in S.(\forall y \in S.A[x,y]))$.  We leave the expansion of this with implications to the imagination.
Restriction can also use other binary predicates:  $(\forall x \, R \, y. A[x]\}$ abbreviates $(\forall x.x \, R \, y \rightarrow A[x])$, for example.

Further, restriction of a quantifier to a particular sort of object is
not always explicitly indicated in the notation.  If we know from the
context that a variable $n$ ranges over natural numbers, we can write
$(\forall n.A[n])$ instead of $(\forall n \in {\cal N}.A[n])$, for
example.  In the chapter on typed theory of sets, all variables will
be equipped with an implicit type in this way.

We do not as a rule negate quantified sentences (or formulas) in
natural language.  Instead of saying ``It is not the case that for all
$x$, $A[x]$'', we would say ``For some $x$, $\neg A[x]$''.  Instead of
saying ``It is not the case that for some $x$, $A[x]$'', we could say
''For all $x$, $\neg A[x]$'' (though English provides us with the
construction ``For no $x$, $A[x]$'' for this case).  ``No men are
mortal'' means ``For all $x$, if $x$ is a man then $x$ is not
mortal''.  The logical transformations which can be carried out on
negated quantified sentences are analogous to de Morgan's laws, and
can be written formally $$\neg(\forall x.A[x]) \leftrightarrow (\exists x.\neg
A[x])$$ and $$\neg(\exists x.A[x]) \leftrightarrow (\forall x.\neg A[x]).$$ Note
that we are not licensing use of these equivalences as proof strategies
before they are proved: as above with de Morgan's laws, these are
introduced here to make a point about the rhetoric of mathematical
English.

Here is a good place to say something formally about the distinction
between the more general ``formula'' and the technical sense of
``sentence'' (I would really much rather say ``sentence'' for both,
following the grammatical rather than the mathematical path).  Any
``sentence'' in the grammatical sense of mathematical language is
called a formula; the actual ``sentences'' in the mathematical sense
are those in which a variable $x$ only occurs in a context such as
$(\forall x.A[x])$, $(\exists x.A[x])$ or even $\{x \mid A[x]\}$ or
$\int_2^3 \,x^2\,dx$ (to get even more familiar) in which it is a
dummy variable.  The technical way of saying this is that a sentence
is a formula in which all occurrences of variables are {\em
bound\/}.



\subsection{Variation:  Hilbert symbols and definite descriptions}

A variant approach to quantification is supported by the use of the {\em Hilbert symbol\/} $(\epsilon x:A[x])$, which may be read ``An $x$ such that $A[x]$ (if there is one)."
This symbol stands for an arbitrarily selected $x$ such that $A[x]$, if there is one, and otherwise for a default object.  The thing to notice here is
that on the intended semantics for the Hilbert symbol, $(\exists x:A[x])$ is equivalent to $A[(\epsilon x:A[x])]$.  Thus the existential quantifier can be defined in terms of the Hilbert symbol, and indeed $(\forall x:A[x])$ can also be expressed as $A[(\epsilon x:\neg A[x])]$.  An attempt to expand any expression with nested quantifiers should reveal why these are not really good practical definitions, but this notion has extensive theoretical uses.

A {\em definite description\/} $(\theta x:A[x])$ (read ``The $x$ such that $A[x]$)" stands for the unique $x$ such that $(\theta x:A[x])$ (if there is such an $x$):  $(\theta x:A[x])$ could be defined in terms of the Hilbert symbol as $(\epsilon x:(\forall y:A[y] \leftrightarrow y=x))$.

We postulate additionally  (if we are using these symbols:  they are not an official part of our logic except when we explicitly say so)  that $$(\forall x:A[x] \leftrightarrow B[x]) \rightarrow (\epsilon x:A[x]) = (\epsilon x:B[x]))$$ and similarly  $$(\forall x:A[x] \leftrightarrow B[x]) \rightarrow (\theta x:A[x]) = (\theta x:B[x])):$$  the object such that $A[x]$ and the object such that $B[x]$ which are arbitarily chosen are the same object if the two sentences are true of the same things.\footnote{This serves to preserve the validity of the rule that logically equivalent formulas can replace one another freely.}  This has the side effect that for any $A[x]$ which is not true for any $x$, \newline $(\epsilon x:A[x])$ is the same default object (and we stipulate that this is the same as $(\theta x:B[x])$ for all $B[x]$ which hold for no $x$ or for more than one $x$).  We will temporarily refer to this default object as $\delta$:  a good choice for $\delta$ is the empty set $\emptyset$.

\section{Proving Quantified Statements and Using Quantified Hypotheses}

To prove the goal ``for all $x$, $A[x]$", introduce a new name $a$ (not
used before anywhere in the context): the new goal is to prove $A[a]$.
Informally, if we can prove $A[a]$ without making any special
assumptions about $a$, we have shown that $A[x]$ is true no matter
what value the variable $x$ takes on.  The new name $a$ is used only
in this proof (its role is rather like that of an assumption in the
proof of an implication).   This rule is called ``universal generalization".

It is permissible and sometimes useful to open the indented block in a proof of ``for all $x$, $A[x]$"
using the new goal $A[a]$ with a numbered line $a=a$:  it is handy to have a first line to reference the start of the block in a uniform way.

To prove the goal ``for some $x$, $A[x]$'', find a specific name $t$
(which may be complex) and prove $A[t]$.  Notice here there may be all
kinds of contextual knowledge about $t$ and in fact that is expected.
It's possible that several different such substitutions may be made in
the course of a proof (in different cases a different witness may work
to prove the existential statement).   This rule is called ``existential instantiation".

If you have posited ``for all $x$, $A[x]$'', then you may further
posit $A[t]$ for any name $t$, possibly complex.  You may want to make
several such substitutions in the course of a proof.   This rule is called "universal instantiation".

\newpage

Using an existential statement is a bit trickier.  If we have posited
``for some $x$, $A[x]$'', and we are aiming at a goal $G$, we may
introduce a name $w$ not mentioned anywhere in the context (and in
particular not in $G$) and further posit $A[w]$: if $G$ follows with
the additional posit, it follows without it as well.  What we are
doing here is introducing a name for a witness to the existential
hypothesis.  Notice that this name is locally defined; it is not
needed after the conclusion $G$ is proved.   This rule is called ``existential generalization" or ``witness introduction".

We present an example.

\begin{description}

\item[Theorem:]  $(\exists x:A[x]) \wedge (\forall x:A[x] \rightarrow B[x]) \rightarrow (\exists x:B[x])$

\item[Proof:]

\begin{description}

\item

\begin{description}

\item

\item[Assume (1):]  $(\exists x:A[x]) \wedge (\forall x:A[x] \rightarrow B[x])$

\item[Goal:]  $(\exists:B[x])$

\item[(2):]  $(\exists x:A[x])$  simplification 1

\item[(3):]  $(\forall x:A[x] \rightarrow B[x])$  simplification 1

\begin{description}

\item[(4):]  $A[w]$ introduce a witness to line 2

\item[Goal:]  $(\exists x: B[x])$  as is usual with EG, this is the goal you already have

\item[(5):]  $A[w] \rightarrow B[w]$  universal instantiation, $x := w$, line 3

\item[(6):]  $B[w]$ mp 4,5

\item[(7):]  $(\exists x:B[x])$  existential instantiation $x := w$ line 6

\end{description}

\item[(8):]  $(\exists x:B[x])$  existential generalization 4-7

\end{description}

\item[(9):]  $(\exists x:A[x]) \wedge (\forall x:A[x] \rightarrow B[x]) \rightarrow (\exists x:B[x])$ deduction 1-8

\end{description}

\end{description}

An argument of the form $P_1,\ldots,P_n \vdash (\forall x:A[x])$ is valid iff $P_1,\ldots,P_n \vdash A[a]$ is valid,
where $a$ is a variable which does not appear in the original argument.  This captures the rule of universal generalization neatly.

An argument of the form $(\exists x:A[x]),P_2,\ldots,P_n \vdash C$ is valid iff $$A[w],P_2,\ldots,P_n \vdash C$$ is valid, where $w$ (for witness) is a variable which does not appear in the original argument.   This captures the rule of existential generalization.

The argument manipulation rules for using a universal premise or an existential conclusion involve a little more machinery, and we will present them the way the Marcel user will see them (to some extent).

We introduce bold face variables such as ${\bf a}$, just for discussion of Marcel's handling of these rules.
The privilege that such a variable (called an instantiable) has is that it may be replaced at any time {\em globally, in all arguments we are working on\/} with any desired term not involving the replaced instantiable or any variable or instantiable introduced later than the replaced instantiable (Marcel manages this using a system of indexing of variables and instantiables).  The formal rule for validity of an argument with instantiables in it is that the argument is valid in the usual sense for {\em some\/} assignment of values to the instantiables.

An argument of the form $(\forall x:A[x]),P_2,\ldots,P_n\vdash C$ is valid iff $$A[{\bf a}],(\forall x:A[x]),P_2,\ldots,P_n \vdash C$$ is valid, where ${\bf a}$ is an instantiable not appearing in the original argument.  This captures the rule of universal instantiation:  the role of the instantiable is to allow us to defer the choice of example.

An argument of the form $P_1,\ldots,P_n \vdash (\exists x:A[x])$ is valid iff $$P_1,\ldots,P_n,\neg(\exists x:A[x]) \vdash A[{\bf w}]$$ is valid, where {\bf w} is an instantiable not appearing the original argument.  This captures the rule of existential instantiation:  the role of the instantiable is to allow us to defer the choice of witness.

Notice that the quantified sentences are not eliminated in the last two rules.  More than one example of a universal statement or witness to an existential statement may need to be introduced in an argument, and we want to preserve exact equivalence of validity of the derived argument with the original argument.

We could eliminate discussion of instantiables in favor of introducing specific terms (not variables), but there is value in allowing the user to defer the selection of examples or witnesses.

\subsection{Reasoning with the Hilbert symbol}

One of the merits of the Hilbert symbol is that it might help to demystify the patterns of reasoning above, if one did happen to be mystified.

The sole relevant reasoning rule for the Hilbert symbol is ``From $A[t]$, deduce $A[(\epsilon x:A[x])$", where $t$ is any expression and $x$ is a variable not appearing in $A[t]$.

Suppose that we can prove $A[a]$ where $a$ is a brand-new constant used nowhere before.  Then we can define $a$ retrospectively
as $(\epsilon x:\neg A[x])$ and we find that $A[a]$ literally is $(\forall x:A[x])$ if we use the definition of the universal quantifier given above.

If we assume $(\forall x:A[x])$, then we have assumed $A[(\epsilon x:\neg A[x])]$.  We argue by contradiction that we can conclude $A[t]$:  if we on the contrary suppose $\neg A[t]$, we can deduce $\neg A[(\epsilon x:\neg A[x])]$ using the basic rule for the Hilbert symbol, which yields a contradiction.

Suppose that we can prove $A[t]$:  then by the basic rule for reasoning with the Hilbert symbol we have $A[(\epsilon x:A[x])]$, thus $(\exists x:A[x])$.

Suppose that we assume $(\exists x:A[x])$.  If we introduce a new symbol $w$ and from $A[w]$ deduce a conclusion $G$ not mentioning $w$, we
can retrospectively define $w$ as $(\epsilon x:A[x])$ and we suddenly realize that we have deduced $G$ from $A[w]$ which simply is the hypothesis
$(\exists x:A[x])$.



\section{Equality and Uniqueness}

For any term $t$, $t=t$ is an axiom which we may freely assert.

If we have posited $a=b$ and $A[a/x]$, we can further posit $A[b/x]$.

We recall from above that if we include the Hilbert symbol in our logic, we add the rule ``if we have posited $(\forall x:A[x] \leftrightarrow B[x])$, we may further posit $(\epsilon x:A[x]) = (\epsilon x:B[x])$."

These are an adequate set of logical rules for equality.

To show that there is exactly one object $x$ such that $A[x]$ (this is
often written $(\exists! x.A[x])$), one needs to show two things:
first, show $(\exists x.A[x])$ (there is at least one $x$).  Then show
that from the additional assumptions $A[a]$ and $A[b]$, where $a$ and
$b$ are new variables not found elsewhere in the context, that we can
prove $a=b$ (there is at most one $x$).\footnote{Note that under the hypothesis $(\exists! x.A[x])$, the definite description notation $(\theta x:A[x])$ has the intended meaning.  A logically rather opaque way of expressing the same condition is ``$(\theta x:A[x])$ exists", though on our convention the definite description denotes a definite object $\delta$ even if it does not have the intended meaning.}

Proofs of uniqueness are often given in the form ``Assume that $A[a]$,
$A[b]$, and $a \neq b$: deduce a contradiction''.  This is equivalent
to the proof strategy just given but the assumption $a\neq b$ is often
in practice never used (one simply proves $a =b$) and so seems to be
an unnecessary complication.

The rules of symmetry and transitivity of equality are consequences of the rules given above.

We demonstrate the validity of the rules

$$\begin{array}{c} a=b \\ \hline b=a \end{array}$$

and

$$\begin{array}{c} a=b \\ b=c \\ \hline a=c\end{array}$$

We demonstrate the validity of the first rule.

\begin{description}

\item[(1):]  $a=b$ premise

\item[(2):]  $a=a$  reflexivity of equality

\item[(3):]  $b=a$ substitution into 2 using $x=a$ as the formula $A$:  $A[a/x]$ is $a=a$ and $A[b/x]$ is $b=a$.  Notice that this way of describing substitution allows us to deal with situations where we do not want to replace all $a$'s with $b$'s.

\end{description}

We demonstrate the validity of the second rule.

\begin{description}

\item[(1):]  $a=b$  premise

\item[(2):]  $b=c$ premise

\item[(3):] $b=a$  symmetry of equality (the rule just proved)

\item[(4):]  $[a]=c$  substitution into 2 using 3 (using the bracket to highlight where the substitution happened).

\end{description}

\section{Dummy Variables and Substitution}

The rules of the previous section make essential use of substitution.
If we write the formula $A[x]$ of the previous section in the form
$\cal A$, recall that the variants $A[a]$ and $A[t]$ mean ${\cal
A}[a/x]$ and ${\cal A}[t/x]$: understanding these notations requires
an understanding of substitution.

And there is something nontrivial to understand.  Consider the
sentence $(\exists x.x=a)$ (this is a sentence if $a$ is a constant
{\em name\/} rather than a variable).  This is true for any $a$, so we
might want to assert the not very profound theorem $(\forall
y.(\exists x.x=y))$.  Because this is a universal statement, we can
drop the universal quantifier and replace $y$ with anything to get
another true statement: with $c$ to get $(\exists x.x=c))$; with $z$
to get $(\exists x.x=z))$.  But if we naively replace $y$ with $x$ we
get $(\exists x.x=x)$, which does not say what we want to say: we want
to say that there is something which is equal to $x$, and instead we
have said that there is something which is equal to {\em itself\/}.

The problem is that the $x$ in $(\exists x.x=y)$ does not refer to any
particular object (even if the variable $x$ does refer to something in
a larger context).  $x$ in this sentence is a ``dummy variable''.
Since it is a dummy it can itself be replaced with any other variable:
$(\exists w.w=y)$ means the same thing as $(\exists x.x=y)$, and
replacing $y$ with $x$ in the former formula gives $(\exists w.w=x)$
which has the intended meaning.

Renaming dummy variables as needed to avoid collisions avoids these
problems.  We give a recursive definition of substitution which
supports this idea.  $T[t/x]$ is defined for $T$ any term or formula,
$t$ any term, and $x$ any variable.  The only kind of term (noun
phrase) that we have so far is variables: $y[t/x]$ is $y$ if $y \neq
x$ and $t$ otherwise; $P(u)[t/x]$ is $P(u[t/x])$; $(u\,R\,v)[t/x]$ is
$u[t/x]\,R\,v[t/x]$.  So far we have defined substitution in such a
way that it is simply replacement of the variable $x$ by the term $t$.
Where $A$ is a formula which might contain $x$, $(\forall y.A)[t/x]$
is defined as $(\forall z.A[z/y][t/x])$, where $z$ is the
typographically first variable not occurring in $(\forall y.A)$, $t$
or $x$.  $(\exists y.A)[t/x]$ is defined as $(\exists z.A[z/y][t/x])$,
where $z$ is the typographically first variable not occurring in
$(\exists y.A)$, $t$, or $x$.  This applies to all constructions with
bound variables, including term constructions: for example, once we
introduce set notation, $\{y \mid A\}[t/x]$ will be defined as $\{z
\mid A[z/y][t/x]\})$, where $z$ is the typographically first variable
not occurring in $\{y \mid A\}$, $t$, or $x$.  The use of
``typographically first'' here is purely for precision: in fact our
convention is that (for example) $(\forall x.A)$ is basically the same
statement as $(\forall y.A[y/x])$ for any variable $y$ not occurring
in $A$ (where our careful definition of substitution is used) so it
does not matter which variable is used as long as the variable is new
in the context.

It is worth noting that the same precautions need to be taken in
carefully defining the notion of substitution for a propositional
letter involved in the rule of substitution.

We also note that the actual substitutions involved in expanding out any nontrivial reasoning with Hilbert symbols would create very large expressions indeed!\footnote{Note to self:  add some exercises expanding on this point}

\section{Are we doing formal logic yet?}

One might think we are already doing formal logic.  But from the
strictest standpoint we are not.  We have introduced formal notations
extending our working mathematical language, but we are not yet
considering terms, formulas and proofs {\em themselves\/} as
mathematical objects and subjecting them to analysis (perhaps we are
threatening to do so in the immediately preceding subsection).  We
will develop the tools we need to define terms and formulas as formal
mathematical objects (actually, the tools we need to formally develop
any mathematical object whatever) in the next section, and return to
true formalization of logic (as opposed to development of formal
notation) in the Logic chapter.

We have not given many examples: our feeling is that this material is so
abstract that the best way to approach it is to use it when one has
some content to reason about, which will happen in the next chapter.
Reference back to our discussion of proof strategy here from actual
proofs ahead of us is encouraged.\footnote{and lab work with Marcel will also give exanples of abstract reasoning with quantifiers.}


\newpage

\section{Exercises}

Prove the following statements using the proof strategies above.  Use
only the highlighted proof strategies (not, for example, de Morgan's
laws or the rules for negating quantifiers, or the use of biconditional theorems to make substitutions).  You may use proof of an
implication by proving the contrapositive, modus tollens and the
generalized rules for proving disjunctions.

\begin{enumerate}

\item Prove the equivalence $$A\rightarrow (B \rightarrow C) \leftrightarrow (A \wedge B) \rightarrow C$$



\item  
Prove $$((P\rightarrow Q) \wedge (Q \rightarrow R)) \rightarrow (P \rightarrow R)$$

This should be very straightforward.

\item Prove $$(A \wedge B) \leftrightarrow (B \wedge A)$$

This should be very straightforward, and very annoying.

\item Prove the equivalence $$\neg(A \rightarrow B) \leftrightarrow (A \wedge \neg B)$$

This will be harder:  remember, when you can't see anything else to do,  use reductio.

\item


 Prove each of the following:
\begin{enumerate}
\item  $$\neg (\forall x.P[x]) \leftrightarrow (\exists x.\neg P[x])$$
\item  $$\neg (\exists x.P[x]) \leftrightarrow (\forall x.\neg P[x])$$
\end{enumerate}

\item  
Prove $$((\exists x.P[x]) \wedge (\forall uv.P[u] \rightarrow Q[v])) \rightarrow (\forall z.Q[z])$$

\item Prove de Morgan's laws (both of them).

\item Verify that $$((A \vee B) \rightarrow C) \leftrightarrow ((A \rightarrow C) \wedge (B \rightarrow C))$$ is a theorem.

\item Verify the rule of {\em destructive dilemma}:

$$\begin{array}{c}

P \rightarrow Q \\

R \rightarrow S \\

\neg Q \vee \neg S \\ \hline

\neg P \vee \neg R

\end{array}$$

An example of verification of a related rule appears as an example in the manual of logical style at the end of the text.

\item  Justify the rules for the existential quantifier using the rules for the universal quantifier and the equivalence of $(\exists x.P[x])$ and $\neg(\forall x.\neg P[x])$.

\item Construct truth tables for $A \leftrightarrow (B \leftrightarrow C)$, $(A \leftrightarrow B) \leftrightarrow C$, and $$(A \leftrightarrow B) \wedge (B \leftrightarrow C).$$  Notice that the first two are the same and the third (which one might offhand think is what the first says) is quite different.  Can you determine a succinct way of explaining what $$A_1 \leftrightarrow A_2 \leftrightarrow \ldots \leftrightarrow A_n$$ says?

\item  (added 2/4/2021) Give a paper proof of $(\exists x:(\forall y:P[y] \rightarrow P[x]))$.  I'll try to provide notes comparing this proof with the Marcel proof we did in class.  Then prove
$(\exists x:(\forall y:P[x] \rightarrow P[y]))$.  I might have you do this one in a Marcel lab.

This one might be really difficult.

\newpage

\item  (added 2/9/2021) Verify the validity of the argument

$$\begin{array}{c}

(\forall x:\neg G[x] \vee \neg H[x]) \\

(\forall x:(J[x] \rightarrow F[x]) \rightarrow H[x]) \\  \hline

\neg (\exists x:F[x] \wedge G[x])

\end{array}$$

Hint:  You will want to assume $(\exists x:F[x] \wedge G[x])$ and reason to a contradiction.

\end{enumerate}

\newpage

We give some solutions.

\begin{description}

\item[1:] Our goal is $$A\rightarrow (B \rightarrow C) \leftrightarrow (A
\wedge B) \rightarrow C.$$  

\begin{description} 

\item[Goal 1:]  $$A\rightarrow (B \rightarrow C) \rightarrow (A
\wedge B) \rightarrow C$$.

\item[Argument for Goal 1:]

Assume $A\rightarrow (B \rightarrow C)$.  Our new goal is $(A \wedge
B) \rightarrow C$.  To prove this implication we further assume
$A\wedge B$, and our new goal is $C$.  Since we have posited $A \wedge
B$, we may deduce both $A$ and $B$ separately.  Since we have posited
$A$ and $A\rightarrow (B \rightarrow C)$ we may deduce $B \rightarrow
C$ by modus ponens.  Since we have posited $B$ and $B \rightarrow C$,
we may deduce $C$, which is our goal, completing the proof of Goal 1.

\item[Goal 2:]  $$(A\wedge B) \rightarrow C \rightarrow A\rightarrow (B \rightarrow C)$$

\item[Argument for Goal 2:] Assume $(A\wedge B) \rightarrow C$.  Our
new goal is $A\rightarrow (B \rightarrow C)$.  To deduce this
implication, we assume $A$ and our new goal is $B\rightarrow C$.  To
deduce this implication, we assume $B$ and our new goal is $C$.  Since
we have posited both $A$ and $B$ we may deduce $A \wedge B$.  Since we
have posited $A \wedge B$ and $(A\wedge B) \rightarrow C$, we may
deduce $C$ by modus ponens, which is our goal, completing the proof of
Goal 2.

\item[Conclusion:] Since the implications in both directions have been
proved, the biconditional main goal has been proved.

\end{description}
\newpage
\item[5a:] Our goal is $$\neg (\forall x.P[x]) \leftrightarrow (\exists x.\neg
P[x]).$$ Since this is a biconditional, the proof involves proving two
subgoals.

\begin{description}

\item[Goal 1:]  $$\neg (\forall x.P[x]) \rightarrow (\exists x.\neg
P[x])$$

\item[Argument for Goal 1:] Assume $\neg (\forall x.P[x])$.  Our new
goal is $(\exists x.\neg P[x])$.  We would like to prove this by
exhibiting a witness, but we have no information about any specific
objects, so our only hope is to start a proof by contradiction.  We
assume $\neg(\exists x.\neg P[x])$ and our new goal is $\perp$.  We note that deducing $(\forall x.P[x])$ as a goal
would allow us to deduce $\perp$ (this is one of the main ways to use
a negative hypothesis).  To prove this goal, introduce an arbitrary
object $a$ and our new goal is $P[a]$.  Since there is no other
evident way to proceed, we start a new proof by contradiction: assume
$\neg P[a]$ and our new goal is $\perp$.  Since we have
posited $\neg P[a]$, we may deduce $(\exists x.\neg P[x])$.  This
allows us to deduce $\perp$, since we have already posited the negation of
this statement.  This supplies what is needed for each goal in turn
back to Goal 1, which is thus proved.

\item[Goal 2:] $$(\exists x.\neg P[x])\rightarrow \neg (\forall
x.P[x])$$


\item[Argument for Goal 2:]


We assume $(\exists x.\neg P[x])$.  Our new goal is $\neg(\forall
x.P[x])$.  To deduce this goal, we assume $(\forall x.P[x])$ and our
new goal is $\perp$.  Our existential hypothesis $(\exists
x.\neg P[x])$ allows us to introduce a new object $a$ such that $\neg
P[a]$ holds.  But our universal hypothesis $(\forall x.P[x])$ allows
us to deduce $P[a]$ as well, so we can deduce $\perp$,
completing the proof of Goal 2.

\item[Conclusion:] Since both implications involved in the
biconditional main goal have been proved, we have proved the main
goal.

\end{description}

\end{description}

\newpage

We present this again in the more explicit format with line numbers and justifications.

\begin{description}

\item[Prove:]  $$\neg (\forall x.P[x]) \leftrightarrow (\exists x.\neg P[x]).$$

\item[Part I:]

\begin{description}



\item

 \item[Assume (1):]  $\neg (\forall x.P[x])$

\item[Goal:]  $(\exists x.\neg P[x])$

\begin{description}

\item[Assume (2):]  $\neg(\exists x.\neg P[x])$ for the sake of a contradiction

\item[Goal:]  $\perp$

\item[Goal:]  $(\forall x.P[x])$ (which will give a contradiction with 1)

\begin{description}

\item Let $a$ be arbitrary.

\item[Goal:]  $P(a)$

\begin{description}

\item[Assume (3):]  $\neg P[a]$ for the sake of a contradiction

\item[Goal:]  $\perp$

\item[(4):]  $(\exists x.\neg P[x])$  EI, 3

\item[(5):]  $\perp$ contradiction 2,4

\end{description}

\item[(6)] $P(a)$  RAA 3-5

\end{description}

\item[(7):]  $(\forall x.P[x])$  UG 3-6

\item[(8):]  $\perp$  contradiction 1,7

\end{description}

\item[(9):]  $(\exists x.\neg P[x])$  RAA 2-8

\end{description}

\item[Part II:]


\begin{description}

\item

\item[Assume (1b):]  $(\exists x.\neg P[x])$

\item[Goal:]  $\neg (\forall x.P[x])$

\begin{description}

\item[Assume (2b)]: $(\forall x.P[x])$, for the sake of a contradiction.
\item[Goal:]  $\perp$

\begin{description}

\item[(3b):]  $\neg P(w)$  introduce witness to hypothesis 1b

\item[(4b):]  $P(w)$  UI line 2b $x:=w$

\item[(5b):] $\perp$  contradiction, 3b,4b

\end{description}

\item[(6b):]  $\perp$  EG 3b-5b

\end{description}

\item[(7b):]  $\neg(\forall x.P[x])$ neg intro 2b-6b

\end{description}

\item[the result to be proved:]  follows by biconditional introduction, 1-9, 1b-7b.

\newpage

\item[12a:]  This proof presents a significant technical challenge which we illustrate.  Thw difficulty is that there isn't any specific object we have
introduced in the environment to use as a candidate to prove the statement by EI.  Watch and marvel at the solution.

\begin{description}

 \item[Goal:]  $(\exists x:(\forall y:P[y] \rightarrow P[x]))$

\item[Goal:] $(\forall z:(\exists x:(\forall y:P[y] \rightarrow P[x])))$  You will see why we introduce the extra quantifier, and how we get rid of it.

\begin{description}

\item[1:]  $a=a$ (let $a$ be arbitary, but provide a line number).

\item[Goal:]  $(\exists x:(\forall y:P[y] \rightarrow P[x]))$

\item[2:]  $(\exists x:P[x]) \vee \neg (\exists x:P[x])$  excluded middle

we proceed with a proof by cases on line 2

\begin{description}

\item[Case 1 (3a):]  $(\exists x:P[x])$

\begin{description}

\item[(4a):]  $P[w]$  introduce a witness to line 3a

\item[Goal:]  $(\forall y:P[y] \rightarrow P[w])$

\begin{description}

\item[5a:]  $b=b$ let $b$ be arbitary

\item[Goal:]  $P[b] \rightarrow P[w]$

(it wont let me indent this block)

\item[Assume (6a):]   $P[b]$

\item[Goal:]  $P[w]$

\item[7a:]  $P[w]$ copied from line 4a

(end block)

\item[8a:]  $P[b] \rightarrow P[w]$ deduction 6a-7a

\end{description}

\item[9a:]  $(\forall y:P[y] \rightarrow P[w])$  UG 5a-8a

\item[10a:]  $(\exists x:(\forall y:P[y] \rightarrow P[x]))$  EI 9a x:=w

\end{description}

\item[11a:]  $(\exists x:(\forall y:P[y] \rightarrow P[x]))$  EG 4a-10a

\newpage

\item[Case 2: (3b):]$\neg (\exists x:P[x])$

\item[Goal:]  $(\forall y:P[y] \rightarrow P[a])$

\begin{description}


\item[4b] $b=b$ let $b$ be arbitrary

\item[Goal:]  $P[b] \rightarrow P[a]$

\begin{description}

\item[Assume(5b):]  $P[b]$

\item[6b:]  $(\exists x:P[x])$  EI 5b x:=b

\item[7b:]  $\perp$  contradiction 3b,6b

\item[8b:]  $\perp \rightarrow P[a]$  theorem proved in the text, false implies anything

\item[9b:]  $P[a]$ mp 7b,8b

\end{description}

\item [10b]  $P[b] \rightarrow P[a]$  deduction 5b-9b

\end{description}

\item[11b]  $(\forall y:P[y] \rightarrow P[a])$  UG  4b-10b

\item[12b:]  $(\exists x:(\forall y:P[y] \rightarrow P[x]))$  EI 11b x:=a



\end{description}

\item[13:]  $(\exists x:(\forall y:P[y] \rightarrow P[x]))$  proof by cases, 2, 3a-11a, 3b-12b  (I should have had 2a, 2b, but life is too short to change all the numbers)



\end{description}

\item[14:]  $(\forall z:(\exists x:(\forall y:P[y] \rightarrow P[x])))$ UG 1-13

\item[15:]  $(\exists x:(\forall y:P[y] \rightarrow P[x]))$  UI 14 (no need to say what $z$ is assigned since no occurrence of $z$ needs to be adjusted)

\end{description}

\end{description}

Isn't that weird?  I could add a rule which allows introduction of an arbitrary object, but...I dont need one!  This also provides extensive examples of the idea of adding
a trivial line mentioning the arbitrary object at the beginning of a UG block.





\newpage

\chapter{Typed theory of sets}

In this chapter we introduce a theory of sets, but not the usual one
quite yet.  We choose to introduce a typed theory of sets, which might
carelessly be attributed to Russell, though historically this is not
quite correct.

\section{Types in General}

Mathematical objects come in sorts or kinds (the usual word is
``type'').  We seldom make any statement about all mathematical
objects whatsoever: we are more likely to be talking about all natural
numbers, or all real numbers, or all elements of a certain vector space, etc.

Further, there are standard ways to produce a new sort of object from an old sort, which can be uniformly applied to all or at least many types:  for example,
if $\sigma$ is a sort and $\tau$ is a sort, we can talk about collections of $\sigma$'s or $\tau$'s,  functions from $\sigma$'s to $\tau$'s, ordered pairs of a $\sigma$ and a $\tau$, and so forth.  These are called {\em type constructors\/} when they are considered in general.

In much of this chapter, every variable
we introduce will have a type, and a quantifier over that variable
will be implicitly restricted to that type.

We begin, however, with a theory without types (which is distinctly different from the usual untyped set theory that we will introduce later).  The typed theory is introduced in a self-contained way
with a remark at the end that the untyped preamble provides a motivation;  the reader of this text or teacher from it may decide to make use of the untyped preamble or not.

\section{Untyped Preamble}

A longer development of this approach appears as an appendix.

Our basic predicates are equality and membership:  atomic sentences $x=y$ and $x \in y$ are at bottom what all our discourse is built out of.

We first introduce an essential identity criterion.  

\begin{description}

\item[Definition:]  An object $x$ is a {\em nonempty set\/} iff $(\exists y:y \in x)$.

\item[Axiom of extensionality:]  $$(\forall xyz: z \in x \rightarrow x=y \leftrightarrow (\forall w:w \in x \leftrightarrow w \in y)).$$  Nonempty sets are equal exactly if they have the same elements.

\end{description}

We define  the notion of ``being of the same kind (or type)".  The definition is motivated by the idea that any set
has elements all of the same kind, and every kind is a set.  Thus if two objects belong to the same set, they should be of the same kind, and further, if two objects are of the same kind there is a set (the common kind itself) to which they both belong.  This of course is informal.

\begin{description}

\item[Definition (being of the same type):]  We say that objects $x,y$ are of the same type, and write $x \sim_\tau y$, iff
$(\exists z:x \in z \wedge y \in z)$.  We expect this to be an equivalence relation, and indieed this will follow from further axioms.

\item[Axiom of types:]  $(\forall x:\exists \tau:x \in \tau \wedge (\forall y:y \in \tau \leftrightarrow x \sim_\tau y))$.

\item[Definition (type of an object):]  For any object $x$, we define $\tau(x)$ as the unique $\tau$ such that $(\forall y:y \in \tau \leftrightarrow x \sim_\tau y)$:  there is one by the axiom of types and only one by the axiom of extensionality (since the axiom of types tells us it is nonempty).   The set $\tau(x)$ witnesses $x \sim_\tau x$, so the relation of being of the same kind is reflexive.

\item[Definition (type hierarchy):]  We define $\tau^{\bf 1}(x)$ as $\tau(x)$ and for each numeral $n$ define $\tau^{\bf n+1}(x)$ as $\tau(\tau^{\bf n}(x))$.  Please note
that this is an infinite sequence of independent definitions and we do not quantify over such superscripts $\bf n$ (we put such superscripts in boldface to remind us that they are constants, never variables).  We haven't yet defined the natural numbers as objects we can quantify over, nor have we analyzed the iterated application of a function as a mathematical operation.  We will!  For the moment, we view the use of superscripts as simply a device for abbreviation.

\end{description}

We have only considered nonempty sets so far.  We introduce empty sets.

\begin{description}

\item[Empty set construction and axiom of empty sets:]  For each $x$ we postulate an object $\emptyset_{\tau(x)}$(the empty set of the same kind as $\tau(x)$)  satisfying the axiom $\emptyset_{\tau(x)} \in \tau(x) \wedge (\forall y:y \not\in \emptyset_{\tau(x)}$.

\item[Definition:]  We say ``$x$ is a set" and write this ${\tt set}(x)$ just in case $$(\exists y:y \in x \vee x = \emptyset_{\tau(y)}).$$  Note that
two sets are equal iff they are of the same kind and they have the same elements, by extensionality and the axiom of empty sets.   Having distinct empty sets of distinct kinds is an unusual feature of this theory.  We refer to non-sets (if there are any) as {\em atoms\/}.  We refer to objects which are not of the same kind as any set as {\em individuals\/} (if there are any).  Clearly an individual is an atom.

\item[Definition:]  We say that $x$ is a subset of $y$ and write $x \subseteq y$ iff $${\bf set}(x) \wedge {\bf set}(y) \wedge x \sim_\tau y \wedge (\forall z:z \in x \rightarrow z \in y).$$

\end{description}

The constructive power of set theory comes from the following axiom.  Any property of objects of a given kind determines a set, of the same kind as the given kind.

\begin{description}

\item[Axiom of comprehension:]  For any sentence $\phi$ of our language and any $u$ not free in $\phi$,

$$(\exists A:{\tt set}(A) \wedge A \sim \tau(u) \wedge (\forall x:x \in A \leftrightarrow x \in \tau(u) \wedge \phi)).$$

There is only one such object $A$ by the axioms of extensionality and empty sets, and we denote it by $\{x \in \tau(u):\phi)$.

This is actually a separate axiom for each formula $\phi$ and so technically it is called an axiom scheme.

\end{description}

There is one more basic axiom of our theory.

\begin{description}

\item[Axiom of binary union:]  $$(\forall AB:(A \sim_\tau B \rightarrow \exists C:(\forall x:x \in C \leftrightarrow x \in A \vee x \in B))).$$  If $A$ and $B$ are sets, there is a unique such object $C$, the union of $A$ and $B$, which we write $A \cup B$.

\end{description}

Now we do some bookkeeping.

\begin{description}

\item[Theorem:]  $\sim_\tau$ is an equivalence relation.  Of course the equivalence classes under this relation are the types $\tau(x)$.

\item[Proof:]  $x \sim_\tau x$ follows from the axiom of types ($x \in \tau(x)$ so $$(\exists z:x \in z \wedge x \in z)$$ [I enjoyed writing that.]

$x \sim_\tau y$, that is, $(\exists z:x \in z \wedge y \in z)$ is equivalent simply as a matter of logic to $y \sim_\tau x$, that is $(\exists z:y \in z \wedge x \in z)$.

Suppose $x \sim_\tau y$ and $y \sim_\tau z$.  It follows by the axiom of types and symmetry of the relation already established that
$x \in \tau(y)$ and $z \in \tau(y)$, which witnesses $x \sim_\tau z$.

\item[Stratification lemma (upward):]  $x \in y \rightarrow y \in \tau^2(x)$

\item[Proof:]  Since $x \in y$, any $z \in y$ also belongs to $\tau(x)$.  So $\{u \in \tau(x):u \in y\}$ exists, belongs to $\tau^2(x)$ by comprehension,
and is equal to $y$ because it is a nonempty set with the same extension.

\end{description}

This lemma handles type inference upward.  Type inference downward is handled by the following

\begin{description}

\item[Stratitification lemma (downward):]  $\tau^2(x) = \tau^2(y) \rightarrow \tau(x) = \tau(y)$.

\item[Proof:]  Suppose $\tau^2(x)=\tau^2(y)$.  Then $\tau(x) \in \tau^2(x)$ and $\tau(y) \in \tau^2(y)=\tau^2(x)$, so
$\tau(x) \sim_\tau \tau(y)$, so $\tau(x) \cup \tau(y)$ exists, and $x$ and $y$ both belong to this union, so $x \sim_\tau y$, 
so $\tau(x)=\tau(y)$.

This is why we have the axiom of binary union.

\end{description}

Combining the previous two results, we get the

\begin{description}

\item[Stratification lemma (full):]  $x \in y \rightarrow (x \in \tau(u) \leftrightarrow y \in \tau^2(u))$

\end{description}



We can define the important notion of power set.

\begin{description}

\item[Definition:]  We define ${\cal P}(x)$, the power set of a set $x$, as $\{y \in \tau(x):y \subseteq x\}$.  Any nonempty set $y$ all of whose elements
belong to $x$ is an element of ${\cal P}(x)$ by downward stratification:  $u \in y$ implies $y \in \tau^2(u)$ and also implies $u \in x$ whence
$x \in \tau^2(u) = \tau(x)$, so $y \in \tau(x)$ so $y \in {\cal P}(x)$.  The only empty element of ${\cal P}(x)$ is of course $\emptyset_{\tau(u)}$, where $u \in x$ or $x = \emptyset_{\tau(u)}$.

\end{description}

One might suppose that ${\cal P}(\tau(x)) = \tau^2(x)$, but the possibility of atoms means we cannot be sure of this.  $\tau$ might or might not be a special case of ${\cal P}$;  in any case they are intellectually distinct, though closely related.

We give an extension of our definition of type hierarchy.

\begin{description}

\item[Definition (type hierarchy, nonpositive indices):]  We extend the notion $\tau^{\bf n}(x)$ to concretely given integer superscripts, not just
natural numbers, defining $\tau^{\bf n-1}(x)$ as the $\tau(u) \in \tau^{\bf n}(x)$, if there is one, and otherwise leave it undefined.  The stratification lemmas together establish that there is at most one such $\tau(u)$.  In particular, $\tau^0(x)$ is the unique $\tau(u)$ which belongs to $\tau(x)$ and includes $x$ as a subset;  if it is undefined, $x$ is an individual.

\end{description}

Our final piece of bookkeeping is to show that for any $x$, the types $\tau^{\bf n}(x)$ are distinct for distinct superscripts.  The observant reader
will notice that we actually cannot say this in generality, because we do not quantify over the superscripts.  But we can prove for each concrete ${\bf n}$ with $n>1$ that $\tau^{\bf n}(x)\neq \tau(x)$.

We warm up for this by showing that $\tau^2(x) \neq \tau(x)$, which is the local version of the infamous Russell paradox.

\begin{description}

\item[Theorem:]  $\tau^2(x) \neq \tau(x)$

\item[Proof:]  Suppose that $\tau^2(x) = \tau(x)$.  Consider the set $R = \{y \in \tau(x): y \not\in y\}$.  We have $R \in \tau^2(x) = \tau(x)$.
Thus $R \in R \leftrightarrow R \not\in R$, a contradiction.

Notice that this implies not only that $\tau(x) \neq \tau^2(x)$, but that these types are disjoint, since distinct equivalence classes are disjoint.

\end{description}

The proof of the result for ${\bf n} >1$ general is quite similar.  It requires a definition.

\begin{description}

\item[Definition:]  We define $\iota(x)$ as $\{y \in \tau(x):y=x\}$.  We define $\iota^{\bf 0}(x)$ as $x$ and $\iota^{\bf n+1}(x)$ as $\iota(\iota^{\bf n}(x))$ for each concrete ${\bf n}$.

Notice that this is the same kind of definition scheme as our definition of the type hierarchy.

\item[Theorem:]  Let ${\bf n}$ be taken to be a concretely given natural number greater than 1.   Then $\tau^{\bf n}(x) \neq \tau(x)$

\item[Proof:]  Consider $R = \{\iota^{\bf n-2}(y)\in \tau^{\bf n-1}(x):\iota^{\bf n-2}(y) \not\in y\}$.  That $\iota^{\bf n-2}(y)\in \tau^{\bf n-1}(x)$ iff $y \in \tau(x)$ for any specific ${\bf n}$ is straightfoward to establish in each case.
$R$ belongs to $\tau^{\bf n}(x)$.  Suppose that $\tau^n(x) = \tau(x)$.  Then  $\iota^{\bf n-2}(R)\in \tau^{\bf n-1}(x)$ and $\iota^{\bf n-2}(R) \in R \leftrightarrow \iota^{\bf n-2}(R) \not\in R$, a contradiction.  It is a very similar argument but with more elaborate bookkeeping.

\end{description}



\newpage


\section{Typed Theory of Sets}

We introduce a typed theory of sets in this section, loosely based on
the historical type theory of Bertrand Russell.  This theory is
sufficiently general to allow the construction of all objects
considered in classical mathematics.  We will demonstrate this by
carrying out some constructions of familiar mathematical systems.  An
advantage of using this type theory is that the constructions we
introduce will not be the same as those you might have seen in other
contexts, which will encourage careful attention to the constructions
and proofs, which furthers other parts of our implicit agenda.  Later
we will introduce a more familiar kind of set theory.  Please note that the development of this section is given independently of the previous section until a remark at the end, but we hope that the relationship between the two approaches will be reasonably clear.

Suppose we are given some sort of mathematical object (natural
numbers, for example).  Then it is natural to consider collections of
natural numbers as another sort of object.  Similarly, when we are
given real numbers as a sort of object, our attention may pass to
collections of real numbers as another sort of object.

Our approach is an abstraction from this.  The basic idea (which we will tweak) is that we introduce a sort of
object which we will call {\em individuals\/} about which
we initially assume nothing whatsoever (we will add an axiom asserting
that there are infinitely many individuals when we see how to say
this).  We also call the sort of individuals {\em type 0\/}.  We then
define {\em type 1\/} as the sort of collections of individuals, {\em
type 2\/} as the sort of collections of type 1 objects, and so forth.
(The tweak is that we actually leave open the possibility that each type $n+1$ contains additional objects over and above the collections of type $n$ objects).

No essential role is played here by natural numbers: we could call
type 0 $\iota$ and for any type $\tau$ let $\tau^+$ be the sort of
collections of type $\tau$ objects, and then the types 0,1,2$\ldots$
would be denoted $\iota,\iota^+,\iota^{\bf ++},\ldots$ in which we can see
that no reference to natural numbers is involved.  This paragraph is
an answer in advance to an objection raised by philosophers: later we
will define the natural number 3 (for example) in type theory: we have
not assumed that we already understand what 3 is by using ``3'' as a
formal name for the type $\iota^{\bf +++}$.

Every variable $x$ comes equipped with a type.  We may write $x^{\bf
3}$ for a type 3 variable, but we will not always do this:  we may write $x$ and expect the type to be deduced from context (type superscripts will be boldface when
they do appear so as not to be confused with exponents or other
numerical superscripts:  we tend to use boldface numerals when we want to emphasize that the use of a numeral in the context under consideration is not a reference to the natural number as a mathematical object).  Atomic
formulas of our language are of the form $x=y$, in which the variables
$x$ and $y$ must be of the same type, and $x \in y$ in which the type
of $y$ must be the successor of the type of $x$.\footnote{Just for fun we give a formal description of the grammatical
requirements for formulas which does not use numerals (in fact,
amusingly, it does not even mention types!).  Please note that we will
not actually {\em use\/} the notation outlined in this paragraph: the
point is that the notation we actually use could be taken as an
abbreviation for this notation, which makes the point firmly that we
are not actually assuming that we know anything about natural numbers
yet when we use numerals as type superscripts.  We use a more
long-winded notation for variables.  We make the following
stipulations: ${\bf x}$ is an individual variable; if $y$ is an
individual variable, so is $y^{\bf '}$; these two rules ensure that we
have infinitely many distinct individual (type 0, but we aren't
mentioning numerals) variables.  Now we define variables in general:
an individual variable is a variable; if $y$ is a variable, $y^{\bf
+}$ is a variable (one type higher, but we are not mentioning
numerals).  Now we define grammatical atomic formulas.  If $x$ is an
individual variable and $y$ is a variable, then $x=y$ is an atomic
formula iff $y$ is an individual variable.  If $x$ is an individual
variable, then $x \in y$ is an atomic formula iff $y$ is of the form
$z^{\bf +}$ where $z$ is an individual variable.  For any variables
$x$ and $y$, $x^{\bf +}=y^{\bf +}$ is an atomic formula iff $x=y$ is
an atomic formula and $x^{\bf +}\in y^{\bf +}$ is an atomic formula
iff $x\in y$ is an atomic formula.  We do not write any atomic formula
which we cannot show to be grammatical using these rules.  The
variable consisting of $x$ followed by $m$ primes and $n$ plusses
might more conveniently be written $x_m^{\bf n}$, but in some formal
sense it does not have to be: there is no essential reference to
numerals here.  The rest of the formal definition of formulas: if
$\phi$ is an atomic formula, it is a formula; if $\phi$ and $\psi$ are
formulas and $x$ is a variable, so are $(\phi)$, $\neg\phi$, $\phi
\wedge\psi$, $\phi \vee\psi$, $\phi\rightarrow \psi$, $\phi \leftrightarrow
\psi$, $(\forall x.\psi)$, $(\exists x.\psi)$ [interpreting formulas
with propositional connectives is made more complicated by order of
operations, but the details are best left to a computer parser!].}

Our theory has axioms.  The inhabitants of every type other than 0 are sets (at least, some of them are).
We believe that sets are equal iff they have exactly the same elements.
This could be expressed as follows:

\begin{description}

\item[$^*$Strong axiom of extensionality:] $$(\forall x.(\forall y.x=y \leftrightarrow
(\forall z.z \in x \leftrightarrow z \in y))),$$ for every assignment of types
to $x,y,z$ that makes sense.

\item[$^*$Proof Strategy:] If $A$ and $B$ are sets, to prove $A=B$,
introduce a new variable $a$, assume $a \in A$, and deduce $a \in B$,
and then introduce a new variable $b$, assume $b \in B$, and deduce $b
\in A$.  This strategy simply unfolds the logical structure of the
axiom of extensionality.

\end{description}

This axiom says that objects of any positive type are equal iff they
have the same elements.  This is the natural criterion for equality
between sets.

Notice that we did not write $$(\forall x^{\bf n+1}.(\forall
y^{\bf n+1}.x^{\bf n+1}=y^{\bf n+1} \leftrightarrow (\forall z^{\bf n}.z^{\bf n} \in x^{\bf n+1} \leftrightarrow z^{\bf n}
\in y^{\bf n+1}))).$$ This would be very cumbersome, and it is not
necessary: it is clear from the form of the sentence (it really is a
sentence!)  that $x$ and $y$ have to have the same type (because $x=y$
appears) and $z$ has to be one level lower in type (because $z \in x$
appears).  One does need to be careful when taking this implicit
approach to typing to make sure that everything one says {\em can\/}
be expressed in the more cumbersome notation:  we will continue to talk about this where appropriate.


Notice that we starred the strong axiom of extensionality; this is
because it is not the axiom we actually adopt.  We take the more
subtle view that in the real world not all objects are sets [and perhaps not all mathematical constructions are implemented as set constructions], so we
might want to allow many non-sets with no elements (it is reasonable
to suppose that anything with an element {\em is\/} a set).  Among the
objects with no elements, we designate a particular object $\emptyset$
as the empty {\em set\/}.

This does mean that we are making our picture of the hierarchy of
types less precise (the tweak that we foreshadowed): type $n+1$ is inhabited by collections of type $n$
objects and also possibly by other junk of an unspecified nature.  A more abstract way of putting it is that our type constructor sending each type $\tau$ to a type $\tau+$ is underspecified:  all we say is that type $\tau+$ includes the collections of type $\tau$ objects.

\begin{description}

\item[Primitive notion:] There is a designated object $\emptyset^{\bf
n+1}$ for each positive type $n+1$ called the {\em empty set\/} of
type $n+1$.  We do not always write the type index.  \footnote{ If we are using the Hilbert symbol in our logic, $(\epsilon x^{\bf n}:A[x^{\bf n}])^{\bf n}$ is of the same type as $x^{\bf n}$, and $(\epsilon x^{\bf n+1}:x^{\bf n+1} \neq x^{\bf n+1})$ [the default object $\delta^{\bf n+1}$ of the same type as $x^{\bf n+1}$] is taken to be $\emptyset^{\bf n+1}$.  The temptation to use $\emptyset^{\bf 0}$ for $\delta^{\bf 0}$ is noted.}

\item[Axiom of the empty set:] $(\forall x.x\not\in \emptyset)$, for
all assignments of a type to $x$ and $\emptyset$ which make sense.

\item[Definition:] We say that an object $x$ (in a positive type) is a
{\em set\/} iff $$x = \emptyset \vee (\exists y.y \in x).$$  We write
${\tt set}(x)$ to abbreviate ``$x$ is a set'' in formulas.  We say
that objects which are not sets are {\em atoms} or {\em urelements\/}:  notice that this only makes sense for objects of positive type.  We do not say that individuals (type 0 objects) are atoms, nor do we say that they are not atoms.

\item[Axiom of extensionality:] $$(\forall xy.{\tt set}(x) \wedge {\tt
set}(y) \rightarrow (x=y\leftrightarrow (\forall z.z\in x
\leftrightarrow z\in y))),$$ for any assignment of types to variables that makes sense.

\item[Proof Strategy:] If $A$ and $B$ are sets, to prove $A=B$,
introduce a new variable $a$, assume $a \in A$, and deduce $a \in B$,
and then introduce a new variable $b$, assume $b \in B$, and deduce $b
\in A$.  This strategy simply unfolds the logical structure of the
axiom of extensionality.

\end{description}



We have already stated a philosophical reason for using a weaker form
of the axiom of extensionality, though it may not be clear that this
is applicable to the context of type theory (one might at first glance
suppose that non-sets are all of type 0); we will see mathematical
reasons for adopting the weaker form of extensionality in the course
of our development (and we will also see mathematical advantages of
strong extensionality).

We have said when sets are equal.  Now we ask what sets there are.
The natural idea is that any property of type $n$ objects should
determine a set of type $n+1$, and this is what we will say:

\begin{description}
\item[Axiom of comprehension:] 

For any formula $A[x]$ in which the
variable $y$ (of type one higher than $x$) does not appear, $$(\exists
y.{\tt set}(y) \wedge (\forall x.x \in y \leftrightarrow A[x]))$$ is an axiom.

\end{description}

This says that for any formula $A[x]$ expressing a property of an
object $x$ (of some type $n$), there is a set $y$ of type $n+1$ such
that the elements of $y$ are exactly the objects $x$ such that $A[x]$.

The axiom of extensionality tells us that there is only one such
object $y$ which is a set (there may be many such objects $y$ if
$A[x]$ is not true for any $x$, but only one of them ($\emptyset$)
will be a set). This suggests a definition:

\begin{description}

\item[Set builder notation:] For any formula $A[x]$, define $\{x \mid
A[x]\}$\footnote{Nowadays (2021) I usually write $\{x :
A[x]\}$, and this form may appear in later edits of the text.} as the unique {\em set\/} of all $x$ such that $A[x]$: an object with exactly these members exists by
Comprehension and there is only one such object which is a set by Extensionality.  If $x$ is
of type $n$, then $\{x \mid A[x]\}$ is of type $n+1$.\footnote{If the Hilbert symbol is used in our logic, we have two comments.  First, $\{x:A[x]\}$ can
be defined as $(\theta x:(\forall y:y \in x \leftrightarrow A[x]))$; in the case where there might  not be a unique object with this extension, the correct one will be chosen magically because the empty set is our default object.  Second, one may want to forbid Hilbert symbols appearing in formulas $A[x]$ used in the Axiom of Comprehension, as this amounts to assuming the Axiom of Choice, of which more below.  Of course, we {\em do\/} as a rule assume the Axiom of Choice, and one might choose to introduce it in this devious way.

A further perhaps amusing observation is that $(\theta x:{\tt set}(x) \wedge A[x])$ can be defined as $\{y:(\exists!x.{\tt set}(x) \wedge A[x]) \wedge (\forall x:{\tt set}(x) \wedge A[x] \rightarrow y \in x)\}$, without any assumption that the Hilbert symbol is in use.}

\item[Proof Strategy:] To use a posit or deduce a goal of the form $t
\in \{x \mid A[x]\}$, replace the posit or goal with the equivalent
$A[t]$.

\end{description}

In our numeral free notation we indicate the grammar requirements for
set abstracts: if $x$ is a variable and $\phi$ is a formula, $\{x^{\bf +}
\mid \phi\}$ can replace any occurrence of $x^{\bf +}$ in a formula and it
will still be a formula.

There are two other axioms in our system, the Axiom of Infinity and
the Axiom of Choice, but some formal development should be carried out
before we introduce them.

And now one may see the point of the untyped preamble.  Fix an object $i$ which is to be taken as inhabiting type {\bf 0}.  Now regard each variable $x^{\bf n}$ as ranging over
values in $\tau^{\bf n+1}(i)$.  Any formula of our typed language can be converted to a formula of the untyped language of the preamble by replacing each quantifier $(\forall x^{\bf n}:\ldots)$ or 
$(\exists x^{\bf n}:\ldots)$ with $(\forall x \in \tau^{\bf n+1}(i):\ldots)$ or 
\newline $(\exists x \in \tau^{\bf n+1}(i):\ldots)$, in which the variable is no longer typed.  Bad formulas $x^{\bf m} = x^{\bf n}$ with $m \neq n$ or $x^{\bf m} \in x^{\bf n}$ with $m+1 \neq n$ become false rather than meaningless.

It is not difficult to show that all the axioms of the typed theory translate to axioms or theorems of the untyped theory in the way indicated.

It is interesting to note that nothing ensures that all objects in the universe of the untyped preamble are in the universe of our type theory:  there doesn't seem to be a way to make such a claim.  There might be other types.  We may revisit the untyped underpinnings of the type theory later.

\newpage

\section{Russell's Paradox?}

At this point an objection might interpose itself.  Consider the
following argument.

For any set $x$, obviously either $x$ is an element of itself or $x$
is not an element of itself.  Form the set $R$ whose elements are
exactly those sets which are not elements of themselves: $R = \{x \mid
x \not\in x\}$.  Now we ask, is $R$ an element of itself?  For any $x$,
$x \in R \leftrightarrow x \not\in x$, so in particular $R \in R \leftrightarrow R
\not\in R$.  This is a contradiction!

This argument, known as {\em Russell's paradox\/}, was a considerable
embarrassment to early efforts to formalize mathematics on the very
abstract level to which we are ascending here.  

Fortunately, it is completely irrelevant to our work here.  This
argument does not work in our system, on a purely formal level,
because $x \in x$ is not a legal formula in the language of our type
theory, so it does not define a property of sets allowing the
introduction of a set by Comprehension!  On a less formal level,
attending to the meaning of notations rather than their formal
structure, we have not introduced the kind of sweeping notion of set
presupposed in the argument for Russell's paradox: for any particular
sort of object $\tau$ (such as type $n$) we have introduced the new
sort of object ``set of $\tau$'s'' or ``$\tau^+$'' (which
we call type $n+1$ in the particular case where $\tau$ is type $n$).
The supposition in Russell's paradox is that we have a type of sets
which contains all sets of objects of that same type.  Ordinary
mathematical constructions do not lead us to a situation where we need
such a type.  If we had a universal sort ${\tt o}$ containing {\em all
objects\/} it might seem that ${\tt o}^+$ would contain all sets of
anything whatsoever (including sets of type ${\tt o}^+$ sets, which
would presumably also be of the universal type ${\tt o}$).  The
argument for Russell's paradox shows that there cannot be such a type
if the Axiom of Comprehension is to apply: either there cannot be a
universal type ${\tt o}$ or the type ${\tt o}^+$ cannot contain all
definable subcollections of ${\tt o}$.  We will introduce untyped set
theories with restrictions on comprehension below.

It is important to notice on a philosophical level that care in the
introduction of the idea of a set has completely avoided the paradox:
there is no embarrassment for our typed notion of set, and our typed
notion of set is true to what we actually do in mathematics.
Russell's paradox was a serious problem for an initial insufficiently
careful development of the foundation of mathematics; it is not
actually a problem for the foundations of mathematics as such, because
the typed notion of set is all that actually occurs in mathematics in
practice (in spite of the fact that the system of set theory which is
customarily used is formally untyped: we shall meet this system in
chapter 3 and see that its restrictions on comprehension can be
naturally motivated in terms of types).

Notice that if $x$ and $y$ are terms of different types, $x=y$ is not
a formula at all.  This does not mean that we say that $x$ and $y$
are distinct: it means that we do not entertain the question as to
whether objects of different types are identical or distinct (for now;
we will have occasion to think about this later).  Similarly, if the
type of $y$ is not the successor of the type of $x$ (for example, if
$x$ and $y$ are of the same type) we do not say $x \in y$ (it is
ungrammatical, not false).  We do not ask whether $x \in x$; we do not
say that it is false (or true) (for now).

If the reader has looked at the untyped preamble, she will have seen an application of the Russell argument to show a fact we cannot express in our typed language.  In the theory of the untyped preamble,
we treat every ill-typed atomic formula as false, rather than taking the liberal view outlined above.  But the liberal view allows this, of course.

\newpage

\section{Simple Ideas of Set Theory}

In this section we develop some familiar ideas of set theory.  In an appendix, the same notions are developed in the theory of the untyped preamble.

We first develop the familiar list notation for finite sets.  Here are
the standard notations for one and two element sets.

\begin{description}

\item[List notation for sets:]  $\{x\}$ is defined as $\{y \mid y=x\}$.
$\{x,y\}$ is defined as $\{z \mid z=x\vee z=y\}$.

\end{description}

It is convenient to define Boolean union and intersection of sets
before giving the general definition of list notation.

\begin{description}

\item[Boolean union and intersection:] If $x$ and $y$ are sets, define
$x \cup y$ as $$\{z \mid z \in x \vee z \in y\}$$ and $x \cap y$ as $$\{z
\mid z \in x \wedge z \in y\}.$$  Notice that though we may informally
think of $x \cup y$ as ``$x$ and $y$'', it is actually the case that
$x \cup y$ is associated with the logical connective $\vee$ and it is
$x \cap y$ that is associated with $\wedge$ in a logical sense.

We also define $a^c$ (the complement of $a$) as $\{x \mid x \not\in
a\}$ and $a-b$ (the set difference of $a$ and $b$) as $a \cap b^c$.

\item[recursive definition of list notation:] $\{x_1,x_2,\ldots,x_n\}$
is defined as $$\{x_1\} \cup \{x_2,\ldots,x_n\}.$$  Notice that the
definition of list notation for $n$ items presupposes the definition
of list notation for $n-1$ items: since we have a definition of list
notation for 1 and 2 items we have a basis for this recursion.

Note that all elements of a set defined by listing must be of the same
type, just as with any set.

\end{description}

There is one more very special case of finite sets which needs special
attention.

\begin{description}

\item[null set:] We have introduced $\emptyset^{\bf n+1}$ as a primitive notion
because we adopted the weak axiom of extensionality.

If we assumed strong extensionality, we could define $\emptyset^{\bf
n+1}$ as $$\{x^{\bf n} \mid x^{\bf n} \neq x^{\bf n}\}$$ (in any event
this set abstract is equal to $\emptyset^{\bf n+1}$!).  Notice that
$\emptyset^{\bf n+1}$ has no elements, and it is by Extensionality
(either form) the only set (of type $n+1$) with no elements.  In this
definition we have used type superscripts, though hereinafter we will
write just $\emptyset$: this is to emphasize that $\emptyset$ is
defined in each positive type and we do not say that the empty sets in
different types are the same (or that they are different).  Notice
that although $x \in x$ is not grammatical, $\emptyset \in \emptyset$
{\em is\/} grammatical (and false!).  It is not an instance of the
ungrammatical form $x \in x$ because the apparent identity of the two
occurrences of $\emptyset$ is a kind of pun.  The pun can be dispelled
by writing $\emptyset^{\bf n+1} \in \emptyset^{\bf n+2}$ explicitly.

\item[universe:] We define $V$ as $\{x \mid x=x\}$.  This is the
universal set.  The universal set in type $n+1$ is the set of all type
$n$ objects.  $V \in V$ is grammatical and true -- but the two
occurrences of $V$ have different reference (this can be written
$V^{\bf n+1} \in V^{\bf n+2}$ for clarification).

\end{description}

Of course we assume that the universal set is not finite, but we do
not know how to say this yet.

The combination of the empty set and list notation allows us to write
things like $\{\emptyset,\{\emptyset\}\}$, but not things like
$\{x,\{x\}\}$: the former expression is another pun, with empty sets
of different types appearing, and the latter expression is
ungrammatical, because it is impossible to make a consistent type
assignment to $x$.  An expression like this can make sense in an
untyped set theory (and in fact in the usual set theory the first
expression here is the most popular way to define the natural number 2, as we
will explain later).

Set builder notation can be generalized.  

\begin{description}

\item[Generalized set builder notation:] If we have a complex term
$t[x_1,\ldots,x_n]$ containing only the indicated variables, we define
$\{t[x_1,\ldots,x_n] \mid A\}$ as $\{y \mid (\exists x_1\ldots
x_n.y=t[x_1,\ldots,x_n] \wedge A)\}$ (where $y$ is a new variable).
We do know that this kind of very abstract definition is not really
intelligible in practice except by backward reference from examples,
and we will provide these!

\item[Examples:] $\{\{x\}\mid x=x\}$ means, by the above convention,
$$\{z \mid (\exists z.z=\{x\}\wedge x=x)\}.$$  It is straightforward to
establish that this is the set of all sets with exactly one element,
and we will see below that we will call this the natural number 1.
The notation $\{\{x,y\}\mid x \neq y\}$ expands out to $\{z \mid
(\exists xy.z = \{x,y\} \wedge x\neq y)\}$: this can be seen to be the
set of all sets with exactly two elements, and we will identify this
set with the natural number 2 below.

\end{description}


We define some familiar relations on sets.

\begin{description}

\item[subset, superset:] We define $A \subseteq B$ as $${\tt
set}(A)\wedge{\tt set}(B) \wedge (\forall x.x \in A \rightarrow x \in
B).$$  We define $A \supseteq B$ as $B \subseteq A$.

\item[Theorem:] For any set $A$, $A \subseteq A$.

\item[Theorem:]  For any sets $A,B$, $A \subseteq B \wedge B \subseteq A \rightarrow A=B$.

\item[Theorem:] For any sets $A,B,C$, if $A \subseteq B$ and $B
\subseteq C$ then $A \subseteq C$.

\item[Observation:] The theorems we have just noted will shortly be
seen to establish that the subset relation is a ``partial order''.

\item[Proof Strategy:] To show that $A \subseteq B$, where $A$ and $B$
are known to be sets, introduce an arbitrary object $x$ and assume
$x\in A$: show that it follows that $x\in B$.

If one has a hypothesis or previously proved statement $A \subseteq B$ and a statement $t \in A$, deduce $t \in B$.

Notice that the proof strategy given above for
proving $A = B$ is equivalent to first proving $A \subseteq B$, then
proving $B \subseteq A$.

\end{description}

The notions of element and subset can be confused, particularly
because mathematicians and math students have a bad habit of saying things like ``$A$ is in $B$'' or
``$A$ is contained in $B$'' both  for $A \in B$ and for $A \subseteq B$.  It
is useful to observe that elements are not ``parts'' of sets.  The
relation of part to whole is transitive: if $A$ is a part of $B$ and
$B$ is a part of $C$, then $A$ is a part of $C$.  The membership
``relation'' is not transitive in a quite severe sense: if $A \in B$
and $B \in C$, then $A \in C$ is not even meaningful in our type
theory!  [In the untyped set theories discussed in chapter 3,
membership is in a quite normal sense not transitive.]  But the subset
relation is transitive: if $A \subseteq B$ and $B \subseteq C$, then
any element of $A$ is also an element of $B$, and so is in turn an
element of $C$, so $A \subseteq C$.  If a set can be said to have
parts, they will be its subsets, and its one-element sets $\{a\}$ for
$a \in A$ can be said to be its atomic parts.



We give a general format for introducing operations, and then
introduce an important operation.

\begin{description}

\item[Definable Operations:] For any formula $\phi[x,y]$ with the
property that $$(\forall xyz.\phi[x,y] \wedge \phi[x,z] \rightarrow
y=z)$$ we define $F_{\phi}(x)$ or $F_{\phi}`x$ as the unique $y$ (if
there is one) such that $\phi[x,y]$.  Note that we will not always
explicitly give a formula $\phi$ defining an operation, but it should
always be clear that such a formula could be given.  Note also that
there might be a type differential between $x$ and $F_{\phi}(x)$
depending on the structure of the formula $\phi[x,y]$.

For any such definable operation $F(x)$, we define $F``x$ for any set
$x$ as $\{F(u)\mid u \in x\}$: $F``x$ is called the (elementwise) {\em
image\/} of $x$ under the operation $F$.

We also support iteration of such operations:  $F^{\bf 0}(x)$ is defined as
$x$ and $F^{\bf n+1}(x)$ is defined as $F(F^{\bf n}(x))$.   The numerals here are in boldface to indicate that no reference to natural numbers as mathematical objects is intended.

\item[Power Set:] For any set $A$, we define ${\cal P}(A)$ as $\{B
\mid B \subseteq A\}$.  The power set of $A$ is the set of all subsets
of $A$.  Notice that ${\cal P}(V^{\bf n})$ is the collection of all
sets of type $n+1$, and is not necessarily the universe $V^{\bf n+1}$,
which might also contain some atoms.

\item[Singleton:]  For any object $x$, we define $\iota(x)=\{x\}$.  The primary use
of this alternative notation for the singleton operation is to allow notations like $\iota^{\bf 3}(x)$ for $\{\{\{x\}\}\}$.

\item[Observation:] $\cal P$ is $F_{\phi}$ where $\phi[x,y]$ is the
formula $(\forall z.z \in y \leftrightarrow z \subseteq x)$ (or just $y = \{z
\mid z \subseteq x\}$).  The operator $\iota$ is $F_{\phi}$ where $\phi$ is $(\forall z.z \in y \leftrightarrow z=x)$.

\end{description}

It is {\bf very important} to notice that ${\cal P}(x)$ is one type
higher than $x$, and similarly that $\iota(x)=\{x\}$ is one type higher than
$x$.



It may be well known to you that union, intersection and complement satisfy the following properties (which demonstrate that the sets in each type with these operations form what is called a {\em Boolean algebra\/}).  You should also notice that these are closely parallel with the properties of disjunction, conjuction, and negation, the logical operations which appear in the definitions of the set operations.
{\tiny
$$\begin{array}{cccc}

{\tt commutative} & A\cup B = B \cup A& & A \cap B = B \cap A \\

{\tt associative}  & (A \cup B) \cup C = A \cup (B \cup C) & & (A \cap B) \cup C = A \cap (B \cap C) \\

{\tt identity}  &  A \cup \emptyset = A& & A \cap V = A \\

{\tt zero} & A \cup V = V & & A \cap \emptyset = \emptyset \\

{\tt idempotent} & A \cup A = A & & A \cap A = A \\

{\tt distributive} & A \cup (B \cap C) = (A \cup B) \cap (A \cup C)& & A \cap (B \cup C) = (A \cap B) \cup (A \cap C) \\

{\tt cancellation} & & (A^c)^c=A&  \\

{\tt deMorgan} & (A \cup B)^c = A^c \cap B^c & & (A \cap B)^c = A^c \cup B^c \\


\end{array}$$}

The properties motivate a style in which $A \cup B$ (or $A \vee B)$ is written $a + b$,  $A \cap B$ (or $A \wedge B)$ is written $ab$, $V$ (or {\tt true}) is written 1
and $\emptyset$ (or {\tt false}) is written 0.  The complement (or negation) operation, which doesn't really correspond to anything in arithmetic, is often written with an overline:  $\overline{a}$ represents $A^c$ (or $\neg A$).

On the next page, we give a proof of a sample  axiom of Boolean algebra.  This is tedious in obvious ways!

\newpage

\begin{description}

\item[Prove:]  $A \cup (B \cap C) = (A \cup B) \cap (A \cup C)$.  The objects to be shown equal are obviously sets:  we use the set equality strategy.

\item[Part I:]

\begin{description}

\item

\item[Assume (1):]  $x \in A \cup (B \cap C) $

\item[Goal:]  $x \in (A \cup B) \cap (A \cup C)$

\item[(2):]  $x \in A \vee x \in B \cap C$  definition of union, 1

We prove the result by cases on 2

\begin{description}

\item[Case I:  assume (2a):]  $x \in A$

\item[Goal:]  $x \in (A \cup B) \cap (A \cup C)$

\item[(3):]  $x \in A \vee x \in B$  addition (2a)

\item[(4):]  $x \in A \vee x \in C$  addition ((2a)

\item [(5):]  $x \in A \cup B$ def union 3

\item[(6):]  $x \in A \cup C$  def union 4

\item[(6.5)]  $x \in  (A \cup B) \wedge x \in  (A \cup C)$ conj 5,6

\item[(7):]  $x \in  (A \cup B) \cap (A \cup C)$  def intersection 6.5

\end{description}

\begin{description}

\item[Case II:  assume (2a):]  $x \in B \cap C$

\item[Goal:]  $x \in (A \cup B) \cap (A \cup C)$

\item[(8):] $x \in B \wedge x \in C$ def intersection 2a

\item[(9):]  $x \in B$ simp 8

\item[(10):]  $x \in C$ simp 8

\item[(11):]  $x \in A \vee x \in B$  addition 9

\item[(12):]  $x \in A \vee x \in C$  addition 10

\item [(13):]  $x \in A \cup B$ def union 11

\item[(14):]  $x \in A \cup C$  def union 12

\item[(15):]  $x \in  (A \cup B) \wedge x \in (A \cup C)$  conj 13,14

\item[(16):]   $x \in  (A \cup B) \cap  (A \cup C)$ def intersection 15
 

\end{description}

\item[(17):] $x \in  (A \cup B) \cap  (A \cup C)$ proof by cases, 2, 2a--7, 2b--16

\end{description}

\newpage

\item[Part II:]

\begin{description}

\item

\item[Assume (18):]  $x \in  (A \cup B) \cap  (A \cup C)$

\item[Goal:]  $x \in A \cup (B \cap C)$

\item[Goal:]  $x \in A \vee x \in B \cap C$ (rewriting goal using definition of union)

\begin{description}

\item[Assume (19):]  $\neg x \in A$

\item[Goal:]  $x \in B \cap C$

\item[(20):]   $x \in A \cup B \wedge x \in A \cup C$  def intersection 17

\item[(21):]  $x \in A \cup B$ simp 20

\item[(22):]  $x \in A \cup C$  simp 20

\item[23):]  $x \in A \vee x \in B$  def union 21

\item[(24):]  $x \in A \vee x \in C$ def union 22

\item[(25):]  $x \in B$  d.s. 19, 23

\item[(26):]  $x \in C$  d.s. 19, 24

\item [(27):]  $x \in B \wedge x \in C$  conj 25, 26

\item[(28):]  $x \in B \cap C$  def int 27

\end{description}

\item[(29):]  $x \in A \vee x \in B \cap C$  alt elim 19-28

\item[(30):]  $x \in A \cup (B \cap C)$  def union 29

\end{description}

\item[the main result is proved:]  by set equality strategy, 1-17, 18-30

\end{description}

\newpage

\subsection{Review of set theory proof strategies}

In the first chapter, you should have noticed our orientation toward letting the form of a statement drive the way we prove it and the way we use it.  This approach of identifying proof strategies can be extended to set theory.  We present some rules along these lines, some of which will already have been mentioned.

Here is a collection of official basic rules for reasoning about sets in our logic.  We assume the type constraints on language stated above, but we don't attach types explicitly to variables here.  The basic notions are membership, equality, the sethood predicate, and set builder notation.

You might want to look back to section 1.11 for basic rules of equality (reflexivity and substitution) as well as familiar derived rules.

\begin{description}

\item[Rule of nonemptiness:]  things with elements are sets.

$$\begin{array}{ccc}

  & t \in u  &  \\ \hline

  & {\tt set}(u)&  \end{array}$$

\item[Rule of abstraction:]  set abstracts (the objects denoted by set builder notation) are sets.  No premises are needed:  this could also simply be stated as an axiom.

$$\begin{array}{ccc}

& &\\ \hline

  & {\tt set}(\{x:P[x]\}) &  

  \end{array}$$

\item[Rule of extensionality for set abstracts.]

$$\begin{array}{ccc}

& (\forall x:P[x] \leftrightarrow Q[x])&\\ \hline

  & \{x:P[x]\} = \{x:Q[x]\} &  

    \end{array}$$

\item[General rule of extensionality:]  the preceding rule can be deduced from this one and the rule of abstraction.  The extensionality rules are further expanded into a more complicated proof strategy using the rules for the universal quantifier and the biconditional below.

$$\begin{array}{ccc}

{\tt set}(t)  &{\tt set}(u)  &  (\forall x:x \in t \leftrightarrow x \in u) \\ \hline

  & t=u &  \end{array}$$

\item[Leibniz rules for equality:] (these could even be taken as a definition of equality).  Notice that this is not extensionality:  this says that things are equal iff they belong to the same sets.  The official rules of equality, reflexivity and substitution, can be deduced from these rules with the aid of other rules of logic and the rules of comprehension below (try it).

$$\begin{array}{ccc}

  & (\forall x: t \in x \leftrightarrow u \in x )&  \\ \hline

  & t=u &  \end{array}$$

$$\begin{array}{ccc}

  & t=u &  \\ \hline

  & (\forall x:t \in x \leftrightarrow u \in x)&  \end{array}$$


\item[First rule of comprehension]

$$\begin{array}{ccc}

  & P[t] &  \\ \hline

  & t \in \{x:P[x]\} &  \end{array}$$

\item[Second rule of comprehension]

$$\begin{array}{ccc}

  & t \in \{x:P[x]\}  &  \\ \hline

  & P[t]  &  \end{array}$$

\end{description}


Here is a discussion of proof strategies.

\begin{description}

\item[to prove that sets are equal:]

Given that $A$ and $B$ are sets, to prove $A=B$, prove $(\forall x:x \in A \leftrightarrow x \in B)$.  This expands out using the proof strategies for the universal quantifier and the biconditional. 

Like the proof of a biconditional,  the proof is divided into two parts.

Let $c$ be arbitary.  Assume $c \in A$ for the sake of argument:  show that $c \in B$.

Let $d$ be arbitrary and assume $d \in B$ for the sake of argument:  show that $d \in A$.

If you have shown these two things, you have shown $A=B$.

\item[to prove subset relations:]

Let $A,B$ be sets.  To prove $A \subseteq B$ is to prove $(\forall x:x \in A \rightarrow x\in B)$.

This expands out to the following strategy:

Let $c$ be arbitary.  Assume that $c \in A$.  Show that $c \in B$.

If you can carry this out, you have shown $A \subseteq B$.

\item[to prove equality using the subset relation:]

Notice that if you have shown $A \subseteq B$ and $B \subseteq A$, you have shown $A=B$.

\item[to use a subset statement:]  If you have $a \in A$ and $A \subseteq B$, deduce $a \in B$.

\item[to prove that an object belongs to a specific set:]

To show that $a \in \{x:P[x]\}$, prove $P[a]$.  This is of course very abstract.  Every defined set theoretical operation amounts to a special case of this.

\item[to use an assertion that an object belongs to a specific set:]

If you have $a \in \{x:P[x]\}$, deduce $P[a]$.  Again, this is very abstract, but every defined set theoretical operation gives us a version of this move.

To show that $a \in \{x:P[x]\}$, prove $P[a]$.  This is of course very abstract.  Every defined set theoretical operation amounts to a special case of this.

\item[a general strategy:  when all else fails, expand definitions.]

To prove or use statements involving defined operations, expand the definitions (and use the comprehension axiom).

For example, to prove $a \in \{b\}$, notice that this is $\{x:x=b\}$, so prove $a=b$.

Similarly, if you have proved or assumed $a \in \{b\}$, you may further deduce $a=b$.

Similarly, $x \in A \cup B$ is equivalent to $x \in A \vee x \in B$, which should be exploited if we want to use or prove a statement about unions.

$x \in A \cap B$ is equivalent to $x \in A \wedge x \in B$, which should be exploited if we want to use or prove a statement about intersections.


$x \in {\cal P}(A)$ is equivalent to $x \subseteq A$ (which will then involve opportunities to apply rules for handling the subset relation).

$x \in \bigcup {\cal A}$ is equivalent to $(\exists A:x \in A \wedge A \in {\cal A})$, which suggests a strategy ``introduce a witness, a set $A \in {\cal A}$ such that $x \in A$".

\item[complex definitions:]  Rinse and repeat.  For example, if we have to use the assertion $a \in b \cup \{c\}$, we can expand this to $a \in b \vee a = c$.  With experience,
this should be pretty automatic.

\end{description}

\newpage

\subsection{Exercises}

\begin{enumerate}

\item Prove $A \cap B = B \cap A$ (using the definition of intersection and the proof strategy indicated for equality of sets:  it comes down to that proof strategy and very easy propositional logic).

\item Prove $(A \cup B)^c = A^c \cap B^c$ (recalling that $A^c$, the complement of $A$ is defined as $\{x \mid x \not\in A\}$).

\item Prove $(A - B)-C = A-(B\cup C)$.

\item Verify the validity of the rule

$$\begin{array}{ccc}

 t \in A& &A \subseteq B \\ \hline

&t \in B&

\end{array}$$  using basic rules of logic and the rules for reasoning with sets given in the previous section.  You may then use this rule (call it ``subset rule").

\item Prove the theorem $A \subseteq B \wedge B \subseteq C \rightarrow A \subseteq C$.  Look at the proof strategies for the subset relation in the text.  Roughly the same proof
justifies the ``rule of transitivity of subset", $$\begin{array}{ccc}
A \subseteq B & & B \subseteq C \\ \hline

& A \subseteq C & \end{array}$$

\item Prove $A \subseteq B \rightarrow {\cal P}(A) \subseteq {\cal P}(B)$.  I did this in class in a rough form:  so obviously I would like something closer to a line by line proof,
and in the text above I have given you finer-grained rules and strategies that you may employ.

\item  Give possible assignments of types to variables in each of the following statements or set expressions, or say that they cannot really be statements in our typed language
(with at least a brief explanation).  There will not be a unique answer, because you can take any assignment of types that works and add one to every type you use (this is an important observation!)

We give some schematics as a reminder:  $$x^{\bf n} = y^{\bf n};  x^{\bf n} \in y^{\bf n+1};  \{x^n:P[x^n]\}^{\bf n+1}.$$  Of course, any defined set operation has
some effect on types:  for example, $(x^{\bf n} \bigcup y^{\bf n})^{\bf n}$, union doesn't move types, but ${\cal P}(x^{\bf n})^{\bf n+1}$, the output of the power set operation is one type higher than the input.  Problem 7 is intended to give you a little tutorial in thinking about types.  I hope I have given enough hints!

\begin{enumerate}

\item  $$(\forall x \in A:(\exists y \in x:y \in B))$$

\item  $$(\forall x:x \in A \wedge y \in x \rightarrow x \in y)$$ [in ordinary set theory, this makes sense and the set $A$ is said to be ``transitive" if this is true.  Here it does not.  Explain why.]

\item  We attempt to define $x^+$ as $x \cup \{x\}$.  Does this make sense in type theory?   If it does, give a possible assignment of types to $x$ and $x^+$.  If it doesn't, explain the problem.  Hint:  you need to think about the type of $\{x\}$ too.

\item  We define the set $\cal F$ of finite sets as $\{x : (\forall I:\emptyset \in I \wedge (\forall xy:x \in I \rightarrow x \in \{y\} \in I) \rightarrow x \in I\}$.  We assure you that this
does indeed make sense.  Find a sensible assignment of types to $\cal F$ and all the variables in its definition.

\item  Compare the similar assertions $\{x :x=x\} \in \{y:y=y\}$ and $\{x :x=x\} \in \{x:x=x\}$.  One of these makes sense in type theory and one of them doesn't.  Explain.

\end{enumerate}

\end{enumerate}

\newpage

\begin{comment}

\section{Digression:  Type theory introduced all over again as an unsorted theory}

In this chapter we introduce a theory of sets, but not the usual one
quite yet.  We choose to introduce a typed theory of sets, which might
carelessly be attributed to Russell, though historically this is not
quite correct.  Our approach to typed set  theory is a little odd in not using typed language,
but typed language will be explained and adopted later.

The development in 2.5 is intended as an alternative approach which might be taken, replacing sections 2.1-4.

\subsection{Types in General}

Mathematical objects come in sorts or kinds (the usual word is
``type'').  We seldom make any statement about all mathematical
objects whatsoever: we are more likely to be talking about all natural
numbers, or all real numbers, or all elements of a certain vector space, etc.

Further, there are standard ways to produce a new sort of object from an old sort, which can be uniformly applied to all or at least many types:  for example,
if $\sigma$ is a sort and $\tau$ is a sort, we can talk about collections of $\sigma$'s or $\tau$'s,  functions from $\sigma$'s to $\tau$'s, ordered pairs of a $\sigma$ and a $\tau$, and so forth.  These are called {\em type constructors\/} when they are considered in general.

In much of this chapter, every variable
we introduce will have a type, and a quantifier over that variable
will be implicitly restricted to that type.  But in our underlying theory variables
can be quantified over the entire universe (though this is not usually useful)
and types are actually specific sets.

\subsection{Typed Theory of Sets}

We introduce a typed theory of sets in this section, loosely based on
the historical type theory of Bertrand Russell.  This theory is
sufficiently general to allow the construction of all objects
considered in classical mathematics.  We will demonstrate this by
carrying out some constructions of familiar mathematical systems.  An
advantage of using this type theory is that the constructions we
introduce will not be the same as those you might have seen in other
contexts, which will encourage careful attention to the constructions
and proofs, which furthers other parts of our implicit agenda.  Later
we will introduce a more familiar kind of set theory.

Suppose we are given some sort of mathematical object (natural
numbers, for example).  Then it is natural to consider collections of
natural numbers as another sort of object.  Similarly, when we are
given real numbers as a sort of object, our attention may pass to
collections of real numbers as another sort of object.

The universe of our theory will be partitioned into special sets called ``types" which can be thought
of as such basic sorts of object.   We suggest an intuitive picture in which there is a basic type
of {\em individuals\/} about which we know nothing, and for each type $\tau$ there is a further type (its `successor type') which contains
at least all sets of type $\tau$ objects, and might contain additional objects.  So we have individuals, sets of individuals, sets of sets of individuals, and so forth.  Our axioms will not immediately suggest this picture:  some theorems will be proved to clarify the import of the axioms.

The undefined notions of our theory are equality, membership, and sethood.

Our theory has axioms governing these notions, as any formal mathematical theory does.

We believe that objects wtih elements are  sets and are equal iff they have exactly the same elements.  We leave open the possibility that there may be some non-sets (atoms), with no elements,  distinguishable from one another,  in various types.
We will also find it desirable to have empty sets in different types which are distinct. 

\begin{description}

\item[Axiom of atoms:]  $$(\forall xy: \neg{\tt set}(x) \rightarrow y \not\in x)$$

\item[Axiom of extensionality:] $$(\forall xyz.z \in x  \rightarrow x=y\leftrightarrow (\forall w.w\in x
\leftrightarrow w\in y)).$$

\end{description}

We have said when (nonempty) sets are equal.  Now we ask what sets there are.

\begin{description}

\item[unbounded set builder notation:]  Let $P[x]$ be a formula in which $A$ does not appear free.  We introduce the notation $\{x \mid P[x]\}$ for the unique object $A$ such that
$(\forall x:x \in A \leftrightarrow P[x])$, if there is such an object.  If there is no object $A$ such that $(\forall x:x \in A \leftrightarrow P[x])$,
or if there is more than one (this latter case can only happen if $(\forall x: \neg P[x])$ holds, by extensionality) we say that $\{x \mid P[x]\}$ does not exist.

This convention for the use of the notation $\{x \mid P[x]\}$  will be superseded when typed language is introduced below.

\end{description}

We give a few words of motivation for the axioms which follow.  Our intention is that our universe is partitioned into particular sets called types.  Any two distinct types are disjoint.  All sets are subollections of types.  Any subcollection of a type which we can define is a set.  On these assumptions, notice that two objects can be expected to be of the same type precisely if there is a set which they both belong to (which is a subcollection of the type to which they both belong).   Thus the relation of being in the same type is definable: $x \sim_{\tau} y$ will be defined as $(\exists z:x \in z \leftrightarrow y \in z)$.  What we want to be able to show is that this relation is an equivalence relation on the universe of all objects whose equivalence classes are sets, and that definable subcollections of these equivalence classes are sets.  The axioms we present are precisely crafted to do this.

We introduce a series of basic constructions of sets which are used to define our types.  Each of these axioms also contains the definition of a notation.

\begin{description}

\item[Axiom of singletons:]   For every object $x$, $\{x\} = \{y\mid y = x\}$ exists.

\item[Axiom of containers:]  For every object $x$, ${\cal B}(x) = \{y \mid x \in y\}$ exists.

\item[Axiom of unions:]  For every set $A$ with a nonempty element, $$\bigcup A = \{x\mid (\exists y:x \in y \wedge y \in A\}$$ exists.



\item[Definition (type):]  The type of $x$, written $\tau(x)$, is defined as $\bigcup {\cal B}(x)$.  Any set $\tau(x)$ may be called a type.

\item[Observations about types:]  Notice that $x \in \tau(y)$ iff $$(\exists z:x \in z \wedge z \in {\cal B}(y)),$$ that is $(\exists z:x \in z \wedge y \in z)$, which is clearly symmetrical:
$x \in \tau(x) \leftrightarrow y \in \tau(x)$.  This suggests that we want a symmetrical notation $x \sim_{\tau} y$ for $x \in \tau(y)$.  Futher, we have $x \sim_{\tau} x$, because $x \in \{x\}$.  We need the further result that $x \sim_{\tau} y \wedge y \sim_{\tau} z \rightarrow x \sim_{\tau} z$, to justify our reading  of $x \sim_{\tau} y$ as ``$x$ has the same type as $y$".  Suppose $x \sim_{\tau} y \wedge y \sim_{\tau} z$.  Then $x \in \tau(y)$ and $z \in \tau(y)$, so $x \sim_\tau z$.   Now we can observe that $x \sim_\tau y$ implies $\tau(x) = \tau(y)$, since, if $x \sim_\tau y$, we have $z \in \tau(x)$ equivalent to $z \sim_\tau x$ which is is turn equivalent to $z \sim_\tau y$ which is in turn equivalent to $z \in \tau(y)$.  Further, it is clear that if two types meet they are identical, from which we see that distinct types are disjoint.

\item[Sophisticated observations about the axioms:]  The existence of singleton set is of course asserted in usual treatments of set theory;  in fact, more usually the existence of pairs is asserted with the existence of singletons as a special case.  Here we do not assert the existence of pairs in full generality, as we want two objects to belong to an unordered pair iff they are of the same type.  What the singletons do for us is provide a set to which any $x$ belongs, which makes $\sim_\tau$ reflexive.  The axiom of containers is just plain unfamiliar in usual treatments of set theory:  it makes sense here because sets
which can contain a given $x$ all belong to a particular type, from which ${\cal B}(x)$ can be extracted as a subollection;  in a more usual treatment, the sets which contain a given $x$ are as it were half the sets in the universe, since every set $a$ belongs to a pair $\{a -\{x\},a \cup \{x\}\}$ of a set which contains $x$ and a set which doesn't.  The axiom of unions is a very usual set theoretical axiom:  its function here is that the union of the container of $x$ is exactly the type of $x$.

\end{description}

We give a general format for introducing operations, which we place here because we immediately want to use the convention on iteration of operations.

\begin{description}

\item[Definable Operations:] For any formula $\phi[x,y]$ with the
property that $$(\forall xyz.\phi[x,y] \wedge \phi[x,z] \rightarrow
y=z)$$ we define $F_{\phi}(x)$ or $F_{\phi}`x$ as the unique $y$ (if
there is one) such that $\phi[x,y]$.  Note that we will not always
explicitly give a formula $\phi$ defining an operation, but it should
always be clear that such a formula could be given.  Note also that
there might be a type differential between $x$ and $F_{\phi}(x)$
depending on the structure of the formula $\phi[x,y]$.

For any such definable operation $F(x)$, we define $F``x$ for any set
$x$ as $\{F(u)\mid u \in x\}$: $F``x$ is called the (elementwise) {\em
image\/} of $x$ under the operation $F$.

We also support iteration of such operations:  $F^{\bf 0}(x)$ is defined as
$x$ and $F^{\bf n+1}(x)$ is defined as $F(F^{\bf n}(x))$.   The numerals here are in boldface to indicate that no reference to natural numbers as mathematical objects is intended:  in particular, such numerical superscripts, even if written merely as letters, are not variables over which one can quantify.



\item[Iteration of types:]  We define $\tau^{\bf 0}(x)$ as $x$ and $\tau^{\bf n+1}(x)$ as $\tau(\tau^{\bf n}(X))$, for each concrete $n$ (such indices are not variables which can be quantified over).  If $x$ is an object of the simplest type we are currently considering, we might refer to $\tau^{\bf n+1}(x)$ as ``type {\bf n}".

\end{description}

Our next axiom expresses the idea that the collection of objects in a given set (for example, a type) with a given property is realized by a unique set in the same type as the given set.

\begin{description}

\item[bounded set builder notation:]  For any set $A$ (in the notation for which the variable $x$ does not appear free) and formula $P[x]$ in which the variable $B$ does not appear free, we define $\{x \in A\mid P[x]\}$ as the unique set $B$ such that
$B \in \tau(A) \wedge (\forall x:x \in B \leftrightarrow x \in A \wedge \phi\}$.

\item[Axiom of comprehension:] 

For any set $A$ and formula $P[x]$, \\ $\{x \in A:P[x]\}$ exists.\footnote{This axiom is more usually called ``separation", but I do not want to rename it in all subsequent references in this book.}

\end{description}

The role of this axiom in the development of our intuitive picture of types may be clearer if we note that the bounding set $A$ can be taken to be a type.

The axiom of comprehension tells us that there is no more than one set in the type of $A$  containing exactly those elements $x$ of $A$ with the property expressed by $P[x]$;  the axiom of extensionality already told us this, except in the special case where there is no $x$ such that $P[x]$:  remember in this case that there may be other objects (atoms) in the same type with the same empty extension.

\begin{description}

\item[Observation:]  A further perhaps amusing observation is that $$(\theta x:x \in S \wedge {\tt set}(x) \wedge A[x])$$ can be defined as $\{y \in S:(\exists!x.A[x]) \wedge y \in x\}$, without any assumption that the Hilbert symbol is in use.

\item[Proof Strategy:] To prove $A=B$, where $A$ and $B$ are sets of the same type,
introduce a new variable $a$, assume $a \in A$, and deduce $a \in B$,
and then introduce a new variable $b$, assume $b \in B$, and deduce $b
\in A$.  This strategy simply unfolds the logical structure of the
axiom of extensionality enhanced with the datum from the axiom of comprehension that each type contains no more than one empty set.


\item[Proof Strategy:] To use a posit or deduce a goal of the form \\  $t
\in \{x \in A \mid P[x]\}$, replace the posit or goal with the equivalent
$t \in A \wedge P[t]$.

\end{description}

There are two other axioms in our system, the Axiom of Infinity and
the Axiom of Choice, but some formal development should be carried out
before we introduce them.

Some very basic facts about the system of types remain to be proved;  we have presented the axioms but we have not unfolded certain consequences essential to the way they are used.

\newpage

\subsection{Russell's Paradox?}

At this point an objection might interpose itself.  Consider the
following argument.

For any set $x$, obviously either $x$ is an element of itself or $x$
is not an element of itself.  Form the set $R$ whose elements are
exactly those sets which are not elements of themselves: $R = \{x \mid
x \not\in x\}$.  Now we ask, is $R$ an element of itself?  For any $x$,
$x \in R \leftrightarrow x \not\in x$, so in particular $R \in R \leftrightarrow R
\not\in R$.  This is a contradiction!

This argument, known as {\em Russell's paradox\/}, was a considerable
embarrassment to early efforts to formalize mathematics on the very
abstract level to which we are ascending here.  

Russell's argument is blocked in the system we have presented because no axiom commits us to the existence
of $\{x \mid x \not\in x\}$.  We {\em are\/} committed to the existence of $\{x \in A\mid x \not\in x\}$ for any fixed $A$, which we might give the nonce name $R_A$.
Comprehension tells us that $R_A \in R_A \leftrightarrow R_A \in A \wedge R_A \not\in R_A$.  From this we can see that $R_A \in A$ would lead to the absurd consequence
$R_A \in R_A \leftrightarrow R_A \not\in R_A$, so we can conclude $R_A \not\in A$.  For any fixed set $A$, we can discover a set which is not in it.

This argument has a practical effect on the project of clarifying the nature of the  type system.  Notice that $R_A \in \tau(A)$ by comprehension, while $R_A \not\in A$.  It follows that for any $x$, $R_{\tau(x)} \in \tau^2(x)$ but
$R_{\tau(x)} \not\in \tau(x)$, from which it follows that $\tau(x) \neq \tau^2(x)$.

The argument can be refined to give a more general result.  Define $\iota(x)$ as $\{x\}$, allowing iteration of the singleton operation.

Define $R^n_A$ as $\{\iota^{\bf n}(x) \in \tau^{\bf n}(A)\mid \iota^{\bf n}(x) \not\in x\}$.  $R^n_A \in \tau^{n+1}(A)$ is evident.  Now consideration of $\iota^{\bf n}(R^n(A)) \in R^n(A)$ shows
that $R^n_A \in A$ is impossible (because if $R^n_A \in A$ holds then $\iota^{\bf n}(R^n_A) \in \tau^{\bf n}(A)$ holds [if $a \in b$ then $\iota(a)$ and $b$ both belong to ${\cal B}(a)$, so $\iota(a) \in \tau(b)$; iterate], and contradiction follows).  If we replace $A$ with
$\tau(x)$, we have shown in general that $\tau(x) \neq \tau^{\bf n+2}(x)$;  all iterated types starting with a fixed type are distinct and thus disjoint.

\newpage

\subsection{Simple Ideas of Set Theory}

In this section we develop some familiar ideas of set theory.

We first develop the familiar list notation for finite sets.  Here are
the standard notations for one and two element sets.

\begin{description}

\item[List notation for sets:]  $\{x\}$ is defined as $\{y \mid y=x\}$.
$\{x,y\}$ is defined as $\{z \mid z=x\vee z=y\}$.

\item[Theorem:]  $\{x,y\}$ exists iff $x \sim_{\tau} y$

\item[Proof:]  If $\{x,y\}$ exists, $x$ and $y$ both belong to it, so $x \sim_{\tau} y$ is immediate.  If $x \sim_{\tau} y$, then $\{z \in \tau(x)\mid z = x \vee z = y\}$ is $\{x,y\}$ by Comprehension.

\end{description}

It is convenient to define Boolean union and intersection of sets
before giving the general definition of list notation.

\begin{description}

\item[Boolean union and intersection:] If $x$ and $y$ are sets in the same type $\tau^2(u)$ define
$x \cup y$ as $$\{z \in \tau(u)\mid z \in x \vee z \in y\}$$ and $x \cap y$ as $$\{z \in \tau(u)
\mid z \in x \wedge z \in y\}.$$  Notice that though we may informally
think of $x \cup y$ as ``$x$ and $y$'', it is actually the case that
$x \cup y$ is associated with the logical connective $\vee$ and it is
$x \cap y$ that is associated with $\wedge$ in a logical sense.

Note that $x \cup y = \bigcup \{x,y\}$ exists if $x \sim_{\tau} y$.  Comprehension gives the existence of $x \cap y = \{z \in x\mid z \in y\} = \{z \in y \mid z \in x\}$ if
$x \sim_{\tau} y$:  if $x$ and $y$ are not of the same type, they will have no common members (if $z$ were a common member of $x$ and $y$, then ${\cal B}(z)$ would witness their being of the same type) and the two definitions of intersection using Comprehension which we give would give empty sets of different types.

If $a, b \in \tau^2(x)$, we define $a^c$ (the complement of $a$) as $\{y \in \tau(x)\mid y \not\in a\}$ and we define $a - b$ as $a \cap b^c$,

\item[recursive definition of list notation:] $\{x_1,x_2,\ldots,x_n\}$
is defined as $$\{x_1\} \cup \{x_2,\ldots,x_n\}.$$  Notice that the
definition of list notation for $n$ items presupposes the definition
of list notation for $n-1$ items: since we have a definition of list
notation for 1 and 2 items we have a basis for this recursion.

Note that all elements of a set defined by listing must be of the same
type for the definition to have the intended effect.

\end{description}



There is one more very special case of finite sets which needs special
attention.

\begin{description}

\item[null set:]  We define $\emptyset_x$ as $\{y \in \tau(x)\mid y \neq y\}$, the unique empty set
in the same type as $\tau(x)$ (the type containing all sets which contain $x$).

\item[universe:] We define $V_x$ as $\{y \in \tau(x) \mid y=y\}$, the universe of $x$ being synonymous with the type of $x$.
\end{description}

Of course we assume that the universal set is not finite, but we do
not know how to say this yet.

The combination of the empty set and list notation allows us to write
things like $\{\emptyset_x,\{\emptyset_x\}\}$, and
$\{x,\{x\}\}$ but they do not have the intended meaning as the purported elements are not of the same type and cannot belong to the same sets.   An expression like this can make sense in an
untyped set theory (and in fact in the usual set theory the first
expression here (without subscripts) is the most popular way to define the natural number 2, as we
will explain later).  We {\em can\/} make sense of $\{\emptyset_{\{x\}},\{\emptyset_x\}\}$, but this is only an implementation of the von Neumann 2 by means of a sort of pun.

Set builder notation can be generalized.  

\begin{description}

\item[Generalized set builder notation:] If we have a complex term
$t[x_1,\ldots,x_n]$ containing only the indicated variables, we define
$\{t[x_1,\ldots,x_n] (\in S) \mid A\}$ as $\{y (\in S) \mid (\exists x_1\ldots
x_n.y=t[x_1,\ldots,x_n] \wedge A)\}$ (where $y$ is a new variable).
We do know that this kind of very abstract definition is not really
intelligible in practice except by backward reference from examples,
and we will provide these!

\item[Examples:] $\{\{x\}\in \tau^2(u) \mid x=x\}$ means, by the above convention,
$$\{z \in \tau^2(u)\mid (\exists x.z=\{x\}\wedge x=x)\}.$$  It is straightforward to
establish that this is the set of all sets (in a certain type) with exactly one element,
and we will see below that we will call this the natural number 1.
The notation $$\{\{x,y\}\in \tau^2(u)\mid x \neq y\}$$ expands out to $\{z\in \tau^2(u)  \mid
(\exists xy.z = \{x,y\} \wedge x\neq y)\}$: this can be seen to be the
set of all sets with exactly two elements (of a certain type), and we will identify this
set with the natural number 2 below.

\end{description}


We define some familiar relations on sets.

\begin{description}

\item[subset, superset:] We define $A \subseteq B$ as $${\tt
set}(A)\wedge{\tt set}(B) \wedge A \sim_{\tau} B \wedge (\forall x.x \in A \rightarrow x \in
B).$$  We define $A \supseteq B$ as $B \subseteq A$.

\item[Theorem:] For any set $A$, $A \subseteq A$.

\item[Theorem:]  For any sets $A,B$, $A \subseteq B \wedge B \subseteq A \rightarrow A=B$.

\item[Theorem:] For any sets $A,B,C$, if $A \subseteq B$ and $B
\subseteq C$ then $A \subseteq C$.

\item[Observation:] The theorems we have just noted will shortly be
seen to establish that the subset relation is a ``partial order''.

\item[Proof Strategy:] To show that $A \subseteq B$, where $A$ and $B$
are known to be sets of the same type, introduce an arbitrary object $x$ and assume
$x\in A$: show that it follows that $x\in B$.

If one has a hypothesis or previously proved statement $A \subseteq B$ and a statement $t \in A$, deduce $t \in B$.

Notice that the proof strategy given above for
proving $A = B$ is equivalent to first proving $A \subseteq B$, then
proving $B \subseteq A$.

\end{description}

The notions of element and subset can be confused, particularly
because mathematicians and math students have a bad habit of saying things like ``$A$ is in $B$'' or
``$A$ is contained in $B$'' both  for $A \in B$ and for $A \subseteq B$.  It
is useful to observe that elements are not ``parts'' of sets.  The
relation of part to whole is transitive: if $A$ is a part of $B$ and
$B$ is a part of $C$, then $A$ is a part of $C$.  The membership
relation is not transitive in a strong sense:  if $A \in B$
and $B \in C$, then $A \not\in C$ by lemmas we have proved about types.  [In the untyped set theories discussed in chapter 3,
membership is in a quite normal sense not transitive.]  But the subset
relation is transitive: if $A \subseteq B$ and $B \subseteq C$, then
any element of $A$ is also an element of $B$, and so is in turn an
element of $C$, so $A \subseteq C$.  If a set can be said to have
parts, they will be its subsets, and its one-element sets $\{a\}$ for
$a \in A$ can be said to be its atomic parts.



\begin{description}

\item[Power Set:] For any set $A$, we define ${\cal P}(A)$ as $\{B
\mid B \subseteq A\}$.  The power set of $A$ is the set of all subsets
of $A$.  Notice that ${\cal P}(V_x)$ is the collection of all
sets of type $\tau^2(x)$, and is not necessarily the universe $V_{\{x\}}$,
which might also contain some atoms.

\item[Singleton:]  For any object $x$, we define $\iota(x)=\{x\}$.  The primary use
of this alternative notation for the singleton operation is to allow notations like $\iota^{\bf 3}(x)$ for $\{\{\{x\}\}\}$.

\item[Observation:] $\cal P$ is $F_{\phi}$ where $\phi[x,y]$ is the
formula $(\forall z.z \in y \leftrightarrow z \subseteq x)$ (or just $y = \{z
\mid z \subseteq x\}$).  The operator $\iota$ is $F_{\phi}$ where $\phi$ is \\$(\forall z.z \in y \leftrightarrow z=x)$.

\end{description}

It is important to notice that ${\cal P}(x)$ and $\{x\}$ do not belong to the same type as $x$.  $x \in {\cal P}(x), x \in \{x\}, x \in \tau(x)$ show that these three objects belong to a common set ${\cal B}(x)$, and so to the type $\tau^2(x) \neq \tau(x)$.

We examine the relationship between power sets and types.

\begin{description}

\item[Theorem:]  If $x \in y$, then ${\cal P}(\tau(x)) \subseteq \tau^2(x) = \tau(y)$.  Further, all elements of $\tau(y) - {\cal P}(\tau(x))$ are atoms.

\item[Proof:]  $x \in y$ and $x \in \tau(x)$ show that $y$ and $\tau(x)$ belong to the common set ${\cal B}(x)$ and so $\tau^2(x)=\tau(y)$.

If $z \in {\cal P}(\tau(x))$, then $z \subseteq \tau(x)$, so $z \sim_\tau \tau(x)$, so $z \in \tau^2(x)=\tau(y)$

If $z \in \tau(y)$ and $z$ is not an atom, then either $z$ is an empty set, in which case $z$ is the empty set belonging to ${\cal P}(\tau(x))$ as there is only one empty set in a type,
or for some $w$, $w \in z$.  Observe that $x$ and $w$ both belong to $y \cup z$, so $w \in \tau(x)$.  But this is true of every $w \in z$, so $x \subseteq \tau(x)$, that is,
$w \in {\cal P}(\tau(x))$.  We have shown that every set in $\tau(y)$ belongs to ${\cal P}(\tau(x))$, so whatever other elements it has are atoms.

\end{description}

A base type (one which is not itself the type of a type) may take two forms.  It either contains nothing but atoms, or a single empty set and possibly some atoms.

It may be well known to you that union, intersection and complement satisfy the following properties (which demonstrate that the sets in each type with these operations form what is called a {\em Boolean algebra\/}).  You should also notice that these are closely parallel with the properties of disjunction, conjuction, and negation, the logical operations which appear in the definitions of the set operations.
{\tiny
$$\begin{array}{cccc}

{\tt commutative} & A\cup B = B \cup A& & A \cap B = B \cap A \\

{\tt associative}  & (A \cup B) \cup C = A \cup (B \cup C) & & (A \cap B) \cup C = A \cap (B \cap C) \\

{\tt identity}  &  A \cup \emptyset = A& & A \cap V = A \\

{\tt zero} & A \cup V = V & & A \cap \emptyset = \emptyset \\

{\tt idempotent} & A \cup A = A & & A \cap A = A \\

{\tt distributive} & A \cup (B \cap C) = (A \cup B) \cap (A \cup C)& & A \cap (B \cup C) = (A \cap B) \cup (A \cap C) \\

{\tt cancellation} & & (A^c)^c=A&  \\

{\tt deMorgan} & (A \cup B)^c = A^c \cap B^c & & (A \cap B)^c = A^c \cup B^c \\


\end{array}$$}

The properties motivate a style in which $A \cup B$ (or $A \vee B)$ is written $a + b$,  $A \cap B$ (or $A \wedge B)$ is written $ab$, $V$ (or {\tt true}) is written 1
and $\emptyset$ (or {\tt false}) is written 0.  The complement (or negation) operation, which doesn't really correspond to anything in arithmetic, is often written with an overline:  $\overline{a}$ represents $A^c$ (or $\neg A$).

On the next page, we give a proof of a sample  axiom of Boolean algebra.  This is tedious in obvious ways!

\newpage

\begin{description}

\item[Prove:]  $A \cup (B \cap C) = (A \cup B) \cap (A \cup C)$.  The objects to be shown equal are obviously sets:  we use the set equality strategy.

\item[Part I:]

\begin{description}

\item

\item[Assume (1):]  $x \in A \cup (B \cap C) $

\item[Goal:]  $x \in (A \cup B) \cap (A \cup C)$

\item[(2):]  $x \in A \vee x \in B \cap C$  definition of union, 1

We prove the result by cases on 2

\begin{description}

\item[Case I:  assume (2a):]  $x \in A$

\item[Goal:]  $x \in (A \cup B) \cap (A \cup C)$

\item[(3):]  $x \in A \vee x \in B$  addition (2a)

\item[(4):]  $x \in A \vee x \in C$  addition ((2a)

\item [(5):]  $x \in A \cup B$ def union 3

\item[(6):]  $x \in A \cup C$  def union 4

\item[(6.5)]  $x \in  (A \cup B) \wedge x \in  (A \cup C)$ conj 5,6

\item[(7):]  $x \in  (A \cup B) \cap (A \cup C)$  def intersection 6.5

\end{description}

\begin{description}

\item[Case II:  assume (2a):]  $x \in B \cap C$

\item[Goal:]  $x \in (A \cup B) \cap (A \cup C)$

\item[(8):] $x \in B \wedge x \in C$ def intersection 2a

\item[(9):]  $x \in B$ simp 8

\item[(10):]  $x \in C$ simp 8

\item[(11):]  $x \in A \vee x \in B$  addition 9

\item[(12):]  $x \in A \vee x \in C$  addition 10

\item [(13):]  $x \in A \cup B$ def union 11

\item[(14):]  $x \in A \cup C$  def union 12

\item[(15):]  $x \in  (A \cup B) \wedge x \in (A \cup C)$  conj 13,14

\item[(16):]   $x \in  (A \cup B) \cap  (A \cup C)$ def intersection 15
 

\end{description}

\item[(17):] $x \in  (A \cup B) \cap  (A \cup C)$ proof by cases, 2, 2a--7, 2b--16

\end{description}

\newpage

\item[Part II:]

\begin{description}

\item

\item[Assume (18):]  $x \in  (A \cup B) \cap  (A \cup C)$

\item[Goal:]  $x \in A \cup (B \cap C)$

\item[Goal:]  $x \in A \vee x \in B \cap C$ (rewriting goal using definition of union)

\begin{description}

\item[Assume (19):]  $\neg x \in A$

\item[Goal:]  $x \in B \cap C$

\item[(20):]   $x \in A \cup B \wedge x \in A \cup C$  def intersection 17

\item[(21):]  $x \in A \cup B$ simp 20

\item[(22):]  $x \in A \cup C$  simp 20

\item[23):]  $x \in A \vee x \in B$  def union 21

\item[(24):]  $x \in A \vee x \in C$ def union 22

\item[(25):]  $x \in B$  d.s. 19, 23

\item[(26):]  $x \in C$  d.s. 19, 24

\item [(27):]  $x \in B \wedge x \in C$  conj 25, 26

\item[(28):]  $x \in B \cap C$  def int 27

\end{description}

\item[(29):]  $x \in A \vee x \in B \cap C$  alt elim 19-28

\item[(30):]  $x \in A \cup (B \cap C)$  def union 29

\end{description}

\item[the main result is proved:]  by set equality strategy, 1-17, 18-30

\end{description}

\newpage

\subsection{Exercises}

\begin{enumerate}

\item Prove $A \cup B = B \cup A$ (using the definition of union and the proof strategy indicated for equality of sets:  it comes down to that proof strategy and very easy propositional logic).

\item Prove $(A \cup B)^c = A^c \cap B^c$ (recalling that $A^c$, the complement of $A$ is defined as $\{x \mid x \not\in A\}$).

\item Prove the theorem $A \subseteq B \wedge B \subseteq C \rightarrow A \subseteq C$.  Look at the proof strategies for the subset relation in the text.

\item Prove $A \subseteq B \rightarrow {\cal P}(A) \subseteq {\cal P}(B)$.

\end{enumerate}

\newpage

\subsection{Typed Language}

To start the work of this section, note that for any objects $x$ and $y$, if $x=y$ we have $\tau(x)=\tau(y)$, and if $x \in y$ we have $\tau^2(x)=\tau(y)$.

We suggest a convention in which every variable is supposed bounded in a type $\tau^n(u)$ (where $n$ can be different for different variables), where $u$ is a variable of the simplest type considered (or just a conveniently chosen constant).  Any bound variable
is supposed to be restricted to the appropriate $\tau^n(u)$.  The bounding variable in a set abstract is also supposed restricted to a type if no explicit bound is given (this supersedes our original convention on the meaning of $\{x \mid P[x]\}$).

This typed language convention is used only with formulas where such an assignment of types $\tau^n(u)$ to variables  can be made  in a way compatible with the conditions on atomic formulas:  in a subformula $x=y$, $x$ and $y$ are assigned the same type, and in a subformula $x \in y$, $x$ and $y$ are assigned successive types (if $x$ is assigned type $\tau^n(u)$, $y$ is assigned type $\tau^{n+1}(u)$).  Such formulas are called
``stratified".   Notice that uniformly raising or lowering all type assignments by a fixed amount will still give a valid type assignment.  We do {\bf not} normally exhibit a specific type assignment:  we commit ourselves simply to the possibility of a type assigment.

It should be noted that type theory is usually presented in a strictly typed language in which each variable has a numerical type as a fixed feature, and $x=y$ is well-formed
iff $x$ and $y$ have the same type while $x \in y$ is well-formed iff the type of $y$ is the successor of the type of $x$.  For us the formula $x \in x$ is false and moreover cannot appear in a formula to which the typed language convention is to be applied:  in a more usual presentation of type theory, $x \in x$ would be meaningless.

It is worth noting that we can traverse types downward as well as upward:  we can define a map $\tau^{\bf -1}$ which sends a type $\tau(x)$ to the unique type (if there is one)
which is an element of $\tau(x)$.  In fact $\bigcup \tau(u)$ is $\tau^{-1}(\tau(u))$ if both exist.  $\tau^{\bf -n}(x)$ can be read as $(\tau^{\bf -1})^n(x)$.  This operation can be used, if appropriate, to extend type assignments downward.

We prove a theorem which makes our use of the typed language convention seem perhaps more reasonable.

\begin{description}

\item[Theorem:]  Each instance $\{x \in A:P[x]\}$ of the axiom of comprehension in which a type assignment is made for each free variable is equivalent to a stratified instance of comprehension in which every quantifier is bounded in a type (a suitable target for the typed language convention).   Of course a type assignment for $x$ is implicit in the choice of $A$.

\item[Proof:]  We describe a transformation of formulas which forces type assignments.  This transformation commutes with negation and disjunction (and other propositional logixl operators).
An atomic formula is fixed by the transformation if the assignments of types to its components are appropriate, and transforms to $x \neq x$ otherwise.
The work is in seeing how it works on quantifiers.   Existentially quantified statements can be viewed as negations of universally quantified statements.  $(\forall u:Q[u])$ transforms into a conjunction of formulas, one for each possible assignment of a type to $u$,
of the form $(\forall u \in \tau:Q^*[u])$, where $Q^*[u]$ is the result of applying the transformation to $Q[u]$ with the given assignment of type to $u$, with the proviso
that if $Q^*[u]$ doesn't contain any occurrence of $u$ (as will happen with all but finitely many type assignments) the quantifier over $u$ is dropped.  This makes the conjunction finite.  We also provide that any subformula which does not contain any variable which is connected to $x$ can simply be read as a truth value, $x=x$ or $x \neq x$ as appropriate  (the relation of connectedness being the smallest transtive relation on variables  extending ``occurs together in an atomic subformula of $P[x]$").   This process clearly terminates in a stratified formula with all quantifiers bounded in types:  it is important to notice that because we work downward, and because we start with type assignments to $x$ and to all free variables and eliminate variables not connected to $x$, we will have type assignments for variables in atomic formulas when we reach them, so they will end up either in their original shape and well-typed, or in the stratified shape $x \neq x$.

\end{description}

When this convention is being used there is a certain vagueness in play:  the lowest type (the objects treated as individuals for the moment) is being left undetermined.  This vagueness turns out actually be be useful in practice.  The underlying idea is that we have no interest in any specific type as the type of individuals:  we are proving general facts true for any type if it is considered as the domain of individuals.

The subsequent text was written with strict types assumed but with type assignments left unspecified.   Any references to ``type {\bf k}" can be read as references
to $\tau^{\bf k+1}(u)$, where $u$ is an object of the type currently viewed as featureless individuals.  If the reader prefers, they can interpret type {\bf 0} as an actual base type.
But notice that our theory in type-free language actually does not prove that there is a base type at all!

A further subtle point is that defined constants may appear with more than one deducible type in the same formula;  with care we can read these sensibly.  $V$ for example
stands for $\{x \mid x=x\}$.  The statement $V \in V$, read $\{x \mid x=x\} \in \{y \mid y=y\}$, is {\em true\/} under the typed variable convention, as the types deduced for the two instances of $V$ are different.  $V \in V$ is true, but it is not a substitution  instance of $x \in x$, which is always a false statement;  the identity of the two occurrences of $V$ is a sort of pun.
It can be clarified as $V_x \in V_{\{x\}}$, which is indeed true.

\newpage

\end{comment}





\section{Finite Number; the Axiom of Infinity; Ordered Pairs}

In the usual untyped set theory, the natural numbers are usually
defined using a clever scheme due to John von Neumann.

\begin{description}

\item[$^*$Definition:] 0 is defined as $\emptyset$.  1 is defined as
$\{0\}$.  2 is defined as $\{0,1\}$.  3 is defined as $\{0,1,2\}$.  In
general, $n+1$ is defined as $n \cup \{n\}$.

\end{description}

The star on this ``definition'' indicates that we do not use it here.
The problem is that this definition makes no sense in our typed
language.  Notice that there is no consistent way to assign a type to
$n$ in ``$n \cup \{n\}$''.  In chapter 3 on untyped set theory, we
will be able to use this definition and we will see that it
generalizes to an incredibly slick definition of ordinal number.

The motivation of our definition of natural number in type theory is
the following

\begin{description}

\item[Circular Definition:] The natural number $n$ is the set of all
sets with $n$ elements.

\end{description}

Of course this will not be acceptable as a formal definition:  we spend the rest of the section showing how we can implement it using a series of formally valid definitions.

It is amusing to observe that the von Neumann definition above can
also be motivated using another

\begin{description}

\item[$^*$Circular Definition:] The natural number $n$ is the set of
all natural numbers less than $n$.

\end{description}

This is starred to indicate that we are not at this point using it at all!


\begin{description}
\item[Definition:]  We define 0 as $\{\emptyset\}$.  

\end{description}

Note that we have thus defined 0 as the set
of all sets with zero (no) elements.

\begin{description}

\item [Definition:] For any set $A$, define $\sigma(A)$ as $\{x \cup \{y\} \mid x \in A \wedge y
\not\in x\}$.  $\sigma(A)$, which we call the {\em successor\/} of $A$, is the collection of all sets obtained by
adjoining a single new element to an element of $A$. 

\item[Alternative Definition:]  For any sets $A,B$, define $A+B$ as $$\{a \cup b\mid a \in A \wedge b \in B \wedge a \cap b = \emptyset\}.$$  Define 1 as $$\{x\mid (\exists y:(\forall z:z \in x \leftrightarrow z=y))\},$$ or equivalently $\{\{x\}:x=x\}$.  It is straightforward to see that $A+1$ defined using these definitions is the same as $\sigma(A)$.  This notion of addition for general sets extends the notion we will eventually define for natural numbers.

\item[Definition:] We define 1 as above or as $\sigma(0)$.  (Observe that 1 is the set
of all one-element sets (singletons).)  We define 2 as $\sigma(1)$ or 1+1, 3 as $\sigma(2)$ or 2+1,
and so forth (and observe that 2 is the set of all sets with exactly
two elements, 3 is the set of all sets with exactly three elements,
and so forth).

\end{description}

Unfortunately, ``and so forth'' is a warning that a careful formal
examination is needed at this point!

\begin{description}

\item[Definition:] We call a set $I$ an {\em inductive set\/} if $0
\in I$ and $$(\forall A.A \in I \rightarrow \sigma(A) \in I).$$  We define
$\cal I$ as the set of all inductive sets.

\end{description}

At this point it is useful to define the unions and intersections of
not necessarily finite collections of sets.  

\begin{description}

\item[Definition:]  For any set $A$, we
define $\bigcup A$ as $$\{x \mid (\exists a \in A.x \in a)\}$$ and
$\bigcap A$ as $$\{x \mid (\forall a \in A.x \in a)\}.$$  (Notice that $x
\cup y = \bigcup \{x,y\}$ and $x \cap y = \bigcap\{x,y\}$.)\footnote{For some purposes, it is useful to modify the definition of $\bigcup A$ so that
when $x$ is an atom, $\bigcup \{x\} = x$.}

\item[Observation:] Notice that $\bigcup A$ and $\bigcap A$ are of
type ${\bf n+1}$ if $A$ is of type ${\bf n+2}$ (we are using boldface here to clearly indicate where I am talking about types rather than natural numbers).

\item[Definition:]  We define ${\mathbb N}$, the set of all natural numbers, as $\bigcap
{\cal I}$, the intersection of all inductive sets.\footnote{In the system of the unsorted preamble, the discussion above can be adapted to
define $0_x$ as $\{\emptyset_x\}$, an element of $\tau^3(x)$, to define $\sigma(A)$ for any set $A$ as $$\{a \cup \{x\} \in \tau^{-1}(\tau(A)):a \in A \wedge x \not\in a\},$$ which is of the same type as $A$, thus allowing definition of $0_x,1_x,2_x\ldots$, the natural numbers in $\tau^3(x)$.  We can then define ${\mathbb N}_x  \in \tau^4(x)$ as the intersection of all inductive sets in $\tau^4(x)$.}

\end{description}

We saw above that 0 has been successfully defined as the set of all
zero element sets, 1 as the set of all one-element sets, 2 as the set
of all two-element sets and so forth (whenever and so forth, etc,
$\ldots$ or similar devices appear in mathematical talk, it is a
signal that there is something the author hopes you will see so that
he or she does not have to explain it!)  So we can believe for each of
the familiar natural numbers (as far as we care to count) that we have
implemented it as a set.  If $I$ is an inductive set, we can see that
(the set implementing) 0 is in $I$ by the definition of ``inductive''.
If the set implementing the familiar natural number $n$ is in $I$,
then (by definition of ``inductive'') the set implementing the
familiar natural number $n+1$ will be in $I$.  So by the principle of
mathematical induction, sets implementing each of the familiar natural
numbers are in $I$.  But $I$ was {\em any\/} inductive set, so for
each familiar natural number $n$, the set implementing $n$ is in the
intersection of all inductive sets, that is in $\mathbb N$ as we have
defined it.  This is why we call inductive sets ``inductive'', by the
way.  How can we be sure that there aren't some other unintended
elements of ${\mathbb N}$?  The best argument we can give is this: if
there is a collection containing exactly the implementations of the
familiar natural numbers, we observe that 0 is certainly in it and
$n+1$ must be in it if $n$ is in it.  So this collection is inductive,
so any element of ${\mathbb N}$, the intersection of all inductive
sets, must belong to this set too, and so must be one of the familiar
natural numbers.  We will see later that there are models of type
theory (and of untyped set theory) in which there {\em are\/}
``unintended'' elements of ${\mathbb N}$.  In such models the
collection of familiar natural numbers must fail to be a set.  How can
this happen when each type ${\bf k+1}$ is supposed to be the collection of {\em
all\/} sets of type ${\bf k}$ objects?  Notice that the axiom of comprehension
only forces us to implement the subcollections of type $\bf k$ which are
definable using a formula of our language as type $\bf k+1$ objects.  So
if there are ``unintended'' natural numbers we will find that no
formula of our language will pick out just the familiar natural
numbers.  If we insist that each type $\bf k+1$ contain {\em all\/}
collections of type $\bf k$ objects, it will follow that we have defined the
set of natural numbers correctly.

\begin{description}

\item[Definition:] We define ${\mathbb F}$, the set of all finite
sets, as $\bigcup {\mathbb N}$.  A set which is not finite (not an
element of ${\mathbb F}$) is said to be {\em infinite\/}.

\end{description}

Since we have defined each natural number $n$ as the set of all sets
with $n$ elements, this is the correct definition of finite set (a
finite set is a set which has $n$ elements for some natural number
$n$, so exactly a set which belongs to $n$ for some $n \in {\mathbb
N}$).

Now we can state a promised axiom.

\begin{description}

\item[Axiom of Infinity:]  $V \not\in {\mathbb F}$

\end{description}

This says exactly that the universe is infinite.

In all of this, we have not issued the usual warnings about types.  We
summarize them here.  For $A+1$ to be defined, a set must be of at
least type 2.  $A+1$ is of the same type as $A$.  Similarly, 0 is of
type at least 2 (and there is a formally distinct $0^{\bf n+2}$ for each
$n$).  Any inductive set must be of at least type 3 and the set of all
inductive sets $\cal I$ is of at least type 4.  $\mathbb N$ is then of
type at least 3 (it being the minimal inductive set) and there is
actually a $\mathbb N^{\bf n+3}$ in each type $\bf n+3$.  An amusing pun which
you may check is $0 \in 1$.  The Axiom of Infinity, like the two
earlier axioms, says something about each type: the universal set over
each type is infinite (it could be written more precisely as $V^{\bf n+1}
\not\in {\mathbb F}^{\bf n+2}$).

We state basic properties of the natural numbers.  These are Peano's
axioms for arithmetic in their original form.  The theory with these
axioms (which makes essential use of sets of natural numbers in its
formulation) is called {\em second-order Peano arithmetic\/}.

\begin{enumerate}

\item $0 \in {\mathbb N}$

\item For each $n \in {\mathbb N}$, $\sigma(n) \in {\mathbb N}$.

\item For all $n \in {\mathbb N}$, $\sigma(n) \neq 0$

\item For all $m,n \in {\mathbb N}$, $\sigma(m)=\sigma(n)\rightarrow m=n$.

\item For any set $I \subseteq {\mathbb N}$ such that $0 \in I$ and
for all $n\in I$, $\sigma(n) \in I$, all natural numbers belong to $I$ (the
principle of mathematical induction).

\end{enumerate}

All of these are obvious from the definition of ${\mathbb N}$ except
axiom 4.  It is axiom 4 that hinges on the adoption of the Axiom of
Infinity.

The principle of mathematical induction (axiom 5) can be presented as
another

\begin{description}

\item[Proof Strategy:] To deduce a goal $$(\forall n \in {\mathbb
N}.\phi[n]),$$ define $A$ as the set $\{n \in {\mathbb N}\mid \phi[n]\}$ and deduce the following goals:

\begin{description}
\item[Basis step:] $0 \in A$

\item[Induction step:] The goal is $(\forall k \in {\mathbb N}\mid k
\in A \rightarrow \sigma(k) \in A)$ (or $(\forall k \in {\mathbb N}\mid k
\in A \rightarrow k+1 \in A)$): to prove this, let $k$ be an arbitary
natural number, assume $k \in A$ (equivalently $\phi[k]$) (called the
{\em inductive hypothesis\/}) and deduce the new goal $\sigma(k) \in A$, or $k+1 \in A$
(equivalently $\phi[\sigma(k)]$, or $\phi[k+1]$).

\end{description}

\end{description}

We prove some theorems about natural numbers.   Our aim is to prove the equivalence of the Axiom of Infinity and Peano's Axiom 4.  We will start by trying this and failing, but the nature of our failure will indicate what lemmas we need to prove for ultimate success.

\begin{description}

\item[$^*$Theorem (using Infinity):]   For all $m,n \in {\mathbb N}$, $m+1=n+1\rightarrow m=n$.

\item[$^*$Proof:]  Suppose that $m$ and $n$ are natural numbers and $m+1=n+1$.   Our aim is to show that $m=n$.   We show this by choosing an arbitrary element
$a$ of $m$ and showing that it belongs to $n$ (and also the converse, but this will be direct by symmetry).  Suppose $x \not\in a$ (we can find such an $x$ by Infinity).
Now $a \cup \{x\} \in m+1$, by definition, so it is in $n+1$.  It seems that from $a \cup \{x\} \in n+1$, $a \in n$ should follow, but this will require more work.   There is certainly
no general result that $x \not\in a$ and $a \cup \{x\} \in A + 1$ implies $a \in A$.   Suppose that $A = \{\{0,1\}\}$.   Then $\{0,1\} \cup \{2\} = \{0,2\} \cup \{1\} \in A +1$ and $1 \not\in \{0,2\}$, but $\{0,2\} \not\in A$.  We do believe that $a \cup \{x\} \in n+1$ and $x \not\in a$ implies $a \in n$, when $n$ is a natural number, but we need to show this.

\end{description}

\begin{description}

\item[Theorem (not using Infinity):] For any natural number $n$, if $x
\in n+1$ and $y \in x$, then $x-\{y\} \in n$.  [an equivalent form is
``if $x \cup \{y\} \in n+1$ then $x-\{y\} \in n$'']

\item[Proof:] Let $A = \{n \in {\mathbb N}\mid (\forall xy.x\in n+1
\wedge y \in x \rightarrow x-\{y\} \in n)\}$, i.e., the set of all $n$
for which the theorem is true.  Our strategy is to show that the set
$A$ is inductive.  This is sufficient because an inductive set will
contain all natural numbers.

\begin{description}

\item[First Goal:]  $0 \in A$

\item[Proof of First Goal:]

The goal is equivalent to the assertion that if $x \in 0+1$ and $y \in
x$, then $x-\{y\} \in 0$.  We suppose that $x \in 0+1 = 1$ and $y \in
x$: this implies immediately that $x=\{y\}$, whence we can draw the
conclusion $x-\{y\} = \{y\}-\{y\} = \emptyset \in 0$, and $x-\{y\} \in
0$ is our first goal.

\item[Second Goal:]  $(\forall k \in A.k+1 \in A)$

\item[Proof of Second Goal:]

Let $k$ be an element of $A$.  Assume that $k \in A$: this means that
for any $x \in k+1$ and $y \in x$ we have $x-\{y\} \in k$ (this is the
inductive hypothesis).  Our goal is $k+1 \in A$: we need to show that
if $u\in (k+1)+1$ and $v \in u$ we have $u-\{v\} \in k+1$.  So we
assume $u \in (k+1)+1$ and $v \in u$: our new goal is $u - \{v\} \in
k+1$.  We know because $u \in (k+1)+1$ that there are $p \in k+1$ and
$q \not\in p$ such that $p \cup \{q\} = u$.  We consider two cases:
either $v=q$ or $v \neq q$.  If $v=q$ then $u-\{v\} = (p \cup
\{q\})-\{q\} = p$ (because $q \not\in p$) and we have $p \in k+1$ so we
have $u -\{v\} \in k+1$.  In the case where $v \neq q$, we have $v \in
p$, so $p -\{v\} \in k$ by the inductive hypothesis, and $u-\{v\} =
(p-\{v\}) \cup \{q\} \in k+1$ because $p-\{v\} \in k$ and $q \not\in
p-\{v\}$.  In either case we have the desired goal so we are done.

\end{description}

\item[$^*$ Theorem:]  If Infinity is false, then Axiom 4 is false.

\item[$^*$ Proof:]  If Infinity is false then $V$ is a finite set, so $V \in n$ for some natural number $n$.   We would like to say then that $\{V\}=n$, so $n+1 = \emptyset$ (there is no way to add a new element to $V$), so $\emptyset \in {\mathbb N}$, and clearly $\emptyset+1=\emptyset$, so $\{V\}+1=\emptyset+1=\emptyset$, but $\{V\} \neq \emptyset$, which  gives a counterexample to Axiom 4.  This argument is not so much incorrect as incomplete:  how do we know that $V \in n$ excludes $n$ having other elements?   The following common sense Lemma fixes this:  we believe that a finite set with $n$ elements will not have any proper subsets with $n$ elements$\ldots$

\item[Theorem (not using Infinity):] If $n$ is a natural number and
$x,y \in n$ and $x \subseteq y$ then $x=y$.

\item[Proof:] Let $A$ be the set of natural numbers for which the
theorem is true: $A = \{n \in {\mathbb N} \mid (\forall xy.x\in n
\wedge y \in n \wedge x \subseteq y \rightarrow x=y)\}$.  Our strategy
is to show that $A$ is inductive.

\begin{description}

\item[First Goal:]  $0 \in A$


\item[Proof of First Goal:]
What we need to prove is that if $x \in 0$ and $y \in 0$ and $x
\subseteq y$ then $x=y$.  Assume that $x \in 0$ and $y \in 0$ and $x
\subseteq y$.  It follows that $x=\emptyset$ and $y = \emptyset$, so
$x=y$.  This completes the proof.  Note that the hypothesis $x
\subseteq y$ did not need to be used.


\item[Second Goal:]  $(\forall k \in A.k+1 \in A)$

\item[Proof of Second Goal:]

Assume $k \in A$.  This means that for all $x,y \in k$, if $x
\subseteq y$ then $x=y$.  This is called the inductive hypothesis.

Our goal is $k+1 \in A$.  This means that for all $u,v \in k+1$, if $u
\subseteq v$ then $u=v$.  Suppose that $u \in k+1$, $v \in k+1$, and
$u \subseteq v$.  Our goal is now $u=v$.  Because $u \in k+1$, there
are $a$ and $b$ such that $u = a \cup \{b\}$, $a \in k$, and $b
\not\in a$.  Because $u \subseteq v$ we have $a = u-\{b\} \subseteq
v-\{b\}$.  $a \in k$ has been assumed and $v-\{b\} \in k$ by the
previous theorem ($b \in v$ because $u \subseteq v$), so $a = v-\{b\}$
by inductive hypothesis, so $u = a \cup \{b\} = (v-\{b\})\cup\{b\} =
v$.

\end{description}



\item[Theorem (not using Infinity):] If there is a natural number $n$
such that $V \in n$, we have $n = \{V\}$, $n+1 = \emptyset \in
{\mathbb N}$, and $n+1 = \emptyset +1$, though $n \neq \emptyset$, a
counterexample to Axiom 4.

\item[Proof:] If $V \in n \in {\mathbb N}$, then for any $x \in n$ we
clearly have $x \subseteq V$ whence $x=V$ by the previous theorem, so
$n = \{V\}$.  That $\{V\}+1 = \emptyset$ is obvious from the
definition of successor (we cannot add a new element to $V$).  It then
clearly follows that $\emptyset$ is a natural number.  $\emptyset +1 =
\emptyset$ is also obvious from the definition of successor, so we get
the counterexample to Axiom 4.

\item[Theorem (using Infinity):] $(\forall mn \in {\mathbb N}.m+1 =
n+1 \rightarrow m=n)$.

\item[Proof:]  Suppose that $m$ and $n$ are natural numbers and $m+1 = n+1$.

We prove that $m=n$ by showing that they have the same elements.

Let $a \in m$ be chosen arbitrarily:  our aim is to show $a \in n$.

Choose $x \not\in a$ (that there is such an $x$ follows from the Axiom
of Infinity, which tells us that the finite set $a$ (finite because it
belongs to a natural number) cannot be $V$).  $a \cup \{x\} \in m+1$.
It follows that $a \cup \{x\} \in n+1$, since by hypothesis $m+1=n+1$.
It then follows that $a = (a \cup \{x\})-\{x\} \in n$ by the first in
our sequence of theorems here.  This is the goal of the first part of
the proof.

In the second part of the proof, we choose $a \in n$ arbitrarily and
our goal is to show $a \in m$.  The proof is precisely the same as the
previous part with $m$ and $n$ interchanged.

So Axiom 4 of Peano arithmetic holds in our implementation.

\end{description}

A familiar construction of finite objects is the construction of
{\em ordered pairs\/}.

\begin{description}

\item[$^*$ordered pair:]  We define $\left<x,y\right>$ as $\{\{x\},\{x,y\}\}$.
Note that the pair is two types higher than its components $x$ and $y$.

\item[Theorem:]  For any $x,y,z,w$ (all of the same type), $\left<x,y\right>=\left<z,w\right>$ iff $x=z$ and $y=w$.

\item[Proof:]  This is left as an exercise.

\item[$^*$cartesian product:] For any sets $A$ and $B$, we define $A
\times B$, the {\em cartesian product\/} of $A$ and $B$, as
$\{\left<a,b\right> \mid a \in A \wedge b \in B\}$.  Notice that this
is an example of generalized set builder notation, and could also be
written as $\{c \mid (\exists ab.c = \left<a,b\right> \wedge a \in A
\wedge b \in B)\}$ (giving a promised example of the
generalized set builder notation definition).

\end{description}

The definitions above are starred because we will in fact not use
these common definitions.  These definitions (due to Kuratowski) are
usable in typed set theory and have in fact been used, but they have a
practical disadvantage: the pair $\left<x,y\right>$ is two types
higher than its components $x$ and $y$.

We will instead introduce a new primitive notion and axiom.

\begin{description}

\item[ordered pair:] For any objects $x^{\bf n}$ and $y^{\bf n}$, we
introduce primitive notation $\left<x^{\bf n},y^{\bf n}\right>^{\bf
n}$ for the ordered pair of $x$ and $y$ and primitive notation
$\pi_1(x^{\bf n})^{\bf n}$ and $\pi_2(x^{\bf n})^{\bf n}$ for the
first and second projections of an object $x^{\bf n}$ considered as an
ordered pair.  As the notation suggests, the type of the pair is the
same as the types of its components $x$ and $y$ (which we call its
{\em projections\/}).  In accordance with our usual practice, we will
omit the type indices most of the time, allowing them to be deduced
from the context.

Notice that the scope of the Axiom of Comprehension is expanded to cover
statements including these notations.

\item[Axiom of the Ordered Pair:]
For any $x,y$, $\pi_1(\left<x,y\right>)=x$ and $\pi_2(\left<x,y\right>)=y$.
 For any $x$,
$x=\left<\pi_1(x),\pi_2(x)\right>$.

\item[Corollary:] For any $x,y,z,w$,
$\left<x,y\right>=\left<z,w\right>\leftrightarrow x=z\wedge y=w$.  The
corollary is usually taken to be the defining property of the ordered
pair; our axiom has the additional consequence that all objects are
ordered pairs.

\item[cartesian product:] For any sets $A$ and $B$, we define $A
\times B$, the {\em cartesian product\/} of $A$ and $B$, as
$\{\left<a,b\right> \mid a \in A \wedge b \in B\}$.  Notice that this
is an example of generalized set builder notation, and could also be
written as $\{c \mid (\exists ab.c = \left<a,b\right> \wedge a \in A
\wedge b \in B)\}$ (giving a promised example of the
generalized set builder notation definition).

We define $A^2$ as $A \times A$ and more generally define $A^{n+1}$ as
$A \times A^n$ (this definition of ``cartesian powers'' would not work
if we were using the Kuratowski pair, for reasons of type).  Notice
that these exponents can be distinguished from type superscripts (when they are
used) because we do not use boldface.

\end{description}

A crucial advantage of a type-level pair in practice is that it allows a nice definition of $n$-tuples for every $n$:

\begin{description}

\item[tuples:] $\left<x_1,x_2,\ldots,x_n\right> =
\left<x_1,\left<x_2,\ldots,x_n\right>\right>$ for $n>2$.

\end{description}

This would not type correctly if the Kuratowski pair were used.  We illustrate the problem.
If we want to represent $\left<x,y,z\right>$ as $\left<x,\left<y,z\right>\right>$, and assign type $n$ to $z$,
then $y$ will also be assigned type $n$, $\left<y,z\right>$ will be assigned type $n+2$, and $x$ will be assigned type
$n+2$!  This can be repaired by using $\left<\iota^{\bf 2}`x,\left<y,z\right>\right>$ instead.  The type of the triple thus implemented will be $n+4$.  Now imagine what this approach would give as the definition of a quintuple of objects of the same type:  progressively longer tuples defined in this way will be of progressively higher type.  We will briefly describe in a later section how the
Kuratowski pair can be used to define $n$-tuples of arbitrary length of the same type independent of $n$.

We show that the Axiom of Infinity follows from the Axiom of Ordered
Pairs (so we strictly speaking do not need the Axiom of Infinity if we
assume the Axiom of Ordered Pairs).

\begin{description}

\item[Theorem:]  The Axiom of Ordered Pairs implies the Axiom of Infinity.

\item[Proof:] We argue that if $A \in n \in {\mathbb N}$ then
$A\times\{0\} \in n$.  $\emptyset$ is the only element of 0 and
$\emptyset\times \{0\}=0 \in {\mathbb N}$.  Suppose that $A
\times\{0\} \in n$ for all $A \in n$.  Any element of $n+1$ is of the
form $A \cup \{x\}$ where $A \in n$ and $x \not\in A$.  $(A \cup
\{x\}) \times \{0\} = (A \times\{0\}) \cup \{\left<x,0\right>\} \in
n+1$.  The claim follows by induction.  Now suppose $V \in N \in
{\mathbb N}$.  It follows that $V \times \{0\} \in N$.  But certainly
$V \times \{0\} \subseteq V$ so by a theorem about finite sets proved
above, $V = V\times\{0\}$, which is absurd.

\end{description}


\subsection{Digression:  The Quine Ordered Pair}

We develop a more complex definition of an ordered pair
$\left<x,y\right>$, due to Willard v. O. Quine, which is of the same
type as its components $x$ and $y$ and satisfies the Axiom of
Ordered Pairs above, but only works if strong extensionality is
assumed.

The definition of the Quine pair is quite elaborate.  The basic idea
is that the Quine pair $\left<A,B\right>$ is a kind of tagged union of
$A$ and $B$ (it is only defined on sets of sets).  Suppose that we can
associate with each element $a$ of $A$ an object ${\tt first}(a)$ from
which $a$ can be recovered, and with each element $b$ of $B$ an object
${\tt second}(b)$ from which $b$ can be recovered, and we can be sure
that ${\tt first}(a)$ and ${\tt second}(b)$ will be distinct from each
other for any $a \in A$ and $b \in B$.  The idea is that
$\left<A,B\right>$ will be defined as $$\{{\tt first}(a) \mid a \in
A\} \cup \{{\tt second}(b) \mid b \in B\}.$$ For this to work we need
the following things to be true for all objects $x$ and $y$ of the
type to which elements of $A$ and $B$ belong:

\begin{enumerate}

\item  For any $x,y$, ${\tt first}(x) = {\tt first}(y) \rightarrow x=y$

\item  For any $x,y$, ${\tt second}(x) = {\tt second}(y) \rightarrow x=y$

\item  For any $x,y$, ${\tt first}(x) \neq {\tt second}(y)$

\end{enumerate}

If these conditions hold, then we can recover $A$ and $B$ from
$\left<A,B\right>$.  An element $x$ of $\left<A,B\right>$ will be of
the form ${\tt first}(a)$ for some $a \in A$ or of the form ${\tt
second}(b)$ for some $b \in B$.  It will be only one of these things,
because no ${\tt first}(x)$ is equal to any ${\tt second}(y)$.
Moreover, if $x={\tt first}(a)$, there is only one $a$ for which this
is true, and if $x={\tt second}(b)$ there is only one $b$ for which
this is true.  So $A = \{a \mid {\tt first}(a) \in \left<A,B\right>\}$
and $B = \{b \mid {\tt second}(b) \in \left<A,B\right>\}$.

Thus if $\left<A,B\right> = \left<C,D\right>$ we have $A = \{a \mid
{\tt first}(a) \in \left<A,B\right>\} = \{a \mid {\tt first}(a) \in
\left<C,D\right>\} = C$ and similarly $B=D$.

The details of the definitions of the needed {\tt first} and {\tt
second} operators follow.  They will actually be called $\sigma_1$ and
$\sigma_2$.

\begin{description}

\item[Definition:] For each $n \in {\mathbb N}$ we define $\sigma_0(n)$
as $n+1$ and for each $x \not\in {\mathbb N}$ we define $\sigma_0(x)$ as
$x$.  Note that $\sigma_0(x)$ is of the same type as $x$.

\item[Observation:] For any $x,y$, if $\sigma_0(x)=\sigma_0(y)$ then
$x=y$.  If $x$ and $y$ are not natural numbers then this is obvious.
If $x$ is a natural number and $y$ is not, then $\sigma_0(x)$ is a
natural number and $\sigma_0(y)$ is not, so the hypothesis cannot be
true.  If $x$ and $y$ are natural numbers the statement to be proved
is true by axiom 4.

\item[Definition:] We define $\sigma_1(x)$ as $\{\sigma_0(y) \mid y \in
x\}$.  We define $\sigma_2(x)$ as $\sigma_1(x) \cup \{0\}$.  We define
$\sigma_3(x)$ as $\{y \mid \sigma_0(y) \in x\}$.  Note that all of these
operations preserve type.

\item[Observation:] $\sigma_3(\sigma_1(x)) = x$, so if $\sigma_1(x) =
\sigma_1(y)$ we have
$x=\sigma_3(\sigma_1(x))=\sigma_3(\sigma_1(y))=y$;
$\sigma_3(\sigma_2(x)) = x$, so similarly if $\sigma_2(x)=\sigma_2(y)$
we have $x=y$; $\sigma_1(x) \neq \sigma_2(y)$, because $0 \not\in
\sigma_1(x)$ and $0 \in \sigma_2(y)$.  This shows that the $\sigma_1$
and $\sigma_2$ operations have the correct properties to play the
roles of {\tt first} and {\tt second} in the abstract discussion
above.

\item[Definition:]  We define $\sigma_1``(x)$ as $\{\sigma_1(y)\mid y \in x\}$, $\sigma_2``(x)$ as $\{\sigma_2(y)\mid y \in x\}$ and  $\sigma_3``(x)$ as $\{\sigma_3(y)\mid y \in x\}$

\item[Definition:] We define $\left<x,y\right>$ as $\sigma_1``(x) \cup
\sigma_2``(y)$.  Note that the pair is of the same type as its
components.

\item[Theorem:] For each set $x$ there are unique sets $\pi_1(x)$ and
$\pi_2(x)$ such that $\left<\pi_1(x),\pi_2(x)\right>=x$.  An immediate
corollary is that for any $x,y,z,w$ (all of the same type),
$\left<x,y\right>=\left<z,w\right>$ iff $x=z$ and $y=w$.

\item[Proof:] $\pi_1(x) = \sigma_3``(\{y \in x \mid 0 \not\in y\})$;
$\pi_2(x) = \sigma_3``(\{y \in x \mid 0 \in y\})$

\end{description}

The Quine pair is defined only at type 4 and above; this is not a
problem for us because we can do all our mathematical work in as high
a type as we need to: notice that the natural numbers we have defined
are present in each type above type 2; all mathematical constructions
we present will be possible to carry out in any sufficiently high
type.

In the theory with weak extensionality, the Quine pair is defined only
on sets of sets (elements of ${\cal P}^{\bf 2}(V)$) in types 4 and above,
but it does satisfy the Axiom of Ordered Pairs on this restricted
domain.  We could in principle use the Quine pair instead of
introducing a primitive pair, if we were willing to restrict
relations and functions to domains consisting of sets of sets.  This
isn't as bad as it seems because all objects of mathematical interest
are actually sets of sets.

 We will not do this (our primitive pair
acts on all objects), but we can use the Quine pair on sets of sets to
justify our introduction of the primitive pair: if we cut down our
universe to the sets of sets in types 4 and above, and use the
relation $x \in' y$ defined as $x\in y \wedge y \in {\cal P}^{\bf 3}(V)$ as
our new membership relation (allowing only sets of sets of sets to be
sets in the restricted world) it is straightforward to verify that our
axioms will hold with the new membership relation and the Quine pair
in the old world (with its associated projection functions) will still
be a pair and projections in the new world satisfying the Axiom of
Ordered Pairs.  We can do even better.  If we replace the natural
numbers $n$ in the definition of the Quine pair in the old world with
$n \cap {\cal P}^{\bf 3}(V)$, the pair in the new world will turn out to
coincide with the new world's Quine pair on sets of sets (because the
objects $n \cap {\cal P}^{\bf 3}(V)$ are the natural numbers in the new
world), and further all pairs of sets will be sets.

We generalize the idea of the previous paragraph.  Suppose we have an expression $W_{\bf n}$ with a type parameter,
satisfying ${\cal P}(W_{\bf n}) \subseteq W_{\bf n+1}$.   Notice that if $i$ is the type of elements of $W_{\bf 0}$ then $n+i$ will be the type of
elements of $W_{\bf n}$.  Now define $x \in_W y$ as $x \in y \wedge x \in W_{\bf n} \wedge y \in {\cal P}(W_{\bf n})$ for $x$ of type $n+i$ and
$y$ of type $n+i+1$.  Define ${\tt set}_W(x)$ as $x \in {\cal P}(W_{\bf n})$ for $x$ of type $n+i+1$.  Think of $W_{\bf n}$ as type $n$ of a ``$W$-world" embedded in the world of our type theory (which we will refer to as the ``real world" when we need to contrast the worlds).  If we have ${\tt set}_W(x)$, we have $z \in_W x \leftrightarrow z \in x$ for any $z$, so if we have  ${\tt set}_W(x)$ and ${\tt set}_W(y)$ 
and for every $z$, $z \in_W x$ iff $z \in_W y$, we also have $z \in x$ iff $z \in y$, and of course $x$ and $y$ are sets (since they belong to a power set) so they are equal.  We have just shown that extensionality holds in the $W$-world.  Notice that an atom in the sense of the $W$-world is either an atom in the real world or a set in the real world which has an element which is not in the $W$-world.  Now let $P[x]$ be any sentence of our language.  Observe that for any $x \in W_{\bf n}$ ($n$ being appropriate to the type of $x$) $P[x] \leftrightarrow x \in \{x \in W_{\bf n} \mid P[x]\}$, so $P[x] \leftrightarrow x \in_W \{x \in W_{\bf n} \mid P[x]\}$ (because $x \in W_{\bf n}$ and $\{x \in W_{\bf n} \mid P[x]\} \in {\cal P}(W_{\bf n})$), so Comprehension holds in the $W$-world.  Now we note that if we define $W_{\bf n}$ as ${\cal P}^{\bf 2}(V^{\bf n+3})$, we do have the relation
${\cal P}(W_{\bf n}) \subseteq W_{\bf n+1}$, as clearly $${\cal P}({\cal P}^{\bf 2}(V^{\bf n+3})) = {\cal P}^{\bf 3}(V^{\bf n+3}) \subseteq {\cal P}^{\bf 2}(V^{\bf n+4}).$$  Note further that if $x,y \in W_{\bf n}$, we also have $\left<x,y\right>$ (the Quine pair of $x$ and $y$) belonging to $W_{\bf n}$, and further $\pi_1(x)$ and $\pi_2(x)$ (the Quine projections of $x$) belong to $W_{\bf n}$, so the Axiom of Ordered Pairs is true in the $W$-world (where the pair is read as the Quine pair of the real world).   The definition of $W_{\bf n}$ is driven by the fact that we need $x,y \in W_{\bf n}$ to be sets of sets (thus the double power set) and we need the elements of their elements to be of a type which contains natural numbers (thus the double power set of $V^{\bf n+3}$, the lowest type universal set which contains natural numbers, which are of types $n+2$).   A further trick
will cause the pair inherited from the larger world to actually be the Quine pair in the $W$-world when its projections are sets of sets in the $W$-world:  in the definition of the Quine pair in the real world, replace natural numbers
$n^{\bf k+2} \in {\mathbb N}^{\bf k+3}$ with their restrictions  $n^{\bf k+2} \cap {\cal P}^{\bf 3}(V^{\bf k-1})$ whenever $k>0$;  this has the effect of replacing the $n$ of the larger world with the $n$ of the $W$-world, so that the modified Quine pair in the real world is exactly the Quine pair in the $W$-world when its projections are sets of sets in the $W$-world.  So we can justify the use of the Axiom of Ordered Pairs, if we have the Axiom of Infinity in the real world, by stipulating that we restrict our attention to this $W$-world henceforth, and we can even preserve the fact that the pair is the Quine pair, though only for sets of sets.  

What we have just given is a sketch of what is called a {\em relative
consistency proof\/}.  Given a model of our type theory with the Axiom
of Infinity, we show how to get a model of our type theory with the
Axiom of Ordered Pairs (but not quite the same model).

Something important is going on here: we are forcibly reminded here
that we are {\em implementing\/} already familiar mathematical
concepts, not revealing what they ``really are''.  Each implementation
has advantages and disadvantages.  Here, the Kuratowski pair has the
advantage of simplicity and independence of use of the Axiom of
Infinity, while the Quine pair (or the primitive pair we have to
introduce because we allow non-sets) has the technical advantage,
which will be seen later to be overwhelming, that it is type level.
Neither is the {\em true\/} ordered pair; the ordered pair notion
prior to implementation is not any particular sort of object: its
essence is perhaps expressed in the theorem that equal ordered pairs
have equal components.  The internal details of the implementation
will not matter much in the sequel: what will do the mathematical work
is the fact that the pair exactly determines its two components.

\newpage

\subsection{Exercises}

\begin{enumerate}

\item Write a definition of the natural number 2 in the form
$\{x \mid \phi[x]\}$ where $\phi$ is a formula containing only
variables, logical symbols, equality and membership.  Hint: the
formula $\phi[x]$ needs to express the idea that $x$ has exactly two
elements in completely logical terms.  How would you say that $x$
has at least two elements?  How would you say that $x$ has at most two elements?

A definition of 1 in this style is $$\{x \mid (\exists y.y \in x)
\wedge (\forall uv.u \in x \wedge v \in x \rightarrow u=v)\}.$$

Another definition of 1 is $$\{x \mid (\exists y.y \in x \wedge (\forall z.z\in x \rightarrow z=y))\}.$$

Notice the different structure of the scopes of the quantifiers in the
two definitions.

\item The usual definition of the ordered pair used in untyped set
theory (due to Kuratowski) is $$\left<x,y\right> =_{\tt def}
\{\{x\},\{x,y\}\}.$$ We will not use this as our definition of ordered
pair because it has the inconvenient feature that the pair is two
types higher than its projections.  What we {\em can\/} do (as an
exercise in thinking about sets) is prove the following basic Theorem
about this pair definition:

$$\left<x,y\right>=\left<z,w\right>\rightarrow x=z \wedge y=w$$

This is your exercise.  There are various ways to approach it: one
often finds it necessary to reason by cases.  if you have seen a proof
of this, don't go look it up: write your own.

\item
Prove the theorem $(\forall xyz.\{x,z\} = \{y,z\} \rightarrow x=y)$
from the axioms of type theory, the definition of unordered pairs
$\{u,v\}$, logic and the properties of equality.  Remember that
distinct letters do not necessarily represent distinct objects.

This could be used to give a very efficient solution to the previous
exercise.


\item  
Prove that the set $\mathbb N^{\bf k+3}$ (the set of natural numbers
in type {\bf k+3}) is inductive.  You don't need to specify types on
every variable (or constant) every time it occurs, but you might want
to state the type of each object mentioned in the proof the first time
it appears.

This proof is among other things an exercise in the careful reading of
definitions.



\item
Prove the following statement using the Peano axioms in the form
stated in the current section: $(\forall n \in {\mathbb N}.n = 0 \vee
(\exists m.m+1=n))$.  You will need to use mathematical induction (in
the set based form introduced above), but there is something very odd
(indeed rather funny) about this inductive proof.

Why is the object $m$ unique in case it exists? (This is a throwaway
corollary of the main theorem: it does not require an additional
induction argument).

\item  
You are given that $n>0$ is a natural number and $a,b$ are not natural
numbers.

Compute the Quine pairs $\left<x,y\right>$ and $\left<y,x\right>$ where $x =
\{\{\emptyset,3\},\{2\},\{0,b\}\}$ and $y = \{\{1,2\},\{n,a\}\}$ 

Given that $\left<u,v\right> =\{\{0,2,4\},\{a,b,2\},\{0\},\{1\},\{a,n\}\}$, what are the sets $u$
and $v$?

\item
Prove that the following are pair definitions (that is, show that they
satisfy the defining theorem of ordered pairs).

\begin{description}

\item[The Wiener pair:] This is the first ordered pair definition in
terms of set theory ever given.
$$\left<x,y\right> = _{\tt def} \{\{\{x\},\emptyset\},\{\{y\}\}\}.$$

Hint:  think about how many elements the sets appearing as components of this definition have.

What is the type of the Wiener pair relative to the types of its projections?

\item[A pair that raises type by one:]  This is due to the author.
Define $[x,a,b]$ as $\{\{x',a,b\}\mid x'\in x\}$.  Define $\left<x,y\right>$ as
$[x,0,1] \cup [x,2,3] \cup [y,4,5] \cup [y,6,7]$, where 0,1,2,3,4,5,6,7 can be any eight distinct objects.   This only serves to construct pairs of sets, like the Quine pair.

\end{description}

\item
We define an {\em initial segment of the natural numbers\/} as a set
$S$ of natural numbers which has the property that for all natural
numbers $m$, if $m+1 \in S$ then $m \in S$.

Does an initial segment of the natural numbers need to contain all
natural numbers?  Explain why, or why not (with an example).

Prove that any nonempty initial segment of the natural numbers includes 0.

How do we prove {\em anything\/} about natural numbers?

\item
Find sets $A$ and $B$ such that $A+1=B+1$ but $A \neq B$.  I found an
example that isn't too hard to describe where $A+1 = B+1 = 3$ (or any
large enough natural number; nothing special about 3).  There are
other classes of examples.  This shows that Axiom 4 is true of natural
numbers but not of sets in general.

Can you describe a set $A$ such that $A+1=A$?

\item Verify the equation $$(A \cup
\{x\}) \times \{0\} = (A \times\{0\}) \cup \{\left<x,0\right>\}$$ found in the proof that the Axiom of Ordered Pairs implies the Axiom of Infinity.  This is an exercise in reading definitions carefully.

\end{enumerate}

\newpage

We give some solutions.

\begin{description}

\item[2.]  We repeat the definition $$\left<x,y\right> =_{\tt def}
\{\{x\},\{x,y\}\}$$ of the Kuratowski pair.  Our goal is to prove
$$(\forall xyzw.\left<x,y\right> = \left<z,w\right> \rightarrow x=z
\wedge y=w).$$

We let $x$,$y$,$z$,$w$ be arbitrarily chosen objects.  Assume that
$\left<x,y\right> = \left<z,w\right>$: our new goal is $x=z \wedge
y=w$.  Unpacking definitions tells us that we have assumed
$\{\{x\},\{x,y\}\}=\{\{z\},\{z,w\}\}$.

We have two things to prove (since our goal is a conjunction).  Note
that these are not separate cases: the result proved as the first
subgoal can (and will) be used in the proof of the second.

\begin{description}

\item[Goal 1:] $x=z$
\item[Proof of Goal 1:]
Because $\{\{x\},\{x,y\}\}=\{\{z\},\{z,w\}\}$, we have either
$\{x\}=\{z\}$ or $\{x\}=\{z,w\}$.  This allows us to set up a proof by
cases.

\begin{description}

\item[Case 1a:] We assume $\{x\}=\{z\}$.  Certainly $x \in \{x\}$;
thus by substitution $x \in \{z\}$, thus by definition of $\{z\}$ (and
by comprehension) we have $x=z$.

\item[Case 1b:] We assume $\{x\}=\{z,w\}$.  Certainly $z \in \{z,w\}$
(by definition of $\{z,w\}$ and comprehension).  Thus $z \in\{x\}$, by
substitution of equals for equals.  Thus $z=x$, so $x=z$.

\item[Conclusion:]  In both cases $x=z$ is proved, so Goal 1 is proved.

\end{description}


\item[Goal 2:] $y=w$

\item[Proof of Goal 2:] Note that we can use the result $x=z$ proved
above in this subproof.

Because $\{\{x\},\{x,y\}\}=\{\{z\},\{z,w\}\}$ we have either $\{x\} =
\{z,w\}$ or $\{x,y\} = \{z,w\}$.  This allows us to set up an argument
by cases.

\begin{description}

\item[Case 2a:] Assume $\{x\} = \{z,w\}$.  Since $z \in \{z,w\}$ and
$w \in \{z,w\}$, we have $z\in \{x\}$ and $w \in \{x\}$ by
substitution, whence we have $x=z=w$.  This implies that $\{z\} =
\{z,w\}$, so $\{\{z\},\{z,w\}\} = \{\{z\}\}$.  Now we have
$\{\{x\},\{x,y\}\}=\{\{z\}\}$ by substitution into our original
assumption, whence $\{x,y\} = \{z\}$, whence $x=y=z$ (the proofs of
these last two statements are exactly parallel to things already
proved), so $y=w$ as desired, since we also have $x=z=w$.

\item[Case 2b:] Assume $\{x,y\} = \{z,w\}$.  Suppose $y \neq w$ for
the sake of a contradiction.  Since $y \in \{x,y\}$, we have $y \in
\{z,w\}$, whence $y=z$ or $y=w$  Since $y \neq w$, we have $y=z$.  Since
$w \in \{x,y\}$ we have $w =x$ or $w=y$.  Since $w \neq y$, we have $w=x$.
Now we have $y=z=x=w$, so $y=w$, giving the desired contradiction, and completing the proof that $y=w$.

\item[Conclusion:] Since $y=w$ can be deduced in both cases, it can be
deduced from our original assumption, completing the proof of Goal 2
and of the entire theorem.

\end{description}

\end{description}

\newpage

\item[5.]  Our goal is $(\forall n \in {\mathbb N}.n = 0 \vee (\exists
m.m+1=n))$.

Define $A$ as the set $\{n \in {\mathbb N} \mid n=0 \vee (\exists m \in{\mathbb N}.m+1=n)\}$.

Our goal is to prove that $A$ is inductive, from which it will follow
that ${\mathbb N} \subseteq A$, from which the theorem follows.

\begin{description}

\item[Basis Step:]  $0 \in A$

\item[Proof of Basis Step:] $0 \in A \leftrightarrow (0 = 0 \vee (\exists
m\in {\mathbb N}.m+1=0))$, and $0=0$ is obviously true.

\item[Induction Step:] $(\forall k \in {\mathbb N}.k \in A \rightarrow
k+1 \in A)\}$.

\item[Proof of Induction Step:] Let $k$ be an arbitrarily chosen
natural number.  Assume $k \in A$.  Our goal is to prove $k+1 \in A$,
that is, $k+1=0 \vee (\exists m \in {\mathbb N}.m+1=k+1)$.  We prove
this by observing that $k \in {\mathbb N}$ and $k+1=k+1$, which
witnesses $(\exists m \in {\mathbb N}.m+1=k+1)$.  Notice that the
inductive hypothesis $k \in A$ was never used at all: there is no need
to expand it.

\end{description}

\newpage

\item[8.]  We define an {\em initial segment of the natural numbers\/} as a set
$S$ of natural numbers which has the property that for all natural
numbers $m$, if $m+1 \in S$ then $m \in S$.

Does an initial segment of the natural numbers need to contain all
natural numbers?  Explain why, or why not (with an example).

{\bf Solution:} No.  The empty set is an initial segment, since the
hypothesis $m +1 \in S$ is false for every $m$ if $S = \emptyset$,
making $m+1 \in S \rightarrow m \in S$ vacuously true.  A nonempty
initial segment not equal to $\mathbb N$ is for example $\{0,1\}$: the
implication can be checked for $m=0$ and is vacuously true for all
other values of $m$.

Prove that any nonempty initial segment of the natural numbers includes 0.

{\bf Solution:} Let $S$ be a nonempty initial segment of the natural
numbers.  Our goal is to show $0 \in S$.  Since $S$ is nonempty, we
can find $m \in S$.  If we could show $(\forall n \in {\mathbb N}.n
\in S \rightarrow 0 \in S)$, we would have $m \in S \rightarrow 0 \in
S$ and $0 \in S$ by modus ponens.

We prove the lemma $(\forall n \in {\mathbb N}.n \in S \rightarrow 0
\in S)$ by mathematical induction.  Let $A = \{n \in {\mathbb N} \mid
n \in S \rightarrow 0 \in S\}$.  We show that $A$ is inductive.

\begin{description}

\item [Basis Step:]  $0 \in S \rightarrow 0 \in S$ is the goal.  This is obvious.

\item[Induction Step:] Let $k$ be an arbitrarily chosen natural
number.  Suppose $k \in A$.  Our goal is $k+1 \in A$.  $k \in A$ means
$k \in A \rightarrow 0 \in S$.  We have $k+1 \in S \rightarrow k \in
S$ because $S$ is an initial segment.  From these two implications
$k+1 \in S \rightarrow 0 \in S$ follows, completing the proof of the
induction step and the lemma.

\end{description}


\newpage

\end{description}

\section{Relations and Functions}

If $A$ and $B$ are sets, we define a {\em relation from $A$ to $B$\/}
as a subset of $A \times B$.  A {\em relation\/} in general is simply
a set of ordered pairs.

If $R$ is a relation from $A$ to $B$, we define $x \,R\,y$ as
$\left<x,y\right>\in R$.  This notation should be viewed with care.
Note here that $x$ and $y$ must be of the same type, while $R$ is one
type higher than $x$ or $y$ (that would be three types higher if we
used the Kuratowski pair).  In the superficially similar notation $x
\in y$, $y$ is one type higher than $x$ and $\in$ does not denote a
set at all: do not confuse logical relations with set relations.  In
some cases they can be conflated: the notation $x \subseteq y$ can be
used to motivate a definition of $\subseteq$ as a set relation
($[\subseteq]=\{\left<x,y\right>\mid x \subseteq y\}$), though we do
not originally understand $x \subseteq y$ as saying anything about a
set of ordered pairs.

If $R$ is a relation, we define ${\tt dom}(R)$, the {\em domain of
$R$\/}, as $\{x \mid (\exists y.x\,R\,y)\}$.  We define $R^{-1}$, the
{\em inverse of $R$\/}, as $\{\left<x,y\right>\mid y\,R\,x\}$.  We
define ${\tt rng}(R)$, the {\em range of $R$\/}, as ${\tt
dom}(R^{-1})$.  We define ${\tt fld}(R)$, the {\em field of $R$\/}, as
the union of ${\tt dom}(R)$ and ${\tt rng}(R)$.  If $R$ is a relation
from $A$ to $B$ and $S$ is a relation from $B$ to $C$, we define
$R|S$, the {\em relative product of $R$ and $S$\/} as
$$\{\left<x,z\right>\mid(\exists y.x\,R\,y \wedge y\,S\,z)\}.$$

The symbol $[=]$ is used to denote the equality relation
$\{\left<x,x\right>\mid x \in V\}$.  Similarly $[\subseteq]$ can be
used as a name for the subset relation (as we did above), and so
forth: the brackets convert a grammatical ``transitive verb'' to a
noun.\footnote{The transformation of relation symbols into terms using brackets is an invention of ours and not likely to be found in other books.}

We define special characteristics of relations.

\begin{description}

\item [reflexive:] $R$ is {\em reflexive\/} iff $x\,R\,x$ for all $x \in {\tt fld}(R)$.

\item[symmetric:]  $R$ is {\em symmetric\/} iff for all $x$ and $y$, $x\,R\,y \leftrightarrow y\,R\,x$.

\item[antisymmetric:]  $R$ is {\em antisymmetric\/} iff for all $x,y$ if $x\,R\,y$ and $y\,R\,x$ then $x=y$.

\item[asymmetric:] $R$ is {\em asymmetric\/} iff for all $x,y$ if $x\,R\,y$
then $\neg y \,R\,x$.  Note that this immediately implies $\neg x\,R\,x$.

\item[transitive:]  $R$ is {\em transitive\/} iff for all $x,y,z$ if $x\,R\,y$ and $y\,R\,z$ then $x\,R\,z$.

\item[equivalence relation:]  A relation is an {\em equivalence relation\/} iff it is reflexive, symmetric, and transitive.

\item[partial order:] A relation is a {\em partial order\/} iff it is
reflexive, antisymmetric, and transitive.

\item[strict partial order:] A relation is a {\em strict partial order\/} iff
it is asymmetric and transitive.  Given a partial order $R$, $R-[=]$
will be a strict partial order.  From a strict partial order $R-[=]$,
the partial order $R$ can be recovered if it has  no ``isolated
points'' (elements of its field related only to themselves).

\item[linear order:] A partial order $R$ is a {\em linear order\/} iff
for any $x,y \in {\tt fld}(R)$, either $x \,R\,y$ or $y\,R\,x$.  Note
that a linear order is precisely determined by the corresponding
strict partial order if its domain has two or more elements.

\item[strict linear order:] A strict partial order $R$ is a {\em strict
linear order\/} iff for any $x,y \in {\tt fld}(R)$, one has $x\,R\,y$,
$y\,R\,x$ or $x=y$.  If $R$ is a linear order, $R-[=]$ is a strict
linear order.

\item[image:]  For any set $A\subseteq {\tt fld}(R)$, $R``A = \{b \mid (\exists a \in A.a\,R\,b)\}$.

\item[extensional:] A relation $R$ is said to be {\em extensional\/}
iff for any $x,y \in {\tt fld}(R)$,
$R^{-1}``(\{x\})=R^{-1}``(\{y\})\rightarrow x=y$: elements of the
field of $R$ with the same preimage under $R$ are equal.  An
extensional relation supports a representation of some of the subsets
of its field by the elements of its field.

\item[well-founded:] A relation $R$ is {\em well-founded\/} iff for each
nonempty subset $A$ of ${\tt fld}(R)$ there is $a\in A$ such that for
no $b\in A$ do we have $b \,R\,a$ (we call this a minimal element of
$A$ with respect to $R$, though note that $R$ is not necessarily an
order relation).

\item[well-ordering:] A linear order $R$ is a {\em well-ordering\/} iff the
corresponding strict partial order $R-[=]$ is well-founded.

\item[strict well-ordering:] A strict linear order $R$ is a {\em strict
well-ordering\/} iff it is well-founded.

\item[end extension:] A relation $S$ {\em end extends\/} a relation $R$
iff $R \subseteq S$ and for any $x \in {\tt fld}(R)$, $R^{-1}``\{x\} =
S^{-1}``\{x\}$.  (This is a nonstandard adaptation of a piece of
terminology from model theory).

\item[function:] $f$ is a {\em function from $A$ to $B$\/} (written
$f:A\rightarrow B$) iff $f$ is a relation from $A$ to $B$, ${\tt dom}(f)=A$,  and for all
$x,y,z$, if $x\,f\,y$ and $x\,f\,z$ then $y=z$.  For each $x\in {\tt
dom}(f)$, we define $f(x)$ as the unique $y$ such that $x\,f\,y$ (this
exists because $x$ is in the domain and is unique because $f$ is a
function).  The notation $f[A]$ is common for the image $f``A$.

\item[warning about function notation:] Notations like ${\cal P}(x)$
for the power set of $x$ should not be misconstrued as examples of the
function value notation $f(x)$.  There is no function $\cal P$ because
${\cal P}(x)$ is one type higher than $x$.  We have considered using
the notation $F`x$ (this was Russell's original notation for function
values) for defined operators in general and restricting the notation
$f(x)$ to the case where $f$ is actually a set function.  If we did
this we would exclude (for example) the notation ${\cal P}(x)$ in
favor of ${\cal P}`x$ (or ${\cal P}`(t)$ for complex terms $t$ that
require parentheses).  If we used the Russell notation in this way we
would also write $\bigcup`x$, $\bigcap`x$ because these operations
also shift type.  We would then prefer the use of $f[A]$ to the use of
$f``A$ for images under functions.  But we have not adopted such a
convention here.

\item[injection:] A function $f$ is an {\em injection\/} (or {\em one-to-one\/}) iff $f^{-1}$ is a
function.

\item[surjection:] A function $f$ is a {\em surjection from $A$ to
$B$\/} or a {\em function from $A$ onto $B$\/} iff it is a function
from $A$ to $B$ and $f``A=B$.

\item[bijection:] A function $f$ is a {\em bijection from $A$ to $B$\/} iff it
is an injection and also a surjection from $A$ to $B$.

\item[composition and restriction:] If $f$ is a function and $A$ is a
set (usually a subset of ${\tt dom}(f)$), define $f\lceil A$ as $f
\cap (A \times V)$ (the {\em restriction of $f$ to the set $A$\/}).
If $f$ and $g$ are functions and ${\tt rng}(g) \subseteq {\tt
dom}(f)$, define $f \circ g$ as $g|f$.  This is called the {\em
composition\/} of $f$ and $g$.  We may now and then write compositions as relative products, when the unnaturalness of the order of the composition operation is a problem.

\item[identity function:] Note that [=] is a function.  We call it the
{\em identity function\/}, and we call [=]$\lceil A$ the {\em identity
function on $A$\/}, where $A$ is any set.

\item[abstraction:] If $T[x]$ is a term (usually involving $x$) define
$(x:A \mapsto T[x])$ or $(\lambda x:A.T[x])$ as
$\{\left<x,T[x]\right>\mid x \in A\}$.  The explicit mention of
the set $A$ may be omitted when it is $V$ or when it is understood
from the form of the term $T[x]$.

\end{description}

\subsection{Exercises}

\begin{enumerate}
\item  I give alternative definitions of injection and surjection from $A$ to
$B$.

A function $f$ is an injection from $A$ to $B$ iff it is a function
from $A$ to $B$ and for all $x,y \in A$, $f(x)=f(y) \rightarrow x=y$.

A function $f$ is a surjection from $A$ to $B$ iff it is a
function from $A$ to $B$ and for all $y \in B$, there exists $x\in A$
such that $f(x)=y$.

Verify that each of these definitions is equivalent to the original one.

\item

Prove that if $f$ is an injection from $A$ to $B$ and $g$ is
an injection from $B$ to $C$, then $g \circ f$ is an
injection from $A$ to $C$.  ($g \circ f$ may be supposed
defined by the equation $(g\circ f)(x)=g(f(x))$) 

Prove that if $f$ is an surjection from $A$ to $B$ and $g$ is
a surjection from $B$ to $C$, then $g \circ f$ is an
surjection from $A$ to $C$.

Use the alternative definitions of ``injection"  and ``surjection" given in the previous problem and proof strategy as
described in chapter 1.

Comment: of course this shows compositions of bijections are
bijections, which will be useful.

\item

We outline how to define an $n$-tuple for arbitrary $n \in {\mathbb N}$ using the Kuratowski pair.  Let $\left<x_1,\ldots,x_n\right>$ be the ``function"
$\{[i,x_i] \mid i \in \{1,\ldots,n\}\}$, where the notation $[i,x_i]$ is to be read as a Kuratowski pair (in this context we need different notations for the Kuratowski pair and the 2-tuple; explain).

How do you pick out $x_i$ given $\left<x_1,\ldots,x_n\right>$ and $i$?

What is the relation between the type of the $n$-tuples and the common type of the $x_i$'s (we do assume that they are all of the same type).

Give a recursive definition of $\left<x_1,\ldots,x_n\right>$ in terms of $\left<x_1,\ldots,x_{n-1}\right>$ and $x_n$, using explicit set operations.  You will need a basis for this recursion (a definition of $\left<\right>$ or $\left<x\right>$).

\end{enumerate}

\newpage

\section{Digression:  The logic of subjects and predicates, or second-order logic}

This section is at a higher philosophical level than the preceding.  It originally appeared at the end of the Proof chapter, as it is about an extension of our logic, but the level of mathematical sophistication seems to require a prior treatment of ordered pairs and relations, so we have moved it here.  In any case, this is not an essential part of our main development.

At the bottom, the subject of logic ought to be completely general:  we ought to be able to talk about the entire universe.
So we declare that the domain over which the variables $x$ varies in $(\forall x.P[x])$ is simply the domain of all things, whatever things there are.

One might look at a unary sentence $P(x)$ or an atomic sentence $x\,R\,y$ and think that two (respectively three) objects are being discussed:
the objects $x$ [resp. $x$ and $y$] and the predicate $P$ [resp. $R$].  

We are going to analyze this impression.  First of all, we simplify matters by reading every unary sentence $P(x)$ as actually having the underlying
form $x\,P\,x$, so that all predicates are of the same sort (binary relations).  Secondly, we consider the difference between sentences
$A[x,y]$ and the atomic predicates $R$ that we are given.  If our sentences $A[x,y]$ are meaningful they too must express relations, so
we give names $\{x \rightarrow y : A[x,y]\}$ for such relations.  The rule for using this construction is that $a \{x \rightarrow y \mid A[x,y]\} b$
is to mean $A[a,b]$.

We allow predicate variables and quantification over the realm of predicates (= binary relations).  For any sentence ${\bf P}[R]$ in which a relation symbol $R$
appears, we allow the formation of sentences $(\forall R.{\bf P}[R])$ and $(\exists R.{\bf P}[R])$.  The rules for manipulating these relation quantifiers are exactly the same as for manipulating the quantifiers over objects.

We state firmly that we are {\em not\/} admitting a new sort of object (relations) over which these variables range.  The objects over which
the variables $x$ range are all the objects, and it can be proved that there can be no identification of the relations with a subset of our usual objects.

We add a further abbreviation $\{x \mid R[x]\}$ for $\{x \rightarrow y : x=y \wedge R[x]\}$.  This ties in with our abbreviation of $R(x)$ as $x\,R\,x$ (the change from brackets to parentheses here is principled!)

Suppose that the relations $R$ are to be identified with some objects.  We can preserve the grammatical distinction by writing ${\tt object}(R)$
for the object to be identified with $R$.  Now consider $\cal R$, a specific relation defined as $\{x \mid (\exists X.x={\tt object}(X) \wedge \neg X(x)\}$.  ${\cal R}({\tt object}({\cal R}))$ is then equivalent to $(\exists X.{\tt object}({\cal R})={\tt object}(X) \wedge \neg X({\tt object}({\cal R})))$, which is clearly equivalent to $\neg {\cal R}({\tt object}({\cal R})))$.  This is impossible.  So relations in general cannot be objects.  This is another analysis of ``Russell's paradox".

We stand by our stricture that the domain of our object variables is the entire universe of objects, so we do not allow relation variables to be regarded as denoting objects.  Nonetheless, we do not regard it as senseless to say that something is true of all predicates.  For example,
$R(0) \wedge (\forall x.R(x) \rightarrow R(x+1))$ expresses the idea that a relation $R$ is inductive, and $(\forall R.(\forall R.R(0) \wedge (\forall x.R(x) \rightarrow R(x+1))) \rightarrow R(3))$ is simply a true statement (3 has all inductive properties).

We resist certain extensions of this logical framework (which is usually called ``second-order logic").

The first extension we resist is the extension to ternary and higher arity relations.  We avoid the necessity to do this by making an assumption about the world:  $$(\exists \Pi_1.(\exists \Pi_2.(\forall xy.(\exists!z.x \Pi_1 z \wedge y \Pi_2 z)) \wedge (\forall z.(\exists!x.x \Pi_1 z) \wedge (\exists!y.y \Pi_2 z))))$$

This asserts the existence of a pairing construction on the universe by asserting the existence of its projection relations.  The unique object $z$ such that $x\Pi_1 z$ and $y \Pi_1 z$ whose existence is declared can be called $(x,y)$, the ordered pair of $x$ and $y$, and a ternary relation $B(x,y,z)$ can be taken as really meaning $x\,B\,(y,z)$ (with similar magic dispelling relations of all higher arities).

The second extension, which is much harder to resist, is the temptation to proceed to logic of third and higher orders.

Formally speaking, to pass to third order logic is to proceed to allow names for objects $\{R \rightarrow S:{\bf P}[R,S]\}$ representing
binary relations on relations $R$ and $S$, and then to admit quantifiers over these, and so forth.  Other more complex classes of relations
and predicates can be imagined.  

We can express all consequences of such a move in logic of second order alone.  The idea is to specify a domain $D_0$ to which we restrict
the object variables of our original language, then introduce a domain $D_1$ and a relation $E$ which satisfies $(\forall R.(\exists r \in D_1.(\forall xy \in D_0.x \,R\,y \leftrightarrow (x,y)\,E\,r)))$.  The objects in $D_0$ will be the genuine objects; the objects in $D_1$ will be (or include) our
relations; the true higher order relations will include the relations of third order alluded to above and indeed all the further strange kinds of relation you might want.

This can be done without enhancing our logic from ``mere" second order logic, and moreover, the picture given is false to our intentions.  We insist
that there are not two tiers of objects in our logic:  the domain of the object variables is all objects.  So while we can simulate a picture in which there
are first order objects, second order objects which capture all relations on first order objects, and third order relations, this is actually not
an enhancement of our world, but a (very interesting) suggestion of how there might be a lot of extra complexity in the objects.  Note that we
can iterate this to obtain fourth order logic, fifth order logic and so forth, and in fact our theory of types  above looks remarkably like such an iteration.

We are not tempted in the direction of third, fourth and higher order logic by thinking that the predicates represent a higher tier of objects:  we know by the argument above that as we add more and more tiers of relations of various orders the domain of objects we are talking about at the base must depart further and further from being {\em all\/} the objects.  There is another subtler temptation, which is to introduce third, fourth and higher order logic not as a higher tier of objects but as a higher tier of $\ldots$ relations.  For certainly relations have properties.  ``$R$ is symmetric" is a perfectly reasonable abbreviation for $(\forall xy.x\,R\,y \leftrightarrow y\,R\,x)$.  Our view is that we have not succumbed to the siren lure of third order logic in this direction as long as we only talk about specific
properties, relations and operations on predicates.  As long as we introduce no variables ranging over predicates of predicates (and woe betide us if we introduce quantifiers over predicates of predicates) we have not advanced to the level of third-order logic.

We give a brief account of what we are doing in the development of our theory of types in terms of the framework of ``second-order logic".

We use $x \in U$ to abbreviate $U(x)$ where $x$ is an object and $U$ is a predicate.

We consider the assertion ``$U$ is a type" as meaning $$((\forall xy \in U.(x,y)\in U) \wedge (\exists V.(\exists E.(\forall S.(\exists s \in V.(\forall x \in U.S(x) \leftrightarrow x \,E\,s))))))$$

For $U$ a type and $V$ a predicate, we read $$(\exists E.(\forall S.(\exists s \in V.(\forall x \in U.S(x) \leftrightarrow x \,E\,s))))$$ as
``$V$ is a power domain for $U$''.   This says that $V$ contains codes for the restriction of every unary predicate to $U$ (which makes
$V$ quite large).

We assert as an axiom that any type has a power domain which is a type.

Type 0 will be a predicate $U_0$; for each concrete natural number, type $U_{i+1}$ will be a power domain over type $U_i$
with membership relation $E_i$ (which as we will see is not the internal membership relation of the type theory).

$x \in_0 y$, where $x$ is type $i$ and $y$ is type $i+1$, is to be read $x\,E_i \,y$.  $x\sim y$ is to be read $x=y$ if $x$
and $y$ are of type 0.  Otherwise it is to be read $(\forall z.z \in_0 x \leftrightarrow z \in_0 y)$.  $x \in y$ is to be read
$x \in_0 y$ if $x$ is of type 0, and otherwise $x \in_0 y \wedge (\forall uv.u\sim v \rightarrow u \in_0 y \leftrightarrow v \in_0 y)$.

The relations $\in$ and $\sim$ in each type can be taken to implement our membership and equality relations for type theory.

A reader should notice that our construction of type theory amounts to iterating the passage to third-order logic which we deprecated, repeatedly.  Here we are using this machinery to implement additional complexity in the domain of objects, which we remarked was a sound reason to be interested in this kind of structure.

It should be noted that our type theory is an entirely ``first-order" theory:  there are no quantifiers over predicates in its language.  As a result, it may have interesting ``first-order" models in which not all restrictions of predicates to the types define sets at higher types, and we will see much later that this is the case.  The idea is that the comprehension axiom asserts that all sets of type $i$ objects defined by statements in the language of type theory
exist; this is not the same as saying (as we do in the framework presented here) that in some sense all sets of type $i$ objects are implemented
at type $i+1$.

On a very technical level, it should be noted that there is no way whatsoever to define a full infinite sequence of types using the framework we have given here:  this does nonetheless support the validity of all reasoning in type theory, because any particular argument in type theory mentions only finitely many types.

\newpage

\section{Defining Functions by Recursion; First-Order Peano Arithmetic}

Recursion is a special technique for defining functions with domain
$\mathbb N$.

Informally, a recursive definition might look like this (this is not a
completely general example): $f(0) = 0$; for each natural number $n$,
$f(n+1)=(f(n)+1)+1$.  This seems somehow suspect because this
definition of $f$ appears to mention $f$ itself in an essential way.

We show that this kind of definition is legitimate.  We begin by
exhibiting the technique of {\em iterative\/} definition of which the
example just given is a special case.

\begin{description}

\item[Iteration Theorem:] For any function $f:D \rightarrow D$ and $a \in D$ (of appropriate
types) there is a unique function $g:{\mathbb N}\rightarrow D$ such
that $g(0)=a$ and $g(n+1) = f(g(n))$ for each $n \in {\mathbb N}$.

\item[Definition:] Where $a$, $f$, $g$ are as in the statement of the
Theorem, we define $f^n(a)$ as $g(n)$.

\item[Proof of Iteration Theorem:]

We begin with a nonce

\begin{description}

\item[Definition:] A set $I$ is said to be {\em $(f,a)$-inductive\/}
iff $\left<0,a\right> \in I$ and $(\forall nx.\left<n,x\right> \in I
\rightarrow \left<n+1,f(x)\right> \in I)$.

\end{description}

Let $g$ be the intersection of all $(f,a)$-inductive sets.  We claim
that $g$ is the desired function.  Note that we do not even know that
$g$ is a function at this point!

We claim that $g$ is a subset of ${\mathbb N}\times D$.  Note that
$\left<0,a\right> \in {\mathbb N} \times D$ and for any
$\left<n,x\right> \in{\mathbb N} \times D$ we also have
$\left<n+1,f(x)\right> \in {\mathbb N} \times D$, so ${\mathbb N}
\times D$ is $(f,a)$-inductive, whence $g \subseteq {\mathbb N}\times
D$.

So we now know that every element of $g$ is an ordered pair whose
first component is a natural number and whose second component is in $D$, which is necessary but not
sufficient for $g$ to be a function with domain the set of natural
numbers and range included in $D$.

We claim that for each natural number $n$ there is exactly one object
$x$ such that $\left<n,x\right>$ is an element of $g$.  Define $A$ as
the set of all natural numbers $n$ such that there is exactly one
object $x$ such that $\left<n,x\right>$ is an element of $g$: we prove
our claim by showing that $A$ is inductive.

We first need to show that $0 \in A$.  We know that
$\left<0,a\right>\in g$, so there is at least one $x$ such that
$\left<0,x\right>\in g$.  Now consider \newline $g' = g- \{\left<0,x\right>\mid
x \neq a\}$.  We claim that $g'$ is 
$(f,a)$-inductive.
$\left<0,a\right> \in g'$ is obvious.  Suppose $\left<n,x\right> \in
g'$.  It follows that $\left<n+1,f(x)\right> \in g$, and in fact that
$\left<n+1,f(x)\right> \in g'$, because $\left<n+1,f(x)\right> \not\in
\{\left<0,x\right>\mid x \neq a\}$.  Since $g'$ is $(f,a)$-inductive,
$g \subseteq g'$.  But $g' \subseteq g$ as well, so $g=g'$, and $a$ is
the only object such that $\left<0,a\right> \in g' = g$, which is what we
needed to show.

Now we need to show that for any $k \in A$ we also have $k+1 \in A$.
Assume $k \in A$, whence there is exactly one $u$ such that
$\left<k,u\right> \in g$.  We need to show that there is exactly one
$v$ such that $\left<k+1,v\right> \in g$.  Since $\left<k,u\right> \in
g$, it follows that $\left<k+1,f(u)\right> \in g$, so there is at
least one such $v$.  Now define $g'$ as $g - \{\left<k+1,w\right> \mid
w \neq f(u)\}$.  We claim that $g'$ is $(f,a)$-inductive.  Clearly
$\left<0,a\right> \in g'$.  Suppose $\left<n,x\right> \in g'$; our aim
is to show $\left<n+1,f(x)\right> \in g'$.  Suppose otherwise for the
sake of a contradiction.  Clearly $\left<n+1,f(x)\right> \in g$: it is
thus necessary that $\left<n+1,f(x)\right> \in \{\left<k+1,w\right>
\mid w \neq f(u)\}$, which implies $f(x) \neq f(u)$ and also that
$n+1=k+1$.  From this it follows that $n=k$, and thus, since
$\left<n,x\right> = \left<k,x\right> \in g$, that $x=u$, whence $f(x)
\neq f(u)$ is impossible, which is the desired contradiction.  We then
have $g=g'$, whence $f(u)$ is the only object $x$ such that
$\left<k+1,x\right> \in g' = g$, whence $k+1 \in A$.

This completes the proof that $g$ is a function from ${\mathbb N}$ to
$D$.  Since $\left<0,a\right>\in g$, we have $g(0)=a$.  Since
$\left<n,g(n)\right> \in g$, we have $\left<n+1,f(g(n))\right> \in g$,
whence $g(n+1)=f(g(n))$.

Now we need to show that $g$ is the unique function with these
properties.  Suppose $g':{\mathbb N} \rightarrow V$, $g'(0) = a$ and
$g'(n+1) = f(g'(n))$.  $\left<0,a\right> \in g'$ is immediate.  If
$\left<n,x\right> \in g'$, then $x=g'(n)$, and
$\left<n+1,g'(n+1)\right> = \left<n+1,f(g'(n))\right> =
\left<n+1,f(x)\right> \in g'$, so $g'$ is $(f,a)$-inductive, whence $g
\subseteq g'$.  $g'$ contains exactly one element with first
projection $n$ for each natural number $n$, which must be the one
element with first projection $n$ belonging to $g$, so $g$ and $g'$
are the same set.

This completes the proof of the Iteration Theorem.

\item[Observation:]   This is more than a technical theorem:  it has some philosophically interesting content.   Our definition of the natural numbers
is based intellectually  on the use of natural numbers to count the elements of sets.   Here we are showing that our logical machinery allows us to implement the arguably quite different basic idea of applying an operation $n$ times to an object.

\item[Recursion Theorem:] For any set $a$ and function $g:({\mathbb
N}\times V)\rightarrow V$, there is a function $h:{\mathbb
N}\rightarrow V$ such that $h(0)= a$ and $h(n+1)=g(n,h(n))$ for each
$n \in {\mathbb N}$.

\item[Proof of Recursion Theorem:] Let $G(\left<n,x\right>)$ be defined as
$\left<n+1,g(n,x)\right>$.  Then $h(n) = \pi_2(G^n(\left<0,a\right>))$. 

\end{description}

There is an alternative way to define $f^n(a)$.  

\begin{description}

\item[Definition:] A set $S$ of natural numbers is an {\em initial
segment of the natural numbers\/} iff for all $n \in{\mathbb N}$, $n+1
\in S \rightarrow n \in S$.

\item[Theorem:] Any nonempty initial segment of the natural numbers
contains 0.

\item[Theorem:] $y = f^n(a)$ iff there is a function $g$ such that the
domain of $g$ is an initial segment $S$ of the natural numbers
including $n$ as an element, $g(0) = a$, for all $m$ such that $m+1
\in S$ we have $g(m+1)=f(g(m))$, and $y=g(n)$.  This formulation is
advantageous because it only appeals to the existence of finite sets.

\end{description}

We summarize the basic properties of $f^n(x)$ in the

\begin{description}

\item[Recursive definition of iteration:]  Where $D$ is a set, $a \in D$ and $f:D \rightarrow D$,

\begin{enumerate}

\item $f^0(a)=a$

\item $f^{n+1}(a)=f(f^n(a))$

\end{enumerate}

The types of $a$ and $n$ in $f^n(a)$ are the same, and the type of $f$ is one higher.  Notice that type indices are bold-faced, so they will not be confused with these indices.

We also define the freestanding notation $f^n$ for the function $$\{\left<x,f^n(x)\right>:x \in D\}.$$

When we iterate an operation which is not a function, as for example in ${\cal P}^{\bf 2}(A)$, the power set of the power set of $A$, we bold-face the index (which in this case
cannot possibly be a type index) to indicate that this is not an example of iteration.  There is in fact no reference to the number 2 (of any type) in this expression at all, any more than there is a reference to 2 when we write a variable $x^{\bf 2}$ with the type index 2.

\end{description}

As examples we can present definitions of addition and multiplication.

Give the nonce name $\sigma$ to the successor function on natural numbers ($\sigma = \{(n,n+1) \mid n \in {\mathbb N}\}$)  We can define
$m+n$ (for any natural numbers $m$, $n$) as $\sigma^n(m)$ (adding $n$
is iterating successor $n$ times).  We can define $m \cdot n$ for any
natural numbers $m$ and $n$ as $(\sigma^m)^n(0)$: to add $m\cdot n$ is
to add $m$ $n$ times.

At this point we can observe that our original definition of $n+1$ and the new definition of $n+1$ in terms of the addition function just defined agree.
If $n$ is a natural number, $n+1$ (read as a sum) is defined as $\sigma^1(n)$ which is $\sigma^{\sigma(0)}(n)$ by the definition of 1, which is $\sigma(\sigma^0(n))$ by an application of the definition of $f^n(a)$, which is $\sigma(n)$ by another application of the definition of $f^n(a)$, which, finally, is $n+1$ in the original sense (the successor of $n$) by definition of the function $\sigma$.

\begin{description}


\item[Demonstrations of some properties of addition and multiplication] using the recursive definition of iteration:
\begin{enumerate}

\item $m+0 = \sigma^0(m) = m$, so $m+0=m$.

\item $m+ \sigma(n) = \sigma^{n+1}(m) = \sigma(\sigma^n(m)) = \sigma(m+n)$, so $m+\sigma(n) = \sigma(m+n)$, or $m+(n+1) = (m+n)+1$.

\item $m \cdot 0 = (\sigma^m)^0(0) = 0$, so $m \cdot 0 = 0$

\item $m \cdot \sigma(n) = (\sigma^m)^{n+1}(0) = \sigma^m((\sigma^m)^n(0)) = \sigma^m(m \cdot n) = m \cdot n + m$, so $m\cdot \sigma(n) = m \cdot n+m$, or $m\cdot(n+1) = m\cdot n + m$.


\end{enumerate}



\end{description}

The recursive (really as we see above ``iterative'') definitions of
addition and multiplication are incorporated into modern formulations
of ``Peano's axioms'', which make no essential reference to sets.  The
theory with these axioms is formally called {\em first-order Peano
arithmetic\/}.

When we reason in first-order Peano arithmetic, we are not reasoning
in our type theory.  But, since we have shown that there is an
intepretation of the axioms of first-order Peano arithmetic in our
type theory, any theorems we prove in first-order Peano arithmetic
will be true in that interpretation.  We will see below that there is
a different interpretation of Peano arithmetic commonly used in
untyped set theory (the von Neumann definition of the natural numbers,
already mentioned above), and anything we prove in arithmetic will
also be true in that interpretation (and in any other we come up
with).

The convention when reasoning in first-order Peano arithmetic is to
assume that all quantifiers are restricted to the natural numbers (we
are not talking about anything else, and notably we are not talking
about sets of natural numbers as we do in the original (second-order)
version of the theory).  Note this particularly in axiom 5.

\begin{enumerate}

\item $0$ is a natural number.

\item For each natural number $n$, $\sigma(n)$ is a natural number.  For all natural numbers $m, n$, $m+n$ and $m \cdot n$ are natural numbers.

\item For all natural numbers $n$, $\sigma(n) \neq 0$

\item For all natural numbers $m,n$, $\sigma(m)=\sigma(n)\rightarrow m=n$.

\item For each formula $\phi[n]$, we adopt as an axiom $\phi[0] \wedge
(\forall k.\phi[k] \rightarrow \phi[\sigma(k)]) \rightarrow (\forall
n.\phi[n])$.  This is the principle of mathematical induction.  Note
that this is not really a single axiom 5, but a suite of axioms
$5_{\phi}$.  Such a suite is called an {\em axiom scheme\/}.  A scheme
is needed because we do not refer to sets here.

\item For all natural numbers $m,n$, $m+0=m$

\item For all natural numbers $m,n$, $m+\sigma(n)=\sigma(m+n)$

\item For all natural numbers $m,n$, $m\cdot 0 = 0$

\item For all natural numbers $m,n$, $m\cdot \sigma(n) = m\cdot n + m$



\end{enumerate}


Since addition is also a primitive operation here, we use a primitive
notation for successor at first rather than the more natural addition
of 1.  Notice the reformulation of mathematical induction in terms of
formulas rather than sets.  This formulation of mathematical induction
is not a statement with a quantifier over formulas (we cannot really
do that for reasons which we may discuss {\em much\/} later on) but an
infinite collection of different axioms, one for each formula $\phi$.
You should notice that the axioms for addition and multiplication
capture the iterative definitions of addition and multiplication given
above.

We give some sample proofs in Peano arithmetic.

\begin{description}

\item[Definition:] $1 = \sigma(0)$ (of course this recapitulates an earlier definition given in the context of our type theory).  Note that it is immediate from
the axioms for addition that $n+1 = n+\sigma(0) = \sigma(n+0) =
\sigma(n)$.  We feel free to use these notations interchangeably.

\item[Proof Strategy:] We give the first order version of mathematical
induction as a proof strategy.

To deduce a goal $(\forall n.\phi[n])$, deduce the following two goals:

\begin{description}

\item[Basis step:]  Deduce $\phi[0]$.

\item[Induction step:] Deduce $(\forall k.\phi[k] \rightarrow
\phi[k+1])$.  Application of prior proof strategy expands this: let
$k$ be an arbitrarily chosen natural number (which might be 0!):
assume $\phi[k]$ (this is called the {\em inductive hypothesis\/}, and
it is useful to emphasize where in an induction proof the inductive
hypothesis is used), and deduce the new goal $\phi[k+1]$.

\end{description}

\item[Theorem:] For each natural number $n \neq 0$, there is a unique
natural number $m$ such that $m+1=n$.

\item[Proof:] We prove by mathematical induction the assertion ``For
each natural number $n$, if $n \neq 0$, then there is a natural number
$m$ such that $m+1=n$''.

For $n=0$ this is trivially true (basis step).

Suppose it is true for $n=k$; then our goal is to prove that it is
true for $n=k+1$ (induction step).

Either $k=0$ or there is an $m$ such that $m+1=k$, by inductive hypothesis.
In either case, there is an $m'$ such that $m'+1 = k+1$, namely $k$ itself.

So the assertion is true for all $n$ by mathematical induction.  What
is strange here is that the inductive hypothesis is not used in this
proof!

The observant reader will notice that we have not yet proved the
theorem.  We have shown that for each nonzero natural number $n$ there
is an $m$ such that $m+1=n$, but we have not shown that this $m$ is
unique yet.  Suppose that $m+1=n$ and also $m'+1=n$: it follows
directly from axiom 4 that $m=m'$.  So we have shown that there can
only be one such $m$ for each $n$ and the proof is complete.

\item[Theorem:]  For each natural number $n$, $0+n=n+0$.

\item[Proof:] We prove this by mathematical induction.

$0+0 = 0+0$ completes the proof of the basis step.

Now for the induction step.  We assume that $0+k=k+0$ and our goal is
to show that $0+\sigma(k) = \sigma(k)+0$.  $0+\sigma(k) = \sigma(0+k)$
by axioms, and $\sigma(0+k) = \sigma(k+0)$ (by inductive hypothesis) $
= \sigma(k) = \sigma(k)+0$.  This completes the proof of the induction
step and of the theorem.

\item[Theorem:]  For any natural numbers $m,n$, $(m+1)+n = (m+n)+1$.

We fix $m$ and prove this by induction on $n$.

The basis step is established by $(m+1)+0 = m+1 = (m+0)+1$.

The hypothesis of the induction step is $(m+1)+k = (m+k)+1$; the goal
is to show $(m+1)+(k+1) = (m+(k+1))+1$.  $(m+1)+(k+1) = ((m+1)+k)+1$
by axiom, which is equal to $((m+k)+1)+1$ by inductive hypothesis,
which is in turn equal to $(m+(k+1))+1$ by axiom, completing the
proof.

\item[Theorem:]  For any natural numbers $m,n$, $m+n=n+m$.

\item[Proof:]  We prove this by (you guessed it!) mathematical induction.

The statement we actually prove by mathematical induction is ``for any
natural number $n$, for any natural number $m$, $m+n=n+m$.''

The basis step is ``For any natural number $m$, $m+0 = 0+m$''.  We
just proved that!

The induction hypothesis is ``For any natural number $m$, $m+k =
k+m$'' (for some fixed natural number $k$) and the induction goal is
``For any natural number $m$, $m+(k+1) = (k+1)+m$''.  Now $m+(k+1) =
(m+k)+1$ by axiom, which is in turn equal to $(k+m)+1$ by inductive
hypothesis, which is equal to $(k+1)+m$ by the previous theorem,
proving the induction goal and completing the proof of the theorem.

\end{description}

We follow our own example in earlier sections on logic and recapitulate a proof of the same theorem (the commutativity of addition) in a more formal style.

Our aim is to prove $(\forall yx.x+y=y+x)$, the commutative law of addition.

We must be proving it by math induction, as we have no other way to do it!

We will prove this by induction on $y$ (as a rule, it is better to do induction on the variable farthest to the right
in an expression you are going to work with, because of the forms of axioms 6-9).

The basis step will be $(\forall x.x+0=0+x)$

The induction hypothesis will be $(\forall x.x+k=k+x)$ ($k$ being an arbitrary number we introduce).

The induction goal will be $(\forall x.x+\sigma(k)=\sigma(k)+x)$

This gives us the following proof outline:

\begin{description}

\item[Goal:]  $(\forall yx.x+y=y+x)$  We prove this by induction on $y$.

\begin{description}

\item[Basis Goal 1:]  $(\forall x.x+0=0+x)$

\item[Let] $k$ be chosen arbitrarily.

\item[Ind Hyp 1:]  $(\forall x.x+k=k+x)$ 

\item[Induction Goal:]  $(\forall x.x+\sigma(k)=\sigma(k)+x)$


\end{description}

We now proceed to fill in the complete proof (though not without further comments about what we are doing!!!)
The reason we are numbering the basis and induction items is that there will be subproofs of this proof which are
induction proofs themselves and have their own bases and induction steps.

\newpage

\begin{description}

\item[Goal:]  $(\forall yx.x+y=y+x)$  We prove this by induction on $y$.

\begin{description}

\item[Basis Goal 1:]  $(\forall x.x+0=0+x)$  We prove this by induction on $x$!

\begin{description}

\item[Basis Goal 2:]  $0+0=0+0$

\item[1:]  $0+0=0+0$  ref =  [That was easy!]

\item[Ind Hyp 2 (2):]   $k+0=0+k$

\item[Ind Goal 2:]  $\sigma(k)+0=0+\sigma(k)$

\item[3:]  $0+\sigma(k)=\sigma(0+k)$  ax 7 $x:=0; y:=k$

\item[4:]  $0+\sigma(k)=\sigma(k+0)$  subs using (2) [the ind hyp] into (3)

\item[5:]  $k+0=k$  ax 6 $x:=k$

\item[6:]  $0+\sigma(k)=\sigma(k)$  subs using line (5) into line (4)

\item[7:]  $\sigma(k)=\sigma(k)+0$  ax 6 $x:=\sigma(k)$

\item[8:]  $0+\sigma(k)=\sigma(k)+0$  trans = 6,7

\item[9:]  $\sigma(k)+0 = 0+\sigma(k)$  symm = 8, and we are done with the basis goal.  I proved this differently than I did in class (I think) though the basic idea is the same.


\end{description}

\item[Let] $k$ be chosen arbitrarily.

\item[Ind Hyp 1 (2):]  $(\forall x.x+k=k+x)$  This is line 2 again because everything in an induction step uses local hypotheses and goes away.  We could even call it line 1, since we are never going to refer to line 1 again, but the original line
1 has not vanished.  It wouldn't do any harm to call this line 10, as long as you know that lines 2-9 above can't be used.

\item[Induction Goal:]  $(\forall x.x+\sigma(k)=\sigma(k)+x)$  I'm going to start working on the left side of this because I can see what to do with it.  I will get as far as I can and then I will see something else that I want to prove...by induction of course.
\begin{description}
\item[Let] $m$ be arbitrary (I'm not going to use $l$ because it looks too much like a 1).  Notice that I'm using the standard technique to deal with a universal quantifier instead of induction.  Sometimes it works!

\item[Goal:]  $m+\sigma(k)=\sigma(k)+m$

\item[3:]  $m+\sigma(k) = \sigma(m+k)$  ax 7 $x:=m, y:=k$  working on left side as I said, because axiom 7 applies.

\item[4:]  $m+\sigma(k)=\sigma(k+m)$  subs using (2) (the ind hyp) into (3).

\item[Goal:]  I see that I need to prove $\sigma(k)+m = \sigma(k+m)$ to complete the proof.  I prove this as a Lemma below, by induction:  actually the Lemma is $$(\forall xy.\sigma(x)+y=\sigma(x+y)),$$a statement which looks rather like axiom 7 but isn't.

The proof of the Lemma is given below; we proceed with the main proof assuming that we have it.

\item[5:]  $\sigma(k)+m = \sigma(k+m)$  Lemma proved below, $x:=k, y:=m$

\item[6:] $m+\sigma(k)=\sigma(k+m)$  symm trans = lines 4 and 5

\end{description}


\end{description}

\end{description}

\end{description}

This completes the proof of the main theorem, once we prove the Lemma, whose free-standing proof follows.
Of course we cannot use commutativity of addition in the proof of the Lemma!

\begin{description}

\item[Lemma:]  $(\forall xy.\sigma(x)+y=\sigma(x+y))$  Im going to prove this by induction on $y$; first I'm going to use the usual strategy for a universal quantifier to get rid of $x$.

\begin{description}

\item[Let] $a$ be chosen arbitrarily.

\item[Goal:]  $(\forall y.\sigma(a)+y = \sigma(a+y))$  {\em This\/} is what we will prove by induction on $y$.

\begin{description}

\item[Basis Goal:]  $\sigma(a)+0 =\sigma(a+0)$

\item[1:]  $\sigma(a)+0=\sigma(a)$  ax 6 $x:=\sigma(a)$

\item[2:]  $a+0=a$  ax 6 $x:=a$

\item[3:]  $\sigma(a+0)=\sigma(a)$  both sides line 2

\item [4:]  $\sigma(a)+0 =\sigma(a+0)$  symm trans = 1,3

\item[Let] $k$ be arbitrary

\item[Ind Hyp (5):]  $\sigma(a)+k=\sigma(a+k)$

\item[Ind Goal:]  $\sigma(a)+\sigma(k) = \sigma(a+\sigma(k))$  Notice that both sides offer opportunities to calculate using axiom 7.

\item[6:]  $\sigma(a)+\sigma(k) = \sigma(\sigma(a)+k)$  ax 7, $x:=\sigma(a), y:=k$  You should notice the opportunity to rewrite using the inductive hypothesis!

\item[7:]  $\sigma(a)+\sigma(k) = \sigma(\sigma(a+k))$  subs into line 6 using line 5 (the ind hyp)

\item[8:]  $a+\sigma(k)=\sigma(a+k)$  ax 7 $x:=a,y:=k$

\item[9:]  $\sigma(a)+\sigma(k) = \sigma(a+\sigma(k))$  subs using (8) into (7)  [an equation can be used to substitute in either order].

\end{description}


\end{description}

\end{description}

That completes the proof of the lemma and the theorem.

Much more natural definitions of the arithmetic operations which use
the intuitive idea that the numbers are sizes of sets are given below,
and in terms of these definitions much more natural proofs of
properties such as the ones just proved can be given.  Proofs in Peano
arithmetic are nonetheless a useful exercise: they apply to quite
different implementations of the natural numbers (another
implementation will be given later): for any implementation, if the
Peano axioms hold, then all the theorems following from the Peano
axioms also hold.

Apparently stronger forms of both induction and recursion are
available, but turn out to be equivalent to the basic forms already
given.  A presentation of these requires some prior discussion of the
familiar order on the natural numbers.

\begin{description}

\item[Definition:] For natural numbers $m,n$, we say $m\leq n$ ({\em
$m$ is less than or equal to $n$\/}) just in case $(\exists k.m+k=n)$.
We define $m<n$ ({\em $m$ is less than $n$\/}) as $m \leq n \wedge m
\neq n$.  We define $m\geq n$ ({\em $m$ is greater than or equal to
$n$\/}) as $n\leq m$, and similarly define $m>n$ ({\em $m$ is greater
than $n$\/}) as $n<m$.

\end{description}

Note that we assume here that such things as the associative and
commutative laws of addition have already been proved.

\begin{description}

\item[Theorem:] For all natural numbers $m,n,k$, if $m+k = n+k$ then
$m=n$.  

\item[Proof:] Fix $m$ and $n$ and prove by induction on $k$.  This is
obvious for $k=0$.  If it is true for $k$ and $m+(k+1) = n+(k+1)$,
then $(m+k)+1 = (n+k)+1$ by addition axiom, $m+k = n+k$ by axiom 4,
and $m=n$ by inductive hypothesis.



\item[Theorem:] The relation $\leq$ on natural numbers just defined is
a linear order. 

\item[Proof:] $n\leq n = n+0$ is immediate.  If $m\leq n$ and $n\leq
m$ then we have $n=m+k$ and $m=n+l$ for some $k$ and $l$, whence
$n=n+0=n+(k+l)$, so $k+l = 0$, whence it is easy to show that $k=l=0$,
so $m=n$.  If $m \leq n$ and $n \leq p$, then for some $k,l$, $m+k =n$
and $n+l = p$, so $(m+k)+l = m+(k+l)=p$.  This shows that $\leq$ is a partial order.  

We need to show further that $(\forall m.(\forall n.m \leq n \vee n \leq m))$.  That $(\forall n.0 \leq n \wedge n \leq 0)$ is evident, because $0 \leq n$ is true for any $n$.  Suppose that $(\forall n.k \leq n \vee n \leq k)$.  We want to show that $(\forall n.k+1 \leq n \vee n \leq k+1)$  Either $n$ is 0, in which case $n \leq k+1$, or for some $m$, $n=m+1$, in which case $k \leq m \leftrightarrow m \leq k$ by inductive hypothesis, whence $k+1 \leq m+1=n \vee n=m+1 \leq k+1$ by axiom 4.

\item[Theorem:]  $m \leq n \leftrightarrow m+k \leq n+k$.

\item  $m+p = n \leftrightarrow (m+k)+p = n+k$

\item[Corollary:] $m<n \leftrightarrow m+k <n+k$

\item[Theorem:] For all $n \in {\mathbb N}$, for all $k \in {\mathbb
N}$, $k \leq n \leftrightarrow k<n+1$.

\item[Proof:] Prove this by induction on $n$.  The basis step requires
us to show that $m \leq 0 \leftrightarrow m < 1$ for all $m$.  If $m \leq 0$,
then since $0 \leq 1$ and $1 \not\leq 0$, $m < 1$ is obvious.  If
$m\neq 0$ then $m =n+1$ for some $n$, so $1\leq m$, thus $m<1
\rightarrow m=0$ (by contrapositive).  Now if $m \leq k \leftrightarrow m <
k+1$, for all $m$, we immediately have $(m+1) \leq (k+1) \leftrightarrow
(m+1)<(k+1)+1$ We certainly also have $0\leq k+1 \leftrightarrow 0<(k+1)+1$,
and since every number is either 0 or a successor we have shown for
all $m$ that $m \leq k+1 \leftrightarrow m < (k+1)+1$

\item[Theorem (Strong Induction, set form):] For any set $A$ of
natural numbers, if $(\forall a \in {\mathbb N}.(\forall x < a.x \in
A)\rightarrow a \in A)$, then $A={\mathbb N}$.

\item[Proof:] Suppose that $A$ is a set of natural numbers and
$(\forall a \in {\mathbb N}.(\forall x < a.x \in A)\rightarrow a \in
A)$.  We define the set $B$ as $\{b \in {\mathbb N}.(\forall x \leq
b.x \in A)\}$.  We show that $B$ is inductive.  Since $B \subseteq A$
is obvious, $B = {\mathbb N} \rightarrow A = {\mathbb N}$.

Since $(\forall x < 0.x \in A)$ is vacuously true, $0 \in A$.  For any
$b \leq 0$, $b=0\in A$, so $0 \in B$.

Now suppose that $k \in B$.  Our goal is to show that $k+1 \in B$.
Since $k \in B$, we have $p \in A$ for all $p \leq k$, and so for all
$p < k+1$.  It then follows that $k+1 \in A$, and since we have $p \in
A$ for all $p<k+1$ as well, we also have $k+1 \in B$.  This completes
the proof that $B$ is inductive, which we have already seen is
sufficient for the proof of the theorem.

\item[Theorem (Strong Induction, property form):] For any formula
$\phi$, $(\forall a \in {\mathbb N}.(\forall x <
a.\phi[x])\rightarrow \phi[a]) \rightarrow (\forall n \in {\mathbb
N}.\phi[n])$.

\item[Proof:] This is proved in the same way as the previous theorem.

\end{description}

There is a form of recursion which is to standard recursion (or
iteration) roughly as strong induction is to standard induction.

\begin{description}

\item[Theorem (Course-of-Values Recursion):] Let $A$ be a set.  Let
${\cal F}$ be the set of all functions with domain a proper initial
segment $$\{m \in{\mathbb N}\mid m <n\}$$ of the natural numbers and
range a subset of $A$ (notice that the function with domain
$\emptyset$ is one of these: set $n=0$).  Let $G$ be any function from
$\cal F$ to $\iota``A$.  Then there is a uniquely determined function
$f:{\mathbb N}\rightarrow A$ such that $\{f(n)\} = G(f \lceil \{m \in
{\mathbb N} \mid m < n\})$ for each $n \in {\mathbb N}$.

\item[Proof:] We define a function $H$ from ${\cal F}$ to ${\cal F}$
as follows.  If $g \in {\cal F}$ has domain $\{m \in {\mathbb N}\mid
m<n\}$, define $H(g)$ as $g \cup (\{n\} \times G(g))$ (recall that
$G(n)$ is the singleton set containing the intended value at $n$ of
the function being constructed).  Now apply the iteration theorem:
define $f(n)$ as $H^{n+1}(\emptyset)(n)$.  It is straightforward to verify
that this function has the desired property.

\item[Example:] An example of a function defined in this way, in which
the value of $f$ at any natural number depends on its values at {\em
all\/} smaller natural numbers, would be $f(n) = 1+\Sigma_{i<n}
f(i)$\footnote{I should give more examples of this kind of function definition and a discussion of the interesting things going on here with types.}

\end{description}

It is a usual exercise in a book of this kind to prove theorems of
Peano arithmetic up to the point where it is obvious that the basic
computational axioms of arithmetic and algebra can be founded on this
basis (and we may do all of this in these notes or in exercises).  It
is less obvious that all usual notions of arithmetic and algebra can
actually be defined in terms of the quite restricted vocabulary of
Peano arithmetic and logic: this is very often asserted but seldom
actually demonstrated.  We supply an outline of how this can be
established.

We give basic definitions without (or with only an indication of)
supporting proofs to indicate that the expressive power of Peano
arithmetic without set language is enough to talk about finite sets of
natural numbers and to define recursive functions.  This is a serious
question because the definition of recursive functions above relies
strongly on the use of sets.  Notice that we use the alternative
formulation of the definition of $f^n(a)$ in this development, because
we only code finite sets of natural numbers as natural numbers here,
and the alternative formulation has the advantage that it only talks
about finite sets.

\begin{description}

\item[Definition:] For natural numbers $m,n$ we say $m|n$ ({\em $n$ is
divisible by $m$\/} or {\em $m$ is a factor of $n$\/}) iff there is a natural
number $x$ such that $m\cdot x = n$.

\item[Definition:] A natural number $p$ is a {\em prime\/} iff it has
exactly two factors.  (One of these factors must be 1 and the other $p
\neq 1$ itself).

\item[Definition:] Let $p$ be a prime.  A natural number $q$ is a {\em
power of $p$\/} iff $p$ is a factor of every factor of $q$ except 1.

\item[Definition:] Let $p$ be a prime and $n$ a natural number.  A
nonzero natural number $m$ {\em occurs in the base $p$ expansion of $n$\/}
just in case $n$ can be expressed in the form $a\cdot q+m\cdot r+s$,
where $q>r>s$ and $q,r$ are powers of $p$.

\end{description}

The underlying idea is that we now have the ability to code finite
sets of natural numbers as natural numbers (and so in fact sets of
sets, sets of sets of sets, and so forth).

\begin{description}

\item[Definition:] Define $x \in_p y$ as ``$x+1$ occurs in the base
$p$ expansion of $y$''.  For any prime $p$ and naturals
$x_1,\ldots,x_n$ all less than $p-1$ define $\{x_1,\ldots,x_n\}_p$ as
the smallest natural number $y$ such that $(\forall z.z\in_p y \leftrightarrow
z=x_1 \vee\ldots\vee z=x_n)$. [there is something to prove here,
namely that there is such a $y$].

\item[Definition:] Define $\left<x,y\right>_{p,q}$ as
$\{\{x\}_p,\{x,y\}_p\}_q$.

\item[Definition:] For any function $f$, we say that $f$ is {\em definable in
Peano arithmetic\/} iff there is a formula $\phi[x,y]$ in the language of
arithmetic such that $\phi[x,y] \leftrightarrow y=f(x)$.

\item[Theorem:] For any function $f$ definable in Peano arithmetic, $y
= f^n(x)$ iff there are primes $p<q<r$ such that there is a natural
number $g$ such that $(\forall m\leq n.(\exists!
y.\left<m,y\right>_{p,q} \in_r g))$ and $\left<0,x\right> \in_r g$ and
$(\forall m<n.(\forall y.\left<m,y\right>_{p,q} \in_r g \rightarrow
\left<m+1,f(y)\right>_{p,q}\in_r g))$.  Note that this is expressible in the
language of Peano arithmetic, so all functions definable by iteration
of definable functions are definable (and functions definable by
recursion from definable functions are also definable since we can
represent pairs of natural numbers as natural numbers and define the
projection functions of these pairs).

\item[Definition:] Define $d(x)$ as $2\cdot x$.  Define $2^n$ as
$d^n(1)$.  Define $x \in_{\mathbb N} a$ as $$(\exists y>x.(\exists
z<2^x.(\exists u.a = u\cdot 2^y+2^x+z))).$$ This expresses that the
$n$th digit in the binary expansion of $a$ is 1, and this supports a
nice coding of finite sets of natural numbers as natural numbers,
which we will have occasion to use later.

\end{description}

\newpage

\subsection{Exercises}

\begin{enumerate}

\item
If I define a function $I_n$ such that $I_n(f) = f^n$ (so for example
$I_3(f)(x) = f^3(x) = f(f(f(x)))$, I invite you to consider the
functions $(I_n)^m$.  For example, compute $(I_2)^3(f)(x)$.  Compute
$(I_3)^2(f)(x)$.  There is an equation $(I_m)^n = I_{F(m,n)}$, where
$F$ is a quite familiar operation on natural numbers, which you can
write and might derive if you do enough experiments.  There is a
serious formal problem with this equation, though, in our type theory.
What is the function $F(m,n)$?  What is the formal problem?

\item Prove  the theorem $$(\forall m:m \neq m+1)$$ of Peano arithmetic.

Indicate each application of an axiom and of an inductive hypothesis.
Do not apply theorems you have not proved yourself on your paper.  You
may identify $\sigma(x)$ and $x+1$ without comment for any natural
number $x$.

\item
 Prove as many of the following as you can in first-order Peano
 arithmetic, not necessarily in the given order (but this is the suggested order).  Your proofs should
 not mention sets or the type theory definitions of the natural
 numbers (this is all just arithmetic from the Peano axioms).

 Use proof strategy.  You can be a little more freeform than
 heretofore, but take pains to make it clear what you are doing.  You
 may use theorems already proved in the notes or already proved by
 you.  You may {\em not\/} use anything else you think you know about
 arithmetic.

In some of these, you may need to prove lemmas as I had to in the proof of the commutative law of addition.

I suggest looking at the proof of the left distributive law which appears after these exercises as a style model.

Do prove at least two of them.

\begin{enumerate}



\item The associative law of addition.

\item The distributive law of multiplication over addition (for this one, I require you to prove the right distributive property (a proof of the left distributive property appears below) without using the commutative law of multiplication.

\item The associative law of multiplication.

\item The commutative law of multiplication.

\end{enumerate}

\item  Prove the {\em Well-Ordering Principle\/}:  for any nonempty set of natural numbers $A$, there is an element $m$ of $A$ such that for all $x \in A$, $m \leq x$.  The usual hint:  how do we prove anything about natural numbers?

\item Assuming ordinary knowledge about elementary algebra of natural numbers and basic properties of divisibility, write a proof by strong induction that any natural number is a finite product of primes.

As a footnote to this, see if you can make a proposal as to how to formally define the notion of a product of a finite list of primes in our formal system.


\end{enumerate}

\newpage

\subsection{A case study:  proof of the left distributive law in formal arithmetic}

In this subsection, we prove the left distributive law $a\cdot (b+c) = a\cdot b +a\cdot c$.  We assume that the associative law of addition

$$(\forall m:(\forall n:(\forall r:(m+n)+r = m+(n+r))))$$

 has already been proved
(it appears in the previous set of exercises.  This proof might be used as a style manual for the formal arithmetic proofs in that exercise set.



\begin{description}

\item[Theorem: ]$(\forall x:\forall y:\forall z:x\cdot(y+z) = x \cdot y + x \cdot z$.  I do assume standard order of operations for addition and multiplication without further comment.

\item[(1):] $a=a$ trivial line for universal generalization

\item[(2):] $b=b$ trivial line for universal generalization

\item[Goal:]  $(\forall z:a\cdot(b+z) = a \cdot b + a \cdot z)$ (I'm not going to indent here:  you need to keep track of the block structure of the proof.)

We prove this goal by mathematical induction.

\item[Basis Goal:]  $a\cdot(b+0) = a \cdot b + a \cdot 0$

\item[style remark:]  We introduce all instances of axioms at the top level (applying UI to the axiom) and then substitute using these
instances into other equations.  We may often start with a trivial equation, an instance of the reflexivity of equality, as here.

\item[(3):]  $a \cdot (b+0) = a\cdot (b+0)$  refl = (an acceptable abbreviation for ``reflexivity of equality")

\item[(4):]  $b+0=b$  UI ax 6 m:=b

\item[(5):]  $a \cdot (b+0) = a \cdot b$  subs 4 into 3 (apply substitution property of equality using equation 4 to equation 3)

\item[(6):]  $a \cdot b = a \cdot b+0$  UI ax 6 m:=b

\item[(7):]    $a \cdot (b+0) =  a \cdot b+0$  trans = 5,6  (this could also be justified by substitution)

\item[(8):]  $a \cdot 0 = 0$ UI ax 8 m:= a

\item[(9):]  $a\cdot(b+0) = a \cdot b + a \cdot 0$  subs 8 into 7 (substitution in either order can be justified this way:  the reader can presumably tell which way we are going).

This completes the proof of the basis goal.

\item[Induction Step:]

\item[Ind Hyp:]  Let $k$ be arbitrarily chosen.  Assume {\bf (10)} $a\cdot(b+k) = a \cdot b + a \cdot k$

\item[Ind Goal:]  $a\cdot(b+\sigma(k)) = a \cdot b + a \cdot \sigma(k)$

\item [(11):]  $a\cdot(b+\sigma(k)) = a\cdot(b+\sigma(k))$  refl =

\item[(12):]   $b + \sigma(k) = \sigma(b+k)$  UI ax 7 m:=b n:=k

\item[(13):]  $a\cdot(b+\sigma(k)) = a\cdot(\sigma(b+k))$  subs 12 into 11

\item[(14):]  $a\cdot(\sigma(b+k)) = a\cdot(b+k) +a$  UI ax 9 m:= a n:=b+k

\item[(14a):] $a\cdot(b+\sigma(k)) =  a\cdot(b+k) +a$ trans = 13,14 (accidentally left out a line and didn'f want to renumber the rest of the proof!)

\item[(15):]   $a\cdot(b+\sigma(k)) = (a\cdot b + a \cdot k)+ a$  subs 10 (ind hyp!) into 14a  

You {\bf must} supply parentheses.  Though we do have associativity of addition by hypothesis, we are still at a level where we show its use
explicitly.

\item[(16):]  $(a\cdot b + a \cdot k)+ a = a\cdot b + (a \cdot k+ a)$  assoc + [m:= $a \cdot b$ n:= $a \cdot k$ r:=a]

\item[(17):]   $a\cdot(b+\sigma(k)) = a\cdot b + (a \cdot k+ a)$ trans = 15,16

\item[(18):]   $a \cdot \sigma(k) = a\cdot k + a$  ax 9 m:= a n:= k

\item[(19):]  $a\cdot(b+\sigma(k)) = a \cdot b + a \cdot \sigma(k)$  subs 18 into 17

\item[(20):]  $(\forall z:a\cdot (b+z) = a \cdot b + a \cdot z)$  mathematical induction 9, 10-19.  Notice that the basis step reference is to a line but the induction step reference must be to a block.

\item[(21):]  The theorem UG 1-20 (really this is two applications of UG but I think you follow).

\end{description}

In this proof I followed a much more careful strategy than I followed in class, which I suggest for the proof exercises.  I started by setting the left side of the theorem equal to itself, then
did a series of substitutions into the right side of this equation until I got the desired right side.  This might be familiar to you, as it is often mandated as a style of proving trig identities.

Please be aware that you cannot prove equations the way you learned to solve them in algebra.  If we have an equation $a=b$ and do the same thing to both sides of it
(getting something like $F[a] = F[b]$) we can then posit $F[a] = F[b]$ (this derived rule appears in some proofs above, labelled ``both side"), but proving $F[a] = F[b]$ will not prove $a=b$ (this style of reasoning, very useful when trying to solve equations, is not reversible).

\newpage

\section{Equivalence Relations, Partitions, and Representatives:  the Axiom of Choice}

\begin{description}

\item[Definition:]  Sets $A$ and $B$ are said to be {\em disjoint\/} just in case $A \cap
B = \emptyset$.

\item[Definition:]A collection $P$ of sets is said to be {\em pairwise disjoint\/} just
in case $$(\forall A \in P.(\forall B \in P.A=B \vee A \cap B = \emptyset)).$$

\item[Definition:]  A collection $P$ of sets is a {\em partition of A\/} iff $\emptyset
\not\in P$, $\bigcup P = A$, and $P$ is pairwise disjoint.  A
partition of $A$ is a collection of nonempty sets which do not overlap
and which cover all of $A$.  We say that a collection $P$ {\em is a
partition\/} iff it is a partition of $\bigcup P$.

\item[Definition:]  If $R$ is an equivalence relation and $x \in {\tt fld}(R)$ we define $[x]_R$,
the {\em equivalence class of $x$ under $R$\/}, as $R``(\{x\}) = \{y \mid x \,R\,y\}$.

\item[Theorem:] If $R$ is an equivalence relation, $P_R= \{[x]_R \mid
x \in {\tt fld}(R)\}$ is a partition of ${\tt fld}(R)$.

\item[Proof:]  Let $R$ be an arbitrarily chosen equivalence relation.
Define $P_R= \{[x]_R \mid x \in {\tt fld}(R)\}$.

Our goal is to prove that $P_R$ is a partition of ${\tt fld}(R)$.
Using the definition of partition, this reduces to three subgoals.

\begin{description}

\item[Goal 1: $\emptyset \not\in P_R$.] Suppose for the sake of a
contradiction that $\emptyset \in P_R$.  By the definition of $P_R$ as
a complex set abstract, this is equivalent to the assertion that
$\emptyset = [x]_R$ for some $x \in {\tt fld}(R)$.  Choose such an
$x$.  $x\,R\,x$ holds because $R$ is reflexive, whence $x \in [x]_R$
by the definition of equivalence class, whence $x \in \emptyset$,
which yields the desired contradiction.  This completes the proof of
Goal 1.

\item[Goal 2: $\bigcup P_R = {\tt fld}(R)$.]  Use the proof strategy for
showing the equality of two sets.  

\begin{description}
\item[2a:] Let $x$ be an arbitrarily chosen element of $\bigcup P_R$:
our new goal is to show $x \in {\tt fld}(R)$.  Since $x \in \bigcup
P_R$, we can choose a set $A$ such that $x \in A$ and $A \in P_R$.
Since $A \in P_R$, we can choose $y$ such that $A = [y]_R$.  $x \in A
= [y]_R$ implies immediately that $y \,R\,x$, whence $x \in {\tt
fld}(R)$, which completes the proof of goal 2a.

\item[2b:] Let $x$ be an arbitrarily chosen element of ${\tt fld}(R)$:
our new goal is to show that $x \in \bigcup P_R$.  Since $x \in {\tt
fld}(R)$, we may choose a $y$ such that one of $x \,R\,y$ or $y\,R\,x$
is true.  But then both are true because $R$ is symmetric, and we have
$x \in [y]_R$.  From $x \in [y]_R$ and $[y]_R \in P_R$, we deduce $x
\in \bigcup P_R$, completing the proof of goal 2b.

\end{description}

Since any element of either set has been shown to belong to the other,
the two sets are equal, completing the proof of Goal 2.

\item[Goal 3: $P_R$ is pairwise disjoint.]  Our goal is to show that
for any elements $A,B$ of $P_R$ we have $A=B \vee A \cap B =
\emptyset$ To prove this, we assume that $A$ and $B$ are distinct and
take $A \cap B = \emptyset$ as our new goal.  We prove this by
contradiction: assume $A \cap B \neq \emptyset$ and our new goal is a
contradiction.  Since $A \cap B \neq \emptyset$, we may choose an $x
\in A \cap B$.  Since $A, B \in P_R$ we may choose $y$ and $z$ such
that $A = [y]_R$ and $B = [z]_R$.  If we had $y=z$ we would have $A=B$
and a contradiction, so we must have $y \neq z$.  $x \in A \cap B =
[y]_R \cap [z]_R$ implies $x \in [y]_R$ and $x \in [z]_R$, whence we
have $x\,R\,y$ and $x\,R\,z$, whence by symmetry and transitivity of
$R$ we have $y\,R\,z$.  We now prove $A = [y]_R = [z]_R = B$, which
will give the desired contradiction since $A$ and $B$ were initially
supposed distinct.

\begin{description}
\item[3a:] Let $u$ be an arbitrarily chosen element of $[y]_R$.  Our
new goal is $u \in [z]_R$.  $u \in [y]_R$ implies $y\,R\,u$, and
$y\,R\,z$ and symmetry imply $z \,R\,y$.  Thus by transitivity of $R$
we have $z\,R\,u$ and so $u \in [z]_R$.  This completes the proof of
goal 3a.

\item[3b:] Let $u$ be an arbitrarily chosen element of $[z]_R$.  Our
new goal is $u \in [y]_R$.  $u \in [z]_R$ implies $z\,R\,u$, which in
combination with $y\,R\,z$ and transitivity of $R$ implies $y\,R\,u$,
which implies $u \in [y]_R$, which completes the proof of goal 3b.
\end{description}
Since the sets $[y]_R=A$ and $[z]_R = B$ have the same elements, it
follows that they are equal, which completes the proof of a
contradiction, from which Goal 3 and the Theorem follow.



\end{description}

\item[Theorem:] If $\cal P$ is a partition of $A$, the relation
$$\equiv_{\cal P} = \{\left<x,y\right>\mid (\exists B \in {\cal P}.x\in
B \wedge y \in B)\}$$ is an equivalence relation with field $A$.

\item[Proof:]  This is left as an exercise.

\item[Observation:] Further, $\equiv_{P_R} = R$ and $P_{\equiv_{\cal
P}}={\cal P}$ for any $R$ and $\cal P$: there is a precise
correspondence between equivalence relations and partitions.

\end{description}

An equivalence relation $R$ represents a way in which elements of its
field are similar: in some mathematical constructions we wish to {\em
identify\/} objects which are similar in the way indicated by $R$.
One way to do this is to replace references to an $x \in {\tt fld}(R)$
with references to its equivalence class $[x]_R$.  Note that for all $x,y$ in
${\tt fld}(R)$ we have $x\,R\,y$ iff $[x]_R=[y]_R$.

It might be found inconvenient that $[x]_R$ is one type higher than
$x$.  In such a situation, we would like to work with a representative
of each equivalence class.  

\begin{description}

\item[Definition:] Let $P$ be a partition.  A {\em choice set\/} for
$P$ is a set $C$ with the property that $B \cap C$ has exactly one
element for each $B \in P$.

\end{description}

A choice set for the partition $P_R$ will give us exactly one element
of each equivalence class under $R$, which we can then use to
represent all elements of the equivalence class in a context in which
$R$-equivalent objects are to be identified.

In some situations, there is a natural way to choose an element of
each equivalence class (a canonical representative of the class).  We
will see examples of this situation.  In the general situation, we can
invoke the last axiom of our typed theory of sets.

\begin{description}

\item[Axiom of Choice:] If $P$ is a partition (a pairwise disjoint set
of nonempty sets) then there is a choice set $C$ for $P$.\footnote{If we include the Hilbert symbol in our logic and allow its use in comprehension, this axiom is not needed:  a choice set for $P$ is definable as $(\{(\epsilon x:x \in A)\mid A \in P\}$.  This would be a way to slip the Axiom of Choice in while attracting even less attention.}

\end{description}

The Axiom of Choice is a somewhat controversial assertion with
profound consequences in set theory: this seemed like a good place to
slip it in quietly without attracting too much attention.

Here we also add some terminology about partial orders.

It is conventional when working with a particular partial order $\leq$
to use $<$ to denote $[\leq]-[=]$ (the corresponding strict partial
order), $\geq$ to denote $[\leq]^{-1}$ (which is also a partial order)
and $>$ to denote the strict partial order $[\geq]-[=]$.

A minimum of $\leq$ is an element $m$ of ${\tt fld}(\leq)$ such that
$m \leq x$ for all $x \in {\tt fld}(x)$.  A maximum of $\leq$ is a
minimum of $\geq$.  A minimal element with respect to $\leq$ is an
element $m$ such that for no $x$ is $x < m$.  A maximal element with
respect to $\leq$ is a minimal element with respect to $\geq$.  Notice
that a maximum or minimum is always unique if it exists.  A minimum is
always a minimal element.  The converse is true for linear orders but
not for partial orders in general.

For any partial order $\leq$ and $x \in {\tt fld}(\leq)$, we define
${\tt seg}_{\leq}(x)$ as $\{y \mid y < x\}$ (notice the use of the
strict partial order) and $(\leq)_x$ as $[\leq] \cap ({\tt
seg}_{\leq}(x))^2$.  The first set is called the {\em segment\/} in $\leq$
determined by $x$ and the second is called the {\em segment restriction\/}
determined by $x$.

For any subset $A$ of ${\tt fld}(\leq)$, we say that an element $x$ of
${\tt fld}(\leq)$ is a lower bound for $A$ in $\leq$ iff $x \leq a$
for all $a \in A$, and an upper bound for $A$ in $\leq$ iff $a \leq x$
for all $a \in A$.  If there is a lower bound $x$ of $A$ such that for
every lower bound $y$ of $A$, $y \leq x$, we call this the greatest
lower bound of $A$, written $\inf_{\leq}(A)$, and if there is an upper
bound $x$ of $A$ such that for all upper bounds $y$ or $A$, we have $x
\leq y$, we call this the least upper bound of $A$, written
$\sup_{\leq}(A)$.

A special kind of partial order is a {\em tree}: a partial order
$\leq_T$ with field $T$ is a {\em tree\/} iff for each $x \in T$ the
restriction of $\leq_T$ to ${\tt seg}_{\leq_T}(x)$ is a well-ordering.
A subset of $T$ which is maximal in the inclusion order among those
well-ordered by $\leq_T$ is called a {\em branch\/}.

\newpage

\subsection{Exercises}

\begin{enumerate}

\item  
Suppose that $P$ is a partition.  

Prove that the relation $\sim_P$ defined by \newline $x \,\sim_P\, y$ iff
$(\exists A \in P.x \in A \wedge y \in A)$ \newline is an equivalence relation.
What is the field of this equivalence relation?

Describe its equivalence classes.  

This is an exercise in carefully writing everything down, so show all
details of definitions and proof strategy, as far as you can.


\item
This question relies on ordinary knowledge about the reals and the
rationals, and also knowledge of Lebesgue measure if you have studied
this (if you haven't, don't worry about that part of the question).

Verify that the relation on real numbers defined by ``$x\,R\,y$ iff
$x-y$ is rational'' is an equivalence relation.  It might be cleaner to consider the 
relation ``$x\,R\,y$ iff
$x-y$ has a terminating decimal expansion''; the result is similar and the equivalence classes are easier to describe.

Describe the equivalence classes under this relation in general.
Describe two or three specific ones.  Note that each of the
equivalence classes is countably infinite (why? [see the next section if you don't know what ``countably infinite" means]), distinct equivalence
classes are disjoint from each other, and so we ``ought'' to be able
to choose a single element from each class.  

Can you think of a way to do this (you will not be able to find one,
but thinking about why it is difficult is good for you)?

Suppose we had a set $X$ containing exactly one element from each
equivalence class under $R$.  For each rational number $q$, let $X_q$
be the set $\{r+q \mid r \in X\}$.  Note that $X_q$ is just a
translation of $X$.

Prove that $\{X_q \mid q \in {\mathbb Q}\}$ is a partition of $\mathbb
R$.  (This will include a proof that the union of the $X_q$'s is the
entire real line).

If you know anything about Lebesgue measure, you might be able to
prove at this point that $X$ is not Lebesgue measurable (if you can,
do so).  It is useful to note that the collection of $X_q$'s is
countable.


\end{enumerate}

\newpage

\section{Cardinal Number and Arithmetic}

We say that two sets are the same size iff there is a one-to-one
correspondence (a bijection) between them.

\begin{description}

\item[Definition:] We say that sets $A$ and $B$ are {\em
equinumerous\/} and write $A \sim B$ just in case there is a bijection
$f$ from $A$ onto $B$.

\item[Theorem:]  Equinumerousness is an equivalence relation.

\item[Indication of Proof:] It is reflexive because the identity
function on any set is a function.  It is symmetric because the
inverse of a bijection is a bijection.  It is transitive because the
composition of two bijections is a bijection.

\item[Definition:] For any set $A$, we define $|A|$, the {\em cardinality
of $A$\/}, as $[A]_{\sim} = \{B \mid B \sim A\}$.  Notice that $|A|$ is
one type higher than $A$.  We define ${\tt Card}$, the set of all {\em cardinal numbers\/}, as $\{|A| \mid A \in V\}$.

\end{description}

The same definitions would work if we were using the Kuratowski pair,
and in fact the cardinals would be precisely the same sets.

We have already encountered some cardinal numbers.

\begin{description}

\item[Theorem:]  Each natural number is a cardinal number.

\item[Proof:]  $|\emptyset| = \{\emptyset\}=0$ is obvious:  there is a bijection from $\emptyset$ to $A$ iff $A = \emptyset$.

 Suppose that $n \in {\mathbb N}$ is a cardinal number: show that
$n+1$ is a cardinal number and we have completed the proof that all
natural numbers are cardinals by mathematical induction.  Let $x$ be
an element of $n$.  There is a $y \not\in x$ because $x \neq V$ (by
the Axiom of Infinity).  It suffices to show $n+1 = |x \cup \{y\}|$.
To show this, we need to show that for any set $z$, $z \in n+1$ iff $z
\sim x\cup \{y\}$.  If $z \in n+1$ then $z = v \cup \{w\}$ for some $v
\in n$ and some $w \not\in v$.  Because $n$ is a cardinal number there
is a bijection $f$ from $x$ to $v$: $f \cup \{\left<y,w\right>\}$ is
readily seen to still be a bijection.  Now let $z$ be an arbitrarily
chosen set such that $z \sim x \cup \{y\}$.  This is witnessed by a
bijection $f$.  Now $f^{-1}``x$ belongs to $n$ because $n$ is a
cardinal number, and thus we see that $v = f^{-1}``x \cup
\{f^{-1}(y)\}$ belongs to $n+1$ (certainly $f^{-1}(y) \not\in f^{-1}``x$), completing the proof.

\end{description}

There is at least one cardinal number which is not a natural number.

\begin{description}

\item[Definition:] We define $\aleph_0$ as $|{\mathbb N}|$.  Sets of
this cardinality are said to be {\em countable\/} or {\em countably
infinite\/}.  Infinite sets not of this cardinality (if there are any)
are said to be {\em uncountable\/} or {\em uncountably infinite\/}.

\end{description}

We provide some lemmas for construction of bijections from other bijections.

\begin{description}

\item[Lemma:] The union of two relations is of course a relation.  The
union of two functions is a function iff the functions agree on the
intersection of their domains: that is, if $f$ and $g$ are functions,
$f \cup g$ is a function iff for every $x \in {\tt dom}(f) \cap {\tt
dom}(g)$ we have $f(x)=g(x)$, or, equivalently but more succinctly, $f
\lceil {\tt dom}(f) \cap {\tt dom}(g) = g \lceil {\tt dom}(f) \cap
{\tt dom}(g)$.  Note that it is sufficient for the domains of $f$ and
$g$ to be disjoint.

\item[Definition:] A function $f$ is said to {\em cohere\/} with a
function $g$ iff $f \lceil( {\tt dom}(f) \cap {\tt dom}(g)) = g \lceil
({\tt dom}(f) \cap {\tt dom}(g))$.

\item[Lemma:] The union of two injective functions $f$ and $g$ is an
injective function iff $f$ coheres with $g$ and $f^{-1}$ coheres with
$g^{-1}$.  Note that it is sufficient for the domain of $f$ to be
disjoint from the domain of $g$ and the range of $f$ disjoint from
the range of $g$.

\item[Lemma:] For any $x$ and $y$, $\{\left<x,y\right>\}$ is an
injection.

\end{description}

Arithmetic operations have natural definitions.

A cardinal $|A|$ is the collection of all sets of the same size as
$A$.  Thus, if $\kappa$ is a cardinal, we mean by ``set of size
$\kappa$'' simply an element of $\kappa$.  This is not true of all
representations of cardinality: if we used a representative set the
same size as $A$ as $|A|$, for example, then a set of size $\kappa$
would be a set equinumerous with $\kappa$ (the representation used in
the usual set theory introduced later is of this latter kind).

We define addition of cardinals.  Informally, a set of size
$\kappa+\lambda$ will be the union of two disjoint sets, one of size
$\kappa$ and one of size $\lambda$.

\begin{description}

\item[Definition (abstract definition of addition):] If $\kappa$ and $\lambda$ are
cardinals, we define $\kappa+\lambda$ as $$\{A \cup B \mid A \in
\kappa \wedge B \in \lambda \wedge A \cap B = \emptyset\}.$$

\end{description}

Notice that this agrees (for cardinal numbers) with the abstract definition of addition of arbitrary sets given in the alternative definition in the initial definition of natural numbers.

There are some things to verify about this definition.  One has to
verify that $\kappa+\lambda$ is nonempty.  If $A \in \kappa$ and $B
\in \lambda$ then $A \times \{0\} \in \kappa$, $B \times \{1\} \in
\lambda$, and these sets are obviously disjoint.  The fact that
cartesian product is a type level operation is crucial here (so
Infinity is required).  One has to verify that $\kappa+\lambda$ is a cardinal.

\begin{description}

\item[Observation:]  $|A| + |B| = |(A \times \{0\}) \cup (B \times \{1\})|$.

\item[Proof:] Suppose $A'$ and $B'$ are disjoint sets with bijections
$f:A \rightarrow A'$ and $g:B\rightarrow B'$.  Then $(\pi_1|f) \cup
(\pi_1|g)$ is a bijection from $(A \times \{0\}) \cup (B \times
\{1\})$ to $A' \cup B'$.  The union of these two injections is an
injection because they have disjoint domains and disjoint ranges, and
the union has the correct domain and range.

\end{description}

It is perhaps preferable to simply take the Observation as the

\begin{description}

\item[Definition (concrete definition of addition):] $|A| + |B|$ is defined as $$|(A
\times \{0\}) \cup (B \times \{1\})|.$$  (It is straightforward to show
that this does not depend on the choice of representatives $A$ and $B$
from the cardinals).

\end{description}

The abstract definition of addition would work if we were using
Kuratowski pairs but the proof that addition is total would be
somewhat harder.  The Observation would be incorrect and in fact would
not make sense because it would not be well-typed.

Notice that the definition of $\kappa+1$ as an addition of cardinals
agrees with the definition of $\kappa+1$ as a set already given in the
development of finite number.

Before discussing multiplication, we consider the notion of being the
same size appropriate to sets at different types.

\begin{description}

\item[Definition (alternative notation for singleton set):] We recall that we defined
$\iota(x)$ as $\{x\}$.  The point of this notation is that it is
iterable: we can use $\iota^{\bf n}(x)$ to denote the $n$-fold
singleton of $x$.  [But do notice that this is not an example of
iteration as $\iota$ is not a function (a function does not raise
type).  The $n$ in $\iota^{\bf n}(x)$ is a purely formal bit of
notation (like a type index) and not a reference to any natural number
in our theory, and this is why it is in boldface]

\item[Definition (singleton image operations):] We define $\iota^{\bf n}``x$, the {\em $n$-fold singleton
image of $x$\/} as $\{\iota^{\bf n}(y)\mid y \in x\}$.  For any relation
$R$, we define $R^{\iota^{\bf n}}$ as
$\{\left<\iota^{\bf n}(x),\iota^{\bf n}(y)\right>\mid x\,R\,y\}$.  We define
$T^{\bf n}(\kappa)$ for any cardinal $\kappa$ as $|\iota^{\bf n}``A|$ for any $A
\in \kappa$.  Note that $A \sim B \leftrightarrow \iota``A \sim
\iota``B$ is obvious: if $f$ is a bijection from $A$ to $B$, then
$f^{\iota}$ will be a bijection from $\iota``A$ to $\iota``B$.  We
define $T^{\bf -n}(\kappa)$ as the unique cardinal $\lambda$ (if there is
one) such that $T^{\bf n}(\lambda)=\kappa$.

\item[Definition (sole element):] We define $\iota^{\bf -1}(\{x\})$ as $x$.  We define
$\iota^{\bf -1}(A)$ as $\emptyset$ if $A$ is not a singleton.
$\iota^{\bf -n}(\iota^{\bf n}(x))$ will be defined as $x$ as one might expect,
if this notation is ever needed.  

\end{description}

The singleton map (or iterated singleton map) is in a suitable
external sense injective, so a set equinumerous with $\iota^{\bf n}``A$,
though it is $n$ types higher than $A$, is in a recognizable sense the
same size as $A$.

The definition of $T^{\bf -n}$ depends on the observation that $T^{\bf n}$ is 
 ``injective" in the sense that $T^n(\mu)=T^n(\nu)\rightarrow \mu=\nu$ for any cardinals $\mu,\nu$ (the scare quotes are needed because $T^n$ cannot actually be viewed as an injection, since it is not a function at all, due to its effect on types) , so if there is a suitable $\lambda$ there is only one.  We
leave open the possibility that $T^{\bf -n}(\kappa)$ is undefined for some
cardinals $\kappa$ and indeed this turns out to be the case.

We discuss the application of the $T$ operation to natural numbers.

\begin{description}

\item[Theorem:]  $T(0)=0$ and $T(n+1) = T(n)+1$.

\item[Corollary:]  $T(1)=1; T(2)=2; T(3)=3\ldots$  But we cannot say
$$(\forall n \in {\mathbb N}.T(n)=n),$$ because this is ungrammatical.

\item[Theorem:] For all natural numbers $n$, $T(n)$ is a natural
number.  For all natural numbers $n$ [not of the lowest possible type]
$T^{\bf -1}(n)$ exists and is a natural number.

\item[Proof:]  We prove both parts by induction, of course.

Our first goal is to prove that $T(n)$ is a natural number for every
natural number $n$.  We observe first that $T(0)=0$ is obvious, as
$\iota``\emptyset = \emptyset$.  Now suppose that $k$ is a natural
number and $T(k)$ is a natural number.  Our aim is to prove that
$T(k+1)$ is a natural number.  Each element of $k+1$ is of the form $A
\cup \{x\}$ where $A \in k$ and $x \not\in A$.  $T(k+1) = |\iota``(A
\cup \{x\})|$.  But $\iota``(A \cup \{x\}) = \iota``A \cup \{\{x\}\}$.
Obviously $\iota``A \in T(k)$ and $\{x\} \not\in \iota``A$, so
$\iota``A \cup \{\{x\}\} \in T(k)+1\in {\mathbb N}$, so $T(k+1) =
T(k)+1 \in {\mathbb N}$.

Our second goal is to prove that $T^{\bf -1}(n)$ exists and is a natural
number for each natural number $n$ (not of the lowest possible type).
Since $T(0)=0$, we also have $T^{\bf -1}(0)=0$, so $T^{\bf -1}(0)$ exists and
is a natural number.  Let $k$ be a natural number such that there is a
natural number $l$ such that $T(l)=k$ (which is equivalent to saying
that $T^{\bf -1}(k)$ exists and is a natural number).  Choose a set $A$ of
cardinality $l$.  Choose $x \not\in A$.  $|A \cup \{x\}| = l+1$ and
$|\iota``(A \cup \{x\})| = |\iota``A \cup \{\{x\}\}| = k+1$ is
obvious, so $T(l+1) = k+1$, whence $T^{\bf -1}(k+1)$ exists and is a
natural number as desired.

\item[Reasonable Convention:] It is reasonable to simply identify the
natural numbers at different types and there is a way to make sense of
this in our notation: allow a natural number variable $n$ of type $k$
to appear at other types with the understanding that where it appears
in a position appropriate for a variable of type $k+i$ it is actually
to be read as $T^i(n)$.  We will not do this, or at least we will
explicitly note use of this convention if we do use it, but it is
useful to note that it is possible.

\item[Rosser's Counting Theorem:] $\{1,\ldots,n\} \in T^{\bf 2}(n)$, for
each positive natural number $n$.

\item[Discussion and Proof:] Of course $\{1,\ldots,n\} = \{m \in
{\mathbb N}\mid 1 \leq m \leq n\}$ has $n$ members, if $n$ is a
concrete natural number.  But the second $n$ we mention is two types
higher than the first one: we fix this by affixing $T^{\bf 2}$ to the second
one, so that both occurrences of $n$ have the same type.

What this actually says is that if we have a set $A$ belonging to a
natural number $n$, we can put $\iota^{\bf 2}``A$ (the set of double
singletons of elements of $A$) into one-to-one correspondence with the
set of natural numbers $\{1,\ldots,n\}$ of the type appropriate for $A
\in n$ to make sense.  This can be proved by induction on the number
of elements in $A$.  If $A$ has one element $a$, clearly there is a
bijection between $\{\{\{a\}\}\}$ and $\{1\}$ (all that needs to be
checked is that these objects are of the same type: the number 1 being
considered satisfies $A \in 1$).  Suppose that for all $A \in n$,
$\iota^{\bf 2}``A \sim \{1,\ldots,n\}$.  We want to show that for any $B \in
n+1$, $\iota``B \sim \{1,\ldots,n+1\}$.  $B = A \cup \{x\}$ for some
$A \in n, x\not\in A$.  There is a bijection $f$ from $\iota^{\bf 2}``A$ to
$\{1,\ldots,n\}$ by inductive hypothesis.  $f \cup \left<\{\{x\}\},n+1\right>$
is easily seen to witness the desired equivalence in size.

\item[Von Neumann's Counting Theorem:] For any natural number $n$,
$$\{m \in {\mathbb N}\mid m<n\} \in T^{\bf 2}(n).$$

\item[Discussion:] This is true for the same reasons.  It is not
really a theorem of von Neumann, but it relates to his representation
of the natural numbers.

\end{description}

Notice that these counting theorems could be written in entirely
unexciting forms if we adopted the Reasonable Convention above.  It
would then be the responsibility of the reader to spot the type
difference and insert the appropriate T operation.  This would have to
be done in order to {\em prove\/} either of these statements.

A fully abstract definition of multiplication would say that
$\kappa\cdot\lambda$ is the size of the union of $\kappa$ disjoint
sets each of size $\lambda$.  To state this precisely requires the $T$
operation just introduced.

\begin{description}

\item[$^*$Definition (abstract definition of multiplication):] $\kappa\cdot\lambda$ is
the uniquely determined cardinal of a set $\bigcup C$ where $C$ is
pairwise disjoint, $C \in T(\kappa)$, and $C \subseteq \lambda$.

\end{description}

The details of making this definition work are quite laborious.
Infinity is required to show that there are such sets for any $\kappa$
and $\lambda$, and Choice is required to show that the cardinal is
uniquely determined.  We regretfully eschew this definition and use a more concrete definition employing the cartesian product:

\begin{description}

\item[Definition (concrete definition of multiplication):] $|A| \cdot |B|$ is
defined as $|A \times B|$.  It is straightforward to show that this
does not depend on the choice of representatives $A, B$ from the
cardinals.

\end{description}

If we were using the Kuratowski pair we would define $$|A| \cdot |B| =
T^{\bf -2}(|A \times B|).$$ It would be harder to show that multiplication
is total.  We would also have $$|A| + |B| = T^{\bf -2}(|(A \times \{0\})
\cup (B \times \{1\})|)$$  if we were using the Kuratowski pair.

The $T$ operation commutes with arithmetic operations:  

\begin{description}

\item[Theorem:] For all cardinal numbers $\kappa$ and $\lambda$,
$T(\kappa)+T(\lambda)=T(\kappa+\lambda)$ and $T(\kappa\cdot \lambda) =
T(\kappa)\cdot T(\lambda)$.

\end{description}

Theorems of cardinal arithmetic familiar from the theory of natural
numbers (and from ordinary experience) have much more natural proofs
in set theory than the inductive proofs given in Peano arithmetic.

\begin{description}

\item[Theorem:] The following identities are true for all cardinal
numbers $\kappa,\lambda,\mu$ (including natural numbers).

\begin{enumerate}

\item $\kappa+0 = \kappa;\kappa\cdot 1 = \kappa$

\item $\kappa\cdot 0 = 0$

\item $\kappa+\lambda = \lambda+\kappa; \kappa\cdot\lambda = \lambda\cdot\kappa$

\item $(\kappa+\lambda)+\mu = \kappa+(\lambda+\mu); (\kappa\cdot\lambda)\cdot\mu=\kappa\cdot(\lambda\cdot\mu)$

\item $\kappa\cdot(\lambda+\mu)=\kappa\cdot\lambda+\kappa\cdot\mu$

\end{enumerate}

\end{description}

All of these admit very natural proofs.

\begin{description}

\item[Sample Proofs:]

\begin{description}
\item
\item[commutativity of multiplication:]

Let $\kappa,\lambda$ be cardinal numbers.  Choose sets $A$ and $B$
such that $\kappa=|A|$ and $\lambda=|B|$.  $\kappa\cdot\lambda =
|A\times B|$ and $\lambda\cdot\kappa = |B \times A|$; what remains is
to show that there is a bijection from $|A \times B|$ to $|B \times
A|$.  The map which sends each ordered pair $\left<a,b\right>$ (for $a
\in A,b\in B$) to $\left<b,a\right>$ does the trick.

\item[associativity of addition:] Let $\kappa,\lambda,\mu$ be cardinal
numbers.  Choose $A,B,C$ such that $\kappa =|A|, \lambda=|B|, \mu =
|C|$.  $(|A|+|B|)+|C| = |A\times \{0\} \cup B \times\{1\}| + |C| =
|(A\times \{0\} \cup B \times \{1\})\times \{0\} \cup C \times \{1\}| =
|\{\left<\left<a,0\right>,0\right>\mid a \in A\} \cup \{\left<\left<b,1\right>,0\right> \mid b \in B\} \cup \{\left<c,1\right>\mid c\in C\}|$

Similarly $|A|+(|B|+|C|) = |\{\left<a,0\right>\mid a \in A\} \cup
\{\left<\left<b,0\right>,1\right>\mid b \in B\} \cup
\{\left<\left<c,1\right>,1\right>\mid c \in C\}|$.

A bijection from $\{\left<\left<a,0\right>,0\right>\mid a \in A\} \cup
\{\left<\left<b,1\right>,0\right> \mid b \in B\} \cup
\{\left<c,1\right>\mid c\in C\}$ to $\{\left<a,0\right>\mid a \in A\}
\cup \{\left<\left<b,0\right>,1\right>\mid b \in B\} \cup
\{\left<\left<c,1\right>,1\right>\mid c \in C\}$ is provided by the
union of the map sending each $\left<\left<a,0\right>,0\right>$ to
$\left<a,0\right>$, the map sending each
$\left<\left<b,1\right>,0\right>$ to $\left<\left<b,0\right>,1\right>$
and the map sending each $\left<c,1\right>$ to
$\left<\left<c,1\right>,1\right>$.  Each of these maps is a bijection,
they have disjoint domains and disjoint ranges, so their union is
still a bijection. The existence of this bijection witnesses the
desired equation.

\end{description}

\end{description}

Important arithmetic properties of the natural numbers {\em not\/}
shared by general cardinals are the {\em cancellation properties\/}.
It is not true in general that $\kappa+\mu = \lambda+\mu \leftrightarrow
\kappa=\lambda$, nor that $\kappa\cdot\mu = \lambda\cdot\mu \wedge \mu
\neq 0 \rightarrow \kappa=\lambda$.  This means that we do not get
sensible notions of subtraction or division.

But the following is a

\begin{description}

\item[Theorem:] For any cardinals $\kappa,\lambda$ and any natural
number $n$, $\kappa+n = \lambda+n \rightarrow \kappa=\lambda$.

\item[Proof:] It suffices to prove $\kappa+1 = \lambda+1 \rightarrow
\kappa=\lambda$: the result then follows by induction.

Suppose $\kappa+1=\lambda+1$.  Let $A$ and $B$ be chosen so that
$\kappa=|A|$, $\lambda=|B|$, and neither $A$ nor $B$ is the universal
set $V$.  Note that if either $A$ or $B$ were the universal set, we
could replace it with $V \times\{0\} \sim V$.  Choose $x \not\in A, y
\not\in B$.  We have $|A \cup \{x\}| = \kappa+1 = \lambda+1 = |B \cup
\{y\}|$.  This means we can choose a bijection $f:(A \cup \{x\})
\rightarrow (B \cup \{y\})$.  Either $f(x)=y$ or $f(x) \neq y$.  If
$f(x)=y$, then $f\lceil A$ is the desired bijection from $A$ to $B$,
witnessing $\kappa=\lambda$.  If $f(x)\neq y$, then $f -
\{\left<x,f(x)\right>\} - \{\left<f^{-1}(y),y\right>\} \cup
\{\left<f^{-1}(y),f(x)\right>\}$ is the desired bijection from $A$ to
$B$ witnessing $\kappa=\lambda$.  In either case we have established
the desired conclusion.

\end{description}

\newpage

\subsection{Exercises}
\begin{enumerate}
\item
Prove that $|{\mathbb N}|+1 = |{\mathbb N}|+|{\mathbb N}| = |{\mathbb
N}|\cdot|{\mathbb N}| = |{\mathbb N}|$.

Describe bijections by arithmetic formulas where you can; in any case
clearly describe how to construct them (these are all familiar
results, or should be, and all of the bijections can in fact be
described algebraically: the formula for triangular numbers can be
handy for this).  I'm looking for bijections with domain ${\mathbb N}$
and range some more complicated set in every case.
\item
Verify the distributive law of multiplication over addition in
cardinal arithmetic, $$|A|\cdot(|B| +|C|) = |A|\cdot|B| +
|A|\cdot|C|,$$ by writing out explicit sets with the two cardinalities
(fun with cartesian products and labelled disjoint unions!) and
explicitly describing the bijection sending one set to the other.  You
do not need to prove that it is a bijection: just describe the sets
and the bijection between them precisely.
\item Prove that $|\left<A,B\right>| = |A|+|B|$ if the pair is taken
to be a Quine pair.

\item
Explain why the relation $A \sim B$ of equinumerousness (equipotence,
being the same size) is an equivalence relation by citing basic
properties of bijections.

The
structure of your proof should make it clear that you understand what
an equivalence relation is. 

You do not need to prove the basic
properties of bijections that are needed; you need only state them.

Your proof should also make it clear that you know what $A \sim B$
means.

What are the equivalence classes under the relation $\sim$ called in
type theory?
\item
In this problem you will indicate a proof of the associative property
of multiplication for cardinal numbers.  

Recall that $|A|\cdot|B|$ is defined as $|A \times B|$.

The goal is to prove that $(|A|\cdot|B|)\cdot|C| =
|A|\cdot(|B|\cdot|C|)$.  Describe sets of these cardinalities and
(carefully) describe a bijection between them.  You do not need to
prove that the map is a bijection.

\item This may be so easy that it is hard.

Prove the lemma stated in the text without proof that for functions $f$ and $g$, $f \cup g$ is a function iff $f \lceil ({\tt dom}(f) \cap {\tt dom}(g)) = g \lceil ({\tt dom}(f) \cap {\tt dom}(g))$.

This looks to me worth doing:  it is an exercise in carefully unpacking definitions and keeping track of what it is that you need to show.

\item Prove that for $m,n$ natural numbers, the definitions of $m+n$ and $m \cdot n$ given here coincide with the definitions based on iteration given earlier.  These are induction proofs, of course.


\end{enumerate}

\newpage

\section{Number Systems}

In this section we give a development of the system of real numbers
from the typed theory of sets.  Part of the point is that this
development is not unique or canonical in any way: we indicate how
alternative developments might go.  The development is full in the
sense that all definitions of mathematical structures are given.  Not
all theorems are proved, though important ones are stated.

We begin with the system ${\mathbb N}^+$ of all nonzero natural
numbers.  We have already defined arithmetic operations of addition
and multiplication on the natural numbers, and it is easy to see that
${\mathbb N}^+$ is closed under these operations.

We now give a construction of the system ${\mathbb Q}^+$ of {\em
fractions\/} (positive rational numbers).  

\begin{description}

\item[Definition:] For $m,n \in {\mathbb N}^+$, we define $m|n$ as
$(\exists x \in {\mathbb N}^+.m\cdot x = n)$.  This is read ``$n$ is
divisible by $m$'' and we say that $m$ is a {\em factor\/} of $n$.

\item[Definition:] For $m,n \in {\mathbb N}^+$, we define ${\tt
gcd}(m,n)$ as the largest natural number $x$ which is a factor of $m$
and a factor of $n$.  If ${\tt gcd}(m,n)=1$, we say that $m$ and $n$
are {\em relatively prime\/}.

\item[Theorem:] If $m\cdot x = m\cdot y$, then $x=y$, where $m,x,y \in
{\mathbb N}^+$.

\item[Definition:] If $m\cdot x = n$, we define $\frac nm$ as $x$
(this is uniquely determined, if defined, by the previous theorem).
Note that this notation will be superseded after the following
definition.

\item[Definition:] We define a {\em fraction\/} as an ordered pair
$\left<m,n\right>$ of nonzero natural numbers such that $m$ and $n$
are relatively prime.  For any ordered pair $\left<m,n\right>$ of
nonzero natural numbers, we define ${\tt simplify}(m,n)$ as
$\left<\frac m{{\tt gcd}(m,n)},\frac n{{\tt gcd}(m,n)}\right>$.  Note
that ${\tt simplify}(m,n)$ is a fraction.  After this point, we use
the notation $\frac mn$ to denote ${\tt simplify}(m,n)$.

\item[Observation:] It is more usual to define an equivalence relation
$\left<m,n\right> \sim \left<p,q\right>$ on ordered pairs of nonzero
natural numbers (usually actually ordered pairs of integers with
nonzero second projection) as holding when $mq=np$ (a proof that this
is an equivalence relation is needed) then define fractions (more
usually general rationals) as equivalence classes under this relation.
The construction given here uses canonical representatives instead of
equivalence classes.

\item[Definition:] We define $\frac mn + \frac pq$ as $\frac
{mq+np}{pq}$ and $\frac mn \cdot \frac pq = \frac{mp}{nq}$.  We define
$\frac mn \leq \frac pq$ as holding iff $mq \leq np$.  We leave it to
the reader to prove that these definitions are valid (do not depend on
the choice of representation for the fractions), that $\leq$ is a
linear order, and that addition and multiplication of fractions have
expected properties.  The complete familiarity of these definitions
may obscure the fact that work needs to be done here.

\end{description}

Now we proceed to define the system of {\em magnitudes\/} (positive
real numbers).

\begin{description}

\item[Definition:] A {\em magnitude\/} is a set $m$ of fractions with
the following properties.

\begin{enumerate}

\item $m$ and ${\mathbb Q}^+ - m$ are nonempty.

\item $(\forall pq \in {\mathbb Q}^+.p \in m \wedge q \leq p
\rightarrow q \in m)$: $m$ is downward closed.

\item $(\forall p \in m.(\exists q \in m.p \leq q))$: $m$ has no
largest element.

\end{enumerate}

\end{description}

The motivation here is that for any positive real number $r$ (as
usually understood prior to set theory) the intersection of the
interval $(0,r)$ with the set of positive rationals uniquely
determines $r$ (and of course is uniquely determined by $r$) and any
set of positive rationals $m$ with the properties given above will
turn out to be the intersection of the set of positive rationals and
$(0,\sup m)$.

\begin{description}

\item[Definition:] For magnitudes $m$ and $n$, we define $m+n$ as
$$\{p+q \mid p \in m \wedge q \in n\}$$ and $m\cdot n$ as
$$\{p\cdot q \mid p \in m \wedge q \in n\}.$$  We define $m \leq
n$ as $m \subseteq n$.  We leave it to the reader to prove that
addition and multiplication of magnitudes always yield magnitudes and
that these operations and the order relation have the expected
properties.

\end{description}

This is where the payoff of our particular approach is found.  It is
more usual to use intersections of intervals $(-\infty,r)$ with all
the rationals (positive, negative and zero) to represent the reals; with this representation of reals
the definition of multiplication is horrible.

We cite a

\begin{description}

\item[Theorem:]  If $m+x = m+y$ then $x=y$, for magnitudes $m,x,y$.

\item[Definition:] If $m+x=n$, we define $n-m$ as $x$ (uniqueness of
$n-m$ if it exists follows from the previous theorem).  This definition
will be superseded by the following definition.

\item[Definition:] We define a {\em real number\/} as an ordered pair
of magnitudes one of which is equal to 1 (where the magnitude 1 is the
set of all fractions less than the fraction 1 = $\frac 11$).  For any
pair of magnitudes $\left<x,y\right>$, we define ${\tt simp}(x,y)$ as
$\left<(x+1)-\min(x,y),(y+1)-\min(x,y)\right>$.  Notice that ${\tt
simp}(x,y)$ will be a real number.  Denote ${\tt simp}(x,y)$ by $x-y$
(superseding the previous definition).

\item[Definition:] We define $(x-y)+(u-v)$ as $(x+u)-(y+v)$.  We
define $(x-y)\cdot(u-v)$ as $(xu+yv)-(xv+yu)$.  We define $x-y \leq
u-v$ as holding precisely when $x+v \leq y+u$.  We leave it to the
reader to establish that everything here is independent of the
specific representation of $x-y$ and $u-v$ used, and that the
operations and the order relation have expected properties.

\end{description}

A considerable amount of overloading is found here.  Addition,
multiplication and order are already defined for nonzero natural
numbers when we start.  In each system, addition, multiplication, and
order are defined: these are different operations and relations in
each system.  Names of nonzero natural numbers, fractions, and
magnitudes are also overloaded: the natural number $n$ is confused
with the fraction $\frac n1$ but it is not the same object, and
similarly the magnitude $\{q \in {\mathbb Q}^+\mid q<p\}$ is not the
same object as the fraction $p$ (and is one type higher than $p$!), and the real number $(m+1)-1$ is not
the same object as the magnitude $m$, though in each case we
systematically confuse them.

Certain important subsystems do not have a place in our development
though they do in more usual developments.

\begin{description}

\item[Definition:] We define the real number 0 as $1-1$.  For each
real number $r=x-y$ we define $-r$ as $y-x$.  We define $r-s$ as
$r+(-s)$ for reals $r$ and $s$.

\item[Definition:] We define the set of {\em integers\/} ${\mathbb Z}$
as the union of the set of all (real numbers identified with) nonzero
naturals, \{0\}, and the set of all additive inverses $-n$ of (real
numbers identified with) nonzero naturals $n$.

\item[Definition:] We define the set of {\em rationals\/} $\mathbb Q$
as the union of the set of all (real numbers identified with)
fractions $p$, \{0\}, and the set of all additive inverses $-p$ of
(real numbers identified with) fractions $p$.

\item[Definition:] For any fraction $q = \frac mn$ we define $q^{-1}$
as $\frac nm$.  For any magnitude $m$, we define $m^{-1}$ as $\{q^{-1}
\mid q \not\in m\}$.  It is straightforward to prove that $m^{-1}$ is
a magnitude and $m \cdot m^{-1} = 1$ for each $m$.  Now define the
reciprocal operation for reals: $((m+1)-1)^{-1} = (m^{-1}+1)-1$ and
$(1-(m+1))^{-1} = 1-(m^{-1}+1)$ for each magnitude $m$, while
$(1-1)^{-1}$ is undefined.  It can be proved that $r \cdot r^{-1}=1$
for each real $r \neq 0$.  Finally, we define $\frac rs$ as $r \cdot
s^{-1}$ for any real $r$ and nonzero real $s$.

\end{description}

We noted above that we have avoided the use of equivalence classes of
ordered pairs at the steps passing to fractions and to signed real
numbers, preferring to use canonical representatives.  Simplification
of fractions is of course a familiar mathematical idea; the canonical
representation of reals we use is less obvious but works just as well.

In this development we have followed the prejudices of the ancient
Greeks as far as possible, delaying the introduction of zero or
negative quantities to the last step.

\newpage

The reals as defined here satisfy the following familiar axioms of a
``complete ordered field''.  Up to a suitable notion of isomorphism,
the reals are the only complete ordered field.

\begin{description}

\item[commutative laws:]  $a+b=b+a$; $a\cdot b = b \cdot a$.

\item[associative laws:] $(a+b)+c = a+(b+c); (a\cdot b)\cdot c = a\cdot (b\cdot c)$.

\item[distributive law:] $a\cdot (b+c) = a\cdot b + a\cdot c$.

\item[identity laws:] $a+0 = a; a\cdot 1 = a$.

\item[inverse laws:] $a+(-a)=0; a\cdot a^{-1} = 1$ if $a \neq 0$.

\item[nontriviality:] $0 \neq 1$

\item[closure of positive numbers:] If $a\geq 0$ and $b \geq 0$ then
$a+b \geq 0$ and $a \cdot b \geq 0$. [note that $a \geq 0$ is a
primitive notion at this point in the development: the reals of the
form $r=(m+1)-1$ are the ones for which we assert $r \geq 0$].

\item[trichotomy:] For each real number $a$, exactly one of the
following is true: $a \geq 0$, $a=0$, $-a \geq 0$.

\item[Definition:] $a\leq b$ iff $b+(-a) \geq 0$.

\item[Theorem:] $\leq$ thus defined is a linear order.

\item[completeness:] Any nonempty set of reals which is bounded above
(in terms of the order just defined) has a least upper bound.

\end{description}

\newpage

\subsection{Exercises}

\begin{enumerate}

\item  Show as many of the properties of the real number system stated as the end of the section true (prove them) or false (exhibit a counterexample) as you can
for

\begin{enumerate}

\item the system of fractions

\item the system of magnitudes

\item the integers

\item the rational numbers

\item the entire system of reals

\end{enumerate}

This is an altogether unreasonable question!

\end{enumerate}

\newpage


\section{Well-Orderings and  Ordinal Numbers}

We recall that a well-ordering is a linear order with the property
that the corresponding strict partial order is well-founded.

\begin{description}

\item[Definition:] A {\em well-ordering\/} is a linear
order $\leq$ with the property that for each nonempty subset $A$ of
${\tt fld}(\leq)$ there is $a \in A$ such that there is no $b\neq a$
in $A$ such that $b\leq a$: such an $a$ is a minimal element of $A$
(in fact, the minimal element is unique because $\leq$ is linear).

\end{description}

In this section, we study the structure of well-orderings.  In this
section we state and prove powerful and highly abstract theorems: for
some concrete discussion of ordinal numbers, look toward the end of
the next section.

\begin{description}

\item[Definition:] Two relations $R$ and $S$ are said to be {\em
isomorphic\/} iff there is a bijection $f$ from ${\tt fld}(R)$ to
${\tt fld}(S)$ such that for all $x,y$, $x\,R\,y \leftrightarrow
f(x)\,S\,f(y)$.  $f$ is said to be an {\em isomorphism\/} from $R$ to
$S$.  We write $R \approx S$ for ``$R$ is isomorphic to $S$''.

\item[Theorem:]  Isomorphism is an equivalence relation on relations.

\item[Definition:] An equivalence class under isomorphism is called an
{\em isomorphism type\/}.

\item[Definition:]  Well-orderings are said to be {\em similar} iff they are isomorphic.

\item[Theorem:]  A relation isomorphic to a well-ordering is a well-ordering.

\item[Definition:] The isomorphism type of a well-ordering is called
its {\em order type\/}.  We write ${\tt ot}(\leq)$ for the order type
$[\leq]_{\approx}$ of $\leq$.  A set is an {\em ordinal number\/} iff
it is the order type of some well-ordering.  The set of all ordinal
numbers is called {\tt Ord\/}.
\end{description}
There are few well-orderings familiar to us from undergraduate mathematics.
Any finite linear order is a well-ordering.

\begin{description}

\item[Theorem:] For any $n \in {\mathbb N}$, any two linear orders
with field of size $n$ are isomorphic and are well-orderings.

\item[Theorem:] A well-ordering is finite iff its converse is also a
well-ordering.

Our use of ``finite'' in the previous theorem might cause confusion,
which will be alleviated by considering the following

\item[Lemma:] A relation (considered as a set) is finite iff its field
is finite.

\item[Definition (finite ordinals):] For each natural number $n$, there is a unique
ordinal number which is the order type of all orders with range of
that cardinality: we also write this ordinal number as $n$, though it
is not the same object as the cardinal number $n$.

\end{description}

An amusing observation, depending crucially on the exact details of
our implementation, is the following relationship between ordinal and
cardinal numbers.
\begin{description}
\item[Theorem:] The ordinal number $n$ is a subset of the cardinal
number $\frac{n(n-1)}2$.
\end{description}

In the usual untyped set theory, with the usual implementations of the
notions of ordinal and cardinal number, the finite cardinals and the
finite ordinals are the same objects.  We will see this in chapter 4.

The usual order on the natural numbers is a well-ordering.  The usual
orders on the integers, rationals and reals are {\em not\/}
well-orderings.  Another example of an infinite well-ordering which is
familiar from calculus is the order on reals restricted to the range
of a strictly increasing bounded convergent  sequence taken together with its
limit.


\begin{description}
\item[Definition:] We define $\omega$ as the order type of the natural
order on the natural numbers.
\end{description}

We give some basic definitions for arithmetic of ordinal numbers.

\begin{description}

\item[Definition (ordinal addition):] For well-orderings $R$ and $S$,
we define another well-ordering $R \oplus S$.  The field of $R \oplus
S$ is ${\tt fld}(R)\times \{0\} \cup {\tt fld}(S)\times \{1\}$.
$\left<x,i\right> (R \oplus S) \left<y,j\right>$ is defined as $i<j
\vee i=0 \wedge j=0 \wedge x\,R\,y \vee i=1 \wedge j=1 \wedge
x\,S\,y$.  Intuitively, we make disjoint orders of types $R$ and $S$
and put the order of type $R$ in front of the order of type $S$.
Finally, we define $\alpha + \beta$ for ordinals $\alpha$ and $\beta$
as ${\tt ot}(R \oplus S)$ for any $R \in \alpha$ and $S \in \beta$.

Another way to put this: for any relation $R$, define $R_x$ as
$\{\left<\left<a,x\right>,\left<b,x\right>\right>\mid a\,R\,b\}$.
Notice that $R\approx R_x$ for any $R$ and $x$ and $R_x \cap S_y =
\emptyset$ for any $R$ and $S$ and any distinct $x$ and $y$.  For any
ordinals $\alpha,\beta$ define $\alpha+\beta$ as ${\tt ot}(R_0 \cup {\tt
(fld}(R_0)\times {\tt fld}(S_1)) \cup S_1)$ where $R \in \alpha$ and $S
\in \beta$.  It is straightforward to establish that $R_0 \cup ({\tt
fld}(R_0)\times {\tt fld}(S_1)) \cup S_1$ is a well-ordering and that
its order type does not depend on which representatives $R$ and $S$
are chosen from $\alpha$ and $\beta$.


\item[Discussion:] An order of type $\omega+1$ is readily obtained:
define $x \,\leq'\,y$ as $$x\in {\mathbb N} \wedge y \in {\mathbb N}
\wedge 0<x\leq y \vee y=0.$$ In effect, we move 0 from its position at
the beginning of the order to the end.  This is the same order type as
that of a strictly increasing sequence taken together with its limit,
which we mentioned above.

The relation $\leq'$ is not isomorphic to the usual $\leq$ on the
natural numbers.  An easy way to see this is that there is a
$\leq'$-largest element of the field of $\leq'$, and this is a
property of relations which is preserved by isomorphism: if $\leq'
\approx \leq$ were witnessed by an isomorphism $f$ then $f(0)$ would
have to be the $\leq$-largest natural number, and there is no such
natural number.

Further, the field of $\leq'$ is the same size as the field of $\leq$
(in fact, it is the same set!): so the theorem that there is a unique
order type of well-orderings of each finite cardinality $n$ does not
generalize to infinite cardinalities.

Observe that an order of type $\omega+\omega$ is a still more complex
well-ordering with field the same size as the field of a relation of
type $\omega$.  A concrete example of such an order would be the order
$$\{\left<x,y\right> \in {\mathbb N}^2 \mid 2|(x-y) \wedge x \leq y
\vee 2\not| x \wedge 2|y\},$$ which puts the odd and even numbers in
their usual respective orders but puts all the odd numbers before all
the even numbers.

\item[Definition (ordinal multiplication):] For well-orderings $R$ and $S$, we define
another well-ordering $R \otimes S$.  The field of $R \otimes S$ is
${\tt fld}(R) \times {\tt fld}(S)$.  $\left<x,y\right> (R \otimes S)
\left<u,v\right>$ is defined as $y\,S\,v \vee y=v \wedge x\,R\,u$.
This is reverse lexicographic order on the cartesian product of the
fields of the relations.   Finally, we define $\alpha \cdot \beta$ for
ordinals $\alpha$ and $\beta$ as ${\tt ot}(R \otimes S)$ for any $R \in
\alpha$ and $S \in \beta$. 

The order $\omega\cdot\omega$ is a still more complex order type whose
field is the same size as that of any relation of order type $\omega$.
There are very complicated well-orderings with countable fields (whose
order types are called {\em countable ordinals\/}).

\end{description}

The algebra of ordinal numbers contains surprises.  Some algebraic
laws do work much as expected, but some basic laws are not inherited
from the algebra of natural numbers.  For example, $\omega+1 \neq
1+\omega = \omega$ and $\omega\cdot 2 \neq 2 \cdot \omega = \omega$.


We now study the natural order relation on the ordinal numbers, which
turns out to be a well-ordering itself (at a higher type).

\begin{description}
\item[Definition:] If $\leq$ is a partial order and $x \in {\tt
fld}(\leq)$, we define ${\tt seg}_{\leq}(x)$ as $\{y \mid y<x\}$
(where $<$ is the strict partial order $[\leq] - [=]$).  ${\tt
seg}_{\leq}(x)$ is the {\em segment\/} determined by $x$.  We define
$\leq_x$ as $[\leq] \cap {\tt seg}_{\leq}(x)^2$; this is the {\em
segment restriction\/} of $\leq$ determined by $x$.

\item[Theorem:]  If $\leq$ is a well-ordering and $x \in {\tt fld}(\leq)$
then $\leq_x$ is a well-ordering.

\item[Lemma:] No well-ordering is isomorphic to one of its own segment
restrictions.

\item[Proof:] Suppose that $\leq$ is a well-ordering, $x$ is in the
field of $\leq$, and $\leq \approx (\leq)_x$ is witnessed by an
isomorphism $f$.  Since $f(x) \neq x$ is obvious ($x$ is not in the
range of $f$!), there must be a $\leq$-least $y$ such that $f(y) \neq
y$.  Let $A={\tt seg}_{\leq}(y)$.  Each element of $A$ is fixed by
$f$.  In $\leq$, $y$ is the least object greater than all elements of
$A$.  In $(\leq)_x$, $f(y)$ is the least object greater than all
elements of $A$.  The two orders agree on the common part of their
field.  Since $f(y)$ is certainly in the field of $\leq$, we have $y
\leq f(y)$ (as otherwise $f(y)$ would be a smaller strict upper bound
for $A$ in $\leq$).  Since $y \leq f(y)$, we have $y$ in the field of
$(\leq)_x$, and $f(y) \leq y$, as otherwise $y$ would be a smaller
strict upper bound for $A$ in $(\leq)_x$.  So $y=f(y)$, which is a
contradiction.

\item[Corollary:] No two distinct segment restrictions of the same
well-ordering can be isomorphic to one another.
\item[Proof:]  One of them would be a segment restriction of the other.

\item[Definition:] We say that a subset $D$ of the field of a
well-ordering $\leq$ is ``downward closed in $\leq$'' iff $(\forall d
\in D.(\forall e \leq d.e \in D))$.

\item[Lemma:]  For any well-ordering $\leq$, a set downward closed in $\leq$
is either the field of $\leq$ or a segment in $\leq$.

\item[Proof:] Let $D$ be a set downward closed in $\leq$.  If $x$
belongs to the field of $\leq$ but does not belong to $D$, then $d <
x$ must be true for all $d \in D$, as otherwise we would have $x \leq
d \in D$, from which $x \in D$ would follow.  This means that if $D$
has no strict upper bound, it must be the entire field of $\leq$.  If
$D$ does have a strict upper bound, it must have a $\leq$-least strict
upper bound $x$ because $\leq$ is a well-ordering.  We claim that $D =
{\tt seg}_{\leq}(x)$ in this case.  If $y \in {\tt seg}_{\leq}(x)$,
then $y$ cannot be a strict upper bound of $D$ because $x$ is the
least strict upper bound of $D$, and so $y$ must be an element of $D$.
If $y$ is an element of $D$, then $y$ must be less than $x$ because
$x$ is a strict upper bound of $D$, that is, $y$ is an element of
${\tt seg}_{\leq}(x)$.  Sets with the same elements are the same.


\item[Theorem:] If $\leq_1$ and $\leq_2$ are well-orderings, then
exactly one of three things is true: either $\leq_1$ and $\leq_2$ are
isomorphic, or $\leq_1$ is isomorphic to a segment restriction
$(\leq_2)_x$, or $\leq_2$ is isomorphic to a segment restriction
$(\leq_1)_x$.

\item[Proof:] Let $\leq_1$ be a well-ordering with field $A$.  Let
$\leq_2$ be a well-ordering with field $B$.  Define $C$ as $\{a \in
A\mid \neg(\exists b \in B.(\leq_1)_a \approx (\leq_2)_b)\}$, the set
of all elements of the field of $\leq_1$ whose segment restrictions
are {\em not\/} isomorphic to a segment restriction in $\leq_2$.  If
$C$ is nonempty, it has a least element $c$.  Each $d <_1 c$ does not
belong to $C$, because $c$ is the $\leq_1$-least element of $C$.
Thus, by the definition of $C$, there is an $e \in B$ such that
$(\leq_1)_d \approx (\leq_2)_e$.  There can be only one such $e$ because
no two segment restrictions of the same well-ordering can be
isomorphic to each other.  Thus there is a function $F$ which maps
each $d <_1 c$ to the unique $e$ such that $(\leq_1)_d \approx
(\leq_2)_e$.  We claim that $F$ is an isomorphism from $(\leq_1)_c$ to
$\leq_2$.  This breaks down into three subclaims: $F$ is an injection,
$F$ is order-preserving, and the range of $F$ is $B$.  For each $d <_1
c$, we have an isomorphism $f$ witnessing $(\leq_1)_d \approx
(\leq_2)_{F(d)}$.  For each $d' < d$, the restriction of $f$ to
${\tt seg}_{\leq_1}(d')$ is an isomorphism from $(\leq_1)_{d'}$ to
$(\leq_2)_{f(d')}$, so in fact $F(d') = f(d')$.  Because the range of
$f$ is the segment in $\leq_2$ determined by $F(d)$, we have $F(d') =
f(d') < F(d)$.  This shows both that $F$ is order preserving and that
it is a bijection.  Further, it shows that the range of $F$ is
downward closed, as we see that the restriction of $F$ to the segment
determined by $d$ is the isomorphism from the segment determined by
$d$ to the segment determined by $F(d)$.  Since the range of $F$ is
downward closed, it must be either $B$ or some ${\tt seg}_{\leq_2}(x)$, so $F$ is
either an isomorphism from $(\leq_1)_c$ to $\leq_2$ or an isomorphism
from $(\leq_1)_c$ to some $(\leq_2)_x$.  The latter case is impossible
by the definition of $c$, so we must actually have $F$ an isomorphism
from $(\leq)_c$ to $\leq_2$, establishing the Theorem in this case.
If the set $C$ is empty, then for every $a \in A$ there is $b \in B$
such that $(\leq_1)_a \approx (\leq_2)_b$.  This $b$ must be unique as no
two distinct segment restrictions of $\leq_2$ can be isomorphic.  For
each $a \in A$, we define $F(a)$ as the unique $b$ such that
$(\leq_1)_a \approx (\leq_2)_b$.  Exactly the same argument just given
shows that $F$ is a bijection, order-preserving, and has a downward
closed range.  From this it follows just as in the first case that $F$
is an isomorphism from $\leq_1$ to either $\leq_2$ or some
$(\leq_2)_x$, establishing that the Theorem is true in this case.  If
$\leq_1 \approx \leq_2$ then we cannot have either $(\leq_1)_x \approx
\leq_2$ or $(\leq_2)_x \approx \leq_1$ because a well-ordering cannot be
similar to one of its segment restrictions.  If we had $\leq_1 \approx
(\leq_2)_x$, and further had $\leq_2 \approx (\leq_1)_y$, witnessed by an
isomorphism $g$, then we would have $\leq_1 \approx (\leq_1)_{g(x)}$,
which is impossible.  This establishes that only one of the three
cases can hold.

\item[Definition:] If $\alpha$ and $\beta$ are ordinal numbers, we
define $\alpha \leq \beta$ as holding iff either $\alpha = \beta$ or
each element of $\alpha$ is isomorphic to a segment restriction in
each element of $\beta$.

\item[Theorem:] The relation $\leq$ defined on ordinal numbers in the
previous definition is a well-ordering.  Where it is necessary to
distinguish it from other orders, we write it $\leq_{\Omega}$.  

\item[Proof:] Let $\alpha$ and $\beta$ be ordinals.  If $\leq_1 \in \alpha$
and $\leq_2 \in \beta$, then either $\leq_1 \approx \leq_2$, in which case
$\alpha=\beta$, or $\leq_1$ is isomorphic to a segment restriction in
$\leq_2$, in which case the same is true for any $\leq_1' \approx
\leq_1$ and $\leq_2' \approx \leq_2$, or $\leq_2$ is isomorphic to a
segment restriction in $\leq_1$, in which case the same is true for
any $\leq_2' \approx \leq_2$ and $\leq_1' \approx \leq_1$.  If more
than one of these alternatives held for any pair of well-orderings,
one of them could be shown to be isomorphic to one of its own segment
restrictions.  Certainly $\alpha \leq \alpha$, so the $\leq$ relation
on ordinals is reflexive.  If $\alpha\leq \beta$ and $\beta \leq
\alpha$ this must be witnessed by isomorphisms between $\leq_1 \in
\alpha$ and $\leq_2 \in \beta$ in both directions, or once again we
would have one of these well-orderings isomorphic to a segment
restriction of itself.  So the $\leq$ relation on ordinals is
antisymmetric.  If we have $\alpha\leq \beta$ and $\beta \leq \gamma$
and we choose $\leq_1,\leq_2,\leq_3$ in $\alpha,\beta,\gamma$
respectively, we have $\leq_1$ isomorphic to $\leq_2$ or a segment
restriction thereof, and $\leq_2$ isomorphic to $\leq_3$ or a segment
restriction thereof, and composition of isomorphisms gives us an
isomorphism from $\leq_1$ to $\leq_3$ or a segment restriction
thereof, thus $\alpha \leq \gamma$, so the $\leq$ relation on ordinals
is transitive and is a linear order.  Now let $\cal A$ be a nonempty
set of ordinals.  Let $\alpha \in {\cal A}$.  Let $\leq_1 \in \alpha$
have field $A$.  Consider the set $B$ of all $a \in A$ such that
$(\leq_1)_a$ belongs to some element of $\cal A$.  If $B$ is
empty, then $\alpha$ is the $\leq$-smallest element of $\cal A$.  If
$B$ is nonempty, choose the smallest $a$ in $B$: ${\tt
ot}((\leq_1)_a)$ is the $\leq$-smallest element of $\cal A$.  So the
relation $\leq$ on the ordinal numbers is a well-ordering, which is
what we set out to prove.

\item[Definition:]  ${\tt
ot}(\leq_{\Omega})$ is called $\Omega$: notice that $\Omega$ is not of
the same type as the ordinals in the field of the relation $\leq_{\Omega}$ of which it is the order type (it is 2 types higher; it would
be 4 types higher if we defined well-orderings using the Kuratowski pair).

\end{description}

\newpage

\subsection{Exercises}

\begin{enumerate}

\item  
Some linear orders are listed.  For each one, state (correctly) that
it is a well-ordering or that it is not.  If it is not, explain
precisely why it is not (this means give an example of something).  If
it is, give its order type (an ordinal number).

\begin{enumerate}

\item $\emptyset$

\item the standard order on the integers restricted to $\{x \in {\mathbb Z}\mid -2 \leq x \leq 2\}$

\item the standard order on the integers restricted to $\{x \in {\mathbb Z}\mid x\leq 0\}$

\item the standard order on the rationals restricted to $\{\frac{n}{n+1} \mid n \in{\mathbb N}\} \cup \{1\}$

\item the standard order on the rationals restricted to $\{\frac{n+1}{n} \mid n \in{\mathbb N}\} \cup \{1\}$

\item the standard order on the reals restricted to the interval $[0,1]$

\end{enumerate}

\item 
Prove that for any natural number $n$, any two linear orders with a
field of size $n$ are isomorphic, and all such linear orders are
well-orderings.  (How do we prove anything about natural numbers?)

\item
Prove that if $R$ and $S$ are well-orderings, so is $R \oplus S$.  You
need to prove that it is a linear order (which will probably require
some reasoning by cases) and prove that it has the additional defining
property of a well-ordering.

Now that you are filled with self-confidence, do the same for $R \otimes S$.

\item Define sets of real numbers such that the restriction of the standard order on the real numbers to that set has each of the following order types:

\begin{enumerate}

\item $\omega+1$

\item $\omega\cdot 3$

\item $3\cdot \omega$

\item $\omega\cdot\omega$

\item $\omega\cdot\omega\cdot\omega$ (OK I suppose this is nasty, but see if you can do it)

\end{enumerate}

\item Prove your choice of the two following annoying propositions (these are annoying in the sense that they are straightforward (even ``obvious'') but there is a good deal to write down).

\begin{enumerate}


\item Isomorphism is an equivalence relation on relations.

\item A relation isomorphic to a well-ordering is a well-ordering.

\end{enumerate}

\end{enumerate}

\newpage


\section{Transfinite Induction and Recursion}

The following theorem is an analogue of mathematical induction for the ordinals.

\begin{description}

\item[Transfinite Induction Theorem:] Suppose $A$ is a set of ordinals
with the following property: $(\forall \alpha \in {\tt Ord}.(\forall
\beta<\alpha.\beta \in A) \rightarrow \alpha \in A)$.  Then $A = {\tt
Ord}$.

\item[Proof:] If $A \neq {\tt Ord}$, then ${\tt Ord}-A$ is a nonempty
set and so contains a least ordinal $\alpha$.  But then obviously
$(\forall \beta<\alpha.\beta \in A)$, so $\alpha \in A$ by assumption,
which is a contradiction.

\item[Transfinite Induction Theorem (bounded form):] Suppose $A$ is a
set of ordinals with the following property: $(\forall \alpha < \gamma.(\forall \beta<\alpha.\beta \in A) \rightarrow \alpha \in A)$.
Then $(\forall \alpha<\gamma.\alpha \in A)$.

\item[Transfinite Induction Theorem (property form):] Suppose
$\phi[\alpha]$ is a formula such that $(\forall \alpha \in {\tt
Ord}.(\forall \beta<\alpha.\phi[\beta]) \rightarrow \phi[\alpha])$.
Then $(\forall \alpha \in {\tt Ord}.\phi[\alpha])$.


\end{description}

This looks like the theorem of strong induction for the natural
numbers.  We can make it look a bit more like the usual formulation of
induction by defining some operations on ordinals.  The alternative
forms are easy to prove and are relevant to untyped set theory where
there is no set containing all ordinals.  [The property form would have
to be restated using a predicate ${\tt Ord}(x)$ in place of a set of
all ordinals to prove theorems about all ordinals in a context where
there is no {\em set\/} of all ordinals.]

\begin{description}

\item[zero:] We define 0 as the smallest ordinal (the order type of
the empty well-ordering).

\item[successor:] For any ordinal $\alpha$, we define the {\em
successor\/} of $\alpha$ as the smallest ordinal greater than
$\alpha$.  No special notation is needed for successor, since it is
easy to show that the successor of $\alpha$ is $\alpha+1$.  Every
ordinal has a successor: for any infinite ordinal $\alpha$ containing
a well-ordering $W$ with minimal element $x$, ${\tt ot}(W - (\{x\}\times
{\tt fld}(W))\cup ({\tt fld}(W) \times \{x\}))$ is $\alpha+1$: the new
order is obtained by moving the minimal element of $W$ from bottom to
top of the order.

\item[limit ordinal:] A nonzero ordinal which is not a successor is
called a limit ordinal.

\end{description}

Now we give a different formulation of Transfinite Induction.

\begin{description}

\item[Transfinite Induction Theorem:] Suppose that $A$ is a set of
ordinals such that $0 \in A$, for every ordinal $\alpha \in A$ we also
have $\alpha+1\in A$, and for any limit ordinal $\lambda$ such that
for all $\beta<\lambda$ we have $\beta \in A$, we also have $\lambda
\in A$.  Then $A = {\tt Ord}$.

\item[Proof:] Again, consider the smallest element of the complement
of $A$ (there must be a smallest if there is any).  It cannot be 0
because $0 \in A$.  It cannot be a successor (because its predecessor
would be in $A$, so it would be in $A$).  It cannot be a limit
(because everything below it would be in $A$, so it would be in $A$).
These are the only possibilities.

\end{description}

We now give an extended example of proof by transfinite induction.
For purposes of this example, we assume familiarity with the real
numbers at the usual undergraduate level.  We have seen in an earlier
section of these notes how to construct the real numbers in our type
theory; mod omitted proofs we are warranted in assuming that they are available at
some type and have familiar properties.

\begin{description}

\item[Definition:] We say that an ordinal $\alpha$ is a {\em countable
ordinal\/} iff the relations which belong to it have countably
infinite fields.  

\item[Lemma:] For any countable ordinal $\alpha$, there is a function
$f:{\mathbb N}\rightarrow {\tt Ord}$ such that for natural numbers
$i<j$ we have $f(i) < f(j)$, $f(i) < \alpha$ for all $i$, and $\alpha$
is the least ordinal greater than all $f(i)$'s.  More briefly, $f$ is
a strictly increasing sequence of ordinals whose least upper bound is
$\alpha$.  We will reserve the right to use the usual notation for
sequences, writing $f(i) = \alpha_i$.

\item[Proof of Lemma:] Let $\alpha$ be a countable ordinal, and let
$\leq$ be a fixed well-ordering of type $\alpha$ with field $A$.
Because $\alpha$ is a countable ordinal, there is an enumeration $a_i$
of the set $A$ (the function $i \in {\mathbb N} \mapsto a_i$ being a
bijection from ${\mathbb N}$ to $A$).  We define a sequence $b_i$
recursively as follows: $b_0 = a_0$.  Once $b_i$ has been defined as
$a_j$, we define $b_{i+1}$ as $a_k$, where $k$ is the least natural
number such that $a_j < a_k$.  The sequence $b$ is strictly
increasing, and every element of $A$ is $\leq$-dominated by some
element of the range of this sequence (because $a_k \leq b_k$ for
every natural number $k$, as is easy to prove by induction).  We can
thus define $f(i) = \alpha_i$ as ${\tt ot}(\leq)_{b_i}$: these
ordinals are clearly all less than the order type $\alpha$ of $\leq$,
they increase strictly as the index increases, and any ordinal less
than $\alpha$, being the order type of some $(\leq)_{a_k}$, is
dominated by some $\alpha_k$.

\item[Definition:] For any subset $X$ of the interval $(0,1]$
in the reals and any $a<b$ real numbers, we define $X_{[a,b]}$ as
$\{(1-x)a + xb \mid x \in X\}$.  This is a scaled copy of $X$ in the
interval $[a,b]$.

For any function $f$ from ${\mathbb N}$ to ${\cal P}((0,1])$ (infinite
sequence of subsets of $(0,1]$), define $f^*$ as $\bigcup
\{f(n)_{[1-2^{-n},1-2^{-n-1}]}\mid n \in {\mathbb N}\}$.  This
construction allows me to put together scaled copies of the infinite
sequence of sets $f(n)$, so that the scaled copies are disjoint and
appear in the same order that the sets appear in the sequence.

\item[Theorem:] For any finite or countable ordinal $\alpha$, we can
find a set of reals $A_{\alpha}\subseteq (0,1]$ such that the order type of the
restriction of the usual linear order on the reals to $A_{\alpha}$ is
a well-ordering of order type $\alpha$.

\item[Proof:] We break the proof into three cases: $\alpha=0$, $\alpha
= \beta+1$ for some $\beta$, or $\alpha$ a limit ordinal.

In any of these cases, we assume that sets $A_{\beta}\subseteq (0,1]$ of reals such
that the usual order on the reals restricted to $A_{\beta}$ has order
type $\beta$ exist for each ordinal $\beta<\alpha$.  Our goal is to
show we can find a set of reals $A_{\alpha}$ such that the order type
of the restriction of the usual linear order on the reals to
$A_{\alpha}$ is a well-ordering of order type $\alpha$.

If $\alpha=0$, $A_{\alpha}=\emptyset$ is a subset of the reals such
that the restriction of the natural order on the reals to this set has
order type $\alpha=0$.

If $\alpha=\beta+1$, we assume the existence of $A_{\beta}$ as above.
The set $(A_{\beta})_{[0,\frac12]} \cup \{1\}$ has the desired
properties: the order type of the natural order on the reals
restricted to this set is clearly $\beta+1$.

If $\alpha$ is a limit ordinal, we have two cases to consider.  If
$\alpha$ is not a countable ordinal, we have nothing to prove.  If
$\alpha$ is a countable ordinal, we select a strictly increasing
sequence $\alpha_i$ such that the least upper bound of its range is
$\alpha$, as a Lemma above shows we are entitled to do.  For each
$\alpha_i$, we are given a set $A_{\alpha_i}\subseteq (0,1]$ of reals
with associated order type $\alpha_i$.  For each $i$, we select a
subset $A_{\alpha_i}'$ of $A_{\alpha_i}$ which we now define.
$A_{\alpha_0}'$ is defined as $A_{\alpha_0}$.  For each $i$,
$\leq_{\mathbb R}\lceil A_{\alpha_{i+1}}$ has a unique segment
restriction of order type $\alpha_i$.  $A_{\alpha_{i+1}}'$ is obtained
by subtracting the field of this segment restriction from
$A_{\alpha_{i+1}}$.  Define $f(i)$ as $A_{\alpha_{i}}'$ and the set
$f^*$ will be the desired set: this is the union of linearly scaled
copies of all the $A_{\alpha_{i}}'$'s, made successively smaller so
they will all fit into $(0,1]$. It should be clear that the union of
such linearly scaled copies has order type $\alpha$.

\item[Theorem:] Any ordinal $\alpha$ which is the order type of the
natural order on a subset $A$ of the reals is finite or countable.

\item[Proof:] Given such an ordinal $\alpha$ and set $A$, we construct
a set $A'$ such that the natural order on $A'$ also has order type
$\alpha$ and all elements of $A'$ are rational numbers (so $A'$ must
be finite or countable).  For each element $a \in A$, either $a$ is
the largest element of $A$ or there is a first element $a'$ of $A$
which is greater than $a$.  This is true because $A$ is well-ordered
by the usual order on the reals.  Assume that we have an enumeration
$q_i$ of the rationals.  Let $q_a$ be the first rational in this
enumeration which is greater than $a$ and less than $a'$ (or simply
the first rational in this enumeration which is greater than $a$, if
$a$ is the largest element of $A$).  It should be evident for all $a,b
\in A$ that $q_a < q_b \leftrightarrow a<b$.  Thus $\{q_a \mid a \in A\}$ is a
set of rationals (thus finite or countable) and the order type of the
natural order on this set is $\alpha$, so $\alpha$ is a finite or
countable ordinal.

\end{description}

We conclude that the order types of well-orderings that we can
construct as suborders of the natural order on the real numbers are
exactly the finite and countable ordinals.  We will see below that
there are uncountable ordinals (this will be our first evidence that
there are infinite sets which are not countably infinite).


We introduce a type raising operation on ordinals analogous to that
already given for cardinals and also traditionally denoted by $T$.

\begin{description}

\item[Definition:] For any relation $R$, we define $R^{\iota}$ as
$\{\left<\iota(x),\iota(y)\right> \mid x\,R\,y\}$ = $\{\left<\{x\},\{y\}\right> \mid x\,R\,y\}$.  Notice that
$R^{\iota}$ is one type higher than $R$ and would seem in some
external sense to be isomorphic to $R$.  $R^{\iota^{\bf n}}$ is similarly
defined as $\{\left<\iota^{\bf n}(x),\iota^{\bf n}(y)\right> \mid x\,R\,y\}$

\item[Definition:] For any ordinal $\alpha$, we define $T(\alpha)$ as
${\tt ot}(R^{\iota})$ for any $R \in \alpha$ (it is easy to show that the
choice of $R$ does not matter).  Of course we can then also define
$T^{\bf n}(\alpha)$ and $T^{\bf -n}(\alpha)$ in the natural ways.

\end{description}

Induction can actually be carried out along any well-ordering, but it
is traditional to translate all transfinite inductions into terms of
ordinals.  A general way to do this involves indexing the elements of
${\tt fld}(\leq)$ for a general well-ordering $\leq$ with ordinals:

\begin{description}

\item[Definition(ordinal indexing):] For any well-ordering $W$, define
$W_{\alpha}$ as the unique element $x$ of ${\tt fld}(W)$ (if there is
one) such that ${\tt ot}((\leq)_x) = \alpha$.  [Note that if $W$ is a
well-ordering of a set of ordinals this is different from
$(W)_{\alpha}$, the segment restriction of $W$ to elements which are
$W$-less than $\alpha$.]

\end{description}

Notice that the type of $\alpha$ is one higher than the type of $W$
and two higher than the type of $W_{\alpha}$ (it would be four higher
than the type of $W_{\alpha}$ if we used the Kuratowski pair).  

$W_{\alpha}$ will be defined for each $\alpha$ iff $\alpha < {\tt ot}(W)$.

Discussion of ordinal indexing in the natural order on the ordinals
themselves requires the following

\begin{description}

\item[Theorem:]  ${\tt ot}((\leq_{\Omega})_{\alpha}) = T^{\bf 2}(\alpha)$

\item[Proof:] This is proved by transfinite induction.  Note that what
it says is that the order type of the segment restriction of the
natural order on the ordinals to the ordinals less than $\alpha$ is
$T^{\bf 2}(\alpha)$.  It is ``obvious'' that this order type is actually
$\alpha$ itself, but of course the order type of the segment
restriction is two types higher than $\alpha$ itself, so it is seen to be
the corresponding ordinal $T^{\bf 2}(\alpha)$ two types higher. 

\end{description}

So $[\leq_{\Omega}]_{\alpha} = T^{\bf -2}(\alpha)$ (not $\alpha$ itself).

Note that $[\leq_{\Omega}]_{\alpha}$ will be undefined for
$\alpha={\tt ot}(\leq_{\Omega})=\Omega$, but
$[\leq_{\Omega}]_{T^{\bf 2}(\Omega)} = \Omega$.  This shows that
$T^{\bf 2}(\Omega)$ is not equal to $\Omega$: in fact $T^{\bf 2}(\Omega) < \Omega$
because $T^{\bf 2}(\Omega)$ is the order type of a segment restriction of
the natural order on the ordinals, whose order type is $\Omega$.

The result that $T^{\bf 2}(\Omega)<\Omega$ (in which there is of course a
kind of punning reference to the sets of ordinals at different types)
shows that there are in effect more ordinals in higher types.
There is no well-ordering in type $k$ as long as the natural order on
the ordinals in type $k+2$.

Now we prove that there are uncountable ordinals.

\begin{description}

\item[Theorem:] There are ordinals which are not finite or countably
infinite (in sufficiently high types), and so there is in particular a
first uncountably infinite ordinal $\omega_1$.

\item[Proof:] Consider the restriction of the natural well-ordering on
the ordinals to the finite and countable ordinals.  This is a
well-ordering, so it has an order type, which we call $\omega_1$.  For
each countable ordinal $\alpha$, the order type of
$(\leq_{\Omega})_{\alpha}$ is $T^{\bf 2}(\alpha)$, and of course
$T^{\bf 2}(\alpha) < \omega_1$, because the former is the order type of a
segment restriction of the latter.  So it cannot be the case that
$\omega_1 = T^{\bf 2}(\alpha)$ for any countable ordinal $\alpha$ (of type
two lower than that of $\omega_1$).  It only remains to show that
every countable ordinal of the same type as $\omega_1$ is of the form
$T^{\bf 2}(\beta)$.  Suppose that $\gamma$ is a countable ordinal of the
same type as $\omega_1$.  $\gamma$ is the order type of some
well-ordering $\leq$ with field the set of natural numbers.  Now
consider $\{\left<T^{\bf -2}(m),T^{\bf -2}(n)\right> \mid m \leq n\}$.  We
know that there is a set of natural numbers two types lower than the
one that $\leq$ orders, because $\gamma$ is of the same type as
ordinals $T^{\bf 2}(\alpha)$ with $\alpha$ countable.  We know that the
$T^{\bf -1}$ operation is total on the natural numbers.  It follows that
the relation just defined makes sense and is of some countable order
type $\beta$, with $\gamma=T^{\bf 2}(\beta)$, so $\gamma<\omega_1$.  But
$\gamma$ is an arbitrary countable ordinal of the type of $\omega_1$,
so $\omega_1$ is uncountably infinite.

\item[Corollary:] There are sets which are infinite but not countably
infinite.

\item[Proof:] The field of any relation of type $\omega_1$ will serve:
the set of finite and countable ordinals is shown to be uncountably
infinite in the argument above.

\end{description}

Here is another very important result about well-orderings whose proof is assisted by ordinal indexing.

\begin{description}

\item[Theorem:]  Suppose that $\leq_1 \subseteq \leq_2$ are well-orderings.
Then ${\tt ot}(\leq_1) \leq {\tt ot}(\leq_2)$.

\item[Proof:] We can prove by an easy transfinite induction that
$[\leq_2]_{\alpha}$ is defined and $[\leq_2]_{\alpha} \leq_2
[\leq_1]_{\alpha}$ for each ordinal $\alpha < {\tt ot}(\leq_1)$.  The
map taking each $[\leq_1]_{\alpha}$ to $[\leq_2]_{\alpha}$ is the desired
isomorphism witnessing ${\tt ot}(\leq_1) \leq {\tt ot}(\leq_2)$.

Of course, when the author says something is easy, that means he or
she doesn't really want to take the trouble to prove it.  We now do
so.

We prove by transfinite induction that $[\leq_2]_{\alpha}$ is defined
and $[\leq_2]_{\alpha} \leq_2 [\leq_1]_{\alpha}$ for each ordinal
$\alpha < {\tt ot}(\leq_1)$.

Note first that an ordinal $\alpha$ is less than ${\tt ot}(\leq_1)$
precisely if it is the order type of some $(\leq_1)_x$, by the
definition of the order on the ordinals, and this $x$ is
$[\leq_1]_{\alpha}$ by the definition of ordinal indexing, so
certainly $[\leq_1]_{\alpha}$ is defined for every $\alpha<{\tt
ot}(\leq_1)$.

We fix an ordinal $\alpha<{\tt ot}(\leq_1)$.  We assume that for every
$\beta<\alpha$, $[\leq_2]_{\beta}$ is defined and $[\leq_2]_{\beta}
\leq_2 [\leq_1]_{\beta}$.  Our goal is to show that
$[\leq_2]_{\alpha}$ is defined and $[\leq_2]_{\alpha} \leq_2
[\leq_1]_{\alpha}$.

Observe that $[\leq_1]_{\alpha}$ exists, and for every $\beta<\alpha$,
$[\leq_2]_{\beta}\leq_2[\leq_1]_{\beta}\leq_2[\leq_1]_{\alpha}$.
($[\leq_1]_{\beta}\leq_1[\leq_1]_{\alpha} \rightarrow
[\leq_1]_{\beta}\leq_2[\leq_1]_{\alpha}$ because $\leq_1 \subseteq
\leq_2$).  This means that there is at least one object which is
$\geq_2$ all the $[\leq_2]_{\beta}$'s for $\beta<\alpha$, so there
must be a $\leq_2$-least such object $x$.  We claim that
$x=[\leq_2]_{\alpha}$.  The objects $\leq_2 x$ are precisely the
$[\leq_2]_{\beta}$'s for $\beta<\alpha$, so the order types of the
initial segments of $(\leq_2)_x$ are precisely the ordinals less than
$\alpha$, so the ordinals less than the order type of $(\leq_2)_x$ are
precisely the ordinals less than $\alpha$, and so its order type
$\ldots$ is $\alpha$ as desired.

\end{description}

Now we develop a construction analogous to recursive definition of
functions of the natural numbers.  Just as transfinite induction is
analogous to strong induction on the natural numbers, so transfinite
recursion is analogous to course-of-values recursion on the natural
numbers.

\begin{description}

\item[Transfinite Recursion Theorem:] We give a nonce definition of
$\cal F$ as the set of all functions whose domains are segments of the
natural order on the ordinals [or on the ordinals less than a fixed
$\gamma$]: $${\cal F} = \{f \mid (\exists \alpha \in {\tt
Ord}.f:{\tt seg}_{\leq_{\Omega}}(\alpha)\rightarrow V)\}.$$ Let $G$
be a function from $\cal F$ to $\iota``V$.  Then there is a unique
function $g$ with domain {\tt Ord} [or with domain the set of ordinals
less than $\gamma$] with the property that for every ordinal $\alpha$
[or for every ordinal $\alpha<\gamma$], $\{g(\alpha)\} = G(g \lceil \{\beta\mid\beta<\alpha\})$.

\item[Proof:] 

We say that a set $I$ is $G$-inductive iff whenever a function $f \in
{\cal F}$ with domain $\{\beta\in {\tt Ord}\mid\beta<\alpha\}$ is a subset of $I$,
$\{\alpha\} \times G(f)$ will be a subset of $I$.  Our claim is that
$g$, defined as the intersection of all $G$-inductive sets, is the
desired function.

We first observe that ${\tt Ord}\times V$ is $G$-inductive, so every
element of $g$ actually is an ordered pair whose first projection is
an ordinal, as we would expect.

We then prove by transfinite induction on $\alpha$ that $g_{\alpha} =
g \cap {\tt seg}_{\leq_{\Omega}}(\alpha)\times V$ is a function with
domain ${\tt seg}_{\leq_{\Omega}}(\alpha)$.  For $\alpha=0$ this is
obvious (the empty set is a function with domain the empty set of all
ordinals less than 0).  Suppose that $g_{\beta}$ is a function with
domain the set of ordinals less than $\beta$: our goal is then to show
that $g_{\beta+1}$ is a function with domain the set of ordinals less
than $\beta+1$.  We claim that $X_{\beta}= g_{\beta} \cup (\{\beta\}
\times G(g_{\beta})) \cup (\{\gamma \mid \gamma > \beta\} \times V)$
is $G$-inductive.  Suppose that $f$ is a function with domain the set
of ordinals less than $\delta$ and $f$ is a subset of $X_{\beta}$.  If
$\delta<\beta$, it follows that $f$ is a subset of $g_{\beta}$ and so
$\{\delta\} \times G(f)$ is a subset of $g$ (because $g$ is
$G$-inductive) and also a subset of $g_{\beta}$ and so of $X_{\beta}$
because the first projection of its sole element is $\delta < \beta$.
If $\delta=\beta$, then $f=g_{\beta}$ and $\{\beta\}\times
G(g_{\beta})$ is a subset of $X_{\beta}$ by construction.  If
$\delta>\beta$, then $G(f)$ is a subset of $X_{\beta}$ because the
first projection of its sole element is $\delta>\beta$.  From this we
can see that $g_{\beta} \cup G(g_{\beta})$ is precisely $g_{\beta+1}$:
$G$-inductiveness of $g$ shows that $g_{\beta} \cup (\{\beta\}\times
G(g_{\beta}))$ must be included in $g$, because $g_{\beta}$ is
included in $g$; $G$-inductiveness of $X_{\beta}$ shows that $g$, and
so $g_{\beta+1}$, does not include any ordered pairs with first
component $\beta+1$ and second component outside of $G(g_{\beta})$.
Clearly $g_{\beta+1}$ is a function, with the same value as
$g_{\beta}$ at each ordinal $< \beta$ and the sole element of
$G(g_{\beta})$ as its value at $\beta$, so its domain is the set of
all ordinals less than $\beta+1$ as desired.  Now we consider the case
of a limit ordinal $\lambda$ with the property that $g_{\beta}$ is a
function for each $\beta<\lambda$.  In this case $g_{\lambda}$ is the
union of all the $g_{\beta}$'s.  The only way it could fail to be a
function is if some two $g_{\beta}$'s had distinct values at some
ordinal.  But this is impossible: it is clear from the definition that
$g_{\beta} \subseteq g_{\beta'}$ for $\beta<\beta'$.  It is also
obvious that the domain of $g_{\lambda}$ is the union of the domains
of the $g_{\beta}$'s, and the union of the segments determined by the
ordinals less than a limit ordinal is the segment determined by that
limit ordinal.

Since $g$ is a relation with domain the set of ordinals and its
restriction to any initial segment of the ordinals is a function, it
is a function.  We showed above that the value of $g_{\beta+1}$ ($g$
restricted to the ordinals less than $\beta+1$) at $\beta$ is the sole
element of $G(g_\beta)$, the value of $G$ at the restriction of $g$ to
the ordinals less than $\beta$, and this is the recurrence relation we
needed to show.  Suppose that $g \neq g'$ were two distinct functions
satisfying this recurrence relation.  Let $\delta$ be the smallest
ordinal such that $g(\delta) \neq g'(\delta)$.  Note that $\{g(\delta)\} =
G(g\lceil \{\gamma\mid\gamma<\delta\}) = G(g'\lceil
\{\gamma\mid\gamma<\delta\}) = \{g'(\delta)\}$ by the shared recurrence
relation and the fact that $g$ and $g'$ agree at ordinals less than
$\delta$, a contradiction.

\end{description}

We give the qualifications needed for a bounded formulation of
recursion in brackets in the statement of the theorem: this is the
form which would be used in untyped set theory but also in many
applications in typed set theory.

We present a variation of the Recursion Theorem:

\begin{description}

\item[Transfinite Recursion Theorem:] Suppose we are given a set $a$,
a function $f$ and a singleton-valued function $F$ (of appropriate
types which can be deduced from the conclusion): then there is a
uniquely determined function $g:{\tt Ord}\rightarrow V$ such that
$g(0) = a$, $g(\alpha+1) = f(g(\alpha))$ for each $\alpha$, and
$g(\lambda)$ is the sole element of $F(\{g(\beta)\mid
\beta<\lambda\})$ for each limit ordinal $\lambda$.

\item[Proof:] This is a special case of the theorem above.  The
function $G:{\cal F}\rightarrow \iota``V$ is defined so that
$G(\emptyset)=\{a\}; G(k) = \{f(k(\alpha))\}$ if $\alpha$ is the
maximum element of the domain of $k$; $G(k) = F(\{k(\beta) \mid
\beta<\lambda\})$ if the limit ordinal $\lambda$ is the supremum of
the domain of $k$.  The stated recurrence relations are then
equivalent to $\{g(\alpha)\} = G(g \lceil \{\beta\mid\beta<\alpha\})$.

\end{description}

The alternative theorem could also be stated in a bounded form.

We define ordinal iteration in a special case.  Suppose $f$ is a
function and $\leq$ is an order on elements of its field understood
from context.  Define $f^0(x)$ as $x$, $f^{\alpha+1}(x)$ as
$f(f^{\alpha}(x))$, and $f^{\lambda}(x)$ as $\sup\{f^{\beta}(x) \mid
\beta<\lambda\}$.  This will uniquely determine a function by either
of the recursion theorems.  It would seem most natural to do this
construction when $f$ was an increasing function in $\leq$ with the
property $x \leq f(x)$.  A common choice of $\leq$ would be the subset
relation.

\newpage

The arithmetic operations on the ordinals defined above can also be
defined by transfinite recursion.

\begin{description}

\item[recursive definition of addition:]
This resembles the iterative definition of addition on the natural numbers.
\begin{enumerate}



\item $\alpha + 0 = \alpha$

\item $\alpha + (\beta+1) = (\alpha+\beta)+1$

\item $\alpha + \lambda = \sup(\{(\alpha+\beta)\mid \beta <\lambda\})$ when $\lambda$ is limit.

\end{enumerate}

\item[recursive definition of multiplication:]
This resembles the iterative definition of multiplication on the
natural numbers.

\begin{enumerate}


\item $\alpha\cdot 0 = 0$

\item $\alpha\cdot(\beta+1) = \alpha\cdot\beta + \alpha$

\item $\alpha\cdot\lambda  = \sup(\{\alpha\cdot\beta\mid \beta<\lambda\})$ when $\lambda$ is limit.

\end{enumerate}

\item[recursive definition of exponentiation:]
Of course a similar definition of exponentiation on natural numbers
could be given (and is actually in effect included here).  There is a
set theoretical definition of exponentiation of ordinals as well, but
it is a bit technical.

\begin{enumerate}


\item $\alpha^0 = 1$

\item $\alpha^{\beta+1} = \alpha^\beta \cdot \alpha$

\item $\alpha^{\lambda} = \sup(\{\alpha^{\beta}\mid \beta<\lambda\})$ when $\lambda$ is limit.

\end{enumerate}


\end{description}

All the ordinal arithmetic operations commute with the $T$ operation:

\begin{description}

\item[Theorem:] For any ordinals $\alpha$ and $\beta$,
$T(\alpha+\beta) = T(\alpha)+T(\beta); T(\alpha\cdot\beta) =
T(\alpha)\cdot T(\beta); T(\alpha^{\beta}) = T(\alpha)^{T(\beta)}$.
$T(\alpha) \leq T(\beta) \leftrightarrow \alpha \leq \beta$; if
$T^{\bf -1}(\alpha)$ exists and $T^{\bf -1}(\beta)$ does not, then
$\alpha<\beta$.

\end{description}

\newpage

We now consider the {\em original\/} application of set theory due to
Cantor, which includes an example of construction of a function by
transfinite recursion.  This involves a further discussion of sets of
reals.

\begin{description}

\item[accumulation point:] If $X$ is a set of reals and $r$ is a real
number, we say that $r$ is an {\em accumulation point\/} of $X$ iff
every open interval which contains $r$ contains infinitely many points
of $X$.  Note that $r$ does not have to be an element of $X$ to be an
accumulation point of $X$.

\item[closed set:]  A set of reals $X$ is said to be {\em closed\/} iff
every accumulation point of $X$ is an element of $X$.

\item[derived set:] For any set $X$ of reals, we define the derived
set $X'$ of $X$ as the set of accumulation points of $X$.

\item[Observations:] Obviously $X$ is closed iff $X' \subseteq X$.
Whether $X$ is closed or not, $X'$ is closed: if any interval
containing $r$ contains infinitely many points of $X'$, then it
contains at least one element of $X'$ (accumulation point of $X$)
because it contains infinitely many, and so it contains infinitely
many points of $X$, and so $r$ is itself an accumulation point of $X$
and thus an element of $X'$.  This means further that if we iterate
applications of the derived set operator, the first iteration may make
our set larger but all subsequent iterations will fix it or remove
elements from it.

\item[iteration of the derived set construction:] This is a definition
by transfinite recursion.  Define $\Delta^X_0$ as $X$.  Define
$\Delta^X_{\beta+1}$ as $(\Delta^X_{\beta})'$.  At limit stages, take
intersections: define $\Delta^X_{\lambda}$ as
$\bigcap\{\Delta^X_{\gamma}\mid\gamma<\lambda\}$ for each limit
ordinal $\lambda$.

\item[Theorem:] For every countable ordinal $\alpha$, there is a set
of reals $A\subseteq (0,1]$ with the property that
$\Delta^A_{\alpha}=\{1\}$ (and so $\Delta^A_{\alpha+1}=\emptyset$).

\item[Proof:] We prove this by transfinite induction on $\alpha$.  If
$\alpha=0$, the set $\{1\}$ has the desired properties.  Suppose that
we have a set $A\subseteq (0,1]$ such that $\Delta^A_{\beta}=\{1\}$.
Let $f$ be the function which sends each natural number $n$ to the set
$A$: $f^* \cup \{1\}$ will have the desired property.  This set
consists of infinitely many successively smaller copies of $A$
approaching the limit point $\{1\}$.  Application of the derived set
operator $\beta$ times will reduce each of the infinitely many scale
copies of $A$ in $f^* \cup \{1\}$ to a single point.  The next
application of the derived set operator will leave just $\{1\}$ (1 is
the only accumulation point).  So $f^* \cup \{1\}$ is the desired set
for which $\beta+1$ applications of the derived set operator yields
$\{1\}$.  Now let $\lambda$ be a countable limit ordinal.  There will
be a strictly increasing sequence $\lambda_i$ of ordinals such that
$\lambda$ is the least ordinal greater than all the $\lambda_i$'s
(this is proved above).  By inductive hypothesis, we may assume that
for each $i$ we have a set $A_i$ such that
$\Delta^{A_i}_{\lambda_i}=\{1\}$.  Define $f(i) = A_i$ (you might note
that this actually requires the Axiom of Choice!).  Define $A=f^* \cup
\{1\}$.  Observe that application of the derived set operator to $A$
$\lambda_i+1$ times eliminates the copy of $A_i$, for each $i$.
Notice that application of the derived set operator $\lambda_i$ times
always leaves $\{1\}$ in the set, as the scaled copies of $A_{j}$ for
$j>i$ still have nonempty image, so clearly 1 will still be an
accumulation point.  It follows from these two observations that the
intersection of all the sets $\Delta^A_{\lambda_i}$, which will be
$\Delta^A_{\lambda}$, will contain no element of any of the original
scaled copies of the $A_i$'s but will contain 1: it will be $\{1\}$ as
required.

\end{description}

The sets shown to exist by this Theorem are in a sense ``discrete''
(they cannot be dense in any interval, or no iteration of the derived
set operation could eliminate them), but have progressively more
complex limit structure calibrated by the countable ordinal $\alpha$.
The applications of these concepts by Cantor to problems in the
convergence of trignonometric series are the original motivation (or
one of the original motivations) for the development of transfinite
ordinals and of set theory.

\newpage

\subsection{Exercises}

\begin{enumerate}

\item

Prove that for any ordinals $\alpha$, $\beta$, $\gamma$, if
$\alpha+\beta=\alpha+\gamma$ then $\beta=\gamma$.'

You can probably prove this by transfinite induction, using the
recursive definitions, but it can be proved using the set theoretic
definition and structural properties of ordinals as well.

Give a counterexample to ``if
$\beta+\alpha=\gamma+\alpha$ then $\beta=\gamma$".

\item In type theory, prove that for all ordinals $\alpha$ and $\beta$, 
if $\alpha+1 = \beta+1$ then $\alpha=\beta$.  This is best proved by
considering actual well-orderings and isomorphisms between them (not
by transfinite induction).

\item
Prove by transfinite induction: Every infinite ordinal can be
expressed in the form $\lambda+n$, where $\lambda$ is a limit ordinal
and $n$ is a finite ordinal, and moreover it can be expressed in this
form in only one way (for this last part you might want to use the
result of the previous problem).

\end{enumerate}

\newpage

\section{Lateral Functions and $T$ operations; Type-Free Isomorphism Classes}

We have observed that cardinals $\kappa$ and $T^{\bf n}(\kappa)$, though of
different types, are in some sense the same cardinal, and similarly
that ordinals $\alpha$ and $T^{\bf n}(\alpha)$, though of different types,
are in some sense the same order type.

We have $T^{\bf n}(|A|) = |B|$ iff $|\iota^{\bf n}``A| = |B|$, that is iff there
is a bijection $f:\iota^{\bf n}``A \rightarrow B$.  The bijection $f$
witnesses the fact that $A$ and $B$ are ``the same size'', by
exploiting the fact that $A$ and $\iota^{\bf n}``A$ are externally ``the
same size''.

We introduce the following definitions.

\begin{description}

\item[Definition (lateral relations):] If $R \subseteq \iota^{\bf n}``A \times B$, we
define $x \,R_{\bf n}\, y$ as holding iff $\iota^{\bf n}(x) \,R\,y$.  Similarly,
if $S \subseteq A \times \iota^{\bf n}``B$, we define $x \,S_{\bf -n}\,y$ as
holding iff $x\,S\,\iota^{\bf n}(y)$.

\item[Definition (description of lateral relations):] We define $A\times_{\bf n}B$
as $\iota^{\bf n}``A\times B$ and \newline $A \times_{\bf -n}B$ as $A \times
\iota^{\bf n}``B$.

\item[Definition (lateral functions):] If $f:\iota^{\bf n}``A \rightarrow B$, we define
$f_{\bf n}(a) = f(\iota^{\bf n}(a))$ for each $a \in A$.  Similarly, if $g:A
\rightarrow \iota^{\bf n}``B$, we define $g_{\bf -n}(a) = \iota^{\bf -n}(g(a))$.

\item[Definition (description of lateral functions):]  $f_{\bf n}:A \rightarrow B$ is
defined as $f:\iota^{\bf n}``A \rightarrow B$; $f_{\bf -n}:A \rightarrow B$ is
defined as $f:A \rightarrow \iota^{\bf n}``B$.

\end{description}

Note that in none of these notations is a boldface subscript actually
part of the name of a function or relation: the boldface subscripts
are always indications of the role the function or relation is playing
in the expression.

This definition allows us to code relations and functions with domains
and ranges of different types.  Note that this definition allows us to
say that $T^{\bf n}(|A|) = |B|$ iff there actually is a (lateral) bijection
from $A$ to $B$!  The definition also allows us to assert that
well-orderings of types $\alpha$ and $T^{\bf n}(\alpha)$ actually are
``isomorphic'' in the sense that there is a lateral function
satisfying the formal conditions to be an isomorphism between them.

We present the Transfinite Recursion Theorem in a slightly different
format:

\begin{description}
\item[Transfinite Recursion Theorem:] We give a nonce definition of
$\cal F$ as the set of all functions whose domains are segments of the
natural order on the ordinals [or on the ordinals less than a fixed
$\gamma$].  Let $G_{\bf -1}:\cal F\rightarrow V$.  Then there is a
unique function $g$ with domain {\tt Ord} [or with domain the set of
ordinals less than $\gamma$] with the property that for every ordinal
$\alpha$ [or for every ordinal $\alpha<\gamma$], $g(\alpha) = G_{\bf
-1}(g \lceil {\tt seg}(\alpha))$.
\end{description}

We give a general ``comprehension'' theorem for functions and relations
with a type differential.

\begin{description}

\item[Theorem:] If $\phi[x^n,y^{n+k}]$ is a formula, there is a set
relation $R$ such that $x \,R_{\bf k}\, y \leftrightarrow \phi[x,y]$ (where
types revert to being implicit in the second formula).

 If $\phi[x^{n+k},y^{n}]$ is a formula, there is a set
relation $R$ such that $x \,R_{\bf -k}\, y \leftrightarrow \phi[x,y]$ (where
types revert to being implicit in the second formula).

If $(\forall x^{n}\in A.(\exists!y^{n+k}.\phi[x^n,y^{n+k}]))$,
then there is a function $f_{\bf k}:A \rightarrow V$ such that
for any $x \in A$, $y = f_{\bf k}(x) \leftrightarrow \phi[x,y]$.

If $(\forall x^{n+k}\in A.(\exists!y^{n}.\phi[x^{n+k},y^{n}]))$,
then there is a function $f_{\bf -k}:A \rightarrow V$ such that
for any $x \in A$, $y = f_{\bf -k}(x) \leftrightarrow \phi[x,y]$.

\item[Corollary:] If $A^n$ and $B^{n+k}$ are sets and there is a
formula $\phi[a,b]$ such that $(\forall a \in A.(\exists!b \in
B.\phi[a,b])) \wedge (\forall b \in B.(\exists!a \in A,\phi[a,b]))$,
then $T^k(|A|) = |B|$.  If $\leq_1^n$ and $\leq_2^{n+k}$ are
well-orderings, and there is a formula $\phi$ such that $(\forall xy.x
<_1 y \leftrightarrow (\exists zw.\phi[x,z]\wedge \phi[y,w]\wedge z<_2 w))
\wedge (\forall zw.z <_2 w \leftrightarrow (\exists xy.\phi[z,x]\wedge
\phi[w,y]\wedge x<_1 y))$, then $T^k({\tt ot}(\leq_1)) = {\tt
ot}(\leq_2)$.

\end{description}


All parts of this theorem are proved by direct application of the
Axiom of Comprehension.  The Corollary expresses the idea that any
external bijection or isomorphism we can describe using a formula is
actually codable by a set and so witnesses appropriate cardinal or
ordinal equivalences.

We note that $T$ operations can be defined for general isomorphism
classes.

\begin{description}

\item[Definition:] For any relation $R$, the isomorphism class
$[R]_{\approx} = \{S \mid R \approx S\}$.  We define $T([R]_{\approx})
= [R^{\iota}]_{\approx}$, where $R^{\iota} =
\{\left<\{x\},\{y\}\right>\mid x \,R\,y\}$, as already defined.  Note
that this is more general than but essentially the same as the T
operation on ordinals.

\end{description}

Now we pursue an extension of the Reasonable Convention proposed above
for natural numbers.  We recall that the $T$ operation on cardinals
witnesses an exact correspondence between the natural numbers at
different types.  This allows us, if we wish, to introduce natural
number variables which can be used in a type-free manner: such a
variable can be shifted into the type appropriate for any context by
appropriate applications of the $T$ operation or its inverse.  All
statements purely in the language of the natural numbers are invariant
under uniform application of the $T$ operation, as we have seen.  Each
occurrence of a natural number variable translates into an occurrence
of a general variable of an appropriate type restricted to the set of
natural numbers at the appropriate type.

This idea can be extended to cardinals and ordinals (and to
isomorphism classes in general), but a further refinement is needed.
The difficulty is that the ordinals in one type are mapped injectively
into but not onto the ordinals in the next type, as we have just seen.
We will see below that the same is true of the cardinals.  The natural
number variables introduced in the previous paragraph are translated
as general variables restricted to the set of all natural numbers
(which is in effect the same set at each type); this cannot work for
the ordinals (or the cardinals): each ordinal bound variable must be
restricted to the ordinals in a specific type (which is equivalent to
restriction to an initial segment of ``all the ordinals'' determined
by the first ordinal not in that particular type (the first ordinal of
the next higher type which is not an image under $T$)).  We can thus
use type-free ordinal variables as long as we require that any such
variable be restricted to a proper initial segment of the ordinals
(the type of the bound will determine the highest type in which we can
be working), and we can treat cardinals similarly.  There is no way to
express a general assertion about all ordinals at whatever type in
type theory.  Just as in natural number arithmetic, all statements
about properties, relations, and operations natural to cardinals and
ordinals are invariant under uniform application of the $T$ operation:
this enables the proposed identifications of cardinals and ordinals at
diverse types to cohere.

This convention would allow the elimination in practice of the
inconvenient reduplication of cardinals, ordinals, and similar
constructions at each type.  We do not use it as yet, but it is
important to us to note that it is possible to use this convention.

\subsection{Lateral functions in the system of the unsorted preamble:  an axiom of embedding}

In the system of section 2.1.1,  where we cannot be sure that all types are indexed by natural numbers as in our basic type scheme,
we postulate the existence of many external embeddings of one type into another.

\begin{description}

\item[Axiom of embedding:]  We postulate a primitive strict partial order $<_\tau$ on types.  We postulate function symbols $I_{xy}$ and $I_{yx}$ such
that we have $I_{xy}(a)$ defined just in case $a \in \tau(x)$, $I_{xy}(a) \in \tau(y)$ if it is defined, and, if $\tau(x) <_\tau \tau(y)$,  $I_{yx}(I_{xy}(a)) = a$ for all $a \in \tau(x)$.  We provide further
that for any $x,y$ there is $z$ such that $\tau(x) <_\tau(z)$ and $\tau(y) <_\tau(z)$ and that further, $\tau(x) <_\tau (y)$ iff $|I_{xz}``(\tau(x))| < |I_{yz}``(\tau(y))|$.

Further clauses which are tempting to add to our axiom, though we believe they are not really needed, are $I_{x\tau(x)}(a) = \{a\}$ and, for $\tau(x) <_\tau (y) <_\tau (z)$,
$I_{yz}(I_{xy}(a)) = I_{xz}(a)$ for all $a \in \tau(x)$.  Another reasonable assumption is that the action of the function symbol $I_{xy}$ is exactly determined by the type of $x$
and the type of $y$.

\end{description}

I want to include a description here of how operations such as $\iota^n(x)$ can actually be defined in terms of the natural numbers of the theory using this axiom.

An effect of this axiom is that isomorphism types (including cardinal and ordinal numbers) can be systematically identified between types.

This axiom implies the existence of base types in the unsorted system.

\newpage

\section{Other Forms of the Axiom of Choice}

The Axiom of Choice is equivalent to some other interesting
propositions (in fact, there are whole books of them but we will only
discuss a few).

\begin{description}

\item[The Well-Ordering Theorem:] Every set is the field of a
well-ordering.  (Equivalently, $V$ is the field of a well-ordering.)

\item[Observation:] It is obvious that the well-ordering theorem
implies the Axiom of Choice: the choice set of a partition can be
taken to be the set of minimal elements in the elements of the
partition under a well-ordering of the union of the partition.  The
interesting part of the result is the converse: the Axiom of Choice is
enough to prove the Well-Ordering Theorem.

\item[Definition:] A {\em chain in a partial order $\leq$\/} is a subset
$C$ of ${\tt fld}(\leq)$ such that $\leq\cap C^2$, the restriction of
$\leq$ to $C$, is a linear order (i.e., any two elements of $C$ are
comparable in the order).

\item[Definition:] A collection of sets is said to be
{\em nested\/} iff it is a chain in the inclusion order: $A$ is a nested
collection of sets iff $(\forall x \in A.(\forall y \in A.x\subseteq y
\vee y \subseteq x))$.

\item[Lemma:] The union of a nested collection of chains in a partial
order $\leq$ is a chain in $\leq$.

\item[Zorn's Lemma:] A partial order with nonempty domain in which
every chain has an upper bound has a maximal element.

\item[Observation:] Let $\cal A$ be the set of all well-orderings of
subsets of a set $A$.  We define $U \leq V$ as holding for $U,V \in
{\cal A}$ iff either $U=V$ or $U$ is a segment restriction of $V$.  A
chain in this well-ordering is a collection $C$ of well-orderings of
$A$ which agree with one another in a strong sense and whose union
will also be a well-ordering of a subset of $A$ and so an upper bound
of the chain $C$ (details of this bit are left as an exercise).  So
Zorn's Lemma would allow us to conclude that there was a maximal
partial well-ordering of $A$ under the segment restriction relation,
which clearly must be a well-ordering of all of $A$ (any element not
in the field of the maximal well-order could be adjoined as a new
largest element of a larger well-ordering for a contradiction).

Since Zorn implies Well-Ordering and Well-Ordering implies Choice, it
only remains to show that Choice implies Zorn to prove that all three
are equivalent (in the presence of the rest of our axioms).

\item[Proof of Zorn's Lemma:] Let $\leq$ be a partial order in which every
chain has an upper bound.

Let $\cal C$ be the set of all chains in $\leq$.  Note that for any
chain $C$ if there is an upper bound of $C$ which belongs to $C$ there
is exactly one such upper bound.  If in addition all upper bounds of
$C$ belong to $C$ then this uniquely determined upper bound is maximal
in $\leq$.  For each chain $C$ in $\leq$, define $B_C$ as the set of
all upper bounds for $C$ which are not in $C$, if there are any, and
otherwise as the singleton of the unique upper bound of $C$ which is
an element of $C$.  All of these sets will be nonempty.  The set $\{\{\left<C,\{b\}\right> \mid b \in B_C\}\mid C \in {\cal C}\}$ is a partition, and so has a choice set.  Notice that the
choice set is a function $F$ which sends each $C \in {\cal C}$ to the
singleton set of an upper bound of $C$, which will belong to $C$ only
if all upper bounds of $C$ belong to $C$ (in which case the upper
bound is maximal).

For each chain $C$, denote the linear order $\leq\cap\, C^2$ by
$\leq_C$.  We call a chain $C$ a {\em special chain\/} iff $\leq_C$ is
a well-ordering and for each $x\in C$ we have $\{x\} = F({\tt
fld}((\leq_C)_x))$.

We can prove by transfinite induction that $\leq_C$ is precisely
determined by its order type (for any special chains $C$ and $D$, if
$\leq_C$ is isomorphic to $\leq_D$ then $\leq_C=\leq_D$).  Suppose
otherwise: then there is a least ordinal to which distinct $\leq_C$
and $\leq_D$ belong.  There must be a $\leq_C$-first element $x$ which
differs from the corresponding $\leq_D$ element $y$.  But this implies
that $(\leq_C)_x = (\leq_D)_y$ whence $\{x\} = F((\leq_C)_x) =
F((\leq_D)_y)=\{y\}$.

This implies further that for any two distinct special chains, one is
a segment restriction of the other.  This further implies that the
union of all special chains is a linear order and in fact a special
chain; call it $E$.  Now $E \cup F(\leq_E)$ is a special chain as
well, which cannot properly extend $E$, so $F(\leq_E)\subseteq E$, so
the sole element of $F(E)$ is a maximal element with respect to
$\leq$.

\newpage

\item[Alternative Proof of Zorn's Lemma:]

Let $\leq$ be a nonempty partial order in which any chain
has an upper bound.  Let $\cal C$ be the set of all chains in $\leq$.

For each chain $C$ in $\leq$ and $x \in {\tt fld}(\leq)$, we say that
$x$ is an appropriate upper bound of $C$ if $x$ is an upper bound of
$C$ and $x \not\in C$ or if $x \in C$ and all upper bounds of $C$ are
elements of $C$.  Notice that if there is an upper bound of $C$
belonging to $C$ there is only one, and also notice that if the unique
upper bound of $C$ belonging to $C$ is the only upper bound of $C$
then it is maximal in $\leq$, because anything strictly greater than
the unique upper bound of $C$ in $C$ would be an upper bound of $C$
not in $C$.

For each chain $C$ in $\leq$, we define $X_C$ as the set of all
ordered pairs $\left<C,\{x\}\right>$ such that $x$ is an appropriate
upper bound of $C$.  Notice that if $C \neq C'$ then $X_C$ and
$X_{C'}$ are disjoint (because elements of the two sets are ordered
pairs with distinct first projections).  Thus $\{X_C \mid C \in {\cal
C}\}$ is a partition, and has a choice set $F$.  Notice that $F$ is a
function, $F:{\cal C} \rightarrow \iota``{\tt fld}(\leq)$, and $F(C)$
for every $C$ is the singleton $\{x\}$ of an appropriate upper bound
$x$ of $C$.

  Define a function $G$ by transfinite recursion: $G(\alpha)$ is
defined as the sole element of $F({\tt rng}(G\lceil
\{\beta\mid\beta<\alpha\}))$ if ${\tt rng}(G\lceil
\{\beta\mid\beta<\alpha\})$ is a chain in $\leq$ and as 0 otherwise.
Transfinite induction shows that ${\tt rng}(G\lceil
\{\beta\mid\beta<\alpha\})$ is a chain in $\leq$ for any ordinal
$\alpha$ (for successor $\alpha$, because $C \cup F(C)$ is always a
chain in $\leq$ if $C$ is a chain in $\leq$, and for limit $\alpha$
because a union of nested chains in $\leq$ is a chain in $\leq$).
Note that $G(\alpha)$ will not be one of the $G(\beta)$'s for
$\beta<\alpha$ unless it is maximal for $\leq$, so $G$ is injective if
$\leq$ has no maximal element.  The range of $G$ is a subset of the
field of $\leq$ with a unique well-ordering under which $G$ is an
increasing function.  The order type of this well-ordering will be the
order type $\Omega$ of the ordinals iff $G$ is injective.  If $G$ is
not injective, it is constant past a certain point and so the order
type of this well-ordering will be that of an initial segment of the
ordinals, so strictly less than $\Omega$.  Now we employ a trick:
consider instead of $\leq$ the order type $\leq^{\iota^{\bf 2}}$ of double
singletons induced by $\leq$.  The well-ordering of the range of the
function $G$ associated with $\leq^{\iota^{\bf 2}}$ will have some order
type $T^{\bf 2}(\alpha)<\Omega$ (because it is a well-ordering of a set of
double singletons) and so cannot be injective, and so $\leq^{\iota^{\bf 2}}$
has a maximal element, from which it follows that $\leq$ itself has a
maximal element.  The point of the trick is that the original working
type we started with might not have had enough ordinals for the
construction of $G$ to exhaust the field of $\leq$.

Throughout this discussion we could have used the lateral function
notation introduced in the previous subsection: $F_{\bf -1}(C)$ is an
upper bound for $C$ for each chain $C$.


\end{description}

NOTE: include examples of use of Zorn's Lemma in other parts of
mathematics.

The Axiom of Choice directly enables us to make choices from pairwise
disjoint collections of sets.  But in fact we can use the Axiom to
show that we can make choices from any collection of nonempty sets.

\begin{description}

\item[Definition:] Let $A$ be a collection of nonempty sets.  A function
$c$ with domain $A$ is called a {\em choice function for $A$\/} iff
$c:A \rightarrow 1$ ($c(a)$ is a one element set for each $a \in A$)
and $c(a) \subseteq a$ for each $a \in A$.  The sole element of $c(a)$
is the item selected from $A$ by the choice function.

It is equivalent to say (using the notation for lateral functions) that
a choice function for $A$ is a function $c_{\bf -1}:A \rightarrow V$ such
that $c_{\bf -1}(a) \in a$ for each $a \in A$.

\item[Theorem:]  Each collection of nonempty sets $A$ has a choice function.

\item[Proof:] The collection $\{\{a\} \times \iota``a \mid a
\in A\}$ is a partition and so has a choice set $c$.  This choice set
is the desired choice function.

\end{description}

We observe that a logical device proposed above and revisited from time to time, but not officially adopted, can be introduced by definition at this point.

\begin{description}

\item[Hilbert symbol:] Let $H$ be a fixed function $V \rightarrow 1$
such that $H \lceil ({\cal P}(V) - \{\emptyset\})$ is a choice function and $H(\emptyset) = \emptyset$ (if the type of the second $\emptyset$ is positive).  We do
not care which one.  Define $(\epsilon x.\phi)$ as the sole element of
$H(\{x \mid \phi\})$ for each formula $\phi$.

\item[Theorem:] For any formula $\phi$, $(\exists x.\phi) \leftrightarrow
\phi[(\epsilon x.\phi)/x]$.  Since $(\forall x.\phi) \leftrightarrow
\neg(\exists x.\neg \phi)$, this means that both quantifiers could be
defined in terms of the Hilbert symbol.

\item[Proof:]  This is obvious.

\end{description}

Note that a systematic use of the Hilbert symbol would imply a choice of
an $H$ in each relevant type.

\newpage

\subsection{Exercises}

\begin{enumerate}

\item
Prove that the union of a nested set of chains in a partial order
$\leq$ is a chain.  A chain is a set $C$ such that for any $x,y \in C$
we have either $x \leq y$ or $y \leq x$; a nested collection of sets
is a set $A$ of sets which is a chain in the subset relation (for any
$x,y \in A$, either $x \subseteq y$ or $y \subseteq x$).
\item  
Prove that the union of a countably infinite collection of countably
infinite sets is countably infinite.  Notice that you already know that
${\mathbb N}\times {\mathbb N}$ is a countable set.

We give the result in more detail: suppose that $F$ is a function with
domain $\mathbb N$ and the property that each $F(n)$ is a countably
infinite set.  Show that $\bigcup \{F(n)\mid n \in {\mathbb N}\}$ is
countable (that is, show that it is the range of a bijection with
domain the set of natural numbers).

Hint: be very careful.  It is fairly easy to see why this is true if
you understand why ${\mathbb N}\times {\mathbb N}$ is a ocuntable set,
but there is an application of the Axiom of Choice involved which you
need to notice; in type theory or set theory without choice there may
be countable collections of countable sets which have uncountable
unions!


\item
Use Zorn's Lemma to prove that every infinite set is the union of a
pairwise disjoint collection of countably infinite sets.

Then prove that if $B$ is a collection of countably infinite sets,
$|\bigcup B| = |\bigcup B|+|\bigcup B|$.  (This exploits the fact that
$|{\mathbb N}|=|{\mathbb N}|+|{\mathbb N}|$; it also requires the
Axiom of Choice).

Notice that this is another proof that $\kappa+\kappa=\kappa$ for any
infinite cardinal $\kappa$.

\end{enumerate}

\newpage

\section{Transfinite Arithmetic of Order, Addition, and Multiplication}

We define the order relation on cardinals in a natural way.

\begin{description}

\item[order on cardinals:] $|A| \leq |B|$ iff there is an injection from
$A$ to $B$.

\end{description}

Implicit in our notation is the claim that $\leq$ is a partial order.
The relation is obviously reflexive and transitive: that it is
antisymmetric is a famous theorem.

\begin{description}

\item[Cantor-Schr\"oder-Bernstein Theorem:] If $|A| \leq |B|$ and $|B|
\leq |A|$ then $|A|=|B|$.

\end{description}

Before proving this theorem we give an example to illustrate why it is
not obvious.  Consider the sets $[0,1] = \{x \in {\mathbb R} \mid 0
\leq x \leq 1\}$ and ${\cal P}({\mathbb N})$, the set of all sets of
natural numbers.

A injection $f$ from $[0,1]$ into ${\cal P}({\mathbb N})$ is defined
by $f(r) = \{k \in {\mathbb N}\mid \frac1{2^{k+1}}$ is a term in the
unique nonterminating binary expansion of $r\}$ while a bijection $g$
from ${\cal P}({\mathbb N})$ into $[0,1]$ is given by $g(A) =
\Sigma_{k\in A}\frac1{10^{k+1}}$.  So it is easy to see that each set
embeds injectively in the other, but it is not at all easy to see how
to construct a bijection which takes one set exactly onto the other.

We now give the slightly delayed


\begin{description}

\item[Proof of the Cantor-Schr\"oder-Bernstein Theorem:] Assume that there
is an injection $f:A \rightarrow B$ and an injection $g:B \rightarrow
A$: our goal is to show that there is a bijection $h$ from $A$ to $B$.
$B$ is the same size as $g``B \subseteq A$, so if we can show $A \sim
g``B$ we are done.  The map $f|g$ sends all of $A$ into $g``B$; we
develop a trick to send it exactly onto $g``B$.  Let $C$ be the
intersection of all sets which contain $A-g``B$ and are closed under
$f|g$.  Let $h_0$ be the map which sends all elements of $C$ to their
images under $f|g$ and fixes all elements of $A-C$.  This is a
bijection from $A$ to $g``B$, so $h_0|g^{-1}$ is a bijection from $A$
to $B$.

\end{description}

Note that this proof does not use the Axiom of Choice.  Beyond this
point we will use the Axiom of Choice freely, and some of the results
we state are not necessarily true in type theory or set theory without
Choice.

\newpage

\begin{description}

\item[Theorem:]  The natural order on cardinals is a linear order.

\item[Proof:] Let $A$ and $B$ be sets: we want to show $|A| \leq |B|$
or $|B| \leq |A|$.  This is easy using the Well-Ordering Theorem: we
choose well-orderings $\leq_A$ and $\leq_B$ of $A$ and $B$
respectively.  If the well-orderings are isomorphic, the isomorphism
between them witnesses $|A|=|B|$ (and so $|A| \leq |B|$.  Otherwise,
one of $\leq_A$ and $\leq_B$ is isomorphic to a segment restriction of
the other, and the isomorphism is the required injection from one of
the sets into the other.

\item[Theorem:]  The natural order on cardinals is a well-ordering.

\item[Proof:] Let $C$ be a set of cardinals.  Our aim is to show that
$C$ has a smallest element in the natural order.  Let $\leq$ be a
well-ordering of a set at least as large as any of the elements of the
union of $C$ (the universe of the appropriate type will work).
Consider the set of all well-orderings of elements of the union of $C$
(note that the union of $C$ is the set of all sets which have
cardinalities in the set $C$).  Every well-ordering in this set will
either be similar to $\leq$ or similar to some segment restriction of
$\leq$.  If all are similar to $\leq$, then all elements of $C$ are
the same and it has a smallest element.  Otherwise consider the set of
all $x$ such that $\leq_x$ is isomorphic to some well-ordering of some
element of the union of $C$: there must be a $\leq$-smallest element
of this set, which corresponds to the smallest element of $C$ in the
natural order.

\item[Theorem:]  There is a surjection from $A$ onto $B$ iff
$|B| \leq |A|$ (and $B$ is nonempty if $A$ is).

\item[Proof:] If there is an injection $f$ from $B$ to $A$, then we
can define a surjection from $A$ to $B$ as follows: choose $b \in B$;
map each element of $A$ to $f^{-1}(a)$ if this exists and to $b$
otherwise.  This will be a surjection.  If $B$ is empty we cannot
choose $b$, but in this case $A$ is empty and there is obviously a
surjection.

If there is a surjection $f$ from $A$ onto $B$, there is a partition
of $A$ consisting of all the sets $f``\{a\}$ for $a \in A$.  Let $C$
be a choice set for this partition.  Map each element $b$ of $B$ to
the unique element of $C\subseteq A$ which is sent to $b$ by $f$.
This map is obviously an injection.

\newpage

\item[Definition:] In type theory or set theory {\em without\/}
Choice, we define $$|A| \leq^* |B|$$ as holding iff there is a surjection
from $B$ onto $A$.  In the light of the previous Theorem, there is no
need for this notation if we assume Choice.

\item[Theorem:] For all cardinals $\kappa$ and
$\lambda$, $\kappa \leq \lambda \leftrightarrow T(\kappa) \leq T(\lambda)$.  If
$T^{\bf -1}(\kappa)$ exists and $T^{\bf -1}(\lambda)$ does not exist then
$\kappa\leq \lambda$.

\end{description}

\begin{description}

\item[Definition (repeated from above):]   We define $\aleph_0$ as
$|\mathbb N|$.  Elements of $\aleph_0$ are called  {\em countably infinite
sets\/}, or simply {\em countable sets\/}.

\item[Theorem:] $\aleph_0+1 = \aleph_0+\aleph_0 = \aleph_0 \cdot
\aleph_0 = \aleph_0$.  It is straightforward to define a bijection
between $\mathbb N$ and $\mathbb N\times \mathbb N$.  The bijections
between the naturals and the even and odd numbers witness the second
statement.  The successor map witnesses the first statement.

\item[Theorem:]  $\aleph_0 = T(\aleph_0)$.

\item[Proof:] This follows from the fact that natural numbers are sent
to natural numbers by the $T$ operation and by its inverse.

\item[Theorem:]  Every infinite set has a countable subset.

\item[Proof:] Let $A$ be an infinite set.  The inclusion order on the
collection of all injections from initial segments of $\mathbb N$ to
$A$ satisfies the conditions of Zorn's Lemma and so has a maximal
element.  

If the maximal element had domain a proper initial segment
of $\mathbb N$, then the set $A$ would be finite, because proper initial segments of $\mathbb N$ are finite sets
and (we claim) an injection which has domain a proper initial segment of $\mathbb N$ and range a proper subset of $A$ is not maximal.

We verify the claim.  Suppose we have an injection $f$ from an initial segment of $\mathbb N$ whose domain is a proper initial segment of $\mathbb N$ (say \newline $\{m \in \mathbb N:m < n\}$) and whose range is not $A$.  Choose $a \in A$ which is not in the range of $f$.  Then $f \cup \{(n,a)\}$ is an element of the collection which properly extends $f$.

Thus the maximal element must have domain $\mathbb N$, and its range is a countably infinite subset of $A$.

NOTE:  We corrected terminology in this proof from ``bijection" to ``injection", and made it more explicit, 5/15/2023.  At the same time, we note the importance for this proof of the fact that every proper initial segment of $\mathbb N$ is a finite set, which is proved above in the discussion of the $T$ operation.

\item[Theorem:]  For every infinite cardinal $\kappa$, $\kappa+1 = \kappa$.

\item[Proof:] Let $A$ be an infinite set.  The inclusion order on the
set of all bijections from $B$ to $B \cup \{x\}$, where $B \cup \{x\}
\subseteq A$ and $x \not\in B$, satisfies the conditions of Zorn's
Lemma and so has a maximal element.  It is nonempty because $A$ has a
countable subset.  If the maximal element is a map from $B$ to $B \cup
\{x\}$ and there is $y \in A-(B\cup \{x\})$, then affixing
$\left<y,y\right>$ to the map shows that the supposed maximal element
was not maximal.

An easier proof of this uses the previous theorem.  $\kappa = \lambda+\aleph_0$ for some $\lambda$, since a set of size $\kappa$ has a countable subset.  It follows
that $\kappa=\lambda+\aleph_0=\lambda+(\aleph_0+1)=(\lambda+\aleph_0)+1=\kappa+1$.

\item[Corollary:] If $n$ is finite and $\kappa$ is an infinite
cardinal then $\kappa+n = \kappa$.

\item[Theorem:] For every infinite cardinal $\kappa$,
$\kappa+\kappa=\kappa$.

\item[Proof:] Let $A$ be an infinite set.  The set of
pairs of injections $f$ and $g$ with ${\tt dom}(f)={\tt dom}(g) = {\tt rng}(f) \cup {\tt rng}(g) \subseteq A$ and
${\tt rng}(f) \cap {\tt rng}(g) = \emptyset$ can be partially ordered by componentwise
inclusion: $$(f,g) \leq (f',g') \leftrightarrow f\subseteq f' \wedge g \subseteq g'.$$  This partial order satisfies the hypotheses of Zorn's Lemma (verifying this is left as an exercise).  It is nonempty
because $A$ has a countable subset. Suppose that a maximal such pair
of bijections $f,g$ in the componentwise inclusion order has been constructed. Let $B$ be the common domain of $f$ and $g$. If there is no countably infinite
subset in $A-B$, then $A-B$ is finite and $|B|=|A|$ by a previous
result and the result is proved: otherwise take a countable
subset of $A-B$ and extend the supposedly maximal pair of  maps to a larger one.

\item[Corollary:]  If $\lambda\leq \kappa$ and $\kappa$ is an infinite cardinal
then $\kappa+\lambda=\kappa$:  note that $\kappa \leq \kappa+\lambda \leq \kappa+\kappa=\kappa$.

\item[Theorem:] For every infinite cardinal $\kappa$,
$\kappa\cdot\kappa = \kappa$.

\item[Proof:] Let $A$ be an infinite set.  The inclusion order on
bijections from $B\times B$ to $B$, where $B \subseteq A$, satisfies
the conditions of Zorn's Lemma.  It is nonempty because $A$ has a
countable subset.  Now consider a maximal function in this order,
mapping $B \times B$ to $B$.  If $A-B$ contains no subset as large as
$B$, then $|B| = |A|$ by the previous result and the result is proved.
Otherwise, choose $B' \subseteq A-B$ with $|B'| = |B|$.  It is then
easy to see from assumptions about $B$ and $B'$ and the previous
result that the map from $B \times B$ to $B$ can be extended to a
bijection from $(B \cup B') \times (B \cup B')$ to $B \cup B'$,
contradicting the supposed maximality of the bijection.

\item[Corollary:]  If $\lambda\leq \kappa$ and $\kappa$ is an infinite cardinal
then $\kappa\cdot\lambda=\kappa$.



\end{description}

The arithmetic of addition and multiplication of infinite cardinals is
remarkably simple.  This simplicity depends strongly on the presence
of Choice.

\newpage

\subsection{Exercises}
\begin{enumerate}
\item A classical argument that $|{\cal R}^2| = |{\cal R}|$ goes as follows.
Suppose that it is granted that $|[0,1]| = |{\cal R}|$ (this takes a
wee bit of work, too, but not too much).  So it suffices to prove that
$|[0,1]|^2=|[0,1]|$.  Map the pair of numbers with decimal expansions
$0.a_1a_2a_3\ldots$ and $0.b_1b_2b_3\ldots$ to the number with
expansion $0.a_1b_1a_2b_2a_3b_3\ldots$.  Unfortunately, this doesn't
quite give us the bijection we want due to problems with decimal
expansions (explain).  Give a corrected description of this map,
taking into account bad features of decimal expansions, and explain
why it is not a bijection from $[0,1]^2$ to $[0,1]$.  Is it an
injection?  A surjection?  Then use a theorem from the notes (giving
all details of its application to this situation) to conclude that
there is a bijection from $[0,1]^2$ to $[0,1]$.
\end{enumerate}

\newpage

\section{Cantor's Theorem}

\subsection{Cardinal Exponentiation and the Theorem}

In this section, we start by defining another arithmetic operation.
We have delayed this because its properties in the transfinite context
are more vexed.

\begin{description}

\item[Definition (function space):] The set of all functions from $A$
to $B$ is called $B^A$.  Note that $B^A$ is one type higher than $A$
or $B$ (it would be three types higher if we were using the Kuratowski
pair).

\item[Definition (cardinal exponentiation):]  We define $|A|^{|B|}$ as $T^{\bf -1}(A^B)$.

\end{description}

The appearance of $T^{\bf -1}$ is required to get a type-level operation
(it would be $T^{\bf -3}$ if we used the Kuratowski pair).  It makes it
formally possible that this operation is partial -- and indeed it
turns out that this operation {\em is\/} partial.

\begin{description}

\item[Definition:] For each subset $B$ of $A$ define
$\chi^A_B$ as the function from $A$ to $\{0,1\}$ which sends each
element of $B$ to 1 and each element of $A-B$ to 0.  We call this {\em the
characteristic function of $B$ (relative to $A$)\/}.  

\item[Observation:] The function sending each $B \subseteq A$ to
$\chi^A_B$ is a bijection.

\item[Theorem:]  $|{\cal P}(A)| = |\{0,1\}^A|$, so $2^{|A|} = T^{\bf -1}(|{\cal P}(A)|).$

\end{description}

Now comes the exciting part.

\begin{description}

\item[Cantor's Theorem:]  For any set $A$, there is no function $f$
from $\iota``A$ onto ${\cal P}(A)$.

\item[Proof:] Suppose otherwise, that is, that there is a function $f$
from $\iota``A$ onto ${\cal P}(A)$.  Consider the set $$R = \{a \in
A \mid a  \not\in  f(\{a\})\}.$$ Since $f$ is onto, $R = f(\{r\})$ for some
$r \in A$.  Now $$r \in R \leftrightarrow r \not\in f(\{r\}) = R$$ is a
contradiction.

\end{description}

This tells us that a set $A$ cannot be the same size as its power set.
The fact that $A$ and ${\cal P}(A)$ are of different types
necessitates the exact form of the theorem.  This implies that if
$2^{|A|}$ exists, that $$T(|A|) =|\iota``A|\neq|{\cal
P}(A)|=T(2^{|A|})$$ so $|A| \neq 2^{|A|}$.  There are at least two
distinct infinite cardinals, $|{\mathbb N}|$ and $2^{|{\mathbb N}|}$
(in high enough types for both to be present).

Since certainly $|\iota``A| \leq |{\cal P}(A)|$ (singletons of
elements of $A$ are subsets of $A$), it follows by
Cantor-Schroder-Bernstein that $|{\cal P}(A)| \not\leq |\iota``A|$, as
otherwise these two cardinals would be equal, so we can write
$|\iota``A| < |{\cal P}(A)|$ and  $\kappa < 2^{\kappa}$.

Further we have the curious result that $|\iota``V| < |{\cal P}(V)|$
must be distinct: there are more sets in any given type than
singletons of sets (of the next lower type).  This implies that $V$ in
any given type is strictly larger than any set of lower type (in the
sense that the elementwise image under an appropriate $\iota^{\bf n}$ of
the lower type set at the same type as $V$ will have smaller
cardinality than $V$): $T^{\bf -1}(|V|)$ is undefined and so is $2^{|V|}$,
which would be $T^{\bf -1}(|{\cal P}(V)|)$.

\subsection{Applications to the Number Systems}

We give some set theoretical facts about familiar number systems.

\begin{description}

\item[Theorem:] ${\mathbb Z} \sim {\mathbb N}$

\item[Proof:] Consider the map which sends 0 to 0, $2n-1$ to $n$ for
each natural number $n>0$ and $2n$ to $-n$ for each $n>0$.  This is a
bijection.

\item[Theorem:] ${\mathbb Q} \sim {\mathbb N}$.

\item[Proof:] There is an obvious injection from $\mathbb Q$ into
${\mathbb Z} \times {\mathbb N}^+$ determined by simplest forms of
fractions.  ${\mathbb Z} \times {\mathbb N}^+ \sim {\mathbb N} \times
{\mathbb N}$ is obvious.  ${\mathbb N}\times {\mathbb N}$ is injected
into $\mathbb N$ by the map $f(m,n) = 2^m3^n$, and of course ${\mathbb
N}$ injects into $\mathbb Q$.  The result follows by the
Cantor-Schr\"oder-Bernstein theorem.

\item[Theorem:] ${\mathbb R} \sim {\cal P}({\mathbb N})$, so
$|{\mathbb R}|>|{\mathbb N}|$.

\item[Proof:] An injection from the interval $[0,1)$ in the reals into
${\cal P}({\mathbb N})$ is defined by sending each real $r$ in that
interval to the set of all natural numbers $i$ such that there is a 1
in the $\frac1{2^i}$'s place in the binary expansion of $r$ which
contains infinitely many 1's.  An injection from ${\cal P}({\mathbb
N})$ to the interval $[0,1)$ sends each set $A$ of natural numbers to
the real number whose base 3 expansion consists of 1's in the
$\frac1{3^i}$'s place for each $i\in A$ and zeroes in all other
places.  It follows by Cantor-Schr\"oder-Bernstein that $$[0,1) \sim
{\cal P}({\mathbb N}).$$ Injections from $(-\frac{\pi}2,\frac{\pi}2)$
into $[0,1)$ and vice versa are easy to define, so
$$(-\frac{\pi}2,\frac{\pi}2) \sim[0,1).$$  Finally, the arc tangent function witnesses $$(-\frac{\pi}2,\frac{\pi}2) \sim {\mathbb R},$$  The cardinal inequality follows from Cantor's Theorem.

\end{description}

The linear orders on $\mathbb Q$ and $\mathbb R$ share a
characteristic which might suggest to the unwary that both sets should
be larger than the ``discrete'' ${\mathbb N}$.

\begin{description}

\item[Definition:] If $\leq$ is a linear order and $A \subseteq {\tt
fld}(\leq)$, we say that $A$ is {\em dense in $[\leq]$\/} iff for each $x<y$
there is $z\in A$ such that $x<z$ and $z<y$ (it is traditional to write
$x<z \wedge z<y$ as $x<z<y$).  We say that $\leq$ itself is merely
{\em dense} iff ${\tt fld}(\leq)$ is dense in $[\leq]$.

\item[Observation:] The natural orders on $\mathbb Q$ and $\mathbb R$
are dense.  ${\mathbb Q}$ is dense in the order on ${\mathbb R}$.

\item[Definition:] A linear order with a finite or countable dense set
is said to be {\em separable\/}.  The immediately preceding example
shows that a separable linear order need not be countable.

\item[Theorem:] Any two dense linear orders with countably infinite
field and no maximum or minimum element are isomorphic.  This is a
characterization of the order on ${\mathbb Q}$ up to isomorphism.

\item[Proof:]  Let $\leq_1$ and $\leq_2$ be two such orders.
Let $\leq^1$ and $\leq^2$ be well-orderings of order type $\omega$
with the same fields as $\leq_1$ and $\leq_2$ respectively.

We define a map $f$ from ${\tt fld}(\leq_1)$ to ${\tt fld}(\leq_2)$ by
a recursive process.  

Initially, we define $a^0_1$ as the $\leq^1$-least element of ${\tt
fld}(\leq_1)$, and define $f(a^0_1)$ as the $\leq^2$-least element of
${\tt fld}(\leq_2)$.  This completes stage 0 of the construction.

Suppose that the values at which $f$ has been defined at the $n$th
stage of our construction are the terms $a^n_i (0 \leq i \leq N)$ of a
finite strictly $\leq_1$-increasing sequence of elements of ${\tt
fld}(\leq_1)$, and further that $f(a^n_i) (0 \leq i \leq N)$ is a
strictly increasing $\leq_2$- sequence.  We define $a^{n+1}_{2i+1}$ as
$a^n_i$ for each $i$ in the domain of $a^n$.  Note that this means
that $f$ is already defined at each of the odd-indexed elements of the
range of $a^{n+1}$ that we will consider in what follows.  We define
$a^{n+1}_0$ as the $\leq^1$-least element of the $\leq_1$-interval
$(-\infty,a^{n+1}_1)$ and $f(a^{n+1}_0)$ as the $\leq^2$-least element
of the $\leq_2$-interval $(-\infty,f(a^{n+1}_1))$.  We define
$a^{n+1}_{2N+2}$ as the $\leq^1$-least element of the
$\leq_1$-interval $(a^{n+1}_{2N+1},\infty)$, and $f(a^{n+1}_{2N+2})$
as the $\leq^2$-least element of the $\leq_2$-interval
$(f(a^{n+1}_{2N+1}),\infty)$.  These selections succeed because
neither order has a maximum or minimum.  For $0 \leq i <N$, we define
$a^{n+1}_{2i}$ as the $\leq^1$-least element of the $\leq_1$-interval
$(a^{n+1}_{2i-1},a^{n+1}_{2i+1})$ and $f(a^{n+1}_{2i})$ as the
$\leq^2$-least element of the $\leq_2$-interval
$(f(a^{n+1}_{2i-1}),f(a^{n+1}_{2i+1}))$.  These selections succeed
because both orders are dense.  It should be clear that the extended
sequence $a^{n+1}$ has the same properties specified for the sequence
$a^n$, so this process can be continued to all values of $n$.
Further, it should be clear that the $m$-th element of the order
$\leq^1$ appears in the domain of $f$ by stage $m$ and the $m$-th
element of the order $\leq_2$ appears in the range of $f$ by stage
$m$, so the definition of $f$ succeeds for all values in the domain of
$\leq_1$, defines a function which is onto the domain of $\leq_2$, and
is clearly a strictly increasing bijection, so an isomorphism.

\item[Definition:] A linear order is said to be {\em complete\/} iff
every subset of the order which is bounded above has a least upper
bound.

\item[Observation:]  The order on ${\mathbb R}$ is complete.

\item[Theorem:] A nonempty separable complete dense linear order with
no maximum or minimum is isomorphic to the order on $\mathbb R$.

\item[Proof:] By the theorem above, the order restricted to the
countable dense subset is isomorphic to the usual order on ${\mathbb
Q}$, from which it follows easily that the entire order is isomorphic
to the usual order on ${\mathbb R}$.

\end{description}

\newpage

\subsection{Cardinals and Ordinals; Cardinal Successor; The Hartogs and Sierpinski Theorems}

For any cardinal $\kappa<|V|$, there are larger cardinals ($|V|$, for
instance).  Since the natural order on cardinal numbers is a
well-ordering, there is a {\em smallest\/} cardinal greater than
$\kappa$.  For finite cardinals $n$, this next largest cardinal is
$n+1$, but of course for infinite $\kappa$ we have $\kappa=\kappa+1$:
we will see below how the ``next'' cardinal is obtained in the
infinite case.

\begin{description}

\item[Definition:] If $\kappa \neq |V|$ is a cardinal number, we
define $\kappa+$ as the least cardinal in the natural order which is
greater than $\kappa$.

\end{description}

Now we take an apparent digression into the relationships between
cardinal and ordinal numbers.  Each ordinal $\alpha$ is naturally
associated with a particular cardinal:

\begin{description}

\item[Definition:] Let $\alpha$ be an ordinal number.  We define ${\tt
card}(\alpha)$ as $|{\tt fld}(R)|$ for any $R \in \alpha$ (the choice
of $R$ makes no difference).

\end{description}

For each finite cardinal $n$ there is only one ordinal number $\alpha$
such that ${\tt card}(\alpha)=n$ (usually written $n$ as well).  But
for any ordinal $\alpha$ such that ${\tt card}(\alpha)$ is infinite,
we find that ${\tt card}(\alpha+1) = {\tt card}(\alpha)+1 = {\tt
card}(\alpha)$: the ${\tt card}$ operation is far from injective.  But
there is an ordinal naturally associated with each cardinal as well:

\begin{description}

\item[Definition:] Let $\kappa$ be a cardinal.  We define ${\tt
init}(\kappa)$ as the smallest ordinal number $\alpha$ such that ${\tt
card}(\alpha)=\kappa$.  There is such an ordinal because any set of
size $\kappa$ can be well-ordered; there is a least such ordinal
because the natural order on ordinals is a well-ordering.

\end{description}

It is important to note that the $T$ operations on ordinals and
cardinals preserve order, addition, multiplication, and
exponentiation.  Intuitively, this is all true because $T(\kappa)$ is
in some external sense the same cardinal as $\kappa$ and $T(\alpha)$
is in some external sense the same order type as $\alpha$.  The proofs
are straightforward but tedious.  One has to take into account the
fact that cardinal exponentiation is a partial operation (which
reflects the fact that there are more cardinals and ordinals in higher
types).

We restate and extend our theorems on the fact that the $T$ operation
commutes with operations of arithmetic.

\begin{description}

\item[Theorem:]  Let $\kappa$ and $\lambda$ be cardinal numbers.  Then
$T(\kappa)\leq T(\lambda) \leftrightarrow \kappa \leq \lambda$, $T(\kappa)+T(\lambda) = T(\kappa+\lambda)$, $T(\kappa)\cdot T(\lambda) = T(\kappa\cdot\lambda)$,
and $T(\kappa^{\lambda}) = T(\kappa)^{T(\lambda)}$ if the former exists.
$T(\kappa+) = T(\kappa)+$.

\item[Theorem:] Let $\alpha$ and $\beta$ be ordinal numbers.  Then
$T(\alpha)\leq T(\beta) \leftrightarrow \alpha \leq \beta$, $T(\alpha)+T(\beta)
= T(\alpha+\beta)$, $T(\alpha)\cdot T(\beta) = T(\alpha\cdot\beta)$,
and $T(\alpha^{\beta}) = T(\alpha)^{T(\beta)}$ (ordinal
exponentiation is total).



\end{description}

We now prove a theorem characterizing the way in which $\kappa+$ is
obtained from $\kappa$ when $\kappa$ is infinite.

\begin{description}

\item[Theorem:] Let $\kappa=|A|\neq |V|$ be an infinite cardinal.  Let
$\Omega_A$ be the set of order types of well-orderings of subsets of
the set $A$ (clearly this does not depend on the choice of the set
$A$).  Then $\kappa+ = {\tt card}(\sup(\Omega_A))$.

\item[Proof:] Let $\gamma = {\tt card}(\sup(\Omega_A))$.  Since a
well-ordering of a set of size $\gamma$ must be longer than any
well-ordering of a subset of $A$, $\gamma>\kappa$.  Now suppose that
$\lambda<\gamma$.  It follows that ${\tt init}(\lambda)<{\tt
init}(\gamma)= \sup(\Omega_A)$, so a well-ordering of a set of size
$\lambda$ is of the same length as the well-ordering of some subset of
a set of size $A$, so $\lambda \leq \kappa$.  Note that the size of
the set of ordinals less than $\sup(\Omega_A)$ is $T^2(\gamma)$, so we
could also define $\gamma$ as $T^{\bf -2}(|{\tt
seg}_{\leq_{\Omega}}(\sup(\Omega_A))|)$.

\end{description}

In the absence of Choice the argument above does not work, but there
is still something interesting to say.

\begin{description}

\item[Definition:] For any cardinal $\kappa = |A| \neq |V|$, define
$\Omega_A$ as the set of order types of well-orderings of subsets of
$A$ and $\aleph(\kappa)$ as ${\tt card}(\sup(\Omega_A))$.

\item[Observation:] The preceding definition is only of interest in
the absence of Choice, as otherwise it coincides with $\kappa+$.  Note
that $\aleph(\kappa)$ is always a cardinal whose elements are
well-orderable.  Note that for syntactical reasons this use of
$\aleph$ is distinguishable from another use to be introduced shortly.

\item[Theorem (not using Choice; Hartogs):] For any cardinal $\kappa$, $\aleph(\kappa)
\not\leq \kappa$.

\item[Proof:] Suppose otherwise.  Let $\kappa=|A|$.  We then have an
injection from a set $B$ of order type $\sup(\Omega_A)$ into $A$.  The
range of this injection supports a well-ordering of type
$\sup(\Omega_A)$.  But the range of this injection is a subset of $A$,
so its order type belongs to $\Omega_A$.  This is a contradiction.

\item[Theorem (not using Choice; Sierpinski):]  $\aleph(\kappa) \leq \exp^3(\kappa)$.

\item[Proof:] Since we are working in choice-free mathematics, it is
advantageous to represent things in different ways.  Any well-ordering
is represented effectively by the set of its initial segments.  We
refer to such a representation of an order as a segment-ordering.  A
segment-ordinal is an equivalence class of segment-ordinals.  Notice
that a segment-ordinal is three types higher (not two types higher)
than the elements of its field.  If $A \in \kappa$, observe that the
set of segment-ordinals of well-orderings of subsets of $A$ is of
cardinality $T^3(\aleph(\kappa))$.  Of course a segment-ordinal is a
set of sets of sets of elements of $A$: the collection of sets of sets
of sets of elements of $A$ is of cardinality $T^3(\exp^3(\kappa))$.
The desired inequality follows.

A related result is $\aleph(\kappa) \leq \exp^2(\kappa^2)$ This is
obtained by noting that the usual ordinals of well-orderings of
subsets of $A$ are sets of sets of pairs of elements of $\kappa$, so
$T^2(\aleph(\kappa)) \leq T^2(\exp^2(\kappa^2))$.  This is most useful
when we know that $\kappa^2=\kappa$: this is not a theorem of
choice-free mathematics, though it is true if elements of $\kappa$ are
well-orderable or if $\kappa$ is of the form $\exp^4(\lambda)$ for
$\lambda$ infinite (this last because the construction of the Quine
pair can then be mimicked in a set of size $\kappa$).

\end{description}


\newpage

\subsection{Hierarchies of Cardinals; A Disadvantage of Strong Extensionality}

We introduce two notations for cardinal numbers.  

\begin{description}

\item[Definition:] Let $\aleph$ be the natural order on infinite
cardinals.  We then define $\aleph_{\alpha}$ for ordinals $\alpha$
using the definition of ordinal indexing of the elements of the field
of a well-ordering.

\item[Definition:] We define $\omega_{\alpha}$ as ${\tt
init}(\aleph_{\alpha})$.

\item[Definition:] Let $\beth$ be the natural order on cardinal
numbers restricted to the smallest set of cardinal numbers which
contains $\aleph_0$, is closed under the power set operation, and
contains suprema of all of its subsets.  We then define
$\beth_{\alpha}$ for ordinals $\alpha$ using the definition of ordinal
indexing.

\end{description}

We can now pose one of the notable questions of set theory, dating to
the beginnings of the subject.  The first infinite cardinal is
$\aleph_0$.  We know by Cantor's Theorem that $|{\cal P}({\mathbb N})|
> |\iota``{\mathbb N}| = \aleph_0$.  We also note that $|{\cal
P}({\mathbb N})| = \beth_1$.  We know by definition of cardinal
successor that $\aleph_1 = \aleph_0+ > \aleph_0$.  We know by the
observation following the theorem above that $\aleph_1$ is the number
of finite and countable ordinals (which is easily shown to be the same
as the number of countable ordinals).  The question that arises is the
status of

\begin{description}

\item[$^*$Cantor's Continuum Hypothesis:] $\beth_1 = \aleph_1$?  Are
there more subsets of the natural numbers than countable order types?

\end{description}

It is called the Continuum Hypothesis because Cantor also knew (as we
will find in the next section) that $\beth_1$ is not only the
cardinality of the the set of subsets of the natural numbers but also
the cardinality of the set of real numbers, or the number of points on
a line (the cardinality of the continuum).  For this reason $\beth_1$
is also called $c$ (for ``continuum'').

A related assertion (which is again a hypothesis not a theorem) is 

\begin{description}

\item[$^*$Generalized Continuum Hypothesis (GCH):] $\aleph_{\alpha} =
\beth_{\alpha}$ for all ordinals for which $\aleph_{\alpha}$ is
defined.

\end{description}

A further question is how far the $\aleph_{\alpha}$'s or
$\beth_{\alpha}$'s continue.  These notations are definitely undefined
for sufficiently large ordinals $\alpha$ (neither is defined for ${\tt
ot}(\aleph)$, by a simple consideration of how ordinal indexing is
defined).  We cannot prove in this system that $\aleph_{\omega}$ is
defined or even that $\aleph_n$ exists for each natural number $n$.
It is true that $\beth_n$ exists among the cardinals of type $n$ sets
for each $n$, but there is a kind of pun going on here.  It is also
true that the sequences of $\aleph$'s and $\beth$'s get longer in
higher types.  Suppose $|V| = \aleph_{\alpha}$ (there will be such an
$\alpha$).  It follows that $T(|V|) = \aleph_{T(\alpha)}$ in the next
higher type, so the strictly larger cardinal $|{\cal P}(V)| \geq \aleph_{T(\alpha)+1}$, so the
sequence is extended in length by at least one.  A similar argument for
the $\beth_{\alpha}$'s is slightly more involved.

With strong extensionality there is a much stronger restriction.
Suppose that the cardinality of type $n$ is $\aleph_{\alpha}$.  It
follows that the largest $\beth_{\beta}$ which is the cardinality of a
type $n+1$ set has $\beta\leq\alpha$.  Further, it follows that the
largest $\beth_{\alpha}$ which is the cardinality of a type $n+2$ set
has $\beta$ no greater than $T(\alpha)+2$.  Iteration of this
observation (and the natural identification of ordinals of different
types via the $T$ operation) allow us to say somewhat loosely that there
can be no $\beth_{\beta}$ in any type with $\beta \geq \alpha+\omega$.
The reason for this restriction is that there is a definable bound on
the size of each type $n+1$ in terms of the size of type $n$.

This gives a concrete motivation for the form of the axiom of
extensionality that we have chosen to use.  We do not want the size of
mathematical structures that we can consider to be strongly bounded by
the size of type 0.

With weak extensionality we can cause much larger $\beth$ numbers to
exist because we can assume that each type $n+1$ is much much larger
than the power set of type $n$ (a sufficiently large set of urelements
is added to support whatever construction we are considering). A
strong assumption which suggests itself is that we can iterate the
cardinal exponentiation operation on cardinals of sets of type $n$
objects along any well-ordering of type $n$ objects (for each type
$n$).  This would give existence of $\beth_{T^{\bf 2}(\alpha)}$ for
each ordinal $\alpha$.

It is useful to note that if we use the convention of type-free
cardinal and ordinal variables outlined above, we can treat the
exponential operation on cardinals as total.  This is achieved in the
underlying translation to typed language by providing that we work in
a type higher than that of any variable appearing in an
exponentiation: the exponential $\kappa^{\lambda}$ is then in effect
read as $T(\kappa)^{T(\lambda)}$, which is always defined.

This means that we can in effect say ``For every cardinal there is a
larger cardinal'' and ``For every ordinal there is a larger ordinal''.
$(\forall \kappa.(\exists \lambda.\lambda > \kappa))$ do not make
sense under the convention, because we have not bounded the
quantifiers.  But $(\forall \kappa < \mu.(\exists
\lambda<2^{\mu}.\kappa<\lambda))$ is true (for any specific $\mu$,
with the convention ensuring that we work in a type where $2^{\mu}$
exists), and expresses the desired thought.

\section{Sets of Reals}

topological stuff?

\section{Complex Type Theories}

complicated type theories and how they can be represented in TSTU;
Curry-Howard isomorphism stuff, perhaps.

\section{Infinite Indexed Families; K\"onig's Theorem}

\section{Partition Relations}

We begin this section by stating an obvious 

\begin{description}

\item[Theorem:] If $X$ is an infinite set, $A$ is a finite set, and
$f:X\rightarrow A$, then there is $a \in A$ such that $f^{-1}``\{a\}$
is an infinite subset of $X$.

\item[Proof:] The preimages of individual elements of $A$ under $f$
are a disjoint finite family of sets covering $X$.  The sum of their
cardinalities is the cardinality of $X$.  If all of them had finite
cardinality, this sum (the cardinality of $X$) would be finite.  But
the cardinality of $X$ is infinite by assumption.

\end{description}

The first major theorem of this section is a generalization of this.

\begin{description}

\item[Definition:] If $X$ is a set and $\kappa$ is a cardinal, we
define $[X]^{\kappa}$ as ${\cal P}(X) \cap \kappa$, the set of all
subsets of $X$ of size $\kappa$.

\item[Definition:]  If $X$ is a set, $n$ is a natural number, $A$ is a finite set, and $f:[X]^n \rightarrow A$, we say that $H$ is a homogeneous set for $f$ iff $H \subseteq X$ and $|f``[H]^n| = 1$.

\item[Theorem (Ramsey):] If $X$ is an infinite set, $n$ is a natural
number, $A$ is a finite set, and $f:[X]^n \rightarrow A$, there is an
infinite homogeneous set $H$ for $f$.

\item[Proof:] For $n=1$, the result follows immediately from the first
theorem of this section.

Assume that the theorem is true for $n=k$ and show that it follows for
$n=k+1$.  Let $X$ be an infinite set, $A$ a finite set, and
$f:[X]^{k+1}\rightarrow A$ a function.  Our goal is to show that there
is an infinite homogeneous set $H$ for $f$.

We define a tree $T_f$.  We well-order $X$ and we assume that $u\,T_f\,v$
is defined for all $u,v \leq x$.  We define $x\,T_f\,u$ as false for $u<x$.
We define $y \, T_f \,x$, for $y < x$, as true iff for all $k$-element
subsets $K$ of ${\tt seg}_{T_f}(y)$, $f(K\cup \{y\}) = f(K \cup \{x\})$.
$\leq_f$ is a tree because the order on any segment in $\leq_f$  agrees
with the underlying well-ordering of $X$.

We introduce some terminology useful in the context of trees.  The
{\em level\/} of an element $x$ of the field of a tree $\leq_T$ is the
order type of ${\tt seg}_{\leq_T}(x)$.  The {\em branching\/} of the
tree at an element $x$ of its field is the cardinality of the set of
all $y$ such that $x$ is the maximal element of the segment determined
by $y$ in the tree.  Such elements $y$ are called successors of $x$ in
the tree.

In the tree $\leq_f$, the branching will be finite at any element of
the field of the tree at finite level.  There will be nontrivial
branching above an element $y$ just in case there are elements $z$,
$w$ such that for any $k$-element subset $K$ of ${\tt
seg}_{\leq_f}(y)$, $f(K\cup \{y\}) = f(K \cup \{z\}) = f(K \cup
\{w\})$, but for some $k-1$-element subset $A$, $f(A \cup \{y,z\})
\neq f(A \cup \{y,w\})$.  A possible new branch above $y$ is
determined by the assignment of a value under $f$ to each $f(A \cup
\{y,z\})$, where $z$ is the next element of the branch.  Since there
are finitely many subsets of the segment determined by $y$ (since its
level is finite) and finitely many values in $A$, the branching at
each element of finite level is finite.  One can further prove that if
the branching at each element of a finite level is finite, each finite
level is a finite set.  It follows that some element of each finite
level is dominated by infinitely many members of $X$ in the tree
order, and further that if some element of finite level is dominated
by infinitely many elements of $X$, it has a successor that is
dominated by infinitely many elements of $X$.  From this it follows
that we can construct a branch of the tree with the property that each
of its elements of finite level is dominated by infinitely many
elements of $X$ (so it has elements of all finite levels and is infinite). 

Any branch $B$ in the tree $\leq_f$ has the property that if $b_1
\leq_f b_2 \leq_f \ldots b_k \leq_f b_{k+1} \leq_f c$, that
$f(\{b_1,b_2,\ldots,b_{k},b_{k+1}\}) = f(\{b_1,b_2,\ldots,b_{k},c\})$:
the value at a $k+1$-element subset of the branch is not changed if
the top element of the set is changed.  Thus we can define a new
function $f^*:[B]^k \rightarrow A$ by $f(\{b_1,b_2,\ldots,b_{k}\}) =
f(\{b_1,b_2,\ldots,b_{k},c\})$ for any $c \geq_f b_k$.  Now let $B$ be
the infinite branch whose existence was shown above.  By inductive
hypothesis, there is an infinite homogeneous set $H$ for $B$ with respect to
$f^*$, which will also be an infinite homogeneous set for $f$.  This
completes the proof.

\end{description}



Ramsey theorem and Erd\"os-Rado theorem: not part of the main agenda,
used for model theory of alternative set theories later.

The Schmerl partition relations, needed for theory of NFUA.

\section{Large Cardinals}

inaccessibles, Mahlos, weakly compact and measurables explained.  This
is prerequisite knowledge for the model theory of strong extensions of
{\em NFU\/}; it can also be used to talk about model theory of {\em
ZFC\/}.

\newpage

\section
{Pictures of Sets:  the Theory of Isomorphism Types of Well Founded Extensional
Relations}

In this section, we show how the type theory we are working in can
naturally motivate a development of the untyped set theory which is
more often used, as the theory of a quite natural class of
mathematical structures which has its own intrinsic interest.

\subsection{Coding Sets in Relations}

We consider the possibility that a set relation $R$ may be used to
represent the membership ``relation'' $\in$.  Toward this end, we
introduce some definitions.

\begin{description}

\item[Definition:] Let $R$ be a relation.  We say that an element $x$
of ${\tt fld}(R)$ {\em codes\/} the set $R^{-1}``\{x\} = \{y \mid y \,
R \, x\}$ relative to $R$.  (if the relation is understood in the
context we may just say that the element $x$ codes the given set).

\end{description}

The Definition ensures that a given domain element codes just one
subset of the field of the relation, but we would also like it to be
the case that a given set is coded by no more than one domain element.

\begin{description}

\item[Definition:] A relation $R$ is said to be {\em [weakly]
extensional\/} iff for all $x$ and $y$ in the field of $R$, if
$R^{-1}``\{x\} = R^{-1}``\{y\}$ then [either $R^{-1}``\{x\} =
R^{-1}``\{y\} = \emptyset$] or $x=y$.

\end{description}

A weakly extensional relation leaves open the possibility of coding a
theory of sets with distinct urelements, such as are allowed to exist
in our type theory: there may be many distinct $R$-minimal objects if
$R$ is weakly extensional, but only one if $R$ is extensional.

Because we are working with set relations, we perforce are at least
tempted to use untyped language.  For example, we can ask the question
whether there is a code for the set $\{x \in {\tt fld}(R) \mid \neg x
\,R\, x\}$ relative to the relation $R$.  The argument for Russell's
paradox shows us that there cannot be such a code (though the set
certainly exists).  In our type theory we cannot even ask the question
which leads to Russell's paradox.

A notion which is difficult (though not entirely impossible) to
develop in type theory is the notion of the collection of elements of
a set, elements of its elements, elements of elements of its elements,
and so forth (a kind of downward closure).  In the theory of coded sets
this is straightforward.

\begin{description}

\item[Definition:] Let $R$ be a relation (we do not require it to be
extensional).  Let $x$ be an element of the field of $R$.  We define
the {\em component\/} of $x$ determined by $R$ as $R \cap D_x(R)^2$,
where $D_x(R)$ is the minimal subset of the field of $R$ which
contains $x$ as an element and contains $R^{-1}``\{y\}$ as a subset
for each of its elements $y$.  We denote the component of $x$
determined by $R$ by $C_x(R)$.

\item[Theorem:] Let $R^*$ be the minimal reflexive, transitive
relation which includes $R$.  Then $C_y(R)$ is $R \cap \{x \mid x
\,R^*\,y\}^2$.

\item[Proof:] $x \in D_x(R)$ is obvious.  Suppose $x \in D_y(R)$ and
$y \in D_z(R)$.  Any set which contains $z$ as an element and which
includes $R^{-1}``\{u\}$ as a subset for each of its elements $u$ must
contain $y$ (by definition of $D_z(R)$ and the fact that $y \in
D_z(R)$) and so further must contain $x$ (by definition of $D_y(R)$
and the fact that $x \in D_y(R)$) so we have shown that $x \in
D_z(R)$.  Thus the relation $x \,S\,y$ defined as $x \in D_y(R)$ is
reflexive and transitive, so $x \,R^*\, y$ implies $x \in D_y(R)$.
Now observe that $\{y \mid y\,R^*\,x\}$ contains $x$ and includes the
preimage under $R$ of any of its elements, so must be included in
$D_y(R)$.  We now see that the field $D_y(R)$ of the component
$C_y(R)$ is precisely $\{x \mid x \,R^* y\}$, from which the result
follows.

\end{description}

There is a notion of isomorphism appropriate to weakly extensional
relations.

\begin{description}

\item[Definition:] If $R$ and $S$ are weakly extensional relations, we
say that $f$ is a {\em membership-isomorphism\/} from $R$ to $S$ if
$f$ is a bijection from the field of $R$ to the field of $S$ such that
$x\,R\,y \leftrightarrow f(x)\,S\,f(y)$ and in addition if
$R^{-1}``\{x\} = S^{-1}``(f(x)) = \emptyset$ it also follows that
$x=f(x)$.

\end{description}

We impose a further condition on relations which we regard as
simulating the membership relation, for which we need to supply a
motivation.

\begin{description}

\item[Definition:] A {\em [weak] membership diagram\/} is a
well-founded [weakly] extensional relation.

\item[Theorem:]  If $R$ is well-founded, so is $R^*-[=]$.

\item[Proof:] Suppose $A$ is a nonempty subset of ${\tt fld}(R^*-[=])$
with no $(R^*-[=])$-minimal element.  Certainly $A$ is a nonempty
subset of ${\tt fld}(R)$.  Let $a$ be an $R$-minimal element of $A$.
There must be $b \neq a$ such that $b \, R^* \, a$ (since there is no
$(R^*-[=])$-minimal element).  But from $b \, R^* \, a$, it is easy to
deduce $(\exists x.x \,R\,a)$, which is a contradiction.

\end{description}

The effect of the well-foundedness restriction is to ensure that if
$R$ and $S$ are membership diagrams and $f$ is a
membership-isomorphism from $R$ to $S$, we can be certain that $x$
with respect to $R$ and $f(x)$ with respect to $S$ always ``represent
precisely the same set''.  It is somewhat difficult to say precisely
what is meant by this (since we do not yet have an independent
understanding of untyped set theory), but a definite result which we
can state is that the membership-isomorphism $f$ is {\em unique\/}:
there can be no other membership-isomorphism from $R$ to $S$.  Suppose
there was another such membership-isomorphism $g$.  There would be an
$R$-minimal $x$ in the domain of $R$ such that $f(x) \neq g(x)$.  If
the $R$-preimage of $x$ were empty, then so would be the $S$-preimages
of $f(x)$ and $g(x)$, but further we would have $x=f(x)=g(x)$,
contradicting the choice of $x$ as a counterexample.  If the
$R$-preimage of $x$ were a nonempty set $A$, then the $S$-preimage of
$f(x)$ would be $f``A$ and the $S$-preimage of $g(x)$ would be $g``A$.
But by minimality of $x$, $f``A = g``A$, so by extensionality of $S$,
$f(x)=g(x)$, contradicting the choice of $x$ as a counterexample.

The informal argument that each element of $x$ designates the same set
relative to $R$ that is designated by $f(x)$ with respect to $S$ has
the same form, but has an essential vagueness dictated by the fact
that we are not actually previously acquainted with the domain of sets
being designated.  If the $R$-preimage of $x$ is empty, then $x=f(x)$:
the two objects represent the same atom.  If the $R$-preimage of $x$
is a nonempty set $A$, then the $S$-preimage of $f(x)$ is $f``A$.  $x$
designates (with respect to $R$) the collection of things designated
by elements of $A$ with respect to $R$.  By the minimality hypothesis,
the things designated with respect to $S$ by elements of $f``A$ are
the same: so $f(x)$ designates the same collection with respect to $S$
that $x$ designates with respect to $R$.

Further, if we are working with relations that are extensional rather
than weakly extensional, the argument above works with isomorphism in
place of membership-isomorphism.

General well-founded relations can be ``collapsed'' to well-founded
[weakly] extensional relations in a suitable sense.

\begin{description}

\item[Theorem:] Let $R$ be a well-founded relation.  Then there is a
uniquely determined equivalence relation $\sim$ on ${\tt fld}(R)$ with
the following property (in which we use the notation $[x]$ for
$[x]_{\sim}$): the relation $R_{\sim} = \{\left<[x],[y]\right>\mid
x\,R\,y\}$ is [weakly] extensional and for each $[x]$ we have the set
of its $R_{\sim}$-preimages exactly the set of $[y]$ such that
$y\,R\,x$.

\item[Proof:] Let $x$ be minimal in $R$ such that $C_x(R)$ does not
have this property.  (Clearly if there is no such $x$, then the unions
of uniquely determined equivalence relations on all $C_x(R)$'s with
the indicated property will give such an equivalence relation on $R$.)
Each $C_y(R)$ for $y \,R\, x$ will support such a unique equivalence
relation, if it is nonempty.  We define the desired equivalence
relation on $C_x(R)$, contrary to hypothesis.  The top $x$ is
equivalent only to itself.  All $R$-preimages of $x$ which have empty
$R$-preimage are either equivalent only to themselves (if we are
working with membership-isomorphism) or equivalent to all such
preimages (if we are working with isomorphism).  Each other element
$y$ of $C_x(R)$ has an associated equivalence relation $\sim$ and
relation $C_y(R)_{\sim}$: define $y \sim z$ as holding if and only if
$C_y(R)_{\sim}$ is [membership]-isomorphic to $C_z(R)_{\sim}$.  By
hypothesis the restriction of the equivalence relation to each proper
component is unique.  Extensionality (and the known uniqueness of
isomorphisms between well-founded [weakly] extensional relations]
leaves us no freedom of choice with respect to defining the
equivalence relation between elements of different components.  So the
equivalence relation obtained is unique.

\end{description}


To see why non-well-founded ``membership diagrams'' are problematic,
consider a diagram containing two elements $x$ and $y$, each related
just to itself.  This codes two sets, each of which is its own sole
element.  Consider another diagram containing two elements $u$ and
$v$, each related just to itself.  Either of the two bijections
between the fields of these relations is a membership-isomorphism (and
indeed an isomorphism) between the relations: there is no way to
determine whether $x$ is to be identified with $u$ or with $v$.

It should be noted that non-well-founded ``membership diagrams'' are {\em
merely\/} problematic, not impossible.  Interesting untyped theories
can be developed in which there are objects which are their own sole
elements (and in which there can be many such objects), and in fact we
will have occasion to see this later.  Indeed, arbitarily complex
failures of well-foundedness of the membership relation are possible
and worthy of study.

\newpage

\subsection{Passing to Isomorphism Types}

The advantage of restricting ourselves to well-founded [weak]
membership diagrams is that for any element $x$ of the field of a
well-founded membership diagram $R$, the intended reference of $x$ is
in effect fixed by the [membership-]isomorphism type of the component
$C_x(R)$.  We can then view the [membership-]isomorphism types of
components of diagrams as the actual objects under study.  When
studying weak membership diagrams, there is an element of
arbitrariness in the choice of atoms, though it is sometimes useful to
have atoms in untyped set theory.  The isomorphism types of
well-founded extensional relations will be our principal study, and we
will see that they correspond precisely to the objects of the usual
untyped set theory, though without strong assumptions we will not see
the {\em entire\/} universe of the usual set theory [in whatever sense
this is possible].

\begin{description}

\item[Observation:] If a [weak] set diagram $R$ is equal to $C_x(R)$
and to $C_y(R)$ where $x$ and $y$ belong to the field of $R$, then
$x=y$.  This condition implies $x \, R^* \, y$ and $y \, R^* x$.
Since $R^*-[=]$ is well-founded, an $(R^*-[=])$-minimal element of
$\{x,y\}$ must be equal to both $x$ and $y$, so $x=y$.

\item[Definition:] A {\em weak set diagram\/} is a weak membership
diagram which is equal to one of its components (and thus must be
nonempty).  A {\em set diagram\/} is a membership diagram which is
either empty or equal to one of its components.  A {\em top\/} of a
[weak] set diagram is either the unique $x$ such that the diagram is
its own component determined by $x$ or (in case the diagram is empty)
any object whatsoever.  A {\em [weak] set picture\/} is the
[membership-]isomorphism class of a [weak] set diagram [or a double
singleton (representing an atom)].  The set of all set pictures is
called $Z$.  The set of all weak set pictures whose elements have
atoms restricted to a set $A$ is called $Z[A]$ (this last will contain
only double singletons of elements of $A$; of course $Z[V]$ contains
all weak set pictures).

\item[Definition:] For any [weak] set diagram $R$ with top $t$, we
define an {\em immediate component\/} of $R$ as a component $C_x(R)$
such that $x\,R\,t$.  Note that the empty set diagram has no
immediate components, but may occur as an immediate component of a
set diagram if the $x \,R\,t$ happens to have empty $R$-preimage: the
handling of elementless objects in weak set diagrams is seen below.
For set pictures $\rho$ and $\sigma$, we define $\rho\,E\,\sigma$ as
holding iff there are $R \in \rho$ and $S \in \sigma$ such that $R$ is
an immediate component of $S$.  For weak set pictures $\rho$ and
$\sigma$, we define $\rho\,E\,\sigma$ as holding iff there are $R \in
\rho$ and $S \in \sigma$ such that $R$ is an immediate component of
$S$, or $\rho$ is a double singleton $\{\{x\}\}$, and $\sigma$ has an
element $S$ with top $t$ such that $x\,S\,t$ and the $S$-preimage of
$x$ is empty (this handles atoms).  It is important to note that no
double singleton is a membership-isomorphism class of weak set
diagrams, so there is no conflict between the two parts of the
definition of $E$ on weak set pictures (the double singleton of the
empty set is a set picture, and the sole elementless object in the
``set theory'' implemented using set diagrams).

\item[Theorem:]  $E$ is a membership diagram (on $Z$ or $Z[A]$).

\item[Proof:] We need to show that $E$ is [weakly] extensional and
that $E$ is well-founded.  Suppose that $\rho$ and $\sigma$ are [weak]
set pictures and $E^{-1}``\{\rho\} = E^{-1}``\{\sigma\}$.  This means
that each immediate component of any $R\in \rho$ is isomorphic to some
immediate component of any $S\in \sigma$ and vice versa. [In the weak
case, any preimage of the top of $R$ which has empty $R$-preimage is
identical to some preimage of the top of $S$ which has empty
$S$-preimage, and vice versa].  There is a unique isomorphism from the
field of each immediate component of $R$ to a uniquely determined
immediate component of $S$ (because no two distinct immediate
components can be isomorphic).  Any two of these isomorphisms will
agree on any common element of their domains.  It follows that the
union of these isomorphisms, taken together with the pair whose first
projection is the top of $R$ and whose second projection is the top of
$S$, yields a [membership-]isomorphism from $R$ to $S$, so
$\rho=\sigma$.  [The fact that it is a membership-isomorphism in the
weak case follows from the bracketed complete sentence above: elements
of $E^{-1}``\{\rho\}$ and $E^{-1}``\{\sigma\}$ which are double
singletons each correspond to identical elements of the other, and
this allows one to define the isomorphism so that it fixes all
elements with empty $E$-preimage].  We have shown that $E$ is [weakly]
extensional.

Suppose that $A$ is a nonempty subset of the field of $E$.  Let $\rho$
be an element of $A$ and let $R \in \rho$.  [If $R$ is a double
singleton, $R$ is $E$-minimal and we are done.]  Define $A_R$ as the
intersection of $A$ with the set of isomorphism types of components
$C_x(R)$ [and double singletons of $R$-minimal elements of the field of
$R$].  There will be a minimal $x$ such that the isomorphism type of
$C_x(R)$ belongs to $A$ [or there will be a double singleton which
belongs to $A$]; the isomorphism class of $C_x(R)$ [or the double
singleton] will be an $E$-minimal element of $A_R$, and so an
$E$-minimal element of $A$.

\item[Observation:] Note that $E$ is two types higher than the [weak]
membership diagrams $R$ with which we started.  If $x$ in the field of
$R$ is at type $k$, then $R$ itself is at type $k+1$, the
[membership-]isomorphism class of $R$ is at type $k+2$, and $E$ is at
type $k+3$.  We see that $E$ is two types higher than the arbitrary
membership diagrams with which we started.  $E$ is a kind of universal
membership diagram, but this type differential will allow us to
completely naturally evade any supposed paradoxical consequences of
this universality.  The situation here is analogous to that for
ordinals: the well-ordering on all ordinals is a kind of universal
well-ordering -- it contains not a suborder isomorphic to each
well-ordering $R$ but a suborder isomorphic to the double singleton
image $R^{\iota^{\bf 2}}$ of each well-ordering $R$.  It is also worth
noting that strict well-orderings with maxima (and the empty strict
well-ordering) are well-founded extensional relations, so there are
elements of $Z$ (or $Z[A]$) naturally related to the ordinal numbers
(and indeed these correspond precisely to the objects (the {\em von
Neumann ordinals\/}) which are normally taken to be the ordinal
numbers in the usual set theory).  One must observe though that a
nonzero ordinal $\alpha$ is implemented in untyped set theory by the
isomorphism class of the strict well-ordering derived from a
well-ordering of order type $\alpha+1$.

\end{description}

There is a type-shifting operation $T$ on [weak] set pictures analogous to
the operations on cardinals and ordinals.

\begin{description}

\item[Definition:] For any [weak] set diagram $R$, define $R^{\iota}$
as usual: this will still be a [weak] set diagram.  Let $\rho$ be the
[membership-]isomorphism class of $R$: then $T(\rho)$ is defined as
the [membership]-isomorphism class of $R^{\iota}$, and it is
straightforward to show that the specific choice of an element $R$ of
$\rho$ has no effect on the definition of $T(\rho)$.  Notice that in
the case of weak set diagrams, atoms are replaced by their singletons
as we pass up one type.  [Define $T(\{\{x\}\})$ as $\{\{\{x\}\}\}$].

\newpage


\item[Theorem:] For all [weak] set pictures $\rho$ and $\sigma$, $\rho
\,E\, \sigma \leftrightarrow T(\rho) \,E\, T(\sigma)$.

\item[Proof:] This follows directly from the precise parallelism of
the structure of $S\in \sigma$ with the structure of $S^{\iota} \in
T(\sigma)$.  If $\rho \,E\, \sigma$, any $S\in \sigma$ has an
immediate component $R \in \rho$, so belonging to $\rho$: it is
immediate that $S^{\iota}\in T(\sigma)$ has an immediate component
$R^{\iota}$ belonging to $T(\rho)$, so $T(\rho) \in T(\sigma)$.
Suppose $T(\rho) \in T(\sigma)$.  Then we can choose an element of
$T(\sigma)$ of the form $S^{\iota}$ where $S \in \sigma$, which will
have an immediate component $R^{\iota} \in T(\rho)$ (any component of
$S^{\iota}$ is obviously a relation singleton image), from which we
discover $R \in \rho$, so $\rho \in \sigma$.  [If the top of $S \in
\sigma$ has an immediate preimage $x$ with empty $S$-preimage, and
$\rho = \{\{x\}\}$, then the top of $S^{\iota}$ has an immediate
preimage $\{x\}$, so $\{\{\{x\}\}\} = T(\rho) \,E\, T(\sigma)$ in this
case as well; if $T(\rho) \in T(\sigma)$ where $\rho=\{\{x\}\}$, the
top of $S^{\iota} \in T(\sigma)$ has an immediate preimage $\{x\}$
with empty $S^{\iota}$-preimage [recall that we can without loss of
generality choose an element of $\sigma$ of the form $S^{\iota}$], we
see that the top of $S \in \sigma$ has the preimage $x$ with empty
$S$-preimage, so $\{\{x\}\} = \rho \,E\,\sigma$].

\item[Theorem:] For each $\rho \in Z$ [$Z[A]$] we have $C_{\rho}(E)
\in T^{\bf 2}(\rho)$.

\item[Proof:] Let $R \in \rho$.  Define $\rho_x$ as the isomorphism
type of $C_x(R)$ for $x \in {\tt fld}(R)$ [or as $\{\{x\}\}$ if $x$ is
$R$-minimal.]  The $\rho_x$'s are exactly the elements of
$D_{\rho}(E)$, $\rho_x \,E\, \rho_y$ iff $x\,R\,y$, but $\rho_x$ is
two types higher than $x$, so we can define a [membership-]isomorphism
sending each $\{\{x\}\}$ to $\rho_x$, witnessing the desired relation between
$R^{\iota^{\bf 2}}$ and $C_{\rho}(E)$.

\item[Theorem (using Choice):] Every subset of $T``Z[A]$ is coded in
$E$.  Every subset of $T``Z$ is coded in $E$.

\item[Proof:] Let $B$ be an arbitrary subset of $T``Z$ [$T``Z[A]$].
Each element of $B$ is of the form $T(\rho)$ We transform each
$R^{\iota}\in T(\rho)$ for each $\rho\in B$ to a different $R'$ still
belonging to $T(\rho)$: $R' =
\{\left<\left<\{x\},R\right>,\left<\{y\},R\right>\right>\mid
x\,R\,y\}$.  The collection of relations $R'$ is pairwise disjoint, so
we can take their union and adjoin all pairs $\left< t,T\right>$ as
new elements, where $t$ is the top of one of the $R'$'s and $T$ is a
fixed new top element (any pair whose second projection does not
belong to $B$ will do).  The resulting relation is well-founded and
has immediate components of exactly the right isomorphism classes, but
it is not extensional.  By the theorem proved above on collapsing
well-founded relations to well-founded [weakly] extensional relations,
we can define an equivalence relation on its field and replace each
element of the field by a representative of its equivalence class
taken from a fixed choice set in such a way as to obtain a [weak] set
diagram which has immediate components with isomorphism classes which
are all and only the elements of $B$.

\item[Theorem (not using Choice):] Every subset of $T^{\bf 2}``Z[A]$ is
coded in $E$.  Every subset of $T^{\bf 2}``Z$ is coded in $E$.

\item[Proof:] Let $B$ be a subset of $T^{\bf 2}``Z$ [$T^{\bf
2}``Z[A]$].  Each element $T^{\bf 2}(\rho)$ of $B$ has a {\em
canonical\/} representative, namely $C_{\rho}(E)$.  These relations
all agree on shared members of their domains (since they are all
subsets of $E$).  Add a new top element $T$ and add all pairs
$\left<\rho,T\right>$ for $T^{\bf 2}(\rho) \in B$ as elements to their
union to obtain a relation with the correct isomorphism classes of
immediate components.

\item[Observation:] The membership diagram $E$ in higher types
faithfully reproduces the membership diagrams in the $E$ relations in
lower types.  Moreover, the $E$ relation in higher types is {\em
complete\/} in an obvious sense on its copy of the domains of the $E$
relation of lower types: it codes all subsets of the domains at lower
types, whereas a specific $E$ relation cannot code all subsets of {\em
its own\/} domain.  For example, a specific relation $E$ cannot code
its own field $Z={\tt fld}(E)$, because it is a well-founded relation
(a code $v$ for the entire field of $E$ would satisfy $v \,E\, v$).
But $T``{\tt fld}(E)$ is coded (in $E$ of a higher type) from which
we can see that more sets are coded in higher types.

\end{description}

\newpage


\subsection{The Hierarchy of Ranks of Set Pictures}

We introduce the analogue here of the cumulative hierarchy of sets in
the usual set theory -- without atoms.  From this point on we restrict
ourselves to membership diagrams, though the results for weak
membership diagrams are quite similar.

\begin{description}

\item[Definition:] For any set $A \subseteq {\tt fld}(E)$, we define $P(A)$
as the set of elements of ${\tt fld}(E)$ which code subsets of $A$.
We say that the subset $A$ is {\em complete\/} if $P(A)$ contains
codes for all subsets of $A$.  Notice that $P(A)$ has the same type as
$A$.

\item[Definition:]  We define the set of {\em ranks in $E$\/} as the intersection of all sets $H$ such that $\emptyset \in H$, $(\forall h \in H.P(h) \in H)$,
and $(\forall A \subseteq H.\bigcup A \in H)$.

\item[Theorem:] The set of ranks itself contains $\emptyset$, is
closed under $P$ and closed under unions of sets of ranks.  The ranks
in $E$ are well-ordered by inclusion.

\item[Theorem:]  ${\tt fld}(E)$ is a rank.

\item[Definition:] Let $\mathbb E$ denote the inclusion order on ranks
in $E$. Then ${\mathbb E}_{\alpha}$ is a general notation for ranks
using our convention on ordinal indexing.

\item[Definition:] Let $\gamma$ be the ordinal such that ${\mathbb
E}_{\gamma}$ is the first incomplete rank.

\item[Theorem:] ${\mathbb E}_{\omega+n}$ is a complete rank in a high enough
type for each familiar natural number $n$.

\item[Theorem:]  $|{\mathbb E}_{\omega + \alpha}| = \beth_{\alpha}$ if
${\mathbb E}_{\omega + \alpha}$ is complete.

\end{description}

The ranks code an iterative process for constructing sets by iterating
the ``power set'' construction which may go through stages indexed by
infinite ordinals.  This is reminiscent of how the world of our type
theory is constructed, except that we lack the ability (or indeed the
need) to pass to transfinite {levels.\footnote{We will explore further
the question as to whether type theory suffers from the lack of
transfinite levels.  But notice that we are able to discuss the
transfinite levels of the cumulative hierarchy in type theory here,
and the possible presence of urelements means that the hierarchy will
not necessarily be truncated at any definite point as it would be in a
strongly extensional development of type theory}.}

The set pictures are isomorphism classes supporting a $T$ operation, so
we can introduce type free variables ranging over set pictures using
the conventions introduced above.  Each set picture variable needs to
be restricted to some definite type, which can be viewed as
restriction of the variable to some set of set picture variables (in
higher types) which can in turn be viewed as restriction of the
variable to the preimage under $E$ of some set picture (if we go up
one more type so that all elements of the original type are images
under T so we have completeness).  Just as we represented the bounding
of ordinal variables in types as bounding in the segment determined by
an ordinal variable, we can represent the bounding of set picture
variables in types or sets within types as a bounding in the preimage
of a set picture under $E$.

The self-contained theory of set pictures thus obtained is an untyped
set theory with $E$ as its membership. 

We outline the proofs of some important theorems of this untyped theory.

\begin{description}

\item[Theorem:] For every set picture $\sigma$ and every formula
$\phi$, there is a set picture $\tau$ such that $(\forall
\rho\,E\,\tau.\rho\,E\,\sigma)$ and $(\forall \rho \in \sigma.\rho \in
\tau \leftrightarrow \phi)$.  

\item[Proof:] Our conventions ensure that we work in a type where
$\sigma=T(\sigma')$ for some $\sigma'$, and the result then follows
from theorems given above: the image under T of any set of set
pictures is coded.

\item[Theorem:] For every set picture $\sigma$, the set of all codes
of subsets of the preimage of $\sigma$ under $E$ is coded.

\item[Proof:] Just as with the result that cardinal exponentiation is
total in the untyped theory of cardinals, this is achieved by clever
definition of our conventions.  We stipulate that if any set picture
$\sigma$ is mentioned, we work in a type high enough that
$\sigma=T(\sigma')$ for some $\sigma'$.  This ensures that any
subcollection of the preimage of $\sigma$ under $E$ is coded (the
burden of the previous theorem) and is further itself also an image
under T, so the collection of all these subsets is also coded (though
it is not necessarily an image under T).  Note that if we further
mention this set (for a specific $\sigma$) we bump ourselves into a
yet higher type (so we can iterate this ``power set'' operation
any concrete finite number of times).

\end{description}

\newpage

\section{Category theory}

We give a brief introduction to category theory as carried out in type theory.

\begin{description}

\item[Definition:]  A category is a tuple $\left<O,M,d,r,\circ,{\tt id}\right>$, where $O$ is the set of {\em objects\/} of the category and $M$ is the set of {\em morphisms\/} of the category, $d:M \rightarrow O$ is the {\em domain\/} function of the category, $r:O \rightarrow M$ is the {\em codomain\/} function of the category, and $\circ \subseteq (M \times M) \times M$ is the {\em composition\/} operation of the category (which is partial, so we do not write $\circ \subseteq {M \times M} \rightarrow M$). The function ${\tt id}:O \rightarrow M$ is the identity morphism constructor.

Certain conditions must be satisfied which are necessary and sufficient for this to be a category.  For each pair of morphisms $f,g$ there is at most one
$h$ such that $\left<\left<f,g\right>,h\right> \in \circ$, and we write $h=f \circ g$ if it exists.  The condition for $g \circ f$ to exist is exactly $r(f)= d(g)$,
and $d(g \circ f)=d(f)$, $r(g \circ f) = r(g)$, for any morphisms $f,g$ for which $g \circ f$ is defined.
$f\circ (g\circ h) = (f \circ g) \circ h$ whenever the compositions involved exist.  ${\tt id}(r(f)) \circ f = f \circ {\tt id}(d(f)) = f$ for all morphisms $f$.

\end{description}

We could economize on components in our tuple if we identified each object $A$ with ${\tt id}(A)$;  $d$ and $r$ would then be functions $M \rightarrow M$
and {\tt id} and $O$ would not be needed as components.  However, in natural examples we do not tend to identify objects with their identity morphisms.

A specific concrete example of a category is the category of sets and functions in any particular type.  The objects of this category are all the sets of
a type $n$.  The morphisms $f$ of the category with $d(f)=A$ and $r(f)=B$ are just exactly the functions $f:A \rightarrow B$.  If $A$ is an object,
that is a set, ${\tt id}(A)$ is the restriction of the identity function to $A$.  The composition operation is the usual operation of composition of functions, restricted appropriately.

The alert reader may notice that we are lying (we are doing so quite deliberately, but will promptly atone).  The difficulty is that for each function $f$ we do not have a unique $r(f)$ such that $f:{\tt dom}(f) \rightarrow r(f)$.  We fix this by clarifying that our ``functions" are actually pairs $\left<f,B\right>$
where $f$ is a function in the usual sense and  ${\tt rng}(f) \subseteq B$, and defining ${\tt id}(A)$ as $\left<{\tt id} \lceil A,A\right>$, the identity function restricted to A, paired with $A$,
and $\left<g,C\right> \circ \left<f,B\right>$ as $\left<g \circ f,C\right>$, where the composition on the left is the usual composition of functions.  Of course $d(\left<f,b\right>) = {\tt dom}(f)$ and $r(\left<f,B\right>)=B$.  We will call these objects ``typed functions" and we may now and then confuse them with their first projections when their intended codomain is evident from context.

We could also have defined $r(B)$ as ${\tt rng}(B)$, but this would have given a different category.  Notice that with this definition the set of morphisms from a set $A$ to a set $B$ would be the set of functions from $A$ {\em onto\/} $B$, so we might call this the category of surjective functions (it contains exactly the same functions, but organized differently).

For any objects $A, B$, we define ${\tt hom}(A,B)$ as the set of morphisms $f$ with $d(f)=A$, $r(f)=B$.  We call these sets homsets.

Just for fun, I provide an alternative representation of category theory which might indicate that it can be freed from dependency on notions of function per se.

A {\em multigraph\/} is a triple  $\left<V,E,g\right>$ where $V$ is the set of vertices of the graph, $E$ is the set of edges, and $g:E \rightarrow V \times V$ tells us where each edge starts and ends:  if $g(e) = \left<a,b\right>$, then $e$ is an edge from $a$ to $b$.  A {\em path} is a multigraph is a finite sequence $p$ of odd length in which $p_{2n}$ is always a vertex and $p_{2n+1}$ is an edge and satisfies $g(p_{2n+1}) = \left<p_{2n},p_{2n+2}\right>$.  A path $p$ with domain $[0,n]$ is said to be a path from $p_0$ to $p_n$.  The concatenation $p \oplus q$ of $p$ and $q$, where $p$ has domain $[0,n]$, $q$ has domain $[0,m]$, and $p_n=q_0$ is defined thus, with domain $[0,m+n]$: $(p\oplus q)_i$ is either $p_i$ or $q_{i-n}$, as appropriate.  A category is then determined by an equivalence relation on paths in a multigraph, with the properties that if $p \sim q$ and $p$ is a path from $a$ to $b$, $q$ is also a path from $a$ to $b$, and which respects concatenation:  if $p \sim r$ and $q \sim s$, then $p \oplus q \sim r \oplus s$.
If the additional condition is imposed that each equivalence class contains exactly one path with domain $[0,0]$ or $[0,2]$ (a path determined by a single edge or vertex), there is a precise correspondence between multigraphs and categories (notice that such multigraphs must contain edges from every $a$ to every $b$ for which there is a path from $a$ to $b$,  where $a \neq b$; it may or may not contain loops at each vertex, and loops will not be equivalent to paths with domain $[0,0]$); in other cases, non-isomorphic multigraphs may generate isomorphic associated categories, basically by adding lots of virtual edges to the underlying multigraph and collapsing some actual edges together.  Notice that in this formulation the morphisms of the category, being equivalence classes of paths, are at a higher type than the objects of the path, which are the vertices.
In the restricted formulation with a single path in each equivalence class with domain $[0,0]$ or $[0,2]$, we can instead use the single edge in the range
of that path as the morphism associated with the equivalence class (or the single vertex if there is no edge), and that is at the same type as the vertices/objects.  In the more general case, we note that paths can be represented at the same type as the edges and vertices in them, by building these finite structures using pairing instead of membership, for example, and then representative paths can be chosen from each equivalence class to serve as morphisms.

The objects of a category are often (but not always) structured sets of some sort and the morphisms are often (but not always) functions which preserve this structure.
For example, there is a category of all groups, in which the morphisms are homomorphisms, and a category of topological spaces, in which the morphisms are homeomorphisms.  A category which is not exactly of this kind is the category whose objects are topological spaces and whose morphisms are equivalence classes of continuous functions under homotopy.

Reflexively and perhaps worryingly, we can define a category of categories.  If $C$ and $D$ are categories (with components as above which we will subscript with their names) a functor from $C$ to $D$ is a function $F$ sending objects of $C$ to objects of $D$ and satisfying $r_D(F(A)=F(r_C(A)),
d_D(F(A)=F(d_C(A)), F(f \circ_C g) = F(f) \circ_D F(g), F({\tt id}_C(A))={\tt id}_D(F(A))$.  A functor preserves category theoretic structure.  Notice that the image of the functor may not be all of $D$ and the map $F$ may not be an injection.  Now, there is a category of categories whose objects
are all the categories and whose morphisms are the functors between categories.

We haven't told the whole story here!  The alert reader should notice that the category of all type $n$ categories must actually be of
type $n+1$, since it is a tuple one of whose components is the set of type $n$ categories.  The concrete example given above, the category of type $n$ sets and functions, is itself a type $n+1$ category.

A category considered as such (without information about the specific natures of its objects and morphisms) is a sort of infinite diagram with dots (objects) connected by directed arrows (morphisms) and a notion of composition which ensures that any path made up of directed arrows can be identified with a single directed arrow.  There is nothing more to it.  Properties of categories which are commonly articulated are often of a character which ensures that a category really does have the structure of some kind of system of sets and functions; at any rate, this is the character of the properties we will introduce.

For example, in the category of sets and typed functions, each singleton set $A=\{a\}$ has the property that there is exactly one arrow (we may use the word ``arrow" to mean ``morphism")  from any object $B$ to $A$ (the constant function with the appropriate value $a$ on $B$).  An object with this property in a general category is called a {\em terminal object\/} for that category.  The empty set $\emptyset$ has the property that there is exactly  one arrow from $\emptyset$ to $A$ for any set $A$:  an object with this property in a general category is called an {\em initial object\/}.  The category of sets and typed functions has just one initial object and many terminal objects, but there is a sort of uniqueness for terminal objects:  for any two terminal objects $A$ and $B$, there is exactly one arrow from $A$ to $B$ and exactly one arrow from $B$ to $A$, and their composition must be
the unique arrow from $B$ to $B$, which is ${\tt id}(B)$.  If $f \circ g = {\tt id}(A)$, we say that $f$ and $g$ are inverses, and that $f$ and $g$ are {\em isomorphisms\/}.  All terminal objects in any category are isomorphic to one another.  Similarly, all initial objects in any category are isomorphic to one another.  When there is an isomorphism from $A$ to $B$, composition with the isomorphism gives an exact correlation of structure between
homsets involving $A$ and corresponding homsets involving $B$.

We describe a situation under which we say that a category ``has products".   For any objects $A$ and $B$, if we believe that $C = A \times B$,
by analogy we expect to have morphisms $\pi_1:C \rightarrow A$ and $\pi_2:C \rightarrow B$ such that for any object $D$ and morphisms $f:D \rightarrow A$ and $D \rightarrow B$, there is a unique morphism $f \times g:D \rightarrow C$ such that $\pi_1 \circ (f \times g) = f$
and $\pi_2\circ (f \times g) = g$.   Notice that the existence of a product of $A$ and $B$ (or of products of any two objects in a category) is not just the assertion of the existence of an object $C = A \times B$ but of the existence of $C$ equipped with ``projection maps" $\pi_1$ and $\pi_2$.
It is straightforward to see that any two products of $A$ and $B$ are isomorphic (though there may be more than one of them!)  The idea
is that $f \times g$ is the function sending $\left<x,y\right> \in A \times B$ to $\left<f(x),g(y)\right>$; of course this is uniquely determined and has the indicated property, but in a general category we do not know that the objects are functions in the usual sense.  The cute thing about this definition is that it not only singles out the cartesian product of sets (up to a bijection) in the category of sets, but it also picks out a correct notion of product
of objects and product of functions in other categories, as for example groups or topological spaces.

Any category $C$ can be transformed into a ``converse" category $C^{\tt op}$, called the {\em opposite\/} or {\em dual\/} category of $C$, by reversing the direction of all arrows, and similarly it is often useful to reverse arrows in a property.  Suppose that for objects $A$ and $B$ we have an object $C$ such that we have morphisms $\sigma_1:A \rightarrow C$ and $\sigma_2:B \rightarrow C$ and for any object $D$ and arrows $f:A \rightarrow D$ and $g:B \rightarrow D$, we have a uniquely determined arrow $f+g$ such that
$(f+g) \circ \sigma_1=f$ and $(f+g) \circ \sigma_2=g$.  Curiously, we have described a construction in the standard theory of sets and functions:
for any sets $A,B$, the disjoint union $(A \times \{0\}) \cup (B \times \{1\})$ equipped with the maps $\sigma_1 = (\lambda a \in A:\left<a,0\right>)$
and  $\sigma_2 = (\lambda b \in B:\left<b,1\right>)$ has these properties.  It may not seem obvious that cartesian product and disjoint union are ``dual" operations, but from the category theoretic standpoint that is how things look.

We continue with a development of ``function spaces" internally to a category.  We want to say what it means for there to be a category $B^A$ which
is in effect the set of functions from $A$ to $B$.  The alert reader will see that we are doing violence to our scheme of types as we work, but we will fix everything up before the end.

The idea for representing $B^A$ using category theory concepts comes from the notion of ``currying", popular in computer science for converting
functions with two arguments to functions with one argument:  where we have a function $f(x,y)$, define a related function $\hat{f}$ such that
$\hat{f}(x)(y)=f(x,y)$.  We note that to keep types straight we actually need $\hat{f}(\{x\})(y)=f(x,y)$.   Doing violence to types which we will duly fix, we want a map ${\tt ev}$ which sends a pair $\left<f,a\right>$ in $B^A \times A$ to $f(a)$ in $B$.  The type fix is the following:  we in fact
consider $\iota``A$ and $\iota``B$ and the set $B^A$ and  provide the map ${\tt ev}$ such that ${\tt ev}(\left<f,\{a\}\right>) = \{f(b)\}$.

We now express the defining property of the exponential $C^B$ and its associated ${\tt ev}:C^B \times B \rightarrow C$ arrow:  For any object $A$ and any arrow $f:A \times B \rightarrow C$, there is a unique arrow $\hat{f}: A \rightarrow C^B$ such that ${\tt ev} \circ (\hat{f} \times 1_B) = f$, where $1_B$ is the map from $B$ to a convenient terminal object.  In our category of sets and typed functions, exponentials $C^B$ exist for any sets of singletons $C,B$, with $C^B$ implemented as the set of functions from $\bigcup B$ to $\bigcup C$ and the map {\tt ev} sending each pair $\left<f,\{b\}\right>$ to $\{f(b)\}$.  The map $\hat{f}$ will be the unique map satisfying $\hat{f}(a) (b)  \in f(a,\{b\})$.  This is easily implemented for all sets  $B,C$ the same size as a set of singletons, as well.

A cartesian closed category is one in which there is a terminal object and there are products and exponentials for every pair of objects.  This cannot be the case in the full category of typed sets and functions, as each arrow $f:B \rightarrow C$ would of course correspond to a unique  arrow $f^*:1 \times B \rightarrow C$ and so would be required to correspond to a unique arrow $\hat{f^*}:1 \rightarrow C^B$:  now observe that if $B=C=V$, there are more functions from $V$ to $V$
than there can be functions from a terminal object (singleton set) to any object (there are more functions from $V$ to $V$ than there are elements of $V$, and so of any object).  This has nothing to do with the specific implementation of cartesian closedness above (which of course does not work for the whole category of typed sets and functions):  it is an argument that no implementation can work.

Say that a set $A$ is small if ${\tt exp}^n(|A|)$ exists for each $n$:  the category of small sets and typed functions between small sets is cartesian closed.  This is a hint as to what the world of sets of untyped set theory has to look like:  sets must be restricted to be small in a suitable sense, as we will see in the next chapter.

We define a notion of {\em small category\/} motivated by cardinality features of the category of small sets and functions just discussed.  Notice
that the homsets of the category of small sets and typed functions between them are small, but moreover one has to pay attention to the type relative to which they are small.  The category of small sets and functions of type ${\bf k+1}$ has homsets at type ${\bf k+2}$ whose cardinalities (in type ${\bf k+3}$) are images under $T$ of
small cardinals of type ${\bf k+2}$ (cardinalities of type ${\bf k+1}$ sets).  So we define a small category as a category whose homsets have cardinalities which are images under $T$ of small cardinals.  The appearance of the $T$ operator forces the correct typing.  Notice that there is no assertion that the set of objects is bounded in size:  it is the collections of arrows between any given pair of objects which are being bounded in size.

An important idea in category theory is that of a {\em natural transformation\/}.   Given two functors $S,T$, both from a category $C$ to a category $B$, a natural transformation from $S$ to $T$ is a function $\tau$ from objects of $C$ to morphisms of $B$ such that $\tau(c)$ is a morphism from
$S(c)$ to $T(c)$ and for any morphism $f$ in $C$ with $d(f)=c$, $r(f)=c'$ we have $T(f) \circ \tau(c) = \tau(c') \circ S(f)$.




\newpage

\chapter{Untyped theory of sets}

In this chapter we introduce the usual untyped set theories (Zermelo
set theory and the stronger {\em ZFC\/}, as well as some intermediate systems) and relate them to type
theory.  We will present (at the end) the view that untyped set theory can be
interpreted as the theory of set pictures (isomorphism types of
certain well-founded extensional relations), which should already be
suggested by the treatment at the end of the previous chapter.

Further, we strongly criticize the idea that the axioms of Zermelo set
theory are somehow essentially {\em ad hoc\/}, as is often suggested
(this is stated with great confidence so often as to be clich\'e).
There are some odd features of the earliest form of the axioms, which
reflect the fact that they appear early in the process of
understanding what can be done with set theory, but Zermelo set theory
is very close to being exactly the abstract theory of set pictures,
and this is not {\em ad hoc\/}.  I do think that something is missing from the formulation of
Zermelo set theory as an independent theory:  adding either the Axiom of Rank or the combination of Foundation and the Mostowski Collapsing Lemma gives a theory with the same mathematical strength and much more satisfactory technical features.

In untyped set theory there is only one kind of object -- sets.  There
may also be atoms if extensionality is weakened to allow them but they
will not be an essentially different sort (type) of object.  Though
this may seem to be quite a different kind of theory, we will see that
the usual untyped set theory is not so distantly related to the typed
theory of sets we have developed as you might think.

Subsections of this section which depend strongly on the presentation of type theory in the previous section are marked with $\dagger$, as are local remarks with such dependencies;  subsections which are part of a mostly self-contained treatment of untyped set theory are unmarked.

\newpage

\section{The original system of Zermelo}

The first modern axiomatic system of set theory was proposed by
Zermelo in 1908.  It is even older than the first publication of the
famous {\em Principia Mathematica\/} of Russell and Whitehead, though
not as old as Russell's first proposal of the theory of types.

The axioms differ somewhat from those in modern treatments.  In this
theory, we have primitive predicates of membership and equality, and
all objects are of the same sort (there are no type restrictions in
our language).

\begin{description}

\item[Axiom of Extensionality:]  $$(\forall xy.(\exists z.z \in x) \wedge (\forall w.w \in x \leftrightarrow w \in y) \rightarrow x=y)$$  In the statement of this axiom, we follow Zermelo's apparent original intention of allowing atoms.  We will usually assume strong extensionality, however.  An informal way of putting this axiom is ``Non-sets have no elements, and sets with the same elements are equal".

\item[Elementary Sets:] There is an object $\emptyset$ such that $(\forall x.x \not\in  \emptyset)$, called {\em the empty set\/}.  For any
objects $x$ and $y$, there is a unique object $\{x\}$ such that $(\forall z.z\in \{x\} \leftrightarrow z=x)$ and a unique object $\{x,y\}$ such that
$(\forall z.z \in \{x,y\} \leftrightarrow z=x \vee z \in y)$.\footnote{Notice that the assertion that for any $x,y$, $\{x,y\}$ exists, implies that all objects are of the same type in the parlance of section 2.1.1.}

\item[Definition:]  For each formula $\phi$, define $\{x \mid \phi\}$ as $\emptyset$ if $(\forall x.\neg \phi)$, and otherwise as the object $A$
such that $(\forall x:x \in A \leftrightarrow \phi)$, if there is such an object.  Otherwise we say that $\{x \mid \phi\}$ does not exist\footnote{This can be modified using the device of ``proper classes" introduced later.}.  Notice that
$\emptyset=\{z\mid z \neq z\}$, $\{x\} = \{z\mid z=x\}$ and $\{x,y\} = \{z\mid z=x \vee z=y\}$.\footnote{If we include the Hilbert symbol in our logic, and stipulate that the default object is $\emptyset$, then $\{x \mid \phi\}$ can be defined as $(\epsilon x:(\forall y:y \in x \leftrightarrow \phi)$, if one is willing to live with the odd result that symbols for sets that cannot exist, such as $\{x \mid x \not\in x\}$, actually represent $\emptyset$.}

Define ${\tt set}(x)$ as $x=\emptyset \vee (\exists y.y \in x)$.  It would be equivalent but a little more mysterious to define ${\tt set}(x)$ as $x=\{y \mid y\in x\}$.

\item[Axiom (scheme) of Separation:] For any formula $\phi[x]$ and set $A$, the set $\{x \mid x \in A \wedge \phi\}$, which we abbreviate $\{x
\in A \mid \phi[x]\}$, exists.  This can be written more rigorously, ``For each formula $\phi[x]$, $$(\forall A.(\exists S.(\forall x.x \in S \leftrightarrow x \in A \wedge \phi[x]))),$$ where $S$ is a variable not appearing in $\phi[x]$".\footnote{This is technically an ``axiom scheme" rather than a single axiom:  there is a distinct axiom for each formula $\phi[x]$.}

More complex set builder notation may be used.  $\{t[x_1,\ldots,x_n] (\in A) \mid \phi\}$ is defined as $\{u (\in A) \mid (\exists x_1,\ldots,x_n.u=t[x_1,\ldots,x_n])\}$, where $u$ is a fresh variable not appearing in $\phi$ or $t[x_1,\ldots,x_n]$, the latter being shorthand for an arbitrary complex notation containing the $x_i$'s.  The notation $A$ for a bounding set (if present) should not depend on $x$.

\item[Digression: comments on Separation and an alternative:]

In Zermelo's original formulation, he simply said that a subcollection of a set defined by an arbitrary property was a set.  He commented on the usual formulation given above that it was not strong enough to realize his conception:  the ways in which subsets can be defined by Separation are constrained by limitations of our language.  It turns out that Zermelo set theory with the form of Separation given here (and so of course with stronger forms perhaps closer to his conception) is strictly more powerful than the type theory of chapter 2.  On the other hand, a restriction of our language motivated by the form of the axiom of Separation will give a theory precisely equivalent in power to our type theory of chapter 2:

\begin{description}

\item[$^*$Bounded Separation:]  For any set $A$ formula $\phi[x]$ in which each quantifier is bounded, the set $\{x \mid x \in A \wedge \phi\}$, which we abbreviate $\{x
\in A \mid \phi[x]\}$, exists.   The precise meaning of ``each quantifier is bounded" is that each quantifier is of the form $(\forall x \in t.\ldots)$ or $(\forall x \in t.\ldots)$ where $t$ is an expression in which the variable $x$ does not occur.

\end{description}

The motivation here is to apply the same restriction of the range of the main bound variable to a set which appears in the axiom of separation to all occurrences of bound variables in instances of separation. 

\item[Axiom of Power Set:] For any set $A$, the set $\{B \mid B \subseteq A\}$
exists.  The definition of $A \subseteq B$ is the usual one:  $$A \subseteq B \equiv_{\tt def} {\tt set}(A) \wedge {\tt set}(B) \wedge (\forall x.x \in A \rightarrow x \in B).$$ 

\item[Axiom of Union:] For any set $A$, the set $\bigcup A = \{x \mid (\exists
y \in A.x \in y)\}$ exists.

\item[Definition:]  A set $I$ such that $\emptyset \in I$
and $(\forall x.x \in I \rightarrow \{x\} \in I)$ is said to be Zermelo-inductive.

\item[Axiom of Infinity:]  There is a set $\cal Z$ such that $x \in \cal Z$ iff $x$ belongs to every Zermelo-inductive set.   This set is known as the set of {\em Zermelo natural numbers\/}:  Zermelo used $\emptyset$ to represent 0, $\{0\}$ to represent 1, $\{1\}$ to represent 2, and generally $\{n\}$ to represent $n+1$.

\item[Definition:]  We define $A \cap B$ as $\{x \in A \mid x \in B\}$ (and $A-B$ as \newline $\{x \in A\mid x \not\in B\}$).  Sets $A$ and $B$ are said to be disjoint iff  $A \cap B = \emptyset$.  A collection ${\cal A}$ is said to be {\em pairwise disjoint\/} iff $(\forall AB:  A \in {\cal A} \wedge B \in {\cal A} \wedge A \neq B \rightarrow A \cap B = \emptyset)$.  A set $C$ is a {\em choice set\/} for a pairwise disjoint collection $\cal A$ iff $(\forall A \in {\cal A}.(\exists x.A \cap C = \{x\}))$, i.e., each element of $\cal A$ shares exactly one element with $C$.

\item[Axiom of Choice:] Any pairwise disjoint collection of nonempty sets has a
choice set.

\end{description}

We give some discussion of the axioms.  Items in this discussion may presuppose knowledge of our previous discussion of untyped set theory, though general mathematical knowledge may substitute for this.

\begin{enumerate}

\item We will usually assume strong extensionality (objects with the same elements
are equal), as is now usual, but here we preserve Zermelo's original intention of allowing atoms.

\item The axiom of elementary sets is more complicated than is necessary.  The
separate provision of the singleton set is not made in the modern
treatment, as $\{x\} = \{x,x\}$ exists if we merely assert the
existence of unordered pairs, and Separation and Infinity together
imply the existence of the empty set ($\emptyset = \{x \in {\cal Z}\mid x
\neq x\}$) or of at least one empty object if strong extensionality is not assumed.

\item Zermelo did not know that the ordered pair could be defined by
$\left<x,y\right> = \{\{x\},\{x,y\}\}$, but note that the existence of the ordered pair
(now in the Kuratowski form) is provided by the axiom of elementary sets.

\item The axiom of separation does not appear to imply any paradoxes.  We
attempt the Russell argument: define $R_A = \{x \in A \mid x \not\in
x\}$.  Observe that $R_A \in R_A \leftrightarrow R_A \in A \wedge R_A \not\in
R_A$.  This would only lead to contradiction if $R_A \in A$, so we
conclude $R_A \not\in A$, whence we conclude that there is no
universal set (for every set $A$ we have specified a set $R_A$ which
cannot belong to it).

\item The axiom of power set and the axiom of union define familiar
constructions.  Note that $x \cup y$ can be defined as $\bigcup
\{x,y\}$.  $x \cap y = \{z \in x \mid z \in y\}$ and $x-y = \{z \in x
\mid z \not\in y\}$ are provided by Separation alone.  Complements do
not exist for any set.  The cartesian product $A \times B$ is
definable as $\{c \in {\cal P}^{\bf 2}(A \cup B) \mid (\exists ab.a\in A
\wedge b \in B \wedge c=\left<a,b\right>)\}$.



\item In a modern treatment, the von Neumann successor $x^+$ is defined as
$x \cup \{x\}$, and the axiom of infinity asserts that there is a
minimal set which contains the empty set and is closed under the von Neumann
successor operation.  It is interesting to observe that neither form
of the axiom of infinity implies the other in the presence of the
other Zermelo axioms (though they are equivalent in the presence of
the axiom of replacement introduced below).

\item It is remarkable that in spite of the fact that Zermelo did not know
how to code the general theory of relations and functions into set
theory (lacking an ordered pair definition) he was able to prove the
Well-Ordering Theorem from the Axiom of Choice in his 1908 paper.
Some day I have to look at how he did it!

\item The axioms of Foundation and Replacement which complete the modern set
theory {\em ZFC\/} were later developments.

\item We describe a minimal model of Zermelo set theory.  The domain of this
model is the union of the sets ${\cal P}^{\bf i}({\mathbb N})$ (for purposes of this paragraph we take $\mathbb N$ to be the set $\cal Z$ of Zermelo natural numbers).  It is
important to note that the Zermelo axioms give us no warrant for
believing that this sequence of sets makes up a set.  Extensionality
certainly holds in this structure (in its strong version).  The empty set belongs to $\mathbb
N$, so is certainly found in this structure.  It is useful at this
point to note that ${\mathbb N} \subseteq {\cal P}({\mathbb N})$ (each
Zermelo natural number is a set of Zermelo natural numbers, 0 being
the empty set and $n+1$ being $\{n\}$); since $A \subseteq B$
obviously implies ${\cal P}(A) \subseteq {\cal P}(B)$, we have (by
repeated application) ${\cal P}^{\bf i}({\mathbb N}) \subseteq {\cal
P}^{i+1}({\mathbb N})$ and so ${\cal P}^{\bf i}({\mathbb N})\subseteq {\cal
P}^j({\mathbb N})$ if $i \leq j$.  The iterated power sets of the set
of natural numbers whose union is our structure are nested.  For any
$x$ and $y$ in the structure, there are $m$ and $n$ such that $x\in
{\cal P}^{\bf m}({\mathbb N})$ and $y\in {\cal P}^{\bf n}({\mathbb N})$: both $x$
and $y$ belong to ${\cal P}^{\bf m+n}({\mathbb N})$, and so $\{x,y\} \in
{\cal P}^{\bf m+n+1}({\mathbb N})$: the structure satisfies the axiom of
elementary sets.  If $A \in {\cal P}^{\bf i}({\mathbb N})$, then ${\cal
P}(A) \in {\cal P}^{\bf i+1}({\mathbb N})$.  If $A \in {\cal P}^{\bf i}({\mathbb
N})$ (for $i>0$), then $\bigcup A \in {\cal P}^{\bf i-1}({\mathbb N})$:
the restriction to positive $i$ is no real restriction because
${\mathbb N} \subseteq {\cal P}({\mathbb N})$.  Infinity obviously
holds since $\mathbb N$ belongs to the structure.  If it is supposed
that Choice holds in the whole universe it certainly holds in this
structure, as a choice set for a partition in ${\cal P}^{\bf i+1}({\mathbb
N})$ will belong to ${\cal P}^{\bf i}({\mathbb N})$

$\dagger$ Notice the similarity between the role of iterated power sets of the
natural numbers in our description of this structure and types in the
theory of the previous chapter.  The only difference is that the
analogues of types here are cumulative.

\end{enumerate}

\newpage

\subsection{Exercises}

\begin{enumerate}

\item  We define $x^+$ as $x \cup \{x\}$.  We use the modern form of the
Axiom of Infinity:  there is a set which contains $\emptyset$ and is closed
under $x \mapsto x^+$.  We implement 0 as $\emptyset$, and if the natural
number $n$ is implemented as the set $x$, $n+1$ is implemented as $x^+$.

We define $\mathbb N$ as the intersection of all sets which contain 0
and are closed under successor.  Explain how we can show that this set
exists using the axioms of infinity and separation.

Show that the axioms of Peano arithmetic are satisfied in this implementation
of $\mathbb N$.  Proofs of axioms 1,2,3,5 should be very straightforward.

Axiom 4 requires you to show that $x \cup \{x\} = y \cup \{y\}$ implies
$x=y$ for all $x,y \in {\mathbb N}$.  Show this using the axioms of Zermelo
set theory ({\em without\/} Foundation).

Hints: how do you prove {\em anything\/} about natural numbers?  You
can begin as an exercise by proving that for no natural number $n$ is
$n \in n$ true, by induction of course.  This is similar to the fact
about natural numbers you need to prove to establish Axiom 4.  I will
give more explicit hints if you visit me with work in progress.

\item 
Write a proof in Zermelo set theory with the modern form of the Axiom
of Infinity (and without Foundation) that no natural number is an
element of itself.  This will of course be an induction proof using
the definitions $0 = \emptyset; n+1 = n^+ = n\cup \{n\}$.  Intense
attention to ``obvious'' detail is needed at this level.  Hint: it
will be useful (and easy) to prove first (by induction of course) that
all natural numbers are transitive:  a set $A$ is said to be transitive iff all elements of elements of $A$ are also elements of $A$.

Even more of a hint: the induction step looks like this.  Suppose $n
\not\in n$.  Our goal is to show $n+1 = n \cup \{n\}$ is not an element of itself.  Suppose otherwise for the sake of a contradiction.  We suppose that
is that $n+1 \in n+1 = n \cup \{n\}$.  So either $n+1 \in n$ (something
bad happens$\ldots$) or $n+1 = n$ (something bad happens$\ldots$).

\end{enumerate}

\newpage


\newpage

\section{Basic set constructions in Zermelo set theory}

In this section we develop basic mathematical constructions in Zermelo
set theory.

We begin with a very basic

\begin{description}

\item[Theorem:]  $(\forall A.(\exists x.x \not\in A))$

\item[Proof:] This theorem follows from Separation alone.  Consider
$$R_A = \{x \in A \mid x \not\in x\}.$$  Suppose $R_A \in A$.  It
follows that $R_A \in R_A \leftrightarrow R_A \not\in R_A$.

\end{description}

It will be seen to follow from the Axiom of Foundation usually added to Zermelo set theory that $x \not\in x$ for any $x$, it
further follows from Zermelo set theory with the Axiom of Foundation that
$R_A = A$ for all $A$.

It is worth noting that this is important in applying the Axiom of Infinity.  It might seem to be a hazard that there might be no Zermelo-inductive set.
But if there were no Zermelo-inductive set, then the set $\cal Z$ of all sets which belong to every Zermelo-inductive set would contain {\em everything\/}, because each object belongs to all elements of the empty set.  But there is no set which contains every object.  So there is a Zermelo-inductive set, and then one argues in a standard way that
$\cal Z$ itself is the smallest Zermelo-inductive set.  Notice though that it is also possible to prove quite directly that $\cal Z$ is Zermelo-inductive.

It is a fundamental characteristic of Zermelo set theory (and of all
stronger theories) that there are no very big sets (such as the
universe $V$).  Many mistake this for a fundamental characteristic of
set theory.

We want to implement relations and functions.  Here it is very
convenient to work with the Kuratowski pair.

\begin{description}

\item[Definition:]  $\left<x,y\right> = \{\{x\},\{x,y\}\}$

\item[Theorem:]  For any sets $x$ and $y$, $\left<x,y\right>$ is a set.  If $\left<x,y\right>=\left<z,w\right>$ then $x=z$ and $y=w$.

\item[Definition:]  $\pi_1(\left<x,y\right>)$ is defined as the unique $u$ belonging to all elements of $\left<x,y\right>$:  note $\pi_1(\left<x,y\right>)=x$.  $\pi_2(\left<x,y\right>)$ is defined as the unique $u$ belonging to exactly one element of $\left<x,y\right>$:  note $\pi_2(\left<x,y\right>)=y$.  For a different definition of the ordered pair, these projection operators will be defined differently, in order to satisfy the same equations.

\end{description}

This theorem is not enough by itself to ensure that we can use the
Kuratowski pair to get an adequate theory of relations.

\begin{description}

\item[Definition:]  $A \cup B = \bigcup\{A,B\}$

\item[Theorem:] $A \cup B$ exists for any sets $A$ and $B$ (this is
clear from the form of the definition).  $A \cup B = \{x \mid x\in A
\vee x \in B\}$.  Notice that the latter definition, which we used as
the primary definition in type theory, is {\em not\/} guaranteed to
define a set by Separation.

\item[Definition:] $A \cap B = \{x \in A\mid x \in B\}$; $A - B = \{x
\in A \mid x \not\in B\}$.  If $A$ is a nonempty set and $B \in A$, $\bigcap A
= \{x \in B \mid (\forall a \in A.x \in A)\}$.

\item[Definition:]  For any natural number $n>2$, we define $\{x_1,x_2,\ldots,x_n\}$ as $\{x_1\} \cup \{x_2,\ldots,x_n\}$.  This is a recursive definition:  we already have a definition of list notation for sets when $n=2$, and here we show how to define list notation when $n$ has any value $k$ greater than 2 on the assumption that we know how to define it when $n=k-1$.  Similarly, we define $\left<x_1,x_2\ldots,x_n\right>$ as $\left<x_1,\left<x_2,\ldots,x_n\right>\right>$.

\item[Theorem:] For any sets $A$ and $B$, $A \cap B$ and $A - B$
exist.  This is obvious from the forms of the definitions.  If $A$ is
nonempty, $\bigcap A$ exists and the definition of the set does not
depend on the choice of the element $B$.

\item[Definition:] $A \times B = \{x \in {\cal P}^{\bf 2}(A \cup B)\mid (\exists
a \in A.(\exists b \in B.x = \{\{a\},\{a,b\}\}))\}$.  We define $A^2$ as $A \times A$, and more generally define $A^{n+1}$ as $A \times A^n$.

\item[Theorem:] $A \times B$ exists for all sets $A$ and $B$.  $A
\times B = \{\left<a,b\right>\mid a \in A \wedge b \in B\}$.  The
existence of $A \times B$ is obvious from the form of the definition.
The trick is to notice that any pair
$\left<a,b\right>=\{\{a\},\{a,b\}\}$ with $a \in A$ and $b \in B$
actually belongs to ${\cal P}^{\bf 2}(A \cup B)$, because $\{a\}$ and
$\{a,b\}$ both belong to ${\cal P}(A \cup B)$.

\item[Definition:] A {\em relation\/} is a set of ordered pairs.  We
define $x\,R\,y$ as $\left<x,y\right> \in R$.

\item[Observation:] Just as in the type theory of chapter 2, not every logical relation
is a set relation.  For example, the logical relation of equality is
not implemented as a set, because $Q=\{\left<x,y\right> \mid x=y\}$
would have the unfortunate property $\bigcup^{\bf 2} Q = V$, the
universal set, which we know does not exist.  For similar reasons,
membership and inclusion are not set relations.

\item[Definition:] For any relation $R$, we define ${\tt fld}(R)$ as
$\bigcup^{\bf 2} R$, ${\tt dom}(R)$ as $$\{x \in {\tt fld(R)} \mid
(\exists y.\left<x,y\right> \in R)\},$$ and ${\tt rng}(R)$ as $$\{y \in {\tt fld(R)} \mid
(\exists x.\left<x,y\right> \in R)\}.$$

\item[Theorem:] The field, domain, and range of a relation $R$ are
sets.  This is evident from the forms of the definitions.  That they
are the intended sets is evident from the fact that if
$\left<x,y\right> \in R$ then $x,y \in \bigcup^{\bf 2} R$.  Moreover,
${\tt fld}(R) = {\tt dom}(R) \bigcup {\tt rng}(R)$ and $R \subseteq
{\tt dom}(R) \times {\tt rng}(R) \subseteq {\tt fld}(R)^2$.

\end{description}

\begin{description}

\item[$\dagger$ Remark:]  Once we have verified that we have an adequate foundation for the
theory of relations, we can import definitions and concepts wholesale
from type theory, always subject to the limitation that we cannot
construct very large collections.  For example we cannot define
cardinals, ordinals, or general isomorphism types as equivalence
classes under equinumerousness or isomorphism, because
equinumerousness, isomorphism, and most of their equivalence classes
are not sets.  However, we can for example import every definition given in section 2.6,
except that the collections $[=]$ and $[\subseteq]$, being too large, cannot be sets:  however, for any set $A$,
$[=] \cap A^2$ and $[\subseteq]\cap A^2$ are sets (one might not like definitions of these using the symbols $[=]$
or $[\subseteq]$, but the sets referred to can be defined as $\{\left<x,y\right>\in A\times A \mid x=y\}$ and $\{\left<x,y\right>\in A \times A \mid x \subseteq y\}$, respectively).  This is a specific example of a general phenomenon:  if $R$ is a relation symbol, we cannot be sure that
$[R] = \{\left<x,y\right>\mid x\, R \, y\}$ exists, but for any sets $A,B$, we do know that $[R] \cap (A \times B) = \{\left<x,y\right>\in A \times B \mid x \,R\,y\}$ exists.

\end{description}

\subsection{Relations and Functions}

We here reproduce the section on terminology for relations and functions with minor changes from the typed set theory chapter.  Most but not all of this is taken from section 2.6.  The fact that few changes are needed makes an implicit point.

If $A$ and $B$ are sets, we define a {\em relation from $A$ to $B$\/}
as a subset of $A \times B$.  A {\em relation\/} in general is simply
a set of ordered pairs.

If $R$ is a relation from $A$ to $B$, we define $x \,R\,y$ as
$\left<x,y\right>\in R$.  This notation should be viewed with care.
  In the superficially similar notations $x
\in y$ or $x \subseteq y$, the symbols $\in, \subseteq$ do not denote sets at all: do not confuse logical relations with set relations.  

If $R$ is a relation, we define ${\tt dom}(R)$, the {\em domain of
$R$\/}, as $\{x \in {\tt fld}(R) \mid (\exists y.x\,R\,y)\}$.  We define $R^{-1}$, the
{\em inverse of $R$\/}, as $\{\left<x,y\right>\in {\cal P}^2(\bigcup^2 R)\mid y\,R\,x\}$.  We
define ${\tt rng}(R)$, the {\em range of $R$\/}, as ${\tt
dom}(R^{-1})$.  We note that ${\tt fld}(R)$, the {\em field of $R$\/}, already defined as $\bigcup^2 R$, is
the union of ${\tt dom}(R)$ and ${\tt rng}(R)$.  If $R$ is a relation
from $A$ to $B$ and $S$ is a relation from $B$ to $C$, we define
$R|S$, the {\em relative product of $R$ and $S$\/} as
$$\{\left<x,z\right>\mid(\exists y.x\,R\,y \wedge y\,S\,z)\}.\footnote{We leave it as an exercise for the reader to find a bound for the elements of this set, witnessing the fact that the set exists by Separation.}$$

The symbol $[=]\lceil A$ might be used to denote the equality relation restricted to $A$, the set
$\{\left<x,x\right>\in A \times A \mid x \in V\}$.  Similarly $[\subseteq]\lceil {\cal P}(A)$ can be
used as a name for a restriction of the subset relation, and so
forth: the brackets convert a grammatical ``transitive verb'' to a
noun.\footnote{The transformation of relation symbols into terms using brackets is an invention of ours and not likely to be found in other books.}

We define special characteristics of relations.  Some of these terms are also used in connection with logical relations which do not determine sets:  for example, the subset relation is reflexive, antisymmetric and transitive.

\begin{description}

\item [reflexive:] $R$ is {\em reflexive\/} iff $x\,R\,x$ for all $x \in {\tt fld}(R)$.

\item[symmetric:]  $R$ is {\em symmetric\/} iff for all $x$ and $y$, $x\,R\,y \leftrightarrow y\,R\,x$.

\item[antisymmetric:]  $R$ is {\em antisymmetric\/} iff for all $x,y$ if $x\,R\,y$ and $y\,R\,x$ then $x=y$.

\item[asymmetric:] $R$ is {\em asymmetric\/} iff for all $x,y$ if $x\,R\,y$
then $\neg y \,R\,x$.  Note that this immediately implies $\neg x\,R\,x$.

\item[transitive:]  $R$ is {\em transitive\/} iff for all $x,y,z$ if $x\,R\,y$ and $y\,R\,z$ then $x\,R\,z$.

\item[equivalence relation:]  A relation is an {\em equivalence relation\/} iff it is reflexive, symmetric, and transitive.

\item[partial order:] A relation is a {\em partial order\/} iff it is
reflexive, antisymmetric, and transitive.

\item[strict partial order:] A relation is a {\em strict partial order\/} iff
it is asymmetric and transitive.  Given a partial order $R$, $$R-[=]=\{\left<x,y\right> \mid x\,R\,y \wedge x \neq y\}$$
will be a strict partial order.  From a strict partial order $R-[=]$,
the partial order $R$ can be recovered if it has  no ``isolated
points'' (elements of its field related only to themselves).

\item[linear order:] A partial order $R$ is a {\em linear order\/} iff
for any $x,y \in {\tt fld}(R)$, either $x \,R\,y$ or $y\,R\,x$.  Note
that a linear order is precisely determined by the corresponding
strict partial order if its domain has two or more elements.

\item[strict linear order:] A strict partial order $R$ is a {\em strict
linear order\/} iff for any $x,y \in {\tt fld}(R)$, one has $x\,R\,y$,
$y\,R\,x$ or $x=y$.  If $R$ is a linear order, $R-[=]$ is a strict
linear order.

\item[image:]  For any set $A\subseteq {\tt fld}(R)$, $R``A = \{b \mid (\exists a \in A.a\,R\,b)\}$.

\item[extensional:] A relation $R$ is said to be {\em extensional\/}
iff for any $x,y \in {\tt fld}(R)$,
$R^{-1}``(\{x\})=R^{-1}``(\{y\})\rightarrow x=y$: elements of the
field of $R$ with the same preimage under $R$ are equal.  An
extensional relation supports a representation of some of the subsets
of its field by the elements of its field.

\item[well-founded:] A relation $R$ is {\em well-founded\/} iff for each
nonempty subset $A$ of ${\tt fld}(R)$ there is $a\in A$ such that for
no $b\in A$ do we have $b \,R\,a$ (we call this a minimal element of
$A$ with respect to $R$, though note that $R$ is not necessarily an
order relation).

\item[well-ordering:] A linear order $R$ is a {\em well-ordering\/} iff the
corresponding strict partial order $R-[=]$ is well-founded.

\item[strict well-ordering:] A strict linear order $R$ is a {\em strict
well-ordering\/} iff it is well-founded.

\item[end extension:] A relation $S$ {\em end extends\/} a relation $R$
iff $R \subseteq S$ and for any $x \in {\tt fld}(R)$, $R^{-1}``\{x\} =
S^{-1}``\{x\}$.  (This is a nonstandard adaptation of a piece of
terminology from model theory).

\item[function:] $f$ is a {\em function from $A$ to $B$\/} (written
$f:A\rightarrow B$) iff $f$ is a relation from $A$ to $B$ and for all
$x,y,z$, if $x\,f\,y$ and $x\,f\,z$ then $y=z$.  For each $x\in {\tt
dom}(f)$, we define $f(x)$ as the unique $y$ such that $x\,f\,y$ (this
exists because $x$ is in the domain and is unique because $f$ is a
function).  The notation $f[A]$ is common for the image $f``A$.

\item[warning about function notation:] Notations like ${\cal P}(x)$
for the power set of $x$ should not be misconstrued as examples of the
function value notation $f(x)$.  There is no function $\cal P$ because
the domain of such a function would be the collection of all sets, which cannot be a set in untyped set theory.

\item[injection:] A function $f$ is an {\em injection\/} (or {\em one-to-one\/}) iff $f^{-1}$ is a
function.

\item[surjection:] A function $f$ is a {\em surjection from $A$ to
$B$\/} or a {\em function from $A$ onto $B$\/} iff it is a function
from $A$ to $B$ and $f``A=B$.

\item[bijection:] A function $f$ is a {\em bijection from $A$ to $B$\/} iff it
is an injection and also a surjection from $A$ to $B$.

\item[composition and restriction:] If $f$ is a function and $A$ is a
set (usually a subset of ${\tt dom}(f)$), define $f\lceil A$ as $f
\cap (A \times V)$ (the {\em restriction of $f$ to the set $A$\/}).
If $f$ and $g$ are functions and ${\tt rng}(g) \subseteq {\tt
dom}(f)$, define $f \circ g$ as $g|f$.  This is called the {\em
composition\/} of $f$ and $g$.  We may now and then write compositions as relative products, when the unnaturalness of the order of the composition operation is a problem.

\item[identity function:] Note that [=] meets the specification of a function except that it fails to be a set.  We call [=]$\lceil A$ the {\em identity
function on $A$\/}, where $A$ is any set:  as the terminology suggests, each of these sets is a function.

\item[abstraction:] If $T[x]$ is a term (usually involving $x$) define
$(x:A \mapsto T[x])$ or $(\lambda x:A.T[x])$ as
$\{\left<x,T[x]\right>\mid x \in A\}$.  The explicit mention of
the set $A$ may be omitted when it is understood from context or 
from the form of the term $T[x]$.

\item[isomorphism of relations:]  Relations $R$ and $S$ are said to be {\em isomorphic\/} iff there is a bijection $f$ from ${\tt fld}(R)$ to ${\tt fld}(S)$ such that for every $x,y \in {\tt fld}(R)$ we have $x \, R \, y$ iff $f(x) \, S \, f(y)$.  Such a bijection is called an {\em isomorphism from $R$ to $S$\/}.  Isomorphism between relations captures the idea that they have the same formal structure in certain sense.  It is a valuable exercise to show that isomorphism is a (non-set) equivalence relation on set relations.  In particular, if $f$ is an isomorphism from $R$ to $S$, then $f^{-1}$ is an isomorphism from $S$ to $R$.
\item[terminology about partial orders:]

It is conventional when working with a particular partial order $\leq$
to use $<$ to denote $[\leq]-[=]$ (the corresponding strict partial
order), $\geq$ to denote $[\leq]^{-1}$ (which is also a partial order)
and $>$ to denote the strict partial order $[\geq]-[=]$.

A minimum of $\leq$ is an element $m$ of ${\tt fld}(\leq)$ such that
$m \leq x$ for all $x \in {\tt fld}(x)$.  A maximum of $\leq$ is a
minimum of $\geq$.  A minimal element with respect to $\leq$ is an
element $m$ such that for no $x$ is $x < m$.  A maximal element with
respect to $\leq$ is a minimal element with respect to $\geq$.  Notice
that a maximum or minimum is always unique if it exists.  A minimum is
always a minimal element.  The converse is true for linear orders but
not for partial orders in general.

For any partial order $\leq$ and $x \in {\tt fld}(\leq)$, we define
${\tt seg}_{\leq}(x)$ as $\{y \mid y < x\}$ (notice the use of the
strict partial order) and $(\leq)_x$ as $[\leq] \cap ({\tt
seg}_{\leq}(x))^2$.  The first set is called the {\em segment\/} in $\leq$
determined by $x$ and the second is called the {\em segment restriction\/}
determined by $x$.

For any subset $A$ of ${\tt fld}(\leq)$, we say that an element $x$ of
${\tt fld}(\leq)$ is a lower bound for $A$ in $\leq$ iff $x \leq a$
for all $a \in A$, and an upper bound for $A$ in $\leq$ iff $a \leq x$
for all $a \in A$.  If there is a lower bound $x$ of $A$ such that for
every lower bound $y$ of $A$, $y \leq x$, we call this the greatest
lower bound of $A$, written $\inf_{\leq}(A)$, and if there is an upper
bound $x$ of $A$ such that for all upper bounds $y$ or $A$, we have $x
\leq y$, we call this the least upper bound of $A$, written
$\sup_{\leq}(A)$.

A special kind of partial order is a {\em tree}: a partial order
$\leq_T$ with field $T$ is a {\em tree\/} iff for each $x \in T$ the
restriction of $\leq_T$ to ${\tt seg}_{\leq_T}(x)$ is a well-ordering.
A subset of $T$ which is maximal in the inclusion order among those
well-ordered by $\leq_T$ is called a {\em branch\/}.

\end{description}

\newpage

\subsection{Exercises}

\begin{enumerate}

\item  Prove the lemma [used in class] that $$(\forall xyzw.\{x,y\} = \{z,w\} \rightarrow ((x=z \wedge y =w)\vee (x=w \wedge y=z)).$$ 

\item  Consider the original  ordered pair definition of Wiener, $\left<x,y\right> \equiv_{\tt def} \{\{\{x\},\emptyset\},\{\{y\}\}\}$.  Prove that this satisfies the basic properties needed for a notion of ordered pair to implement relations in set theory:

\begin{enumerate}

\item $$(\forall xyzw.\left<x,y\right>=\left<z,w\right> \rightarrow x=z \wedge y=w)$$

\item  For any sets $A,B$, $A \times B = \{\left<x,y\right>\mid x \in A \wedge y \in B\}$ exists.

\item For any set $R$ of ordered pairs, ${\tt dom}(R) = \{x \mid (\exists y.x\,R\,y)\}$ and  ${\tt rng}(R) = \{y \mid (\exists x.x\,R\,y)\}$ exist.

\end{enumerate}

Hint:  think about how many members various sets involved in this definition have.  By contrast, you cannot tell how many members $\{\{x\},\{x,y\}\}$ has, in general.  Do you see why not?


\item  Consider the ordered pair definition $\left<x,y\right> \equiv_{\tt def} \{\{x,0\},\{y,1\}\}$ [0 being defined as $\emptyset$ and 1 as $\{\emptyset\}$]. Prove that this satisfies the basic properties needed for a notion of ordered pair to implement relations in set theory:

\begin{enumerate}

\item $$(\forall xyzw.\left<x,y\right>=\left<z,w\right> \rightarrow x=z \wedge y=w)$$

\item  For any sets $A,B$, $A \times B = \{\left<x,y\right>\mid x \in A \wedge y \in B\}$ exists.

\item For any set $R$ of ordered pairs, ${\tt dom}(R) = \{x \mid (\exists y.x\,R\,y)\}$ and  ${\tt rng}(R) = \{y \mid (\exists x.x\,R\,y)\}$ exist.  

\end{enumerate}



\item  Show that if we use our official definition of the ordered pair $$\left<x,y\right> \equiv_{\tt def} \{\{x\},\{x,y\}\}$$ that the theorem $$(\forall xyzw.\left<x,y\right>=\left<z,w\right> \rightarrow x=z \wedge y=w)$$ is true.

\item The natural number 1 was defined by Frege as the collection of all sets with exactly one element.  Express ``$x$ has exactly one element" as a formula $\phi[x]$ using only propositional logic, quantifiers, equality and membership, and give a definition of the Frege natural number 1 in the form $\{x \mid \phi[x]\}$.  Then prove that this set does not exist in Zermelo set theory.  The first part of the question is readily answered by looking in chapter 2:  it would be a better idea not to do this.

\item Show that if $R$ and $S$ are set relations (sets of ordered pairs), their relative product $R|S = \{\left<x,z\right> \mid (\exists y.x \,R\,y \wedge y \,S\,z)\}$ exists, by finding a suitable set $U$ such that $R|S = \{\left<x,z\right> \in U \mid (\exists y,x \,R\,y \wedge y \,S\,z)\}$ (from which it follows that the set $R|S$ exists by Separation).

\item  Prove directly that the set $\cal Z$ whose existence is asserted by the axiom of infinity is Zermelo-inductive.  That is, prove that $\emptyset \in \cal Z$, then, assuming that $x \in \cal Z$, show that $\{x\} \in \cal Z$ must follow.

\item Is $\{\{x,0\},\{y,1\},\{z,2\}\}$ (suppose that 2 is defined as $\{\{\emptyset\}\}$) an adequate definition of the ordered triple?  Given an arbitrary set of this form, can we determine its first, second, and third component?

\end{enumerate}

\newpage

\section{Case study:  the implementation of the number systems in untyped set theory}

In this section we will implement familiar systems of numbers in set theory.   Part of the aim is to shed light on what it means to found mathematics on set theory.  A general theme is that though we are identifying mathematical objects which we understand before studying set theory with certain sets, we are not really claiming to reveal anything about these objects, and moreover the identifications depend on decisions that could have been made differently:  in some cases we will describe more than one alternative implementation of a concept, and we try to make it clear that choosing a different implementation would not change the underlying mathematics.

\subsection{The natural numbers}

We begin with the arithmetic of the natural numbers.  Peano proposed a set of five axioms describing the arithmetic of the natural numbers in the nineteenth century\footnote{We note that Peano's original axiom set used 1 as a primitive instead of 0.}.  It is worth noting that while these axioms do not impose an implementation of numbers themselves as sets, they do make essential use of sets.  Later in the section we will give an alternative (and weaker) formulation not dependent on set theory at all.

\begin{description}

\item[Primitive notions:]  A constant 0, a unary operation $\sigma$ (successor), and the set $\mathbb N$ of natural numbers.

\item[Axiom 1:]  $0 \in \mathbb N$.

\item[Axiom 2:]  For each $x \in {\mathbb N}$, $\sigma(x) \in {\mathbb N}$.

\item[Axiom 3:]  For each $x \in \mathbb N$, $\sigma(x) \neq 0$.

\item[Axiom 4:]  For each $x,y \in \mathbb N$, $\sigma(x)=\sigma(y) \rightarrow x=y$

\item[Axiom 5:]  For each $S \subseteq \mathbb N$, if $0 \in S$ and $(\forall x \in S:\sigma(x) \in S)$, then $S=\mathbb N$.

\end{description}

We will give in this section and the following section not one but three implementations of Peano arithmetic.  We will choose one of them as the official representation for our purposes, but we could equally well have chosen one of the others, and our mathematics would be essentially the same.

We review the definition of ``inductive set" and the Axiom of Infinity.

\begin{description}

\item[Definition:]  A {\em Zermelo-inductive set\/} is defined as a set $I$ such that $\emptyset \in I$  and $(\forall x \in I:\{x\} \in I)$.

\item[Axiom of Infinity:]  We assert the existence of the set $\cal Z$  of all $n$ such that for every Zermelo-inductive set $I$, $n$ belongs to $I$.

\item[First implementation of Peano arithmetic:]  We implement 0 as $\emptyset$, $\sigma$ as $\{\left<x,\{x\}\right> \in {\cal Z} \times {\cal P}({\cal Z})\mid x \in {\cal Z}\}$, and $\mathbb N$ as $\cal Z$.  Notice the bounding of the definition of $\sigma$ to verify that this set actually exists.  To confirm that this is an implementation, we need to verify that the translations of each of the axioms hold:

\begin{description}

\item[Axiom 1:]  $\emptyset \in \cal Z$ holds because $\emptyset$ belongs to each Zermelo-inductive set, by the definition of ``Zermelo-inductive", and to belong to each Zermelo-inductive set is to belong to $\cal Z$.

\item[Axiom 2:]  We verify that for all $x \in \cal Z$, $\{x\} \in \cal Z$.  Choose an $x \in \cal Z$ arbitrarily.  Choose a Zermelo-inductive set $I$ arbitrarily.  Because $x \in \cal Z$, we have $x \in I$, by the definition of ``Zermelo-inductive".  Because $I$ is Zermelo-inductive, we have $\{x\} \in I$.  $I$ was chosen arbitrarily, so we have that $\{x\}$ belongs to every Zermelo-inductive set, and so that $\{x\}$ belongs to $\cal Z$.  The element $x \in \cal Z$ was chosen arbitrarily, so we have verified our claim.

\item[Observation:]  The demonstrations of the interpreted Axioms 1 and 2 are together a direct proof that $\cal Z$ is itself Zermelo-inductive.

\item[Axiom 3:]  For each $x \in \cal Z$, $\{x\} = \emptyset$ is obviously false, since the first set has an element and the second does not.

\item[Axiom 4:]  We verify that for each $x,y \in \cal Z$, $\{x\}=\{y\} \rightarrow x=y$.  Suppose $\{x\}=\{y\}$.  Because $\{x\}$ is defined as $\{z\mid z=x\}$ (and exists by the Axiom of Elementary Sets) we have $x \in \{x\}$ (since $x=x$).  Thus by substitution we have $x \in \{y\} = \{z \mid z=y\}$, so we have $x=y$.  This is all quite obvious, but it is worth noting that such obvious things really can be proved.

\item[Axiom 5:]  We need to verify that if $S \subseteq \cal Z$ and $\emptyset \in S$ and $(\forall x\in S:\{x\} \in S)$, it follows that $S=\cal Z$.  This is very direct:  the conditions imply immediately that $S$ is Zermelo-inductive, whence it follows that $\cal Z \subseteq S$  ($\cal Z$ is a subset of any Zermelo-inductive set, since an element of $\cal Z$ belongs to all Zermelo-inductive sets and so to that specific one), whence it follows that $S = \cal Z$ by Extensionality (if $A \subseteq B$ and $B \subseteq A$, then $A$ and $B$ are sets with the same elements and so are equal).

\end{description}

\end{description}

Because of the possibility of this interpretation, we may refer to elements of $\cal Z$ as ``Zermelo natural numbers".  This is not our official interpretation, but it is not a bad one, and we will indicate in this section and the next one how we would proceed if we chose to use it as our official implementation.

Our official interpretation relies on a different choice of implementation of the successor operation, and actually on a different form of the Axiom of Infinity, which turns out not to be provable in Zermelo's original theory.

We reformulate the definition of ``inductive set" and the Axiom of Infinity.

\begin{description}

\item[Definition:]  For any set $x$, we define $x^+$ as $x \cup \{x\}$.

\item[Definition:]  A {\em von Neumann-inductive set\/} is defined as a set $I$ such that $\emptyset \in I$  and $(\forall x \in I:x^+ \in I)$.

\item[Axiom of Infinity$^*$:]  We assert the existence of the set $N$  of all $n$ such that for every von Neumann-inductive set $I$, $n$ belongs to $I$.

\item[Second (and official)  implementation of Peano arithmetic:]  We implement 0 as $\emptyset$, $\sigma$ as $\{\left<x,x \cup \{x\}\right> \in {N} \times {\cal P}(N))\mid x \in {N}\}$, and $\mathbb N$ as $N$.  Notice the bounding of the definition of $\sigma$ to verify that this set actually exists.  To confirm that this is an implementation, we need to verify that the translations of each of the axioms hold:

\begin{description}

\item[Axiom 1:]  $\emptyset \in N$ holds because $\emptyset$ belongs to each von Neumann-inductive set, by the definition of ``von Neumann-inductive", and to belong to each von Neumann-inductive set is to belong to $N$.

\item[Axiom 2:]  We verify that for all $x \in N$, $x^+ \in N$.  Choose an $x \in N$ arbitrarily.  Choose a von Neumann-inductive set $I$ arbitrarily.  Because $x \in N$, we have $x \in I$, by the definition of ``von Neumann-inductive".  Because $I$ is von Neumann-inductive, we have $x^+ \in I$.  $I$ was chosen arbitrarily, so we have that $x^+$ belongs to every von Neumann-inductive set, and so that $x^+$ belongs to $N$.  The element $x \in N$ was chosen arbitrarily, so we have verified our claim.

\item[Observation:]  The demonstrations of the interpreted Axioms 1 and 2 are together a direct proof that $N$ is itself von Neumann-inductive.

\item[Axiom 3:]  For each $x \in N$, $x^+ = \emptyset$ is obviously false, since the first set has an element and the second does not.

\item[Axiom 4:]  We want to verify that for each $x,y \in \cal Z$, $x^+=y^+ \rightarrow x=y$.  We will ask the reader to prove this, with some guidance, in an exercise.

\item[Axiom 5:]  We need to verify that if $S \subseteq N$ and $\emptyset \in S$ and $(\forall x\in S:x^+ \in S)$, it follows that $S=N$.  This is very direct:  the conditions imply immediately that $S$ is von Neumann-inductive, whence it follows that $N \subseteq S$  ($N$ is a subset of any von Neumann-inductive set, since an element of $N$ belongs to all von Neumann-inductive sets and so to that specific one), whence it follows that $S = N$ by Extensionality (if $A \subseteq B$ and $B \subseteq A$, then $A$ and $B$ are sets with the same elements and so are equal).

\end{description}

\end{description}

We make an important observation at this point.  As long as our implementation has the characteristic that $\mathbb N$ is defined as the intersection of all sets $I$ which
contain 0 and satisfy $(\forall x \in I:\sigma(x) \in I)$, and we can verify that this set exists, the verification of Axioms 1, 2, and 5 goes exactly as above.  Only the verifications of Axioms 3 and 4 will depend on the details of what object is chosen to implement 0 and what operation is chosen to implement $\sigma$.  The reader can confirm this by reading the parallel demonstrations of Axioms 1, 2, and 5 given in the two implementations given so far, which do not depend in any way on any specific information about 0 or $\sigma$.

A major practical advantage of the von Neumann representation is that the von Neumann implementation of each natural number $n$ is $\{0,\ldots,n-1\}$, a set with $n$ elements, which facilitates reasoning about counting (discussed in the next subsection).  A further and more profound advantage is that this representation generalizes naturally to a representation of transfinite ordinals, which is not the case for the Zermelo representation.

We claim to have a representation of the natural numbers at this point, but the reader may notice that we have not defined even such basic concepts as addition and multiplication.  We proceed to repair this lack.

\begin{description}

\item[Iteration Theorem:]  For each set $A$ and function $f:A \rightarrow A$, and element $a \in A$, there is a unique function $g:{\mathbb N} \rightarrow A$ such that $g(0)=a$ and for each $n \in \mathbb N$, $g(\sigma(n))=f(g(n))$.  Once the theorem is proved, we introduce the notation ${\tt iter}_{f,a}$ for the unique function $g$ and the notation $f^n(a)$ for $g(n)$ [this last may also serve to make our motivation clear].

\item[Proof of the Iteration Theorem:]

Fix a set $A$, a function $f:A \rightarrow A$, and an element $a \in A$.

\begin{description}

\item[Definition:]   We define an $(f,a)$-inductive set as a set $I\subseteq {\mathbb N} \times A$ such that $\left<0,a)\right> \in I$ and for all $\left<n,x\right> \in I$, we also have $\left<\sigma(n),f(a)\right>\in I$.   Further define $g$ as the set of all elements of ${\mathbb N} \times A$ which belong to all $(f,a)$-inductive sets.

\end{description}

We first prove that $g$ is a function, that is, for each $n \in \mathbb N$, there is exactly one $x \in A$ such that $\left<n,x\right> \in g$.  We prove this by induction on $n$.

\begin{description}

\item[basis:]  We claim that there is exactly one $x$ such that $\left<0,x\right>\in g$.  We claim in fact that $x=0$.  We know that $\left<0,a\right>\in g$, because $\left<0,a\right>$ belongs to every $(f,a)$-inductive set, and that is the criterion for belonging to $g$.  Now suppose that $y \neq 0$; our aim is to show that $\left<0,y\right>\neq g$.  Let $I$ be an $(f,a)$-inductive set:  we claim that $J=I - \{\left<0,y\right>\}$ is also $(f,a)$-inductive.  Certainly $\left<0,a\right>\in J$, since $\left<0,a\right>\in I$ and
$\left<0,a\right>\neq \left<0,y\right>.$   Suppose $\left<n,z\right>\in J$.  It follows that $\left<\sigma(n),f(z)\right>\in I$, because $I$ is $(f,a)$-inductive and $J \subseteq I$; but also $\left<\sigma(n),f(z)\right>\neq \left<0,y\right>$ by Axiom 3, so $\left<\sigma(n),f(z)\right>\in J$, so $J$ is $(f,a)$-inductive, so $\left<0,y\right>\not\in g$, since $\left<0,y\right>\not\in J$, an $(f,a)$-inductive set.  This completes the proof of the basis.

\item[induction step:]  We assume for a fixed $k \in \mathbb N$ that there is exactly one $x$ such that $\left<k,x\right>\in g$, and show that there is exactly one $y$ such that $\left<\sigma(k),y\right>\in g$.   There is at least one such $y$, namely $f(x)$, because $\left<\sigma(k),f(x)\right>\in g$, since each $(f,a)$-inductive set contains $\left<k,x\right>\in g$, and so contains $\left<\sigma(k),f(x)\right>$.  Now consider any $z \neq f(x)$:  our aim is to show $\left<\sigma(k),z\right> \not\in g$.  Let $I$
be any $(f,a)$-inductive set:  we show that $J=I - \{\left<\sigma(k),z\right>\}$ is also $(f,a)$-inductive.  $\left<0,a\right>\in I$ of course, and $\left<0,a\right>\neq \left<\sigma(k),z\right>$ by Axiom 3, so $\left<0,a\right>\in J$.  Now suppose that $\left<n,w\right>\in J$:  certainly $\left<\sigma(n),f(w)\right> \in I$, but further 
$\left<\sigma(n),f(w)\right>\neq \left<\sigma(k),z\right>$, because if this equation held we would have $n=k$ by Axiom 4, and we know that if $n=k$ we have $w=x$, so $z=f(x)$, which contradicts our choice of $z$.  And thus $\left<\sigma(n),f(w)\right>\in J$, so $J$ is $(f,a)$-inductive, whence $\left<\sigma(k),z\right> \not\in J$ cannot belong to $g$, which completes the proof of the induction step.

\end{description}

Since $g$ is a function, we can now see that $g(0)=a$ and $g(\sigma(n))=f(g(n))$.   We further claim that for any function $g'$ such that $g'(0)=a$ and $g'(\sigma(n))=f(g'(n))$,
we have $g'(n)=g(n)$ for each $n \in \mathbb N$, whence $g=g'$, establishing uniqueness.  This is an easy induction.  $g(0)=a=g'(0)$ is obvious.  If $g(k)=g'(k)$,
then $g(\sigma(k))=f(g(k))=f(g'(k))=g'(\sigma(k))$.

This completes the proof of the Iteration Theorem.

\end{description}

We state some identities for the iteration notation $f^n(a)$.  Notice that, where $g$ is the unique function provided by the Iteration Theorem, $f^0(a)=g(0)=a$, so we obtain the identity $f^0(a)=a$.   We further obtain $f^{\sigma(n)}(a) = g(\sigma(n))=f(g(n))=f(f^n(a))$, so we have the identity $f^{\sigma(n)}(a)=f(f^n(a))$.  It is also worth noting that we can define the function $f^n$ as $$\{\left<x,y\right>\in A \times A\mid y=f^n(x)\},$$ and this will define a function, even though the notation $f^n(a)$ was not originally defined as a function application notation;  it is safe to read it that way, anyway.  In English, $f^n(x)$ is defined as ${\tt iter}_{f,x}(n)$, recalling that ${\tt iter}_{f,x}$ is the unique function $g$ from $\mathbb N$ to $A$  provided by the Iteration Theorem for which $g_x(0)=x; g_x(\sigma(n))=f(g_x(n))$, for all $n \in \mathbb N$.  Something rather subtle is going on here:  $f^n$ is a function whose value at each $x\in A$ is determined by applying a function depending on $x$ to $n$.

We can now define some familiar operations.

\begin{description}

\item[definition of addition of natural numbers:]  For natural numbers $m,n$, we define $m+n$ as $\sigma^n(m)$.  Note that if we define 1 as $\sigma(0)$, we can represent $\sigma(n)$ in the more familiar form $n+1$.

\item[definition of multiplication of natural numbers:]  For each natural number $n$, define ${\tt add}_n$ as $\{\left<m,m+n\right>\in {\mathbb N}^2 \mid m \in {\mathbb N}\}$, the function which adds $n$.  Define $m \cdot n$ as $({\tt add}_m)^n(0)$.

\item[arithmetic rules from the iteration theorem:]

We derive rules for addition and multiplication which are given as additional axioms for Peano arithmetic when it is formulated independently of set theory.

\begin{enumerate}

\item $n+0 = \sigma^0(n) = n$

\item $m+\sigma(n) = \sigma^{\sigma(n)}(m) = \sigma(\sigma^n(m)) = \sigma(m+n)$

\item $n \cdot 0 = ({\tt add}_n)^0(0) = 0$

\item $$m \cdot \sigma(n) = ({\tt add}_m)^{\sigma(n)}(0) = {\tt add}_m(({\tt add}_m)^n(0)) = {\tt add}_m(m \cdot n) = m\cdot n + m$$

\end{enumerate}

\end{description}

The formulation of Peano arithmetic independently of set theory adds addition and multiplication as new primitive notions (with closure properties added as part of axiom 2), takes the equations proved just above as Axioms 6-9, and modifies Axiom 5 to assert for any
formula $\phi(x)$ for which $\phi(0)$ is true and $(\forall k \in \mathbb N:\phi(k) \rightarrow \phi(\sigma(k)))$ is true, we obtain $(\forall n \in \mathbb N:\phi(n))$.  We will not make use of this more restricted formulation, since we have no reason to refrain from using set concepts.  For us, ``Axioms" 6-9 are consequences of Axioms 1-5 with Axiom 5 in its original form involving sets rather than open sentences.

More general forms of recursion can be implemented, and indeed our Iteration Theorem is a somewhat unusual formulation.

\begin{description}

\item[Recursion Theorem:]  Let $A$ be any set, let $a \in A$, and let $f:{\mathbb N} \times A \rightarrow A$.  Then there is a unique function $g$
such that $g(0)=a$ and for all $n \in \mathbb N$, we have $g(n+1)=f(n,g(n))$.

\item[Proof:]  Define $F(\left<n,x\right>) = \left<n+1,f(n,g(n))\right>$.  Then the function $g$ is definable using the Iteration Theorem as
$g(n)=\pi_2(F^n(\left<0,a\right>)$.

\end{description}

We give an example to illustrate yet more complex forms of recursion.

\begin{description}


\item[Recursion example (Fibonacci numbers):]  Define $f({\mathbb N}) \times {\mathbb N} \rightarrow {\mathbb N}$ by $f(\left<m,n\right>) = \left<n,m+n\right>$.  Then the $n$th Fibonacci number $F(n)$ can be defined as $\pi_1(F^n(\left<1,1\right>))$.

We introduce an even stronger form of recursion.

\item[Theorem (course of values recursion):]  Let $\cal F$ be the set of all functions whose domain is a set $\{m \in {\mathbb N} \mid m<n\}$ for $n \in {\mathbb N}$ (of course, if we use the von Neumann definition of the natural numbers, this domain is $n$ itself) and whose range is $A$.  Let $F:{\cal F} \rightarrow A$.  Then there is a uniquely determined function $G:{\mathbb N} \rightarrow A$ such that $$G(n) = F(G \lceil \{m \in {\mathbb N}\mid m<n\}).$$ for each natural number $n$.

\item[Proof of course-of-values recursion theorem:]  Define $H:{\cal F} \rightarrow {\cal F}$ as follows:  if $f$ has domain $\{m \in {\mathbb N}\mid m<n\}$, define $H(f)$
as $f \cup \{\left<n,F(f)\right>\}$.  Now define $G$ as $\bigcup\{H^n(\emptyset) \in {\cal P}({\cal F}^2) \mid n \in \mathbb N\}$.

\end{description}

The reader still may be suspicious of our claim that this is an adequate axiomatization of arithmetic.  We refer him or her to section 2.8 in the previous chapter, in which various basic results of arithmetic are proved from the Peano axioms (including 6-9), the formal differences being the convention that all quantifiers are taken to be restricted to the natural numbers, since natural numbers are the only objects we are talking about (so there is no reason to write $(\forall k \in {\mathbb N}:\phi)$:  this is abbreviated to $(\forall k :\phi)$).

We have another, quite abstract point, to make about implementations.  We are planning to use the von Neumann implementation, under which $0=\emptyset, \sigma(x)=x^+$,
and $\mathbb N$ is the set $N$ of von Neumann natural numbers.  We want to argue that if we are careful, we can translate any theorem proved in set theory (not just arithmetic) which mentions natural numbers (using the von Neumann interpretation) into a theorem proved in set theory which mentions natural numbers using any other implementation we might choose.  The idea is that we can eliminate all consideration of which sets the natural numbers are, and rely on no fact about the natural numbers other than which one is zero and which natural numbers are successors of which other natural numbers.  The key point to observe is that if $n$ is a natural number, $x \in n$ is equivalent to $x < n$, which we define
(as we did in section 2.8 above) as $(\exists k\in {\mathbb N}.x+k = n) \wedge x \neq n$.   Set theory allows us to deduce that $x \in 0$ is false and that $x \in \sigma(n) = n \cup \{n\}$ iff $x \in n \vee x = n$.  We proved in section 2.8 that $x<0$ is false and that $x \leq n \leftrightarrow x < n+1$, which is equivalent to $x<\sigma(n) \leftrightarrow x<n \vee x=n$.  Set theory allows us to deduce that $(\forall k \in {\mathbb N}:k \in m \leftrightarrow k \in n) \rightarrow m=n$, by applying Extensionality.  The assertion $(\forall k \in {\mathbb N}:k<m \leftrightarrow k<n) \rightarrow m=n$ follows from the fact that $\leq$ is a linear order, proved in section 2.8:  suppose that $m \neq n$;  then either $m \leq n$ or $n \leq m$;  suppose $m \leq n$ without loss of generality;  then $m<n$, but of course $\neg m<m$, so $(\forall k \in {\mathbb N}:k<m \leftrightarrow k<n)$ does not hold.  To make the statement (and proof) of a theorem involving natural numbers implementation-independent, make sure that all references to what objects are elements of a natural number $n$ are replaced with references to which objects are natural numbers less than $n$ (a concept which can be defined independently of the choice of implementation).

\subsubsection{Exercises}

\begin{enumerate}

\item Prove Axiom 4 for the von Neumann implementation:  that is, prove that for any von Neumann  natural numbers $x,y$, $x^+ = y^+ \rightarrow x=y$.

First prove this using the special assumption that for all sets $x,y$, it is not the case both that $x \in y$ and $y \in x$.  You should find that this makes it possible to prove
that $x^+ = y^+ \rightarrow x=y$ for all sets $x,y$, and so of course for all natural numbers $x,y$.   [This special assumption is a consequence of the Axiom of Foundation, which we will introduce and add to our official set theory later, but which we do not have yet.]

This should give a hint about how to prove the result for all natural numbers without the special assumption:  demonstrate by induction that for all natural numbers $x,y$, it is not the case that both $x \in y$ and $y \in x$ hold.  Prove this theorem, then show how to use it to prove the von Neumann version of Axiom 4 without making any special assumptions.

A further hint:  a set $x$ is said to be {\em transitive\/}  iff for all $y \in x$, for all $z \in y$, we also have $z \in x$ (a transitive set contains all elements of its own elements).  It is useful to prove (by induction of course) that all von Neumann natural numbers are transitive.

The moral here is that if we have Foundation (which we will have), Axiom 4 for our implementation is almost as easy as it is for the Zermelo implementation.  But it is useful to see that the validity of the von Neumann interpretation does not depend on assuming Foundation as an axiom.

\item  Prove by mathematical induction that for all natural numbers $n$, $\sigma^n(0)=n$.  (the notation $\sigma^n(0)$ is a special case of the notation $f^n(a)$ defined in the Iteration Theorem).  You might want to look at the identities for the notation $f^n(a)$ which I supply in a new paragraph after the proof of the Iteration Theorem.  This isn't precisely hard, but you have to pay attention to what you write.

\item Prove the associative law of addition as a theorem of our set theory.  Prove it quite straightforwardly using more or less the same strategy I used to prove commutativity of addition in section 2.8 (or in class):  be aware that you have to prove it by induction, and set up the appropriate basis step and induction step and prove them using axioms 6 and 7, proving further statements by induction if necessary.


\end{enumerate}

\newpage

\subsection{The natural numbers and counting elements of sets}

In this section, we will discuss the original use of the natural numbers for counting finite sets.

\begin{description}

\item[Definition:]  Fix a set $X$.  We define a sequence of subsets of ${\cal P}(X)$.  Define $[X]^0$ as $\{\emptyset\}$.  Define $[X]^{n+1}$ as $\{u \cup \{x\} \in {\cal P}(X)\mid u \in [X]^n \wedge x \in X \setminus u\}$.  

There is a function $K$ such that $K(n) = [X]^n$ by the Iteration Theorem, with $A = {\cal P}^2(X), a=\{\emptyset\}$, and $f$ defined by $$f(U) =  \{u \cup \{x\} \mid u \in U \wedge x \in X - u\}$$ for $U \in {\cal P}^2(X)$.

The natural reading of $[X]^n$ is ``the collection of subsets of $X$ with $n$ elements".

\item[Definition:]  The collection $[X]^{<\omega}$ of finite subsets of $X$ is defined as the intersection of all sets $I \subseteq {\cal P}(X)$ with the property that $\emptyset \in I$
and for every $u \in I, x \in X$, $u \cup \{x\}\in I$.  A set $I$ with this property is said to be ``$X$-finite-inductive".  The natural reading of $[X]^{<\omega}$ is ``the collection of finite subsets of $X$", and we say that a set is a finite subset of $X$
iff it belongs to  $[X]^{<\omega}$.  We say that a set $X$ is finite iff $X \in [X]^{<\omega}$, and infinite iff it is not finite.

\item[Observation:]  It is provable that for any $A \subseteq X$, $A \in [X]^{<\omega}$ iff $$(\exists n \in {\mathbb N}\mid A \in [X]^n\}.$$ Prove that $\bigcup \{[X]^n\mid n \in \mathbb N\}$ satisfies the right closure property to show that $[X]^{<\omega} \subseteq \bigcup \{[X]^n\mid n \in \mathbb N\}$, then prove by mathematical induction on $n$ that
each $[X]^n \subseteq [X]^{<\omega}$.

\end{description}

We introduce an important concept of set theory whose possibilities we will not be exhausting in this section.

\begin{description}

\item[Definition:]  Sets $A$ and $B$ are said to be {\em equinumerous\/} or less formally, to be of the same cardinality, or, even less formally, to be of the same size, iff there is a bijection $f:A \rightarrow B$ from $A$ onto $B$.

\item[Theorem:]  The relation $\sim$ is reflexive, symmetric, and transitive:  for any set $A$, the restriction of $\sim$ to ${\cal P}(A)^2$ is an equivalence relation.  (The qualification of the claim that $\sim$ is an equivalence relation is that it is not a set).

\item[Proof of Theorem:]  The identity relation is a bijection from $A$ to $A$, so $A \sim A$.  If $A \sim B$, then there is $f:A \sim B$, a bijection from $A$ to $B$, and $f^{-1}:B \rightarrow A$ is a bijection from $B$ to $A$, so $B \sim A$ as well.  If $A \sim B$ and $B \sim C$, then there is $f:A \rightarrow B$, a bijection from $A$ onto $B$, and $g:B \rightarrow C$, a bijection from $B$ onto $C$, and $g \circ f$ is a bijection from $A$ onto $C$, so $A \sim C$ as well.

\item[Finite Counting Theorem:]  If $A,B \in [X]^{<\omega}$, then $$A \sim B \leftrightarrow (\exists n \in {\mathbb N}:A \in [X]^n \wedge B \in [X]^n),$$ and further $[X]^m \cap [X]^n = \emptyset$ when $m \neq n$.

\item[Proof of Theorem:]  We have already shown that for any $A \in [X]^{<\omega}$ there is $n$ such that $A \in [X]^n$.

We prove by induction on $n$ that if $A \in [X]^n$ and $B \in {\cal P}(X)$, then $A \sim B \leftrightarrow B \in [X]^n$.

For all $A \in [X]^0$ and for all subsets $B$ of $X$, we immediately have $A \sim B$ iff $B$ is empty, that is, $B$ is also in $[X]^0$.

Now suppose that for any $A \in [X]^k$, we have, for all $B$ subsets of $X$, that $A \sim B$ iff $B \in [X]^k$.    Suppose that $C \in [X]^{k+1}$ and $D \in {\cal P}(X)$.  Our aim is to show that $C \sim D \leftrightarrow D \in [X]^{k+1}$.  $C = E \cup \{x\}$ for some $E \in [X]^k$ and $x \not\in E$.  If $C \sim D$ is witnessed by a bijection $f$,
then $D = f``C = f``E \cup \{f(x)\}$.  By ind hyp $f``E$ belongs to $[X]^k$ and obviously $f(x) \not\in f``E$, so $D \in [X]^{k+1}$.  Now suppose that $D \in [X]^{k+1}$,
so $D = F \cup \{y\}$ where $F \in [X]^k$ and $y \not\in F$.  By ind hyp, $E \sim F$, and if $g:E \rightarrow F$ is a bijection onto $F$ witnessing this, then $g \cup \{\left<x,y\right>\}$ is a bijection witnessing $C \sim D$.

This completes the proof by induction of the first claim.

Now we show by induction on $n$ that for all $A \in [X]^n$ and $m \neq n$ a natural number, we have $A \not\in [X]^m$.

This is evident for $n=0$, as if $A \in [X]^n$ we have $A$ empty, and for any $m \neq 0$, all elements of $[X]^m$ are nonempty.

Fix $k \in \mathbb N$, and assume that for all $A \in [X]^k$ and $m \neq k$, $A \not\in [X]^m$.  Suppose that $A \in [X]^{k+1}$ and also $A \in [X]^p$ with $p \neq k+1$ (our aim is a contradiction).
We know that $p \neq 0$, so $p=m+1$ for some $m$.  We have $A = B \cup \{x\}$ for some $B \in [X]^k$ and $x \not\in B$, and $A = C \cup \{y\}$ for some $C \in [X]^m$
and $y \not\in C$.  Now if $x=y$ we have $B=C$ which contradicts the induction hypothesis.  If $x \neq y$, we observe that ${\tt id}_A \setminus \{\left< y,y\right>\} \cup \{\left<y,x\right>\}$ is a bijection from $B$ to $C$, so $B \sim C$, which contradicts the inductive hypothesis.  In both cases, as soon as we know $B \sim C$, we know that $C \in [X]^k$ by the first claim, and then by inductive hypothesis we know that $[X]^k$ is the only set of this form to which $C$ belongs, so $k=m$, which is a contradiction.

This completes the proof of the second claim and the entire theorem.

\end{description}

We present a further formulation of the Axiom of Infinity and implementation of the natural numbers in Zermelo set theory, which we will not use, but which we offer for comparison with the treatment of the natural numbers in chapter 2.

\begin{description}

\item[Axiom of Infinity$^{**}$:]  There is a set $\cal I$ such that $\cal I \not\in [{\cal I}]^{<\omega}$:  i.e., $\cal I$ is not finite.

The status of this version of the Axiom of Infinity is that it is weaker than (that is, a consequence of) either of the other forms we have given (neither of which is deducible from the other).  Neither of the other axioms can be proved from this one.

\item[An alternative implementation of the natural numbers:]

We give an alternative implementation of the natural numbers, based directly on counting sets, like the implementation in chapter 2.

\begin{description}

\item[zero:]  Define 0 as $\{\emptyset\}$.

\item[successor:]  Define $\sigma(n)$ (for $n \in {\cal P}^2({\cal I})$) as $$\{u \cup \{x\} \in {\cal P}({\cal I}) \mid u \in n \wedge x \not\in u\}.$$

\item[$\cal I$-inductive:]  We say that a set $I$ is $\cal I$-inductive iff $0 \in I$ and $(\forall n:n \in I \rightarrow \sigma(n) \in I)$.

\item[definition of $\cal N$:]  We define $\cal N$ as the collection of all elements of ${\cal P}^2({\cal I})$ which belong to all $\cal I$-inductive sets.

\item[axioms 1,2, and 5:]  Verification of these axioms goes exactly as in the other two implementations we have discussed.

\item[axiom 3:]  We need to show that for any $n \in {\cal N}$, we have $\sigma(n) \neq 0$.  This is straightforward:  each element of $\sigma(n)$ is of the form
$u \cup \{x\}$ and so must have an element, and 0 has $\emptyset$ as an element, which does not have any elements.

\item[axiom 4:]  We need to show that for any $m,n \in {\cal N}$, if $\sigma(m)=\sigma(n)$ then $m=n$.  This requires an extended argument.

We notice first that for each $n \in \mathbb N$, for each $u \in n$, we have $u \in [{\cal I}]^{<\omega}$.  The reason for this is that if we take any $\cal I$-finite-inductive set
(a set $I \subseteq {\cal P}({\cal I})$ such that $\emptyset \in I$ and $$(\forall ux:u \in I \wedge x \in {\cal I} \rightarrow u \cup \{x\} \in I))$$ we can prove by an obvious direct induction that each natural number $n$ is a subset of $I$.

It then follows that no natural number is empty.  0 is nonempty.  Suppose that $n$ is nonempty and $u \in n$.  We have shown that $u \in [{\cal I}]^{<\omega}$, and we know that 
${\cal I} \not\in [{\cal I}]^{<\omega}$ and that $u$ is a subset of $\cal I$, so $u$ is a proper subset of $\cal I$, there is $x \in {\cal I} -u$, and $u \cup \{x\} \in \sigma(n)$, so $\sigma(n)$ is nonempty, and all elements of $\mathbb N$ are nonempty sets by induction.

We prove a lemma.  For each $n$, if $u \in \sigma(n)$ and $x \in u$, then $u - \{x\} \in n$ (taking an element away from an $n+1$-element set gives an $n$-element set).  If $u \in [{\cal I}]^{1}$ and $x \in u$, then $u$ is $\emptyset \cup \{y\}$ for some $y$, so $u = \{y\}$, whence $y=x$, so $u \setminus \{x\} = \emptyset \in [{\cal I}]^0$.  Now fix $k \in \mathbb N$ and suppose that for every $u \in [{\cal I}]^{k+1}$ and $x \in u$ we have $u \setminus \{x\} \in [{\cal I}]^k$.  Suppose that $u \in [{\cal I}]^{k+2}$ and $x \not\in u$.  Our aim is to show that $u-\{x\}\in [{\cal I}]^{k+1}$.  Because $u \in [{\cal I}]^{k+2}$, $u$ is of the form $v \cup \{y,z\}$ where $v \in [{\cal I}]^k$, $y \not\in v$, $v \cup \{y\} \in [{\cal I}]^{k+1}$, and $z \not\in v \cup \{x\}$.  If $x=z$, we have $u \setminus \{z\} = v \cup \{y\} \in [{\cal I}]^{k+1}$ as desired.  Otherwise we have $x \in v \cup \{y\}$, and by inductive hypothesis we have $(v \cup \{y\}) \setminus \{x\} \in [{\cal I}]^{k}$, and since $z \not\in v \cup \{y\} \setminus \{x\}$, we have $((v \cup \{y\}) \setminus \{x\})\cup \{z\} = u \setminus \{x\} \in [{\cal I}]^{k+1}$ as required.  This completes the proof by induction of the lemma.

Now suppose that $\sigma(m)=\sigma(n)$.   Both $m$ and $n$ are nonempty.  An element of $\sigma(m)$ must be of the form $u \cup \{x\}$ where $u \in [{\cal I}]^m$ and $x \not\in u$, and also of the form $v \cup \{y\}$ where $v \in [{\cal I}]^n$ and $y \not\in v$.  Now if $x=y$ we have $u=v$ and $m=n$ immediately.   If $x \neq y$, observe that since $u \cup \{x\}$ is in $[{\cal I}]^{m+1}$ and $y \in u$, we must have $v=(u \cup \{x\})\setminus \{y\} \in [{\cal I}]^m$, whence $v$ is in both $[{\cal I}]^m$ and $[{\cal I}]^m$, whence $m=n$.  This completes the verification of Axiom 4.

\end{description}

\end{description}

If one assumes either the Zermelo or von Neumann form of the Axiom of Infinity, it is easier to prove that the representation just given works.
One first proves that $\cal Z$ or $N$ is an infinite set (in the case of $N$, this is easy, as $n \in [N]^n$ is straightforward to prove).  One then observes that the new representation of $n$ is $[{\cal I}]^n$ (where the superscript represents the previous implementation of $n$).  Axiom 3 remains easy.
Axiom 4 holds because if $[{\cal I}]^{m+1}=[{\cal I}]^{n+1}$, then $m+1=n+1$ by the Finite Counting Theorem, so $m=n$ by Axiom 4 for the original implementation of the natural numbers, whence of course $[{\cal I}]^m=[{\cal I}]^n$.

We are interested in this representation because of its evident relationship to the implementation given in chapter 2.  It is also interesting to present a representation of the natural numbers in Zermelo set theory which does not actually rely on explicitly postulating a zero set and a successor operation, but just assumes in the abstract that there is an infinite set.  It is worth noticing that the existence of {\em any\/} implementation of the natural numbers implies that the representation of $n$ as $[{\cal I}]^n$ works;  to show this we need to demonstrate that $\mathbb N$ is an infinite set no matter how the natural numbers are implemented, a proof of which is suggested as an exercise.  So this is demonstrably the weakest form of the Axiom of Infinity for Zermelo set theory.\footnote{A lacuna in this last representation is that we are told nothing about the identity of the postulated infinite set $\cal I$.  We put an evil suggestion on the table (which contradicts the axiom of foundation):  what if ${\cal I} = \mathbb N$ (this last being understood in terms of the third implementation)?  This can be made to work but is admittedly quite weird.}

Now addition and multiplication can be defined in natural ways that we learned in elementary school (and which generalize to addition and multiplication operations on infinite cardinals).

\begin{description}

\item[Definition:]  The cardinality $|A|$ of a finite set $A$ is defined as the natural number $n$ such that $A \in [A]^n$.  That this definition makes sense for each finite set is established by results proved above.


 Since we are using the von Neumann representation, we can further observe that $|A|$ is the unique natural number such that $A \sim n$, though this requires some additional care.
A straightforward induction shows that for each $n \in \mathbb N$, we have $n \in [n]^n$:  each von Neumann natural number $n$ is an $n$-element subset of itself. 

 We need the further lemma that $X \subseteq Y \rightarrow [X]^n \subseteq [Y]^n$.  A set $A$ belongs to $[X]^0$ iff it belongs to $[Y^0]$.  Suppose for a fixed $k$ that
for any set $B$, $B$ belongs to $[Y]^k$ if it belongs to $[X]^k$.   Suppose that $A \in [X]^{k+1}$.  There must be $x \in X$ such that $A = B \cup \{x\}$, $x \not\in B$, and $B \in [X]^k$
and so $B \in [Y]^k$ by ind hyp.  But then $x \in Y$ and $x \not\in B$ so $A = B \cup \{x\} \in [Y]^{k+1}$.

  Then we can argue that if $A \in [A]^n$, we can further argue using the lemma that both $A$ and $n$ belong to $[A \cup n]^n$, whence we have $A \sim n$.  Conversely, if $A \sim n$ we want to argue that $A \in [A]^n$:   this seems best proved by induction;  if $A \sim 0$, then $A$ is the empty set and $A \in [A]^0$;  if for a fixed $k$, for any set $B$, $B \sim k$ implies $B \in [B]^k$, consider an arbitrary $A$ such that $A \sim k+1$.  If $f$ is a bijection from $A$ onto $k+1$,
$f^{-1}``k$ is equinumerous to $k$ and so $f^{-1}``k \in [f^{-1}``k]^k$ by ind hyp, whence $f^{-1}``k \in [A]^k$ by an earlier lemma, whence $A= f^{-1}``k \cup \{f^{-1}(k)\} \in [A]^{k+1}$ as desired.

\item[Definition:]  Suppose $|A|=m$ and $|B|=n$.  Define $m\oplus n$ as $$|(A \times \{0\}) \cup (B \times \{1\})|,$$ and define $m \otimes n$ as $|A \times B|$.  We further state our eventual intention to define $|A|+|B|$ as $$|(A \times \{0\}) \cup (B \times \{1\})|$$ and $|A|\cdot |B|$ as $|A \times B|$ for all sets $A,B$, when we have a more general notion of cardinality (under which $|A|=|B| \leftrightarrow A \sim B$ will hold for all sets $A, B$).

These meanings are assigned to $\oplus$ and $\otimes$ only in this section;  the intention is to show that they are the same as + and $\cdot$.

\end{description}

Some theorems need to be proved to verify that the last definition makes sense, and then of course we want to prove that $m \oplus n = m+n$ and $m \otimes n = m \cdot n$.

The definition given for addition can be given a more abstract form:

\begin{description}

\item[Abstract definition of addition:]  Suppose that $|A|=m$ and $|B|=n$, and that $A \cap B = \emptyset$.  Define $m \oplus n$ as $|A \cup B|$.  (This could also be stated for general sets when we have a general definition of cardinality). 

\end{description}

To verify that this makes sense, we need to show that if $|A| = |A'| = m$ and $|B| = |B'|=n$, and $A \cap B = A'\cap B' = \emptyset$, it follows that $|A \cup B| = |A' \cup B'|$, if the latter cardinals exist.
Theorems already proved establish that there are bijections $f$ from $A$ onto $A'$ and $g$ from $B$ onto $B'$:  it is then straightforward to see that $f \cup g$ is a bijection from $A \cup B$ to $A' \cup B'$, whence if either of the cardinals $|A \cup B|$ and $|A' \cup B'|$ are defined as natural numbers, the other must also be defined and must be the same.  So the definition succeeds, in the sense that the value determined for $m \oplus n$ (if any) will not depend on the choice of the sets $A$ and $B$.  Notice that the concrete definition above is thus equivalent:  for any $A$, $B$, we have $A \sim A \times \{0\}$, $B \sim B \times \{1\}$, and $A \times \{0\} \cap B \times \{1\}$ empty.  That $m \oplus n$ is always defined follows from the next result:

\begin{description}

\item[Theorem:]  For any natural numbers $m,n$, $m \oplus n = m + n$.

\item[Proof:]  We prove this by induction on $n$.  If $n=0$, we have $|A|\oplus 0 = |(A \times \{0\} \cup \emptyset\times \{1\}| =  |A \times \{0\}| = |A|$ for any finite set $A$, and of course
$|A|+0$ is also $|A|$.

Suppose that the result is true for $n=k$.  Let $|B|=k+1$, so $B = C \cup \{x\}$ with $|C|=k$ and $x \not\in C$.  For any finite set $A$ we have
$|A| \oplus (k+1) = |(A \times \{0\}) \cup (B \times \{1\}| = |(A \times \{0\} \cup C \times \{1\}) \cup \{(x,1)\}| = |A \times \{0\} \cup C \times \{1\}|+1 = (|A| \oplus |C|)+1=(|A|\oplus k)+1) = (|A| +k)+1) = |A|+(k+1)$ as desired.

\item[Theorem:]  For any natural numbers $m,n$, $m \otimes n = m \cdot n$.

\item[Proof:]  This is left as an exercise.

\end{description}

While we are in the spirit of elementary school, we do frame an

\begin{description}

\item[Abstract definition of multiplication:]  For any $m,n$ natural numbers, define $m\otimes n$ as $|\bigcup X|$, where $X$ is a pairwise disjoint collection with $|X|=n$ and
$(\forall A \in X:|A|=m)$.  (This could also be stated for general sets if we had a general definition of cardinality, though only if the Axiom of Choice is assumed, as we will discuss later).  The alert reader may recall that we ran into serious trouble trying to frame this definition in the type theory of chapter 2, but here it seems more innocent.  However, where sets of sets are being invoked, presuming innocence may not be the best strategy.  We recommend using the concrete definition above to prove the previous theorem.

\end{description}

We note that we get much nicer proofs (basically elementary school proofs) of a number of well known properties of arithmetic.  Each of the parts of the following theorem can be proved by unpacking arithmetic operations using the $\oplus$ and $\otimes$ definitions, then explicitly constructing bijections which witness the stated equations.

\begin{description}

\item[Theorem:]  The following ``axiomatic" assertions of arithmetic hold.  We feel free to use the usual symbols for addition and multiplication.

\begin{description}

\item[commutative laws:]  $|A| + |B|=|B|+|A|$;  $|A|\cdot |B|$

\item[associative laws:]  $(|A|+|B|)+|C| = |A|+(|B|+|C|)$;   $(|A|\cdot|B|)\cdot|C| = |A|\cdot(|B|\cdot|C|)$

\item[distributive law:]  $|A| \cdot (|B|+|C|) = |A|\cdot |B| + |A| \cdot |C|$

\item[identity properties and zero law:]  $|A|+0=|A|$;  $|A| \cdot 0 = 0$; $|A|\cdot 1=|A|$

\end{description}

\item[sample proof:]  We prove the distributive law.  $|A| \cdot (|B|+|C|)$ is the cardinality of the collection of pairs $(a,d)$ where $d \in B \times\{0\} \cup C \times \{1\}$,
that is the collection of pairs of one of the forms $\left<a,\left<b,0\right>\right>$ or   $\left<a,\left<c,1\right>\right>$ with $a \in A, b \in B , c \in C$.  On th eother hand, $|A|\cdot |B| + |A| \cdot |C|$ consists
of elements of one of the forms $\left<\left<a,b\right>,0\right>$ or $\left<\left<a,c\right>,1\right>$.   The bijection witnessing the claimed equation of cardinals is then
$$\{\left<\left<a,\left<d,i\right>\right>,\left<\left<a,d\right>,i\right>\right> \mid a \in A \wedge ((d \in B \wedge i=0) \vee (d \in C \wedge i=1))\}.$$

\end{description}

It is important to note that our definitions of addition and multiplication of not necessarily finite cardinals will be the same, and the proof of the previous theorem will carry over to possibly infinite cardinals.  Not all sensible results of arithmetic do carry over, however.  Notably, the cancellation property of addition does not hold.  This can be seen informally by considering
that $|1 + {\mathbb N}| = |0 + {\mathbb N}|$ will be expected to be true, as a bijection from $\{(1,0)\} \cup {\mathbb N} \times \{1\}$ to $(\emptyset \times \{0\}) \cup ({\mathbb N} \times \{1\}) = {\mathbb N} \times \{1\}$ is readily constructed (map $\left<1,0\right>$ to $\left<0,1\right>$ and map each $\left<n,1\right>$ to $\left<n+1,1\right>$), but of course $1 \neq 0$.  However, the following restricted form of cancellation does hold for general cardinals:

\begin{description}

\item[Theorem:]  For any sets $A,B$, $|A|+1 = |B|+1$ implies $|A|=|B|$.

\item[Proof:]  We are given a bijection $f$ from a set $A \cup \{x\}$ (with $x \not\in A$) to a set $B \cup \{y\}$ (with $y \not\in B$).  If $f(x)=y$, the restriction of $f$ to
$A$ witnesses $|A|=|B|$.  If $f(x) \neq y$, the map $f \setminus \{\left<x,f(x)\right>\} \setminus \{\left<f^{-1}(y),y\right>\} \cup \{f^{-1}(y),f(x)\}$ is a bijection witnessing $|A|=|B|$.  Note that this depends in no way on $A$ and $B$ being finite sets.

\item[Theorem:]  For any sets $A,B$ and $n \in \mathbb N$, $|A|+n = |B|+n \rightarrow |A|=|B|$.

\item[Proof:]  Apply the previous result repeatedly.


\end{description}

\newpage

\subsection{The positive rationals}

We now begin the construction of the real number system.  It is usual to begin with the system $\mathbb N$ of natural numbers (which we have in hand), then construct the system
$\mathbb Z$ oof integers, then construct the system $\mathbb Q$ of rationals, then construct the reals from the rationals.

We will take a somewhat different approach, perhaps reminiscent of ancient Greek prejudices against zero and negative numbers.  We will begin by restricting our attention to
${\mathbb N}^+ = {\mathbb N} \setminus \{0\}$, the system of positive integers.  We will then construct the system of {\em fractions\/} representing the positive rational numbers
${\mathbb Q}^+$.  We will then construct the system of {\em magnitudes\/} representing the positive real numbers ${\mathbb R}^+$, and only at the final step will we introduce zero and the negative reals to obtain $\mathbb R$.  We will see advantages in the simplicity of the definitions of arithmetic operations.  

Further, we will not make any use of equivalence classes as is usual in these constructions, because choosing representative elements is easy at each point where an equivalence class construction would seem to be recommended.

\begin{description}

\item[Definition:]  For each $m,n \in {\mathbb N}^+$, we define the fraction $\frac mn$ as $$\left<m \,{\tt div}\, {\tt gcd}(m,n),n \,{\tt div}\, {\tt gcd}(m,n)\right>.$$  The development of the theory of greatest common denominators (and of the operator {\tt div} of integer division) in the positive natural numbers should present no difficulties for our reader.  We define ${\mathbb Q}^+$ as the set of all fractions.  We will as usual in this sort of construction regard $\left<n,1\right>$ as the fraction implementing the positive natural number $n$, though it is not the same object.

\item[Definition:]  We define $\frac mn + \frac pq$ as $\frac{mq+np}{nq}$.  We define $\frac mn \cdot \frac pq$ as $\frac{mp}{nq}$.  We define $\frac mn \leq \frac pq$ as
$mq \leq np$.  There are things to verify here:   one needs to check that choice of $m,n,p,q$ in the representations of fractions does not affect these definitions.

\end{description}

The reader should find it easy to believe that we have implemented the familiar system of positive rational numbers in our set theory at this point.  To check this in detail is an exercise in algebra, not set theory.

\subsection{Magnitudes (positive reals)}

This is the center of the construction, where we employ Dedekind's device of ``cuts" for representing reals as sets of rationals.  We only use the left set of the Dedekind cut, and we derive substantial formal advantages from only choosing to construct the positive reals in this way.

\begin{description}

\item[Definition:]  We say that a set $r \subseteq {\mathbb Q}^+$ is a {\em magnitude\/} if and only if (1) $r$ is {\em nontrivial} ($r$ and ${\mathbb Q}^+ -r$ are both nonempty), (2) $r$ is {\em downward closed} (for each $p \in r$ and $q \leq p$ we also have $q \in r$), and (3) $r$ is {\em open\/}:  $r$ has no largest element.   We define ${\mathbb R}^+$ as the set of all magnitudes.  For each rational $p$, $\{q \in {\mathbb Q}^+ \mid q \leq p \wedge q \neq p\}$ is taken to be the magnitude implementing the positive rational
$p$, though it is not the same object.  In general, speaking as if we had informal prior knowledge of the real numbers, the trick is that we implement a positive real $r$ as $\{q \in {\mathbb Q}^+ \mid q < r\}$.

\item[Definition:]  For magnitudes $r,s$, we define $r+s=\{p+q \mid p \in r \wedge q \in s\}$,   $r\cdot s=\{p\cdot q \mid p \in r \wedge q \in s\}$, and $r \leq s$ as $r \subseteq s$.

It is valuable to note that for any nonempty set $A$ of magnitudes which is bounded above, $\sup A = \bigcup A$, and for any nonempty set of magnitudes $A$ which is bounded below, $\inf A = \bigcap A$.  Actually, the statement about set intersections of sets of reals and greatest lower bounds  is not quite true:  this is the subject of an exercise.

\end{description}

The reader should find it easy enough to believe that we have implemented the positive reals at this point:  verifying this would represent quite a lot of work in the general area of analysis.

\subsection{The real number system}

The final step of implementation of the real number system amounts to allow free rein to the operation of subtraction, and it should be reminiscent of the construction of the fractions.

\begin{description}

\item[Definition:]  For each pair of magnitudes $m,n$, the formal difference $m-n$ is defined as $\left<(1+m)\ominus {\tt min}(m,n),(1+n)\ominus {\tt min}(m,n)\right>$, where the partial operation $\ominus$ of subtraction of magnitudes is defined in the natural way.  The set $\mathbb R$ of all real numbers is defined as the set of all formal differences of magnitudes.  We implement 0 (this is not of course the familiar natural number 0)
as $\left<1,1\right>$, $+m$ as $\left<1+m,1\right>$ and $-m$ as $\left<1,1+m\right>$ (in which 1 is the magnitude 1, not the familiar natural number).

\item[Definition:]  We define $(m-n)+(p-q)$ as $(m+p) -(n+q)$.  We define $(m-n)\cdot (p-q)$ as $(mp+nq) - (mq+np)$.  We define $(m-n) \leq (p-q)$ as $m+q \leq n+p$.

\end{description}

That the real number system has been implemented successfully at this point is an algebraic exercise.

The notations ${\mathbb N}^+, {\mathbb Q}^+$ and ${\mathbb R}^+$ might sometimes be used to describe subsets of the real number system with which their elements are ``identified".  The more usual systems $\mathbb Z$, $\mathbb Q$ must be understood as subsets of $\mathbb R$ in our development, as they are not way stations on the road to constructing the reals in our particular approach.

\newpage

\section{\small Preliminaries for transfinite arithmetic of cardinals and ordinals}

We will now depart from implementation of familiar bits of mathematics and strike out into the uncharted territory of the infinite.

We intend to generalize the notation of cardinal $|A|$ which we have defined for all finite sets $A$ to all sets.  We state our

\begin{description}

\item[Formal Intention:]  With each set $A$ we intend to associate a set $|A|$, the cardinality of $A$, satisfying the condition that for
all sets $A,B$ we have $A \sim B \leftrightarrow |A|=|B|$.

We intend to define $|A|+|B|$ as $|(A \times \{0\}) \cup (B \times \{1\})|$ and $|A|\cdot |B|$ as $|A \times B|$ for all sets $A,B$, as signalled above.

It happens that this intention cannot be realized in Zermelo set theory without additional assumptions.  It is however possible to define
$[X]^{|A|}$ as $\{B \in {\cal P}(X) \mid A \sim B\}$, which gives a representation of cardinals for subsets of any fixed set $X$.

We do immediately define the cardinality $|\mathbb N|$ as the set $\mathbb N$ itself, though when this set is considered as a cardinal we will write it $\aleph_0$ (which is read ``aleph-null").  Sets of cardinality $\aleph_0$ are called {\em countably infinite sets\/}.  Sets which are neither finite nor countably infinite will be called {\em uncountable\/} or {\em uncountably infinite\/} sets.

The details of implementations of cardinality will be given below.

\end{description}

We begin by discussing some facts about the cardinal $\aleph_0$.

\begin{description}


\item[Theorem:]  $\aleph_0 + \aleph_0 = \aleph_0$;  $\aleph_0\cdot \aleph_0 = \aleph_0$.

\item[Proof:]  To prove each of these statements, we need to exhibit an appropriate bijection.

A bijection from $\mathbb N$ to ${\mathbb N}\times \{0\} \cup {\mathbb N} \times \{1\}$ is defined by $f(2n) = \left<n,0\right>; f(2n+1) = \left<n,1\right>$.  This justifies $\aleph_0 + \aleph_0 = \aleph_0$.

A bijection from $\mathbb N$ to ${\mathbb N} \times {\mathbb N}$ is defined thus:  $g(2^m(2n+1)-1) = \left<m,n\right>$.  Each positive integer factors uniquely into a product of a power of two and an odd number from which ``coordinates" can be extracted as indicated;  subtracting one covers all the natural numbers.  This justifies $\aleph_0\cdot \aleph_0 = \aleph_0$.

\end{description}

This much was known in the Middle Ages or even in ancient times.  However, the former understanding was that infinite sets behaved in paradoxical ways, but all infinite sets were infinite in the same sense.  It was a modern discovery that there are actually different sizes of infinite sets.

\begin{description}

\item[Definition:]  We define comparison relations between cardinals.  We say that $A \preceq B$ holds iff there is an injection from $A$ to $B$, and that $A \preceq^* B$ holds iff $A$ is empty or there is a surjection from $B$ onto $A$;  these are two different ways of saying that the set $A$ is no larger than the set $B$.  We then intend to assert $|A| \leq |B|$ iff $A \preceq B$, and $|A| \leq^* |B|$ iff $A \preceq^* B$.  We define $A \prec B$ as holding if $A \preceq B \wedge A \not\sim B$.

Important observations need to be made whose details will be filled in later.  The starred forms are equivalent to the unstarred forms in the presence of the Axiom of Choice.  The unstarred forms are preferable:  there is a nice theorem (not requiring Choice) to the effect that $|A| \leq |B| \wedge |B| \leq |A| \rightarrow |A|=|B|$ (the Schr\"oder-Bernstein theorem).  In the absence of Choice, the starred version has no such nice property.  The relation $\leq$ on cardinals is clearly reflexive and transitive;  the theorem establishes that it is at least a partial order.  We will see later that the assertion that the the order on cardinals is a total order is equivalent to the Axiom of Choice.

We define $|A|<|B|$ as $|A| \leq |B| \wedge |A|\neq |B|$.

\item[Schr\"oder-Bernstein Theorem:]  If $A \preceq B$ and $B \preceq A$, then $A \sim B$.  Thus $|A| \leq |B|$ and $|B| \leq |A|$ together imply $|A|=|B|$.

\item[Proof:]  Let $f:A \rightarrow B$ and $g:B \rightarrow A$ be injective.  The map $f$ sends $A$ to a subset $f``A$ of $B$, the same size as 
$A$.  The idea is to adjust $f$ so that it maps $A$ exactly to $B$.  Let $C$ be the set $\{(f\circ g)^n(c) \mid c \in B \setminus f``A \wedge n \in {\mathbb N}\}$.  Define $h$ as $(f \circ g)^{-1}$ on $C$ and as the identity on $B \setminus C$.  Then $h \circ f$ is a bijection from $A$ onto $B$.

\end{description}

Now we will demonstrate the following result which was a historic surprise:

\begin{description}

\item[Theorem:]  For every set $A$, $|A| < |{\cal P}(A)|$.

\item[Proof:]  Clearly $|A| \leq |{\cal P}|(A)|$:  the map sending $a \in A$ to $\{a\} \in {\cal P}(A)$ witnesses this.

Suppose that $|A|=|{\cal P}(A)|$.  This would give us a bijection $f:A \rightarrow {\cal P}(A)$.  Now define the set $R = \{a \in A \mid a \not\in f(a)\}$.  Let $r = f^{-1}(R)$.  Now observe that $r \in R$ iff $r \not\in f(r)=R$, a contradiction.  Notice the close relationship to Russell's paradox here.

So we have established $|A|<|{\cal P}(A)|$.

\end{description}

A particular consequence of this is $|\mathbb N|<|\mathbb R|$.  The key is that it is straightforward to establish $\mathbb R \preceq {\cal P}(\mathbb N)$ and ${\cal P}(\mathbb N) \preceq \mathbb R$ [we leave construction of the required injections as an exercise:  think about representations of the reals in base 2 (though base 3 might be more convenient to avoid unwanted identifications)], so these two sets have the same cardinality, and the theorem above shows $|\mathbb N|<|{\cal P}(\mathbb N)|$.

This is as far as we'll go with the theory of cardinals for now.  As it happens the usual official representation of cardinals depends on the prior development of a representation for ordinals, and further relies on the Axiom of Choice.  So we will start discussing well-orderings and ordinal numbers.

\begin{description}

\item[Definition:]  A {\em well-ordering\/} is a binary relation $\leq$ which is reflexive, antisymmetric, transitive and total (a linear order),
and has the further property that  for each nonempty subset $A$ of ${\tt fld}(\leq)$ there is an element $a$ such that $(\forall b\in A:a \leq b)$,
a $\leq$-minimal element of $A$.  Clearly this element is unique.

\item[Familiar well-orderings:]  Of the orders encountered in undergraduate mathematics before this point, remarkably few are well-orderings.
A linear order on a finite set is a well-ordering.  It is amusing to observe that a relation $\leq$ and its inverse $\leq^{-1}$ are both well-orderings iff $\leq$ is a linear order on a finite set.  

The usual order on the natural numbers is a well-ordering:  suppose that $A \subseteq \mathbb N$ has no minimal element;  clearly $0 \not\in A$, as if otherwise 0 would be minimal in $A$  (and for all $m \leq 0$, $m \not\in A$, though it may seem odd for us to say this);  now suppose for a fixed $k$ not only that $k \not\in A$, but that
for all $m \leq k$, $m \not\in A$:  it follows that $k+1 \not\in A$ because otherwise it would be minimal in $A$, and further $m \not\in A$ for any
$m \leq k+1$.  We have proved by induction that a subset of $\mathbb N$ with no minimal element in the usual order is empty, and so this order is a well-ordering.

A final sort of order familiar to undergraduates which is a well-ordering is the usual order on a convergent increasing sequence of real numbers together with its limit.

The reader should check to their own satisfaction that the usual orders on integers, rationals and reals are not well-orderings.

\item[Isomorphism:]  If $R$ and $S$ are binary relations, we say that $R$ and $S$ are {\em isomorphic\/} (written $R \approx S$) iff there
is a bijection $f$ from ${\tt fld}(R)$ onto ${\tt fld}(S)$ with the property that for all $x,y \in {\tt fld}(R)$, $x \, R \, y \leftrightarrow f(x) \,S\,f(y)$.  The notion of isomorphism captures the idea that the relations $R$ and $S$ have the same formal structure.

\end{description}

We now state the formal intention which will lead to our eventual definition of ordinal numbers.

\begin{description}

\item[Formal Intention:]  We intend to associate with each well-ordering $\leq$ a set ${\tt ot}(\leq)$, called the {\em order type\/} of $\leq$, with the rule that for any well-orderings $\leq_1$ and $\leq_2$ we have ${\tt ot}(\leq_1) = {\tt ot}(\leq_2)$ iff $[\leq_1] \approx [\leq_2]$.  Note the similarity to our formal intentions with regard to cardinality.  A set which is an order type will also be called an {\em ordinal number\/}.

\item[Reminder of segment notations:]  When $\leq$ represents a well-ordering, we will let $x < y$ represent $x \leq y \wedge x \neq y$.  When $\leq$ is a well-ordering and $x \in {\tt fld}(\leq)$, we define ${\tt seg}_{\leq}(x)$ as
$\{y \in {\tt fld}(\leq)\mid y < x\}$.  We define $(\leq)_x$ as $[\leq] \cap {\tt seg}_{\leq}(x)^2$.  Note that $(\leq)_x$ is also a well-ordering.

\item[Foreshadowing of the ordinal definition:]  We will indicate the usual definition of order type (due to von Neumann).  The difficulty is that it cannot be proved in Zermelo set theory that all well-orderings have order types in this sense:  $${\tt ot}(\leq) = \{{\tt ot}((\leq)_x)\mid x \in {\tt fld}(\leq)\}$$.

The reader should be able to convince themselves that for each natural number $n$, the order type of a linear order on a set of size $n$ under this definition is precisely the von Neumann natural number $n$ (the set of all smaller natural numbers).  Further, the order type of the usual order on the natural numbers is $\mathbb N$ itself:  we write $\omega$ for $\mathbb N$ when we consider it the first infinite ordinal number.  The order type
of the convergent series with limit is $\{0,1,2,\ldots,\omega\}$, which we will call $\omega+1$.

The embarrassment in Zermelo set theory is that the ordinal $\omega \cdot 2$ which is the order type of the order on natural numbers defined by
``$m \leq_{\tt bogus} n$ iff $m$ is even and $n$ is odd, or $m$ and $n$ have the same parity and $m \leq n$" cannot be shown to exist.  It is easy enough to describe this set:  we first define $\alpha+1$ for any ordinal as $\alpha \cup \{\alpha\}$:  this is the same as the successor definition for von Neumann naturals, and it is clear that if $\alpha$ is the order type of an particular order, $\alpha+1$ will be the order type of an order extending that order by adding one more item at the end.  Now $\omega\cdot 2$ is nothing more mysterious than $$\{0,1,2,3,\ldots,n,\ldots,\omega,\omega+1,\omega+2,\omega+3, \ldots,\omega+n\ldots\}$$

\end{description}

We demonstrate briefly that $\omega \cdot 2$ cannot be shown to exist.  The idea is that we can suppose that the universe consists exactly of the elements of the sets $\mathbb N$, ${\cal P}(\mathbb N)$, ${\cal P}^2(\mathbb N),\ldots$, and no other sets:  it is fairly straightforward to determine that the axioms of Zermelo set theory hold in this structure.  Then observe that  the ordinal $\omega$ appears first in ${\cal P}(\mathbb N)$ and more generally the ordinal $\omega+i$ appears first in ${\cal P}^{i+1}(\mathbb N)$.  There is no iterated power set of the set of natural numbers which contains all the elements of the von Neumann ordinal $\omega \cdot 2$, so this von Neumann ordinal does not appear in this structure.

We can say this more precisely.

\begin{description}

\item[Iterated power sets of $\mathbb N$:]  We define $x = {\cal P}^i(\mathbb N)$ as meaning ``There is a sequence $s$ with domain $i+1$
such that $s(0)={\mathbb N}$, for each $j < i$ we have $s(j+1) = {\cal P}(s(j))$, and $x = s(i)$".

\item[$^*$Axiom of restriction:]  For every set $x$, there is a natural number $i$ such that $x \in {\cal P}^i({\mathbb N})$.

\item[Observations:]  We will not assume the axiom of restriction.  But it is closely related to the idea of the cumulative hierarchy which we will soon introduce.  Note that if $\mathbb N$ is defined as the Zermelo natural numbers, it follows from the axiom of restriction that the set $N$ of von Neumann natural numbers does not exist (each von Neumann natural exists, but they do not all belong to any fixed iterated power set of the natural numbers).  Similarly, if $\mathbb N$ is defined as the von Neumann natural numbers, our official position, the axiom of restriction implies that the set of Zermelo natural numbers does not exist.

Further, the axiom of restriction allows a definition of cardinality!  Define $|A|$ as $\{B \in {\cal P}^{{\tt sup}(\{ i+1: A \not\preceq {\cal P}^i(\mathbb N)\})}(\mathbb N)\mid A \sim B\}$.  To read this, recognize that ${\tt sup}(\{ i+1: A \not\preceq {\cal P}^i(\mathbb N)\})$ is the successor of the smallest natural number $j$ such that $A \preceq {\cal P}^j(\mathbb N)$.  We define $|A|$ as the set of all sets $B$ which are equinumerous with $A$ and which belong to the first iterated power set of $\mathbb N$ which contains a set that large.  Similarly we could define ${\tt ot}(\leq)$ as
$\{B \in {\cal P}^{{\tt sup}(\{ i+4: {\tt fld}(\leq) \not\preceq {\cal P}^i(\mathbb N)\})}(\mathbb N)\mid [\leq] \approx B\}$:  define ${\tt ot}(\leq)$ as the set of all binary relations isomorphic to $\leq$ on the first iterated power set of the natural numbers which is large enough to support such a relation.  Under this definition, there is no difficulty defining $\omega \cdot 2$ (or much larger ordinal numbers):  it is the collection of all well-orderings on subsets of $\omega$ which are isomorphic to $<_{\tt bogus}$, which appears in a fairly low indexed iterated power set of the natural numbers.  This paragraph indicates that the problem with representing ordinals and cardinals in Zermelo is not that it is not strong enough;  the axiom of restriction does not make Zermelo set theory stronger;  it is more that the axioms of Zermelo set theory are not precise enough about the structure of the world of sets.   The axiom of restriction does make Zermelo set theory more precise in this respect.  The axioms we will introduce later both make the picture of the world of sets more precise and make the theory considerably stronger, in the sense that the extended theory proves the existence of more and bigger sets.

\end{description}

At this point our aim is to prove some facts about the structure of well-orderings.

\begin{description}


\item[Theorem:]  Any downward closed subset of the field of a well-ordering $\leq$ is a segment ${\tt seg}_{\leq}(x)$ or the whole of ${\tt fld}(\leq)$.

\item[Theorem:]  No well-ordering $\leq$ is isomorphic to any of its segment restrictions $(\leq)_x$.

\newpage

\item[Theorem:]  For any well-orderings $\leq_1$ and $\leq_2$ exactly one of the following is true:

\begin{enumerate}

\item  $\leq_1 \approx \leq_2$.

\item  For some $x \in {\tt fld}(\leq_1)$, $(\leq_1)_x \approx \leq_2$.

\item  For some $x \in {\tt fld}(\leq_2)$, $\leq_1 \approx (\leq_2)_x$.

\end{enumerate}

\item[Proofs:]  The proofs of all of these statements are {\em exactly\/} as in section 2.12.

\item[Definition:]  For ordinals $\alpha$ and $\beta$, we define $\alpha \leq_{\Omega} \beta$ as holding iff for some (and so for any) well-orderings $\leq_1, \leq_2$ such that
$\alpha = {\tt ot}(\leq_1)$ and $\beta = {\tt ot}(\leq_2)$ we have $\leq_1$ isomorphic either to $\leq_2$ or to some segment restriction $(\leq_2)_x$ for $x \in {\tt fld}(\leq_2)$.
We will usually write just $\leq$ for $\leq_{\Omega}$ except where some confusion might otherwise occur.

\item[Theorem:]  The relation $\leq_{\Omega}$ is reflexive, antisymmetric, transitive, and total, and for any set of ordinals $A$, there is a $\leq_{\Omega}$-minimal ordinal in $A$.  We do not say that $\leq_{\Omega}$ is a well-ordering only because it is not a set, as we will see shortly.

\item[Proof:]  This is proved in section 2.12, and the proof can be read using our definitions and works in exactly the same way.

\item[Theorem:]  The order type of the restriction of $\leq_{\Omega}$ to the set of ordinals $\beta$ such that $\beta<\alpha$ is $\alpha$, if this set exists.

\item[Proof:]  Assume that $\{\beta \mid \beta <_{\Omega} \alpha\}$ exists.  From this it follows that \newline $[\leq_{\Omega}] \cap \{\beta \mid \beta <_{\Omega} \alpha\}^2$ exists.
Suppose that this statement were  not true for some $\alpha$.   Then we can argue that there is a smallest ordinal $\gamma$ for which it is not true.  If $\{\beta \mid \beta<\alpha\}$
contains an ordinal for which the theorem is not true, then the smallest element $\gamma$ of $\{\beta \mid \beta<\alpha\}$ is the smallest counterexample.  Otherwise $\gamma=\alpha$ itself is the smallest counterexample.  So, let $\gamma$ be the smallest ordinal such that the order type of the restriction of $\leq_{\Omega}$ to
ordinals less than $\gamma$ is not $\gamma$.  Let $\gamma'$ be the actual order type of this restriction.  For each $\delta<\gamma$, the order type of the restriction of $\leq_{\Omega}$ to ordinals $<\delta$ has order type $\delta$, and so we see that every $\delta<\gamma$ is also $<\gamma'$, from which it follows that $\gamma \leq \gamma'$.
If $\gamma<\gamma'$, it follows that some segment restriction of the natural well ordering on ordinals $<\gamma$ must be of order type $\gamma$, but this is impossible:
each such segment restriction is the order type of the natural well-ordering on ordinals $<$ some $\delta<\gamma$, and this order type is by choice of $\gamma$ equal to $\delta<\gamma$.  So the order type of the natural order on the ordinals $<\gamma$ is $\gamma$, which is a contradiction.

$\dagger$ It is important to note that the statement proved here cannot be proved in type theory and in fact would not make sense in type theory, because the ordinal $\alpha$ would occur at more than one type.

\item[Theorem:]  There is no set of all ordinals, and $\leq_{\Omega}$ is not a set.

\item[Proof:]  If there were a set ${\tt Ord}$ of all ordinals, then $\leq_{\Omega}$ would be a set, and in fact a well-ordering, and so there would be an ordinal $\Omega = {\tt ot}(\leq_{\Omega})$.  We would have $\Omega \in {\tt Ord} = {\tt fld}(<_{\Omega})$.  By the previous theorem, for any ordinal $\alpha$, we have ${\tt ot}((\leq_{\Omega})_{\alpha}) = \alpha$ so in particular ${\tt ot}((\leq_{\Omega})_{\Omega}) = \Omega$.  But equally clearly ${\tt ot}((\leq_{\Omega})_{\Omega})$ is strictly less than ${\tt ot}(<_{\Omega})= \Omega$;  a well-ordering cannot be isomorphic to one of its segment restrictions.

\end{description}

We now develop our official definition of ordinals and state an axiom required to make it work.

\begin{description}

\item[Definition:]  A {\em system of von Neumann ordinal notation\/} is a function $f$ whose domain is the field of a well-ordering $\leq$ and which satisfies the condition $f(x) = \{f(y) \mid y \leq x \wedge y \neq x\}$ for all $x$ in the domain of $\leq$.

\item[Theorem:]  If $f$ is a system of von Neumann ordinal notation on the field of $\leq$ and $f(x)=f(y)$, then $x=y$: systems of von Neumann ordinal notation are injective.

\item[Proof:]  Let $x$ be the $\leq$-minimal element of the field of $\leq$ such that $f(x)=f(y)$ for some $y \neq x$.    Clearly $x \leq y$.  We are supposed to have
$$\{f(z) \mid z \leq x\wedge z \neq x\} = \{f(w) \mid w \leq y\wedge w \neq y\}.$$  Now observe that $f(x)$ belongs to the second set but cannot belong to the first.

\item[Theorem:]  If well-orderings $\leq$ and $\leq'$ are isomorphic and support systems of von Neumann ordinal notation $f$ and $g$,
then for each $x$ in the domain of $\leq$, $f(x) = g(y)$ iff $(\leq)_x \approx (\leq)_y$.

\item[Proof:]  Consider the smallest $x$ for which this is not the case.  We have $(\leq)_x$ isomorphic to $(\leq)_{h(x)}$, this being witnessed by the restriction of $h$ to the segment determined by $x$.  We have $f(z) = g(h(z))$ for each $z \leq x$.  And it follows that $f(x) = \{f(z) \mid z \leq x\} = \{g(h(z) \mid z \leq x\} = \{g(w) \mid w \leq' h(z)\} = g(h(x))$.  On the other hand, if $f(x) = g(y)$ for some $y \in {\tt fld}(\leq ')$, we have $f(x) = \{f(z) \mid z \leq x\} = \{g(h(z)) \mid z \leq x\} = g(h(z))$, whence $f(x)=g(y)$ since systems of von Neumann ordinal notation are injective.

\item[Axiom of Ordinals:]  On every well-ordering $\leq$, there is a system of von Neumann ordinal notation.

\item[Remark about the Axiom of Ordinals:]  This axiom or something stronger  is needed:  the ordinal $\omega \cdot 2+1$ can be shown not to support a system of von Neumann ordinal notation under the Axiom of Restriction.  The Axiom of Ordinals will be seen to be a consequence of the more powerful Axiom of Replacement that we will adopt later.

\item[Definition:]  We define ${\tt ot}((\leq)_x)$ as $f(x)$, where $f$ is a system of von Neumann ordinal notation on $\leq$.  It is straightforward to verify that every well-ordering
$\leq$ can be expressed in the form $(\leq')_x$ for some ``larger" well-ordering $\leq'$, and straightforward to establish that the value of ${\tt ot}(\leq)$ computed by the definition above does not depend on the choice of $\leq'$.  We refer to order types defined in this way as {\em von Neumann order types\/}, and we refer to any von Neumann order type as a {\em von Neumann ordinal number\/} or (usually) simply as an ordinal number.

\end{description}

\subsubsection{Exercises}

\begin{enumerate}

\item  Using the official definititions of $[X]^n$ and $[X]^{<\omega}$, prove that the von Neumann natural number $n$ is an $n$ element set
(that $n \in [\mathbb N]^n$ is probably easiest to prove).  Perhaps much harder, prove that $\mathbb N$ is an infinite set, using the official definition of
$[X]^{<\omega}$.  This all hinges on details of definitions, and in no way on common sense!

\item Prove the theorem $m \otimes n = m \cdot n$, verifying that the set based definition of multiplication is equivalent to the Peano arithmetic definition.  You may use the result about addition already established in the notes in your proof.

\item Present the proofs of the associativity of addition and multiplication of cardinal numbers of sets in the same style in which I presented the proof of the distributive property of multiplication over addition.  An adequate description of the bijection witnessing the claimed equation between cardinals in each case is all that is wanted.

\item  Prove that the sum of two magnitudes is a magnitude.  You may assume all familiar properties of positive rational numbers (the elements of magnitudes).  You have a definition of the sum of two magnitudes:  the point is to show that this set satisfies the defining conditions to be a magnitude.

\item I claimed (correctly) in class and in the notes that the supremum of a nonempty set $A$ of magnitudes which is bounded above is
the union $\bigcup A$ of the set $A$.  I also claimed that the infimum of a nonempty set $A$ of magnitudes which is bounded below is the intersection $\bigcap A$ of the set $A$.  This statement about the infimum (greatest lower bound) is not quite true!  Describe an exception and indicate how to correct this statement.  This hinges on details of the definition of magnitude.

\item Identify the smallest natural number $n$ such that $\mathbb R \in {\cal P}^n(\mathbb N)$.   This should be easy bookkeeping.

\item  Describe an injection from $\mathbb R$ into ${\cal P}(\mathbb N)$ and an injection from ${\cal P}(\mathbb N)$ into $\cal R$.  These maps do not need to be onto:  the point is to show that each set is the same size as a subset of the other.  Ordinary knowledge of the reals is all that is needed (this doesn't depend in any way on my fancy constructions).  Think about base 2 (or, for reasons having to do with unintended identifications, base 3) representations of the reals.

\item  Prove that a linear order $\leq$ has finite domain iff both $\leq$ and $\leq^{-1}$ are well-orderings.

\end{enumerate}

\newpage

\section{Zorn's Lemma, The Well-Ordering Theorem, and the official definition of cardinality}

In this section we will realize Zermelo's aim in the definition of his axioms for set theory:  we will prove the Well-Ordering Theorem, that every set is the field of a well-ordering.  Amachronistically, we will do this by proving Zorn's Lemma, a theorem whose proof is formally quite similar to the Well-Ordering Theorem, from which the Well-Ordering Theorem is easily proved, and which is technically very useful in set theory and in mathematics generally.

\begin{description}

\item[Definition:]  Fix a partial order $\leq$.  A {\em chain\/} in $c$ is defined as a linear order which is a subset of $\leq$.  An upper bound for a chain $\leq_c$ in $\leq$ is an $x \in {\tt fld}(\leq)$ such that for all $c \in {\tt fld}(\leq_c)$ we have $c \leq x$.  A {\em maximal element\/} for $\leq$ is
an $m \in {\tt fld}(\leq)$ such that for all $n$, $m \leq n$ implies $m=n$.

\item[Definition (ordinal indexing):]  Let $\leq$ be a well-ordering and let $\alpha$ be an ordinal.  We introduce the notation $[\leq]_{\alpha}$ for the unique $x$, if there is one, such that
the order type of $(\leq)_x$ is $\alpha$.

\item[Zorn's Lemma:]  Let $\leq$ be a partial order with the property that every chain in $\leq$ has an upper bound.  Then $\leq$ has a maximal element.

\item[Proof of the Well-Ordering Theorem from Zorn's Lemma:]  Let $A$ be a set.
Let $X$ be the set of well-orderings whose fields are subsets of $A$.  Define a partial order on $X$ by $[\leq_1] \leq [\leq_2]$ iff
$[\leq_1]=[\leq_2] \vee (\exists a \in A:[\leq_1]=[(\leq_2)_a])$:  that is, if $\leq_2$ is an end extension of $\leq_1$.  Every chain in this order on $X$ has a well-ordering of a subset of $A$ as the  union of its field, which is an end-extension of each element of the union of its field.  So there is a maximal element in the end extension order on $X$ by Zorn's Lemma, and this maximal element must be a well-ordering of all of $A$:  any order on a subset of $A$ whose field is not all of $A$ can be end-extended by adding another element of $A$ as a new largest element.

\item[Proof of Zorn's Lemma:]  Let $\leq$ be a partial order in which every chain has an upper bound.  Let $X$ be the set of all pairs
$(\leq_c,x)$, where $x$ is an upper bound for $\leq_c$, and moreover $x \in {\tt fld}(\leq_c)$ if and only if $x$ is maximal in $\leq$ (clearly we can always choose an upper bound for $\leq_c$ which is not in the field of $\leq_c$, except in the case where $\leq_c$ has a maximal element which is also a maximal element in $\leq$).   We now define $Y$ as the partition of $X$ whose elements are the sets $$Y_{\leq_c} = \{\left<\leq_c,x\right>\mid \left<\leq_c,x\right> \in X\}.$$  Now choose a choice set $F$ for the partition $Y$.  Notice that $F$ is a function which sends each chain in 
$\leq$ to one of its upper bounds, belonging to the field of the chain only if it is maximal in $\leq$.

Define a {\em special chain\/} in $\leq$ as a chain $\leq_s$ in $\leq$ which is a well-ordering and has the property that for each $x$ in the field of $\leq_s$ we have \newline $F((\leq_s)_x)=x$.

We claim that if $\leq_1$ and $\leq_2$ are special chains, either the two special chains are equal or one is a segment restriction of the other.  If this is not the case, it follows that there is an $\alpha$ such that $[\leq_1]_{\alpha} \neq [\leq_2]_{\alpha}$ (both being defined), and there will be a smallest such $\alpha$.  But then for this smallest $\alpha$ the segment restrictions $(\leq_1)_{[\leq_1]_{\alpha}}=(\leq_1)_{[\leq_2]_{\alpha}}$, from which it follows that $[\leq_1]_{\alpha}=F((\leq_1)_{[\leq_1]_{\alpha}})=F((\leq_1)_{[\leq_2]_{\alpha}})=[\leq_2]_{\alpha}$, which is a contradiction.

We rephrase the previous paragraph in a way which does not use ordinal indexing:  we do want it to be clear that existence of von Neumann ordinals is not required for this proof.
If $\leq_1$ and $\leq_2$ are special chains which are not equal, and neither of which is a segment restriction of the other, there must be a $\leq_1$-minimal $x$ and a $\leq_2$-minimal
$y$ such that $x \neq y$ and $(\leq_1)_x = (\leq_2)_y$, whence it follows that $x = F((\leq_1)_x) =  F((\leq_2)_y) = y$, which is a contradiction.

From this it follows that the set union of all special chains is itself a special chain:  this is again the union of a collection of well-orderings ordered by end extension, so it is a well-ordering as noted above, and further it clearly satisfies the special chain property.  Let the union of all special chains be denoted by $\leq_S$.  Now consider $F(\leq_S)$: this must be an upper bound for $<_S$:  if it is not in the field of $<_S$ then $<_S$ can be extended to a longer special chain by appending $F(\leq_S)$;  but this is impossible because $<_S$ is the union of all special chains.  Thus $F(<_S)$ is in the field of $<_S$, from which it must be maximal in $\leq$.

\end{description}

Now that we know that all sets can be well-ordered, we can present the official definition of cardinality.

\begin{description}

\item[Definition:]  For any set $A$, we define $|A|$ as the smallest ordinal $\alpha$ such that there is a well-ordering $\leq$ with field $A$ such that ${\tt ot}(\leq)=\alpha$.

\item[Observation:]  If this definition of cardinality is used, $|A| \sim A$.

\item [Proof of observation:]  $|A|$ is the order type of a well-ordering with field $A$, and the von Neumann order type of any well-ordering is equinumerous with the field of that well-ordering.  

\end{description}

Now we combine observations of the last two sections to raise a classic question.  We have shown that there is an uncountable set (for example
${\cal P}(\mathbb N)$).  It follows that there is a smallest uncountable ordinal, which we will call $\omega_1$.  Clearly $\omega_1$ is a cardinal
(for example $|\omega_1|=\omega_1$):   when we consider it as a cardinal we call it $\aleph_1$.

Now we can raise a famous question (Cantor's Continuum Hypothesis):

\begin{description}

\item[Question:]  Is it the case that $|{\cal P}({\mathbb N})| (= |{\mathbb R}|) = \aleph_1$?

\subsubsection{Exercises}

\begin{enumerate}

\item  Prove that the union of a collection of well-orderings which are linearly ordered by end-extension is a well-ordering.  Give an example of a collection of well-orderings which are linearly ordered merely by inclusion (the subset relation) whose union is {\em not\/} a well-ordering.

\end{enumerate}

\end{description}

\newpage

\section{The cumulative hierarchy picture and Replacement}

This section outlines the development of the cumulative hierarchy picture of the world of sets, and the principle of ``limitation of size" (collections are sets if they are small in a suitable sense).

\subsection{Basic definitions of ordinal and cardinal numbers in untyped set theory; the cumulative hierarchy introduced}

We now present the definitions of cardinal and ordinal number which are
usually used in {\em ZFC\/}.  We give those definitions (due to von
Neumann) but they have the limitations that they do not necessary work
in Zermelo set theory without Replacement (not all well-orderings can
be shown to have order types, nor can all sets be shown to have
cardinals) and the von Neumann definition of cardinal depends
essentially on the Axiom of Choice, as the Scott definition does not.

Of course this section, up to the Axiom of Ordinals, redevelops the notion of von Neumann ordinal already introduced in a different style in the previous section.

The informal motivation of the von Neumann definition of natural numbers and general ordinals is the following

\begin{description}

\item[$^*$Circular Definition:]  Each ordinal is the set of all preceding ordinals.

\item[Development:]  Thus the first ordinal 0 is $\emptyset$, 1 is $\{0\}$, 2 is $\{0,1\}$, 3 is $\{0,1,2\}, \ldots$.  And further, $\omega$ is the set of all finite ordinals
$\{0,1,2\ldots\}$, $\omega+1$ is $\{0,1,2,\ldots,\omega\}$, $\omega \cdot 2$ is  $\{0,1,2,\ldots,\omega, \omega+1,\omega+2\ldots\}$, and so forth.

\end{description}

We give a formal definition with the same effect.

\begin{description}

\item[Definition:]  A {\em transitive set\/} is a set $A$ such that $A \subseteq {\cal P}(A)$.  Equivalently, for all $x$, $y$, if $x \in y$ and $y \in A$, then $x \in A$ (this might suggest why the word ``transitive" is used).

\item[Definition:] A {\em (von Neumann) ordinal number\/} is a
transitive set which is strictly well-ordered by the membership
relation.  Equivalently, it is a set which is transitive, not a member of itself, and well-ordered by the inclusion (subset) relation.

\item[$\dagger$Observation (depending on section 3.7.1 below):] In our implementation of Zermelo set theory in set
pictures, a von Neumann ordinal number $\alpha$ is implemented by the
isomorphism type of strict well-orderings of type $\alpha+1$ (except
for 0, which is implemented by the order type of the usual empty
order).  In ${\mathbb E}_{T^2(\lambda)}$, only the ordinals less than
$\lambda$ are implemented in this way.  If $\lambda=\omega\cdot 2$,
this definition is not useful: the only infinite well-orderings with
order types are of the form $\omega+n$, but there are much longer
order types that are realized (such as $\omega_1$).  A hypothesis
adequate to make this definition useful is ``$\beth_{T^2(\lambda)}$
exists for each ordinal $\lambda$'' in the ambient type theory.  The
Axiom of Replacement of {\em ZFC\/} makes this definition usable (and
is much stronger).

\item[Definition:] The {\em (von Neumann) order type\/} of a
well-ordering $W$ is the von Neumann ordinal $\alpha$ such that the
union of the restrictions of the membership and equality relations to $\alpha$ is isomorphic to $W$.   Equivalently, the subset relation restricted to $\alpha$ is isomorphic to $W$.

\item[Definition:] The {\em (von Neumann) cardinality\/} of a set $A$
is the smallest von Neumann ordinal which is the order type of a
well-ordering of $A$.

\end{description}

Zermelo set theory cannot prove the existence of any von Neumann ordinals other than the finite ones, in its original formulation.  In modern reformulations, the axiom of infinity is often given as asserting the existence of the von Neumann ordinal $\omega$, in which case the first ordinal whose existence cannot be proved is $\omega \cdot 2$.

We handle this provisionally by extending Zermelo set theory with an

\begin{description}

\item[Axiom of Ordinals:]  For each well-ordering $\leq$, there is a von Neumann ordinal $\alpha$ (necessarily unique) which we denote
by ${\tt ot}(\leq)$, read {\em the order type of $\leq$\/}, such that the subset relation restricted to $\alpha$ is isomorphic to $\leq$.

\item[Remark:]  This axiom ensures that every well-ordering has a corresponding von Neumann ordinal to serve as its order type.

\end{description}

A further axiom gives an intuitive description of the way that we now envision the universe of sets as being built.  We start with the empty set and build the universe in stages indexed by the ordinals, taking power sets at successor stages and taking unions at limit stages.

\begin{description}

\item[Axiom of Levels:]  For each ordinal $\alpha$, there is a set $V_{\alpha}$, this scheme satisfying

\begin{enumerate}

\item $V_0 = \emptyset$

\item $V_{\alpha+1} = {\cal P}(V_{\alpha})$

\item for $\lambda$ a limit ordinal, $V_{\lambda}= \bigcup \{V_{\beta}\mid \beta<\lambda\}$.

\end{enumerate}

In addition, for every set $x$ there is an ordinal $\alpha$ such that $x \in V_{\alpha}$.

\end{description}

The Axiom of Ordinals and the Axiom of Levels are not as strong as the Axioms of Replacement and Foundation usually adjoined to Zermelo set theory,
but they do give a good picture of the structure which the universe of untyped set theory is usually intuitively understood as having.

We present a formulation of the construction of levels which makes it clear that the notation $V_{\alpha}$ can be defined in the language of set theory (it is not required
that we introduce a new primitive notion $V_{\alpha}$ to be able to talk about the levels).

\begin{description}

\item[Definition:] A {\em subhierarchy\/} is a set $H$ which is
well-ordered by inclusion and in which each successor in the inclusion
order on $H$ is the power set of its predecessor and each non-successor in the
inclusion order on $H$ is the union of all its predecessors in that
order.  A {\em rank\/} is a set which belongs to some subhierarchy.

\item[Theorem:] Of any two distinct subhierarchies, one is an initial
segment of the other in the inclusion order.  So all ranks are
well-ordered by inclusion.

\item[Proof:]  Let $H_1$ and $H_2$ be subhierarchies.

Suppose first that $H_1$ is included in $H_2$.  From this it follows that $H_1$ is an initial segment of $H_2$, unless there is an element
$h_2 \in H_2\setminus H_1$ which is a subset of some $h_1 \in H_1$.  Choose the minimal $h_1$ in the inclusion order on $H_1$ such that
there is $h_2$ with this property, and choose the minimal such $h_2$ for the given $h_1$.  If $h_2$ is a successor in the inclusion order on
$H_2$, then it is the power set of some $h_3 \in H_2$, and this $h_3$ must also belong to $H_1$ by minimality of $h_2$.  But then the successor
of $h_3$ in the inclusion order on $H_1$ exists (because $h_1$ properly includes $h_2$ and so $h_3$) and is the power set of $h_3$,
and of course $h_2$ is also the power set of $h_3$ because it is the successor of $h_3$ in the order on $H_2$, so $h_2 \in H_1$ which is a contradiction.  If $h_2$ is limit, it is the union of all elements of the domain of $H_2$ properly included in $H_2$, all of which actually are elements of $H_1$.  There is a first element of $H_1$ properly including all of these sets, because $h_1$ properly includes $h_2$ and so properly includes all of these sets.  But this first element is the union of all the elements of $H_1$ properly included in it, and so in fact is $h_2$, so $h_2 \in H_1$, again a contradiction.  We have established that if $H_1$ is included in $H_2$, then $H_1$ is an initial segment of $H_2$ in the inclusion order.

Now suppose that $H_1$ is not included in $H_2$.  There must then be a first $h_2$ in the inclusion order on $H_2$ such that $h_2 \not\in H_1$.
We claim that in this case $H_1$ is an initial segment of $H_2$ in the inclusion order.  Certainly the collection $H_3$ of subsets of $h_2$ which belong to
$H_1$ is such an initial segment.  Suppose that there is $h_1 \in H_1$ which is not a subset of $h_2$;  choose the minimal such $h_1$.  If $h_1$ is
a successor in the inclusion order on $H_1$, it must be the power set of some $h_3 \in H_1$ which is a subset of $h_2$.  Now we argue that $h_2$
must also be the power set of $h_3$:  $h_3$ is included in $h_2$, and if $h_2$ were not its immediate successor in $H_2$ then its immediate successor would be the power set of $h_3$ and would be included in $h_2$ as a subset, so $h_1$ would be included as a subset in $h_2$, which is a contradiction.  But also $h_1$ cannot be equal to $h_2$ because $h_1 \in H_1$ and $h_2 \not\in H_1$.  We are left in the case where $h_1$ is the union of all elements of $H_3$.  Now we observe that there must be a first $h_4$ in $H_2$ which includes all elements of $H_3$, because $h_2 \in H_2$ includes all elements of $H_3$, and this
must be the union of all elements of $H_3$ as well.  But this means that $h_4 \in H_2$ is equal to $h_1 \not\in H_2$, which is a contradiction.\footnote{This is gruesome.  It might make a classroom exercise?  Or I may just need to rewrite it.}

\newpage

\item[Alternative definition of hierarchy given in lecture:]  Alternatively, we define a {\em hierarchy along a well-ordering $\leq$\/} as a function $h$ with domain ${\tt fld}(\leq)$
and satisfying the following conditions:

\begin{enumerate}

\item The image of the $\leq$-first element of ${\tt fld}(\leq)$ is $\emptyset$.

\item If $y$ is the immediate successor of $x$ in the order $\leq$, then $h(y) = {\cal P}(h(x))$

\item If $y$ is an element of ${\tt fld}(\leq)$ which is not an immediate successor in the order $\leq$, then $h(y) = \bigcup \{h(z) \mid z \leq y \wedge z \neq y\}$.

\item Conditions 1-3 can be replaced by the single condition ``For all $x \in {\tt fld}(\leq)$, $h(x) = \bigcup \{{\cal P}(h(y)) \mid y \leq x\}$".

\end{enumerate}

\item[Comments on the alternative definition:]  It should be clear that the intention is that for any $x \in {\tt fld}(\leq)$, that $h(x) = V_{{\tt ot}((\leq)_x)}$.  We can then give an alternative definition of rank:
a {\em rank\/} as any set which belongs to the range of any hierarchy along any well-ordering.  It is straightforward to prove that for any well-orderings $\leq_1$ and $\leq_2$, with hierarchies $h_1$ and $h_2$ along them, $h_1(x) = h_2(y) \leftrightarrow (\leq_1)_x \approx (\leq_2)_y$, for any $x$ in the field of $\leq_1$ and $y$ in the field of $\leq_2$: all hierarchies agree in a suitable sense.

\item[Axiom of Rank:]  Every set is a subset of some rank.


\item[Definition:] For any formula $\phi[x]$, define $r_{\phi[x]}$ as
the minimal rank $r$ such that $(\exists x \in r.\phi[x])$, or as the
empty set if there is no such rank $r$.  Define $\{x :: \phi\}$ as $\{x
\in r_{\phi[x]}\mid\phi[x]\}$.  $\{x :: \phi[x]\}$ is obviously a set for all
formulas $\phi[x]$.

\item[Definition:] $A \sim B$ iff there is a bijection from $A$ onto
$B$, as in type theory.  $|A|$, the {\em Scott cardinal\/} of $A$, is
defined as $\{B :: B \sim A\}$.  For any relations $R$ and $S$, we say
that $R \approx S$ iff there is a bijection $f$ from ${\tt fld}(R)$
onto ${\tt fld}(S)$ such that $x\, R \, y \leftrightarrow
f(x)\,S\,f(y)$.  We define the {\em Scott isomorphism type\/} of $R$
as $\{S :: S \approx R\}$.  Scott isomorphism types of well-orderings
are called {\em Scott order types\/} (of the well-orderings:
${\tt ot}(W)$ is the Scott order type of a well-ordering $W$) or {\em
Scott ordinal numbers\/} (as a class).

\end{description}

Notice that this ``Scott trick" allows us to recover the ability to define isomorphism types of objects
as (restricted) equivalence classes.

Now that we have defined Scott ordinals we can define the notation $V_{\alpha}$.

\begin{description}


\item[Definition:] For any subhierarchy $h$, introduce the nonce
notation $\leq_h$ for the inclusion order restricted to $h$.  If
$\alpha$ is a Scott ordinal we define $V_{\alpha}$ as the rank $A$ (if
there is one) such that ${\tt ot}((\leq_h)_A) = \alpha$ for any
$\leq_h$ with $A$ in its field.  It is straightforward to show that
${\tt ot}((\leq_h)_A)$ is the same ordinal for any $\leq_h$ with $A$
in its field, and that $A$ is uniquely determined by $\alpha$.  We can also use the alternative definition:  define $V_{\alpha}$ as
the value of any hierarchy along a well-ordering $\leq$ of order type $\alpha+1$ at the $\leq$-maximum element of ${\tt fld}(\leq)$.

\end{description}

In Zermelo set theory with the Rank Axiom we can prove that every set
belongs to some $V_{\alpha}$ but we cannot prove the existence of
$V_{\omega\cdot 2}$.  But we do have the ability to define order types
for every well-ordering and cardinals for every set using the Scott
definitions.  And we can then recover the full Axiom of Levels as the

\begin{description}

\item[Axiom of Hierarchy:]  For each Scott ordinal $\alpha$, $V_{\alpha}$ exists.  Equivalently, there is a subhierarchy isomorphic to each well-ordering (or a hierarchy along each well-ordering, using the alternative definition).

\end{description}

It is an immediate consequence of the Axiom of Hierarchy that each von Neumann ordinal exists (the von Neumann $\alpha$ exists in $V_{\alpha+1}$).  So the axioms of Rank and Hierarchy have the same effect as the axioms of Ordinals and Levels.  Once we have the axiom of hierarchy, we can if we prefer use von Neumann ordinals instead of Scott ordinals to index $V_{\alpha}$'s and serve as order types of well-orderings.



\newpage


\newpage

\subsection
{More on the von Neumann definitions of ordinal and cardinal number}

We introduced the perhaps mysterious traditional definition of {\em
ordinal number\/} due to von Neumann above:

\begin{description}

\item[Definition:] An {\em ordinal number\/} is a transitive set $A$
which is strictly well-ordered by membership (i.e., the restriction $$[\in] \cap\, A^2= \{\left<x,y\right> \in A \times A\mid x \in y\}$$ of
the membership relation to $A$ is the strict partial order
corresponding to a well-ordering).  Or ``a transitive set of
transitive sets none of which are self-membered''.

\item[Observation:] This is equivalent to ``$x$ is an ordinal
iff $x$ is a transitive set, no element of $x$ is self-membered, and
$x$ is well-ordered by inclusion''.  This has the merit that our
preferred definition of well-ordering is used.  Let $x$ be an ordinal
by this definition.  For each $y \in x$, and each $z \in y$, we have
$z \in x$ because $x$ is transitive, so we have either $z \subset y$
or $y \subseteq z$.  But $y \subseteq z$ is impossible because this
would imply $z \in z$.

\item[Definition:] For any ordinal $\alpha$, use $\in_{\alpha}$ to
represent $[\in]\cap\,\alpha^2$ (which we know is a strict
well-ordering).

\item[Theorem:] For any ordinal $\alpha$ and any $\beta \in \alpha$,
$\beta$ is an ordinal, $\beta = {\tt seg}_{\in_{\alpha}}(\beta)$ and
$\in_{\beta} = (\in_{\alpha})_\beta$.

\item[Proof:] $\delta \in \gamma \in \beta \rightarrow \delta
\in_{\alpha} \gamma \in_{\alpha} \beta$ ($\gamma, \delta \in \alpha$
because $\alpha$ is transitive) and this implies $\delta \in_{\alpha}
\beta$ and so $\delta \in \beta$ because $\in_{\alpha}$ is a partial
order.  Thus $\beta \in \alpha$ is transitive.  $[\in]\cap\,\beta^2$ is a
strict well-ordering because it is a suborder of $[\in]\cap\,\alpha^2$.
Further, it is evident that $\in_{\beta} = (\in_{\alpha})_{\beta}$:
the order on $\beta$ is the segment restriction of the order on
$\alpha$ determined by $\beta$, because $\beta$ is identical to the
segment in the order on $\alpha$ determined by $\beta$: $\beta =
\{\gamma \in \alpha \mid \gamma \in \beta\}$ (this uses transitivity
of $\alpha$) $= \{\gamma \mid \gamma \in_{\alpha} \beta\}$, which is what we mean by ${\tt seg}_{\in_{\alpha}}(\beta)$

\item[Theorem:] For any two ordinal numbers $\alpha$ and $\beta$,
exactly one of the following is true: $\alpha=\beta$, $\alpha\in \beta
\wedge \alpha\subseteq \beta$, $\beta \in \alpha \wedge \beta
\subseteq \alpha$.  Any set of ordinal numbers is thus linearly
ordered by $\subseteq$: moreover, this linear order is a
well-ordering.

\item[Proof:] By a basic theorem on well-orderings proved above, we
know that there is either an isomorphism from $\in_{\alpha}$ to
$\in_{\beta}$, an isomorphism from $\in_{\alpha}$ to some
$(\in_{\beta})_{\gamma} = \in_{\gamma}$ for some $\gamma \in \beta$ or
an isomorphism from $\in_{\beta}$ to some $(\in_{\alpha})_{\gamma} =
\in_{\gamma}$ for some $\gamma \in \alpha$.  It is then clearly
sufficient to show that for any ordinals $\alpha$ and $\beta$, if
$\in_{\alpha} \approx \in_{\beta}$, then $\alpha=\beta$.  Suppose for
the sake of a contradiction that $f:\alpha \rightarrow \beta$ is an
isomorphism from $\in_{\alpha}$ to $\in_{\beta}$ and that there is
some $\gamma \in \alpha$ such that $f(\gamma) \neq \gamma$.  There is
then a $\in_{\alpha}$-least such $\gamma$.  We have $\gamma$ as the
$\in_{\alpha}$-least element of $\alpha$ such that $f(\gamma) \neq
\gamma$.  The objects which are $\in_{\beta} f(\gamma)$ are exactly
those $f(\delta)$ such that $\delta \in_{\alpha} \gamma$ (this is just
because $f$ is an isomorphism).  We can read $\in_{\alpha}$ and
$\in_{\beta}$ simply as membership, and we remind ourselves that for
any $\delta<\gamma$ $f(\delta) = \delta$, and thus we see that
$\gamma=f(\gamma)$ because they have the same members, which is a
contradiction.

That $\alpha \in \beta \rightarrow \alpha\subseteq \beta$ expresses
the fact that ordinals are transitive sets.  $\subseteq$ is a partial
order on sets and what we have shown so far indicates that it is a
linear order on ordinals.  To see that it is a well-ordering, we need
to show that any nonempty set $A$ of ordinals has a $\subseteq$-least
element: since $A$ is nonempty, we can choose $\alpha \in A$; either
there is some $\beta \in \alpha$ which is an element of $A$ or there
is not.  If there is none, then $\alpha$ is the $\subseteq$- (and
$\in$-) least element of $A$; otherwise the $\subseteq$- (and $\in$-)
least element of $\alpha$ which belongs to $A$ will be the
$\subseteq$- (and $\in$-) least element of $A$: there is such an
element because $\in$ is a strict well-ordering of $\alpha$ and so
$\subseteq$ is a well-ordering of $\alpha$.

\item[Definition:] For any well-ordering $\leq$, we define
${\tt ot}(\leq)$ as the ordinal $\alpha$ (if any) such that the
well-ordering of $\alpha$ by $\subseteq$ is isomorphic to $\leq$.

\item[Definition:] For any set $A$, we define $|A|$, the cardinality of $A$,  as the minimal
ordinal $\alpha$ in the inclusion order such that $A \sim \alpha$.  Ordinals which are cardinals are also called initial ordinals.  For any ordinal $\alpha$, we define $\beth_{\alpha}$ as the infinite cardinal with the property that the order type of the inclusion order on smaller infinite cardinals is $\alpha$.
\end{description}

This definition of cardinal does not make sense unless we assume the 
Axiom of Choice (so that every set can in fact be well-ordered) and at least
the Axiom of Ordinals (so that every well-ordering has a von Neumann order type).
The Scott definition is available as an alternative if the Axiom of Rank (or the Axiom of Levels) is present.  We are assuming in general in this section
that we are assuming either Ordinals and Levels, or Rank and Hierarchy (each of these pairs of axioms has the same effect).

Note that in the usual set theory we identify a cardinal number with
its initial ordinal: these are not the same object in type theory,
though of course they are closely related.  This is another of those
differences between possible implementations of mathematical concepts
in set theory that one should watch out for (in the Scott
implementation of cardinals and ordinals in Zermelo set theory with the Axiom of Rank, a
cardinal is not identified with its initial ordinal).  The fact that
though we identify these concepts formally in {\em ZFC\/} we do not
actually think of them as having the same mathematical content is
witnessed by the fact that we use different notations for $\mathbb N$
(the set of natural numbers), $\omega$ (the first infinite ordinal)
and $\aleph_0$ (the first infinite cardinal) although these are all
implemented as exactly the same object!  Note that in type theory they
are all different.

It is important to notice that just as there can be no set $V$ of all
sets in Zermelo set theory, there can be no set ${\tt Ord}$ of all
ordinals (so transfinite induction and recursion must be stated in
property-based or restricted forms in this theory).  For the ordinals
are strictly well-ordered by membership in an obvious external sense:
if there were a set $\Omega$ which contained all ordinals, it would be
an ordinal, so we would have $\Omega \in \Omega$, and this is
impossible again by the definition of ordinal.  This is a version of
the Burali-Forti paradox, another of the classical paradoxes of set
theory.

\newpage

\subsubsection{Exercises}

\begin{enumerate}

\item  
The Scott definition of a natural number $n$ 
is that it is the collection of all sets of size $n$ and rank as low
as possible.  Remember the rank of a set $A$ is the first ordinal
$\alpha$ such that $A$ is a subset of $V_{\alpha}$.  Write down as
many Scott natural numbers as explicit sets as you can stand to.  Work
out the sizes of the next few (how many elements do they have? -- go
up to 20 or so?)  All you need for this is an understanding of what
$V_0, V_1, V_2\ldots$ (the finite ranks) are, and some familiar
combinatorics.  You might also want to see what you can say about the
Scott natural number 60000 versus the Scott natural 70000.  
There is a dramatic difference (smiley).



\item  The Axiom of Foundation asserts that for any nonempty set $x$
there is a set $y \in x$ such that $x \cap y = \emptyset$.

One way of understanding this is that this axiom says that if we look
at $[\in] \cap\, x^2$ (the membership relation on $x$) that it must have
a ``minimal'' element -- ``minimal'' is in scare quotes because
membership is not an order relation.  A ``minimal'' element $y$ will
have empty preimage under the membership relation restricted to $x$ --
that is, it will have no elements in common with $x$.

Use the Axiom of Foundation (along with the other axioms of course) to prove
the following:

\begin{enumerate}

\item There is no set $x$ such that $x \in x$.

\item There is no sequence $s$ such that for all $n \in {\mathbb N}$ we have
$s_{n+1} \in s_n$.

\end{enumerate}

The strategy to follow is this: in each part, identify a set which
would have no ``minimal'' element in the membership relation.


\end{enumerate}

\newpage

\subsection{The Axiom of Replacement and {\em ZFC\/}}

We introduce the missing assumption of the usual set theory which
makes it possible to prove that the von Neumann definitions of ordinal
and cardinal number are total.  The axioms of Replacement and Foundation imply
the axioms of Ordinals and Levels (or Rank and Hierarchy) and are in fact considerably stronger.

\begin{description}

\item[class function notation:] If we have a formula $\phi[x,y]$ such
that for every $x$ there is at most one $y$ such that $\phi[x,y]$, we
introduce notation $y = F_{\phi}(x)$ for the unique $y$ associated
with a given $x$.  Notice that $F_{\phi}$ is not understood to be a
set here.

\item[Axiom (scheme) of Replacement:] If we have a formula $\phi[x,y]$ such
that for every $x$ there is at most one $y$ such that $\phi[x,y]$, and
define $F_{\phi}(x)$ as above, then for every set $A$, the set
$\{F_{\phi}(x) \mid x \in A\}$ exists.

This is called an axiom scheme rather than an axiom, because we actually have a distinct axiom for each formula, in a technical sense.

\end{description}

The Axiom of Replacement can be used then to justify the recursive
definition of the $V_{\alpha}$'s above.  What the axiom of replacement
says, essentially, is that any collection we can show to be the same
size as or smaller than a set is in fact a set.

\begin{description}

\item[Theorem:] ${\tt ot}(\leq)$ exists for every well-ordering $\leq$.

\item[Proof:] Let $\leq$ be a well-ordering such that ${\tt ot}(\leq)$
does not exist.  If there are $x$ such that ${\tt ot}((\leq)_x)$ does
not exist, define $\leq_0$ as $(\leq)_x$ for the smallest such $x$;
otherwise define $\leq_0$ as $\leq$ itself.  In either case $\leq_0$
is a well-ordering which has no order type with the property that all
of its initial segments have order types.  We now define a formula
$\phi[x,\alpha]$ which says ``$\alpha$ is the order type of
$(\leq_0)_x$'' (the tricky bit is showing that we can say this).
Notice that once we do this we are done: $\{F_{\phi}(x) \mid x \in
{\tt fld}(\leq_0)\}$ will be the first von Neumann ordinal after all
the order types of segment restrictions of $\leq_0$, which will be the
order type of $\leq_0$ contrary to assumption.

$\phi[x,\alpha]$ is defined as ``if $f$ is a (set) function with
domain an initial segment of $\leq_0$ containing $x$ and having the
property $f(y) = \{f(z) \mid z \leq_0 y\}$ for each $y$ in its domain,
then $f(x) = \alpha$''.  It is straightforward to prove that exactly
one such function $f$ exists for each initial segment of $\leq_0$ (its
extendability at limits in $\leq_0$ uses Replacement).

We have already seen that provision of this formula leads to a
contradiction to our original assumption.

\item[Corollary:]  The von Neumann cardinal $|A|$ exists for every set $A$.

\item[Proof:] There is a well-ordering of $A$, whose order type is an
ordinal with the same cardinality of $A$.  Either this is the smallest
ordinal (in the inclusion order) with this property, in which case it
is $|A|$ itself, or it has elements which have this property, among
which there must be a smallest, which is $|A|$.

\item[Theorem:]  $V_{\alpha}$ exists for each $\alpha$.

\item[Proof:] Consider the smallest ordinal $\lambda$ for which
$V_{\lambda}$ does not exist (it is obviously a limit ordinal if it
exists).

Find a formula $\phi[\alpha,A]$ which says ``$A = V_{\alpha}$'' and we
are done, because we can then define the set $\{F_{\phi}(\alpha) \mid
\alpha \in \lambda\}$, and the union of this set will be $V_{\lambda}$
contrary to assumption.

The formula $\phi[\alpha,A]$ says ``there is a function $f$ whose
domain is an ordinal $\beta$ such that $\alpha \in \beta$, and $f(0) =
\emptyset$, $f(\gamma+1) = {\cal P}(f(\gamma))$ if $\gamma+1\in \beta$,
and $f(\mu) = \bigcup\{f(\gamma) \mid \gamma\in \mu\}$ for each limit
ordinal $\mu\in \beta$, and $f(\alpha) = x$''.  The fact that there is
a unique such function $f$ for each $\beta<\lambda$ is readily shown:
Replacement is used to show extendability of $f$ at limit ordinals.

\end{description}

Zermelo set theory augmented with the Axioms of Replacement and Foundation is known as
{\em ZFC\/} (Zermelo-Fraenkel set theory with Choice).  This is the system
of set theory which is most commonly used.

The Axiom of Foundation has been mentioned:  we restate it.

\begin{description}

\item[Axiom of Foundation:]  For each set $x$, there is $y \in x$ such that $x \cap y = \emptyset$

\end{description}

The intention of this axiom is to assert that the membership relation restricted to any set is well-founded.

An informal way to explain the motivation of the axiom is that the sets are constructed in well-ordered stages.  If $x$ is any set, there must be a first stage at which an element $y$ of $x$ is constructed (possibly more than one, but we select one).  The idea then is that $y$, if it is constructed at a stage of positive index, must have all of its elements constructed at earlier stages, so none of them belong to $x$, so $x \cap y = \emptyset$.  If $y$ were constructed at stage 0 we would need to say more:  but in our construction of ranks, stage 0 has no elements!  Now we can observe that the Axiom of Restriction given earlier also implies Foundation for the same reason:  stage 0 in the construction underlying the axiom of Restriction is $\mathbb N$, and the further remark to be made is that if the element $y$ is constructed at stage 0 it is a natural number, and we can further choose $y$ to be the smallest natural number belonging to $x$:  its elements (whether we use the von Neumann or the Zermelo implementation of $\mathbb N$) are smaller natural numbers and so do not belong to $x$.

It is useful to note that the Axiom of Separation is almost redundant in the presence of Replacement.  

\begin{description}

\item[Theorem:]  Zermelo set theory without Separation (and with the assertion that the empty set exists:  this is part of the Axiom of Elementary Sets in the original formulation, but in more usual formulations it is deduced from Separation and the existence of any set at all, such as the one provided by Infinity), with the addition of the Axiom of Replacement,  proves Separation.

\item[Proof:]  Let $\phi[x]$ be a formula and let $A$ be a set.  If $\neg(\exists x:\phi[x])$ then $\{x \in A\mid \phi[x]\} = \emptyset$ exists.  Otherwise, choose $a$ such that $\phi[a]$ and define $\psi[x,y]$ as $\phi[x] \wedge y=x \vee \neg \phi[x] \wedge y=a$.  Clearly $\psi$ is a functional formula,
so $\{y \mid (\exists x:x \in A \wedge \psi[x,y])\}$ exists, and this set is $\{x \in A \wedge \phi[x]\}$.

\end{description}

Although the Axiom of Replacement is sufficient to make the von
Neumann definitions of cardinality and order type succeed, it is
certainly not necessary.  A weaker axiom with the same effect is the Axiom of Levels or the Axiom of Hierarchy, which can also be stated as the

\begin{description}

\item[Axiom of Beth Numbers:] For every Scott ordinal $\alpha$,
$\beth_{\alpha}$ exists.

\end{description}

We define things in terms of Scott ordinals because we do not wish to
presume that the von Neumann ordinal $\alpha$ exists; that is what we
are trying to prove.  A set of size $\beth_{\alpha}$ must be included
in a rank $V_{\beta}$ with $\beta\geq \alpha$, and the von Neumann
ordinal $\alpha$ will be present in $V_{\beta+1}$.  Notice that the
Axiom of Rank plays an essential role in this argument: existence of
large $\beth$ numbers in the original Zermelo set theory does not have
any effect on existence of von Neumann ordinals.

Another axiom which works is the stronger

\begin{description}

\item[Axiom of Beth Fixed Points:]  For every cardinal $\kappa$, there is a cardinal $\lambda>\kappa$ such that $\beth_{\lambda} = \lambda$.

\end{description}

A consequence of Foundation and Replacement which is often useful is the Mostowski Collapsing Lemma which we now present.

\begin{description}

\item[Review of definitions:]  If $R$ is a relation, we define $R``A$ as $\{y\mid (\exists x \in A:x\,R\,y\}$.   Thus $R^{-1}``A$ is
$\{x\mid (\exists y \in A:x\,R\,y\}$.  A relation $R$ is {\em well-founded\/} iff for every subset $A$ of ${\tt fld}(R)$ there is $a \in A$ such that
$R^{-1}``\{a\}$ is empty ($a$ is $R$-minimal).  Notice that a nonempty well-ordering is not well-founded, but a strict well-ordering is.

\item[Mostowski Collapsing Lemma:]  For every well-founded relation $R$, there is a unique function $f$ with domain ${\tt fld}(R)$ such that
$$(\forall a \in {\tt fld}(R) : f(a) = \{f(b) \mid b \,R\,a\}).$$

\item[Proof of Lemma:]  Define ${\tt cl}_R(a)$ for each $a \in {\tt fld}(R)$ as the intersection of all sets $I$ such that $a \in I$ and $R^{-1}``I \subseteq I$:  this can be thought of as the downward closure of $\{a\}$ under $R$.

Consider the set $D$ of all elements $a$ of ${\tt fld}(R)$ such that $R \cap {\tt cl}_R(a)^2$ satisfies the condition stated in the Lemma,
that is, there is a unique function $f_a$ with domain ${\tt cl}_R(a)$ such that $(\forall b \in {\tt cl}_R(a) : f_a(b) = \{f(c) \mid c\,R\,b\})$.

If $D = {\tt fld}(R)$ then observe that the set $F$ of all such functions $f_a$ exists by Replacement (the function $f_a$ is a unique object associated with each $a \in {\tt fld}(R)$ in a way which can be defined by a formula) and $\bigcup F$ is the desired function $f$.  To see this observe that
if any two functions $f_a, f_b \in F$ both have $c$ in their domain, the restrictions of each of $f_a$ and $f_b$ to ${\tt cl}_R(c)$ (which is a subset of
both ${\tt cl}_R(a)$ and ${\tt cl}_R(b)$) must actually be the function $f_c$, so $f_a$ and $f_b$ agree at $c$, and $\bigcup F$ is a function
(and certainly satisfies the stated conditions).

If $D \neq {\tt fld}(R)$ then there must be an $R$-minimal $d \in {\tt fld}(R) -D$.  Each $c \, R \, d$ has an associated function $f_c$,
If we extend the union of the functions $f_c$ for $c \, R\,d$ (which must be a function by the same argument above) to a function $g$ with the additional value $g(d) = \{f_c(c) \mid c \, R \, d\}$, the function $g$ satisfies the conditions to be $f_d$ (and clearly is the only function which can do this), and so $d \in D$, which is a contradiction.

\item[Theorem:]  For every set $A$, there is a well-founded relation $R$ with unique function $f$ associated to it by the Lemma such that
$f$ is the identity function on the field of $R$ and $A$ is in the range of $f$ (so $f(A)=A$).

\item[Proof:]  Suppose otherwise.  Let $B$ be a counterexample.  We seek a further special counterexample $C$.  If $B$ has no elements which are counterexamples, then let $C=B$;  otherwise let $C$ be the $\in$-minimal counterexample belonging to $B$.  In either case, $C$ is a counterexample
and none of its elements are counterexamples.  With each $D \in C$, associate $R_D$, the intersection of all well-founded relations with $D$ in their range satisfying the condition that the associated function $f_D$ provided by the Mostowski Collapsing Lemma is the  identity function.  Take the union of the sets $R_D$ and add the pair $\left<C,C\right>$ to it as an element.  This relation $R_C$ satisfies the desired conditions, which is a contradiction.

\end{description}

This is a weird way of putting a proof of a useful result.  For each set $z$, a relation $R$ which is well-founded, has $z$ in its field,
and has the associated function $f$ equal to the identity function on the field of $R$ must actually be the restriction of the membership relation to ${\tt fld}(R)^2$.
The field of the intersection of all such sets must be a transitive set containing $z$, and in fact the smallest one (because membership on any transitive set containing $z$ satisfies the indicated conditions!).  This set is called ${\tt TC}(\{x\})$, the transitive closure of $\{x$\}. (${\tt TC}(x)$ would differ in not containing $x$ as a member).  There is an exercise which addresses proving the existence of this set in a different way.

An alternative pair of axioms to adjoin to Zermelo set theory which would tidy up such questions as existence of versions of the natural numbers and von Neumann order types of every well-ordering would be the combination of Foundation and the Mostowski Collapsing Lemma.  This would correct technical problems with the original formulation of Zermelo set theory, but in a different way.  Proving the existence of $V_{\omega}$ would be straightforward.  One could not prove the existence of $V_{\omega\cdot 2}$, but one could prove the existence of the von Neumann order type of any set well-ordering:  one could prove the existence of lots of sets for which one could not prove the existence of the rank of the cumulative hierarchy to which one would expect them to belong.  A world satisfying the Zermelo axioms and Foundation and Mostowski could be built in countably many stages:  let $H_0$ be $V_{\omega}$.  Define $H_{n+1}$ as the set of elements of ranges of Mostowski collapse functions on well-founded relations included in  $H(n)^2$.

\newpage


\subsubsection{Exercises}
\begin{enumerate}
\item  This proof will use Replacement.

In the usual axiom set it is rather more involved than it seems it
ought to be to show that every set is a subset of a transitive set
(this is easily shown in cumulative type theory or in Zermelo set
theory with the rank axiom, but the usual formulation of Zermelo set
theory or {\em ZFC\/} has the Foundation axiom, which is weaker).

I give an outline of a proof which you need to complete (there are models
in the notes for the proof).

Let $X$ be a set.  We want to prove that there is a transitive set
which contains $X$.  The idea is to prove that the collection of sets
$\{\bigcup^{\bf n}(X)\mid n \in {\mathbb N}\}$ exists.  Then you can show that
the union of this set is transitive and contains $X$ as a subset.

Fill in the details.  To prove the existence of $\{\bigcup^{\bf n}(X)\mid n
\in {\mathbb N}\}$ by Replacement you need a formula $\phi[n,x]$ which
says ``$x =\bigcup^{\bf n} X$''.  As I said, there are models for this in
the notes.

Why does it follow immediately from ``$X$ is a subset of a transitive
set'' that $X$ is an element of some transitive set as well?

Define the {\em transitive closure\/} ${\tt TC}(x)$ for any $x$ as the intersection of all transitive sets including $x$ as a subset:
this set contains exactly the elements of $x$, elements of elements of $x$, elements of elements of elements of $x$, and so forth.  It exists by this exercise.
Note that ${\tt TC}(\{x\})$ contains $x$ as an element in addition.

\item Prove using the axioms of Zermelo set theory and the axioms of Replacement and Foundation that every set $x$ belongs to some rank $V_{\alpha}$.  You may use the result established in the section that $V_{\alpha}$ exists for any ordinal $\alpha$.  Hint:  Suppose that there is a set $x$ which belongs to no rank of the cumulative hierarchy.  Consider an $\in$-minimal element of ${\tt TC}(\{x\})$ which does not belong to any rank $V_{\alpha}$; of course you have to say why the hypotheses imply that there is such an object, and why bad things follow from this.

You will note that this is a proof that our Axiom of Rank is a consequence of the axioms of {\em ZFC\/}.  Of course, you cannot assume the Axiom of Rank in your argument.

\item

This question is intended to address the question of just how weird a
model of Zermelo set theory without the Axiom of Rank can be.



Work in {\em ZFC\/}.  Define a set $A$ as {\em bounded\/} iff its
transitive closure (defined in the first exercise) contains finitely many von Neumann natural numbers.
We refer to the first von Neumann natural not in the transitive
closure of $A$ as the bound of $A$.  Verify the following points:

\begin{enumerate}

\item  If $a$ and $b$ are bounded, $\{a,b\}$ is bounded.

\item If $A$ is bounded, ${\cal P}(A)$ is bounded (but the bound might
go up by one -- do you see why?), and $\bigcup A$ is bounded (with the
same bound).  You should also show that the bounds of the sets ${\cal
P}^k(A)$ eventually increase with $k$.

\item  The set of Zermelo natural numbers is bounded.

\item If the bound of $A$ is $n$, the set ${\cal P}^{>n}(A)$ of all
subsets of $A$ with more than $n$ elements is also bounded with bound
$n$ (note that this can be iterated).

\item Apply the points above to argue that the collection of all
bounded sets in the universe of {\em ZFC\/} is a model of Zermelo set
theory in which the set of von Neumann natural numbers does not exist,
in which $\{{\cal P}^n(X) \mid n \in {\mathbb N}\}$ does not exist for
any $X$, and in which there are sets at least as large as any set which exists
in the universe of {\em ZFC\/} (for this last point you may assume without proof that for an infinite set $A$, ${\cal P}(A)$ is the same size as ${\cal P}^{>n}(A)$;  of course I would enjoy it if you could prove this.)  Hint:  show that there is a bounded set at least as large as $V_{\alpha}$ for each $\alpha$.

\end{enumerate}

\end{enumerate}

\newpage

\subsection{Transfinite Induction and Recursion}

The reader may already have the idea that we have been engaging in transfinite arguments and constructions analogous to arguments by induction and constructions by recursion analogous to familiar arguments by induction and constructions by recursion on the natural numbers.  In this section, we confirm this impression explicitly.

\begin{description}

\item[Transfinite Induction Theorem:]  Let $\phi[x]$ be any formula.   If we can show that for any ordinal $\alpha$, $(\forall \beta <_{\Omega} \alpha: \phi[\beta]) \rightarrow \phi[\alpha]$, it follows that for any ordinal $\alpha$, $\phi[\alpha]$.

\item[Proof:]  Suppose that  for any ordinal $\alpha$, $(\forall \beta <_{\Omega} \alpha: \phi[\beta]) \rightarrow \phi[\alpha]$, and further that for some ordinal $\gamma$, $\neg\phi[\gamma]$.  We can assume that $\gamma$ is the smallest such ordinal, since $\leq_{\Omega}$ is a ``well-ordering", apart from failing to be a set.  But then we have
$(\forall \beta \leq_{\Omega} \gamma: \phi[\beta])$, from which we can deduce $\phi[\gamma]$ by hypothesis, which is a contradiction.

Notice that this theorem formally resembles strong induction on the natural numbers.

\end{description}

We give a form which looks a little bit more like the usual formulation of induction on the natural numbers.

\begin{description}

\item[Definition:]  A {\em limit ordinal\/} is an ordinal which is not zero and which is not a successor.

\item[Transfinite Induction Theorem (three-case form):]  Suppose that $\phi[\alpha]$ is a formula which implies that $\alpha$ is an ordinal, and we can prove
that

\begin{enumerate}

\item  $\phi[0]$

\item $(\forall \alpha:\phi[\alpha] \rightarrow \phi[\alpha+1])$

\item For each limit ordinal $\lambda$, $(\forall \beta<_{\Omega}\lambda:\phi[\beta]) \rightarrow \phi[\lambda]$.

\end{enumerate}

It follows that $\phi[\alpha])$ holds for all ordinal $\alpha$.

\item[Proof:]  This follows easily from the previous form.  The thing is to show that the hypotheses imply that for any ordinal $\alpha$, $(\forall \beta \leq_{\Omega} \alpha: \phi[\beta]) \rightarrow \phi[\alpha]$.  This is asserted for limit ordinals as one of the hypotheses.  It is always true for $\alpha=0$, vacuously.  For successors $\alpha=\alpha'+1$, $(\forall \beta \leq_{\Omega} \alpha'+1: \phi[\beta])$ implies $\phi[\alpha']$ which in turn implies $\phi[\alpha'+1]$ by the hypotheses.

\end{description}

Now we introduce transfinite recursion.  The most general form is analogous to course-of-values recursion on the natural numbers, in which the value of $f(n)$ is defined in terms of the entire restriction of $f$ to $\{m \in {\mathbb N}\mid m < n\}$.

\begin{description}

\item[Transfinite Recursion Theorem:]  Suppose that we can uniformly define an operation $G$ which acts on functions whose domain is an ordinal ($G$ here has to be a class function notation, as it has to act on any function whose domain is an ordinal, and there is no set of all functions whose domains are ordinals).  Then on any ordinal $\alpha$ there is
a function $F$ such that for each $\beta \leq_{\Omega} \alpha$ we have $F(\beta)=G(F \lceil \beta)$, and all such functions $F$ satisfying this condition (but with possibly different ordinal domains) agree on the intersections of their domains.

\item[Transfinite Recursion Theorem (three case form):]   Suppose we have defined a constant $x$ and operations $F$ and $G$ (defined as class function notations).  Then for any limit ordinal $\alpha$ there is a function $H$ with domain $\alpha$ such that

\begin{enumerate}

\item $H(0) = x$

\item $H(\beta+1) = F(H(\beta))$ for all $\beta<\alpha$

\item $H(\lambda) = G(\{H(\beta)\mid \beta<\lambda\})$ for each limit ordinal $\lambda<\alpha$.  Note that $\{H(\beta)\mid \beta<\lambda\}$ exists by Replacement if $H(\beta)$ has successfully been defined for each $\beta<\lambda$.

\end{enumerate}

Further, such functions $H$ with domains distinct ordinals agree on the intersections of their domains.  This theorem could be thought of as a kind of transfinite analogue of the Iteration Theorem.

\end{description}

The reader should recognize the definition of the ranks $V_{\alpha}$ of the cumulative hierarchy as an example of transfinite recursion (presented in the three-case form).  Since the functions we construct in either form of transfinite recursion actually agree on common parts of their domains if they have different ordinals as their domains, we have in effect defined an operation on {\em all\/} ordinals in each case, though this operation cannot be realized by a single set function.

Inductions and recursions restricted to ordinals below a specific ordinal $\gamma$ are readily handled using the general forms, by making the formula $\phi[\alpha]$ true for all $\alpha\geq_{\Omega} \gamma$ in the case of induction, and by defining the operations $G$ (or $F$ and $G$) as taking default values (such as $\emptyset$) at functions whose domain is an ordinal $\geq_{\Omega} \alpha$ in the case of the first form of the transfinite recursion theorem, or at ordinals $\geq_{\Omega} \alpha$ or sets too big to be subsets of the range of $H \lceil \alpha$  in the case of the three-case form.

As an example of transfinite recursion, we present definitions of addition and multiplication of ordinal numbers.

\begin{description}

\item[Definition:]  We define $\alpha+\beta$ for ordinals $\alpha,\beta$:

\begin{enumerate}

\item  $\alpha+0=\alpha$

\item $\alpha+(\beta+1)=(\alpha+\beta)+1$

\item $\alpha+\lambda = \sup(\{\alpha+\beta\mid \beta<\lambda\})$, for $\lambda$ limit.

\end{enumerate}

\item[Definition:]  We define $\alpha\cdot \beta$ for ordinals $\alpha,\beta$:

\begin{enumerate}

\item  $\alpha\cdot 0=0$

\item $\alpha\cdot (\beta+1)=(\alpha\cdot \beta)+\alpha$

\item $\alpha\cdot \lambda = \sup(\{\alpha\cdot \beta\mid \beta<\lambda\})$, for $\lambda$ limit.


\end{enumerate}

\end{description}

\newpage

\subsubsection{Exercises}

\begin{enumerate}

\item  Demonstrate that addition of ordinals is not commutative by demonstrating that $1+\omega \neq \omega+1$.

Demonstrate that multiplication of ordinals is not commutative by demonstrating that $2 \cdot \omega \neq \omega \cdot 2$.

\item  Write a recursive definition of exponentiation of ordinals, modelled on the definition of multiplication above.  Use your definition to compute $2^{\omega}$ and $\omega^2$.   Describe a well-ordering of order type
$\omega^2$.  See if you can describe a well-ordering of type $\omega^{\omega}$ (it is fairly easy to construct one using familiar concepts, but I don't know that anyone will come up with it).

\item (Project!)  Prove the Transfinite Recursion Theorem.

\item (Project!)  Prove the three-case form of the Transfinite Recursion theorem, then show that the three-case form implies the original form.


\end{enumerate}

\newpage

\section{The theory of infinite ordinal and cardinal numbers in untyped set theory}

This section brings together basic results we will need about arithmetic of possibly infinite (transfinite) ordinal and cardinal numbers.  Some of these results depend on the Axiom of Choice, which has a powerful simplifying effect on cardinal arithmetic.  We'll try to indicate where this happens.

\subsection{Transfinite ordinal arithmetic}

It is useful to recall that for any ordinal $\alpha$, $\alpha+1 = \alpha \cup \{\alpha\}$:  the successor operation for ordinal numbers is the same as that for the (von Neumann) natural numbers, which are of course precisely the finite ordinal numbers.

We give set theoretic definitions for ordinal addition and multiplication (which can be shown to be equivalent to the recursive definitions given above).

\begin{description}

\item[Definition:]  If $\leq_1$ is a well-ordering of order type $\alpha$ and $\leq_2$ is a well-ordering of order type $\beta$, then the set
$\leq_1 \oplus \leq_2$ is defined as $$\{\left<\left<c,i\right>,\left<d,j\right>\right> \mid (c \leq_1 d \wedge i=0 \wedge j=0) \vee (c \leq_2 d \wedge i=1 \wedge j=1)$$ $$ \vee (c \in {\tt fld}(\leq_1) \wedge d \in {\tt fld}(\leq_2) \wedge i=0\wedge j=1)\}.$$  It is straightforward to determine
that $\leq_1 \oplus \leq_2$ is a well-ordering and that its order type, which we call $\alpha+\beta$, is completely determined by $\alpha$ and $\beta$.

Notice that this is not the same use of $\oplus$ as the one above in the discussion of addition of natural numbers.

Note that what we are doing in effect is creating disjoint copies of $\leq_1$ and $\leq_2$ and putting the copy of $\leq_1$ before the copy of $\leq_2$.

\item[Definition:]  If $\leq_1$ is a well-ordering of order type $\alpha$ and $\leq_2$ is a well-ordering of order type $\beta$, then the set
$\leq_1 \otimes \leq_2$ is defined as $$\{\left<\left<c,d\right>,\left<e,f\right>\right>\mid c \leq_2 e \, \wedge \, d \in {\tt fld}(\leq_1)\,  \wedge \, f \in {\tt fld}(\leq_1) \, \wedge\,  (c=e \,\rightarrow \, d \leq_1 f)\}.$$   It is straightforward to determine
that $\leq_1 \otimes \leq_2$ is a well-ordering and that its order type, which we call $\alpha\cdot \beta$, is completely determined by $\alpha$ and $\beta$.

Notice that this is not the same use of $\otimes$ as the one above in the discussion of multiplication of natural numbers.

Note that what we are doing in effect is creating disjoint copies of $\leq_1$ indexed by elements of the field of  $\leq_2$ and putting the copies of $\leq_1$ in the order dictated by their indices in the field of $\leq_2$.

\end{description}

Note at once that these operations are not commutative.  $1+ \omega = \omega$ is immediate from this definition, while $\omega+1$ is the successor of $\omega$.  $2 \cdot \omega = \omega$, while $\omega\cdot 2 = \omega+\omega$.  In exercises above, you are expected to use the recursive definitions of the same operations  to verify these facts.  Verifying that the two definitions are equivalent is likely to appear as an exercise below. 

Addition does have identity 0 and multiplication has identity 1 (left and right).  Multiplication has the zero property (left and right).  Addition and multiplication of ordinals are both associative.   $\alpha \cdot (\beta + \gamma) = \alpha\cdot \beta + \alpha\cdot \gamma$, but the left associative property of multiplication over addition does not hold (this is easy:  $(1+1) \cdot \omega \neq 1 \cdot \omega + 1 \cdot \omega$).

We give a

\begin{description}
\item[recursive definition of exponentiation of ordinal numbers:]
Of course a similar definition of exponentiation on natural numbers
could be given (and is actually in effect included here).  There is a
set theoretical definition of exponentiation of ordinals as well, but
it is a bit technical.

\begin{enumerate}


\item $\alpha^0 = 1$

\item $\alpha^{\beta+1} = \alpha^\beta \cdot \alpha$

\item $\alpha^{\lambda} = \sup(\{\alpha^{\beta}\mid \beta<\lambda\})$ for $\lambda$ limit.

\end{enumerate}

\end{description}

\newpage


\subsection{Transfinite cardinal arithmetic}

In this section we will discuss basic properties of order, addition, multiplication and exponentiation of transfinite cardinals assuming the Axiom of Choice.
We may have some things to say about what can be proved without Choice.

The Cantor-Schr\"oder-Bernstein theorem proved above establishes that the order $\leq$ already defined on cardinal numbers is a partial order.

We briefly discuss the official definition of cardinals in a slightly different way.

\begin{description}

\item[Definition:]  An {\em initial ordinal\/} is an ordinal $\alpha$ such that for every $\beta<\alpha$, $|\beta|<|\alpha|$ (equivalently, for no $\beta<\alpha$ do we have
$\beta \sim \alpha$).  For any set $A$ define $|A|$ as the unique initial ordinal such that $A \sim \alpha$.  That there is such an ordinal we show as follows:  by the Well-Ordering Theorem, there is a well-ordering $\leq_A$ with field $A$;  for each such well-ordering there is an ordinal $\alpha={\tt ot}(\leq_A)$ such that $(\subseteq \lceil \,\alpha) \approx \,<_A$;  isomorphic relations have fields of the same size (because an isomorphism between two relations is a bijection between their fields, among other conditions), so $\alpha \sim A$;  the smallest ordinal $\alpha$ isomorphic to a well-ordering with field $A$ will be an initial ordinal equinumerous with $A$ (if there were any ordinal $\beta$ less than this $\alpha$ with $\alpha \sim \beta$, it would have $|\beta| = |A|$ whence one could define a well-ordering of $A$ of type $\beta$ contrary to choice of $\alpha$).

Note that we will use different notation for one and the same object when it is considered as an ordinal and when it is considered as a cardinal.  For example, $\omega$, the first countable ordinal, is the same object as $\aleph_0$, the cardinality of all countably infinite sets (and for that matter both are the same as $\mathbb N$, the set of natural numbers),
and similarly, $\omega_1$, the first uncountable ordinal, and $\aleph_1$, the first uncountable cardinal, are the same object.  Note though that if we used different implementations of cardinals and/or ordinals, these notations would have different referents, though theorems about cardinals and ordinals as such would tend to remain the same.

\item[Theorem:]  The natural order on cardinals is a linear order.

\item[Proof:]  What we need to show is that for any two sets $A$ and $B$ we can construct either an injection from $A$ to $B$ or an injection from $B$ to $A$.  The well-ordering theorem allows us to select a well-ordering $\leq_A$ of $A$ and a well-ordering $\leq_B$ of $B$.  Now we know that there is either an isomorphism between $<_A$ and $<_B$, or an isomorphism between $<_A$ and some $(<_B)_b$ or an isomorphism between $<_B$ and
some $(<_A)_a$.  Now recall that an isomorphism is a bijection between the fields of the relations involved.  Thus we have either a bijection from $A$ to $B$, or a bijection from $A$ to a subset of $B$, or a bijection from $B$ to a subset of $A$, whence we either have an injection from $A$ into $B$
or an injection from $B$ into $A$ as required, so either $|A| \leq |B|$ or $|B| \leq |A|$.

Notice that the axiom of choice was used here.  We will prove in the not too distant future that the assertion that the natural order on cardinals is a linear order implies the axiom of choice:  these two assertions are exactly equivalent.  This is perhaps somewhat surprising, since linear ordering of size seems a very natural assumption about sizes of sets.

\item[Theorem:]  The natural order on cardinals is a well-ordering.

\item[Proof:] Let $C$ be a set of cardinals.  Our aim is to show that
$C$ has a smallest element in the natural order.  Let $\leq$ be a
well-ordering of the power set of the union of $C$ (which is an ordinal  larger as a set than any of the cardinals in $C$).
Consider the set of all well-orderings of elements of $C$.  Every well-ordering in this set will be isomorphic to some segment restriction of
$\leq$.  Consider the set of
all $x\in {\tt fld}(\leq)$ such that $\leq_x$ is isomorphic to some well-ordering of some
element of $C$: there must be a $\leq$-smallest element $m$
of this set, and the order type of $(\leq)_m$ will be the smallest element of $C$ (it has to be the order type of a well-ordering of some element of $C$, but moreover the order type of a shortest such well-ordering, which will be the element of $C$ itself, since the element of $C$ is a cardinal).

\item[Theorem:]  There is a surjection from $A$ onto $B$ iff
$|B| \leq |A|$ (and $B$ is nonempty if $A$ is).

\item[Proof:] If there is an injection $f$ from $B$ to $A$, then we
can define a surjection from $A$ to $B$ as follows: choose $b \in B$;
map each element of $A$ to $f^{-1}(a)$ if this exists and to $b$
otherwise.  This will be a surjection.  If $B$ is empty we cannot
choose $b$, but in this case $A$ is empty and there is obviously a
surjection.

If there is a surjection $f$ from $A$ onto $B$, there is a partition
of $A$ consisting of all the sets $f``\{a\}$ for $a \in A$.  Let $C$
be a choice set for this partition.  Map each element $b$ of $B$ to
the unique element of $C\subseteq A$ which is sent to $b$ by $f$.
This map is obviously an injection.

\item[Definition:] In set theory {\em without\/}
Choice, we define $$|A| \leq^* |B|$$ as holding iff there is a surjection
from $B$ onto $A$.  In the light of the previous Theorem, there is no
need for this notation if we assume Choice.



\item[Definition (repeated from above):]   We define $\aleph_0$ as
$|\mathbb N|$.  Elements of $\aleph_0$ are called  {\em countably infinite
sets\/}, or simply {\em countable sets\/}.

\item[Theorem:] $\aleph_0+1 = \aleph_0+\aleph_0 = \aleph_0 \cdot
\aleph_0 = \aleph_0$.  It is straightforward to define a bijection
between $\mathbb N$ and $\mathbb N\times \mathbb N$.  The bijections
between the naturals and the even and odd numbers witness the second
statement.  The successor map witnesses the first statement.


\item[Theorem:]  Every infinite set has a countable subset.

\item[Proof:] Let $A$ be an infinite set.  The inclusion order on the
collection of all bijections from initial segments of $\mathbb N$ to
$A$ satisfies the conditions of Zorn's Lemma and so has a maximal
element.  If the maximal element had domain a proper initial segment
of $\mathbb N$, then the set would be finite.  So the maximal element
is a bijection from $\mathbb N$ to a subset of $A$.

\item[Theorem:]  For every infinite cardinal $\kappa$, $\kappa+1 = \kappa$.

\item[Proof:] Let $A$ be an infinite set.  The inclusion order on the
set of all bijections from $B$ to $B \cup \{x\}$, where $B \cup \{x\}
\subseteq A$ and $x \not\in B$, satisfies the conditions of Zorn's
Lemma and so has a maximal element.  It is nonempty because $A$ has a
countable subset.  If the maximal element is a map from $B$ to $B \cup
\{x\}$ and there is $y \in A-(B\cup \{x\})$, then affixing
$\left<y,y\right>$ to the map shows that the supposed maximal element
was not maximal.

An easier proof of this uses the previous theorem.  $\kappa = \lambda+\aleph_0$ for some $\lambda$, since a set of size $\kappa$ has a countable subset.  It follows
that $\kappa=\lambda+\aleph_0=\lambda+(\aleph_0+1)=(\lambda+\aleph_0)+1=\kappa+1$.

\item[Corollary:] If $n$ is finite and $\kappa$ is an infinite
cardinal then $\kappa+n = \kappa$.

\item[Theorem:] For every infinite cardinal $\kappa$,
$\kappa+\kappa=\kappa$.

\item[Proof:] Let $A$ be an infinite set.  The set of
pairs of bijections $f$ and $g$ with ${\tt dom}(f)={\tt dom}(g) = {\tt rng}(f) \cup {\tt rng}(g) \subseteq A$ and
${\tt rng}(f) \cap {\tt rng}(g) = \emptyset$ can be partially ordered by componentwise
inclusion: $$(f,g) \leq (f',g') \leftrightarrow f\subseteq f' \wedge g \subseteq g'.$$  This partial order satisfies the hypotheses of Zorn's Lemma (verifying this is left as an exercise).  It is nonempty
because $A$ has a countable subset. Suppose that a maximal such pair
of bijections $f,g$ in the componentwise inclusion order has been constructed. Let $B$ be the common domain of $f$ and $g$. If there is no countably infinite
subset in $A-B$, then $A-B$ is finite and $|B|=|A|$ by a previous
result and the result is proved: otherwise take a countable
subset of $A-B$ and extend the supposedly maximal pair of  maps to a larger one.

\item[Alternative Proof:]  We prove by transfinite induction that for any ordinal $\alpha = \lambda+n$, where $\lambda$ is 0 or limit and $n$ is finite (every ordinal can be written in this way in one and only one way -- exercise), we have $2 \cdot\alpha = \lambda+2n$.  Note that the arithmetic operations here are operations on {\em ordinals\/}.  We prove this using three-case induction (in which $\lambda$ always stands for a limit ordinal and $n$ for a finite ordinal):

\begin{description}

\item[zero:]   $0=0+0$.  0=0+0;  $2 \cdot 0 = 0 = 0+2\cdot 0$.

\item[successor:]  Let $\alpha=\lambda +n$.  Suppose that $2\cdot\alpha=2 \cdot(\lambda+n) = \lambda+2\cdot n$.  Then $2\cdot (\alpha+1)=2 \cdot ((\lambda+n)+1) [= 2 \cdot ((\lambda+(n+1))] = 2\cdot(\lambda +n) +2 = (\lambda + 2\cdot n)+2 = \lambda+2\cdot (n+1)$, verifying the claim for $\alpha+1=\lambda+(n+1)$.

\item[limit:]  Suppose that $\mu$ is limit and for every $\beta=\lambda+n<\mu$ we have $2\cdot(\lambda+n) = \lambda+2\cdot n$.  Then $2 \cdot \mu = {\tt sup}_{\lambda+n <\mu}(\lambda+2\cdot n)$.   This supremum is less than or equal to $\mu$ because for any $\lambda+n <\mu$ we have $\lambda+2\cdot n<\mu$ as well.  The supremum is greater than or equal to $\mu$ because any $\alpha<\mu$ is of the form $\lambda+n<\mu$, so $\alpha+1$ is of the form $\lambda+(n+1)<\mu$, and is strictly less than $2 \cdot (\alpha+1)=
\lambda+2n+2$, which is less than or equal to the supremum in question.  So $2\cdot \mu = 2 \cdot (\mu + 0) = \mu = \mu+2\cdot 0$ as desired.

So we have for any cardinal $\kappa$ that $2 \cdot \kappa = \kappa$ by the Lemma, the finite part being zero.  It is very important to note that this is ordinal multiplication.
Now $2 \cdot \kappa$ is the order type of an order $\leq_1 \otimes \leq_2$, where the cardinality of the field of $\leq_1$ is 2 and the cardinality of the field of $\leq_2$ is
$\kappa$.  The field of this relation is ${\tt fld}(\leq_1) \times {\tt fld}(\leq_2)$ which is of the form $\{x\} \times {\tt fld}(\leq_2) \cup \{y\} \times {\tt fld}(\leq_2)$ where
$x$ and $y$ are the two elements of the field of $\leq_1$.  This set is of cardinality $\kappa+\kappa$ (here we mean cardinal addition).  The field of a well-ordering of type $\kappa$ is of course of size $\kappa$.  Isomorphic well-orderings have fields of the same sizes, so the ordinal fact $2 \cdot \kappa = \kappa$ implies the cardinal fact $\kappa+\kappa=\kappa$.

\end{description}

\item[Corollary:]  If $\lambda\leq \kappa$ and $\kappa$ is an infinite cardinal
then $\kappa+\lambda=\kappa$:  note that $\kappa \leq \kappa+\lambda \leq \kappa+\kappa=\kappa$.



\item[Theorem:] For every infinite cardinal $\kappa$,
$\kappa\cdot\kappa = \kappa$.

\item[Proof:] Let $A$ be an infinite set.  The inclusion order on
bijections from $B\times B$ to $B$, where $B \subseteq A$, satisfies
the conditions of Zorn's Lemma.  It is nonempty because $A$ has a
countable subset.  Now consider a maximal function in this order,
mapping $B \times B$ to $B$.  If $A-B$ contains no subset as large as
$B$, then $|B| = |A|$ by the previous result and the result is proved.
Otherwise, choose $B' \subseteq A-B$ with $|B'| = |B|$.  It is then
easy to see from assumptions about $B$ and $B'$ and the previous
result that the map from $B \times B$ to $B$ can be extended to a
bijection from $(B \cup B') \times (B \cup B')$ to $B \cup B'$,
contradicting the supposed maximality of the bijection.

\item[Corollary:]  If $\lambda\leq \kappa$ and $\kappa$ is an infinite cardinal
then $\kappa\cdot\lambda=\kappa$.



\end{description}

The arithmetic of addition and multiplication of infinite cardinals is
remarkably simple.  This simplicity depends strongly on the presence
of Choice.

We now introduce exponentiation of cardinals.

\begin{description}

\item[Definition:]  $B^A$ is defined as the set of functions from the set $A$ to the set $B$.  $|A|^{|B|}$ is defined as
$|A^B|$.  It is a defect of this traditional notation that if $\kappa$ and $\lambda$ are considered as sets, the meaning of
$\kappa^{\lambda}$ (the set of functions from $\lambda$ to $\kappa$) is the same size as but not the same object as the referent of
$\kappa^{\lambda}$ when $\kappa$ and $\lambda$ are considered as cardinals (the cardinality of the set just mentioned).

\item[Observation:]  $2^{|A|} = |{\cal P}(A)|$, because there is a one to one correspondence between subsets $B \subseteq A$ and
characteristic functions $\chi_B:A \rightarrow \{0,1\}$ such that $\chi_B(a) = 1$ iff $a \in B$ (every element of $2^A$ is a $\chi_B$).  Thus we have $2^{|A|}>|A|$ by the theorem of Cantor already proved.

\end{description}

We present some rules of exponentiation which should look familiar (though the generalizations require justification).  These are combinatorial results which do not depend on Choice:  writing out explicitly how to get bijections witnessing these equations of cardinality might be good exercises.
\begin{description}
\item[Rules of exponentiation:]

\begin{enumerate}

\item[]

\item $\kappa^1 = \kappa; \kappa^0 = 1$; $1^{\kappa}=1; 0^{\kappa}=1$ if $\kappa=0$, 0 otherwise.

\item  $\kappa^{\lambda + \mu} = \kappa^{\lambda} \cdot \kappa^{\mu}$

\item  $(\kappa^{\lambda})^{\mu} = \kappa^{\lambda \cdot \mu}$

\item $(\kappa\cdot \lambda)^{\mu} = \kappa^{\mu} \cdot \lambda^{\mu}$

\end{enumerate}

\end{description}

We prove a sample result combining our various tools.

\begin{description}

\item[Theorem (AC -- do you see where?):]  For any infinite cardinal $\kappa$, $\kappa^{\kappa}=2^{\kappa}$.

\item[Proof:]  Let $|A|=\kappa$.  $\kappa^{\kappa}$ is the cardinality of $A^A$, which is a subset of ${\cal P}(A^2)$, which is the same size as
$2^{A^2}$, whose cardinality is $2^{\kappa^2} = 2^{\kappa \cdot \kappa} = 2^{\kappa}$, the size of the collection of subsets of $A$ or functions from $A$ to 2.  On the other hand there are clearly no more functions from $A$ to 2 than there are from $A$ to $A$ (an injection can easily be presented).  So $$\kappa^{\kappa} \leq 2^{\kappa^2} = 2^{\kappa \cdot \kappa} = 2^{\kappa} \leq \kappa^{\kappa}.$$

\end{description}

We now present some information about the structure of the entire system of cardinals.   We know that for every cardinal $\kappa$, the cardinal $2^{\kappa}>\kappa$.  It follows that there is a smallest cardinal greater than $\kappa$.

\begin{description}

\item[Definition:]  For any infinite cardinal $\kappa$, we define $\kappa+$ as the smallest cardinal greater than $\kappa$.

\end{description}

The cardinal $\kappa+$ coincides with the notion defined in the next definition, if the axiom of choice holds.  Note that if we admit the possibility that the axiom of choice might fail, we need to use an alternative definition of cardinal number, such as the Scott definition, under which $|A|$ is the set of the form $\{B \sim A \mid B \in V_{\alpha}\}$ for the smallest $\alpha$ for which this set is nonempty.

\begin{description}

\item[Definition:]  For any infinite cardinal $\kappa$ and set $A$ of size $\kappa$, we define $\aleph(\kappa)$ as the supremum of the set of order types of well-orderings of subsets of $A$.  This is known as the Hartogs aleph function.  It is clear that it does not depend on the choice of $A$.  

\item[Lemma:]  If $\alpha$ is the order type of a well-ordering of a subset of an infinite set $A$, so is $\alpha+1$.

\item[Proof:]  $A$ has a countable subset $B$;  $B \setminus \{x\}$ is countable, so $A \setminus \{x\} = (A \setminus B) \cup (B \setminus \{x\}) \sim (A \setminus B) \cup B = A$.
So $A \setminus \{x\}$ also has a well-ordering of order type $\alpha$, to which $x$ can be appended to give a well-ordering of order type $\alpha+1$.

\item[Theorem (not using AC):]  It is not the case that $|\aleph(\kappa)| \leq \kappa$.  It follows immediately that $\aleph(\kappa)$ is an initial ordinal.  But note that for the moment this does not mean that we identify $|\aleph(\kappa)|$ with $\aleph(\kappa)$, as we may be using a different cardinal implementation.

\item[Proof of theorem:]  Let $A$ be of cardinality $\kappa$.  If $|\aleph(\kappa)| \leq \kappa$, then $\aleph(\kappa)$ is the order type of the field of a well-ordering of a subset of $A$.  But then
$\aleph(\kappa)+1$ is also the order type of a well-ordering of a subset of $A$, and this contradicts the definition of $\aleph(\kappa)$:  $\aleph(\kappa)$ must be the supremum of all order types of well-orderings of subsets of $A$.

\item[Theorem (using AC):]  For any infinite cardinal $\kappa$, $\kappa+ = \,\aleph(\kappa)$.

\item[Proof of theorem:]  Let $A$ be of cardinality $\kappa$.  By AC, $A$ has a well-ordering, so $\aleph(\kappa) \geq \kappa$.  By the previous theorem and the linearity of the natural order on cardinals, $\aleph(\kappa)>\kappa$.  Now any ordinal strictly less than $\aleph(\kappa)$ is the order type of a well-ordering of a subset of $A$, so cannot be
greater than $\kappa$, whence $\aleph(\kappa)=\kappa+$.

\item[Theorem:]  The Axiom of Choice is true iff the natural order on cardinal numbers is a linear order.

\item[Proof:]  We have already shown that if the axiom of choice is true, the natural order on cardinal numbers is a linear order.

Suppose that the natural number on cardinal numbers is a linear order.  Let $A$ be an infinite set of cardinality $\kappa$.  We must have either $\kappa \leq \aleph(\kappa)$
or $\aleph(\kappa)\leq \kappa$.  The latter is impossible by an earlier theorem.  So we have $\kappa \leq \aleph(\kappa)$, and any subset of a well-orderable set is well-orderable:  there is a well-ordering on $A$ because it is the same size as a subset of the set of order types of well-orderings of subsets of $A$, which is evidently well-orderable.

\end{description}

We now present some information about the structure of all cardinals (assuming Choice):

\begin{description}

\item[Definition:]  The notation $\aleph_0$ is already defined.  We define $\aleph_{\alpha+1}$ as $\aleph_{\alpha}+$.  We define $\alpha_{\lambda}$, for $\lambda$ limit,
as ${\tt sup}(\{\aleph_{\beta} \mid \beta<\lambda\})$.

\item[Observation:]  Under the axiom of choice, all cardinals are of the form $\aleph_{\alpha}$.

\item[Definition:]  The notation $\beth_0$ is defined as $\aleph_0$.  We define $\beth_{\alpha+1}$ as $2^{\beth_{\alpha}}$.  We define $\beth_{\lambda}$, for $\lambda$ limit,
as ${\tt sup}(\{\beth_{\beta} \mid \beta<\lambda\})$.

\item[Observation:]  Independently of Choice, the cardinals $\beth_{\alpha}$ are exactly the cardinalities of the ranks of the cumulative hierarchy with infinite index.
$\beth_{\alpha}=|V_{\omega+\alpha}|$.  For $\beta \geq \omega^2$, $\beth_{\beta} = |V_{\beta}|$.  The Generalized Continuum Hypothesis is equivalent to the assertion
that $\beth_{\alpha} = \aleph_{\alpha}$ for each $\alpha$.

\end{description}


We introduce a further idea bearing on structure of cardinals.

\begin{description}

\item[Definition:]  The {\em cofinality\/} of a partial order $\leq$ is the infimum of the order types of unbounded well-ordered chains in $\leq$.  Because the natural order on ordinals is a well-ordering,
there will be a well-ordered chain in $\leq$ which is unbounded and has the cofinality of $\leq$ as its order type.

\item[Definition:]  The cofinality of a cardinal $\kappa$, written ${\tt cf}(\kappa)$, is the cofinality of the inclusion order on $\kappa$ itself. 

\item[Observation:]   The cofinality of ${\tt cf}(\kappa)$ is
${\tt cf}(\kappa)$;  this is easily seen, because an unbounded well-ordered chain of minimal length in the inclusion order on ${\tt cf}(\kappa)$ will clearly be isomorphic to an unbounded well-ordered chain in the inclusion order on $\kappa$ itself:  if there were such a well-ordered chain of order type $<{\tt cf}(\kappa)$ in the inclusion order on ${\tt cf}(\kappa)$, there would be such a chain in the inclusion order on $\kappa$ itself, contradicting the definition of ${\tt cf}(\kappa)$.

\item[Observation:]  ${\tt cf}(\kappa)$ is an initial ordinal, and thus a cardinal.   Suppose that $|{\tt cf}(\kappa)|<{\tt cf}(\kappa)$.  Choose a bijection $f$ from $|{\tt cf}(\kappa)|$ to ${\tt cf}(\kappa)$.  For each $\alpha<|{\tt cf}(\kappa)|$, define $a_{\alpha}$ as ${\tt sup}_{\beta<\alpha}({\tt max}(a_{\beta}+1,f(\beta)+1))$.  The objects $a_{\alpha}$ are strictly increasing as $\alpha$ increases, and make up an unbounded [because any element $f(\alpha)$ of ${\tt cf}(\kappa)$ is less than or equal to $a_{\alpha}$] well-ordered chain in the inclusion order on ${\tt cf}(\kappa)$ of order type $|{\tt cf}(\kappa)|$, whence we cannot have $|{\tt cf}(\kappa)|<{\tt cf}(\kappa)$.  It might seem possible that
${\tt sup}_{\beta<\alpha}({\tt max}(a_{\beta}+1,f(\beta)+1))$ might be ${\tt cf}(\kappa)$ for some $\alpha<|{\tt cf}(\kappa)|$:  but this is ruled out because we would then have an unbounded well-ordered chain of order type $\alpha<{\tt cf}(\kappa)$ in ${\tt cf}(\kappa)$.

\item[Observation on implementation-dependence:]  This discussion strongly depends on the use of the von Neumann ordinals and the use of initial ordinals to implement cardinals.
An implementation-independent presentation which would work for different implementations of ordinals and cardinals is possible, and its character might be divined by taking a look at definitions of this notion in chapter 2.

\item[Definition:]  A cardinal $\kappa$ is {\em regular} iff ${\tt cf}(\kappa)=\kappa$.  Note that cofinalities of cardinals are regular cardinals.  A cardinal which is not regular is said to be {\em singular\/}.

\item[Observation:]  $\aleph_0$ is regular.

\item[Theorem (AC!):]  For any infinite cardinal $\kappa$, $\kappa+$ is regular.

\item[Proof:]  Suppose otherwise.  Then there would be an unbounded chain of order type $\lambda \leq \kappa$ in the inclusion order on $\kappa+$.  Let $C$ be this chain:
$C_{\alpha}$ is the element of this chain, if any, such that the order type of the segment in $C$ determined by $C_{\alpha}$ is $\alpha$.  We can assume that $C_0 \neq 0$.  For each $C_{\alpha}$ choose
a surjection $f_{\alpha}$ from $\kappa$ onto $C_{\alpha}$:  this is possible (though only with the use of the Axiom of Choice)  because any initial segment of an order of type $\kappa+$ is of cardinality $\leq \kappa$.  We then define
a surjection from $\kappa \times \lambda$ onto $\kappa+$:  map $\left<\alpha,\beta\right>$ to $f_{\beta}(\alpha)$.  This means that $\kappa+\leq |\kappa \times \lambda| = \kappa$, which is a contradiction.



\end{description}

Note that this theorem applies to the case $\kappa=\aleph_1$, or indeed any $\aleph_n$ for $n$ finite.

\begin{description}

\item[Definition:]  A cardinal $\lambda$  is said to be {\em strong limit\/} iff for each cardinal $\kappa<\lambda$ we have $2^{\kappa}<\lambda$.  $\beth_{\omega}$ is the smallest uncountable strong limit cardinal.  Note that $\beth_{\omega}$ is singular, having cofinality $\omega$.

\item[Definition:]  A regular strong limit cardinal is said to be {\em inaccessible\/}.   We cannot give an example of one of these, as our current axioms cannot prove that there is one.

\end{description}

As our final major point in this section, we prove K\"onig's Theorem and explore some of its consequences.

\begin{description}

\item[Definition:]  We define infinite sums and products of cardinals.  Let $F$ be a function from an index set $I$ whose range is a set of cardinals.  We define $\Sigma_{i \in I} F(i)$ as
$|\{\left<x,i\right> \mid i \in I \wedge x \in F(i)\}|$, and we define $\Pi_{i \in I} F(i)$ as $|\{f \mid {\tt dom}(f) = I \wedge (\forall x \in I:f(x) \in F(i))\}|$.  

It is important to note that the Axiom of Choice is required to establish the more general assertion that if $A$ is a function from the index set $I$ to sets such that $|A(i)| = F(i)$, that  $\Sigma_{i \in I} F(i)$ is 
$|\{\left<x,i\right> \mid i \in I \wedge x \in A(i)\}|$ and $\Pi_{i \in I} F(i)$ as $|\{f \mid {\tt dom}(f) = I \wedge (\forall x \in I:f(x) \in A(i))\}|$.   The difficulty is that one needs not only bijections witnessing each assertion $A(i) \sim F(i)$, but also a uniform way to choose one such bijection for each $i$, in order to establish that the cardinalities of the last two expressions do not depend on the choice of the map $A$. 

Note also that this definition would have to be rephrased if we were using a different definition of cardinality, under which the cardinality of the set was not itself a set of the same cardinality.  Under a different definition of cardinality, it would be likely to be necessary to use the more general form which depends on the Axiom of Choice.

\item[K\"onig's Theorem (depends on AC):]  Let $F$ and $G$ be functions with the same nonempty domain $I$ whose ranges are sets of cardinals.
Suppose further that $0<F(i) < G(i)$ for each $i \in I$.  It follows that $\Sigma_{i\in I}F(i) < \Pi_{i \in I} G(i)$.

\item[Proof of K\"onig's Theorem:]  Suppose on the contrary that $\Sigma_{i\in I}F(i) \geq \Pi_{i \in I} G(i)$.  It follows that there is a surjection
$H$ from $|\{\left<x,i\right> \mid i \in I \wedge x \in F(i)\}|$ onto $|\{f \mid {\tt dom}(f) = I \wedge (\forall x \in I:f(x) \in G(i))\}|$.  Each set
$A_i$ defined as $\{\left<x,i\right> \mid x \in F(i)\}$ is of cardinality $F(i)$ and the set $\pi_i``H``A_i\subseteq G(i)$ [where $\pi_i(f)$ is defined as $f(i)$] cannot cover $G(i)$ because $F(i) < G(i)$.  Choose an element $k(i) \in G(i) - \pi_i``H``A_i$ for each $i \in I$.  The function $k$ belongs to $$|\{f \mid {\tt dom}(f) = I \wedge (\forall x \in I:f(x) \in G(i))\}|,$$ and cannot belong to the range of $H$ (since by construction no element of any $A_i$ can be mapped to $k$ by $H$, and the union of the $A_i$'s is the entire domain of $H$).

\item[Consequences of K\"onig's Theorem:]  Cantor's theorem is a consequence:  $\kappa = \Sigma_{i\in \kappa} 1 < \Pi_{i \in I} 2 = 2^{\kappa}$.  Of course, Cantor's theorem can be proved without choice, so this is not an optimal proof of this result.

A very interesting corollary is that $\kappa^{{\tt cf}(\kappa)} > \kappa$ for any $\kappa$.  Let $\lambda = {\tt cf}(\kappa)$ and
let $\alpha_i<\kappa$ for $i \in \lambda$ be an unbounded strictly increasing sequence in $\kappa$.  Then $\kappa = \Sigma_{i \in \lambda}\alpha_i < \Pi_{i \in \lambda} \kappa = \kappa^{\lambda} = \kappa^{{\tt cf}(\kappa)}$.

Now we can show that the cofinality of $2^{\aleph_0}$, the cardinality of the reals, is uncountable.  If the cofinality of $2^{\aleph_0}$ is countable,
then the preceding result establishes that $(2^{\aleph_0})^{\aleph_0}>2^{\aleph_0}$.  But $(2^{\aleph_0})^{\aleph_0} = 2^{\aleph_0 \cdot \aleph_0} = 2^{\aleph_0}$, so this is impossible.  It turns out that this is basically the only provable limitation on which cardinal $2^{\aleph_0}$ can be.

\end{description}

\newpage

\subsection{Exercises}

\begin{enumerate}


\item  Let $\leq_{\mathbb N}$ be the usual order on the natural numbers.  Use the operations $\oplus$ and $\otimes$ to describe
orders of type $\omega+\omega$ and $\omega\cdot \omega = \omega^2$;  draw illustrations of these orders sufficient to convince me that you know what they look like.

\item  Prove that the two definitions of ordinal addition that you have been given (the one by transfinite recursion and the one using $\oplus$) actually agree.  This should be a proof by transfinite induction.

\item  Prove that a chain of injective functions in the inclusion order has an injective function as the union of its range (recall that for us a chain is actually a linear order;  its range is the set which carries the linear order).

\item Verify that the componentwise inclusion order on pairs of bijections which appears in the first proof of $\kappa+\kappa=\kappa$ satisfies the hypotheses of Zorn's Lemma.



\item Verify rules 2 and 4 of exponentiation.  Rule 3 is especially tricky, and you will receive additional credit (and praise) if you can prove it.
Remember that sets $B^A$ between which you are building bijections are sets of functions, and your bijections need to involve clever definitions of functions taking functions to other functions.

\item Explain why the Well-Ordering Theorem implies the axiom of choice (in the presence of the other axioms).

\item  Show that for any cardinal $\kappa$, if $\lambda\leq \kappa$ and $\{\alpha_{\beta}\}_{\beta<\lambda}$ is an increasing unbounded sequence of length $\lambda$ in $\kappa$, then $\Sigma_{i \in \lambda}\alpha_i= \kappa$.  Hint:  you can map the standard set whose cardinality is  $\Sigma_{i \in \lambda}\alpha_i= \kappa$ into $\kappa \times \lambda$ (this is straightforward), which is a set of size $\kappa$ (why?);  the trick then is to see how to map $\kappa$ injectively into this set, which will complete the proof.  Hint:  consider the sets $\alpha_{\beta+1} - \alpha_{\beta}$.

\item Prove that $(\beth_{\omega})^{\aleph_0} = 2^{\beth_{\omega}}$.  Hint:  there is a standard set of size $\beth_{\omega}$ that you can think of, namely, the union of the iterated power sets of the natural numbers.  Consider how you can use a sequence of elements of this set to approximate an arbitrary subset of this set.

\item (Project, entirely optional)  A set theoretical definition of ordinal exponentiation $\beta^{\alpha}$ can be given.  One expects this to have something to do with functions from a set with an order of type $\alpha$ on it to a set with an order of type $\beta$ on it.  In fact, this is true with a subtle modification.
If ${\tt ot}(\leq_1) = \alpha$ and ${\tt ot}(\leq_2)= \beta$, the order $\leq_2^{\leq_1}$ has domain the set of functions from ${\tt fld}(\leq_1)$
to ${\tt fld}(\leq_2)$ which are equal to 0 at all but finitely many elements of ${\tt fld}(\leq_1)$.  An order statement $f \, \leq_2^{\leq_1}\,g$ is evaluated by considering the largest value $x$ in ${\tt fld}(\leq_1)$ at which the functions $f$ and $g$ disagree (there is a largest such value because $f$ and $g$ have the value 0 except at a finite number of inputs), and returning the truth value of $f(x) \, \leq_2 \, g(x)$.

The project is to verify that this definition is equivalent to the recursive one given above.  I imagine it is rather difficult.

\end{enumerate}


\newpage

\section{Logically regimented set constructions and the definition of $L$}

The reader should be aware of the correlation between propositional logic (the logic of ``and", ``or", and ``not") and the Boolean algebra of sets (the logic of intersection, union and complement (relative to a fixed universe in the context of untyped set theory).   In this section, we extend this idea to include the operations of the logic of quantifiers.  We will use this in our development of G\"odel's constructible universe.

\begin{description}

\item[Definition:]  A ``predicate set" over a domain $D$ is a collection of elements of $D^{\infty}$ (defined as the set of  natural-number-indexed sequences of elements of $D$ which are eventually constant) which has a {\em predicate order\/} in a sense we now define:  a subset $X$ of $D^{\infty}$ has predicate order $n\in \mathbb N$ iff for each $f \in X$, for every $g \in D^{\infty}$, if $f \lceil n = g \lceil n$, then $g \in X$.

The intention is that a predicate set is in all cases a set of sequences of elements of the domain indexed by all natural numbers, but if the predicate takes $n$ arguments, then whether a sequence belongs to the associated predicate set depends only on the first $n$ terms of the sequence (those with indices $<n$, since indexing starts at 0).  Notice that a predicate set with predicate order $n$ is also of predicate order $m$ for all $m \geq n$.

\item[Basic constructions of predicate sets:]   If $P$ and $Q$ are predicate sets of predicate order $n$, then $P^c = D^{\infty} \setminus P$ and $P \cap Q$ are predicate sets of predicate order $n$.  This gives us support for the propositional operations of conjunction and negation, and so for all the operations of propositional logic.

For any $f \in D^{\mathbb N}$, define $f_{i,j}$ as $(f \lceil ({\mathbb N}-\{i,j\}) \cup \{\left<i,f(j)\right>,\left<j,f(i)\right>\}$.  For any predicate set $P$ of order $n$,
$P_{i,j} = \{f_{i,j} \mid f \in P\}$ is a predicate set of predicate order the maximum of $n,i,j$.

For any predicate set $P$, define $\exists P$ as $\{\left<0,d\right> \cup f \lceil {\mathbb N}^+\mid f \in P \wedge d \in D\}$.  Define $\exists_i P$ as $(\exists P_{0,i})_{0,i}$.  Define $\forall P$
as $(\exists P^c)^c$ and $\forall _iP$
as $(\exists_i P^c)^c$.  These operations implement quantification on predicate expressions.  If $P$ is a predicate set of predicate order $n$, each of these sets is also a predicate set of predicate order $n$.

For any unary predicate $P$, define $[P]$ as $\{f \in D^{\infty} \mid P(f(0))\}$.  This is clearly a predicate set of predicate order 1.  For any logical relation $R$, define $[R]$ as
$\{f \in D^{\infty} \mid f(0) \, R \, f(1)\}$.  This is clearly a predicate set of predicate order 2.

\item[Representations of propositions by predicate sets:]

\begin{description}

\item

\item[atomic formulas:]  Define $[Px_i]$ as $[P]_{0,i}$.  Define $[x_i \, R \, x_j]$ as $([R]_{0,i})_{1,j}$, when $i \neq j$ and $i \neq 1$ and $j \neq 0$..  If $i \neq j$ and $i=1$ and $j \neq 0$, define it as $([R]_{0,i})_{0,j}$.  If $i \neq j$ and $i \neq 1$ and $j=0$, define it as $([R]_{0,i})_{ij}$.  If $i=1$ and $j=0$ define it as $[R]_{01}$.  If $i=j$, define it as
$\exists_{i+1}([x_i \, R\, x_{i+1} \wedge x_i = x_{i+1}])$.]

\item[propositional operations:]  Define $[\neg \phi]$ as $[\phi]^c$.  Define $[\phi \wedge \psi]$ as $[\phi] \cap [\psi]$.   For any other propositional connectives, generate such definitions by redefining the connectives in terms of conjunction and negation.

\item[quantifiers:]  Define $[(\forall x_i \in D:\phi)]$ as $\forall_i([\phi])$.  Define $[(\exists x_i \in D:\phi)]$ as $\exists_i([\phi])$.

\end{description}

\end{description}

The point is that for any proposition $\phi$, this procedure will create the set $[\phi]$ of all sequences $f$ such that if each variable $x_i$ is assigned the value $f(i)$, the proposition $\phi$ is assigned the truth value ``true".  This could be modified to allow different domains for different variables, by defining our universe as the collection of functions $f$
such that for each $i \in {\mathbb N}$, $f(i) \in D(i)$, where $D$ is a function from natural numbers intended to take each $i$ to the set to which the variable $x_i$ is to be bounded.  This would require care in the use of the operators $\cdot_{ij}$, as they would transpose not only values but intended domains.

We can strengthen our position further by representing formulas $\phi$ as sets themselves. 

\begin{description}

\item[predicate and relation symbols:]  For any set $x$, we allow $\left<0,x\right>$ to be an atomic unary predicate symbol and $\left<1,x\right>$ to be an atomic binary relation symbol.  We provide that $\left<0,0\right>$ represents the predicate {\tt set} of sethood (if atoms are considered) and that for each $x \in D$,  $\left<0,\{x\}\right>$ represents the predicate $P_x(y)$ defined as $y=x$, and $\left<1,0\right>$ and $\left<1,1\right>$ represent = and $\in$, respectively.   We could also provide constants, but we note that a constant $c$ can always be handled by introducing a predicate $C$ such that $Cx_i$ means $x_i=c$, and we have arranged to be able to do this by providing for each $x \in D$ a predicate true exactly of $x$.  It does simplify things that the only ``nouns" in the language we implement here are the variables $x_i$ for natural numbers $i$.  We could introduce other predicates, but for our immediate purposes we will not need to do this.

\item[atomic formulas:]  We let $\left<2,P^*,n\right>$ represent $Px_n$, where $P^*$ represents the logical predicate $P$.  We let $\left<3,R^*,m,n\right>$ represent
$x_m \,R \,x_n$, where $R^*$ represents the logical relation $R$.

\item[propositional logic:]  We let $\left<4,\phi^*,\psi^*\right>$ represent $\phi \wedge \psi$, and $\left<5,\phi^*\right>$ represent $\neg\phi$, where $\phi^*$ represents
$\phi$ and $\psi^*$ represents $\psi$.

\item[quantifiers:]  We let $\left<6,\phi^*,i\right>$ represent $(\exists x_i.\phi)$, where $\phi$ is represented by $\phi^*$.

\item[other logical operations:]  One may add more clauses for further logical operations and quantifiers or view them as always abbreviating constructions using the operations given, which are adequate.

\end{description}

We can then associate with every set which is an expression according to the definition just given a predicate set which it is intended to represent, given intended representations for
each symbol $\left<0,x\right>$ and $\left<1,x\right>$.  The notation ${\tt ref}$ below everywhere abbreviates ${\tt ref}_{\cal U,R}$, where $\cal U, R$ are specific functions explained in the first clause.

\begin{description}

\item[predicate and relation symbols:]  We define ${\tt ref}(\left<0,x\right>)$ as a predicate set ${\cal U}(x)$ of order 1 for each $x$ we use as a unary predicate symbol in our language.
We define ${\tt ref}(\left<1,x\right>)$ as a predicate set ${\cal R}(x)$ of order 2 for each $x$ we use as a binary predicate symbol in our language.  We stipulate that
${\cal U}(0)$ is $\{f \in D^{\infty} \mid {\tt set}(f(0))\}$ [if we make use of atoms, which for the most part we will not] and that for each $x \in D$, ${\cal U}(\{x\}) = \{f \in D^{\infty}\mid f(0)=x\}$ [so that we have a predicate which picks out each individual member of $D$, whether we can actually characterize it with a formula or not],   and ${\cal R}(0)$ is $\{f \in D^{\infty} \mid f(0)=f(1)\}$ and $R(1)$ is  $\{f \in D^{\infty} \mid f(0)\in f(1)\}$

\item[atomic formulas:]  

Define ${\tt ref}(\left<2,P,n\right>)$ as ${\tt ref}(P)_{0,n}$, if $\pi_1(P)=0$ and ${\tt ref}(P)$ is defined. 

For any natural numbers $m,n,p$, define $(mn)p$ as $n$ if $p=m$, $m$ if $p=n$, and otherwise as $p$.

 Define ${\tt ref}(\left<3,R,m,n\right>$ as $(({\tt ref}(R))_{0,m})_{(0,m)1,n}$, when $m\neq n$ and further $\pi_1(R)=1$ and ${\tt ref}(R)$ is defined.  Just swapping $m$ with 0 and $n$ with 1 does not always work.

Define ${\tt ref}(\left<3,R,m,m\right>$ as $${\tt ref}(\left<6,\left<4,\left<3,R,m,m+1\right>\left<3,\left<1,0\right>,m,m+1\right>\right>,m+1\right>)$$ when $\pi_1(R)=1$ and ${\tt ref}(R)$ is defined.  The idea here is that $x_m \, R \, x_m$ is equivalent to $(\exists x_{m+1} : x_m \,R\, x_{m+1} \wedge x_m = x_{m+1})$, and that is what that nasty expression does.

\item[propositional logic:]  Define ${\tt ref}(\left<4,\phi,\psi\right>$ as ${\tt ref}(\phi) \cap {\tt ref}(\psi)$ where ${\tt ref}(\phi)$ and ${\tt ref}(\psi)$ are defined.  Define ${\tt ref}(\left<5,\phi\right>)$ as $({\tt ref}(\phi))^c$, where ${\tt ref}(\phi)$ is defined.

\item[quantifiers:]   Define ${\tt ref}(\left<6,\phi,i\right>)$ as $\exists_i({\tt ref}(\phi))$, when ${\tt ref}(\phi)$ is defined.

\item[other logical operations:]  One may add more clauses for further logical operations and quantifiers or view them as always abbreviating constructions using the operations given, which are adequate.

\end{description}

The added power here is that we have imported all the formal sentences of our logical language into our mathematical universe, and have assigned meanings to all sentences, subject to the condition that all quantifiers are restricted to the particular domain set $D$ (or to sets $D(i)$ in the alternative approach we sketched).

We can further establish that both the collection of sets representing logical formulas and the function ${\tt ref}_{U,R}$ just defined are actually sets.

\begin{description}

\item[Definition:]  Let $D$ be a set.  Let $\cal U$ and $\cal R$ be functions with range included in $D^{\infty}$ and satsifying further conditions described above.  A $\cal U,R$-{\em formula-inductive set\/} is a set $I$ with the following closure properties:

\begin{enumerate}

\item $\left<0,P\right>$ is in $I$ iff $P \in {\tt dom}(\cal U)$.  $\left<1,R\right>$ is in $I$ iff $R \in {\tt dom}(\cal R)$.

\item  $\left<2,P,n\right>$ is in $I$ iff $n$ is a natural number, $\pi_1(P)=0$, and $\pi_2(P) \in {\tt dom}(\cal U)$.  $\left<3,R,m,n\right>$ is in $I$ iff $m,n$ are natural numbers,
$\pi_1(R) = 1$ and $\pi_2(R) \in {\tt dom}(\cal R)$.

\item $\left<4,P,Q\right>$ and $\left<5,P\right>$ are elements of $I$ if $P$ and $Q$ belong to $I$.

\item $\left<6,P,n\right>$ belongs to $I$ iff $P \in I$ and $n$ is a natural number.


\end{enumerate}

\item[Definition:]  ${\cal L}_{\cal U,R}$ is defined as the intersection of all $\cal U,R$-formula inductive sets.  The existence of a set including the natural numbers and the domains of $\cal U$ and $\cal R$ and closed under pairing will provide such an inductive set to start with:  a limit rank of the cumulative hierarchy containing both of these sets does the trick, and in fact less is needed (just a rank of infinite index including those sets)  if one uses a different ordered pair definition.  Note that the countable set $V_{\omega}$ suffices if the domains of $\cal U$ and $\cal R$ are finite (and so may harmlessly be supposed inhabited by natural numbers).   The letter $\cal L$ should suggest ``language":  these are the sentences of a formal language, internalized as objects of our set theory.

\item[Definition:]  Where $D, \cal U, R$ are as above, a ${\tt ref}_{\cal U, R}$-inductive set is a set $I$ which is a relation with domain $L_{\cal U,R}$ and range $D^{\infty}$ and has the following closure properties:

\begin{enumerate}

\item $\left<\left<0,x\right>,{\cal U}(x)\right>$ belongs to $I$ for each $x$ in the domain of $\cal U$.   $\left<\left<1,x\right>,{\cal R}(x)\right>$ belongs to $I$ for
each $x$ in the domain of $\cal R$.

\item  $\left<\left<2,P,n\right>,X_{0,n}\right>$ belongs to $I$ if $\left<P,X\right>$ belongs to $I$ and $\pi_1(P)=1$.

\item $\left<\left<3,R,m,n\right>,(X_{0,m})_{(0m)1,n}\right>$ belongs to $I$ if $\left<R,X\right>$ belongs to $I$, $\pi_1(R)=1$, and $m\neq n$.

\item $\left<\left<3,R,m,m\right>,X\right>$ belongs to $I$ if $$\left<\left<6,\left<4,\left<3,R,m,m+1\right>,\left<3,\left<1,0\right>,m,m+1\right>\right>,m+1\right>,X\right>$$ belongs to $I$.

\item $\left<\left<4,P,Q\right>,X \cap Y\right>$ is in $I$ if $\left<P,X\right>$ and $\left<Q,Y\right>$ are in $I$.

\item $\left<\left<5,P\right>,X^c\right>$ is in $I$ if $\left<P,X\right>$ is in $I$.

\item $\left<\left<6,P,n\right>,\exists_iX\right>$ is in $I$ if $\left<P,X\right>$ is in $I$.

\end{enumerate}

\item[Definition:]  The relation ${\tt ref}_{\cal U,R}$ is defined as the intersection of all ${\tt ref}_{\cal U,R}$-inductive sets.  Notice that ${\cal L}_{\cal U,R} \times D^{\infty}$ is such an inductive set.  It might be an instructive exercise to prove that this is a function with domain ${\cal L}_{\cal U,R}$.   Once it is seen to be a function, it is seen to satisfy the conditions in its earlier informal definition.

\end{description}

Notice that we can characterize a formula in $L_{\cal U,R}$ as simply true if its image under ${\tt ref}_{\cal U,R}$ is the universal predicate set $D^{\infty}$, and false if its image is $\emptyset$.  it is very important to observe that we have only defined this notion of truth for formulas in which every formula is bounded in the set $D$.  We could further adapt this to allow each variable $x_i$ to be bounded in a set $D_i$ as suggested above, but in any case all quantifiers must be bounded in sets.

We can now define the constructible universe $L$ of G\"odel.  It is the ``union" (not a set) of a sequence of ranks indexed by the cumulative hierarchy, but not the same ranks as in the case of the universe $V$.

For any set $D$, and the minimal functions ${\cal U}$ and ${\cal R}$ defined exactly as above with no additional predicates, define ${\tt Def}(D)$ as the collection of all sets
$E$ such that $\{f \in D^{\infty} \mid f(0) \in E\}$ is in the range of ${\tt ref}_{\cal U,\cal R}$.  In other words ${\tt Def}(D)$ is the collection of all subsets of $D$ definable
in first-order logic with all quantifiers bounded in $D$, with no primitive logical notions available other than membership, equality, and each element of $D$ considered as a constant [and in addition sethood if atoms are admitted].

We then define

\begin{enumerate}

\item $L_0 = \emptyset$ [or the set of all atoms if atoms are present and make up a set].

\item $L_{\alpha+1} = {\tt Def}(L_{\alpha})$

\item $L_{\lambda}= \bigcup\{L_{\beta} \mid \beta<\lambda\}$ for $\lambda$ limit.

\end{enumerate}

The whole exertion of this section was in showing that ${\tt Def}$ can indeed be defined internally to our set theory.  We say that a set is {\em constructible} if it is an element of some $L_{\alpha}$, and we refer to the collection of all constructible sets (which is certainly not a set) as the {\em constructible universe\/} $L$.   We will assume that there are no atoms unless we specifically state otherwise.

It is perhaps worth observing that the existence of $L_{\alpha+1}$ follows from the existence of $L_{\alpha}$ and the axioms of Zermelo set theory, but the existence of $L_{\lambda}$ in general requires Replacement (or at least the axiom of hierarchy).

\newpage

\subsection{Defining the well-ordering on $L$}

We now examine what sets we can construct in $L$.

\begin{description}

\item[Theorem:]  If $\phi$ is a formula in which every quantifier is bounded in a set in $L$, then $\{x \in L_{\alpha} \mid \phi\}$ belongs to some $L_{\beta}$.  

\item[Proof of Theorem:]   Quantifiers bounded to specific sets are definable because individual objects (such as the specific sets to which the quantifiers are bounded) have predicates which pick them out.

It is not necessarily the case that $\beta=\alpha+1$:  what we can say is that if $\gamma$ is the maximum of all the ordinals $\delta$ such that some object named in $\phi$ belongs to $L_{\delta}$, then $\{x \in L_{\alpha} \mid \phi\}$ belongs to $L_{\max(\alpha,\delta)+1}$,
because $\{x \in L_{\alpha} \mid \phi\} = \{x \in L_{\max(\alpha,\delta)+1} \mid x \in L_{\alpha} \wedge \phi\}$, and this clearly belongs to 
${\tt Def}(L_{\max(\alpha,\delta)})$.

\end{description}

It is important to observe that every finite subset of $L_{\alpha}$ belongs to $L_{\alpha+1}$.  Iterated application of this fact tells us that every
bijection from a natural number to a set $A$ exists in $L_{\alpha+3}$ if $A$ is in $L_{\alpha}$, and so the collection of finite subsets of $L_{\alpha}$ is in something like $L_{\alpha+4}$, definable as the collection of subsets of $L_{\alpha}$ which are the same size as an element of $\omega$.

It is important to note that there is no {\em prima facie} reason to believe that all countably infinite subsets of a given set of $L_{\alpha}$ are found in $L_{\beta}$ for
any $\beta>\alpha$ whatsoever.

If $D \in L_{\alpha}$, every pair of an element of $D$ and a natural number exists in $L_{\alpha+2}$, and every element of $D^{\infty}$ exists in
$L_{\alpha+3}$, and the set $D^{\infty}$ itself exists in $\alpha+4$ (the test on a countable sequence for whether it is in $D^{\infty}$ has to do with whether its range is finite, and one can already identify finite subsets of $D$ at the level of $L_{\alpha+4}$.  Nothing hangs on getting these small finite numbers exactly right.

For any set $x$, we can define ${\tt TC}(\{x\})$ as the collection of all $y$ such that there is a finite sequence $s$ with domain $n+1$ such that $s_0=y$, $s_{i} \in s_{i+1}$ for each $i<n$, and $s(n)=x$.  If $x \in L_{\alpha}$, all terms of this sequence will be in $L_{\alpha}$ because $L_{\alpha}$ is a transitive set, and it will be definable in $L_{\alpha+4}$ or so.

We claim that $x = {\tt Def}(D)$ is definable by a formula with every quantifier bounded in $L$.  This is rather involved.  We can certainly describe $D^{\infty}$.  We can describe the minimal sets $\cal U$ and $\cal R$.  

That an object $x$ belongs to ${\cal L}_{\cal U,R}$ is equivalent to the existence of a finite subset of ${\tt TC}(\{x\})$ with certain closure properties:  define the parse tree of $x$ as the intersection of all sets $T$ which contain $x$ and have the following closure properties:

\begin{enumerate}

\item if $y=\left<2,P,n\right>\in T$ then $P \in T$

\item if $y = \left<3,R,m,n\right>\in T$ then $R \in T$

\item if $y = \left<4,P,Q\right>\in T$ then $P,Q \in T$

\item if $y = \left<5,P\right> \in T$ then $P \in T$

\item if $y= \left<6,P,n\right> \in T$ then $P \in T$

\end{enumerate}

The parse tree of any $x$ must be finite (there cannot be an infinite descending sequence of ``subterms" of $x$ by Foundation).
We say that $x \in L_{\cal U,R}$ iff every element of the parse tree of $x$ which does not have first projection between 2 and 5 inclusive
is either $\left<0,0\right>$ (in case we are considering atoms) or a $\left<0,\{x\}\right>$ for $x \in D$, or $\left<1,0\right>$ or $\left<1,1\right>$.

That a pair $\left<x,P\right>$, with $x \in {\cal L}_{\cal U,R}$ and $P \in D^{\infty}$,  belongs to ${\tt ref}_{\cal U,R}$ is similarly witnessed by a finite set, the intersection of all sets $T$ which contain
$\left<x,P\right>$ and have closure conditions

\begin{enumerate}

\item  If $\left<\left<4,P,Q\right>,A\right> \in T$, then there are $B$ and $C$ such that $\left<P,B\right> \in T$ and
$\left<Q,C\right> \in T$ and $A = B \cap C$.

\item If $\left<\left<5,P\right>,A\right> \in T$, then $\left<P,A^c\right> \in T$.

\item If $\left<\left<6,P,n\right>,A\right> \in T$, then there is $B$ such that $\left<P,B\right> \in T$ and $\exists_n(B) = A$.

\item If $\left<\left<2,P,n\right>,A\right> \in T$ then $\left<P,A_{0,n}\right>\in T$

\item If $\left<\left<3,R,m,n\right>,A\right> \in T$ then there is $B$ such that $\left<P,B\right>\in T$ and $A = (B_{0,m})_{(0m)1,n}$.

\item If $\left<\left<3,R,m,m\right>,A\right> \in T$ then the mutant translation of it given above also belongs to $T$.

\end{enumerate}
 If $\left<\left<0,X\right>,A\right>$ or  $\left<\left<1,X\right>,A\right>$ belong to the intersection of all such $T$, then the value of $A$ is required to  be determined by the intended semantics in the obvious way.


The minimal such set will be finite, with first coordinates restricted to the parse tree of $x$ (mod additions caused by the repeated arguments clause for relations, which will only be applied finitely many times).

Now, since we can define the formula $x={\tt Def}(D)$ in such a way that we can use it in the definitions of sets in $L$, this means that we
can say what it means for a set to be an $L_{\alpha}$:  we can assert that there is a sequence indexed by an ordinal in which each term is the image under {\tt Def} of the previous term (if there is a previous term) or the union of all previous terms if there is no previous term.

Now we can define a well-ordering on $L$ (global on the entire universe).  Certainly $L_0$ is well-ordered (uniquely).  Suppose for each $\beta<\alpha$
that we have well-ordered $L_{\beta}$, and that further that the order that we have defined on each $L_{\gamma}$ with $\gamma<\beta$ is a segment restriction
of the order on $L_{\beta}$.  It follows immediately that we have a well-ordering on $L_{\alpha}$ if $\alpha$ is limit, and if the order on each $L_{\beta}$ for $\beta<\alpha$ is defined in a uniform way which can be described in language bounded to an $L_{\alpha+i}$ for a small finite $i$;  it remains to show how to get
a well-ordering on $L_{\alpha}$ if $\alpha=\beta+1$.  We have an order on $L_{\beta}$ already and we know that we want each element of
$L_{\beta}$ to appear before all elements of $L_{\alpha} - L_{\beta}$ in this order.  So, to decide which order two elements of $L_{\alpha} - L_{\beta}$
are to be placed in, we appeal to an order on ${\cal L}_{\cal U,R}$, and we place $x$ before $y$ iff $x$ is defined in ${\tt Def}(L_{\beta})$ by a formula which appears before any formula defining $y$.  Since we can define formulas and their references in $L$ in a way which supports definition of sets in $L$, all that remains is to define an order on the formulas in ${\cal L}_{\cal U,R}$.  Lexicographic order suffices:  order first by the natural number initial in the formula, then recursively by the same order on simpler formulas for subformulas, by numerical order for numeral components,
and (tricky last point) by the order already defined for $L_{\beta}$ for constants in $L_{\beta}$ appearing as components of $\cal U$.  Now, the successor case shows us how the order
on $L_{\beta+1}$ is determined in a uniform way from the order on $L_{\beta}$, which allows us to meet the condition stated above which is required at limit ordinals:  we need to say not only that we have defined $L_{\gamma}$ for each $\gamma<\alpha$ limit, but also that each $L_{\gamma+1}$ is determined by $L_{\gamma}$ as discussed in the successor case (which is expressible in suitably bounded language), in addition to the assertion already made that for $\delta<\gamma<\alpha$, the order on $L_{\delta}$ is a segment restriction of the order on $L_{\gamma}$.


Now we can define a Hilbert symbol $(\epsilon x:\phi[x])$ for any formula $\phi$:  $y=(\epsilon x:\phi[x])$ means that $\phi[y]$ is true
and $y$ is the least object in the well-order on $L$ for which this is true, or that there is no $x$ such that $\phi[x]$ and $y= \emptyset$.  This works even if $\phi$ is an unbounded formula.

\newpage

\subsection{$L$ satisfies the axioms of ZFC}

We show that the universe $L$ of constructible sets satisfies the axioms of ZFC.  We remark that though we have no collection $L$, we can make sense of ``$x$ is in $L$" as meaning
``There is an ordinal $\alpha$ such that $x \in L_{\alpha}$".

We consider the axioms one by one.

\begin{description}

\item[Extensionality:]  We want to show that if $x$ and $y$ are in $L$ and they have the same elements which belong to $L$, then they are the same set.  The crucial point here is that
$L$ is transitive:  if $x$ is in $L$, there is a first $L_{\alpha}$ to which $x$ belongs, which must be an $L_{\beta+1} = {\tt Def}(L_{\beta})$.  Any $y \in x$ then clearly belongs to
$L_{\beta}$ and so is in $L$.  Note that not only $L$ but each $L_{\alpha}$ is transitive for just these reasons.  Thus if $x$ and $y$ are in $L$, and have the same elements belonging to $L$, then they have the same elements in the real world $V$ (the collection of all sets of our set theory) and so they are equal.

\item[Pairing:]  If $x \in L_{\alpha}$ and $y \in L_{\beta}$ then $\{x,y\} \in L_{{\tt max}(\alpha,\beta)+1}$ for obvious reasons.

\item[Union:]  If $x \in L_{\alpha}$, then the union of $x$ is definable in $L_{\alpha+1}$ as usual, since all elements of $x$ and elements of elements of $x$ also belong to $L_{\alpha}$,
which is a transitive set as noted above.

\item[Infinity:]  $\omega \in L$.

\item[Foundation:]  This follows directly from Foundation in $V$ and the fact that $L$ is transitive (the fact that $L$ is transitive isn't even needed, but it makes it easier).

\item[Power Set:]  This is the first hard case.  Suppose $A \in L_{\alpha}$.  The collection ${\cal P}(A) \cap L$ is a set by Separation in the real world.  Let $\phi(x,\beta)$ mean
'``$\beta$ is the smallest ordinal such that $x \in L_{\beta}$.  This is a functional formula.  By Replacement in the real world $V$, the collection of all ordinals $\beta$ such
that $\phi(x,\beta)$ holds for some $x \in {\cal P}(A) \cap L$ is a set.  The union of this set is an ordinal $\gamma$, and ${\cal P}(A) \cap L$ appears in $L_{\gamma+1}$, defined
as the collection of all elements of $L_{\gamma}$ which are included in $A$ as a subset.  ${\cal P}(A) \cap L$, once it is seen to be an element of $L$, witnesses the truth of the Axiom of Power Set with all quantifiers restricted to $L$.  Notice that many or most subsets of $A$ which are in $V$ may not ever appear in $L$ at all:  this power set may be quite impoverished when viewed as it were from the outside.

\item[Separation:]  To prove separation (and replacement) we need to give some definitions and prove a lemma.

\begin{description}

\item[Definition:]  Where $M$ is a subcollection of the universe, what we mean by saying that a formula $\phi$ is true in $M$ should be formalized (we have already been talking about this informally).  We read $M \models \phi$ as ``$M$ says that $\phi$ is true" or ``$\phi$ is true in $M$".

\begin{enumerate}

\item $M \models x \in y$ or $M \models x=y$ means the same thing as $x \in y$ or $x=y$ just in case $x \in M$ and $y \in M$.

\item $M \models \neg \phi$ just in case $\neg(M \models \phi)$.  $M \models (\phi \wedge \psi)$ just in clase $(M \models \phi) \wedge (M \models \psi)$.

\item $M \models (\exists x:\phi)$ just in case $(\exists x \in M:M \models \phi)$.  Notice the important restriction of the quantifier here.

\end{enumerate}

Notice that this definition works equally well if $M$ is a set $A$ or if $M$ is a ``collection" defined by a formula, such as $L$.

\item[Definition:]  If $M \subseteq N$ are collections, we say that $M$ agrees with $N$ about $\phi[x_1,\ldots,x_n]$, where the $x_i$'s are all the free variables in $\phi$,
iff $M \models \phi[t_1,\ldots,t_n]$ if and only if $N \models \phi[t_1,\ldots,t_n]$ for all $t_1,\ldots, t_n \in M$.

\item[Lemma:]  For every formula $\phi$ and ordinal $\beta$, there is an $L_{\alpha}$ with $\alpha>\beta$ which agrees with $L$ about $\phi$.

\item[Proof of Lemma:] Fix a formula $\phi$.

 For any formula $\psi[x,x_1,\ldots,x_n]$, define $\beta_{\phi}(x_1,\ldots,x_n)$ as the first ordinal such that $L_{\beta_{\phi}(x_1,\ldots,x_n)}$ contains a $t$
such that $L \models \phi[t,x_1,\ldots,x_n]$, or 0 otherwise.  Define ${\tt cl}_{\phi}(A)$ for any set $A \in L$ as the supremum of all $\beta_{\psi}(x_1,\ldots,x_n)$ for $\psi$ a subformula of $\phi$
and values of $x_i$'s taken from $A$.  Define ${\tt cl}_{\phi}^0(A)$ as $A$ and ${\tt cl}_{\phi}^{n+1}(A)$ as ${\tt cl}_{\phi}({\tt cl}^n_{\phi}(A))$.  The various closure sets exist by applications of Replacement.  It is important to notice that we are dealing with only finitely many formulas at a time:  we can define $\beta_{\phi}(x_1,\ldots,x_n)$  for concretely given formulas (finitely many of them) but we cannot uniformly define what it means for an arbitrary formula to be true in $L$ inside $L$ (or even inside $V$!).

It is straightforward to establish that ${\tt cl}^{\omega}(A)$, the union of all ${\tt cl}^n_{\phi}(A)$'s agrees with $L$ about $\phi$ when all values of free variables in $\phi$ are
taken from ${\tt cl}^{\omega}(A)$:  this is proved by induction on the definition of $\models$, and relies on the fact that for any subformula of $\phi$ and any choice of free variables
from ${\tt cl}^{\omega}(A)$ other than $x$, we have arranged for a witness to $(\exists x:\phi)$ to exist in ${\tt cl}^{\omega}(A)$ (in the next ${\tt cl}^n_{\phi}(A)$ after the first one which contains all the values assigned to free variables) just in case it exists in $L$.

Now the desired $L_{\alpha}$ is ${\tt cl}^{\omega}(L_{\beta})$.

\end{description}


Now we can complete the proof of separation.  Let $A$ be a set in $L_{\alpha}$ and let $\phi$ be a formula (in which there may be unbounded quantifiers over $L$).  Let
$L_{\gamma}$ be a level of $L$ above $L_{\alpha}$ which agrees with $L$ about $\phi$ (notice that this works if free variables appear in $\phi$, too).  The set $\{x \in A\mid \phi\}$
in the sense proper to $L$ then appears in $L_{\gamma+1}$, defined as the collection of all elements of $L_{\gamma}$ which belong to $L$ and satisfy $\phi$ as localized to $L_{\gamma}$.

\item[Replacement:]  Suppose $A \in L_{\alpha}$ and suppose $\phi[x,y]$ is a functional formula.  Find an $L_{\gamma}$ with $\gamma>\alpha$ which agrees with $L$ about
$\phi$.  The collection of $y$ such that $L$ says that for some $x \in A$, $\phi[x,y]$ is definable in $L_{\gamma+1}$ since $L_{\gamma}$ agrees with $L$ about $\phi$ (and so contains all the needed images!).

\item[Choice:]  We showed in the previous section that $L$ sees a well-ordering of every set.  So if $P$ is a partition in $L$, well-order $\bigcup P$ using that partition and let $C$ be the collection of first elements in that order of elements of $P$.

\end{description}

\newpage

\subsection{$L$ satisfies GCH}

We first observe that the cardinality of $L_{\alpha}$ is the same as the cardinality of $\alpha$, for each infinite ordinal $\alpha$.  We argue for this by transfinite induction.  This is as true in $L$ as it is true in $V$.

Clearly $|L_{\omega}| = \omega$:  $L_{\omega}$ is the union of countably many finite sets $L_n$.

Suppose that $L_{\beta}$ has the same cardinality as $\beta$.  $L_{\beta+1}={\tt Def}(L_{\beta})$ obviously has cardinality at least that of $\beta$, since it includes $\beta$ as a subset.   We show that it has cardinality 
at most that of $\beta$:  each element of ${\tt Def}(\beta)$ is associated with one or more finite length expressions in a language with $|\beta|$ symbols, so the size of ${\tt Def}(\beta)$ is bounded by the size of the collection of such expressions.  The collection of such expressions of length $n$ is at most $|\beta|^n=|\beta|$, and the collection of all such expressions of all lengths has size at most $|\beta| \cdot \aleph_0=|\beta|$.

Suppose that $L_{\beta}$ is of size $|\beta|$ for each $\beta<\lambda$ limit.  The union is of size no more than $|\lambda| \cdot |\lambda|=\lambda$ (it is no larger than a disjoint union of 
copies of all the $L_{\beta}$'s).  It is clearly of size at least $|\lambda|$ since it contains all elements of $\lambda$.

Let $A \in L_{\alpha}$ and suppose that $|A|=|\alpha|$ according to $L$ (we can arrange this for any cardinal $|\alpha|$ in the sense of $L$ by considering $A=\alpha \in L_{\alpha+1}$).  Each $B \in {\cal P}(A) \cap L$ appears first in some $L_{\gamma}$.  We claim (and must prove below) that $|\alpha|=|\gamma|$ and moreover that this is true in $L$.  We note first that this will prove our result:  every subset of $A$ must then appear in $L_{(|\alpha|+)_L}$, which $L$ itself sees as having cardinality $|\alpha|+$ (the ordinal $(|\alpha|+)_L$ which $L$ thinks is the next cardinal after $|\alpha|$ might be of the same cardinality as $\alpha$ as far as $V$ is concerned, because $L$ might be missing some bijections!).  So $L$ sees the cardinality of ${\cal P}(A)$ (which it sees as $2^{|\alpha|}$) as bounded
by $|\alpha|+$, and on the other hand the cardinality of ${\cal P}(A)$ must be at least $|\alpha|^+$ by Cantor's theorem.

It remains to establish our claim.

Recall that we defined a Hilbert symbol  $(\epsilon x:\phi[x])$ above, as the $L$-first object $x$ such that $\phi[x]$, or else 0 if there is no such object.  We define a closure in a way similar to the way we defined a closure in the Lemma proving Separation above, though this one will be a smaller set.  We begin with all elements of $L_{\alpha}$ and all elements of the set $B$
(which has no more than $|\alpha|$ elements).  We go through countably many steps.  At each step we add $(\epsilon x:\phi[x,t_1,\ldots,t_n])$, with $\phi$ being interpreted in the sense
of $L_{\gamma}$, and all $t_i$'s added at previous stages.  Notice that here we can refer to all formulas uniformly:  we know how to determine whether any formula at all is true with its quantifiers restricted to $L_{\gamma}$ and its parameters taken from objects present at the previous stage, using our logically regimented set construction machinery.   After $\omega$ steps, we have built a structure $M$ which agrees with $L_{\gamma}$ about {\em every} formula:  this is established by an induction on the definition of $\models$.

Further,
this structure $M$ is of size $|\alpha|$, for the same sort of reason that each $L_{\alpha}$ is of size $|\alpha|$:  it is built using finite strings of symbols taken from a previous stage which may be supposed of size $|\alpha|$ in an inductive argument.

Now we take a Mostowski collapse of $M$.  $M$ just thinks it is an $L_{\alpha}$ (because it has delusions that it is $L_{\gamma}$).  But membership on $M$ is extensional (if two elements of $M$ are distinct, we might find that some elements of these elements of $M$ are not in $M$, but because $L_{\gamma}$ knows that they are different, some element of their symmetric difference (describable by a Hilbert symbol) is in $M$) and well-founded (because it is a subrelation of true membership on $L_{\gamma}$, which is well-founded) so we can collapse $M$ using a Mostowski collapse.

Each element of the collapsed set $M^*$ actually has exactly the members the theory of $M^*$ says it has, and in fact $M^*$ is an $L_{\beta}$:  we can see this  by induction on
$\delta<\beta$.  Let $\delta$ be the first ordinal such that $L_{\delta}$ as $M^*$ sees it is not the real $L_{\delta}$.  This $\delta$ cannot be 0.  It cannot be a limit $\lambda$ because
it is then the union of all the things $M^*$ sees as $L_{\chi}$ for $\chi<\delta$, and this really will be $L_{\delta}$ because $M^*$ correctly identifies all the earlier ones.
Finally, if $M^*$ identifies $L_{\chi}$ correctly, it also identifies ${\tt Def}(L_{\chi})$ correctly (because the elements of ${\tt Def}(L_{\chi})$ are defined using finite expressions
built up from symbols taken from $L_{\chi}$) so the first bad $L_{\delta}$ cannot be a successor $L_{\chi+1}$ either!  $M^*$, since it thinks it is $L_{\gamma}$, sees itself as either
${\tt Def}(L_{\delta})$ for some $\delta$ or a union of $L_{\delta}$'s, and since it has a correct understanding of what sets are levels of $L$, this means that it is itself a level of $L$.

So we have an $L_{\beta}$ of size $|\alpha|$ (it is the same size as $M$) which contains all elements of $B$.  $B \in L_{\beta}$ because it is actually defined in exactly the same way it was defined in $L_{\gamma}$!  But $|\beta|=|\alpha|$, so the first possible $\gamma$ must also have been of this cardinality (it is a final weird consequence of this argument that if we chose the first possible $\gamma$ that in fact $\gamma=\beta$, and nothing much really happened in the collapse).

\newpage

\subsection{Final remarks about $L$}

If we accepted the universe of constructible sets as the universe of sets, we would thereby answer almost all questions about set theory.   Why do we not believe that $L$ is the universe?

I intend to add more (but still brief) discussion of this question.

\subsubsection{Exercises}

\begin{enumerate}

\item  Read the section on $L$ and send me any remarks you have about typos, points of confusion, and so forth.

\item  Determine which $L_{\alpha}$ contains the set of all finite functions with domain and range included in $\omega$ (this is a bookkeeping problem).

\item  If the set $D$ belongs to $L_{\alpha}$, determine the smallest finite $n$ for which it can be shown that $D^{\infty} \in L_{\alpha+n}$ (this is again bookkeeping, like the previous problem).  You might want to write out actual definitions of typical elements of $D^{\infty}$ to see how this works.

Explain why I cannot expect to be able to define $D^{\mathbb N}$ in $L$ (if $D$ is infinite).

\item  Prove that for each finite $n$, $L_n = V_n$.  (This is straightforward).

Prove that $L_{\omega} = V_{\omega}$ (give the brief justification on the basis of what we have already shown).

Now prove that $L_{\omega+1} \neq V_{\omega+1}$ (something we have shown recently will show this immediately).

It might be that $L_{\alpha} = V_{\alpha}$ will have {\em only\/} the finite ordinals and $\omega$ as solutions for $\alpha$.  If $V \neq L$,
there are very simple situations under which this will happen (describe such a situation).

If $V=L$, there is a next $\alpha$ above $\omega$ such that $L_{\alpha} = V_{\alpha}$, and you should be able to explain what this value
of $\alpha$ is and why on the basis of things shown recently.

\item (this might be rather evil)  If $D$ is a transitive set, demonstrate that the set $X=\{A \in D \mid |A|=1\}$ belongs to ${\tt Def}(D)$ by showing how to construct a subset $Y$ of $D^{\infty}$ using the axioms of cylindrical algebra such that the set $\{y(0) \mid y \in Y\}=X$.   Explain why it is necessary for me to assume that $D$ is a transitive set in order for you to be able to define this.  My mental model of this is that you will need to carry out a series of definitions of
subsets of $D^{\infty}$ using the cylindrical algebra operations based on the defining formula of $X$.



\end{enumerate}

\newpage

\section{Theories with proper classes}

In this section, we outline approaches to foundations basically similar to what we have done so far, modified to allow us to speak of large collections like the Russell class as objects.  The key idea is that the very large collections (which we call ``classes") cannot themselves be members of classes.

We present the axioms of a theory of this kind.  The primitive predicates of the theory are equality and membership.

\begin{description}

\item[The empty class; definitions of atom and class:]  There is a distinguished object $\emptyset$ with no elements, which we call the empty class.  Objects with no elements are called atoms.  Objects with elements and the empty class are called classes.

\item[Axiom of Extensionality:]    Classes with the same elements are equal.

\item[Definition:]  We say that $x$ is a {\em set\/} iff $(\exists y:x \in y)$:  elements are sets.  A class which is not a set is called a {\em proper class\/}.

\item[Axiom of Comprehension:]  For any formula $\phi[x]$, there is a class $\{x \in V\mid \phi[x]\}$ such that for each $a$,
$a \in \{x \in V\mid \phi[x]\}$ if and only if $a$ is a set and $\phi[a]$.  We define $V$ as $\{x \in V\mid x=x\}$.  $V$ is the class of all sets.

\item[Axiom of Elementary Sets:]  $\emptyset$ is a set.  For any sets $x,y$, $\{x,y\}$ is a set.

\item[Axiom of Power Set:]  For any set $x$, ${\cal P}(x)$ is a set.

\item[Axiom of Union:]  For any set $x$, $\bigcup x$ is a set.

\item[Axiom of Infinity:]  $\omega$ is a set.

\item[Axiom of Limitation of Size:]  For any class $A$, $A$ is a proper class if and only if there is a class bijection from $A$ to $V$.

\item[Axiom of Foundation:]  Each class has an element disjoint from itself.

\end{description}

We summarize why the sets of this theory satisfy the axioms of ZFC.  Extensionality for sets  (in its strong form if we assume that there are no atoms) follows immediately from Extensionality for classes.

The empty class is a set by Elementary Sets.  Pairing, Power Set, Union, and Infinity for sets are explicitly provided.  Foundation is explicitly provided.

Choice holds, strangely enough, by Limitation of Size.  The class of all von Neumann ordinals which are sets exists by Comprehension, is obviously a von Neumann ordinal, and cannot be a set on pain of the Burali-Forti paradox.  Thus by Limitation of Size the class of all von Neumann ordinals can be placed in one-to-one correspondence with the universe.  The class of von Neumann ordinals can be well-ordered, and from this well-ordering we obtain a well-ordering of the universe $V$, which gives us Choice (in a very strong form, in fact).

Replacement holds, because if there is a functional relation from a set $A$ to a class $C$, the functional relation can be implemented as a class bijection,
and since $A$ is not of the same cardinality as $V$, and $C=f``A$ cannot have larger cardinality than $A$, it follows that $C$ is not of the same cardinality
as $V$, and so is a set.  Separation we have seen follows from Replacement and the existence of the empty set.

The axiom of limitation of size gives a different and stronger approach to what properties have extensions which are sets than the approach implicit in separation (that a property has an extension which is a set if its extension is included in something already known to be a set).  The idea is that the common property of sets is that they are smaller than the universe.

The theory that we have described, which is called Morse-Kelley set theory, is somewhat stronger than ZFC.  To get a theory of essentially the same strength as ZFC, restrict the Axiom of Comprehension to apply only to formulas $\phi$ in which every quantifier is restricted to a class.

\subsection{Pocket set theory, or, who said mathematicians don't have a sense of humor?}

Pocket set theory is a theory with sets and classes which doesn't allow very large collections (or does it?).  It is based on the observation that
the only cardinals which ``occur in nature" are $\aleph_0$ and $c$, the cardinality of the set of natural numbers and the cardinality of the reals.
Its axiomatics are also just plain funny.

The primitive predicates of pocket set theory are equality and membership. General objects of the theory are called {\em classes\/}.  For simplicity we rule out atoms.

\begin{description}

\item[Axiom of Extensionality:]    Classes with the same elements are equal.

\item[Definition:]  We say that $x$ is a {\em set\/} iff $(\exists y:x \in y)$:  elements are sets.  A class which is not a set is called a {\em proper class\/}.

\item[Axiom of Comprehension:]  For any formula $\phi[x]$, there is a class $\{x \in V\mid \phi[x]\}$ such that for each $a$,
$a \in \{x \in V\mid \phi[x]\}$ if and only if $a$ is a set and $\phi[a]$.  We define $V$ as $\{x \in V\mid x=x\}$.  $V$ is the class of all sets.

\end{description}

Thus far the theory is almost the same as the theory with sets and classes given above.  We do not postulate the axiom of pairing (we will be able to prove it) but we define unordered pairs, ordered pairs and class bijections as usual (though as yet we do not know that there are any).

\begin{description}

\item[Definition:]  A class $A$ is {\em infinite\/} iff there is a class bijection from $A$ to a proper subset of $A$.  We say that two classes
are the same size iff there is a class bijection from one of the classes to the other.

\item[Axiom of Infinite Sets:]  There is an infinite set, and all infinite sets are the same size.

\item[Axiom of Proper Classes:]  All proper classes are the same size, and no proper class is the same size as a set.

\end{description}

We now prove a series of theorems, getting at the end to the point where we can see the shape of the world of pocket set theory.

\begin{description}

\item[Theorem:]  The Russell class $R= \{x \in V\mid x \not\in x\}$ is a proper class.

\item[Proof:]  This is familiar.

\item[Theorem:]  The empty class $\{x \in V \mid x \not\in x\}$ is a set.

\item[Proof:]  Otherwise the empty class is a proper class and so is the same size as the Russell class, so the Russell class is empty.  Let $I$ be an infinite set.
Then $\{I\} = \{x \in V \mid x \in I\}$ is a set, because it is clearly not the same size as the Russell class.  It is not infinite,
because it clearly (having one element) cannot be the same size as a proper subclass of itself.  But $\{I\}$ is then clearly a member of the Russell class
(as it is not an element of itself, being distinct from its infinite sole element), which is a contradiction.

\item[Theorem:]  For any set $x$, $\{x\} = \{y \in V\mid y=x\}$ is a set.

\item[Proof:]  Suppose that $\{x\}$ is a proper class for some set $x$.  Thus the Russell class is the same size as $\{x\}$ and has exactly one element.  Let $I$ be an infinite set.  $\{I,\emptyset\} = \{y \in V \mid y=I \vee y = \emptyset\}$ is then a set, because there clearly cannot be a class bijection from this class to $\{x\}$.  $\{I,\emptyset\}$ clearly belongs to the Russell class, as it cannot be equal to either of its elements by reason of size.
$\emptyset$ is a set and also belongs to the Russell class.  But the Russell class is supposed to have exactly one element, so we have arrived at a contradiction.

\item[Theorem:]  For any sets $x$, $y$, $\{x,y\} = \{x \in V\mid z=x \vee z=y\}$ is a set.  This is readily seen, as $\emptyset$, $\{\emptyset\}$
and $\{\{\emptyset\}\}$ are all elements of the Russell class, so it cannot be placed into a one to one correspondence with $\{x,y\}$.

\item[Theorem:]  For any sets $x,y$ the ordered pair $\left<x,y\right>$ is a set.  For any logical relation $R$, there is a class
$\{\left<x,y\right> \in V \mid x\,R\,y\}$ implementing $R$.  Thus, if there is a logically describable bijection between two classes, there is actually a class bijection between them.

\item[Theorem:]  The class of von Neumann ordinals is a proper class.  It is obviously a proper class von Neumann ordinal, which we will call $\omega_1$.

\item[Proof:]  The reasons for this are familiar.

\item[Theorem:]  The universe $V$ can be well-ordered.

\item[Proof:]  By the axiom of proper classes, $V$ is the same size as $\omega_1$.

\item[Theorem:]  There is an infinite ordinal.

\item[Proof:]  An infinite set $I$ will be the same size as a subclass of $\omega_1$, which will be the same size as an initial segment of $\omega_1$, which will be an ordinal.  This ordinal is a set because it is the same size as $I$ and it is infinite because $I$ is infinite.  

\item[infinite set ordinals discussed:]  We define $\omega$ as the first infinite ordinal, which we know to be a set.  All ordinals $\alpha$ with
$\omega \leq \alpha <\omega_1$ are sets (because they belong to $\omega_1$ and the same size as $\omega$ because they are infinite.  We see that $\omega$ is the familiar ordinal of that name, and we see that $\omega_1$, being the first uncountable ordinal, is also the familiar ordinal of that name.

\item[real numbers implemented:]  Natural numbers are represented as elements of $\omega$ as usual.  Positive rationals can be represented as pairs of positive natural numbers (and so are sets).  Reals can then be defined as initial segments of the positive rationals as elsewhere in these notes:  these are countable classes and thus sets.
The class of reals is the same size as the power class of $\omega$ for the usual reasons.  The power class of $\omega$  is larger than $\omega$ for the usual reasons.  Because it is larger than $\omega$ it is not a set and so is the same size as $\omega_1$.  Thus we obtain not only Choice from our version of Limitation of Size (the axiom of proper classes can be viewed thus) but also the Continuum Hypothesis.

\end{description}

There are two ways to view this.  We may suppose that we have a system in which all collections are very small (they being no larger than the system of real numbers) or we may suppose that our view is that the class of reals is very large.

\newpage

\subsection{The theory of the unsorted preamble with proper classes}

Somehow this seems to evade the fundamental motivation of the theory, but it does work.  Like other theories with proper classes, the real charm comes from a precise criterion for which classes are proper.

We develop a first order theory with equality and membership as primitives.

\begin{description}

\item[Definition:]  An object $x$ is an {\em element\/} (and we write ${\tt elt}(x)$) just in case $(\exists y:x \in y)$.

\item[Definition:]  We define $x \sim_\tau y$ as $(\exists z:{\tt elt}(z) \wedge x \in z \wedge y \in z)$.  We say that $x$ and $y$ are of the same type:  to be of the same type is to belong to the same {\em element\/}.

\item[Axiom of types:]  For each element $x$, there is an element $\tau(x)$ such that for all $y$, $y \in \tau(x) \leftrightarrow y \sim_\tau x$.  This object is called the type of $x$.  

We write $\tau^1(x)$ for $\tau(x)$ and for each numeral $n$ we write $\tau^{n+1}(x)$ for $\tau(\tau^n(x))$.  A type $\tau(x)$ not of the form $\tau^2(y)$ for some $y$ we may call a base type.

\item[$\sim_\tau$ is an equivalence relation on elements:]  For any element $x$, the condition $x \sim_\tau x$ holds trivially.  $x \sim_\tau y \leftrightarrow y \sim_\tau x$ is a theorem of first order logic.
If $x \sim_\tau y$ and $y \sim_\tau z$ then $z \sim_\tau y$ and so both $x$ and $z$ belong to $\tau(y)$, so $x \sim_\tau z$.

\item[Axiom of non-elements:]  An object $x$ is a non-element iff there are $y$ and $z$ such that $y \in x$, $z \in x$, and $\neg y \sim_\tau z$.  Notice that this implies immediately that
objects with no elements are elements and objects with one element are elements.

\item[Construction and axiom of empty sets:]  For each element $x$, there is an object $\emptyset_x \in \tau^2(x)$ with no elements.  This is called the empty set over $x$ or the empty set included in $\tau(x)$.   We say that an object $x$ is a {\em class\/} if it has elements or is an empty set.  

We say that an object $x$ is a {\em set\/} if it is a class and an element.
We say that an object $x$ is an {\em atom\/} if it is not a class (since a non-class has no elements, it is an element).  A class which is not an element may be called a {\em proper class\/}.

\item[Axiom of extensionality:]  Classes with the same elements are equal.  Notice that we have left open the possibility that there are many atoms in each type.

\item[Axiom of comprehension:]  For every formula $\phi$ in which $A$ does not appear free, $(\exists A:(\forall x:x \in A \leftrightarrow {\tt elt}(x) \wedge \phi))$.  If $(\exists x:{\tt elt}(x) \wedge \phi)$ holds the witness $A$ is unique by extensionality and we use the notation  $\{x:\phi\}$ for the witness $A$.  Notice that this axiom implies that there are empty objects, so there are elements.

\item[Definition:]  We define $x \subseteq y$ as holding for sets $x,y$ if $$(\forall z \in x:x \in y) \wedge ((\neg(\exists z:z \in x) \rightarrow (\exists u:x = \emptyset_u) \wedge x \sim_\tau y):$$  all elements of $x$ are elements of $y$ and if $x$ is empty, $x$ is the empty set of the same type as $y$ (there is at most one such empty set by extensionality).

\item[Axiom of intersection:]  $$(\forall xyz:{\tt elt}(x) \wedge {\tt elt}(y) \wedge z \in x \wedge z \in y \rightarrow x \sim_\tau y):$$  sets which meet are of the same type.

\item[Observation:]  For any nonempty set $A$ and property $\phi$ which holds of some element of $A$, $\{x:x \in A \wedge \phi\}$ is a set.  Every such set has a common element with $A$, so by the axiom of intersection is of the same type as $A$.   If $x \in A$, $\emptyset_x$ is of the same type as $A$.  Thus the collection of all subsets of $A$, which we write 
${\cal P}(A)$ and call the power set of $A$, is a set, since it does not have two distinct elements of different types.  The power set of an empty set clearly exists since it is just a singleton.

Further, if $x \in y$, we have $x \in \tau(x)$ so by intersection $y \sim_\tau \tau(x)$, so $y \in \tau^2(x)$.  The statement ``for any element $y$, $x \in y \rightarrow y \in \tau^2(x)$", in the presence of the previously given axioms is equivalent to the axiom of intersection.

\item[Axiom of levels:]  For any elements $x,y$, $$\tau^2(x) = \tau^2(y) \rightarrow \tau(x)=\tau(y).$$  We define $\tau^{-1}(\tau^2(x))$ as $\tau(x)$ and otherwise leave
$\tau^{-1}(y)$ undefined.  We define $\tau^{-(n+1)}(x)$ as $\tau^{-1}(\tau^{-n}(x))$ for positive numerals $n$.

\item[Observations:]  From the axiom of levels  it follows that if $u \in x$ and $v \in y$ and $x \sim_\tau y$
we have $x \in \tau^2(u)$ and $y \in \tau^2(v)$, so $\tau^2(u) = \tau(x)=\tau(y) =\tau^2(v)$ so $\tau(u) = \tau(v)$.

Further, it follows that if $x,y$ are sets and $x \sim_\tau y$ (and one of these sets is nonempty) that $\{z:z \in x \vee z \in y\}$ is a set (which we call $x \cup y$), since all of its elements are of the same type.
and further that for any set all of whose elements are sets and which contains a nonempty set, $\bigcup A = \{x:(\exists a \in A:x \in a)\}$ is a set, for the same reason.  The union of two empty sets or the union of a set of empty sets is defined in the obvious way.  Either the binary axiom of union or the general axiom of union implies the axiom of levels in the presence of the other axioms.

\item[Definitions:]  Define $\iota(x)$ or $\{x\}$ as $\{y:y=x\}$, which we know is a set.  Define $\iota^0(x)$ as $x$ as $\iota^{n+1}(x)$ as $\{\iota^n(x)\}$.  Define $I(x)$ as the intersection of all classes $C$ which contain $x$ as an element and satisfy $(\forall y:y \in C \rightarrow \{y\} \in C)$.  This can be called the class of iterated singletons of $x$.

\item[$^*$Axiom of hierarchy:] $$(\forall xy:(\exists z:z \in \tau(x) \wedge z \in I(y) \vee z \in\tau(y) \wedge z \in I(x))):$$  for any $x,y$, there is an iterated singleton of $x$ of the same type as $y$ or there is an iterated singleton of $y$ of the same type as $x$.

This axiom provides that the hierarchy of types looks exactly as we expect.  It might seem that a bottomless type hierarchy  is supported, but it is not:  we have enough machinery to prove that there must be base types.


\end{description}


\newpage

\section{Forcing}

We tend to see how forcing works through the lens of concepts developed (somewhat independenty?  I'm not sure of the history) for the model theory of constructive logic.  This sometimes causes our terminology to be nonstandard.  We also take a different approach to implementation of membership and equality which avoids mutual recursion between the definitions of forcing of membership and equality conditions, though this does have some other costs.

We also need a result from model theory.  By an inner model, we mean a model whose membership relation is a subset of the true membership relation.

\begin{description}

\item[Theorem:]  Let $T$ be any set theory with a transitive inner model which is a set (by inner model, I mean that the membership relation of the model is actually the membership relation of the real world). Then $T$ has a countable transitive model (ctm for short). 

\item[Proof:]  Start with any transitive inner model of $T$.  Put a well-ordering on the model, and define $(\epsilon x.\phi[x])$ as the first object $t$
such that $\phi[t]$ if there is one, and as $\emptyset$ otherwise (or some suitable default object in the model of $T$).  Start with the empty set as the first stage;  at each stage add all referents of Hilbert symbols $(\epsilon x.\phi(x,t_1,\ldots,t_n))$ where the values $t_i$ of the free variables are taken from previous stages, and all quantifiers are understood as bounded in $T$.  Go through $\omega$ stages.  The resulting structure is countable (because every object in it is represented by a finite string written in a finite or countable alphabet)  and its membership relation is well-founded and extensional but not necessarily transitive.   Take a Mostowski collapse and you get a model which again satisfies the same sentences but which is a countable transitive inner model of the original theory.

We can prove the existence of such a model of any theory which has a transitive inner model which is a set.  We cannot prove in ZFC itself
that ZFC has a transitive inner model which is a set, and so we cannot prove that ZFC has a ctm.  If there is an inaccessible cardinal, the existence
of such a transitive inner model follows, and so the existence of a ctm of ZFC (because if $\kappa$ is inaccessible, $V_{\kappa}$ is a transitive inner model of ZFC).  We also get such a model if we have a magic oracle which tells us what sentences are true in ZFC and a global order on ZFC;  these tools would allow us to define the Hilbert symbols and build a ctm of full ZFC.  Also note that we can definitely build a ctm of Zermelo set theory, because $V^{\omega\cdot 2}$ is a transitive inner model.

\end{description}

We now work inside a ctm of ZFC (however obtained) or inside a ctm of a suitable weaker theory.

Now for the logical concepts.  

I'm going to outline a semantics originally developed for constructive logic to motivate forcing.  Suppose we have a set $P$ with an order $\leq$ on it.
The elements of $P$ are called {\em conditions} and represent states of knowledge.  $p \leq q$ means that we have more information in condition $q$ than we do in condition $p$ (but compatible with condition $p$).  We say that conditions $p$ and $q$ are {\em compatible\/} iff there is a condition $r$ such that $p \leq r$ and $q \leq r$;  conditions which are incompatible represent states of knowledge which are inconsistent with one another.
We resist the usual habit of writing the partial order in the other sense:  most workers use $\geq$ where we use $\leq$.

\begin{description}

\item[splitting property:]  We require that $P$ have the {\em splitting property\/}:  for every $p \in P$, there are $q$ and $r$ such that $p \leq q$, $p \leq r$, and $q$ and $r$ are incompatible

\item[Definition (truth value sets):]  We call a subset $\tau$ of $P$ a {\em truth value set\/} if it satisfies two conditions:

\begin{enumerate}

\item For every $p \in \tau$, for every $q \geq p$, $q \in \tau$.

\item If $(\forall q \geq p: \exists r \geq q:r \in \tau)$, then $p \in \tau$.

\end{enumerate}

\item[forcing sets:]  We associate with each proposition $\phi$ a truth value set $[\phi]$:  we refer to $[\phi]$ as the set of conditions which force $\phi$ (so $p \in [\phi]$ can be read ``$p$ forces $\phi$" or less formally, ``$p$ asserts $\phi$").  The idea is that $[\phi]$ is the set of stages of knowledge at which we are able to determine that $\phi$ is true.
The first condition in the definition of truth value sets is thus motivated:  if we believe $\phi$ at the stage $p$, then we will still believe $\phi$ at any stage $q \geq p$.

\item[some clauses in the recursive definition of forcing sets:]  We regard our operations of propositional and first-order logic as being defined
in terms of negation, conjunction, and the universal quantifier.

\begin{enumerate}

\item The forcing set $[\neg \phi]$ is defined as $\{p \in P\mid (\forall q \geq p:q \not\in [\phi])\}$.  A condition forces $\neg \phi$ if neither that condition nor any stronger condition forces $\phi$.

The definition of forcing sets of negations motivates the second condition in the definition of truth value sets:  this condition is precisely what is needed to ensure that $[\neg \neg \phi] = [\phi]$.

\item The forcing set $[\phi \wedge \psi]$ is defined as $[\phi] \cap [\psi]$:  a condition forces $\phi \wedge \psi$ iff it forces both $\phi$ and $\psi$.

\item The forcing set $[(\forall x \in N:\phi[x])]$ is defined as $\bigcap_{x \in N}[\phi[x]]$, where $N$ is the set of names of objects in our domain of discourse.  The reason that we speak of names of elements of the domain of discourse rather than elements themselves is that equations between names will have nontrivial forcing sets:  there may be names for elements of our domain which under some conditions are names for the same object and under some conditions are names for different objects.  A condition forces $(\forall x \in N:\phi[x])$ iff it forces $\phi[x]$ for every name.

\end{enumerate}

\end{description}

The definitions of forcing sets for disjunctions and existential quantifiers follow from the definitions given and the natural definitions 
of disjunction and the existential quantifier in terms of negation, conjunction, and the universal quantifier, but they may be a bit unexpected.
$[\phi \vee \psi]$ is not $[\phi] \cup [\psi]$, which might not be a truth value set at all, but the smallest truth value set which contains this set.
For example, the forcing set $[\phi \vee \neg \phi]$ is  the entire set $P$, though certainly there will be conditions at which we will not have decided whether $\phi$ or $\neg\phi$ holds:  however, above any condition $q \geq p$ there is a condition $r$ at which either $\phi$ or $\neg \phi$ is asserted.

The semantics for $[(\exists x:\phi[x])]$ then follow from the definitions already given, but again might seem to require comment:  $p$ asserts
$(\exists x \in N:\phi[x])$ just in case for every condition $q \geq p$ there is $r \geq q$ such that $r$ asserts $\phi[t]$ for some name $t$ of an object in our domain of discourse, but it does not necessarily follow from $p \in [(\exists x \in N:\phi[x])]$ that there is any name $t$ such that $p \in [\phi[t]]$.

We will now introduce our domain of names.   We build a subclass of names of sets in our ctm of ZFC (or whatever theory).
We build the names in a hierarchy reminiscent of others we have seen.

\begin{enumerate}

\item $N_0 = \emptyset$

\item $N_{\alpha+1} = {\cal P}(N_{\alpha} \times P)$

\item $N_{\lambda}=\bigcup_{\beta<\lambda} N_{\beta}$ for $\lambda$ limit.

\end{enumerate}

So a name is a relation between names (of lower rank) and conditions.  The intention is that if $(x,p) \in y$ ($x$ and $y$ being names) that
$p \in [x \in y]$.  Things are trickier than that, though.

We define the rank of a name as the smallest $\alpha$ such that the name belongs to $N_{\alpha+1}$.

\noindent {\bf Note on restrictions on names which would tidy up this treatment}

The following additional restrictions on names would clear up many difficulties below.

\begin{enumerate}

\item Define $x_p$ (differently from below, without recursion, but for the names we construct with these restrictions the operation will be the same) as the set of
all $(y,q) \in x$ such that $q \geq p$.

\item Require that if $(x,p) \in y \in N$ and $q \geq p$, then $x_p=x$ and $(x_q,q) \in y$ as well.  Notice that $x_p = x$ ensures that if $(u,r) \in x$ we have $r \geq p$, and this further applies to $u$, to first projections of elements of $u$, and so forth.

\item Require that if $(x,p) \not\in y$ then there must be $q \geq p$ such that for all $r\geq q$, $(x_r,r) \not\in y$:  this ensures that the set of $p$ such that
$(x_p,p) \in y$ is a truth value set.

The actual effect of this note on the definition above is to restrict $N_{\alpha+1}$ to be not the entire set ${\cal P}(N_{\alpha} \times P)$ but the subset of this set determined by the restrictions just stated.

\end{enumerate}

The standard approach involves a mutual recursion between definitions of $[x \in y]$ and $[x=y]$.  It is sound, and even fairly easy to believe, but it is hard to demonstrate convincingly that one is not reasoning in a circle.  We take a different approach.

\begin{description}

\item[reducing names under a condition:]  For any name $x$ and condition $p$, we define $x_p$ as the set of all $(y_p,q) \in x$ such that $(y,p) \in x$ and $q$ is compatible with $p$.  Notice that this is a definition by transfinite recursion:  we can suppose when defining $x_p$ that we have already defined $y_p$ for all $y$ of rank less than the rank of $x$, and moreover that $y_p$ is of rank less than or equal to the rank of $y$, so we can establish by an induction parallel to the recursive definition that the rank of $x_p$ is less than or equal to the rank of $x$.

\noindent{\bf Note about restricted names:}  Notice that if we restrict names as in the note above, the recursive definition will be the same:  the only incompatible things removed will be actual elements of $x$;  elements of elements of $x$ which are not removed will themselves not be removed because their associated conditions will be compatible with $p$.  The entire development can proceed as below, and the recursion in the definition of $x_p$ is removed.

\item[weak membership and its forcing sets:]  $[$`$x$'$ \in_0 y]$ is the smallest truth value which contains each $p$ such that for some $q \leq p$ and $x'$ such that $x'_p=x_p$ (notice that this is an appeal to the literal equality of reduced names in the ctm, not a recursive appeal to the membership on elements of the domain for which we define forcing sets below), we have $(x',q) \in y$.   We will see below that $\in_0$ is a relation between names (as weak elements) and elements of our domain (as ``weak sets"):  it does not respect the membership relation on objects of the domain on the left, though it does on the right, as we will see in the next clause:  this is why we put the letter on the left in quotes.

\noindent{\bf Note on the situation with restricted names:}  If we use restricted names, this simplifies to ``$[$`$x$'$ \in_0 y]$ is the smallest truth value which contains each $p$ such that we have $(x_p,p) \in y$", and in fact the restrictions ensure that it simply is the collection of conditions $p$ such that $(x_p,p) \in y$.

\item[equality and its forcing sets:]  $[x = y]$ is defined as $[(\forall z\in N:$`$z$'$ \in_0 x \leftrightarrow$`$ z$'$ \in_0 y)]$.  For $p$ to be in this truth value set  means that for every $q \geq p$,
$x$ and $y$ have the same elements in the sense of $\in_0$ under condition $q$.  This definition makes it clear that we did not have to put $y$ in quotes in the definition of $\in_0$.

{\bf Note on the situation with restricted names:}  With restricted names, we have a much simplified situation.  If $p \in [x=y]$ we have for any condition $q \geq p$ that
$(z_q,q) \in x \leftrightarrow (z_q,q) \in y$ (because $z$ forces that $z$ is weakly in $x$ under condition $q$ iff it forces that $z$ is weakly in $y$) but then because of the restrictions on names we see that $x_p$ and $y_p$ then have exactly the same elements in the normal sense, so $x_p=y_p$.

\item[sethood and true membership and their forcing sets:]  The forcing set [${\tt set}(x)$] is defined as $[(\forall zw\in N. z=w \rightarrow $`$z$'$ \in_0 x \leftrightarrow $`$w $'$\in_0 x)]$.   The forcing set $[x \in y]$ is then defined as $[$`$x$'$ \in_0 y \wedge {\tt set}(y)]$.

It is possible for there to be names $x$, $y$, $z$ and a condition $p$ such that $p \in [y=z]$, $p \in [$`$y$'$ \in_0 x]$ and $p \in [\neg $`$z$'$ \in_0 x] = [$`$z$'$ \not\in_0 x]$.  We will then have that $p \in [\neg{\tt set}(x)]$, because $x$ contains one name for the element of the domain which $p$ says is named by both $y$ and $z$, and does not contain another name for the same object, under the given condition.  When an object $x$ is forced by condition $p$ to be a set,
this means that every condition $q \geq p$ which forces an equation $[y=z]$ forces $[$`$y$'$ \in_0 x]$ iff it forces $[$`$z$'$ \in_0 x]$.

{\bf Note on the situation with restricted names:}  This distinction collapses.  $p \in [{\tt set}(x)]$ always holds, and $[x \in y]$ is the same set as $[$`$x$'$ \in_0 y]$.

\item[name closures:]  For every name $x$, we can construct a name $x^*$ (called its ``name closure") such that the conditions $[{\tt set}(x^*)]$
and $[(\forall y \in N:$`$y$'$\in_0 x^* \leftrightarrow (\exists z\in N.y=z \wedge$`$ z$'$\in_0 x)]$ both are simply the set $P$.  The idea is that $x^*$ is obtained by fattening up $x$ to obtain a name which satisfies the closure condition defining sets, that names for the same object are always either both included or both excluded once they are forced to be names of the same object and either of them is forced into or out of the set.  The exact
way that this is achieved is that for each condition $p$ which forces $[y=z]$ and for which there is a condition $q \leq p$ such that $(y,q) \in x$, we
add $(z_p,p)$ to $x^*$.

Please note that the notation $x^*$ for the name closure of $x$ is only used in the preceding and following paragraphs:  stars are used for various different purposes in this chapter.

We discuss the reasons why this works.  First of all, we note a useful fact about $p \in [y=z]$.  Any $u$ which is forced to be a weak element of
$y$ by $p$ or a stronger condition $q$ is witnessed as such by an element $(u',r)$ of $y$ where $r$ is compatible with $p$ (resp. $q$),
and $u'_p=u_p$.  Now $u$ must also be forced to be a weak element of $z$ by the same condition, by an element $(u'',r')$ of $z$ where
$r'$ is compatible with $p$ (resp. $q$) and $u''_p = u_p$.  It follows from these considerations that $y_p$ and $z_p$ have the same first projections of elements:  for every $(u_p,r) \in y_p$, there must be a $(u_p,r') \in z_p$ (possibly with a different condition) and vice versa.  This means that
$y_p$ and $z_p$, though they may not be exactly the same sets, are of the same rank, and implies that the fattening process which builds $x^*$
does not increase rank, and so must succeed in defining a set.  Further, it should be clear that we do not add weak elements to $x^*$ which are not equal to some weak element of $x$:  any ``new" weak element $z'$ of $x^*$ will have $z'_p$ equal to one of the $z_p$'s for which $(z_p,p)$
was added by the fattening process (driven by the fact that $p \in [y=z]$ for some $y$) and then $z'=y$ will also be forced by $p$ (for the general reason that $y=z$ is forced by $p$ iff $y_p=z_p$ is forced by $p$).

\noindent{\bf Note on the situation with restricted names:}  Name closures are not needed.  All names are names for sets if restricted names are used.

\end{description}


As noted above, this membership relation respects equality on names.  We now of course have weak extensionality:  there are distinct atoms
with no elements.

We will eventually throw away the atoms.

\begin{description}

\item[Definition (filter in $P$, dense set in $P$):]  Now we are going to do black magic due to the fact that we are in a countable model.  A {\em filter\/} is a subset  $F$ of $P$ which represents an effort to decide what is actually true.  The defining conditions of a filter $F$ are that if $p$ belongs to $F$, and $q \leq p$ ($q$ represents less information than $p$) it follows that
$q$ belongs to $F$ as well.  If two elements $p$ and $q$ both belong to $F$, they are compatible:  there is $r$ such that $r \geq p$, $r \geq q$, and $r \in F$.  Now comes the black magic.   We call a subset $D$ of $P$ {\em dense\/} iff for every $p \in P$ there is $q \geq p$ which belongs to $D$.
A dense set of conditions is a set of conditions which is somehow impossible to avoid.   We have already provided that a dense truth value is actually true.

\item[Definition and construction of a generic filter in $P$:]  We construct a {\em generic filter\/} $G$, which is defined as a filter which meets every dense subset of $P$.  This will not be an object in our ctm.
We can build it, on the outside, by listing all the dense sets in a list $D_i$, choosing a condition $p_0$, and then at each step choosing $p_{i+1}$ so that it is greater than $p_i$ and
included in $D_i$ (since there is an element of $D_i$ greater than any element of $P$).  Further, of course, anything $\leq$ an element of $G$ will belong to $G$.  Notice that if two things belong to $G$, they belong because they are $\leq$ a $q_i$ and a $q_j$, and both will be $\leq$
the larger-indexed of these two.

$G$ is easily seen not to be in the ctm, because $P - G$ is dense (because we assumed the splitting property), and so cannot have been one of the $D_i$'s.

\end{description}

We now use $G$ to collapse each condition in $P$ in effect to a truth-value, thus collapsing names to sets (with some care).

\begin{description}

\item[$G$ makes our logic sensible:]  We claim that for every proposition $\phi$, either there is $p \in G$ which forces $\phi$ or there is $p \in G$ which forces $\neg \phi$.  The reason for this is that the set of $p \in P$ which either force $\phi$ or force $\phi$ is dense in $P$, and so must contain an element of $G$.  It is also the case that we cannot have a $p \in G$ which forces $\phi$ and a $q \in G$ which forces $\neg \phi$, as any two conditions in $G$ must be compatible.  We refer to this as the decision property:  $G$ must in a certain sense settle the truth value of each formula.

We deal further with the logic of the forcing model.  As we noted above, for every formula $\phi$ there is either $p \in G$ which forces $\phi$ or $p \in G$ which forces $\neg \phi$:  the collection of conditions which either force $\phi$ or force $\neg \phi$ is not the set of all conditions, but it is dense, so it contains some $p \in G$.  If we define $G \vdash \phi$ as
$(\exists p \in G:p \in [\phi])$, we see that $G \vdash \neg \phi$ is equivalent to $\neg(G \vdash \phi)$.  It should be clear that $G \vdash (\phi \wedge \psi)$ iff $(G \vdash \phi) \wedge (G \vdash \psi)$ and that $G \vdash (\forall x \in N.\phi[x])$ iff $(\forall x \in N:G \vdash \phi[x])$.  We will verify below that what $G$ says about atomic sentences comports with what happens in our model and so that $G$'s logic agrees exactly with ours.  It is particularly worth noting, though it does follow from what we have already said, that $G \vdash \phi \vee \psi$ does imply that either $G \vdash \phi$ or $G \vdash \psi$ holds:  if some $p \in G$ forces $\phi \vee \psi$, it does not necessarily force one of the disjuncts, but the set of $q \geq p$ which force one of the two disjouncts is dense, and so includes an element of $G$.  Similarly, if $G \vdash (\exists x.\phi[x])$, so there is $p \in G$ which forces $\phi[x]$, the set of $q \geq p$ for which there is a name $t$ such that
$G \vdash \phi[t]$ is dense above $p$, though it may not contain $p$, so it does contain some $q \in G$, so $G \vdash \phi[t]$ for some particular name $t$.

\item[Definition (equivalence of names):]  Define an equivalence relation $\sim_G$ on names by $x \sim_G y \leftrightarrow (\exists p \in G:x_p = y_p)$.  This is a modified version of the relation of literal identity between names, taking advantage of the generic filter to know which information is irrelevant and can be thrown away by reducing names.

\item[Definition (elements of the first approximation to our model):]  Define $\underline{x}$, for any name $x$, as the collection of all names $z$ such that for some $(y,p) \in x$, with $p \in G$, we have $z \sim_G y$, and moreover $z$ is of minimal rank in the collection of names $\{w \in N \mid w \sim_G y\}$ (that is, $z$ belongs to the lowest indexed set $N_{\alpha}$ with nonempty intersection with this set).

\item[weak membership in model elements defined and related to forcing:]  We define $z \in_0 \underline{x}$ as $(\exists y \in \underline{x}:z \sim_G y)$.  We claim that $z \in_0 \underline{x}$ iff there is $p \in G$ such that $p \in [$`$z$'$\in_0 x]$.
Suppose $z \in_0 \underline{x}$.  Then there is $u$ in $\underline{x}$ such that $z \sim_G u$, so there is $(y,q) \in x$ with $q \in G$ and $y_p=u_p=z_p$ for some $p \geq q$ also in $G$, and this establishes $p \in [$`$z$'$\in_0 x]$.  Now suppose that for some $p \in G$, $p \in [$`$z$'$\in_0 x]$.  This means that there is $(y,q) \in x$, with $q \leq p$,
such that $y_p=z_p$.  Now $q \in G$ because $G$ is a filter, and we see that $y \sim_G z$, from which it follows that both $y$ and $z$ stand in the relation $\sim_G$ to some element $u$ of minimal rank in their common equivalence class, which belongs to $\underline{x}$, so we have established that $z \in_0 \underline{x}$.

Note that we have shown that $G \vdash $`$z$'$\in_0 x$ iff $z \in_0 \underline{x}$.

\item[equality of model elements related to forcing:]  Now we claim that $\underline{x} = \underline{y}$ iff for some $p \in G$, $p \in [x=y]$.
\begin{enumerate}
 \item Suppose $\underline{x} = \underline{y}$.  We need to show that for some $p \in G$,
$p \in [x=y]$.  It appears easier to prove the contrapositive.  Suppose that for no $p \in G$ do we have $p \in [x=y]$.  The set of conditions $q$ such that $q \in [x=y]$ or $q \in \neg[x=y]$ is dense in $P$, so there must be $p \in G$ such that $p \in [\neg x=y]$ (decision property).   This means that $p$ forces $(\exists z:($`$z$'$ \in _0 x \wedge \neg$`$ z$'$ \in_0 y) \vee ($`$z$'$ \in _0 y\wedge \neg$`$ z$'$ \in_0 x))$, so for some stronger $q \in G$ we have a specific name $z$ such that either $q$ forces `$z$'$ \in_0 x$ and $q$ forces $\neg $`$z$'$ \in_0 y$, or vice versa, which implies $z \in\underline{x} \wedge z \not\in \underline{y}$, or vice versa, and in either case $\underline{x} \neq \underline{y}$ as desired.

\item Suppose that some $p \in G$ forces $x=y$.  We aim to show that $\underline{x}=\underline{y}$.  Suppose that $z \in_0 \underline{x}$.  It follows that for some $q \in G$,
$q$ forces $z \in_0 x$.  A condition $r$ stronger than both $p$ and $q$ will force $z \in_0 x$ and $x=y$, or equivalently $z \in_0 x \leftrightarrow z\in_0 y$, so in fact it also forces $z \in_0 y$ so $z \in \underline{y}$.  The argument is symmetrical so $\underline{x} = \underline{y}$.

\end{enumerate}

Note that we have shown that $G \vdash x=y$ iff $\underline{x} = \underline{y}$.

\item[sethood and membership proper defined for model elements:]  We now define ${\tt set}(\underline{x})$ as holding iff $(\forall zw: \underline{z} = \underline{w} \rightarrow (z \in_0 x \leftrightarrow w \in_0 x)$, and define
$\underline{x} \in_1 \underline{y}$ as holding iff $x \in_0 \underline{y} \wedge {\tt set}(\underline{y})$.   Arguments precisely similar to ones we have already given show that ${\tt set}(\underline{x})$ holds exactly if ${\tt set}(x)$ is forced by some $p \in G$, and $\underline{x} \in_1 \underline{y}$ holds iff there is $p$ in $G$ which forces $x \in y$.

We are claiming that $G \vdash {\tt set}(x)$ iff ${\tt set}(\underline{x})$ and $G \vdash x \in y$ iff $\underline{x} \in_1 \underline{y}$, completing the verification that $G$ assigns values to atomic sentences in a way which comports with our model.

{\bf with restricted names$\ldots$}  $\in_1$ will coincide with $\in_0$ and there will be no atoms.  $G \vdash {\tt set}(x)$ will always hold.

\item[membership of model elements is extensional:]  That the relation $\in_1$ is (weakly) extensional should be clear.  We have shown above that the model elements $\underline{x}$ are equal  iff they have the same names
as elements in the weak sense;  the same is true if we restrict our attention to the model elements treated as sets, and their extensions under $\in_1$ determine their extensions under $\in_0$, so sets with the same associated extension of model elements under $\in_1$ have the same associated extension of names under $\in_0$, and so are equal.  Model elements
$\underline{x}$ of which ${\tt set}(\underline{x})$ does not hold have no elements in the sense of $\in_1$:  they are treated as atoms.

{\bf with restricted names$\ldots$}  Membership of model elements will be strongly extensional.

\item[membership of model elements is well-founded:]  Now we need to show that $\in_1$ is a well-founded relation.  For any model element $\underline{x}$, we define the rank of $x$ as the smallest rank of a name $y$ such that
$\underline{x}=\underline{y}$.  Now suppose that $\underline{z} \in_1 \underline{x}$, where $x$ has the same rank as  $\underline{x}$. This implies that  $\underline{z} \in_0 \underline{x}$, from which we can conclude that for some $p \in G$ we have that $p$ forces `$z$'$ \in_0 x$, from which we can conclude that there is $u$ with $u_p=z_p$ (for which $p$ forces $u=z$) and $q \leq p$ such
that $(u,q) \in z$.  Now $\underline{u} = \underline{z}$ and the rank of $u$ is less than the rank of $x$, so the rank of $\underline{z}$ is less than the rank of $\underline{x}$, and we conclude that $\in_1$ is well-founded.

\item[the forcing model proper constructed by a modified Mostowski collapse:]  Thus we can carry out a modified Mostowski collapse sending each model element $\underline{x}$ to a new model element $\underline{x}^*$ with $\underline{x}^* = (\underline{x},\omega_1)$  for each atom  $\underline{x}$ (that is the real uncountable $\omega_1$, not in our ctm), and $\underline{x}^*$ defined (by transfinite recursion on rank) as $\{\underline{y}^* \mid \underline{y} \in_1 \underline{x}\}$. when ${\tt set}(\underline{x})$.  Notice that we will have $\underline{x}^* = \underline{y}^*$ iff $\underline{x} = \underline{y}$, by an induction parallel to the recursion in the definition.  The use of the real $\omega_1$ is a technical device to prevent the accidental construction of the set implementing an atom as the set implementing some set as well.

In what follows, our model elements are the objects $\underline{x}^*$, but we will refer to an object of our model  $X=\underline{x}^*$ as the object $X$ with name $x$ (and of course many other names as well).   We refer to $X$ as the ``collapse" of $x$.  We then free up the star to mean something else, as we will no longer use it to refer to the product of the Mostowski collapse carried out here.

{\bf with restricted names$\ldots$}  There are no atoms so the Mostowski collapse is unmodified, and the curious case with the true $\omega_1$ involving atoms does not occur.

\end{description}



We then hope to restrict our attention entirely to the pure sets, those whose transitive closures contain no atoms.  {\bf with restricted names$\ldots$}  there is no need to do this.

As we will see, this process of collapse will sometimes produce sets which were not in our original ctm (certainly atoms are not in the ctm, nor is any set with an atom in its transitive closure:  the question is what pure sets are present).  Copies of all sets in the original ctm
will be present:  if we have names $\hat{y}$ for all elements of a set $x$ in the original ctm, define $\hat{x}$ as the name closure of
$\{\hat{y} \mid y \in x\} \times P$, and $\hat{x}$ will name $x$.

We want to verify that all axioms of ZFC actually hold in the new structure (with the qualification that extensionality is weakened).

\begin{description}

\item[Extensionality:]  If $x$ and $y$ have elements and have the same elements in the new structure, they are certainly equal.  I have a feeling there is more to be said about this.

\item[Pairing:] If $x$ and $y$ have names $x^*$ and $y^*$, take the name $(x^* \times P) \cup (y^* \times P)$, close it up so it names a set, and it will collapse to $\{x,y\}$.

\item[Union:]  If $x$ has a name $x^*$, define a name $z^*$ for the union of the object named by $x^*$ as follows:  for each $(y^*,p)$ in $x^*$
and each $(u^*,q) \in y^*$, $z^*$ must contain each $(u^*,r)$ such that $r \leq p$ and $r \leq q$;  $z^*$ is the result of applying name closure to the smallest relation compatible with these conditions.  The collapse of $z^*$ will be the union of the collapse of $x^*$.

\item[Power Set:]  Let $x^*$ be a name for $x$.  Construct all possible  names $y^*$ relating each element of the domain of $x^*$ to a possibly proper subset of the upward closure of the set of elements of $P$ to which $x^*$ relates it.  Let $Y$ be the set of such names.  Take the name $Y \times P$ and close it up if necessary so that it names a set.  This will name the power set of $x$.

\item[Separation:]  To build $\{x \in A \mid \phi(x)\}$,  relate each name $x^*$ in the domain of $A^*$ to the intersection of $[x^* \in A^*]$ and $[\phi[x^*]]$.  Close the result so that it names a set. 

\item[Infinity:]  $\hat{\omega}$.

\item[Choice:]  For any name $x^*$, build a name for a map from an ordinal $\alpha$ or an initial segment thereof to the elements of $x$:  well-order the domain of $x$ and let $\alpha$ be the order type of this well-ordering.  Build a name for a function by building a name for a set of ordered pairs, sending each $y^*$ to $\beta<\alpha$ (using the name $\hat{\beta}$ for each such ordinal) under any condition under which each preceding domain element of $x$ has either been shown not to be in $x$ or already mapped to some $\gamma<\beta$.   Close the name computed so that it is the name of a set.

\item[Replacement:]  To compute the image of a set with name $A^*$ under a putatively functional formula $\phi[x,y]$, take as your domain all names
$y^*$ of minimal set theoretic rank such that for some condition $p$, $p$ says that $\phi[x,y]$ is functional and $p$ says that $\phi[x^*,y^*]$, for some $x^*$ such that $p \in [x \in A]$, and relate each $y^*$ to each element of the smallest truth value set containing all such conditions.  There is some trickiness about seeing that Replacement in the original ctm ensures that the name is actually a set:  the trick is contained in the requirement that we choose names of minimal rank.  Close the name up if necessary to ensure that it names a set.

\end{description}

The new structure is still countable, because it is no larger than the collection of names, which is a subset of the original ctm.  For every object
$x$ in the original ctm, there is a name $\hat{x}$ defined as the name closure of $\{\hat{y}\mid y \in x\} \times P$, which in fact names the original set $x$.  There is at least one
new object in the structure, namely the collapse of the name closure of $\{(\hat{p},p)\mid p \in P\}$, which contains $p$ just in case $p \in G$.  This is precisely the set $G$, which we know was not in the original ctm (it cannot be, because $P \setminus G$ is dense in $P$ by the splitting property, and $G$ meets every dense subset of $P$ in the ctm).

Finally, we can expunge the atoms because we can define what it means to be an atom (not equal to $\emptyset$ and having no elements)
and then we can define what it means to be a pure set (having no atoms in one's transitive closure), and restrict our domain to pure sets.
The domain of pure sets will still satisfy all the axioms, and thus will be a countable transitive model of ZFC.

\subsubsection{Exercises}

\begin{enumerate}

\item Please carefully read the section on forcing and communicate with me about any typos or errors you find and any points that confuse you.

\item  Prove that for any formulas $\phi$ and $\psi$ and condition $p$, $p$ forces $\phi \rightarrow \psi$ if and only if $(\forall q \geq p : q \in [\phi] \rightarrow q \in [\psi]\}$.

\item Draw a picture of conditions arranged in a finite diagram (with the relation $\leq$ between conditions indicated) with indications of where propositions are forced satisfying the following perhaps unexpected conditions.  These may {\em all\/} be things I actually sketched on the board.

\begin{enumerate}

\item  A condition $p$ (with incompatible conditions $q$ and $r$ pictured above it) forces $\phi \vee \psi$ without forcing either $\phi$ or $\psi$ (next to each condition in your diagram, indicate which atomic statements it forces).

\item  A condition $p$ (with incompatible conditions $q$ and $r$ pictured above it) forces $(\exists x \in N:\phi[x])$ while there is no name $y$ such that
$p$ forces $\phi[y]$ (again, label each condition with the set of atomic statements that it forces:  you may leave the predicate $\phi[x]$ completely abstract and use arbitrary names like $a,b$.

\item  In this part you have to reason about names.  Give a diagram of conditions and actual names $x,y$ (presented as subsets of $N \times P$:  the names in the domains of $x$ and $y$ may be letters $a,b$) such that incompatible conditions $q$ and $r$ force $x=y$ and $x \neq y$ respectively.

\item  Give an example of a name $x$ and a condition $p$ such that $p$ forces $a \in x$ ($a$ can just be a letter) but neither $p$ nor any condition weaker than $p$ is in the range of $x$
(there is no pair of the form $(u,p)$ or even of the form $(u,q)$ with $q \leq p$ belonging to $x$).

\end{enumerate}

\item  Prove that if $x.y \in N$, and $x_p=y_p$, and $q \geq p$, then $x_q = y_q$.  Notice that this proof will involve transfinite induction:  you prove this assertion for $x$ and $y$ with the maximum of the ranks of $x$ and $y$ being $\alpha$ under the assumption that it is known to be true for all $u,v$ with the ranks of both $u$ and $v$ less than $\alpha$.  If you get after me, I'll try to prove some result by a similar induction in class.

{\bf Partial Solution}:  Since I had the definition of $x_p$ wrong in the notes, I'll give a strong hint on this one.  To begin with, we might want to make sure that
the notation $x_p$ is well-defined.  We do this by defining a notation $x_p^{\alpha}$ decorated with an ordinal rank.  If $x \in N_{\alpha}$,
we define $x_p^{\alpha}$ as the set of all $(y_p^{\beta},q)$ such that $\beta<\alpha$, $y \in N_{\beta}$, $(y,q) \in x$, and $q$ is compatible with $p$.  This is more clearly a definition by transfinite recursion:  notice that for any $(y,q) \in x$, $y$ has to belong to an $N_{\beta}$ with $\beta<\alpha$, so we have already defined $y^{\beta}_p$.

We argue by transfinite induction on $\alpha$ that if $x \in N_{\alpha}$, then $x^{\gamma}_p = x^{\alpha}_p$ for all $\gamma>\alpha$:
Suppose $(z,q)$ belongs to $x^{\alpha}_p$,  Then $q$ is compatible with $p$, and for some $y$, $z=y_p^{\beta}$ with $\beta<\alpha$ and $y \in N_{\beta}$ and $(y,q) \in x$.  Note, though, that $\beta<\alpha$ implies $\beta<\gamma$ as well, so $(z,q) \in x^{\gamma}_p$ on the same evidence.  The converse is slightly harder.  Suppose $(z,q)$ belongs to $x^{\gamma}_p$.  It follows that $q$ is compatible with $p$
and for some $\beta<\gamma$ and $y \in N_{\beta}$ we have $z=y^{\beta}_p$ and $(y,q) \in x$.  Now, because $(y,q) \in x$, we actually
have $y \in N_{\beta'}$ for some $\beta'<\alpha$ (not merely $<\gamma$).  By inductive hypothesis we have $y^{\beta'}_p = z= y^{\beta}_p$,
and so we also have the evidence required that $(z,q) \in x^{\alpha}_p$.

I'll lecture the full solution on Wednesday (and put it here in the notes).  For the moment, I'll give you a hint.   You now have everything
you need to prove by transfinite induction that if $q \geq p$, then $(x_p)_q = x_q$.  Prove this by transfinite induction on ranks of names, and the result claimed above follows immediately.

\item Prove that if $x_p = y_p$ for any condition $p$, it follows that $p \in [x=y]$, and further that if $p \in G$ in addition it follows that $\underline{x} = \underline{y}$.  I believe that the result of the previous problem will be used in this argument.

\item (challenge problem, I have no idea if this is reasonable):  Suppose that a condition $p$ forces the condition that $x^*$ is the name of a partition.
Try to construct a name $y^*$ such that the same condition $p$ will force $y^*$ to be a choice set for $x^*$.

\end{enumerate}

\subsection{Independence of CH}

Now build the classic example.  We construct a model which believes that there is a set of real numbers of size $\aleph_2$, demonstrating independence of the Continuum Hypothesis.

Our partial order will be the inclusion relation (not the inverse inclusion relation, as is usually stated, because we use the converse order from usual presentations;  this is purely a technicality) on the set of finite functions from subsets of $\omega_2 \times \omega$ to $\{0,1\}$.

The union of a generic filter $G$ on this set will be a function from $\hat{\omega_2} \times \omega$ to $\{0,1\}$ (where $\hat{\omega_2}$ here means the fake $\omega_2$ of the countable transitive model, which is actually some countable ordinal).  To see this, observe that each element of the generic filter
will be a finite function from a subset of $\hat{\omega_2} \times \omega$ to $\{0,1\}$, and for any $\alpha \in \hat{\omega_2}$ (the fake $\omega_2$) the set $D$ of conditions
$p$ such that $(\alpha,m)$ is in the domain of $p$ is dense (because any condition at all can be extended with $((\alpha,m),0)$ or $((\alpha,m),1)$ as desired, so $G$ meets $D$, and of course all elements of $G$ must agree on which of 0 or 1 is supplied as a value, because any two elements of $G$ must be compatible).

 For each $\alpha \in \hat{\omega_2}$, there will be a subset
$r_{\alpha}$ = $\{m \in \omega \mid ((\alpha,m),1) \in \bigcup G\}$, and the forcing model will contain a function sending each $\alpha\in \hat{\omega_2}$ to $r_{\alpha}$:  a name for this is easy to construct -- put a pair $(\alpha,m)$ into the name under condition $p$ (that is, add a pair $(\alpha,p)$ as an element of the name) just in case $((\alpha,m),1)\in p$.

All the $r_{\alpha}$'s are different:  if $p$ is any condition and $\alpha \neq \beta$ are elements of $\hat{\omega_2}$, $p$ can always be extended with 
$((\alpha,n),0)$ and $((\beta,n),1)$ for a large enough $n$, so the set of conditions under which the forcing model must put some number $n$ into $r_{\alpha}$ and exclude it
from $r_{\beta}$ is dense, and so such a condition must belong to $G$.

Thus the forcing model contains a bijection from $\hat{\omega_2}$ to the natural numbers, which would seem to imply that we were done.  However, there is a little more work to do.
The problem is that it is not obvious that $\hat{\omega_2}$ is actually $\omega_2$ in the opinion of the forcing model (as it is in the option of the original ctm).

We briefly describe a different forcing model to show that there is a real issue.  Use the partial order of inclusion on finite injective maps from subsets of
$\omega$ to subsets of $\hat{\omega_2}$.  Let $H$ be a generic filter in this partial order.  The union of $H$ will be a bijection from $\omega$ to $\hat{\omega_2}$, and so the resulting forcing model will think (correctly, unlike the original ctm) that $\hat{\omega_2}$ is a countable ordinal!

We say that a partial order $\leq$ on $P$ (in the original ctm) satisfies the ccc (countable chain condition:  the name is traditional, but an error:  it is really the countable antichain condition) iff there is
no uncountable set of mutually incompatible elements in the field of $\leq$.  We show that if forcing under $\leq$ creates a bijection between distinct cardinals in 
the original ctm, then $\leq$ fails to satisfy the ccc in the original ctm.

Let $\kappa < \lambda$ be infinite  cardinals, and let $f$ be a name in the forcing model produced from $P$ for a bijection from $\kappa$ to $\lambda$.  $\lambda$ is uncountable 
in the original ctm.  For each $\alpha < \lambda$, there must be a condition $p$ which forces for some $\beta<\kappa$ the assertion $f(\beta) = \alpha$.   Now we use the Pigeonhole Principle.  For each $\alpha$ in $\lambda$ there are one or more $\beta$'s in $\kappa$ which work.  Because $\lambda$ is greater than $\kappa$, there must be a specific $\beta$ which works for $\lambda$ distinct $\alpha$'s.   But then there are conditions $p_{\alpha}$ for $\lambda$ distinct $\alpha$'s such that for one and the same $\beta$,
$p_{\alpha}$ forces $f(\beta) = \alpha$, and these $p_{\alpha}$'s make up a pairwise incompatible collection of conditions, violating the ccc.

It follows that we can show that $\hat{\omega_2}$ is the $\omega_2$ of our original forcing model (because the forcing model will have the same cardinals as the original ctm) if we can show that the partial order we started with has the ccc.

Suppose that we have an uncountable set of incompatible conditions in our original partial order and argue to a contradiction.

Choose a single condition $p_0$ (partial function from a subset of $\omega_2 \times \omega$ to $\{0,1\}$) in the uncountable mutually incompatible set of conditions.
Each pair of conditions in this set has associated with it the subset of $\omega_2 \times \omega$ on which both conditions have values and the values disagree.

For the single condition $p_0$ we have chosen, there are only finitely many possible values for this distinguished subset in relation to any other condition, and there are countably many values for the size of the other condition.  So there is an uncountable collection $B_0$  of conditions in the uncountable mutually incompatible set which have the same set of disagreement
in relation to the single condition $p_0$ chosen initially and which are all the same size.

So we now have a mutually incompatible uncountable set of conditions all of which agree on a certain finite subset of their domains, and all of which are the same size as sets.
We can choose a single element of this mutually incompatible set and repeat the process:  choose an element $p_1$ of $B_0$ and construct a subset $B_1$ of $B_0$ all elements of which are the same size (in fact the common size of elements of $B_0$) and all of which disagree with $p_1$ on the same nonempty finite set.  However, this will fail after finitely many steps, because when we construct each $p_n$, we discover a new set $B_n$ all of whose elements disagree with each of the $p_i$'s constructed so far and agree with other
on a finite subset of their domains, larger at each step, and all elements of each $B_n$ are of the same size (the common size of all the elements of $B_0$).

Thus our partial order satisfies the ccc, and our forcing model has at least $\omega_2$ distinct subsets of the natural numbers, and so has $\omega_2$ distinct real numbers, so CH fails there.

\newpage

\section{Independence of Choice}

In this section, we will explicitly show that the axiom of choice is independent of ZFC if atoms are allowed.

The modified set theory ZFCA we work in has Extensionality weakened to allow atoms.  Further, we assert that there is a set $\mathbb A$ of all atoms.  The other axioms are as before (including choice).

It is straightforward to interpret ZFCA with the set of atoms of any desired size in ZFC.  Choose a set $A$.  We will redefine the membership relation $\in$ to give a new relation $\in'$ under which the axioms of ZFCA will hold, and all and only the elements of $A \times \{0\}$ become atoms.  The definition is $x \in' y$ iff either $y \not\in A \times \mathbb N$ and $x \in y$ or $y = (a,n+1)$ for some $a \in A$ and $n \in \mathbb N$ and $x \in (a,n)$.   We leave it to the reader to work out the details.

In ZFCA, there is a cumulative hierarchy as there is in ZFC:

We give a definition by transfinite recursion of the hierarchy in ZFCA:

\begin{enumerate}

\item $V_0 = \mathbb A$.

\item $V_{\alpha+1} = V_{\alpha} \cup {\cal P}(V_{\alpha})$.  (this definition preserves the idea that the levels of the hierarchy are cumulative.  If we simply took power sets, we would for example first construct sets with some elements sets and some elements atoms at level $\omega$, which would be odd.)

\item $V_{\lambda} = \bigcup_{\beta<\lambda}V_{\beta}$ for $\lambda$ limit.

\end{enumerate}

We are interested in bijections from $\mathbb A$ to $\mathbb A$, which we refer to as permutations of the atoms.  For each permutation $\pi$, we indicate how to extend the definition of the notation $\pi(A)$ to sets $A$.   We do this by defining a sequence $\pi^{\alpha}$ of functions by transfinite recursion:

We define $\pi^0$ as the permutation $\pi$ itself, considered as a map from $V_0$ to $V_0$.

When we have defined $\pi^{\beta}$ for each $\beta<\alpha$, a permutation of $V_{\beta}$, and moreover we have that each pair $\pi^{\beta}, \pi^{\gamma}$ for $\beta<\gamma<\alpha$ agree on the intersection $V_{\beta}$ of their domains, we indicate how to define $V_{\alpha}$.

If $\alpha$ is a successor $\delta+1$. we define $\pi^{\alpha}(A)$ for each $A \in V_{\alpha}$ as $\pi^{\delta}``A$ on ${\cal P}(V_{\delta})-V_{\delta}$ and as $\pi^{\delta}(A)$ for $A \in V_{\delta}$.   

If $A$ is in the intersection of $V_{\delta}$ and ${\cal P}(V_{\delta})$, $A$ must belong to some $V_{\epsilon+1}-V_{\epsilon}$ where $\epsilon<\delta$, and $\pi^{\epsilon+1}(A) = \pi^{\epsilon}``A$ agrees with $\pi^{\delta}``A$ and with $\pi^{\delta}(A)$, in both cases by the inductive hypothesis on agreement of functions with index less than $\alpha$.

 It is then clear, since $\pi^{\delta}$ is a permutation of $V_{\delta}$,
that $\pi^{\alpha}$ is a permutation of $V_{\alpha} = V_{\delta+1}$.  

 We also need to prove that $\pi^{\alpha}$ agrees with each $\pi^{\beta}$ for $\beta<\alpha$.  Suppose this were not the case.  Then there would be a minimal $\beta$ such that
$\pi^{\beta}$ disagreed with $\pi^{\alpha}$ at some $x$, and for this value of $\beta$, a minimal $\gamma$ such that such an $x$ could be found in $V_{\gamma+1} - V_{\gamma}$ (it is quite clear that such an $x$ will not be an atom).  But it then follows that $\pi^{\alpha}$ (and $\pi_{\delta}$)  agrees with $\pi_{\gamma}$ and $\pi^{\beta}$ on all the elements of $x$, whence $\pi^{\alpha}(x) = \pi^{\delta}``x = \pi^{\gamma}``(x) = \pi^{\gamma+1}(x) = \pi^{\beta}(x)$ by the way the functions are defined and the inductively hypotheses about agreement.

Now if $\alpha$ is limit, we define $\pi^{\alpha}(x)$ for each $x \in V_{\alpha}$ as the common value of all $\pi^{\beta}(x)$'s for $x \in V_{\beta}$ (common by inductive hypothesis).
This defines a permutation of $V_{\alpha}$ because we know by inductive hypothesis that each of the $\pi_{\beta}$'s included in it is a permutation of $V_{\beta}$.  It agrees with all lower indexed maps directly by the way it is defined.

We can then define $\pi(A)$ (admittedly an abuse of notation) as $\pi^{\alpha}(A)$ for any $\alpha$ such that $A \in V_{\alpha}$.  We have thus converted the set of permutations
of the atoms into a set-sized collection of class permutations acting on the entire universe.

Observe that for any $x$ and $y$, $x = y \leftrightarrow \pi(x)=\pi(y)$ and $x \in y \leftrightarrow \pi(x) \in \pi(y) = \pi``y$.  Further observe that $(\forall x:\phi[x])$ is equivalent to $(\forall x:\phi(\pi(x)))$ (and ditto for existential quantifiers) because $\pi$ acts as a permutation on the entire universe.

From this it follows for any formula $\phi[x,a_1,\ldots,a_n]$ with all free variables listed in the vector that  $\phi[x,a_1,\ldots,a_n] \leftrightarrow \phi[\pi(x),\pi(a_1),\ldots,\pi(a_n)]$.
From this it follows that if $\{x \mid \phi[x,a_1,\ldots,a_n]\}$ exists, then $\pi(\{x \mid \phi[x,a_1,\ldots,a_n]\}) = \{\pi(x) \mid \phi[x,a_1,\ldots,a_n]\} = \{\pi(x) \mid \phi[\pi(x),\pi(a_1),\ldots,\pi(a_n)]\} = \{x \mid \phi[x,\pi(a_1),\ldots,\pi(a_n)]\}$ (the last move exploiting the fact that $\pi$ acts as a permutation of the entire universe).

We define a subclass of the universe relative to any group $G$ of permutations of the atoms.  We say that an object $x$ has support $S$, a finite set of atoms, iff every permutation $\pi \in G$ such that for each $s \in S$, $\pi(s)=s$, also satisfies $\pi(x) =x$.  We claim that for any group $G$ of permutations, the class of sets (and atoms) with support satisfies all the axioms of ZFCA except possibly Choice.

That Extensionality holds is evident.  The reasons why all the other axioms hold are exactly the same.  Each such axiom asserts the existence of a more or less complicated set
$\{x \mid \phi[x,a_1,\ldots,a_n]\}$, with parameters $a_1,\ldots,a_n$.  Each parameter $a_i$ (being taken from our subclass) has a support $S_i$.  The union of the sets $S_i$ is finite, and any permutation $\pi$ in $G$ which fixes each element of $\bigcup_{1 \leq i \leq n}S_i$ will fix each $a_i$ and so by the formula $\pi(\{x \mid \phi[x,a_1,\ldots,a_n]\}) = \{x \mid \phi[x,\pi(a_1),\ldots,\pi(a_n)]\}$ proved above will fix $\{x \mid \phi[x,a_1,\ldots,a_n]\}$.

Now we show that choice does not hold in some specific models of this kind.  Let $G$ be the entire group of permutations of the atoms.  Let $\mathbb A$ be infinite.  A set
$A \subseteq {\mathbb A}$ will have finite support $S$ in the set of all permutations of the atoms only if it is finite or if ${\mathbb A} - A$ is finite.  So this model of ZFA contains a
set $\mathbb A$ which it believes to be infinite and to have no subset $A$ such that both $A$ and $\mathbb A$ are infinite.  It is straightforward to prove in ZFC or ZFCA that an infinite set must have a subset of size $\omega$, which in turn has two disjoint infinite subsets, so choice fails in this model.

We introduce another model to implement Russell's famous example of infinitely many pairs of socks.  Let the set of atoms $\mathbb A$ support a partition $P$ into infinitely many two element sets (the atoms are socks and the elements of $P$ are pairs).  Let $G$ be the set of all permutations of $\mathbb A$ whose action fixes each element of $P$ (each permutation  may fix each sock or exchange it with its mate).  We claim that no choice set $C$ for $P$ can have finite support with respect to these permutations.  Suppose it did have a support $S$.  $S$ is a finite set, so there is a pair of socks $p=\{a,b\}$ which is disjoint from $S$.  The permutation which sends $a$ to $b$, $b$ to $a$, and fixes every other sock belongs to the group $G$, fixes each element of $S$, but moves $C$, so $S$ was not a support of $C$:  this contradiction makes our point.

The pairs of socks example is nice because it actually gives us an explicit failure of the Axiom of Choice as usually written:  we actually have a partition $P$ of the atoms which is in the model ($P$ is fixed by the action of every permutation in $G$, so it has support the empty set!) which does not have a choice set in the model.  It is less obvious how to get
a partition without a choice set in the first model where we used all permutations of the atoms (it is not going to be a partition of the set of atoms, because the set of atoms
has no infinite partitions in that model!).

Unfortunately, we have not shown that Choice fails in ZFC by this argument.  The problem is that the well-founded sets of ZFC (without the wiggle room afforded by atoms) are a rigid structure:  we have no way to define a permutation $\pi$ of the universe such that $\pi(A) = \pi``A$ for every $A$ which is not simply the identity.

We will outline vaguely how to do this using forcing technology.  In fact, we can use the exact model we used to establish the independence of Choice.  The key is to use permutations of the partial order $P$ used in our forcing to define a notion of support for {\em names\/}, and to allow only names with support in our forcing model.  The partial order $P$ used in the construction above was the inclusion order on finite subsets of $\omega_2 \times \omega$.  Our group of permutations acts on the $\omega_2$ columns of the set $\omega_2 \times \omega$ (in effect permuting the $\omega_2$ generic subsets of $\omega$  we were constructing in the original construction, without permuting the other dimension which tells which natural numbers belong to each of the subsets of $\omega$).  A name will have support $S$ (a finite subset of $\omega_2$)
of any permutation of the columns which fixes all the elements of $S$ fixes the name.  It is technically exciting to show that our forcing constructions go through if we only allow use of names with support in this sense.  In the forcing model we end up with, we have certainly not added $\omega_2$ reals, but we have added a large set of reals on which it is impossible to put a well-ordering (it seems quite reasonable that it would be difficult to produce a symmetric name for such an ordering, at any rate), so choice fails.  It will not be the case that this set of reals is as formless as the set of atoms in our first construction of atoms:  even without choice, one can show that there cannot be an infinite set of reals which cannot be partitioned into two infinite sets:  being a set of reals, this collection certainly has a linear order (since the reals do!), which the set of atoms in the first model construction above cannot have.  I want to say more about this$\ldots$

\subsubsection{Exercises}

\begin{enumerate}

\item  Show without the Axiom of Choice that any finite partition has a choice set.  Then show in the first model above that any partition of the set of atoms has a choice set.

A serious challenge is to present a set and a partition of that set in the first model (using all permutations of the infinite set of atoms) which does not have a choice set.  It is possible to reverse engineer what the set and the partition should be from a proof of the Well-Ordering Theorem or Zorn's Lemma$\ldots$

\item  Show that the set of atoms in each of the two models we have presented cannot be linearly ordered.  The strategy is to suppose that there is a linear order realized by a set in our model, choose a finite support of this set (which must exist by definition of the model) then show that in fact the support$\ldots$cannot be a support.

A challenge:  produce a model in the same way (using a different group of permutations $G$) which gives a model in which the atoms are linearly ordered but not well-orderable.
I wouldn't regard it as implausible that you could come up with the right group of permutations:  showing that it would kill a well-ordering might be hard.

\item Show that the construction of a model of ZFCA from a model of ZFA at the beginning of the section works.  Again, this is a serious challenge.   See if you can verify a few axioms.

\item Prove in ZFA that if a definable class permutation of the universe $\pi$ has the property $\pi(A) = \pi``A$ for every set $A$, then it is the identity permutation:  $\pi(A) = A$ for every set $A$.

\end{enumerate}

\newpage

\section{$\dagger$ Bridges from untyped set theory to typed set theory}

This subsection introduces relationships between the untyped theory of sets developed above and the typed theory of sets developed previously.

\subsection{$\dagger$The intended interpretation of Zermelo set theory in set pictures; the Axiom of Rank; transitive closures and Foundation}

Our intention in this section is to show how Zermelo set theory can be
interpreted in subsets of the set $Z$ of set pictures with the
relation $E$ standing in for membership, and to observe that when
Zermelo set theory is implemented in this way certain additional
axioms hold which make the system easier to work with.

Any sentence of the language of untyped set theory can be translated
into a sentence of our type theory by replacing each occurrence of
$\in$ with the relation $E$ and bounding each quantifier in the set
$Z$ (all in some fixed type).  In fact, instead of bounding it in $Z$,
we bound it in ${\mathbb E}_{\lambda}$, where $\lambda \geq
\omega\cdot 2$ is a limit ordinal.  We assume that each rank below
rank $\lambda$ is complete, so we are assuming at least the existence
of $\beth_{\omega}$.

We claim that (under the assumption that all types below $\lambda$ are
complete), the translations of the axioms of Zermelo set theory into
the language of type theory are true, so we have a way to understand
untyped set theory in terms of our type theory.

\begin{description}

\item[Extensionality:] Sets with the same elements are the same.  Zermelo allowed atoms (non-sets) in his original
formulation, and we allow for that possibility in the previous presentation of his axioms, but we will assume here that all objects are sets.

\item[Verification of Extensionality:] This follows from the fact that
$E$ is a membership diagram, and so an extensional relation (and the
fact that $E$ end extends the restriction of $E$ to any ${\mathbb
E}_{\lambda}$; the preimage of any element of the field of the
restriction under the restriction is the same as its preimage under
$E$ itself, so extensionality of $E$ implies extensionality of the
restriction.

\item[Elementary Sets:] The empty set $\emptyset$ exists.  For any
objects $x$ and $y$, $\{x\}$ and $\{x,y\}$ are sets.

\item[Verification of Elementary Sets:] The equivalence class of the
empty set diagram belongs to ${\mathbb E}_{\lambda}$ and has empty
preimage under $E$, so satisfies the translation of the properties of
the empty set.  Let $a, b \in {\mathbb E}_{\lambda}$.  $a \in {\mathbb
E}_{\alpha}$ and $b \in {\mathbb E}_{\beta}$ for some $\alpha,\beta <
\lambda$.  Because $\lambda$ is limit, $\max(\alpha,\beta)+1 <
\lambda$, and since ${\mathbb E}_{\max(\alpha,\beta)}$ is a complete
rank, $\{a,b\}$ has an $E$-code in ${\mathbb E}_{\max(\alpha,\beta)+1}
\subseteq {\mathbb E}_{\lambda}$

\item[Separation:] For any property $\phi[x]$ and set $A$, the set $\{x
\in A \mid \phi[x]\}$ exists.

\item[Verification of Separation:] Any $A\in {\mathbb E}_{\lambda}$
belongs to some rank ${\mathbb E}_{\alpha+1}$ for $\alpha<\lambda$
(every element of $Z$ first appears in a successor rank).  The formula
$\phi[x]$ translates to a formula $\Phi[x]$ in the language of type
theory.  The set $\{x \in {\mathbb E}_{\alpha} \mid x \,E\, A \wedge
\Phi[x]\}$ exists by comprehension in type theory and has an $E$-code
in ${\mathbb E}_{\alpha+1}$ because ${\mathbb E}_{\alpha}$ is a
complete rank.

\item[Power Set:] For any set $A$, the set $\{B \mid B \subseteq A\}$
exists.  The definition of $A \subseteq B$ is the usual one.

\item[Verification of Power Set:] Any $A\in {\mathbb E}_{\lambda}$
belongs to some rank ${\mathbb E}_{\alpha+1}$ for $\alpha<\lambda$
(every element of $Z$ first appears in a successor rank).  $\alpha+1$
and $\alpha+2$ are also less than $\lambda$ because $\lambda$ is
limit.  The translation of $B \subseteq A$ asserts that the
$E$-preimage of $B$ is a subset of the $E$-preimage of $A$.  Each $B$
whose $E$-preimage is a subset of the $E$-preimage of $A$ also belongs
to ${\mathbb E}_{\alpha+1}$ (because each element of its $E$-preimage
belongs to ${\mathbb E}_{\alpha}$ and ${\mathbb E}_{\alpha}$ is
complete), so the set of all such $B$ has an $E$-code in ${\mathbb
E}_{\alpha+2}$, because ${\mathbb E}_{\alpha+1}$ is complete.

\item[Union:] For any set $A$, the set $\bigcup A = \{x \mid (\exists
y \in A.x \in y)\}$ exists.

 \item[Verification of Union:] Any $A\in {\mathbb E}_{\lambda}$
belongs to some rank ${\mathbb E}_{\alpha+1}$ for $\alpha<\lambda$
(every element of $Z$ first appears in a successor rank).  The
translation of $(\exists y \in A.x \in y)$ into the language of type
theory asserts that $x$ is in the $E|E$-preimage of $A$, which is a
subset of ${\mathbb E}_{\alpha}$, so has an $E$-code in ${\mathbb
E}_{\alpha+1}$, because ${\mathbb E}_{\alpha}$ is complete.

\item[Infinity:]  There is a set $I$ such that $\emptyset \in I$
and $(\forall x.x \in I \rightarrow \{x\} \in I)$.  

\item[Verification of Infinity:] Define a relation on the ordinals
$\leq \omega$ by $x\,R\,y \leftrightarrow y=x+1 \vee y=\omega$.  The
isomorphism type of this relation is the implementation of $I$.

\item[Choice:] Any pairwise disjoint collection of nonempty sets has a
choice set.

\item[Verification of Choice:] The translation of the property ``$P$
is a pairwise disjoint collection of nonempty sets'' into the language
of type theory is ``$P$ is an element of ${\mathbb E}_{\lambda}$ such
that the $E$-preimages of the elements of its $E$-preimage are
nonempty and disjoint''.  $P\in {\mathbb E}_{\lambda}$ belongs to some
rank ${\mathbb E}_{\alpha+1}$ for $\alpha<\lambda$.  Each element of
the $E$-preimage of an element of the $E$-preimage of $P$ belongs to
${\mathbb E}_{\alpha}$.  By the Axiom of Choice in type theory, the
pairwise disjoint collection of nonempty $E$-preimages of the elements
of the $E$-preimage of $P$ has a choice set, which is a subset of
${\mathbb E}_{\alpha}$, so has an $E$-code because ${\mathbb
E}_{\alpha}$ is a complete rank.

\end{description}

Furthermore, the translation of the axioms of Zermelo set theory into
the theory of all set pictures expressed with type-free set picture
variables are true, with a qualification, for essentially the same
reasons given above.  The qualification is that Separation will only
work for formulas in which every quantifier is bounded in a set,
because we cannot translate sentences which do not have this property
from the language of type-free set picture variables back into the
language of type theory.  The version of Zermelo set theory with this
restriction on Separation is called ``bounded Zermelo set theory'' or
``Mac Lane set theory'', the latter because Saunders Mac Lane has
advocated it as a foundational system.  Notice that the translation of
Mac Lane set theory into the type-free theory of set pictures does
{\em not\/} require the assumption that $\beth_{\omega}$ exists: the
only axiom that requires that $\lambda$ be limit in the development
above is Power Set, and the verification of the translation of Power
Set in the theory of all set pictures is given at the end of the
section om the hierarchy of ranks of set pictures (basically, one can introduce the ``power set'' of
any particular ``set'' one mentions by working in a higher type).

We state an additional axiom which holds in both the implementations
of Zermelo set theory given here, but which fails to hold in some
eccentric models of Zermelo set theory.  This axiom expresses the idea
that every element of ${\mathbb E}_{\lambda}$ belongs to some rank
${\mathbb E}_{\alpha}$.

\begin{description}

\item[Observation:] The Kuratowski pair $\{\{x\},\{x,y\}\}$ of two
sets $x$ and $y$ is easily seen to be a set, and the proof that this
is a pair goes much as in type theory.  We can then define relations
(and in particular well-orderings) just as we did in type theory.

\item[Definition:] A {\em subhierarchy\/} is a set $H$ which is
well-ordered by inclusion and in which each successor in the inclusion
order on $H$ is the power set of its predecessor and each non-successor in the
inclusion order on $H$ is the union of all its predecessors in that
order.  A {\em rank\/} is a set which belongs to some subhierarchy.

\item[Theorem:] Of any two distinct subhierarchies, one is an initial
segment of the other in the inclusion order.  So all ranks are
well-ordered by inclusion.

\item[Axiom of Rank:]  Every set is a subset of some rank.

\item[Verification of the Axiom of Rank:] Each $A \in {\mathbb
E}_{\lambda}$ belongs to some ${\mathbb E}_{\alpha}$,
$\alpha<\lambda$.  Each ${\mathbb E}_{\alpha}$ has an $E$-code, which
we will call $V_{\alpha}$, because it is a complete rank.  For any
$\beta$, $\{V_{\alpha} \mid \alpha<\beta\}$ has an $E$-code, which we
will call $H_{\alpha}$, because it is a subset of ${\mathbb
E}_{\beta+1}$.  It is straightforward to verify that $H_{\alpha}$
satisfies the stated properties for a subhierarchy (translated into the
language of type theory), whence we have the translation of
``$V_{\alpha}$ is a rank'', and ``$A \subseteq V_{\alpha}$'', so the
translation of ``$A$ is a subset of some rank'' holds.

\end{description}

The Axiom of Rank has many useful consequences.  We give two of them
here.

\begin{description}

\item[Definition:] We say that a set $A$ is {\em transitive\/} iff
$(\forall x \in A.(\forall y \in x.y \in A))$.  It is worth noting
that a set is transitive (in our interpretation in type theory) iff
any set diagram belonging to the set picture implementing $A$ is a
transitive relation.

\item[Theorem:]  Every set is included in a transitive set.

\item[Proof:] It is straightforward to prove by transfinite induction
along the inclusion order that all ranks are transitive.  By the Axiom
of Rank every set is included in a rank.

\item[Definition:] For any set $A$, we define $r_A$ as the minimal
rank in the inclusion order including $A$ as a subset.  We define
${\tt TC}(A)$, the {\em transitive closure\/} of $A$, as $\{x \in r_A
\mid$ every transitive set including $A$ includes $x\}$.  This exists
by Separation and is the minimal transitive set in the inclusion order
which includes $A$ as a subset.  ${\tt TC}(\{A\})$, which also
contains $A$ as an element, will sometimes be of more interest.

\item[Observation:] That sets have transitive closures is not provable
in Zermelo set theory as originally formulated.  The usual proof in
{\em ZFC\/} requires the very powerful Axiom of Replacement.  This is
deceptive, as Zermelo set theory with the Axiom of Rank is not
essentially stronger than Zermelo set theory (it is possible to
interpret the latter in the former), while the Axiom of Replacement
makes Zermelo set theory far stronger.

\item[Theorem (the Axiom of Foundation):] Every set $A$ has an element
$x$ such that $x$ is disjoint from $A$.

\item[Proof:] Let $r$ be the minimal rank in the inclusion order which
includes an element of $A$ as an element, and let $x \in r \cap A$.
Each element of $x$ belongs to a rank properly included in $r$, so $x$
is disjoint from $A$.

\end{description}

The Axiom of Foundation is frequently (anachronistically) adjoined to
Zermelo set theory as an additional axiom.

We observed above that the modern form of the Axiom of Infinity and
the original form do not imply each other in the presence of the other
axioms.  They do imply each other in the presence of the Axiom of
Rank.  For the Axiom of Rank, combined with the existence of any
infinite set, implies that there is a minimal infinite rank
$V_{\omega}$ in the inclusion order, and both the Zermelo natural
numbers and the von Neumann natural numbers are definable subsets of
$V_{\omega}$ (since all of the elements of either are clearly of
finite rank).  It is also amusing to note that the Axioms of Pairing
and Union can be omitted in the presence of the Axiom of Rank.
$\{a,b\}$ can be derived using Separation as a subset of the power set
of $r_a \cup r_b$ (this binary set union exists because it is actually
one of the ranks $r_a$ and $r_b$), and $\bigcup A$ can be derived
using Separation as a subset of ${\tt TC}(A)$.

\newpage

\subsection{$\dagger$Digression: Interpreting typed set theory as Mac Lane set theory}

``Mac Lane set theory" is the version of Zermelo set theory with Separation replaced by Bounded Separation.

Mac Lane set theory can be interpreted in typed
set theory with strong extensionality, using our entire universe of
typed objects.  We begin by postulating an operator $J$ which is
injective ($J(x)=J(y) \rightarrow x=y$) and sends type 0 objects to
type 1 objects.  An example of such an operator is the singleton
operator $\iota$.  Any such $J$ can be thought of as implemented by a
function $\iota``V^{\bf 0} \rightarrow V^{\bf 1}$.

We now indicate how to extend the $J$ operator to all types.  If $J$
is defined for type $n$ objects we define $J(x^{\bf n+1})$ as
$\{J(y^{\bf n}) \mid y^{\bf n} \in x^{\bf n+1}\}$.  Briefly, $J(x) =
J``x$.  It is easy to see that $J$ is injective on every type: we have
$J(x)=J(y) \leftrightarrow x=y$, no matter what the common type of $x$ and $y$.
By the definition of $J$ at successive types, we further have $J(x)
\in J(y) \leftrightarrow x\in y$, no matter what the successive types of $x$
and $y$.

In our interpretation of untyped set theory, we identify every object
$x$ of whatever type with each of its iterated images $J^n(x)$ under
the $J$ operator: in this way each type $n$ is seen to be embedded in
type $n+1$.  If $x$ is of type $m$ and $y$ is of type $n$, we have
$x=y$ in the interpretation iff $J^n(x)=J^m(y)$ (note that both of
these terms are of the same type $m+n$) and we have $x \in y$ in the
interpretation iff $J^{n}(x) \in J^{m+1}(y)$ (in which the terms have
successive types $m+n$ and $m+n+1$).  Notice that if $x$ and $y$ are
of the same type $n$, $J^n(x) = J^n(y) \leftrightarrow x=y$, and if
$x$ and $y$ are of successive types $n$ and $n+1$, $J^{n+1}(x) \in
J^{n+1}(y) \leftrightarrow x\in y$: where equality and membership make
sense in type theory, they coincide with equality and membership in
the typed theory.

If $x,y,z$ have types $m,n,p$, and we have $x=y$ and $y=z$, we have
$J^n(x)=J^m(y)$ and $J^p(y)=J^n(z)$.  Further applications of $J$ to
both sides of these formulas show that transitivity of equality works:
$J^{n+p}(x)=J^{m+p}(y)$ and $J^{m+p}(y)=J^{m+n}(z)$ are implied by the
previous equations and imply $J^{n+p}(x)=J^{m+n}(z)$, which in turn by
injectivity of $J$ implies $J^p(x)=J^m(z)$ which is the interpretation
of $x=z$.  Reflexivity and symmetry of equality present no
difficulties.  The substitution property of equality requires some
technical detail for its verification which we do not (NOTE: yet) give
here.

We verify that some of the axioms of Mac Lane set theory hold in this
interpretation.

We discuss the Axiom of Extensionality.  Suppose that $x$ is of
type $n$ and $y$ is of type $n+k$.  If $k=0$ and $x$ and $y$ have the
same elements, then $x=y$ by the axiom of extensionality of type
theory.  Otherwise, if for all $z$ of type $m$ we have $z\in x$ iff $z
\in y$ in the interpreted theory, this means we have $J^n(z) \in
J^m(x)$ iff $J^{n+k}(z) \in J^m(y)$, and further $J^{n+k}(z) \in
J^{m+k}(x)$, whence $J^m(y) = J^{m+k}(x)$, whence $J^n(y) =
J^{n+k}(x)$, whence $x=y$ in the interpretation, which is what is
wanted.

Now we discuss the Axiom of Bounded Separation.  We want to show the
existence of $\{x \in A \mid \phi[x]\}$ in the untyped theory, where
$\phi$ is a formula in membership and equality (it should not mention
the predicate of typehood, which does not translate to anything in the
language of type theory, though of course it may mention specific
types) we suppose that every quantifier in $\phi[x]$ is restricted to
a set.  Assign referents to each free variable appearing in $\{x \in A
\mid \phi[x]\}$, then assign each bound variable the type one lower
than that assigned to the set to which it is restricted ($A$ in the
case of $x$, the bound on the quantifier in the case of quantified
sets; if the bound is type 0, make the variable type 0 as well), then
apply our interpretation of the untyped language in the typed language
(adding applications of $J$ to variables in such a way as to make
everything well-typed).  For example, $\{x \in A \mid x \not\in x\}$
would become $\{x \in A \mid x \not\in J(x)\}$, with $x$ being
assigned type one lower than that assigned to $A$ (unless $A$ was
assigned a referent of type 0, in which case we would have $\{x \in
J(A) \mid x \not\in J(x)\}$).  The resulting set abstract exists in
our typed theory and has the right extension in the interpretation.
If there were unbounded quantifiers in $\phi[x]$, there would be no
way to interpret them in terms of our typed theory, which does not
allow any way to quantify over objects of all types.

(NOTE: more axioms to be supplied.  Rank will not necessarily hold
here; the form of infinity which holds depends on the exact form of
$J$.  This development is more {\em ad hoc\/} and more closely related
to the original form(s) of Zermelo set theory).

Something like this interpretation can also be carried out in the
version of type theory with weak extensionality.  We detail the
modifications of the construction.

The operation $J$ must be defined at atoms in each positive type.  $J$
is defined on type 0 as an injective operation raising type by 1, as
above.  If $J$ is defined on type $n$ objects, we define it on type
$n+1$ {\em sets\/} as before: $J(x^{\bf n+1}) = J``(x^n)$.  There are
no more than $T|V^{\bf n+1}|$ elementwise images under $J$ in type
$n+2$: since $T|V^{\bf n+1}|<|{\cal P}(V^{\bf n+1})|$ by Cantor's
theorem, we can choose as many distinct further elements of ${\cal
P}(V^{n+1})$, i.e., {\em sets\/} in $V^{n+2}$, as we need as images of
the type $n+1$ atoms under $J$.  The result $x \in y \leftrightarrow J(x) \in
J(y)$ now holds only if $y$ is a set, and for this reason we modify
the interpretation of $x \in y$ in the untyped theory (where $x$ and
$y$ have types $m$ and $n$ respectively in type theory) to $J^{n+1}(x)
\in J^{m+1}(y)$; if $x$ were of type 0 and $y$ were an urelement of
whatever type the original interpretation $J^n(x) \in J^m(y)$ would
not work correctly. 

\newpage

\subsection{$\dagger$Translation between Type Theory and Set Theory}

\begin{description}

\item[Importing results from type theory:]
We make a general claim here that mathematical results can be imported
from type theory to untyped set theory.  It is useful to give a
uniform account of how such a general claim can be justified (which
also makes it clear exactly what is claimed).

Just as we can translate the language of Zermelo set theory into the
language of type theory in a way which makes the axioms true\footnote{With qualifications discussed in section 3.7.1.}, so we
can translate the language of type theory into the language of untyped
set theory in a way which makes the axioms true -- and so makes all
the theorems true.

Let $\phi$ be a formula of the language of type theory mentioning $n$
types.  Let $X_0,X_1,\ldots X_{n-1}$ be a sequence of sets such that
${\cal P}(X_i) \subseteq X_{i+1}$ for each appropriate index $i$.  The
translation $(\phi)_X$ is defined as follows: each quantifier over
type $i$ is restricted to $X_i$.  Each formula $x \in y$, where $x$ is
of type $i$ and $y$ is of type $i+1$ is translated as $x \in y \wedge
y \in {\cal P}(X_{i})$ (elements of $X_{i+1} - {\cal P}(X_i)$ are
interpreted as urelements); formulas of the form $x=y$ are interpreted
as $x=y$.  Such a translation is also feasible if there is an infinite
sequence with the same properties, but it is not a theorem of Zermelo
set theory that there are such sequences.  A specific version which we
will write $[\phi]_X$ has $X_i = {\cal P}^{\bf i}(X)$ for a fixed set
$X$: a nice feature of this version is that we can generate as many
terms of the sequence as we need in a uniform way.  It is
straightforward to verify that as long as $X_0$ is infinite the
translations of all axioms of type theory into the language of untyped
set theory are true.  It can further be noted that expressions $T$
representing sets in the language of type theory will also have
translations $(T)_X$ where $X$ is a sequence or $[T]_X$ where $X$ is a
set.

This makes a wide class of mathematical assertions readily portable
from type theory to set theory.  For example, all of our assertions
about cardinal and ordinal arithmetic have readily determined
analogues in untyped set theory
\end{description}


We discuss how to transfer mathematical concepts and theorems from
type theory to set theory.

We have already seen that any formula of the language of type theory
can be translated to a formula $[\phi]_X$ (where $X$ is an infinite set)
which asserts that $\phi$ holds in a model of type theory in which $X$
is type 0, ${\cal P}(X)$ is type 1, and in general ${\cal P}^{\bf n}(X)$ is
type $n$.  $[\phi]_X$ is obtained by rereading membership and equality
as the relations of the untyped theory and restricting each type $n$
variable to ${\cal P}^{\bf n}(X)$.  For each axiom $\phi$ of type theory (in
each of its explicitly typed versions), it is straightforward to show
that $[\phi]_X$ is a theorem of Zermelo set theory.  So for any theorem
$\phi$ of type theory we have $[\phi]_X$ a theorem of Zermelo set
theory, and in fact we also have ``for all infinite sets $X$,
$[\phi]_X$'' a theorem of Zermelo set theory. 

Every object $t$ we can define in the language of type theory has
analogues $t_X$ for each infinite set $X$.  This presents an obvious
problem (a stronger version of the ambiguity of type theory which our
avoidance of type indices partially obscures).  All our definitions of
specific objects, with a few exceptions such as $\emptyset$, refer to
different objects depending on the choice of the parameter $X$.  For
example the number $3^{\bf n+2}$ is implemented as $[{\cal P}^{\bf n}(X)]^3$,
the set of all subsets of ${\cal P}^{\bf n}(X)$ with exactly three elements.
Just which set this is varies with the choice of $X$ (and $n$).

A possible conceptual problem with the theory of functions can be
dispelled: in type theory, we can prove easily that the functions from
a set $A$ to a set $B$ defined using Kuratowski pairs correspond
precisely to those defined using Quine pairs (they are at different
types but this ceases to be so inconvenient when we are translating to
untyped set theory).  So the question of which sets are the same size
and which relations are isomorphic is settled in the same way no
matter which pair definition one uses.

Nonetheless, the theory of cardinals and ordinals can be stated in
untyped set theory as the theory of specific objects.  Here we suppose
that we use von Neumann ordinals as the implementation of ordinal
numbers, and von Neumann initial ordinals as the implementation of
cardinals.  A sentence $(|A|=\kappa)_X$ asserts that $A$ belongs to a
certain cardinal $\kappa_X$.  This translates to an assertion $|A| =
\kappa$ in the language of untyped set theory, now not meaning $A \in
\kappa_X$ but $A \sim \kappa$, where $\kappa$ is the first von Neumann
ordinal which is equinumerous with an element (and so with all
elements) of $\kappa_X$.  Further, it is important to note that for
any cardinal $(\kappa^{\bf n})_X$ the von Neumann initial ordinal
associated with it will be the same as the von Neumann initial ordinal
associated with $(T(\kappa)^{\bf n+1})_X$: this gives a concrete
meaning to our erstwhile intuitive feeling that $\kappa$ and
$T(\kappa)$ are in fact the same cardinal.  Very similar
considerations apply to order types $(\alpha)_X$ and corresponding von
Neumann ordinals $\alpha$ (and we get the analogous result that the
ordinals $(\alpha^{\bf n})_X$ and $(T(\alpha)^{\bf n+1})_X$ correspond
to the same von Neumann ordinal $\alpha$).  Further, nothing but
technical points would differ if we used the Scott cardinals and
ordinals here instead of the von Neumann cardinals and ordinals.
Since we have a translation of ordinals and cardinals of type theory
to ordinals and cardinals of the untyped set theory, we can translate
the operations of addition and multiplication from type theory to
untyped set theory directly.  It might seem that we cannot translate
cardinal exponentiation so directly, but here we observe that though
$(\kappa^{\lambda})_X$ is not always defined, it is always the case
that $(T(\kappa)^{T(\lambda)})_X$ is defined (and will be
$T(\kappa^{\lambda})_X$ if the latter is defined); since the T
operation is now understood to be the identity, we see that cardinal
exponentiation is now a total operation.  The way in which the
definitions of cardinals and ordinals are transferred from type theory
to set theory ensures that theorems of cardinal and ordinal arithmetic
transfer as well.  Notice that Cantor's Theorem now takes the form
$\kappa < 2^{\kappa}$: there is no largest cardinal (from which it
follows that there can be no universal set, as certainly $|V| \geq
2^{|V|}$ would hold; the argument from the untyped form of Cantor's
theorem and the naive supposition that there is a universal set to a
contradiction is called {\em Cantor's paradox\/}).

Although we have just defined operations of cardinal and ordinal
arithmetic in terms of the interpreted type theory with $X$ as type 0,
it is perfectly possible to state definitions of these operations
which do not depend on the notation $[\phi]_X$.  The recursive
definitions of operations of ordinal arithmetic are inherited directly
by untyped set theory from type theory.  The definitions of $|A|+|B|$
as $|(A \times \{0\} \cup B \times \{1\}|$, $|A|\cdot|B|$ as $|A
\times B|$, and $|A|^{|B|}$ as $|A^B|$ work perfectly well in untyped
set theory (always remembering that the set theoretical meaning of
$|A|$, though not its mathematical function, is quite different).  But
the correspondence between the arithmetic of interpreted type theory
and the arithmetic of untyped set theory is important in seeing that
theorems can be relied upon to transfer fairly directly from type
theory to set theory.

Results we have given above imply that certain statements which can be
shown to be true in the version of Zermelo set theory interpreted in
our type theory with strong extensionality are inconsistent with {\em
ZFC\/}.  We showed above that $\beth_{\alpha}$ does not exist for some
$\alpha$ in these models (to be precise, if the cardinality of the set
corresponding to type 0 is $\aleph_{\beta}$, we can prove that
$\beth_{\beta+\omega}$ does not exist in that model (whereas in {\em
ZFC\/} we have $|V_{\omega+\alpha}| = \beth_{\alpha}$ for each ordinal
$\alpha$, so all $\beth_{\alpha}$'s must exist)) However, there are
models of Zermelo set theory obtained from models of type theory with
weak extensionality in which {\em ZFC\/} holds.  This might seem not
be possible since there is a sequence of sets $V^{\bf i}$ (the sets
corresponding to the types) such that any cardinal is less than some
$|V^{\bf i}|$ (since it is the cardinal of a set of some type): by
Replacement it might seem that the countable sequence of $V^{\bf i}$'s
would be a set (because it is the same size as the set of natural
numbers), so its union would be a set, which would have cardinality
larger than any $|V^{\bf i}|$.  But this argument does not work,
because there is no formula defining the sequence of $V^{\bf i}$'s (as
there is in the models based on type theory with strong
extensionality, where $V^{\bf i+1} = {\cal P}(V^{\bf i})$).  Moreover,
we will apply simple model theory below to show that for any model of
{\em ZFC\/} there is a model obtainable from a model of type theory
with weak extensionality in which the same statements of the language
of set theory are true [that is a very convoluted sentence, I know].

The serious difference in power between untyped set theory and typed
set theory has to do with the ability to quantify over the entire
universe.  This is just a difference in what we can {\em say\/} if we
use Bounded Separation, but if we adopt the full axiom of Separation
we can define sets in terms of global facts about the universe.  This
is best indicated using an example.

\begin{description}

\item[Theorem:] For each natural number $n$, there is a unique
sequence $s$ of sets with domain the set of natural numbers $\leq n$
such that $s_0 = {\mathbb N}$ and for each $i<n$, $s_{i+1}={\cal
P}^n({\mathbb N})$.

\item[Proof:]  Prove this by mathematical induction.  The set of natural
numbers $n$ for which there is such a sequence $s$ clearly includes 0
($s=\left<0,{\mathbb N}\right>$) and if it includes $k$ will also include $k+1$
(if $s$ works for $k$, $s \cup \left<k+1,{\cal P}(s_k)\right>$ works for $k+1$).

\item[Discussion:] In type theory with base type countable, sets
interpreting these sequences do not all exist in any one type, so no
assertion of type theory can even express the fact that they all
exist.  This statement is of course very badly typed, but a similar
assertion would be the statement that there is a sequence of cardinals
such that $s_0 = \aleph_0$ and $s_{i+1}=2^{s_i}$ for each $i$, and
this would present the same problem: in type theory with base type
countable, the sequence $\beth_0,\beth_1\,\beth_2,\ldots$ is not entirely
present in any one type.  The mere statement of the theorem cannot be
expressed in type theory because the quantifier over sequences $s$ is
not bounded in a set (and for this same reason this theorem cannot be
proved using Bounded Separation: the subset of the natural numbers
which needs to be shown to be inductive cannot be shown to exist).

\end{description}

\newpage

\chapter{Logic}

\section{Formalization of Syntax and Substitution}

%This should include formal language and substitution.  Formulas and
%terms are mathematical objects.  Formal definition of substitution (so
%we handle bound variables correctly).  Countability of the usual
%languages.  Discussion of larger languages?

In this section we discuss the representation of bits of syntax
(formulas and terms) by mathematical objects.  We will thereafter
identify the syntactical objects of our language with these
mathematical objects.

An obvious way to do this would be to represent ASCII characters by
natural numbers, then represent character strings as functions from
finite initial segments of $\mathbb N$ to ASCII characters.  But the
definition of formal operations on syntax with this definition would
be inconvenient.

Our representation will be typically ambiguous, as with all our
representations of mathematical objects in type theory: syntactical
objects will exist in all types above a certain minimum type (which we
really will not care about determining).  Though we work in type
theory it should be clear how the same construction could be done in
Zermelo set theory.

\newpage

\section{A toy example (calculator arithmetic)}

Our full type theory has a quite complex language, so we provide a preliminary example of construction of a formal language within our type theory
and definition of semantics for it (intended values for all the expressions of the language as represented within our theory).  The objects we use to represent expressions of calculator arithmetic will all be of the same type, some fixed $n+2$.

The language we consider is the language of calculator arithmetic.

Each individual digit is assigned its usual meaning (0,1,2,3,4,5,6,7,8,9).

Strings of digits are to be assigned their usual meanings:  if $D$ has been assigned a value $x$
already, and $d$ is a digit whose value $r$ we already of course know, the value of $Dd$ (understood as a string concatenation)
will be $10x+r$.

General expressions will include all the strings of digits and all sums and products of expressions.
So we expect $(102+5)\cdot 13$ to be an expression, for example.

Trying to represent our symbols as strings is certainly possible, but would require reasoning about
mathematical representations of parentheses which would be quite unpleasant.  We take a different tack,
which handles grouping without parentheses.

\begin{description}

\item[digits:]  Each digit $n \in \{0,1,2,3,4,5,6,7,8,9\}$ is represented by $(0,n)$.  Using quotes, we define `$n$' as $(0,n)$.

\item[base ten numbers:]  A base ten number with one digit $n$ is represented by the digit $(0,n)$.
A base ten number $N$ whose last digit is $n$ and which has more than one digit is represented by
$(1,D,(0,n))$, where $D$ is the representation of $\frac{N-n}{10}$.  So for example 5 is coded by $(0,5)$,
12 is coded by $(1,(0,1),(0,2))$ and 365 is coded by $(1,(1,(0,3),(0,6)),(0,5))$.  Using quotes, if `$d$' is a digit (so $d$ is a known small number)
and `$D$' is a decimal numeral, `$Dd$' (the string obtained by appending `$d$' to `$D$') is defined as $(1,`D',`d')$.

\item[arithmetic expressions:]  A base ten number by itself is an arithmetic expression.

If $E$ and $F$ are arithmetic expressions, $(2,E,F)$ represents the formal sum of these two notations 
and $(3,E,F)$ represents the product of these two notations.

Using quotes, if `$E$' is a calculator expression and `$F$' is a calculator expression, we define `$(E+F)$' as $(2,`E',`F')$ and `$(E \cdot F)$' as $(3,`E',`F')$.

\end{description}

The translation of the calculator notation ``$(102+5)\cdot 13$"  will then be $(3,(2,(1,(1,(0,1),(0,0)),(0,2)),(0,5)),(1,(0,1),(0,3)))$.

The point here is that the individual notations are objects internal to our type theory, rather than symbols.  An alternative way to do this would be to postulate something like character strings as objects of our theory, but it is instructive that we do not need any new primitive ideas to implement this.

It is important to note that these notations represent pieces of notation, not numbers.  ``2+3" (a piece of notation) is not the same thing as ``5" or ``3+2", though these pieces of notation represent the same numbers.  And all three of these are different complex pairs.

It is not enough for us to be able to represent each individual piece of calculator notation.  We want to be able to say that something is a piece of calculator notation internally to our mathematical language.  We define the sets corresponding to the categories of notation we have discussed.

\begin{description}

\item[the set of digits:]  The set $D$ of digits is easy to define:  this is the set $\{0\} \times \{0,1,2,3,4,5,6,7,8,9\}$.

\item[the set of base ten numbers:]  We will define the set $T$ of base ten numbers.  

We say that a set $I$ is $T$-inductive iff $D \subseteq I$ and $\{1\} \times I \times D \subseteq I$.  We call the set of all $T$-inductive sets ${\cal T}$ and define
$T$ as $\bigcap {\cal T}$, the collection of all objects which belong to every $T$-inductive set.

Certainly the collection of all base ten numbers is $T$-inductive.  And any $T$-inductive set (a collection $I$ which contains all the digits and has the property that any triple $(1,x,d)$
where $x$ is in the collection and $d$ is a digit) must contain all the base ten numbers.  An example of a $T$-inductive set which is not the collection of all base ten numbers
(other than the trivial example, the universe), is the collection $J=(\{0\} \times V)\cup (\{1\} \times V \times D)$ of all things which are either a pair with first component 0 or
a triple with first component 1 and last component a digit.

\item[the set of calculator expressions:]  We define the set $E$ of calculator expressions.

We say that a set $I$ is $E$-inductive iff $T \subseteq I$ and $\{2\} \times I \times I \subseteq I$ and $\{3\} \times I \times I \subseteq I$.
We call the set of all $E$-inductive sets ${\cal E}$, and we define $E$ as $\bigcap {\cal E}$, the collection of all objects which belong to any $E$-inductive set.
It should be clear both that the collection of calculator expressions should be $E$-inductive, and that any $E$-inductive set that we define will contain all the calculator expressions.

\end{description}

Finally, of course, the most interesting thing about a piece of notation is {\em what it means\/}.  We will define a function $v:E \rightarrow {\mathbb N}$ which will represent the natural number value which we expect a piece of calculator notation to denote (what display do you get when you type this notation into the calculator?).  A function like $v$ is called a ``valuation".

We list conditions which we expect to hold  of $v$.

\begin{description}

\item[valuation of digits:]  $v((0,n)) = n$ about sums up our expectations.  This is the simple case.  So we expect $((0,n),n) \in v$ for each $n \in \{0,1,2,3,4,5,6,7,8,9\}$.

\item[valuation of base ten numerals:]  A base ten numeral which is a digit we already know how to handle.  A base ten numeral of the form $(1,D,(0,n))$ will
have $v((1,D,(0,n)) = (10\cdot v(D)) + n$.  So if $(D,x) \in v$, we expect $((1,D,(0,n)),10\cdot x+n) \in v$.

\item[valuation of calculator expressions:]  A calculator expression which is a base ten numeral I already know how to handle.  We expect $v((2,e,f)) = v(e)+v(f)$
and $v((3,e,f))=v(e) \cdot v(f)$.  We express this a little differently:  if $(e,x) \in v$ and $(f,y) \in v$, we expect $((2,e,f),x+y)\in v$ and $((3,e,f),x\cdot y) \in v$.

\item[the appropriate kind of inductive set:]  A set $I$ is $v$-inductive iff $((0,n),n) \in I$ for each $n \in \{0,1,2,3,4,5,6,7,8,9\}$, and
 if $(D,x) \in I$ and $D \in T$ and $x \in {\mathbb N}$ we have $((1,D,(0,n)),10\cdot x+n) \in I$, and if $(e,x) \in I$ and $(f,y) \in I$, and $x, y \in {\mathbb N}$, we have $((2,e,f),x+y)\in I$ and $((3,e,f),x\cdot y) \in I$.

\item[the definition of $v$:]  Define ${\cal V}$ as the collection of all $v$-inductive sets and define $v$ as $\bigcap {\cal V}$, the collection of objects which belong to every
$v$-inductive set.  This way of constructing a function may make us queasy, and should remind us of the proof of the Iteration Theorem.  An informal argument that this
is correct should exploit the observation in effect already made that our first three points show that $v$, considered as a set of pairs, actually is $v$-inductive -- and further
that any $v$-inductive set should actually contain all the pairs in $v$, so their intersection is exactly $v$.

\newpage

\noindent{\bf Exercises}

The first two problems may be all that you do, but I do invite you to think about the two harder problems which follow.

\begin{enumerate}

\item  For each of the following nested pair expressions, determine whether it actually is a mathematical representation of a calculator expression (including digits and base ten numbers), and if it is one report its value.
Show steps in your calculation.

\begin{enumerate}

\item  $(2,(1,(0,1),(0,3)),(0,5)$

\item  $(1,(0,1),(1,(0,3),(0,4)))$

\item  $(3,(2,(0,5),(0,3)),(1,(0,1),0,0))$

\item   $(1,(2,(0,2),(0,3)),(0,5))$


\end{enumerate}

\item Write the nested pair expressions which represent  the following expressions of calculator arithmetic.  You do not have to compute values.

\begin{enumerate}

\item $5+4$

\item $(2+3)+4$

\item $2+(3+4)$

\item $15 \cdot 234$


\end{enumerate}

\item This is a challenge problem:  determine why I need to say ``$d \in T$" in the clause ``if $(d,x) \in I$ and $d \in T$ and $x \in {\mathbb N}$ we expect $((1,d,(0,n)),10\cdot x+n) \in I$"
(hint:  give an example of an illegal expression (something not in $E$) at which we would be forced to evaluate $v$ if we did not include this condition).  One of the parts of the first problem is relevant!

\item Another challenge problem:  show that the set of pairs $V \times {\mathbb N}$ is $v$-inductive -- this shows that every value of $v$ is a natural number.  ($V \times {\mathbb N}$ is the set of all ordered pairs whose second component is a natural number).

\end{enumerate}


\end{description}

\newpage

\section{A formal syntax for  our type theory}

We initially give a recursive definition of notation taken from logic
and set theory as mathematical objects.

We begin with variables.  The triple $\left<0,m,n\right>$ will
represent a bound variable $x^{\bf m}_n$ and the triple
$\left<1,m,n\right>$ will represent a free variable (or ``constant'')
$a^{\bf m}_n$ for natural numbers $m,n$.  The reasons why we want
bound and free variables will become evident later.  That is, we
define `$x^m_n$' as $\left<0,m,n\right>$ and `$a^m_n$' as
$\left<1,m,n\right>$.

The triple $\left<2,n,t\right>$, where $t$ is a term, will represent
the sentence $P_n(t)$ ($P_n$ being a unary predicate).  The quadruple
$\left<3,n,t,u\right>$ will represent the sentence $t \,R_n\,u$ ($R_n$
being a binary predicate (logical relation)).  We read
$\left<3,0,t,u\right>$ as $t\subseteq u$ and $\left<3,1,t,u\right>$ as $t=
u$.  That is, we define `$P_n(t)$' as $\left<2,n,\right.$`$t$'$\left.\right>$ and 
`$t \,R_n\,u$' as $\left<3,n,\right.$`$t$'$,$`$u$'$\left.\right>$.

The triple $\left<4,n,t\right>$ ($t$ being a term) stands for $F_n(t)$
($F_n$ being a function symbol).  The quadruple $\left<5,n,t,u\right>$
($t$ and $u$ being terms) stands for $t\,O_n\,u$, $O_n$ being a binary
function (operation) symbol.  That is, `$F_n(t)$' is defined as
$\left<4,n,\right.$`$t$'$\left.\right>$ and `$t\,O_n\,u$' is defined as
$\left<5,n,\right.$`$t$'$,$`$u$'$\left.\right>$.

We reserve $F_0$ and $F_1$ to stand for the projection operators,
and $O_0$ to stand for the ordered pair.

Note that all predicate and function symbols are typically ambiguous
(can be used with arguments of many types).  Binary relation symbols
are assumed to be type level and functions are assumed to have one or
both inputs and their output all of the same type.

The triple $\left<6,n,t\right>$ represents $\iota^{\bf n}(t)$ and the triple
$\left<7,n,t\right>$ represents $\bigcup^{\bf n} t$.    That is, `$\iota^{\bf n}(t)$'
is defined as $\left<6,n,\right.$`$t$'$\left.\right>$ and `$\bigcup^{\bf n} t$' is defined
as $\left<7,n,\right.$`$t$'$\left.\right>$. [It is important to note that we intend in our semantics to extend the union operation
so that $\bigcup \{x\}$ is equal to $x$ even if $x$ is an atom.]

Note that we can
now represent $t \in u$ as $\{t\} \subseteq u$.  The reason why we choose this apparently odd representation of the membership
relation is that we can then allow all formal relations in the syntax to be type-level, which makes the definition of
the type of an expression simpler.

The quadruple $\left<8,n,v,\phi\right>$ (where $\phi$ is a formula and
$v$ is a bound variable) is read $(Q_nv.\phi)$, where $Q_n$ is a quantifier.
We reserve $Q_0$ as $\exists$ and $Q_1$ as $\forall$.  That is, `$(Q_nv.\phi)$' is defined as $\left<8,n,\right.$`$v$'$,$`$\phi$'$\left.\right>$.

We briefly recall that the symbol $(\epsilon x.\phi)$ represents an arbitrarily chosen element of $\{x \mid \phi\}$ for each $\phi$, and a default object (which we could take to be $\emptyset$ if the type of the expression is positive) if $\{x \mid \phi\}$ is empty:  the Axiom of Choice allows us to assume
that in each type we have a suitable choice function which picks an element from each nonempty set.

The quadruple $\left<9,n,v,\phi\right>$ (where $\phi$ is a formula and
$v$ is a bound variable) represents a term $(B_nv.\phi)$ constructed by binding on a
formula.  We read $\left<9,0,v,\phi\right>$ as $(\epsilon v.\phi)$,
the Hilbert symbol.    That is, `$(\epsilon v.\phi)$'
(in particular) is defined as $\left<9,0,\right.$`$v$'$,$`$\phi$'$\left.\right>$.   Note that `$\{v \mid \phi\}$' can be read as
`$(\epsilon A.(\forall v.\{v\} \subseteq A \leftrightarrow \phi))$', where $A$ is the first variable of appropriate type not found elsewhere in the expression.

Alternatively, we could allow
$\left<9,0,\right.$`$v$'$,$`$\phi$'$\left.\right>$ to represent $\{x \mid \phi\}$, in which case our rules for typing this expression would be different; but the Hilbert symbol is actually very useful in formal logic, though less familiar to us, and we prefer to provide it as a syntactical primitive.

The pair $\left<10,\phi\right>$ represents $\neg \phi$.  The triple
$\left<11,\phi,\psi\right>$ represents $\phi \vee \psi$.  That is,
`$\neg \phi$' is defined as $\left<10,\right.$`$\phi$'$\left.\right>$ and `$\phi \vee \psi$'
is defined as $\left<11,\right.$`$\phi$'$,$`$\psi$'$\left.\right>$.  We could equally
well use the construction $\left<11,n,\phi,\psi\right>$ and provide ourselves with
a potentially infinite supply of binary propositional connectives:  $\left<11,n,`\phi',`\psi'\right>$ would be taken to code
`$\phi \oplus_n \psi$', and we would reserve $\oplus_0, \oplus_1, \oplus_2,\oplus_3$ for $\wedge,\vee,\rightarrow,\leftrightarrow$.

The above is not precisely mathematical as it relates mathematical
objects to pieces of notation.  We proceed to develop a thoroughly
mathematical account of syntax and semantics using this informal
account as motivation.  For readability, we will allow ourselves to
use quoted terms and formulas much of the time.

\begin{description}

\item[Definition:] This is a nonce definition.  A syntactical pair of
sets is a pair of sets $\left<T,F\right>$ with the following
properties, motivated by the idea that $T$ is an approximation to the
set of terms and $F$ is an approximation to the set of formulas.

\begin{enumerate}

\item For any natural numbers $m,n$, $\left<0,m,n\right>$ and
$\left<1,m,n\right>$ belong to $T$.  Objects $\left<0,m,n\right>$ are
called bound variables.

\item For any natural number $n$ and any $t \in T$,
$\left<2,n,t\right> \in F$.

\item For any natural number $n$ and $t,u \in T$,
$\left<3,n,t,u\right> \in F$.

\item For any natural number $n$ and $t \in T$, $\left<4,n,t\right>$,
$\left<6,n,t\right>$, $\left<7,n,t\right> \in T$

\item For any natural number $n$ and $t,u \in T$,
$\left<5,n,t,u\right> \in T$.

\item For any natural number $n$, , bound variable $v$, $\phi \in F$,
and $t,u \in T$, $\left<8,n,v,\phi\right> \in F$, and
$\left<9,n,v,\phi\right> \in T$.

\item For any $\phi,\psi \in F$, $\left<10,\phi\right>\in F$ and
$\left<11,\phi,\psi\right>\in F$.


\end{enumerate}

\item[Definition:] A {\em formal term set\/} is any set which is the
first projection $T$ of a syntactical pair $\left<T,F\right>$.  A {\em
formal proposition set\/} is any set which is the second projection
$F$ of a syntactical pair $\left<T,F\right>$.  A {\em formal term\/}
is an object which belongs to all formal term sets.  A {\em formal
proposition\/} is an object which belongs to all formal proposition
sets.

\item[Theorem:] If $\cal T$ is the set of all formal terms and $\cal
F$ is the set of all formal propositions, then $\left<{\cal T},{\cal
F}\right>$ is a syntactical pair of sets.

\end{description}

The two sets $\cal T$ and $\cal F$ are defined by mutual recursion.
It is natural to prove theorems about formal terms and propositions
using structural induction.  We will write formal terms and
propositions using ordinary typography, and in fact to the best of our
ability forget the intricacies of numerals and pairing that underly
the formal definition (particularly since the details are largely
arbitrary and could be changed wholesale without affecting the
subsequent development).

Terms have type, and considerations of type determine that some terms
are ill-formed.  $x^{\bf m}_n$ and $a^{\bf m}_n$ have type $m$.
$F(t)$ has the same type as $t$.  $t\,O\,u$ has type $m$ iff $t$ and
$u$ have the same type $m$ and is ill-typed otherwise.  $(\epsilon
x^{\bf m}_n.\phi)$ (this is the Hilbert symbol) has type $m$.  A
formula $t\,R\,u$ will only be considered well-formed if $t$ and $u$
have the same type.  If $t$ has type $n$, $\iota^{\bf k}(t)$ has type $n+k$
and $\bigcup^{\bf k}(t)$ has type $n-k$ if $n\geq k$ and is considered
ill-formed otherwise.  These clauses are enough to determine the typing
(and well-formedness) of all terms and formulas by recursion.

\newpage

Now we give the formal definition of substitution.  We define
$u[t/x_i]$ (the result of replacing $x_i$ with $t$ in the term $u$)
and $\phi[t/x_i]$ (the result of replacing $x_i$ with $t$ in the
formula $\phi$) at the same time.  Here we leave off the type index:
the type requirement is that $t$ and $x_i$ have the same type.

\begin{enumerate}

\item $x_j[t/x_i]$ is defined as $t$ if $i=j$ and as $x_j$ otherwise.

\item $a_j[t/x_i]$ is defined as $a_j$.

\item $F(u)[t/x_i]$ is defined as $F(u[t/x_i])$.

\item $(u \,O\, v)[t/x_i]$ is defined as $u[t/x_i]\,O\,v[t/x_i]$.

\item $(Bx_j.\phi)[t/x_i]$ is defined as
$(Bx_k.\phi[x_k/x_j][t/x_i])$, where $x_k$ is the first variable not
found in $(Bx_j.\phi)[t/x_i]$.  
The only $B$ that we use is the Hilbert symbol (which we use to express set abstraction as indicated above).

\item $P(u)[t/x_i]$ is defined as $P(u[t/x_i])$.

\item $(u \, R \, v)[t/x_i]$ is defined as $u[t/x_i]\,R\,v[t/x_i]$.

\item $(Qx_j.\phi)[t/x_i]$ is defined as
$(Qx_k.\phi[x_k/x_j][t/x_i])$, where $x_k$ is the first variable not
occurring in $(Qx_j.\phi)[t/x_i]$. 

\item $(\neg\phi)[t/x_i]$ is defined as $\neg\phi[t/x_i]$ and $(\phi
\vee \psi)[t/x_i]$ is defined as $\phi[t/x_i] \vee \psi[t/x_i]$.

\end{enumerate}

To justify that this definition works takes a little thought.  The
notion of length of a term or formula can be defined by a natural
recursion (we do not give the mind-numbing details here).  Then
observe that the substitution of $t$ for $x_i$ in any given formula
$P$ is may be defined in terms of other substitutions supposed already
defined, but these are always substitutions into strictly shorter
formulas.

Our formulation of syntax differs from usual formulations in defining
a single universal formal language, which is specifically adapted to
the needs of type theory, though it can also be used for single-sorted
first order theories.  The adaptation to first-order theories is
straightforward: simply do not use variables of type other than zero or
the singleton or union operations.  The language would need to be
extended for more complicated multi-sorted theories (more complicated
type theories): we will not discuss this.  The language could be
extended with $n$-ary predicate and function symbols for $n>2$, of
course.  It can obviously be cut down by specifying limited
collections of constants, unary and binary predicate symbols, and
unary and binary function symbols.

\subsection{Exercises}

\begin{enumerate}

\item Using the definitions of formal syntax above, write out the
mathematical object coding the formula $$(\forall x^3_1.(\exists
x^3_2.x^3_1\, R_2\, x^3_2)).$$

\item What is the term or formula coded by 
$$\left<8,1,\left<0,0,1\right>,\left<3,2,\left<0,0,1\right>,\left<1,0,1\right>\right>\right>?$$

\item  Formalize the process of substituting $2+3$ for $x$ in the sentence $(x+y)+z = x+(y+z)$, starting with the formal expression $$((x+y)+z = x+(y+z))[(2+3)/x]$$ and proceeding agonizing step by agonizing step as I did in class.  In words, this means ``replace $x$ with $2+3$ in the expression $(x+y)+z = x+(y+z)$" and should give just the result you expect.  But anyone doing formal syntax should expand something like that once (grin).

\item  The expression $(\forall x:  (\forall y:  c(x+y) = cx + cy))$  is true of course.  Suppose we replace $c$ with $y$.  Then we might think we are saying $(\forall x:  (\forall y:  y(x+y) = yx + yy))$, that is, $(\forall x:(\forall y: y(x+y)  = yx+y^2))$.  Now certainly this is true, but it is not really the statement which I mean when I say to replace $c$ with $y$ in the original statement.  Can you see why not?  What would the offical result of this substitution look like (you do not need to write formal expansions, just write out the intended
sentence, using whatever choice of variables seems reasonable).

\end{enumerate}

\newpage

\section{Formalization of Reference and Satisfaction}

In this section we define the notions of {\em meaning\/} and {\em
truth\/}.  That is, given an interpretation of the nonlogical symbols
of our language, we show how to formally define the referent of each
term and the truth value of each formula, mod assignments of values
to all variables.

We first need to set the stage.  A domain of objects is needed to
support our interpretation.  In fact, we supply a sequence $D_n$ of
domains, one for each $n \in {\mathbb N}$, with $D_n$ intended to be
the collection of type $n$ objects. 

 Note that all the sets $D_n$ are
actually of the same type in the sense of our working type theory.  If
we restrict our language to the first-order as indicated above, we
only need a single domain $D$.   We will use $M$ to represent the type of the elements of $D_n$ (the type of the objects that terms of our language stand for).  We will stipulate that the terms of our language are also of type $M$.  It follows that each of the domains $D_n$ is of type $M+1$ and the sequence $D$ is of type $M+2$ (a map sending type $M+1$ natural numbers to type $M+1$ sets).

We associate a value $a^n_i \in D_n$ with each constant $a^{\bf n}_i$.
With each unary predicate $P_i$ we associate a set $P^n_i \subseteq
D_n$ for each $n$ (because our language is typically ambiguous we need
an interpretation of each predicate over each type).  With each binary
relation symbol $R_i$ we associate a set $R^n_i \subseteq D_n \times
D_n$ for each $n$.  Similarly each unary function symbol $F_i$ is
associated with functions $F^n_i : D_n \rightarrow D_n$, and each
binary operation symbol $O_i$ with functions $O_n:D_n^2 \rightarrow
D_n$.  An injective map $\iota_{n+1}:D_n \rightarrow D_{n+1}$ is
provided for each $n$, and a map $\bigcup_n:D_{n+1}\rightarrow D_n$
with the property $\bigcup_n(\iota_{n+1}(x)) = x$ for each $x \in D_n$.  We define $\iota_{n,k}$
as the identity map on $D_n$ if $n=k$ and for $n<k$ define $\iota_{n,k}$ as $\iota_{k} \circ \iota_{n,k-1}$:  the operation $\iota_{n,k}$ implements the representation of $(k-n)$-fold singletons of type $n$ objects.  We define $\bigcup_{n,k}$
as the identity map on $D_n$ if $n=k$ and for $n>k$ define $\iota_{n,k}$ as $\bigcup_{k} \circ \bigcup_{n,k+1}$:  the operation $\bigcup_{n,k}$ implements the representation of $(n-k)$-fold unions of type $n$ objects.  It is useful to note that I have indexed these operations so that the index (the second one, if there are two) is the object language type of the output.
(The existence of these latter maps imposes requirements on the
sequence of sets $D_n$: the sets in the sequence must be of increasing
size).  To support the Hilbert symbol we provide
a function $H_n$ from nonempty subsets of $D_n$ for
each $n$: $H_n(A) \in 1$ for all $A$; if $A \subseteq D_n$, then
$H_n(A) \subseteq A$ if $A \neq \emptyset$; $H_n(\emptyset)$ is
defined and belongs to $\iota``D_n$ but is otherwise unspecified.

A {\em structure\/} for our formal language is determined by a map $D$
sending a possibly proper initial segment of the natural numbers to
domains $D_n$, ``singleton'' and ``union'' maps $\iota^{\bf n+1}:D_n
\rightarrow D_{n+1}$ and $\bigcup^{\bf n}:D_{n+1} \rightarrow D_n$ as above,
modified choice functions $H_n$ as above (if the Hilbert symbol is to
be used), and some partial functions implementing constants,
predicates and functions as indicated above: where $m,n$ are natural
numbers, $A(m,n)$ will be the element $a^m_n$ of $D_n$ used as the
referent of $a^{\bf m}_n$, $P(m,n)$ will be the subset $P^m_n$ of
$D_m$ intended to be the extension of the predicate $P_n$ in type $m$,
$R(m,n)$ will be the subset $R^m_n$ of $D_m \times D_m$ intended to be the
extension of the logical relation $R_n$, $F(m,n)$ is the element
$F^m_n$ of $D_m^{D_m}$ (recall that $B^A$ is the set of functions from $A$ to $B$) representing the action of the function symbol
$F_n$ in type $m$, and $O(m,n)$ is the element $O^m_n$ of
$D_m^{D_m\times D_m}$ representing the action of the operation symbol $O_n$ in
type $m$.  The length of the domain sequence and the domain of the
partial function determine the subset of our universal language which
is used in the obvious way.

The binding constructions used in the discussion which follows are
limited.  The only term construction binding propositions we provide
is the Hilbert symbol $(\epsilon x.\phi[x])$ which may be read ``an
$x$ such that $\phi[x]$ if there is one (chosen in an unspecified
manner if there are more than one) or a default object if there is no
such $x$''.  All definable term binding constructions (including the
set builder notation) can be defined in terms of the Hilbert operator.
The only quantifiers we provide are the usual ones (which can in fact
also be defined in terms of the Hilbert operator!).  It is not
difficult to extend the discussion to general binders, but it would
further complicate already very elaborate recursive definitions.

A possibly partial function $E$ on variables such that $E(x^{\bf n}_i)
\in D_{n}$ for each variable $x^{\bf n}_i$ in the domain of $E$ is
called an {\em environment\/}.  If $E$ is an environment we define
$E[d/x_i^{\bf n}]$ as the environment which sends $x^{\bf n}_i$ to $d$
and agrees with $E$ everywhere else (this may be an extension of $E$
if $E$ is not defined at $x_i^{\bf n}$).  Notice that each environment is an object
of type $M+1$.  [ If we restricted ourselves to finite partial functions as environments, it is possible
to use type $M$ objects built with ordered pairing.]

We will now recursively define functions $\cal R$ and $\cal V$ (named
with ``reference'' and ``valuation'' in mind).  These functions take
two arguments, an environment and a term:  strictly speaking, because of typing, they need to be type $M+2$ functions taking an environment and the singleton of a term as arguments.  They are partial functions:
they are sometimes undefined.  Strictly speaking, these functions are
defined relative to a structure and would be written $\cal R_S$ and
$\cal V_S$ if we wanted to explicitly specify a structure $S$ we were
working with.  We use the informal notation $a^n_i$ for $A(n,i)$, $P^n_i$ for $P(n,i)$, and so forth.  The domains of these
functions are restricted to the language appropriate to the structure
(and further restricted depending on the extent to which $E$ is
partial).

We define $\chi(\phi)$ as 1 if $\phi$ is true and 0 if $\phi$ is false.  Note that $\chi$ is a truly weird operation
taking a sentence of our metalanguage to a number; all uses of this device can actually be eliminated, but it is a convenience.  It {\em is\/} possible to
define $\chi(\phi)$ quite honestly, as $\{x\mid (\phi \wedge x \in 1) \vee (\neg \phi \wedge x \in 0)\}$.

\newpage

Now for the horrible recursive definition.

\begin{enumerate}

\item ${\cal R}(E,x^{\bf n}_i) = E(x^{\bf n}_i)$ (if this is defined).

\item ${\cal R}(E,a^{\bf n}_i) = A(n,i)$.

\item ${\cal R}(E,F_i(t))$ is defined as $F(n,i)({\cal R}(E,t))$, where
$n$ is the type of $t$ (as long as ${\cal R}(E,t)$ is defined).

\item ${\cal R}(E,\iota^{\bf k}(t))$ is defined as $\iota_{n,n+k}({\cal
R}(E,t))$, where $n$ is the type of $t$, as long as the
embedded reference is defined. 

\item ${\cal R}(E,\bigcup^{\bf k}(t))$ is defined as $\bigcup_{n,n-k}({\cal
R}(E,t))$, where $n$ is the type of $t$, as long as the
embedded reference is defined. 

\item ${\cal R}(u \,O_i\, v)$ is defined as $O(n,i)({\cal
R}(E,u),{\cal R}(E,v))$ just in case ${\cal R}(E,u)$ and ${\cal
R}(E,v)$ are defined and $u$ and $v$ have the same type $n$.

\item ${\cal V}(E,P(u))$ is defined as $\chi({\cal R}(E,u) \in P(n,i))$,
where $n$ is the type of $u$, as long as the embedded reference is
defined.

\item ${\cal V}(E,(u \, R_i \, v)$ is defined as $\chi(\left<{\cal
R}(E,u),{\cal R}(E,v)\right> \in R(n,i)$), as long as the embedded
references are defined and $u$ and $v$ have the same type $n$.

\item ${\cal V}(E,\neg\phi)$ is defined as $\chi(\neg({\cal
V}(E,\phi)=1))$, and ${\cal V}(E,\phi\vee\psi)$ is defined as $\chi({\cal
V}(E,\phi)=1 \vee {\cal V}(E,\psi)=1)$, as long as the embedded
valuations are defined.

\item ${\cal V}(E,(Qx^{\bf n}_j.\phi))$ is defined as $\chi((Qd \in D_n.{\cal
V}(E[d/x^{\bf n}_i],\phi)=1))$, where $Q$ is either $\exists$ or
$\forall$, as long as the embedded valuation is defined.   Please notice that the quantifiers on the left side of the definition are in quotes
and on the outside are real quantifiers of our metalanguage (restricted to the appropriate $D_n$).

\item ${\cal R}(E,(\epsilon x^{\bf n}_i.\phi))$ is defined as the sole
element of $H_n(\{d \in D_n\mid {\cal V}(E[d/x^{\bf n}_i],\phi)=1\})$
if the valuation is defined.

\end{enumerate}

Notice as with substitution that the reference and valuation functions are
defined recursively.  Reference and valuation for a particular term or formula
may appeal to reference or valuation for another formula or term, but always
a strictly shorter one.

Although our language is restricted for convenience in framing these
definitions, the full language of type theory is supported with
suitable definitions.  If equality and subset relations are primitive,
we define $x \in y$ as $\iota(x) \subseteq y$, $\phi \rightarrow \psi$
as $\neg \phi \vee \psi$, $\phi \wedge \psi$ as $\neg(\neg \phi \vee
\neg \psi)$, $\phi \leftrightarrow \psi$ as $\phi \rightarrow \psi \wedge \psi
\rightarrow \phi$, and $\{x \mid \phi\}$ as $(\epsilon A.(\forall x.x
\in A \leftrightarrow \phi))$.

A further technical note is that ${\cal R}_S$ and ${\cal V}_S$ are lateral operations:  they actually
take a type $M+1$ environment and a type $M$ term to a type $M$ term or truth value.
They can of course be transformed into sets by encasing the second argument and the value produced
in singleton brackets, but we will not do this.    We {\em will\/} suppose that we have done this in any context where we presuppose that we have identified
an ${\cal R}_S$ and ${\cal V}_S$ as actual objects of our type theory.

\newpage

\subsection{Exercises}

\begin{enumerate}
\item Use the definitions of reference and satisfaction to evaluate the
following expressions, if $D_0 = \{1,2,3\}$ and the following
information about the environment and interpretation is given.  Notice
that we really do not need to worry about types in this example.

$A(0,1)=3$ (that is, the intended referent of $a^0_1$ is 3).

$P(0,1) = \{1,2\}$

$R(0,1) = \{\left<1,1\right>,\left<2,2\right>,\left<3,3\right>\}$ (the
equality relation).

$E(x^0_n)= 1$ for all $n$ (the environment $E$ assigns every type 0
variable the value 1).

Show the reasoning behind your evaluation in detail.  The intended
evaluations are quite obvious: the point is to show that the nasty
definitions in the notes actually get us there, so detail must be
seen.  This is an exercise in step by step unpacking of definitions.

\begin{enumerate}

\item ${\cal R}(E,a^0_1)$

\item ${\cal R}(E,x^0_5)$

\item ${\cal V}(E,P_1(a^0_1))$

\item ${\cal V}(E,P_1(x^0_1))$

\item ${\cal V}(E,x^0_2\,R_1\,a^0_1)$

\item ${\cal V}(E,x^0_2\,R_1\,x^0_5)$

\item ${\cal V}(E,(\exists x^0_2.x^0_2\,P_1\,a^0_1))$

\end{enumerate}

\item  Why didn't we define ${\cal V}(E,(\forall x.\phi)$ as ``$\chi$(for all terms $t$, ${\cal V}(E,\phi[t/x])$=1)"?  Such a scheme,
called ``substitutional quantification", does have fans.  But it is not equivalent to our scheme.  
Can you see why?  Hint:  it is making a very strong assumption about the capabilities of our formal language.


\end{enumerate}

\newpage
\section{Formal Propositional Sequent Calculus}

We introduce sequent notation.  

\begin{description}

\item[Definition:] A {\em sequent\/} is an ordered pair
$\left<\Gamma,\Delta\right>$ of finite sets of formulas.  We write
sequents $\Gamma \vdash \Delta$.  The set $\{A\}$ (where $A$ is a
formula) is simply written $A$ in a sequent; the set $\Gamma \cup
\{A\}$ is written $\Gamma,A$; notation for the empty set is omitted.

\item[Definition:] A sequent $\Gamma\vdash\Delta$ is {\em valid\/} iff
every interpretation under which $\cal V$ is defined for all elements
of $\Gamma$ and $\Delta$ [we will presume this condition for all
interpretations and sequents hereinafter] and under which ${\cal
V}``\Gamma \subseteq \{1\}$ has $1 \in {\cal V}``\Delta$ (every interpretation
which makes {\em all\/} statements in $\Gamma$ true makes {\em some\/}
statement in $\Delta$ true).

\item[Lemma:] $\Gamma,A \vdash \Delta,A$ is a valid sequent for any
formula $A$ and sets $\Gamma$ and $\Delta$.

\item[Lemma:] $\Gamma,\neg A \vdash \Delta$ is a valid sequent iff
$\Gamma \vdash A,\Delta$ is a valid sequent.

\item[Lemma:] $\Gamma \vdash \neg A,\Delta$ is a valid sequent iff
$\Gamma,A\vdash \Delta$ is a valid sequent.

\item[Lemma:] $\Gamma,A\vee B \vdash \Delta$ is a valid sequent iff
both $\Gamma,A \vdash \Delta$ and $\Gamma,B \vdash \Delta$ are valid
sequents.  Note that this is a formalized version of the strategy of
proof by cases.

\item[Lemma:] $\Gamma \vdash A \vee B,\Delta$ is a valid sequent iff
$\Gamma\vdash A,B,\Delta$ is a valid sequent.

\end{description}

We introduce a weaker notion of valuation appropriate when we are
considering propositional logic only.

\begin{description}

\item[Definition:] A {\em propositional valuation\/} is a partial
function $\cal V$ which sends each formula in its domain to either 0
or 1, and which sends any formula $\neg\phi$ to $1-{\cal V}(\phi)$ and
any formula $\phi \vee \psi$ to ${\cal V}(\phi) + {\cal V}(\psi) -
{\cal V}(\phi)\cdot{\cal V}(\psi)$ (in each case iff the valuations of
subformulas are defined).

\item[Observation:] All valuations in the sense of the previous
section are propositional valuations, but not vice versa.

\item[Definition:] A propositionally valid sequent is one in which any
propositional valuation which is defined on all formulas involved and
sends all formulas on the left to 1 sends some formula on the right to
1.  Note that all propositionally valid sequents will be valid, but
not vice versa (a formula which is not propositionally valid may be
valid for other logical reasons).

\item[Observation:]  All the Lemmas above remain true when ``valid''
is replaced with ``propositionally valid''.

\end{description}

\begin{description}

\item[Theorem:] If a sequent $\phi$ is propositionally valid,
applications of the rules above will inevitably show this.  If a
sequent $\phi$ is not propositionally valid, applications of the rules
above will inevitably reduce the sequent to a form from which a
valuation witnessing its invalidity can be extracted.

\item[Proof:] Any application of the rules above converts a sequent
with $n$ disjunctions and negations in it to one or two sequents with
$n-1$ disjunctions and negations each.  So sufficiently many
applications of the rules will convert any sequent into a collection
of sequents in which all formulas are atomic (or quantified), but in
any event do not have accessible disjunctions or negations.  If each
of these sequents has a formula in common between its left and right
sets, the sequent is valid.  If one of these sequents does not have a
formula in common between its left and right sides, a valuation
assigning 1 to each formula on the left and 0 to each formula on the
right witnesses the fact that the original formula is
(propositionally) invalid.  The total number of steps will be no more
than $1+2+4+\ldots+2^n = 2^{n+1}-1$ (which means that proofs of
complex sequents may be impractically large!), because if we start
with a sequent with $n$ connectives and organize our work into steps
in which we apply a single rule to each sequent, at step $k$ we will
obtain no more than $2^k$ formulas of length $n-k$.

\end{description}

So we have given a complete formal account of propositional logic.

It is worth noting that a form of the rules above can be given in
which all sequents have the empty set or a singleton set on the right.
Many readers will be comfortable with many premisses but only a single
intended conclusion (the case of the empty set represents the goal of
a contradiction).  This can be done purely mechanically: apply the
rules in the forms given above, then, if there is more than one
formula on the right, convert all but one of them to their negations
and move them to the left.  In the case of the negation rule, move the
original conclusion to the left; in the case of the right rule for
disjunction, move the second disjunct to the left.  The theorem still
holds.

The given rules can be used to derive rules for the other
propositional connectives.  These resemble the proof strategies that
we have developed in the chapter on Proof, with the notable exception
that the left rule for implication seems different (although it does
support the modus ponens and modus tollens strategies we expect).  The
resemblance of the sequent rules to our proof strategies is clearer in
the single-conclusion forms (though the left rule for implication
remains eccentric).

We can present sequent proofs as mathematical objects.

\begin{description}

\item[Definition:] 

An axiom (a sequent with nonempty intersection between the left and
right side) is a proof of its own validity.

If the validity of sequent $A$ follows from the validity of sequent
$B$ by an application of a sequent rule, and $C$ is a proof of $B$,
then $\left<A,C\right>$ is a proof of $A$.

If the validity of sequent $A$ follows from the validity of sequents
$B$ and $C$ by an application of a sequent rule, and $D$ is a proof of
$A$ and $E$ is a proof of $C$, then $\left<A,\left<D,E\right>\right>$
is a proof of $A$.

Being an instance of one of the sequent rules is mathematically
definable, so the notion of being a sequent proof is mathematically
definable (the class of sequent proofs is the smallest class with the
closure conditions just described).

Note that the addition of more sequent rules will cause only minor
adjustments to this definition.

A sequent is {\em provable\/} if there is a proof of it.  A sentence
$\phi$ is provable iff the sequent $\vdash \phi$ is provable.

\end{description}

\newpage

We give the propositional sequent rules in a useful format.  In each
entry, the validity of the sequent below the line is equivalent to all
the sequents above the line being valid.

\begin{center}
\AxiomC{$\Gamma,A\fCenter{\,\vdash\,}A,\Delta$}
\DisplayProof
\end{center}

\begin{center}
\Axiom$\Gamma\fCenter{\,\vdash\,}A,\Delta$
\UnaryInf$\Gamma,\neg A\fCenter{\,\vdash\,}\Delta$
\DisplayProof
\end{center}

\begin{center}
\Axiom$\Gamma,A,B\fCenter{\,\vdash\,}\Delta$
\UnaryInf$\Gamma,A\wedge B\fCenter{\,\vdash\,}\Delta$
\DisplayProof
\end{center}

\begin{center}
\Axiom$\Gamma,A \fCenter{\,\vdash\,}\Delta$
\Axiom$\Gamma,B \fCenter{\,\vdash\,}\Delta$
\BinaryInf$\Gamma,A\vee B\fCenter{\,\vdash\,}\Delta$
\DisplayProof
\end{center}

\begin{center}
\Axiom$\Gamma\fCenter{\,\vdash\,}A,\Delta$
\Axiom$\Gamma,B\fCenter{\,\vdash\,}\Delta$
\BinaryInf$\Gamma,A\rightarrow B\fCenter{\,\vdash\,}\Delta$
\DisplayProof
\end{center}

\begin{center}
\Axiom$\Gamma,A\rightarrow B,B\rightarrow A\fCenter{\,\vdash\,}\Delta$
\UnaryInf$\Gamma,A \leftrightarrow B\fCenter{\,\vdash\,}\Delta$
\DisplayProof
\end{center}

\begin{center}
\Axiom$\Gamma,A\fCenter{\,\vdash\,}\Delta$
\UnaryInf$\Gamma\fCenter{\,\vdash\,}\neg A,\Delta$
\DisplayProof
\end{center}

\begin{center}
\Axiom$\Gamma\fCenter{\,\vdash\,}A,B,\Delta$
\UnaryInf$\Gamma\fCenter{\,\vdash\,}A\vee B,\Delta$
\DisplayProof
\end{center}

\begin{center}
\Axiom$\Gamma\fCenter{\,\vdash\,}A,\Delta$
\Axiom$\Gamma\fCenter{\,\vdash\,}B,\Delta$
\BinaryInf$\Gamma\fCenter{\,\vdash\,}A\wedge B,\Delta$
\DisplayProof
\end{center}

\begin{center}
\Axiom$\Gamma,A\fCenter{\,\vdash\,}B,\Delta$
\UnaryInf$\Gamma\fCenter{\,\vdash\,}A\rightarrow B,\Delta$
\DisplayProof
\end{center}

\begin{center}
\Axiom$\Gamma,A\fCenter{\,\vdash\,}B,\Delta$
\Axiom$\Gamma,B\fCenter{\,\vdash\,}A,\Delta$
\BinaryInf$\Gamma\fCenter{\,\vdash\,}A \leftrightarrow B,\Delta$
\DisplayProof
\end{center}

\newpage


\section{Formal First-Order Sequent Calculus:  the Completeness, Compactness and L\"owenheim-Skolem Theorems}

For first-order reasoning, we need to introduce sequent rules for
quantification.

\begin{description}

\item[Lemma (Cut Rule):]  $\Gamma \vdash \Delta$ is valid iff
$\Gamma,A \vdash \Delta$ and $\Gamma \vdash A,\Delta$ are both valid.

\end{description}

This may seem like a purely propositional rule, though we did not need
it in the previous section.  As we will see in a later subsection, we
do not need it here either, but it is very convenient.

We give the sequent rules for quantifiers (and the Hilbert symbol).

\begin{description}

\item[Lemma:] $\Gamma,(\exists x.\phi[x])\vdash \Delta$ is valid iff
$\Gamma,\phi[a]\vdash \Delta$ is valid, where $a$ is a
constant which does not appear in the first sequent.

\item[Lemma:] $\Gamma \vdash (\forall x.\phi[x]),\Delta$ is valid iff
$\Gamma\vdash \phi[a],\Delta$ is valid, where $a$ is a constant which
does not appear in the first sequent.

\item[Lemma:] $\Gamma \vdash (\exists x.\phi[x]),\Delta$ is valid iff
$\Gamma \vdash \phi[t],(\exists x.\phi[x]),\Delta$ is valid, where $t$
is any term.

\item[Lemma:] $\Gamma, (\forall x.\phi[x])\vdash \Delta$ is valid iff
$\Gamma, (\forall x.\phi[x]),\phi[t] \vdash \Delta$ is valid, where
$t$ is any term.

\end{description}

The sequent rules for equality are the following.

\begin{description}

\item[Lemma:]  Any sequent of the form $\Gamma\vdash t=t,\Delta$
is valid.  We take these as axioms.

\item[Lemma:]  $\Gamma,t=u\vdash \phi[t],\Delta$ is valid iff
$\Gamma,t=u\vdash \phi[u],\Delta$ is valid.

\end{description}

Here are the rules for the Hilbert symbol.

\begin{description}

\item[Lemma:] For any term $t$, $\Gamma \vdash \Delta$ is valid iff
$\Gamma, \phi[(\epsilon x.\phi)/x] \vdash \Delta$ is valid and $\Gamma
\vdash \phi[t/x],\Delta$ is valid.  If the existential quantifier is
defined in terms of the Hilbert symbol, its rule can be derived from
this rule (and the rule for the universal quantifier from the rule for
the existential quantifier).  Note that the Cut Rule is actually a
special case of this rule.

\item[Lemma:] $\Gamma,\phi[(\epsilon x.\psi[x])/x],(\forall x.\psi[x]
\leftrightarrow \chi[x]) \vdash \phi[(\epsilon x.\chi[x])/x],\Delta$.

\end{description}

Here is a lemma about provability which follows from common features
of all our rules.

\begin{description}

\item[Lemma:] If $\Gamma\vdash\Delta$ is provable using our sequent
rules then $\Gamma,\Gamma' \vdash \Delta,\Delta'$ is also provable
using our sequent rules for any finite sets $\Gamma'$, $\Delta'$.

\end{description}

These rules correspond precisely to our proof strategies for proof of
quantified goals and use of quantified hypotheses.  Our definition of
proofs as formal objects can be extended to first order logic proofs
by adding these sequent rules.

We now prove a constellation of results which show that first order
logic is {\em complete\/} (any valid sequent can be proved using the
rules we have given) but also cast some doubt on just how strong
first-order logic is.

\begin{description}

\item[Observation:] The sets of formal terms and formulas are
countably infinite.  It is obvious that they have countably infinite
subsets, so they are not finite.  A quick way to see that they are
just countably infinite is to observe that all our objects (formulas,
terms, sequents, and proofs) are built from natural numbers by pairing
and the construction of finite sets, and that finite sets and pairs of
natural numbers can be implemented as natural numbers, as we showed
above.  So the sets of terms and formulas could be understood as
infinite sets of natural numbers.  The formulation we use is
advantageous because it is clearly adaptable to larger languages (we
might for example want uncountably many constants).  This argument
also adapts to larger languages: for any set of an infinite
cardinality $\kappa$, objects in the set can be used to code pairs of
objects in the set by the theorem $\kappa^2 = \kappa$ of cardinal
arithmetic, so if we for example have $\kappa$ constants and otherwise
the usual finite or countable complement of symbols we will have
formula and term sets of size $\kappa$.

It is also important to note that the Construction which follows is
valid for restricted languages.  Limiting the number of constants,
predicates, functions, relations and/or operators to a finite set does
not affect the construction.  Completely eliminating the Hilbert
symbol does not affect the Construction.  Using just one type or a
finite subset of the types does not affect the Construction.

\item[Construction:] Let $\Gamma$ and $\Delta$ be possibly infinite
sets of formulas with the property that for any finite $\Gamma_0
\subseteq \Gamma$ and $\Delta_0 \subseteq \Delta$, $\Gamma_0 \vdash
\Delta_0$ is not provable, and which are such that infinitely many
constants of each type do not appear in any formula in either of these
sets (this is not an essential limitation: constants $a_i$ used can be
replaced with $a_{2i}$ in each type, freeing up infinitely many
constants). Then there is a countably infinite structure in which each
formula in $\Gamma$ and the negation of each formula in $\Delta$ is
satisfied.

For purposes of this proof we use only negation, disjunction, and the
existential quantifier as logical operations (all the others are
definable in terms of these and their proof rules are derivable from
the rules for these and their definitions).

The fact that the model constructed will be countably infinite will be
evident, because the elements of the model will be terms.

We provide an enumeration $F_i$ of all the formulas of our language in
which no bound variable appears free (every bound variable is in the
scope of a quantifier over that variable), and in which shorter
formulas appear before longer formulas.

We define sequences of finite sets of formulas $\Gamma_i$ and $\Delta_i$
which will have the following properties.

\begin{enumerate}

\item Each $\Gamma_i,\Gamma' \vdash \Delta_i,\Delta'$ is not a
provable sequent for any finite subsets $\Gamma',\Delta'$ of
$\Gamma,\Delta$ respectively.

\item $\Gamma_{i} \subseteq \Gamma_{i+1}$; $\Delta_{i} \subseteq \Delta_{i+1}$

\item Each formula $F_i$ appears in $\Gamma_{i+1} \cup \Delta_{i+1}$

\end{enumerate}

The motivation is that the set $\Gamma_{\infty}$ which is the union of
all the $\Gamma_i$'s will be the set of true statements of the model
to be constructed and the set $\Delta_{\infty}$ which is the union of
all the $\Delta_i$'s will be the set of false statements of the model
to be constructed.

\newpage

$\Gamma_0=\Delta_0=\emptyset$.  The conditions are clearly satisfied
so far.

If $\Gamma_i$ and $\Delta_i$ are defined and the conditions are
supposed satisfied so far, we have next to consider where to put
$F_i$.

\begin{enumerate}

\item If $\Gamma_i,\Gamma' \vdash F_i,\Delta'$ is not provable
for any finite subsets $\Gamma',\Delta'$ of $\Gamma,\Delta$
respectively, set $\Gamma_{i+1}=\Gamma_i$ and
$\Delta_{i+1}=\Delta_i\cup \{F_i\}$.

\item If $\Gamma_i,\Gamma'\vdash F_i,\Delta'$ is provable for some
$\Gamma'\subseteq \Gamma$ and $\Delta'\subseteq \Delta$, then it
cannot be the case for any finite subsets $\Gamma'',\Delta''$ of
$\Gamma,\Delta$ respectively that $\Gamma_i,\Gamma'',F_i\vdash
\Delta''$ is provable, as we would then be able to prove
$\Gamma',\Gamma''\vdash\Delta',\Delta''$ using the Cut Rule.  If $F_i$
is not of the form $(\exists x.\phi[x])$, we define $\Gamma_{i+1}$ as
$\Gamma_{i}\cup \{F_i\}$ and $\Delta_{i+1} = \Delta_i$.  If $F_i$ is
of the form $(\exists x.\phi[x])$, let $a$ be the first constant of
the same type as $x$ which does not appear in any formula in
$\Gamma,\Delta, \Gamma_i$ or $\Delta_i$, let $\Gamma_{i+1}$ be defined
as $\Gamma_{i}\cup \{(\exists x.\phi[x]),\phi[a]\}$ and let
$\Delta_{i+1}$ be defined as $\Delta_i$ [an important alternative is
to use the Hilbert symbol $(\epsilon x.\phi[x])$ instead of $a$].
Note that if $\Gamma_{i},(\exists x.\phi[x]),\phi[a],\Gamma' \vdash
\Delta_i,\Delta'$ were provable, so would $\Gamma_{i},(\exists
x.\phi[x]),\Gamma' \vdash \Delta_i,\Delta'$, and we have already
pointed out that the latter cannot be proved for any subsets
$\Gamma',\Delta'$ of $\Gamma,\Delta$ respectively in this case.  [If
the alternative approach is used, note that if $\Gamma_{i},(\exists
x.\phi[x]),\phi[(\epsilon x.\phi[x])/x],\Gamma' \vdash
\Delta_i,\Delta'$ were provable, then $\Gamma_{i},(\exists
x.\phi[x]),\Gamma' \vdash \Delta_i,\Delta'$ would also be provable].
\end{enumerate}

\end{description}

The discussion shows that the conditions required continue to hold at
each stage of the construction.  So the definition succeeds and we
obtain sets $\Gamma_{\infty}$ and $\Delta_{\infty}$ whose union is the
set of all formulas and whose properties we now investigate.

We are able to show that the following Lemmas hold.

\begin{description}

\item[Lemma:]  $\Gamma_{\infty}$ and $\Delta_{\infty}$ are disjoint.

\item[Proof:] If they had a common element $A$, then some $\Gamma_i$
and $\Delta_i$ would have that common element, and
$\Gamma_i\vdash\Delta_i$ would be an axiom of sequent calculus.

\item [Lemma:]  $\Gamma \subseteq \Gamma_{\infty}$; $\Delta \subseteq \Delta_{\infty}$

\item[Proof:]  Consider what happens to $F_i$ in either of these sets
at the appropriate stage of the Construction.

\item[Lemma:] $\neg\phi \in \Gamma_{\infty} \leftrightarrow \phi \in
\Delta_{\infty}$; equivalently, $\neg\phi\in \Gamma_{\infty}$ iff $\phi$
is not in $\Gamma_{\infty}$.

\item[Proof:]  Otherwise for some $i$, $\Gamma_{i}$ would contain both
$\phi$ and $\neg\phi$ or $\Delta_i$ would contain both $\phi$ and
$\neg\phi$.  In either case $\Gamma_i \vdash \Delta_i$ would be provable.

\item[Lemma:] $\phi \vee \psi \in \Gamma_{\infty}$ iff either $\phi
\in \Gamma_{\infty}$ or $\psi \in \Gamma_{\infty}$.

\item[Proof:] Otherwise we would either have $\phi \vee \psi$ in
$\Gamma_{\infty}$ and both $\phi$ and $\psi$ in $\Delta_{\infty}$, in
which case $$\Gamma_i \vdash \Delta_i$$ for some $i$ would take the
form $\Gamma_i,\phi\vee\psi \vdash \phi,\psi,\Delta_i$, which would be
provable, or we would have $\phi\vee\psi \in \Gamma_{\infty}$ and
either $\phi \in \Delta_{\infty}$ or $\psi \in \Delta_{\infty}$,
and thus some $\Gamma_i \vdash \Delta_i$ would take one of the forms

$$\Gamma_i,\phi\vdash \phi\vee\psi,\Delta_i$$

or 
$$\Gamma_i,\psi\vdash \phi\vee\psi,\Delta_i,$$

both of which are provable.

\item[Lemma:]  $(\exists x.\phi[x]) \in \Gamma_{\infty}$ iff
there is a term $t$ such that $\phi[t] \in \Gamma_{\infty}$.

\item[Proof:] If $(\exists x.\phi[x]) = F_i$ and $(\exists x.\phi[x])
\in \Gamma_{\infty}$ then some $\phi[a]$ is also in $\Gamma_{\infty}$
by a specific provision of the construction.  If $(\exists x.\phi[x]) \in \Delta_{\infty}$ and there is some $\phi[t] \in \Gamma_{\infty}$, then some
$\Gamma_i\vdash\Delta_i$ takes the form
$$\Gamma_i,\phi[t]\vdash(\exists x.\phi[x]),\Delta_i$$

and this is  provable.

\item[Lemma:]  $t=t \in \Gamma_{\infty}$ for any term $t$.  If $t=u \in \Gamma_{\infty}$ and $\phi[t] \in \Gamma_{\infty}$ then $\phi[u] \in \Gamma_{\infty}$.

\item[Proof:]  Immediate from the form of the sequent rules for equality.

\item[Lemma:] The relation $=_n$ on terms of type $n$ which holds
between terms $t$ and $u$ of type $n$ just in case $t=u \in
\Gamma_{\infty}$ is an equivalence relation.

\item[Proof:]  $t=u \vdash u=t$ and $t=u,u=v\vdash t=v$ are provable.

\item[Lemma:]  For any term $t$, if $\phi[t/x] \in \Gamma_{\infty}$
then $\phi[(\epsilon x.\phi[x])/x] \in \Gamma_{\infty}$.  

\item[Proof:] $\phi[t/x] \vdash \phi[(\epsilon x.\phi[x])/x]$ is
provable.

\item[Lemma:]  If $(\forall x.\phi[x] \leftrightarrow \psi(x)) \in \Gamma_{\infty}$
then $(\epsilon x.\phi[x]) = (\epsilon x.\psi[x]) \in \Gamma_{\infty}$.

\end{description}

Now we can define the interpretation of our language that we want.
The elements of $D_n$ are the terms of type $n$ in our language.
$a^n_i$ is actually defined as $a^{\bf n}_i$ (each constant is its own
referent).  $F^n_i$ is the map which sends each type $n$ term $t$ to
the term $F_i(t)$.  $O^n_i$ sends each pair of type $n$ terms
$\left<t,u\right>$ to the term $t\,O_i\,u$.  $P^n_i$ is the set of all
terms $t$ of type $n$ such that $P_i(t) \in \Gamma_{\infty}$.  $R^n_i$
is the set of all pairs of type $n$ terms $\left<t,u\right>$ such that
$t \, R_i \, u \in \Gamma_{\infty}$.  The functions $H_n$ are chosen
so that $H_n(\{t \mid \phi[t/x] \in \Gamma_{\infty}\})$ is the formal
term $(\epsilon x.\phi)$.

The idea here is that we construct a model in which each term is taken
to represent itself.  The atomic formulas are evaluated in a way
consistent with the idea that $\phi \in \Gamma_{\infty}$ simply means
``$\phi$ is true in the term model'', and the lemmas above show that
complex terms and formulas are evaluated exactly as they should be for
this to work.  We conclude that for each formula $\phi \in \Gamma$,
$\phi$ is satisfied in the term model, and for each formula $\phi \in
\Delta$, $\phi$ is not satisfied ($\neg \phi$ is satisfied) in the
term model.

\begin{description}

\item[Definition:] For any environment $E$ whose range consists of
closed terms and term or proposition $T$, we define $T[E]$ as
$T[E(x_1)/x_1][E(x_2)/x_2]\ldots[E(x_n)/x_n]$ where $n$ is the largest
index of a variable which occurs free in $T$.

\item[Theorem:] In the interpretation of our language just described,
${\cal V}(E,\phi) = 1 \leftrightarrow \phi[E] \in \Gamma_{\infty}$ for each formal
sentence $\phi$, and ${\cal R}(E,t) = t[E]$ for each formal term $t$.

\item[Indication of Proof:] This is proved by induction on the
structure of formal terms and propositions.  The Lemmas above provide
the key steps.

\end{description}

\newpage

The following theorems follow from considering the Construction and
following Theorem.

\begin{description}

\item[Completeness Theorem:]  Any valid sequent has a proof.

\item[Proof:] This is equivalent to the assertion that any sequent
which is not provable is invalid.  A sequent $\Gamma\vdash\Delta$ is
invalid precisely if there is an interpretation of the language under
which $\Gamma$ consists entirely of true statements and $\Delta$
consists entirely of false statements.  The Construction shows us how
to do this for any sequent which cannot be proved.

\item[Definition:] A collection $\Gamma$ of sentences is {\em
consistent\/} iff there is an interpretation under which all of them
are true.

\item[Compactness Theorem:] Any collection of sentences any finite
subcollection of which is consistent is consistent.

\item[Proof:] Let $\Gamma$ be a collection of sentences any finite
subcollection of which is consistent.  This implies that
$\Gamma_0\vdash \emptyset$ is invalid for each finite $\Gamma_0
\subseteq \Gamma$.  This means that $\Gamma\vdash \emptyset$
satisfies the conditions of the Construction so there is an
interpretation in a term model under which all the sentences in
$\Gamma$ are true.

\item[L\"owenheim-Skolem Theorem:] Any consistent set of sentences has
a finite or countable model.  If it has models of every finite size it
has an infinite model.

\item[Proof:] Any consistent set of sentences satisfies the conditions
of the Construction, and so has a term model, which is countable (or
finite).  If the theory has models of every finite size, it is
consistent with the theory resulting if we adjoin new constants $a_i$
indexed by the natural numbers with axioms $a_i \neq a_j$ for each $i
\neq j$, by Compactness.  A model of this extended theory will of
course be infinite.

\end{description}

The relation $=^n$ on $D_n$ implementing on each type $n$ will not be
the equality relation on $D_n$, but it will be an equivalence
relation.  We can convert any model in which equality is represented
by a nontrivial equivalence relation into one in which the equality
relation is represented by the true equality relation on each type by
replacing model elements of type $n$ with their equivalence classes
(or representatives of their equivalence classes) under $=^n$.

If the logic is extended to support our type theory, equality is
definable.  The relation $(\forall z.x\in z \rightarrow y \in z)$
provably has the properties of equality in the presence of the axiom
of comprehension.  Unfortunately, as we will see in the next section,
full type theory does not satisfy the Completeness Theorem.

Although the set-theoretical definition of the Hilbert symbol involves
Choice (and if we add type theory as part of our logic without some
care, the properties of the Hilbert symbol will imply Choice) the
Hilbert symbol adds no strength to first-order logic.  If we have any
theory not using the Hilbert symbol, we can use the Construction
(without Hilbert symbols) to build an interpretation of the language
of the theory in which all sentences are evaluated, and then (since
the domain of this interpretation is countable), add the order $t \leq
u$ on terms defined by ``the first term equal to $t$ in the
interpretation appears no later than the first term equal to $u$ in the
interpretation in a given fixed order on terms''.  Then define
$(\epsilon x.\phi[x])$ as the first object in this order such that
$\phi$.  The definition of $\leq$ extends to the new Hilbert terms,
and all formulas involving the defined Hilbert symbol have valuations
determined in the interpretation.

The alternative version of the Construction in which existential
statements are witnessed by Hilbert symbols instead of new constants
has the immediate merit that one does not need infinitely many free
constants and the additional merit that every object in the term model
is definable from the basic concepts of the theory (in the original
version of the Construction, the witnesses have an anonymous quality).

If our language is made larger by providing an uncountable collection
of constants, predicates, and/or function symbols, say of uncountable
size $\kappa$, the Construction still works, with the modification
that ``$\Gamma \vdash \Delta$ is provable'' should systematically be
read ``for some finite $\Gamma_0 \subseteq \Gamma$ and $\Delta_0
\subseteq \Delta$ $\Gamma_0 \vdash \Delta_0$ is provable''.  The
difficulty is that the construction will pass through stages indexed
by ordinals, and once $\alpha \geq \omega$ we will have
$\Gamma_{\alpha}$ and $\Delta_{\alpha}$ infinite sets.  Note that we
are not talking here about modifications which would make terms or
formulas of the language themselves into infinite objects (such as
infinite conjunctions or disjunctions).  The Compactness Theorem is
thus seen to hold for languages of all sizes, and likewise the
L\"owenheim-Skolem Theorem can be extended to assert that any theory
with infinite models has models of each infinite size $\kappa$: to
ensure that there are many distinct objects in a term model, add
enough constants $a_{\alpha}$ with axioms $a_{\alpha} \neq a_{\beta}$
for each $\alpha \neq \beta$.  Any finite collection of these new
axioms will be consistent with any theory which has infinite models,
and the Construction will give an interpretation under which all the
new constants are distinct.

NOTE (everything to end of section):

Think about Omitting Types theorem here or later.

{\em TNT\/} is a nice exercise for this section.  Also showing that
type theory is distinct from Zermelo by showing that there are models
of type theory with more natural numbers than types.

Section 6
is soon enough for development of the logic of the set constructor,
but some allowance for the set constructor (and its type regime)
should be added to syntax (which will require changes in my remarks).
Add remarks about single-sorted theories being readily supported here,
and more complex multi-sorted theories possible but not needed.

\newpage

\subsection{Exercises}
\begin{enumerate}

\item 
Express the axioms of group theory in the language of first order
logic (you do not need types and you do not need to use numerical
codings).  Groups are exactly models of this theory.  A group is
said to have {\em torsion\/} if there is an element $g$ of the group
and a natural number $n$ such that $g^n$ is the identity element $e$
of the group.  A group is said to be {\em torsion-free} if it does not
have torsion.  Prove that there is no formula $\phi$ in our formal
language for group theory which is true of exactly the groups with
torsion.  Hint: use compactness.  Suppose that $\phi$ is a formula
which is true in every group with torsion.  Consider the sentences
$\tau_n$ which say ``there is a $g$ such that $g^n=e$'' for each
concrete natural number $n$.  Notice (explain) that each of these
sentence can be written in our formal language.  Verify that the
infinite set of sentences $\{\phi,\neg\tau_1,\neg\tau_2,\neg\tau_3
\ldots\}$ satisfies the conditions of the Compactness Theorem (give
details).  Draw the appropriate conclusion.

Explain why this tells us that $(\exists n \in {\cal N}.g^n=e)$ is not
equivalent to any sentence in our formal language for group theory.

\item
The L\"owenheim-Skolem Theorem tells us that every theory with a
finite or countable language has a finite or countable model.  Our
untyped set theory has a countably infinite language, so has countably
infinite models.  

But in untyped set theory Cantor's Theorem $|A| < |{\cal P}(A)|$
holds.  As an exercise in porting results from type theory to set
theory, write out the proof of Cantor's Theorem in untyped set theory.
Hint: you do not need to make finicky use of the singleton operator in
your argument.

Finally, if $A$ is an infinite set in a model of untyped set theory,
either $A$ is not countably infinite (in which case we have an
uncountable set) or $A$ is countably infinite and $|A| < |{\cal
P}(A)|$, in which case ${\cal P}(A)$ is an uncountable set (according
to the model).  Yet the whole model may be countably infinite, and so
certainly any infinite subsets of the model are countably infinite.
Why is this not a contradiction (this argument is called {\em Skolem's
paradox\/})?  Hint: I'm using what look like the same words in
different senses here; explain exactly how.

\end{enumerate}

\newpage

\section{Cut Elimination for First-Order Logic}

\newpage

\section{Incompleteness and Undefinability of Truth}

We say that a term $t$ is closed iff all bound variables appearing in
it are actually bound by some quantifier (or Hilbert symbol).  A
closed formula in this sense is a sentence.  Each closed term $t$ has
a referent which we may write ${\cal R}(t)$ (the choice of environment
will not affect the reference of a closed term).  There are terms
`$t$' such that ${\cal R}(`t$'$)=t$: `$t$' has as its referent the
formal term $t$ itself.  There is a recursive procedure (using our
definition of syntax) which would allow us to define a function which
sends every formal term $t$ to such a formal term `$t$'.  Similarly we
can define a function sending each formal sentence $p$ (considered as
a mathematical object) to a formal term `$p$' such that ${\cal
R}(`p$'$)=p$.

An additional convention will make this easier to see: let the
operator $O_1$ be reserved to represent the ordered pair, and the
constants $a_{2n}$ to represent the natural numbers $n$.  Since all terms are
built from natural numbers by pairing, easy recursive definitions of
`$t$' in terms of $t$ and `$p$' in terms of $p$ can be given.

Now we can prove some quite surprising theorems.

\begin{description}

\item[G\"odel's First Incompleteness Theorem:] There is a sentence of
our language which is true but cannot be proved.

\item[Proof:] Define a predicate $G$ of formulas $p$ as follows:
$G(p)$ says ``$p$ is a formula with one free variable $x$ and
$p[`p$'$/x]$ is not provable''.  We have seen in the previous sections
that everything here is definable.  Let $g$ represent the formula
$G(p)$ as a mathematical object.  $G(g)$ says that $g$ is a formula
with one free variable (it has one free variable $p$ as you can see
above) and $g[`g$'$/p]$ is not provable.  But $g[`g$'$/p]$ is the
statement $G(g)$ itself.  If $G(g)$ is true, it cannot be proved.  If
$G(g)$ is false, it can be proved and is therefore true.  So $G(g)$ is
true but not provable.

There are some subtleties here if there are unintended objects among
our proofs (we discussed this possibility for the natural numbers
earlier).  The sentence $G(g)$ cannot be provable, as we would then
have a concrete proof whose existence falsifies what it proves.
Suppose that $G(g)$ could be decided by being proved false: this would
show that there is a ``proof'' of $G(g)$, but that might be an
``unintended object'' that we would never actually find.

This loophole can be closed by modifying the definition of $G$ (a
trick due to Rosser).  Instead of constructing a statement which
asserts its own unprovability, construct by the same technique a
statement which asserts that if it is provable there is a shorter
proof of its negation (a notion of numerical measure of size of proofs
can readily be defined recursively).  If a concrete proof of this
statement were given, there would be a proof of its negation which was
shorter, and so also concrete.  If a concrete disproof of this
statement were given, then the statement would be true (as no shorter
statement could be a proof): this would make a concrete proof of the
statement possible.  Whether or not there are unintended ``proofs'' or
``disproofs'' of this statement, the statement must actually be
undecidable.

\end{description}

This theorem applies not only to our type theory but also to bounded
Zermelo set theory, Zermelo set theory and {\em ZFC\/} (where all our
constructions can be carried out) and even to arithmetic (our whole
formal development of the notion of provability can be carried out
entirely in arithmetic: all we need is a notion of ordered pair
definable in arithmetic, and we have shown that enough set theory can
be defined in arithmetic that Kuratowski pairs of natural numbers can
be coded as natural numbers.  Even our semantics can be defined in
arithmetic, with the stipulation that environments have to be partial
functions from variables to domain elements (since they must be
finite) and domains $D_n$ need to be defined by formulas rather than
given as sets.

A corollary of G\"odel's First Incompleteness Theorem is 

\begin{description}

\item[G\"odel's Second Incompleteness Theorem:] Our type theory (or
untyped set theory, or arithmetic) cannot prove its own consistency.

\item[Indication of Proof:] The underlying idea is that to
prove consistency is to prove that some statements cannot be proved.
If the Rosser sentence can be proved, we can prove that all sentences
can be proved (because if the Rosser sentence has a proof, so does its
negation, and so does everything).  So if we can prove consistency we
must be able to prove that the Rosser sentence cannot be proved.  But
if we can prove that the Rosser sentence cannot be proved, then we can
prove that the Rosser sentence is (vacuously) true (and so we have
proved it contrary to hypothesis).

There are problems of level here.  To actually prove that all this
works requires results such as ``if we can prove $\phi$, then we can
prove that $\phi$ is provable,'' and some other similar proofs along
the same lines.

\end{description}

We have never found the First Incompleteness Theorem particularly
surprising: there was never any reason to suppose that we could prove
everything that happens to be true in mathematics.  The Second
Incompleteness Theorem is a bit more alarming (we cannot prove that
the reasoning techniques in our working theory are free from paradox
{\em in that theory\/}).  The next result is quite alarming (and
requires more care to understand).

\begin{description}

\item[Tarski's Theorem:]  The predicate of formulas $p$ of the language
of our type theory (or of untyped set theory, or of arithmetic) which asserts
that $p$ is true cannot be defined in the same theory.

\item[Proof:]  Suppose there there is such a definable predicate {\tt true}.
Define $T(p)$ as ``$p$ is a predicate with one free variable $x$ and
$\neg{\tt true}(p[`p$'$/x])$''.  Let $t$ be the mathematical object representing
$T(p)$.  Then $T(t)$ asserts that $T(t)$ itself is not true.  This is simply impossible.  There can be no truth predicate (of formal sentences).

\end{description}

It is easy to misunderstand this.  For any statement $\phi$ in our
informal mathematical language (of whichever theory) we can say
``$\phi$ is true''; this simply means $\phi$ and has nothing to do
with Tarski's theorem.  What we cannot do is define a predicate of
formal mathematical objects $\Phi$ coding sentences $\phi$ of the
language of our working theory in such a way that this predicate is
true of $\Phi$ exactly if the corresponding formula $\phi$ is true in
our theory.  This is quite weird, since the missing predicate can be
understood as a predicate of natural numbers (in any of these
theories, if we construe the pair of the formalization of syntax as
the pair definable on the natural numbers).

The reader should notice the formal analogy between these results
(especially Tarski's Theorem) and Russell's paradox.  Unfortunately
here the self-application $p[`p$'$/x]$ cannot be exorcised as $x \in x$
was by our type discipline: the self-application is meaningful so
something else has to give.

It is important to notice that the problem here is not that our
theories are too weak.  Any theory sufficiently strong in expressive
power to describe provability (which amounts to having enough
arithmetic) has these features.  It should be noted that stronger
theories can prove consistency of weaker theories.  For example, type
theory does prove the consistency of arithmetic (because one can build
a set model of arithmetic in type theory).

\chapter{Model Theory}

NOTE:  An earlier note said that all of this should be conducted in type theory.  I'm not so certain, particularly as I approach Math 522 in fall 2017.


\section{Ultrafilters and Ultrapowers}

\begin{description}

\item[Definition:]  Let $\leq$ be a partial order.  A nonempty subset $F$ of ${\tt fld}(\leq)$  is a {\em filter in $\leq$\/}  iff it has the properties that for every $x,y \in {\tt fld}(\leq)$ there is some $z$ such that $z \leq x$ and $z \leq y$ and that for every $x,y$ if $x \in F$ and $x \leq y$ implies $y \in F$.  A filter in $\leq$ is proper iff it is not the entire field of $F$.  A filter in $\geq$ is called an {\em  ideal in $\leq$\/}.

\item[Definition:]  This is a maximally abstract definition of filters and ideals.  For our purposes in this section, the partial order $\leq$ will always be the subset
relation on ${\cal P}(X)$ for some fixed set $X$.  So, for the rest of this section, a filter on $X$ is a subset of ${\cal P}(X)$ which is a filter in the subset relation on ${\cal P}(X)$ in the sense just defined.  Further, an {\em ultrafilter\/} on $X$ is a filter $U$ on $X$ with the property that for each $A \subseteq X$,
exactly one of $A$ and $X-A$ belongs to $U$.  Note that for each $x \in X$, the set $U_x = \{A \in {\cal P}(X)\mid x \in A\}$ is an ultrafilter on $X$;
such ultrafilters are called {\em principal\/} ultrafilters on $X$.  An ultrafilter on $X$ which is not of the form $U_x$ for any $x \in X$ is called a {\em nonprincipal\/} ultrafilter on $X$.

\item[Theorem:]  Let $X$ be an infinite set.  Then there is a nonprincipal ultrafilter on $X$.

\item[Proof:]  Choose a well-ordering $W$ of ${\cal P}(X)$.  We define the ultrafilter $U_W$ by transfinite recursion.  Suppose that we have determined for each
$\beta<\alpha$ whether $W_{\beta} \in U_W$.  We provide that $W_{\alpha} \in U_W$ iff $W_{\alpha} \cap \bigcap_{\beta \in F}W_{\beta}$ is an infinite set for each finite set $F$ of ordinals less than $\alpha$ such that $W_{\beta} \in U_W$ for each $\beta \in F$.  Notice that the case $F=\emptyset$ tells us that $W_\alpha$ is infinite.

We verify that $U_W$ is an ultrafilter on $X$.

The intersection of any finite subset of $U_W$ is an infinite set:  we can see this by considering the last element of the finite set in terms of the well-ordering
$\leq$ and applying the definition of $U_W$.  A set $A$ fails to belong to $U_W$ exactly if there is a finite subcollection $F$ of $U_W$ such that
the intersection of $F \cup \{A\}$ is finite:  clearly if there is such a subcollection $A$ is not in $U_W$, and if there is no such subcollection the recursive definition will place $A$ in $U_W$.  

We show that if $A$ belongs to $U_W$ and $A \subseteq B$, then $B$ must belong to $U_W$:  suppose
$B$ did not belong to $U_W$; it follows that there is a finite subcollection $F$ of $U_W$ such that the intersection of $F \cup \{B\}$ is finite,
from which it follows that the intersection of $F \cup \{A\}$ is finite, from which it follows that $A$ is not an element of $U_W$.  We show that
if $A$ and $B$ belong to $U_W$, there is $C \in U_W$ such that $C \subseteq A$ and $C \subseteq B$:  a suitable $C$ is $A \cap B$, for which it is clear 
that any finite subcollection $F$ of $U_W$ has the intersection of $F \cup \{A \cap B\}$ infinite because this is equal to the intersection of
$(F \cup \{A\}) \cup \{B\}$.  This verifies that $U_W$ is a filter on $X$.

It cannot be the case that $A$ and $X-A$ are both in $U_W$ because their intersection is not infinite;
nor can it be the case that both are not in $U_W$, because we would then have finite subsets $F$ and $G$ of $U_W$ with the intersection
of $F \cup \{A\}$ finite and the intersection of $G \cup \{X-A\}$ finite, so all but finitely many of the members of $\bigcap F$ would be outside
$A$ while all but finitely many of the members of $\bigcap G$ would be in $A$, so $\bigcap(F \cup G)$ would be finite, which is impossible.  This verifies that $U_W$ is an ultrafilter on $X$.

$U_W$ is a nonprincipal ultrafilter because any principal ultrafilter $U_x$ has a finite element $\{x\}$.

Note that the Axiom of Choice is used here (we have actually shown that there is a nonprincipal ultrafilter on $X$ if ${\cal P}(X)$ can be well-ordered).
This use of choice is essential:  it is consistent with the other axioms of type theory or set theory that there is no nonprincipal ultrafilter on any infinite set.
It is easy to show that any ultrafilter on a finite set is principal.

\item[Definition:]  Let $X$ be an infinite set and let $U$ be a nonprincipal ultrafilter on $X$.  Let $A$ be any set (not necessarily of the same type as $X$).
Let $f$ and $g$ be two maps from $X$ to $A$ (these may be lateral!).  We define $f \sim_U g$ as holding iff $\{x \mid f(x)=g(x)\} \in U$.
It is easy to see that $\sim_U$ is an equivalence relation:  reflexivity and symmetry are trivial, while transitivity follows from the fact that
$U$ is a filter:  if $\{x \mid f(x)=g(x)\}\in U$ and $\{x \mid g(x)=h(x)\} \in U$, then $\{x \mid f(x) = g(x) \wedge g(x) = h(x)\} \in U$, being the intersection of two elements of $U$, and its superset $\{x \mid f(x)=h(x)\}$ is also in $U$.
We define $A^U$, the {\em ultrapower\/} of $A$ with respect to $U$, as the collection of equivalence classes under $\sim_U$.   With each $a \in A$ we associate $a^* \in A^U$, defined as the equivalence class under $\sim_U$ of the constant function on $X$ with value $a$.  Note that the domain of $\sim_U$ is the collection of functions from $X$ to $A$, and that we have indicated how to define this even if $A$ and $X$ are not of the same type.

\item[Definition:]  Let $X$ be an infinite set and let $U$ be a nonprincipal ultrafilter on $X$.  Let $A$ and $B$ be sets (not necessarily of the same
type) and let $R$ be a (possibly lateral) relation from $A$ to $B$.  For $[f]$ in $A^U$ and $[g]$ in $B^U$, we define $[f]\,R^U\,[g]$ as holding
iff $\{x \mid f(x)\,R\,g(x)\} \in U$ (it is straightforward to show that this does not depend on the choice of the representatives $f$ and $g$ of the elements of $A^U$ and $B^U$).   Note that $a^* \,R^U \,b^* \leftrightarrow a\,R\,b$.

\item[Construction:]  We view $A^U$ as a kind of extension of $A$, with each element $a$ of $A$ corresponding to the element $a^*$ of $A^U$.
We are going to define an extension of the language we use to talk about $A$ to a language which talks about $A^U$.  In fact, we are
going to carry out such an extension for any collection of domains we wish to consider, all at once.

For any open sentence $\phi(x_1,\ldots,x_n)$ with no free variables other than $x_1,\ldots,x_n$, in which each $x_i \in A_i$,
we define a sentence $\phi^*([f_1],\ldots,[f_n])$ for any fixed $[f_i]\in A_i^U$ as meaning $\{x \mid \phi(f_1(x),\ldots,f_n(x))\} \in U$.

If $f_i \equiv_U g_i$ for each $i$, then $\phi^*([f_1],\ldots,[f_n])$ asserts that $\{x \mid \phi(f_1(x),\ldots,f_n(x))\}$ is an element of $U$,
and, because intersections of elements of $U$ are in $U$, so is $\{x \mid \phi(f_1(x),\ldots,f_n(x)) \wedge f_1(x)=g_1(x) \wedge \ldots \wedge f_n(x)=g_n(x)\}$, which is a subset of $\{x \mid \phi(g_1(x),\ldots.g_n(x))\}$, so this latter set is in $U$, so $\phi^*([g_1],\ldots,[g_n])$.  The argument is completely symmetrical that shows that $\phi^*([g_1],\ldots,[g_n])$ implies $\phi^*([f_1],\ldots,[f_n])$, so the choice of representatives in
our notation for elements of $A_i^U$'s is immaterial.

We note that if $\phi(x_1,\ldots,x_n)$ is $\neg\psi(x_1,\ldots,x_n)$, then $\phi^*([f_1],\ldots,[f_n])$ is equivalent to $\{x \mid \neg\psi(f_1(x),\ldots,f_n(x))\} \in U$, which is equivalent to $\{x \mid \psi(f_1(x),\ldots,f_n(x))\} \not\in U$, because $U$ is an ultrafilter, which is in turn equivalent to
$\neg \psi^*([f_1],\ldots,[f_n])$.  In other words, the meaning of negation in the translated language is what we expect.

If $\phi(x_1,\ldots,x_n)$ is $\psi(x_{s_1},\ldots,x_{s_p}) \wedge \chi(x_{t_1},\ldots,x_{t_q})$, then $\phi([f_1],\ldots,[f_n])$ is equivalent to
$\{x \mid \psi(f_{s_1}(x),\ldots,f_{s_p}(x)) \wedge \chi(f_{t_1}(x),\ldots,f_{t_q}(x))\} \in U$, which is equivalent to $\{x \mid \psi(f_{s_1}(x),\ldots,f_{s_p}(x))\} \in U \wedge \{x \mid \chi(f_{t_1}(x),\ldots,f_{t_q}(x))\} \in U$, because subsets $A$ and $B$ of $X$ both belong to $U$ iff their intersection belongs to $U$, and this is in turn equivalent to $\psi^*([f_{s_1}],\ldots,[f_{s_p}]) \wedge \chi^*([f_{t_1}],\ldots,[f_{t_q}])$.  In other words, the meaning
of conjunction in the translated language is what we expect.

If $\phi(x_1,\ldots,x_n)$ is $(\exists y.\psi(y,x_1,\ldots,x_n))$, then $\phi^*([f_1],\ldots,[f_n])$ is equivalent to $\{x \mid (\exists y.\psi(y,f_1(x),\ldots,f_n(x))\} \in U$.  If there is a $g$ such that $\{x \mid \psi(g(x),f_1(x),\ldots,f_n(x))\} \in U$, then certainly $\{x \mid (\exists y.\psi(y,f_1(x),\ldots,f_n(x))\} \in U$, because $\{x \mid \psi(g(x),f_1(x),\ldots,f_n(x))\} \subseteq\{x \mid (\exists y.\psi(y,f_1(x),\ldots,f_n(x))\}$.
Now suppose that  $\{x \mid (\exists y.\psi(y,f_1(x),\ldots,f_n(x))\} \in U$ .  Define a function $g$ such that for each $x$ such that $(\exists y.\psi(y,f_1(x),\ldots,f_n(x))$ we have $\psi(g(x),f_1(x),\ldots,f_n(x))$:  this is an application of the Axiom of Choice.  Now we have $\{x \mid \psi(g(x),f_1(x),\ldots,f_n(x))\} = \{x \mid (\exists y.\psi(y,f_1(x),\ldots,f_n(x))\}\in U$ for this particular $g$.  So we have shown that $\phi^*([f_1],\ldots,[f_n])$ iff
there is a $[g]$ such that $\psi^*([g],[f_1],\ldots,[f_n])$.  This means that the existential quantifier over any $A_i$ in the base language translates to the existential quantifier over $A_i^U$ in the extended language (here we moved the quantified argument into first position, but it should be clear that we do not really lose any generality by doing this).

Note that if $\phi(x_1,\ldots,x_n)$ is $\psi(a,x_1,\ldots,x_n)$, then $\phi^*([f_1],\ldots,[f_n])$ is equivalent to $\{x \mid \psi(a,f_1(x),\ldots,f_n(x))\} \in U$, which is equivalent to $\psi^*(a^*,[f_1],\ldots,[f_n])$, which indicates that constants taken from domains $A_i$ behave naturally
in the extended language.

In the last two paragraphs, we have done manipulations on the first argument of an open sentence which can of course be done on any argument;
since we can change the indexing of the arguments (and so of the domains) of a fixed open sentence it should be clear that we do not lose generality.

Note finally that if $\phi(x_1,\ldots,x_n)$ is true for any assignment of values to the $x_i$'s from the appropriate $A_i$'s, then
$\{x \mid \phi(f_1(x),\ldots,f_n(x))\} = X \in U$ for any choice of $f_i$'s, so $\phi^*([f_1],\ldots,[f_n])$ is always true.  Translations of
general truths about the $A_i$'s hold true in the extended language over the $A_i^U$'s.


\end{description}

\section{Technical Methods for Consistency and Independence Proofs}

There is a political point to be made here: all of these things can be
done in type theory, quite naturally, and can thence be exported to
{\em NFU\/} without reference to the usual set theory.

Writing in fall 2017 for Math 522 development (in which some or all of these topics will be covered) my thinking is that I will certainly want to do these in untyped set theory;  but perhaps I should indicate the outlines of both approaches for the same reasons stated above.

\subsection{Frankel-Mostowski Methods; The Independence of Choice}  Possible Math 522 target.

\subsection{Constructibility and the Minimal Model of Type Theory}  Certainly a Math 522 target.

Build the Forster term model of type theory.  Also, prove the
consistency of CH and GCH (though this might get forced forward
after the logic section, because there is model theory involved.).

\subsection{Forcing and the Independence of CH}  Certainly a Math 522 target.

The treatment of constructibility in the previous subsection is
precisely that in the usual set theory (the fact that all the work is
done in $Z$ should make this clear.  Our treatment of forcing is
somewhat different from the treatement in the usual set theory: this
can be seen from the fact that it handles atoms, which the usual
techniques do not, and also from the fact that it {\em creates\/}
atoms.  The differences are technical: the basic idea is the same.
What we do show by this method is that it appears that it is not
necessary to do recursion along the cumulative hierarchy to do forcing
(as is commonly done).

\subsection{Generalizing the $T$ operation}

NOTE: this note might better belong somewhere else, but these
considerations are needed here.

Certain collections, such as the natural numbers, are ``the same'' in
each sufficiently high type.  This is usually witnessed by a $T$
operation.  Some collections on which a $T$ operation is defined get
larger at each type; these are of less interest to us here.

$T$ operations are defined on cardinals and on ordinals (more
generally on isomorphism types) already.  We point out that if we have
defined $T$ operations on sets $A$ and $B$, there is a natural way to
define a $T$ operation on ${\cal P}(A)$ (for $a \subseteq A$, define
$T^{{\cal P}(A)}(a)$ as $T^A``a$), on $B^A$ (so that
$T^{B^A}(f)(T^A(a)) = T^B(f(a))$, and on $A\times B$ (so that $T^{A
\times B}(\left<a,b\right>) = \left<T^A(a),T^B(b)\right>$).  We
superscript $T$ operations with their intended domains here for
precision:  we will not usually do this.

There is a uniform way to define $T$ operations on sets with a certain
kind of symmetry.  

\begin{description}

\item[Definition:] We call a bijection $f:V \rightarrow V$ a
{\em permutation of the universe\/}.  We use $\Pi$ as a nonce notation for the
set of all permutations of the universe.  Define $j(f)$ so that
$j(f)(x) = f``x$ for all $x$ ($j(f)$ is undefined on sets with
urelements as members).  Define $j^{\bf n}(f)$ in the obvious way.
Further, we define the operation $j^{\bf n}(\iota)$ similarly (with
due respect to the fact that $\iota$ is itself a type-raising
operation, but the definition works formally).  A set $A$ is {\em
$n$-symmetric\/} iff $j^{\bf n}(f)(A) = A$ for all permutations of the
universe $f$ of the appropriate type.  Notice that this implies that
$A \in {\cal P}^{\bf n}(V)$.  We define a $T$ operation on
$n$-symmetric objects $A$ for each $n$: $$T(A) = \{j^{\bf
n-1}(f)(j^{\bf n-1}(\iota)(a)) \mid a \in A \wedge f \in \Pi\}.$$

\item[Observation:] The generalized $T$ operation here would coincide
with all $T$ operations defined up to this point, if we used the
Kuratowski ordered pair, or if we presumed that the type-level ordered
pair coincided with the Quine ordered pair on sets and restricted all
use of pairing to sets of sets (as would happen if we assumed strong
extensionality).  For cardinal numbers are 2-symmetric, isomorphism
types are 4-symmetric if defined in terms of Kuratowski pairs and
2-symmetric if defined in terms of Quine pairs, and the definitions
given above for power sets, function spaces, and cartesian products
will coincide with appropriate $T$ operations of this kind on power
sets, function spaces and cartesian products (taking into account the
effect on the degree of symmetry of these set constructions).

\subsection{Forcing:  Basic Definitions}

We fix a definable partial order $\leq_P$ with field $P$ which
supports a $T$ operation with the property that $T``(\leq_P)=\leq_P$
(which of course implies that $T``P=P$). This is of course a pun: what
is being said is that the definition of $P$ with all types raised by
one will give the image under the $T$ operation of the original
partial order $P$.  Such an order $P$ will be defined and essentially
``the same'' structure in all types above a certain level.

The set $P$ will be in some sense the space of ``truth values'' for
the forcing interpretation.  Each element of $\leq_P$ represents an
(incomplete) ``state of information''; the relation $p \leq_P q$ tells
us that the state of information described by $q$ extends the state of
information described by $p$ (the opposite convention is often used!).
If neither $p \leq_P q$ nor $q \leq_P p$, the states of information
described by $p$ and $q$ are to be understood to be incompatible.

``The objects of type $n$'' of our forcing interpretation are
relations $x$ from $V^{\bf n}$ to $P$, that is, subsets of $V^{\bf
n}\times P$, with the property that $\left<y,p\right> \in x \wedge q
\geq_P p \rightarrow \left<y,q\right>$.  Notice that the type $n$
objects of the forcing model are actually certain type $n+1$ objects.
The type $n+1$ objects which will be interpreted as type $n$ objects
are called {\em names\/}.  Those familiar with treatments of forcing
in the usual set theory should notice that we are {\em not\/}
requiring names to be relations from {\em names\/} to elements of $P$:
this would introduce a recursion on the type structure, which is
something always to be avoided in type theory.  We will see below how
difficulties which might be supposed to arise from this freedom in the
construction of names are avoided.

The central definition of the forcing interpretation is the definition
of a notation $p \vdash \phi$ for formulas $\phi$ of type theory,
which is intended to tell us when a condition $p$ gives us sufficient
information to decide that an assertion $\phi$ is true.

The central theorem of the forcing interpretation will be that $p
\vdash \phi$ is true for each axiom $\phi$, that $p \vdash \phi$ can
be deduced from $p \vdash \psi$ whenever $\phi$ can be deduced from
$\psi$ by a rule of logic.  It will further be clear that we cannot
prove $\neg(p \vdash \phi \wedge \neg\phi)$ (unless we can prove a
contradiction in type theory itself).  It is very important to notice
that this is not metamathematics: $p \vdash \phi$ is not an assertion
about a mathematical object '$\phi$' coding the assertion $\phi$ as in
the development of G\"odel's theorem or Tarski's theorem, and we are
not building a set model of type theory (this cannot be done in type
theory by those very theorems!).  Of course we may associate with set
models of type theory (if there are any) set models of type theory
generated by applying a forcing interpretation to those set models,
and this will be of some interest.

\begin{description}

\item[Definition:] We define $${\mathbb N}_P = \{ x \in {\cal P}(V
\times P)\mid (\forall y.(\forall p \in P.(\forall q \geq_P
p.\left<y,p\right> \in x \rightarrow \left<y,q\right>\in x)))\}$$ as
the set of {\em $P$-names\/}.  We define the notation $p \vdash \phi$
recursively.  We suppose all logical operators defined in terms of
$\wedge, \neg, \forall$.

\begin{description}

\item[negation:] $p \vdash \neg\phi$ is defined as $(\forall q \geq_P
p.\neg(q \vdash \phi))$.  Informally, ``no matter how much information
we add to $p$, we will not verify $\phi$''.

\item[conjunction:] $p \vdash \phi \wedge \psi$ is defined as $(p
\vdash \phi) \wedge (p \vdash \psi)$.  This appears simple enough, but
one should note that if one expands out the definition of disjunction
or implication in terms of the given definitions of negation and
conjunction one does not get this nice distributivity.

\item[universal quantification:] $p \vdash (\forall x.\phi)$ is
defined as $(\forall {\bf x} \in {\mathbb N}_P.p \vdash \phi[{\bf
x}/x])$.  Again, this definition looks very direct, but it is
instructive to analyze the expansion of $p \vdash (\exists
x.\phi[x])$.

\item[pseudo-membership:] (this will not be the interpretation of
membership, for reasons that will become evident, but it makes the
definition easier): for any $x, y$, $p \vdash x \in^*
y$ iff $y \in {\mathbb N}_P \wedge (\forall q \geq_P T^{-1}(p).(\exists
r \geq_P q.\left<x,r\right> \in y))$.  Note the necessity of the
introduction of the $T$ operator so that we have a well-formed
assertion of type theory.  Note also that $x$ here is any object at
all (of appropriate type) while $y$ is a name of the next higher type.

Pseudo-membership does not officially appear in formulas of our
language; this notation is used only in the definitions of equality
and membership for the forcing interpretation.

\item[equality:] Let $x$ and $y$ be names.  $p \vdash x=y$ is defined
as $$(\forall z.(p \vdash z \in^* x) \leftrightarrow (p \vdash z \in ^*
y)).$$  Names are asserted to be equal as soon as we have enough
information to see that they have the same pseudo-members.

\item[sethood:] $p \vdash {\tt set}(x)$ is defined as $$(\forall y.(p
\vdash y \in^* x) \rightarrow y \in {\mathbb N}_P \wedge (\forall z.(p
\vdash y = z) \rightarrow (p \vdash z \in^* x))).$$ $p$ says that $x$
is a set iff anything that $p$ thinks is a pseudo-element of $x$ is a
name and any name that $p$ thinks is equal to an pseudo-element of $x$
$p$ also thinks is an pseudo-element of $x$.  We will see that under
these conditions we can drop the ``pseudo-''.

\item[membership:]  $p \vdash x \in y$ is defined as $(p \vdash x \in^* y) \wedge (p \vdash {\tt set}(y))$.

The idea here is that we convert the names whose pseudo-extension does
not respect equality to urelements.  This is how we avoid recursion on
type in our definitions (along with the fact that we use typically
ambiguous partial orders on forcing conditions).

\item[type-shifting convention:] Notice that in atomic formulas we
have $p$ at the same type as the highest type of one of the arguments.
Hereafter we stipulate $p \vdash \phi$ iff $T(p) \vdash \phi$; the
type of $p$ may freely be shifted.  It would otherwise be difficult to
type conjunctions, and it should be clear that this will introduce no
conflicts.

NOTE: in the context of NF(U) this will be clear if the set $P$ is
strongly cantorian.  What can be done (if anything) with cantorian
partial orders needs to be cleared up [when it is cleared up the exact
way we proceed here might need to be modified].



\end{description}

\end{description}



\end{description}

\chapter{Saving the Universe:  Stratified Set Theories}

This section concerns a class of untyped set theories which are
related to type theory (as Zermelo set theory and {\em ZFC\/} also
are) but in a different way.  The first theory of this class was
introduced by Quine in his ``New foundations for mathematical logic''
(1937) and so is called {\em NF\/}, which is short for ``New
Foundations''.  {\em NF\/}, as we shall see, is a very strange theory
for rather unexpected reasons.  We shall ignore historical precedent
and start by introducing {\em NFU\/} (New Foundations with
urelements), which is much more tractable.  {\em NFU\/} was shown to
be consistent by R. B. Jensen in 1969.

Most of the theories of this class share the perhaps alarming
characteristic that they assert the existence of a universal set.

\section{Introducing {\em NFU\/}}

The starting point of the line of thought which led Quine to ``New
Foundations'' but which will lead us first to {\em NFU\/} (due to
careful planning) is an observation which we have already exploited.
The types of our type theory are very similar to one another (in terms
of what we can prove).  We have used this observation to avoid
cluttering our notation with endless type indices.  We begin by
carefully stating the facts already known to us (at least implicitly)
about this ambiguity of type and considering some extrapolations.

\subsection{Typical Ambiguity Examined}

If we suppose that each variable $x$ in the language of our type
theory actually comes with a type index ($x^{\bf n}$ is the shape of
the typical type $n$ variable), we can define an operation on
variables: if $x$ is a variable of type $n$, we define $x^+$ as the
variable of type $n+1$ obtained by incrementing the type index which
$x$ is supposed to have (though we continue our convention of not
expressing it).  This allows us to define an operation on formulas: if
$\phi$ is a formula of the language of type theory, we define $\phi^+$
as the result of replacing every variable $x$ (free or bound) in
$\phi$ with the type-incremented $x^+$.  The same operation can be
applied to terms: $\{x \mid \phi\}^+ = \{x^+ \mid \phi^+\}$, and
$(\epsilon x.\phi)^+ = (\epsilon x^+.\phi^+)$.

Our first observation is that for any formula $\phi$, $\phi^+$ is also
a formula, and for any term $T$, $T^+$ is also a formula.  The
converse is also true.  Further, if $\phi$ is an axiom, $\phi^+$ is
also an axiom (in fact, the converse is also true).  Further, if
$\psi$ can be deduced from $\phi$ by any logical rule, $\psi^+$ can
also be deduced from $\phi^+$, whence it follows that if $\phi$ is a
theorem of type theory, $\phi^+$ is also a theorem of type theory.  In
this case, the converse is not necessarily the case, though the
converse does hold in {\em TNT\/}.  This means that anything we know
about a particular type (and a number of its successors) is also true
in each higher type (and a number of its corresponding, appropriately
type-shifted successors).  Further, any object we can construct in
type theory has a correlate constructed in the same way at each higher
type.  We have exploited this phenomenon, which Whitehead and Russell
called ``systematic ambiguity'' in the more complex system of their
{\em Principia Mathematica\/}, which most workers in the area of {\em
NF\/} now call ``typical ambiguity'', and which is a rather extreme
example of what computer scientists call {\em polymorphism\/}, to make
it almost completely unnecessary to mention specific type indices in
the typed set theory section of this book.

Quine made a daring proposal in the context of a type theory similar
to ours (in fact, differing only in the assumption of strong
extensionality).  He suggested that it is not just the case that
provable statements are the same at each type, but that the same
statements are true in each type, and that the objects at the
different types with correlated definitions do not merely serve as the
subjects of parallel theorems but are in fact the same objects.  The
theory which results if this proposal is applied to our type theory is
an untyped set theory, but rather different from the theory of Zermelo
developed above.

In this theory we have not a universal set $V^{\bf n+1} = \{x^{\bf n}
\mid x^{\bf n} = x^{\bf n}\}$ for each $n$, but a single set $V = \{x
\mid x=x\}$.  We have already shown that it follows from the Axiom of
Separation of Zermelo set theory that there can be no such set $V$
(whence it follows that if this new theory is coherent it does not
satisfy the Axiom of Separation).  We do not have a $3^{\bf n + 1}$
which contains all the three-element sets of type $n$ objects, but a
single object 3 which is the set of all three-element sets.

We will give the precise definition of this theory in the next
section.  What we will do now is prove a theorem due to Specker which
will make the connections between various forms of typical ambiguity
clearer.  For the rest of this section, we discuss theories framed in
languages in which variables are typed and which satisfy the condition
that for any formula $\phi$ is well-formed if and only if $\phi^+$ is
well-formed.  Further, we require that the language of the theory be
closed under the basic logical operations familiar to us from above,
and that whenever the rules allow us to deduce $\phi$ from $\psi$
[neither formula mentioning any maximum type] we are also able to
deduce $\phi^+$ from $\psi^+$.  It is required that every context in
which a term can occur dictates the type of that term exactly.

We consider the following suite of axioms.

\begin{description}

\item[Ambiguity Scheme:] For each sentence $\phi$ (formula with no
free variables) for which $\phi^+$ is well-formed, $\phi
\leftrightarrow \phi^+$

\end{description}

With any theory $T$ in typed language, we associate a theory
$T^{\infty}$ whose sentences are simply the sentences of $T$ with all
type distinctions removed.  A model of $T^{\infty}$, if there is one,
is a model of the typed theory $T$ in which all the types are actually
the same.  Notice that $T^{\infty}$ is automatically the same as $(T +
Amb)^{\infty}$, where $Amb$ is the ambiguity scheme above, because
$Amb^{\infty}$ is a set of tautologies.

Note that the language of $T^{\infty}$ allows things to be said which cannot be said in the typed language of $T$:  sentences like $a \in a$ are well-formed,
and a completion of a consistent $T^{\infty}$ would assign truth values to such sentences.

\begin{description}

\item[Theorem (Specker):] For any theory in typed language which is
well-behaved in the ways outlined above, $T^{\infty}$ is consistent
iff $T + Amb$ is consistent.

\item[Proof:] It is obvious that the consistency of $T^{\infty}$
implies the consistency of $T+Amb$.

Suppose that $T + Amb$ is consistent.  Our goal is to show that
$T^{\infty}$ has a model.  We first observe that this is obvious if
the language of $T$ contains the Hilbert symbol (or any construction
with equivalent logical properties).  For $T+Amb$, being consistent,
can be extended to a complete theory, which has a model consisting
entirely of closed terms $T$ built using the Hilbert symbol.  We can
then identify the term $T$ with the term $T^+$ for every $T$.  No
conflict can occur: any assertion $\phi(T)$ has the same truth value
as $\phi^+(T^+)$ (and these identifications and equivalences can be
indefinitely iterated) [and no weird variants such as $\phi^+(T)$ are
meaningful].  The truth value of $\phi(a_1,\ldots,a_n)$ with any
Hilbert symbol arguments $a_i$ however weirdly typed can be
established by raising the type of $\phi$ sufficiently high that the
types expected for its arguments are higher than the types of any of
the $a_i$'s then raising the types of the arguments $a_i$ to the
correct types, then evaluating this well-typed formula.

To complete the proof we need to show that any typed theory $T+Amb$
can be extended to include a Hilbert symbol in a way which preserves
the truth of all sentences and allows $Amb$ to be extended to the new
sentences.  Since $T+Amb$ is consistent, we can suppose it complete.
We list all Hilbert symbols, stipulating that a Hilbert symbol must
appear after any Hilbert symbol which occurs as a subterm of it in the
list.  We assume that before each Hilbert symbol is introduced we have
a deductively closed theory which contains all instances of $Amb$
appropriate to its language (i.e., not instances which mention Hilbert
symbols not yet introduced).  We introduce the Hilbert symbol
$a=(\epsilon x.\chi[x])$.  We then find a maximal collection of
sentences $\phi[a]$ which includes $\chi[a]$, contains all type-raised
copies of its elements, and is consistent.  For any conjunction $\Phi$ of these
sentences we have $(\exists x.\Phi^{+^i}(x))$ consistent for any $i$, so we
can consistently add all $\phi^{+^i}[a^{+^i}]$ to our theory.

We now assume that we have a
complete set of sentences $\Phi[a,a^+,\ldots,a^{+^k}]$ consistent with
our theory and closed under + (we have just dealt with the base case
$k=1$).  We show that we can get a complete set of sentences
$\phi[a,a^+,\ldots,a^{+^{k+1}}]$ consistent with our theory and closed
under +.  Suppose $\psi[a,a^+,\ldots,a^{+^{k+1}}]$ is a sentence which
we wish to consider.  We consider the status of sentences $(*):
(\exists x.\psi[a,a^+,\ldots,a^{+^{k}},x] \wedge
\Phi^+[a^+,\ldots,a^{+^{k}},x])$ and $(*^{\neg}): (\exists
x.\neg\psi[a,a^+,\ldots,a^{+^{k}},x] \wedge
\Phi^+[a^+,\ldots,a^{+^{k}},x]$), where the $\Phi^+[a^+,\ldots,a^{+^{k}},x]$ represents type shifted versions of as large a conjunction of sentences from the complete set $\Phi$ as desired, which are already decided in our
theory (because they mention blocks of $k$ successive type-shifted
versions of $a$).  We see that if
$\psi[a,a^+,\ldots,a^{+^{k}},a^{+^{k+1}}]$
(resp. $\neg\psi[a,a^+,\ldots,a^{+^{k}},a^{+^{k+1}}]$ is consistent
with our theory then this statement must have already been decided as
true (otherwise we would be able to disprove
$\psi[a,a^+,\ldots,a^{+^{k}},a^{+^{k+1}}]$
(resp. $\neg\psi[a,a^+,\ldots,a^{+^{k}},a^{+^{k+1}}]$) from prior
assumptions).  This means that we can extend the sequence of $k$ type
shifted versions of $a$ with a new term in such a way that the ``type
shifted sequence'' starting with $a^+$ and extended with $x$ has as
many of the known properties of blocks of $k$ type shifted versions of
$a$ as we want, and the sequence of $k+1$ elements satisfies $\psi$
(resp. $\neg\psi$).  These properties can include the ability
(expressed in the formula $(*)$ (resp. $(*^{\neg})$ above, which can
be used to extend $\Phi$) to further extend the sequence as many times
as desired, while also preserving the property that blocks of $k+1$
elements of the extended sequence satisfy (type shifted versions of)
$\psi$ (resp. $\neg \psi$).  Compactness then tells us that we can
assume that all blocks of $k+1$ type shifted versions of $a$ satisfy
$\psi$ (resp. $\neg\psi$).  This means that we can proceed (again by
compactness) to find a maximal collection of consistent sentences
$\psi[a,a^+,\ldots,a^{+^{k}},a^{+^{k+1}}]$ such that the closure of
this set under + is consistent with our previous theory.  Repeating
this process for all $k$ gives us a theory with the new Hilbert symbol
adjoined which extends $Amb$ as desired.  Repeating this process for
all Hilbert symbols gives the desired extension of $T+Amb$ with
Hilbert symbols, and with the scheme $Amb$ extended appropriately to
Hilbert symbols.

\end{description}

\subsection{Definition and Consistency of {\em NFU\/}}

We refer to the typed theory of sets which is our working theory as {\em TSTU\/} (excluding for the moment the axioms of Infinity, Ordered Pairs, and Choice).  We refer to
{\em TSTU\/} + strong extensionality as {\em TST\/}.  We define {\em NFU\/} (for the moment) as {\em TSTU$^{\infty}$\/}, and define {\em NF\/}  (``New Foundations") as {\em TST$^{\infty}$\/}.

In this section we will expand a bit on how to understand the theory {\em NFU\/}, prove its consistency, and observe that the method of proof extends to a stronger theory which we will then make the referent of the name {\em NFU\/}.

{\em NFU\/} is an untyped set theory, like the theories of chapter 4.  The axioms of {\em NFU\/} are exactly the axioms obtained from axioms of Extensionality and Comprehension of {\em TSTU\/} by disregarding all distinctions of type between the variables.
Impossible axioms like $\{x \mid x \not\in x\}$ do not appear as instances of Comprehension because $x \not\in x$ is not the shape of any formula of the language of {\em TSTU\/}:  we drop the type distinctions, but this does not introduce identifications between variables.

We recapitulate the axioms of {\em NFU\/}.

\begin{description}

\item[Primitive notion:] There is a designated object $\emptyset$ called the {\em empty set\/}.

\item[Axiom of the empty set:] $(\forall x.x\not\in \emptyset)$.

\item[Definition:] We say that an object $x$ is a
{\em set\/} iff $x = \emptyset \vee (\exists y.y \in x)$.  We write
${\tt set}(x)$ to abbreviate ``$x$ is a set'' in formulas.  We say
that objects which are not sets are {\em atoms} or {\em urelements\/}.

\item[Axiom of extensionality:] $$(\forall xy.{\tt set}(x) \wedge {\tt
set}(y) \rightarrow x=y\leftrightarrow (\forall z.z\in x
\leftrightarrow z\in y)),$$

\end{description}

In these axioms, the only changes we make are complete omission of references to types and type indices.  The comprehension axiom is trickier.

\begin{description}
\item[*Axiom of comprehension:] 

For any formula $A[x]$ obtained by ignoring type distinctions in a formula of the language of type theory in which the
variable $y$ (of type one higher than $x$) does not appear, $$(\exists
y.(\forall x.x \in y \leftrightarrow A[x])).$$

\end{description}

We star this because it is not the form of the axiom we will use.

\begin{description}

\item[Definition:]  A formula $\phi$ of the language of set theory is said to be ``stratified" iff there is a function $\sigma$ (called a {\em stratification\/} of $\phi$)
from variables to natural numbers (or, equivalently, integers) such that for each atomic formula $x=y$ appearing in $\phi$ we have $\sigma(x)=\sigma(y)$ and for
each atomic formula $x \in y$ appearing in $\phi$ we have $\sigma(x)+1 = \sigma(y)$.  Note that for a formula in equality and membership alone, to be stratified
is precisely equivalent to being obtainable from a formula of the language of type theory by ignoring type distinctions

\item[Axiom of stratified comprehension:]  For any stratified formula $A[x]$ in which the variable $y$ does not appear, $$(\exists
y.(\forall x.x \in y \leftrightarrow A[x])).$$

\end{description}

The axiom of extensionality tells us that there is only one such
object $y$ which is a set (there may be many such objects $y$ if
$A[x]$ is not true for any $x$, but only one of them ($\emptyset$)
will be a set). This suggests a definition:

\begin{description}

\item[Set builder notation:] For any stratified formula $A[x]$, define $\{x \mid
A[x]\}$ as the unique {\em set\/} of all $x$ such that $A[x]$: this exists by
Comprehension and is uniquely determined by Extensionality. 

\end{description}

We show that {\em NFU\/} is consistent.  We have shown above that it suffices to demonstrate that {\em TSTU\/}+Amb is consistent.

Let $\Sigma$ be any finite collection of sentences of the language of {\em TSTU\/}.  Let $n$ be chosen so that $\Sigma$ mentions only types $0-(n-1)$.  Choose a sequence of sets
$X_i$ such that $|{\cal P}(X_i)| \leq |\iota``X_{i+1}|$ for each $i$.  Choose injective maps $f_i:{\cal P}(X_i) \rightarrow \iota``X_{i+1}$ for each $i$ and define relations
$x \in_i y$ as $x \in X_i \wedge y \in X_{i+1} \wedge x \in f_i^{-1}(\{y\})$  (where of course this is understood to be false if  $f_i^{-1}(\{y\})$ is undefined).  It is easy to see
that the resulting structure is a model of TSTU:  the interpretation of a sentence of TSTU is obtained by replacing each type $i$ variable with a variable restricted to $X_i$,
and replacing each occurrence of $\in$ in an atomic formula $x \in y$ with $x \in _i y$, where $i$ is the type of $x$.  It should be easy to see that the interpretation of each axiom is true.  Notice that this construction is carried out in our type theory,
with the types of all the elements of the $X_i$'s being the same fixed type whose identity does not matter for our purposes.

Now observe further that for any strictly increasing sequence $s$ of natural numbers, the sequence $X^s$ defined by $X^s_i = X_{s_i}$ determines an interpretation of TSTU in exactly the same way.
We observe that the sentences $\Sigma$ determine a partition of the $n$-element sets $A$ of natural numbers as follows:  consider a sequence $s$ such that $s``\{0,\ldots,n-1\}=A$ and note the truth values
of the sentences of $\Sigma$ in the models $X^s$ (which will be entirely determined by the first $n$ terms of $X^s$).  This is  a partition of the $n$ element subsets of $\mathbb N$ into no more than $2^{|\Sigma|}$ parts, which by Ramsey's theorem has
an infinite homogeneous set $H$.  Now consider any $X^s$ such that $s``{\mathbb N} \subseteq H$:  the interpretations of all sentences $\phi \leftrightarrow \phi^+$ for $\phi$ in the axiom scheme Amb will be true in such models.  We have shown that every finite subset of Amb is consistent with TSTU,
so by Compactness TSTU + Amb is consistent, so by Specker's theorem on ambiguity, {\em NFU\/} is consistent.

We have used more mathematical power than we need here.  We have assumed in effect that $\beth_{\omega}$ exists (because we assume the existence of an infinite sequence $X_i$).  This is not strictly necessary:  we can use a more refined form of Ramsey's theorem and
show the existence of homogeneous sets of sufficient size in sufficiently long finite sequences of $X_i$'s.  However, we do not regard the existence of $\beth_{\omega}$ as a dubious assumption.

The method of proof used here extends to any extension of {\em TSTU\/} with ambiguous axioms.  For example {\em NFU\/} + Infinity + Choice is shown to be consistent by this argument.   Further, we can add the axiom of Ordered Pairs as well:  add predicates $\pi_1$ and $\pi_2$ with the additional rules
that typing for formulas $x \, \pi_i \, y$ follows the same rules as typing for formulas $x=y$ and additional axioms $(\forall x.(\exists! y.x \pi_i y))$ (each $\pi_i$ is a function of universal domain) and $(\forall xy.(\exists! z.z \pi_1 x \wedge z \pi_2 y))$.  These axioms hold in our working theory,
and can be made to hold in the $X_i$'s by stipulating that each $X_i$ is infinite and providing bijections $\Pi:(X_i \times X_i) \rightarrow X_i$ for each $i$, and interpreting $x \pi_j y$ between type $i$ objects as holding iff $y=\pi_j(\Pi_i(x))$.

Hereinafter we will usually mean {\em NFU\/} + Ordered Pairs + Choice when we refer to NFU.

We further note that a weaker form of stratification can be used.  We say that a formula $\phi$ is {\em weakly stratified\/} iff
the formula $\phi'$ is stratified which is obtained by replacing each occurrence of each variable free in $\phi$ with a distinct variable.  Another way of putting this is that there is a function $\sigma$ satisfying the conditions for a stratification, but only in atomic formulas in which both variables are bound.  The reason that stratified comprehension entails weakly stratified comprehension is that the existence of each set $\{x \mid \phi\}$
is a special case of the existence of the sets $\{x \mid \phi'\}$ (existing by stratified comprehension)  in which certain variables free in $\phi'$ (and so implicitly universally quantified in the axioms of comprehension in question) happen to take on the same values.   An example:  the set $\{x,\{y\}\}$
exists for each value of $x,y$ (an instance of stratified comprehension) so the set $\{x,\{x\}\}$ exists for each $x$ (an instance of weakly stratified comprehension).

We further note that stratification can be extended to a language with terms, if a stratification must take the same value at $(\epsilon x.\phi)$ that it does at $x$ (the structure of $\phi$ then dictating type differentials between $x$ and any parameters in the term), and noting that any term
construction can be supposed implemented by a Hilbert epsilon term.  This can be handled in the consistency proof by fixing choice functions to identify referents of Hilbert epsilon terms in the $X_i$'s.

This proof allows us to bootstrap our working theory from {\em TSTU\/} with Ordered Pairs and Choice to {\em NFU\/} with Ordered Pairs and Choice, if we are so inclined:  we can adopt the view that the types of our theory, which are suspiciously similar because we have been careful to keep our methods of proof
over them entirely uniform, are in fact all the same domain.  We will explore the consequences of taking this perhaps odd view.

(NOTE:  we certainly want to consider the Boffa model construction as well.  For this we need enough model theory to get models with automorphisms.)


\subsection{Mathematics in {\em NFU\/}}

(NOTE:  Counting is so useful that it might show up in the base development.)

We do not start with a clean slate when we consider doing mathematics in {\em NFU\/}, because all the mathematics we have done in {\em TSTU\/} can be imported.  However, the interpretation of {\em NFU\/} is different in interesting ways.

The language of {\em NFU\/} is larger.  Sentences such as $x \in x$ are well-formed as they are not in typed language.  Further, a sentence like $V \in V$ which we wrote but construed as a sort of pun in typed language is to be taken seriously in 
{\em NFU\/}:  the universal set $V$ has {\em everything\/} as an element, including itself.  From this it follows that $(\exists x. x \in x)$ is a theorem of {\em NFU\/}, since the universal set is a witness.

We have proved Cantor's theorem $|\iota``A|<|{\cal P}(A)|$ which tells us that the power set of $A$ is larger than $A$.  But in {\em NFU\/} we of course know that ${\cal P}(V) \subseteq V$.   This does not contradict anything we proved in type theory, because in type theory the referents of the two
$V$'s are not supposed to be the same.  In {\em NFU\/} Cantor's Theorem tells us that $|\iota``V| < |{\cal P}(V)| \leq |V|$, so we see that the singleton map (which from an external standpoint we can see is a one-to-one correspondence) cannot be a set in {\em NFU\/}.

The unstratified form of Cantor's Theorem which is true in the untyped set theories of chapter 4 cannot hold in general in {\em NFU\/}, but it can hold under special circumstances.

\begin{description}

\item[Definition:]  A set $A$ is said to be {\em cantorian\/} iff $|A| = |\iota``A|$.

\end{description}

This is precisely what is needed to get the unstratified theorem ``if $A$ is a cantorian set, $|A|=|\iota``A|<|{\cal P}(A)|$".  We see that all cantorian sets are smaller than their power sets.  Consideration of how this fact is witnessed suggests a stronger property.

\begin{description}

\item[Definition:]  A set $A$ is said to be {\em strongly cantorian\/} iff $(\iota \lceil A) = \{(a,\{a\}) \mid a \in A\}$ is a set.

\end{description}

Obviously a strongly cantorian set is cantorian.  The stronger property has considerably stronger consequences.

What all of this already tells us is that a model of {\em NFU\/} is not a model of {\em TSTU\/} of the natural kind in which every collection of type $i$ objects is a type $i+1$ object.  Every element of the non-function $\iota = \{(x,\{x\}) \mid x \in V\}$ is an object in our model of
{\em NFU\/}, but the collection of all these pairs cannot be an element of the model on pain of contradiction.

We give a much sharper result of the same kind.  We proved above that $T^2(\Omega) < \Omega$ (recall that $\Omega$ is the order type of the ordinals).  In {\em TSTU\/} this assertion was a kind of pun, but here all references to $\Omega$ are references to the same object.
It is straightforward to prove that $\alpha<\beta \leftrightarrow T(\alpha)<T(\beta)$, from which it follows that $\Omega > T^2(\Omega) > T^4(\Omega)>T^6(\Omega)>\ldots$.  This observation has two different rather alarming consequences.  One is that a certain {\em countable\/}
collection of objects of a model of {\em NFU\/} cannot be a set:  if the smallest collection containing $\Omega$ and closed under $T^2$ were a set, it would be a set of ordinals with no smallest element, which is impossible.  The other is that from a certain external standpoint, the ordinals
of a model of {\em NFU\/} are not well-ordered.

We investigate the mathematics of the properties ``cantorian" and ``strongly cantorian".

\begin{description}

\item[Theorem:]  Concrete finite sets are cantorian.  Power sets of cantorian sets are cantorian.  Cartesian products of cantorian sets are cantorian.  Function spaces from cantorian sets to cantorian sets are cantorian.


\item[Proof:]  Sets of concrete finite sizes are obviously the same size as their images under the singleton operation.  We will find that asserting this for all finite sets is a stronger assertion than we can prove from our current axioms.  The other assertions follow
from the existence of bijections between ${\cal P}(\iota``A)$ and $\iota``({\cal P}(A))$, between $\iota``A \times \iota``B$ and $\iota``(A \times B)$ and between $\iota``B^{\iota``A}$ and $\iota``(B^A)$:  from the ability to define these maps it clearly follows
that if $A$, $B$ are the same size as $\iota``A$, $\iota``B$, respectively, then ${\cal P}(A)$, $A \times B$, $B^A$ are the same size as  $\iota``{\cal P}(A)$, $\iota``(A \times B)$, $\iota``(B^A)$, respectively, which is what is to be shown.

\item[Theorem:]  Concrete finite sets are strongly cantorian.  Power sets of cantorian sets are strongly cantorian.  Cartesian products of cantorian sets are strongly cantorian.  Function spaces from cantorian sets to cantorian sets are strongly cantorian.

\item[Proof:]  If $A$ is a concrete finite set, $(\iota \lceil A)$ can be given as a concrete finite set.  Again, showing that this is true for all finite sets turns out not to be provable with our current axioms.  Construct $(\iota\lceil {\cal P}(A))$ as $(B:{\cal P}(A) \mapsto (A : {\cal P}(\iota``V) \mapsto \{\bigcup A\})((\iota \lceil A)``B))$.  Construct $(\iota\lceil(A \times B))$ as $((a,b): A \times B \mapsto ((\{x\},\{y\}):(\iota``V) \times (\iota``V) \mapsto \{(x,y\})((\iota\lceil A)(a),(\iota\lceil B)(b)))$.  We leave the similar construction of $(\iota \lceil B^A)$ as an exercise.

\item[Theorem:]  A subset of a strongly cantorian set is strongly cantorian.

\item[Proof:]  If $B \subseteq A$, $(\iota \lceil B) = (\iota \lceil A) \lceil B$.

\end{description}

The last theorem is one reason why ``strongly cantorian" is a much stronger property.  Here is a further, more profound reason.

\begin{description}

\item[Subversion Theorem:]  Let $\phi$ be a formula in which some quantified variables are restricted to strongly cantorian sets.
Let $\phi'$ be the formula obtained by replacing each occurrence of each variable bounded in a strongly cantorian set $A$ with a distinct variable bounded in $A$
(replacing single universal quantifiers over $A$  with blocks of universal quantifiers over $A$ or single existential quantifiers over $A$ with blocks of existential quantifiers over $A$ as needed).
If $\phi'$ is stratified then $\{x \mid \phi\}$ exists.  Equivalently, if there is a function which meets the conditions to be a stratification
of $\phi$ in each atomic subformula containing two bound variables neither of which is bounded in $A$, then $\{x \mid \phi\}$ exists.

\item[Proof:]  The formula $\phi'$ can be modified in such a way as to change the value assigned to a variable $a$ restricted to the strongly cantorian set
$A$ freely.  Let $\iota_A$ represent the singleton map restricted to $A$, for each of the strongly cantorian sets $A$ appearing as bounds of quantifiers in $\phi$.   To raise the type assigned to $a$ by one, replace $a$ with the term
$(\epsilon x.x \in \iota_A(a))$.  To lower the type assigned to $a$ by one, replace $a$ with $\iota_A^{-1}(\{a\})$.  Now each variable in $\phi'$
which is bounded in a strongly cantorian set $A$ can be assigned a type in such a way that the desired additional equations between variables needed to give equivalence with $\phi$ can be adjoined while preserving stratification.

\end{description}

NOTE:  other specifically NFU mathematics include unstratified inductive definitions (von Neumann ordinals, notions of well-foundedness, etc.)
and T-sequences and related ideas.

\subsection{There are Urelements}



\section{Extensions of {\em NFU\/}}

\subsection{The Axiom of Counting; $\omega$-Models.}

But perhaps Counting will be covered in the first part?

unstratified induction?  The $\omega$-model construction;
$\alpha$-models; NFU$^*$.

\subsection{The Axiom of Cantorian Sets; the Axiom of Large Ordinals}

this will provide an occasion for $T$-sequences.  Interpretation of
{\em ZFC\/} in this theory (cute eliminations of $T$).  $n$-Mahlos,
fancy partition relations, model theory.

\subsection{The Axiom of Small Ordinals; the BEST model}

ASO with and without CS and Large Ordinals.  weakly compact; nearly
measurable.  Solovay stuff.  The BEST model.

\section{The Extensional Subsystems}

\subsection
{Ambiguity in Theories with Finitely Many Types; {\em NF$_3$\/}}

Our type theory {\em TSTU\/} has natural subtheories defined simply by
restricting the number of types.  Similar considerations apply to
variants of our type theory.

\begin{description}

\item[Definition:] {\em TSTU$_n$\/} is defined as the subtheory of
{\em TSTU\/} with type indices $\geq n$ excluded from the language.
Other type theories will have subscripted variants defined in the same
way.

\end{description}

The situation in three types is very special.

\begin{description}

\item[Theorem:] For any infinite model of {\em TSTU$_3$\/} with either
the same concrete finite number of atoms at each type or infinitely
many atoms at each type, there is a model of {\em TSTU$_3$\/}$^\infty$
with exactly the same theory.

\item[Proof:] By model theory, there is a countable model of {\em
TSTU$_3$\/} with the same theory.  We want a further refinement: we
want a countable model with the property that each infinite set can be
partitioned into two infinite sets.  Suppose our initial countable
model lacks this property: there are then infinite sets which can only
be partitioned into finite and cofinite pieces.  Construct an
ultrapower of the model using an ultrafilter on the natural numbers.
This will give a model of the theory with the splitting property (but
not a countable one).  Build a countable model with the same theory as
this model, but being sure to include some specific constant
(referring to a set of nonstandard finite size) in your theory.  The
resulting model will be countable, will have the splitting property
(because we will have partitions of any infinite set with one
partition of the fixed nonstandard size), and will have exactly the
same theory as the original model (if we exclude references to the
special constant from our language).

Now we show that in any countable model of {\em TSTU\/} there is an
isomorphism between types $0-1$ and types $1-2$.  First of all, the
conditions in the statement of the theorem combined with the
countability of the model are enough to ensure that we have a
bijection from the type 1 atoms onto the type 2 atoms.  Now we handle
the sets.  We fix an order on the type 1 sets and an order on the type
2 sets, each of type $\omega$.  When we have mapped the first $n$ sets
of type 1 to sets of type 2, and also the first $n$ sets of type 2
have been assigned inverse images in type 1, we assume that we have
matched them in such a way that the sizes of the corresponding
compartments in Venn diagrams determined by the type 1 sets assigned
images and the type 2 sets assigned inverse images is correct: for any
intersection of the type 1 sets and their complements, if the
intersection is of concrete finite size $n$ the corresponding
intersection of type 2 sets and their complements will be of the same
concrete finite size $n$, and if the intersection is (countably)
infinite the corresponding intersection of the type 2 sets and their
complements will be countably infinite.  We show how to continue this
process (note that the conditions are vacuously satisfied initially).
Match the first set of type 1 not yet assigned an image with the first
set in the order on type 2 sets which has not yet been matched and has
the correct intersection sizes with the correlates of all finite
intersections of the previously mapped type 1 sets.  The splitting
property is needed here to ensure that if the new type 1 set has
infinite and co-infinite intersection with one of the compartments of
the Venn diagram determined by the previous set that we can choose a
type 2 set with appropriate intersection sizes to associate with it.
Choose an inverse image for the first type 2 set as yet not assigned
an inverse image in exactly the same way.  Notice that the map between
types 1 and 2 determines a map between types 0 and 1 by considering
singletons.  Note that the amount of comprehension needed in the type
theory considered is very limited: all that is needed is existence of
singletons, complements and finite unions.

If $f$ is the isomorphism, we take type 0 as the model and define $x
\in_N y$ as $x \in_M f(y)$ (where $\in_M$ is the membership relation
of the model.  Note that for any $x^{\bf 0} \in_M y^{\bf 1}$ we have
$x^{\bf 0} \in_M f^{-1}(y^{\bf 1})$ equivalent and for any $x^{\bf 1}
\in_M y^{\bf 2}$ we have $f^{-1}(x^{\bf 1}) \in_M f^{-2}(y^{\bf 2})$
This model $N$ will be a model of {\em TSTU$_3$\/}$^{\infty}$: this
should be evident.

It should be evident from these considerations that all models of {\em
TSTU$_3$\/} satisfying the conditions on numbers of atoms (which are
describable in terms of sets of sentences satisfied in their theories)
also satisfy $Amb$ (noting that the scheme $\phi \leftrightarrow
\phi^+$ must be restricted to formulas not mentioning type 2).

\item[Definition:] Define {\em NF$_3$\/} as the theory whose axioms
are Strong Extensionality and those instances ``$\{x \mid \phi\}$
exists'' of Stratified Comprehension which can be stratified using a
stratification with range $\{0,1,2\}$ (note that the stratification
will send $x$ to 0 or 1, since it must assign 1 or 2 to $\{x \mid
\phi\}$).

\item[Corollary:]  {\em NF$_3$\/} is consistent.


\item[Proof:]  In the previous Theorem, fix the number of atoms at 0.

\item[Observation:] This is the first consistent fragment of New
Foundations which we have identified which has strong extensionality.
It is important to notice that, unlike {\em NF\/}, this is {\em not\/}
a weird theory involving considerations strange to ordinary
mathematics.  {\em Every\/} infinite model of {\em TST$_3$\/} has a
correlated model of {\em NF$_3$\/} which satisfies the same sentences
when types are dropped.  {\em NF$_3$\/}, though it may seem
unfamiliar, is ubiquitous and should be of considerable interest in
foundations of mathematics.

\end{description}

We go on to consider Ambiguity for {\em TSTU\/}$_n$ with $n>3$.

\begin{description}

\item[Theorem:] $TSTU_n^{\infty}$ is consistent iff $TSTU_n + Amb$ is
consistent.

\item[Proof:] Notice that our proof above depended on being able to
iterate the + operation as far as wanted; this is spoiled by the
presence of a top type.  We will fix this problem using a trick.

We can cleverly delete all reference to the bottom type of our
language.  We define $[\subseteq]^2$ as the collection of all sets
$\{x \mid x \subseteq A\}$ where $A$ is a fixed type 1 set (it is
important to recall that an urelement is not a subset of anything).
We define $1^{\bf 1}$ as usual as the set of all singletons.  We now observe
that $x^{\bf 0} \in y^{\bf 1}$ is equivalent to the assertion that
$\{x\} \subseteq y$, which is in turn equivalent to ``$\{x\}$ belongs
to every element of $[\subseteq]^{\bf 2}$ which contains $y$''.  We can now
replace all references to specific type 0 objects by references to
singletons and all quantifiers over type 0 with quantifiers over
$1^{\bf 1}$, redefining membership in type 0 objects appropriately.

This doesn't give us anything obvious for free, as we have our special
constants $1^1$ and $[\subseteq]^2$ to consider.  We further observe
that it is a theorem that $1^{\bf 1}$ is a subset of the domain of
$[\subseteq]^{\bf 2}$ and for every (type 2) subset $A$ of $1^{\bf 1}$
there is a unique type 1 object $a$ in the range of $[\subseteq]^{\bf
2}$ such that ``$\{x\} \subseteq a$'' (a fact expressible without
mentioning type 0) iff $\{x\} \in A$. 

Now $Amb$ tells us that there are objects $1^{\bf 0}$ and
$[\subseteq]^{\bf 1}$ with the type-shifted version of the same
property noted above for $1^1$ and $[\subseteq]^1$.  These can be
reinterpreted as the singleton set on a new type $-1$ and the
inclusion relation on type 0 objects construed as ``sets'' of type
$-1$ objects.  This means that $TSTU_n + Amb$ interprets $TSTU_{n+1}$
(we can reindex so that the new type $-1$ becomes type 0).  We can
further use ambiguity to ensure that as much as we wish to be true
about $1^{\bf 0}$ and $[\subseteq]^{\bf 1}$ is the type-shifted
analogue of what is true about $1^{\bf 1}$ and $[\subseteq]^{\bf 2}$
[we cannot show that there are specific relations which have exactly
the same properties, merely that there is a relation with any finite
selection of type shifted versions of the properties of $1^{\bf 1}$ and
$[\subseteq]^{\bf 2}$], and thus show by compactness that the extension of
$Amb$ can consistently hold as well.  So the consistency of $TSTU_n +
Amb$ for $n>3$ implies the consistency of $TSTU_{n+1} + Amb$, whence it
implies the consistency of $TSTU + Amb$, whence it implies the
consistency of New Foundations.

\item[Corollary:]  $TST_4+Amb$ is consistent iff $NF$ is consistent.

\item[Observation:] The profound difference between the case $n=3$ and
the case $n=4$ in the strongly extensional case is of interest here.

\item[Observation:] The proofs above will also work in some other type
theories.

\end{description}


Make the point that NF style considerations are natural and ubiquitous
in 3-typed mathematics.

This should include the proof of consistency of NFU using 3-type machinery
and the Pigeonhole Principle instead of Ramsey's theorem.

Mathematics in three types, functions without pairs.  FM methods in
the first section would avoid an inversion here.

\subsection{Predicativity; NFP; The Ramified Theory of Types Interpreted in NFP;  NFI}

\section{Finite Universes:  {\em NFU\/} + ``the universe is finite''.}

also NFU and nonstandard analysis?  

\section{New Foundations}

\subsection{History of {\em NF\/}; Errors of Quine}

Specker trees, all the bad stuff.  A section on FM methods in type
theory would help here as it would provide an occasion in the first
part to carefully discuss choice-free mathematics.  Orey's metamathematical results; of course they also work in {\em NFU\/}.

\section{Technical Methods for Consistency and Independence Proofs in {\em NF(U)\/}}

\subsection{Forcing in Type Theory and Set Theory}

Introduce the method of forcing in {\em NFU\/} at least and possibly
in type theory and ordinary set theory.  Prove the independence of the
continuum hypothesis.  Forcing in {\em NF\/}, of course.  But this may
continue a section on forcing in the type theory part.

\subsection{Frankel-Mostowski Permutation Methods}

Prove the independence of the Axiom of Choice from type theory
(certainly) and possibly from {\em NFU\/} and/or ordinary set theory.
The initial parts of this may occur in the type theory part.

\section{Cut Elimination in Type Theory and Set Theory}

Prove cut elimination in type theory and {\em SF\/}.  Maybe other
applications of Marcel's weak extensional collapse.

\section{Stratified Combinatory Logic and $\lambda$-Calculus}

\section{Rieger-Bernays Permutation Methods}

Explore the consistency and independence proofs obtainable, and the
set based notion of ``well-foundedness'' and related ideas.
Unstratified implementations of numerals.

\section{Limitations of Universal Constructions}

The existence of universal objects is not magic.  Cartesian closedness
failing for the category of all sets and functions is an advantage.

\chapter{Philosophy of Set Theory}

General considerations about the relative merits of the various
systems considered here and about the sufficiency of each as a
foundational system.  Comments on the general weirdness of {\em NF\/}
and the real nature of the {\em NF\/} consistency problem belong here.

\newpage

\chapter{Appendix:  Manual of Logical Style}

This is a handout I give students at various levels with logical rules in it in the same style as the text.

\section{Introduction}

This document is designed to assist students in planning proofs.  I
will try to make it as nontechnical as I can.

There are two roles that statements can have in a proof: a statement
can be a claim or goal, something that we are trying to prove; a
statement can be something that we have proved or which we have shown
to follow from current assumptions, that is, a statement which we can
{\em use\/} in the current argument.\footnote{which we called a {\em posit\/} in chapter 2:  I have not used this term in other contexts where I have distributed this style manual.}  It is very important not to
confuse statements in these two roles: this can lead to the fallacy of
assuming what you are trying to prove (which is well-known) or to the
converse problem, which I {\em have\/} encountered now and then, of
students trying to prove things that they already know or are entitled
to assume!

In the system of reasoning I present here, we classify statements by
their top-level logical operation: for each statement with a
particular top-level operation, there will be a rule or rules to
handle goals or claims of that form, and a rule or rules to handle
{\em using\/} statements of that form which we have proved or are
entitled to assume.

In what follows, I make a lot of use of statements like "you are entitled to assume $A$".  Notice that if you can flat-out {\em prove\/} $A$ you are entitled to assume $A$.  The reason I often talk about being entitled to assume $A$ rather than having proved $A$ is that one is often proving things using assumptions which are made for the sake of argument.

\section{Conjunction}

In this section we give rules for handling ``and''.  These are so simple that we barely notice that they exist!

\subsection{Proving a conjunction}

To prove a statement of the form $A \wedge B$, first prove $A$, then prove
$B$.

This strategy can actually be presented as a rule of inference:

$$\begin{array}{l} A \\ B \\ \hline A \wedge B \end{array}$$

If we have hypotheses $A$ and $B$, we can draw the conclusion $A \wedge B$:  so a strategy for proving $A \wedge B$ is to first prove $A$ then prove $B$.  This gives a proof in two parts, but notice that there are no assumptions being introduced in the two parts:  they are not separate cases.

If we give this rule a name at all, we call it ``conjunction".

\subsection{Using a conjunction}

If we are entitled to assume $A \wedge B$, we are further entitled to assume $A$ and $B$.  This can be summarized in two rules of inference:

$$\begin{array}{l} A \wedge B \\ \hline A \end{array}$$

$$\begin{array}{l} A \wedge B \\ \hline B \end{array}$$

This has the same flavor as the rule for proving a conjunction:  a conjunction just breaks apart into its component parts.

If we give this rule a name at all, we call it ``simplification".

\section{Implication}

In this section we give rules for implication.  There is a single basic rule for implication in each subsection, and then some derived rules which also involve negation, based on the equivalence of an implication with its contrapositive.  These are called derived rules because they can actually be justified in terms of the basic rules.  We like the derived rules, though, because they allow us to write proofs more compactly.

\subsection{Proving an implication}

\begin{description}

\item[The basic strategy for proving an implication:]  To prove $A \rightarrow B$, add $A$ to your list of assumptions and prove $B$; if you can do this, $A \rightarrow B$ follows without the additional assumption.

Stylistically, we indent the part of the proof consisting of statements depending on the additional assumption $A$:  once we are done proving $B$ under the assumption and thus proving $A \rightarrow B$ without the assumption, we discard the assumption and thus no longer regard the indented group of lines as proved.

This rule is called ``deduction".

\item[The indirect strategy for proving an implication:]  To prove $A \rightarrow B$, add $\neg B$ as a new assumption and prove $\neg A$:  if you can do this, $A \rightarrow B$ follows without the additional assumption.  Notice that this amounts to proving $\neg B \rightarrow \neg A$ using the basic strategy, which is why it works.

This rule is called ``proof by contrapositive" or ``indirect proof".

\end{description}

\subsection{Using an implication}

\begin{description}

\item[modus ponens:]  If you are entitled to assume $A$ and you are entitled to assume $A \rightarrow B$, then 
you are also entitled to assume $B$.  This can be written as a rule of inference:

$$\begin{array}{l} A \\A \rightarrow  B \\ \hline B \end{array}$$

\item[when you just have an implication:]  If you are entitled to assume $A \rightarrow B$, you may at any time adopt $A$ as a new goal, for the sake of proving $B$, and as soon as you have proved it, you also are entitled to assume $B$.  Notice that no assumptions are introduced by this strategy.  This proof strategy is just a restatement of the rule of {\em modus ponens\/} which can be used to suggest the way to proceed when we have an implication without its hypothesis.

\item[modus tollens:]  If you are entitled to assume $\neg B$ and you are entitled to assume $A \rightarrow B$, then 
you are also entitled to assume $\neg A$.  This can be written as a rule of inference:

$$\begin{array}{l} A \rightarrow  B \\ \neg B \\ \hline \neg A \end{array}$$

Notice that if we replace $A \rightarrow B$ with the equivalent contrapositive $\neg B \rightarrow \neg A$, then this becomes an example of modus ponens.  This is why it works.

\item[when you just have an implication:]  If you are entitled to assume $A \rightarrow B$, you may at any time adopt $\neg B$ as a new goal, for the sake of proving $\neg A$, and as soon as you have proved it, you also are entitled to assume $\neg A$.  Notice that no assumptions are introduced by this strategy.  This proof strategy is just a restatement of the rule of {\em modus tollens\/} which can be used to suggest the way to proceed when we have an implication without its hypothesis.

\end{description}

\section{Absurdity}

The symbol $\perp$ represents a convenient fixed false statement.   The point of having this symbol is that it makes the rules for negation much cleaner.

\subsection{Proving the absurd}

We certainly hope we never do this except under assumptions!  If we are entitled to assume $A$ and we are entitled to assume $\neg A$, then we are entitled to assume $\perp$.  Oops!  This rule is called {\em contradiction\/}.

$$\begin{array}{r} A \\ \neg A \\ \hline \perp \end{array}$$

\subsection{Using the absurd}

We hope we never really get to use it, but it is very useful.  If we are entitled to assume $\perp$, we are further entitled to assume $A$ (no matter what $A$ is).  From a false statement, anything follows.  We can see that this is valid by considering the truth table for implication.

This rule is called ``absurdity elimination".

\section{Negation}

The rules involving just negation are stated here.  We have already seen derived rules of implication using negation, and we will see derived rules of disjunction using negation below.

\subsection{Proving a negation}

\begin{description}

\item[direct proof of a negation (basic):]  To prove $\neg A$, add $A$ as an assumption and prove $\perp$.  If you complete this proof of $\perp$ with the additional assumption, you are entitled to conclude $\neg A$ without the additional assumption (which of course you now want to drop like a hot potato!).  This is the direct proof of a negative statement:  proof by contradiction, which we describe next, is subtly different.

Call this rule ``negation introduction".

\item[proof by contradiction (derived):]  To prove a statement $A$ of any logical form at all, assume $\neg A$ and prove $\perp$.
If you can prove this under the additional assumption, then you can conclude $A$ under no additional assumptions.  Notice that the proof by contradiction of $A$ is a direct proof of the statement $\neg\neg A$, which we know is logically equivalent to $A$; this is why this strategy works.

Call this rule ``reductio ad absurdum".

\end{description}

\subsection{Using a negation:}

\begin{description}

\item[double negation (basic):]  If you are entitled to assume $\neg\neg A$, you are entitled to assume $A$.  Call this rule ``double negation elimination".

\item[contradiction (basic):]  This is the same as the rule of contradiction stated above under proving the absurd:
if you are entitled to assume $A$ and you are entitled to assume $\neg A$, you are also entitled to assume $\perp$.  You also feel deeply queasy.

$$\begin{array}{r} A \\ \neg A \\ \hline \perp \end{array}$$

\item[if you have just a negation:] If you are entitled to assume $\neg A$, consider adopting $A$ as a new goal:  the point of this is that from $\neg A$ and $A$ you would then be able to deduce $\perp$ from which you could further deduce whatever goal $C$ you are currently working on.  This is especially appealing as soon as the current goal to be proved becomes $\perp$, as the rule of contradiction is the only way there is to prove $\perp$.

\end{description}

\section{Disjunction}

In this section, we give basic rules for disjunction which do not involve negation, and derived rules which do.  The derived rules can be said to be the default strategies for proving a disjunction, but they {\em can\/} be justified using the seemingly very weak basic rules (which are also very important rules, but often used in a ``forward" way as rules of inference).   The basic strategy for using an implication (proof by cases) is of course very often used and very important.  The derived rules in this section are justified by the logical equivalence of $P \vee Q$ with both $\neg P \rightarrow Q$ and $\neg Q \rightarrow P$:  if they look to you like rules of implication, that is because somewhere underneath they are.

\subsection{Proving a disjunction}

\begin{description}

\item[the basic rule for proving a disjunction (two forms):]  To prove $A \vee B$, prove $A$.   Alternatively, to prove $A \vee B$, prove $B$.
You do {\em not\/} need to prove both (you should not expect to be able to!)

This can also be presented as a rule of inference, called {\em addition\/}, which comes in two different versions.

$$\begin{array}{r} A \\ \hline A \vee B \end{array}$$

$$\begin{array}{r} B \\ \hline  A \vee B \\ \end{array}$$

\item[the default rule for proving a disjunction (derived, two forms):]   To prove $A \vee B$, assume $\neg B$ and attempt to prove $A$.  If $A$ follows with the additional assumption, $A \vee B$ follows without it.  

Alternatively (do not do both!):  To prove $A \vee B$, assume $\neg A$ and attempt to prove $B$.  If $B$ follows with the additional assumption, $A \vee B$ follows without it.

Notice that the proofs obtained by these two methods are proofs of $\neg B \rightarrow A$ and $\neg A \rightarrow B$ respectively, and both of these are logically equivalent to $A \vee B$.  This is why the rule works.  Showing that this rule can be derived from the basic rules for disjunction is moderately hard.

Call both of these rules ``disjunction introduction", or ``alternative elimination".


\end{description}


\subsection{Using a disjunction}

\begin{description}

\item[proof by cases (basic):]  If you are entitled to assume $A \vee B$ and you are trying to prove $C$, first assume $A$ and prove $C$ (case 1);
then assume $B$ and attempt to prove $C$ (case 2).  

Notice that the two parts are proofs of $A \rightarrow C$ and $B \rightarrow C$,
and notice that $(A \rightarrow C) \wedge (B \rightarrow C)$ is logically equivalent to $(A \vee B) \rightarrow C$ (this can be verified using a truth table).

This strategy is very important in practice.


\item[disjunctive syllogism (derived, various forms):]  If you are entitled to assume $A \vee B$ and you are also entitled to assume $\neg B$, you are further entitled to assume $A$.  Notice that replacing $A \vee B$ with the equivalent $\neg B \rightarrow A$ turns this into an example of modus ponens.


If you are entitled to assume $A \vee B$ and you are also entitled to assume $\neg A$, you are further entitled to assume $B$.  Notice that replacing $A \vee B$ with the equivalent $\neg A \rightarrow B$ turns this into an example of modus ponens.

Combining this with double negation gives further forms:  from $B$ and $A \vee \neg B$ deduce $A$, for example.

Disjunctive syllogism in rule format:

$$\begin{array}{r}  A \vee B \\ \neg B \\ \hline A \end{array}$$

$$\begin{array}{r}  A \vee B \\ \neg A \\ \hline B \end{array}$$

Some other closely related forms which we also call ``disjunctive syllogism":

$$\begin{array}{r}  A \vee \neg B \\ B \\ \hline A \end{array}$$

$$\begin{array}{r}  \neg A \vee B \\ A \\ \hline B \end{array}$$

\end{description}

\section{Biconditional}

Some of the rules for the biconditional are derived from the definition of $A \leftrightarrow B$ as $(A \rightarrow B) \wedge (B \rightarrow A)$.  There is a further very powerful rule allowing us to use biconditionals to justify replacements of one expression by another.

\subsection{Proving biconditionals}

\begin{description}

\item[the basic strategy for proving a biconditional:]  To prove $A \leftrightarrow B$, first assume $A$ and prove $B$; then (finished with the first assumption) assume $B$ and prove $A$.  Notice that the first part is a proof of $A \rightarrow B$ and the second part is a proof of $B \rightarrow A$.

Call this rule ``biconditional deduction".

\item[derived forms:]  Replace one or both of the component proofs of implications with the contrapositive forms.  For example one could first
assume $A$ and prove $B$, then assume $\neg A$ and prove $\neg B$ (changing part 2 to the contrapositive form).

\end{description}

\subsection{Using biconditionals}  The rules are all variations of modus ponens and modus tollens.   Call them biconditional modus ponens (bimp)
or biconditional modus tollens (bimt) as appropriate.

If you are entitled to assume $A$ and $A \leftrightarrow B$, you are entitled to assume $B$.

If you are entitled to assume $B$ and $A \leftrightarrow B$, you are entitled to assume $A$.

If you are entitled to assume $\neg A$ and $A \leftrightarrow B$, you are entitled to assume $\neg B$.

If you are entitled to assume $\neg B$ and $A \leftrightarrow B$, you are entitled to assume $\neg A$.

These all follow quite directly using modus ponens and modus tollens and one of these rules:

If you are entitled to assume $A \leftrightarrow B$, you are entitled to assume $A \rightarrow B$.

If you are entitled to assume $A \leftrightarrow B$, you are entitled to assume $B \rightarrow A$.

The validity of these rules is evident from the definition of a biconditional as a conjunction.

\section{Calculating with biconditionals}  Let $F$ be a complex expression including a propositional letter $P$.  For any complex expression $C$
let $F[C/P]$ denote the result of replacing all occurrences of $P$ by $C$.

The replacement rule for biconditionals says that if you are entitled to assume $A \leftrightarrow B$ and also entitled to assume $F[A/P]$,
then you are entitled to assume $F[B/P]$.    Also,  if you are entitled to assume $A \leftrightarrow B$ and also entitled to assume $F[B/P]$,
then you are entitled to assume $F[A/P]$.  

The underlying idea which we here state very carefully is that $A \leftrightarrow B$ justifies substitutions of $A$ for $B$ and of $B$ for $A$ in complex expressions.  This is justified by the fact that all our operations on statements depend only on their truth value, and $A \leftrightarrow B$ is equivalent to the assertion that $A$ and $B$ have the same truth value.

This rule and a list of biconditionals which are tautologies motivates the ``boolean algebra" approach to logic.

\section{Universal Quantifier}

This section presents rules for $(\forall x.P(x))$ (``for all $x$, $P(x)$") and for the restricted form $(\forall x \in A.P(x))$ (``for all $x$ in the set $A$, $P(x)$").  Notice that $(\forall x \in A.P(x))$ has just the rules one would expect from its logical equivalence to $(\forall x.x \in A \rightarrow P(x))$.

\subsection{Proving Universally Quantified Statements}

To prove $(\forall x.P(x))$,  first introduce a name $a$ for a completely arbitrary object.  This is signalled by a line ``Let $a$ be chosen arbitrarily".  This name should not appear
in any earlier lines of the proof that one is allowed to use.  The goal is then to prove $P(a)$. Once the proof of $P(a)$ is complete, one has proved $(\forall x.P(x))$ and should regard the block beginning with the introduction
of the arbitrary name $a$ as closed off (as if ``Let $a$ be arbitrary" were an assumption).  The reason for this is stylistic:  one should free up the use of the name $a$ for other similar purposes later in the proof.

To prove $(\forall x \in A.P(x)$, assume $a \in A$ (where $a$ is a name which does not appear earlier in the proof in any line one is allowed to use):  in the context of this kind of proof it is appropriate to say ``Let $a \in A$ be chosen arbitrarily" (and supply a line number so the assumption $a \in A$ can be used).  One's goal is then to prove $P(a)$.  Once the goal is achieved, one is entitled to assume
$(\forall x \in A.P(x))$ and should not make further use of the lines that depend on the assumption $a \in A$.  It is much more obvious in the restricted case that one gets a block of the proof that one should close off (because the block uses a special assumption $a \in A$), and the restricted case is much more common in actual proofs.

These rules are called ``universal generalization".  The line reference would be to the block of statements
from ``Let $a[\in A]$ be chosen arbitrarily" to $P(a)$.

\subsection{Using Universally Quantified Statements}

If one is entitled to assume $(\forall x.P(x))$ and $c$ is any name for an object, one is entitled to assume $P(c)$.

If one is entitled to assume $(\forall x \in A.P(x))$ and $c \in A$, one is entitled to assume $P(c)$.

These rules are called ``universal instantiation".  The reference is to the one or two previous lines used.

\newpage

As rules of inference:

$$\begin{array}{r} (\forall x.P(x)) \\ \hline P(c) \end{array}$$

$$\begin{array}{r} (\forall x\in A.P(x)) \\c \in A \\ \hline P(c) \end{array}$$

\section{Existential Quantifier}
This section presents rules for $(\exists x.P(x))$ (``for some $x$, $P(x)$", or equivalently ``there exists an $x$ such that $P(x)$") and for the restricted form $(\exists x \in A.P(x))$ (``for some $x$ in the set $A$, $P(x)$" or ``there exists $x$ in $A$ such that $P(x)$").  Notice that $(\exists x \in A.P(x))$ has just the rules one would expect from its logical equivalence to $(\exists x.x \in A \wedge P(x))$.

\subsection{Proving Existentially Quantified Statements}

To prove $(\exists x.P(x))$, find a name $c$ such that $P(c)$ can be proved.  It is your responsibility to figure out which $c$ will work.

To prove $(\exists x\in A.P(x))$ find a name $c$ such that $c \in A$ and $P(c)$ can be proved.  It is your responsibility to figure out what $c$ will work.

A way of phrasing either kind of proof is to express the goal as ``Find $c$ such that [$c \in A$ and] $P(c)$", where $c$ is a new name which does not appear in the context:  once a specific term $t$ is identified as the correct value of $c$, one can then say ``let $c = t$" to signal that one has found the right object.  Of course this usage only makes sense if
$c$ has no prior meaning.

This rule is called ``existential introduction".  The reference is to the one or two lines used.

As rules of inference:

$$\begin{array}{r} P(c) \\ \hline (\exists x.P(x))\end{array}$$

$$\begin{array}{r} c \in A \\ P(c) \\ \hline (\exists x \in A.P(x))\end{array}$$

\subsection{Using Existentially Quantified Statements}

Suppose that one is entitled to assume $(\exists x.P(x))$ and one is trying to prove a goal $C$.  One is allowed
to further assume $P(w)$ where $w$ is a name which does not appear in any earlier line of the proof that one is allowed to use, and prove the goal $C$.  Once the goal $C$ is proved, one should no longer allow use of the block of variables
in which the name $w$ is declared (the reason for this is stylistic:  one should be free to use the same variable $w$ as a ``witness" in a later part of the proof; this makes it safe to do so).   If the statement one starts with is $(\exists x \in A.P(x))$ one may follow $P(w)$ with the additional assumption $w \in A$.

This rule is called ``witness introduction" or ``existential generalization".  The reference is to the line $(\exists x[\in A].P(x))$ and the block of statements from $P(w)$ to $C$.


\section{Proof Format}

Given all these rules, what is a proof?

A proof is an argument which {\em can be\/} presented as a sequence of numbered statements.  Each numbered statement is either justified
by a list of earlier numbered statements and a rule of inference [for example, an appearance of $B$ as line 17 might be justified by an appearance of $A$ as line 3 and an appearance of $A \rightarrow B$ as line 12, using the rule of modus ponens] or is an assumption with an associated goal (the goal is not a numbered statement but a comment).  Each assumption is followed in the sequence by an appearance of the associated goal as a numbered statement, which we will call the resolution of the assumption.  The section of the proof consisting of an assumption, its resolution, and all the lines between them  is closed off in the sense that no individual line in that section can be used to justify anything appearing in the proof after the resolution, nor can any assumption in that section be resolved by a line appearing in the proof after the resolution.  In my preferred style of presenting these proofs, I will indent the section between an assumption and its resolution (and further indent smaller subsections within that section with their own assumptions and resolutions).  The whole sequence of lines from the assumption to its resolution can be used to justify a later line (along with an appropriate rule of course):  for example, the section of a proof between line 34:  assume $A$:  goal $B$ and line 71:  $B$ could be used to justify line 113 $A \rightarrow B$ (lines 34-71, deduction
); I do not usually do this (I usually write the statement to be proved by a subsection as a goal at the head of that section, and I do not usually use statements proved in such subsections later in the proof), but it is permitted.

I used to be in the habit of omitting the resolution of a goal if it was immediately preceded by an assumption-resolution section (or sections in the case of a biconditional) which could be used as its line justification:  this seemed like a pointless repetition of the goal, which would already appear just above such a section.   I would state the resolution line if it was going to be referred to in a later line justification.  The idea was that the statement of a goal followed by a block of text that proves it is accepted as a proof of that statement; the only reason to repeat the statement with a line number is if it is going to be referenced using that line number.  However, I have learned that students prefer closing lines.

Note the important italicized phrase ``can be".  A proof is generally presented in a mathematics book as a section of English text including math notation where needed.  Some assumptions may be assumed to be understood by the reader.  Some steps in reasoning may be omitted as ``obvious".   The logical structure will not be indicated explicitly by devices like line numbering and indentation; the author will rely more on the reader understanding what he or she is writing.  This means that it is actually quite hard to specify exactly what will be accepted as a proof; the best teacher here is experience.  A fully formalized proof can be specified (even to the level where a computer can recognize one and sometimes generate one on its own), but such proofs are generally rather long-winded.

\section{Examples}

These examples may include some general comments on how to write these proofs which you would not include if you were writing this proof yourself.  I also included resolution lines (restatements of goals after they are proved) which I do not usually include.

\newpage

\begin{description}

\item[Theorem:]  $((P \wedge Q) \rightarrow R) \leftrightarrow (P \rightarrow (Q \rightarrow R))$

\item[Proof:]

The statement is a biconditional.  The proof is in two parts.

\begin{description}

\item[Part 1:]

\begin{description}

\item[Assume (1)]  $(P \wedge Q) \rightarrow R)$

\item[Goal:]  $(P \rightarrow (Q \rightarrow R)$

Now we use the strategy for proving an implication.

\begin{description}

\item[Assume (2)]  $P$

\item[Goal:]  $Q \rightarrow R$

\begin{description}

\item[Assume (3)]  $Q$

\item[Goal:]  $R$

\item[Goal:]  $P \wedge Q$ (so that we can apply m.p. with line 1)

\item[4] $P \wedge Q$ (from lines 2 and 3)

\item[5] $R$ rule of modus ponens with lines 1 and 4.  This is the resolution of the goal at line 3.

\end{description}

\item[6]  $Q \rightarrow R$ lines 3-5.  This is the resolution of the goal at line 2, which I used to omit.

\end{description}

\item[7]  $P \rightarrow (Q \rightarrow R)$  lines 2-6 This is the resolution of the goal at line 1, which I used to be in the habit of omitting.

\end{description}

\newpage
\item[Part 2:]

\begin {description}

\item[Assume (8):] $P \rightarrow (Q \rightarrow R)$

\item[Goal:]  $(P \wedge Q) \rightarrow R$

\begin{description}

\item[Assume (9):]  $P \wedge Q$

\item[Goal:] $R$

\item[Goal:]  $P$ (looking at line 1 and thinking of modus ponens)

\item[10]  $P$ from line 9

\item[11]  $Q \rightarrow R$  mp lines 10 and 8.

\item[Goal:]  $Q$  (looking at line 4 and thinking of modus ponens)

\item[12] $Q$  from line 9

\item [13] $R$ lines 11 and 12, rule of modus ponens.  This is the resolution of the goal at line 9.

\end{description}

\item[14]  $(P \wedge Q) \rightarrow R$  This is the resolution of the goal at line 8, which I used to omit.

\end{description}

\end{description}

\item[15]  $((P \wedge Q) \rightarrow R) \leftrightarrow (P \rightarrow (Q \rightarrow R))$  lines 1-14.  I used to omit this as it just recapitulates the statement of the theorem already given.  If I did omit it, I would also restart the numbering at 1 at the beginning of Part 2.

\newpage

\item[Theorem:]  $\neg(P \wedge Q) \leftrightarrow (\neg P \vee \neg Q)$

\item[Proof:]

\begin{description}

\item[Part 1:]

\begin{description}

\item[Assume (1):]  $\neg(P \wedge Q)$

\item[Goal:]  $\neg P \vee \neg Q$

We use the disjunction introduction strategy:  assume the negation of one alternative and show that the other alternative follows.

\begin{description}

\item[Assume (2):]  $\neg\neg P$

\item[Goal:]  $\neg Q$

\begin{description}

\item[Assume (3):]  $Q$

\item[Goal:]  $\perp$ (a contradiction)

\item[Goal:]  $P \wedge Q$ (in order to get a contradiction with line 1)

\item[4]  $P$  double negation, line 3

\item[5]  $P \wedge Q$ (lines 3 and 4)

\item[6]  $\perp$  1,5 contradiction .  This resolves the goal at line 3.

\end{description}

\item[7] $\neg Q$  lines 3-6 negation introduction.  This resolves the goal at line 2.

\end{description}

\item[8]  $\neg P \vee \neg Q$  2-7 disjunction introduction.  This resolves the goal at line 1.


\end{description}

\newpage


\item[Part 2:]

\begin{description}

\item[Assume (9):]  $\neg P \vee \neg Q$

\item[Goal:]  $\neg(P \wedge Q)$

\begin{description}

\item[Assume (10):]  $P \wedge Q$

\item[Goal:]  $\perp$ (a contradiction)

We use the strategy of proof by cases on line 9.
\begin{description}
\item[Case 1 (9a):]  $\neg P$

\item[Goal:]  $\perp$

\item[11]:  $P$ from line 10

\item[12]: $\perp$ 9a, 11 contradiction (this resolves the goal after 9a)

\item[Case 2 (9b):]  $\neg Q$

\item[Goal:] $\perp$

\item[13]  $Q$ from line 10

\item[14]  $\perp$  9b, 13 contradiction (this resolves the goal after 9b)

\end{description}

\item[15]  $\perp$ 9, 9a-14 proof by cases.

\end{description}

\item[16]  $\neg (P \wedge Q)$  9-15 negation introduction.  This resolves the goal at line 9.




\end{description}



\end{description}

\item[17]  $\neg(P \wedge Q) \leftrightarrow (\neg P \vee \neg Q)$ 1-16, biconditional introduction.

\newpage


\item[Rule of Inference (Constructive Dilemma):]

We verify that

$$\begin{array}{r} P \vee Q \\ P \rightarrow R \\ Q \rightarrow S \\ \hline R \vee S \end{array}$$

is a valid rule of inference.

If we are verifying a rule of inference we assume the hypotheses to be true then adopt the  conclusion as our goal.

\begin{description}

\item[1]  $P \vee Q$  premise

\item[2] $P \rightarrow R$  premise

\item[3] $Q \rightarrow S$  premise

\item[Goal:]  $R \vee S$

We use proof by cases on line 1.

\begin{description}

\item[Case 1 (1a):]  $P$

\item[Goal:]  $R \vee S$

\item[4] $R$  1a, 2, modus ponens

\item[5]  $R \vee S$  addition, line 4.  This resolves the goal at line 1a.



\item[Case 2 (1b):]  $Q$

\item[Goal:]  $R \vee S$

\item[6]  $S$  3,1b, modus ponens

\item[7]  $R \vee S$  addition, line 6.  This resolves the goal at line 1b.

\end{description}

\item[8]  $R \vee S$  proof by cases, 1, 1a-7.  And this is what we set out to prove.

\end{description}


\newpage


Here is a quantifier example.

\item[Theorem:]  $$(\forall x:P[x]) \wedge (\forall y:P[y] \rightarrow Q[y]) \rightarrow (\forall z:Q[z])$$

\begin{description}

\item[Assume (1):]  $(\forall x:P[x]) \wedge (\forall y:P[y] \rightarrow Q[y])$

\item[Goal:]  $(\forall z:Q[z])$

\begin{description}

\item Let $a$ be arbitrary.

\item[(2):] $a=a$ (optional)

\item[Goal:]  $Q[a]$

\item[(3):]  $(\forall x:P[x])$  simp 1

\item[(4):]  $(\forall y:P[y] \rightarrow Q[y])$ simp 1

\item[(5):]  $P[a]$  UI 3 $x:=a$

\item[(6):]  $P(a) \rightarrow Q(a)$  UI 4 $y:=a$

\item[(7):] $Q[a]$ mp 5,6

\end{description}

\item[(8):]  $(\forall z:Q[z])$  UG 2-7  [you may share my temptation to put a line number on ``Let $a$ be arbitrary" and start the UG block there; or, as shown here, use an optional line $a=a$ for this purpose]

\end{description}

\item[(9):]  The theorem:  deduction 1-8.

\end{description}


\newpage

\chapter{Appendix:  Unsorted Preamble}

A brief introduction to this approach now appears at the beginning of chapter 2.  The notation used there may need some reconciliation with notation used here, and the axiomatics are slightly different, but the discussion here is sufficient to establish that the theories presented are the same.

\section{An unsorted theory of sets motivated by types}

We begin with an unsorted theory (in which quantifiers range over all objects), which is a theory of sets, though it is not the usual untyped theory of sets.

We introduce the membership relation $\in$ as a primitive.

In the context of our unsorted set theory, we can represent the notion of being the same type.  We certainly expect that all elements of a set will be of the same kind (type).
We further expect that for any kind of object, there will be a set of all objects of that kind.  So being of the same type will coincide precisely with belonging to some common set.

\begin{description}

\item[Definition:]  We say that $x$ and $y$ are of the same type, written $x \sim_\tau y$, iff $(\exists z:x \in z \wedge y \in z)$.

\end{description}

The terminology suggests that $\sim_\tau$ is an equivalence relation.  That this relation is symmetric is a truth of logic.  We provide an axiom which makes it easy to demonstrate
that it is reflexive and transitive.

\begin{description}

\item[Axiom of types:]  $(\forall x:(\exists y:x\in y \wedge (\forall z:z \in y \leftrightarrow z \sim_\tau x))).$  For each $x$, we define $\tau(x)$ as the set provided by this
axiom which contains $x$ and contains all objects of the same type as $x$.  We call this set the type of $x$.

\item[$\sim_\tau$ is an equivalence relation:]  That $x \sim_\tau y \leftrightarrow y \sim_\tau x$ is a tautology of first order logic.  From $x \in \tau(x)$ it follows
immediately that $x \sim_\tau x$.  Suppose $x \sim_\tau y$ and $y \sim_\tau z$.  Then $z \sim_\tau y$ (symmetry) so $x \in \tau(y) \wedge z \in \tau(y)$, so
$(\exists u:x \in u \wedge y \in u)$, so $x \sim_\tau z$.

\end{description}

We provide that objects with elements are sets, and that sets with the same elements and of the same type are equal.

\begin{description}

\item[Axiom of the empty set:]  We introduce a primitive construction of an object $\emptyset_x$ for each object $x$, and the axiom $(\forall xy:\emptyset_x \sim_\tau \tau(x) \wedge y \not\in x).$   We refer to $\emptyset_x$ as the empty set over $x$.

\item[Definition of sethood:]  We define ${\tt set}(x)$ ($x$ is a set) as $$(\exists y:y \in x \vee x=\emptyset_y).$$

\item[Axiom of (weak) extensionality:]  $$(\forall xy: {\tt set}(x) \wedge {\tt set}(y) \wedge x \sim_\tau y \wedge (\forall z:z\in x \leftrightarrow z \in y) \rightarrow x=y)$$

\end{description}

It has become stylish in foundations of mathematics to assume that all objects are sets (so there would be no more than one object with no elements).  We leave open the possibility that there are many objects with no elements, for more than one reason.



We postulate that every property of objects of a particular kind determines a set.

\begin{description}

\item[Axiom scheme of comprehension:]  For each formula $\phi$ in which $A$ is not free, we have an axiom $$(\forall x:(\exists A: {\tt set}(A) \wedge A \sim_\tau \tau(x) \wedge (\forall y:y \in A \leftrightarrow x \sim_\tau y \wedge \phi))),$$  For each $x$, the witness $A$ (unique by extensionality) is denoted by $\{y \sim_\tau x:\phi\}$.  

\item[Notational observations:]  We note that $\emptyset_x$ (the empty set over the type of $x$) can be expressed as $\{y \sim_\tau x:y \neq y\}$.  We note that $\tau(x)$ (the type of $x$) can be written as $$\{y \sim_\tau x:y = y\},$$ but the axiom of types does have additional content, since it ensures that $x$ is in this set and so that it is nonempty.  We may also write this $V_x$ (the universe containing $x$).  We define $\{y \in \tau(x):\phi\}$ as $\{y \sim_\tau x:\phi\}$.  



\end{description}

We provide a further axiom regulating types.  If two nonempty sets are of the same type, their respective elements are of the same type as well.

\begin{description}

\item[Axiom of levels:]  For any $x,y,z,w$, if $x \sim_\tau y$ and $z \in x$ and $w \in y$, then $z \sim_\tau w$.

\item[Exercise:]  Show that in the presence of the other axioms, the axiom of levels is equivalent to each of the following assertions:

\begin{description}

\item  [the axiom of union:]  $$(\forall A:\exists U:{\tt set}(U) \wedge (\forall x:x \in U \leftrightarrow (\exists a:a \in A \wedge x \in a))).$$  If $A \in \tau^3(x)$ and ${\tt set}(A)$,
there is a unique such $U$ in $\tau^2(x)$, which we denote by $\bigcup A$.

\item[the axiom of binary union:]  $$(\forall AB: {\tt set}(A) \wedge {\tt set}(B) \wedge A \sim_\tau B \rightarrow (\exists U:{\tt set}(U) \wedge (\forall x:x \in U \leftrightarrow x \in A \vee x \in B))).$$  For any particular sets $A,B$ of the same type, we introduce the notation $A \cup B$ for the witness $U$ to this assertion.

\end{description}

\item[Exercise:]  Verify that the axioms of types and levels both follow if one postulates the existence of $\{x\} = \{y:y=x\}$ for each $x$ (axiom of singletons), the existence of
${\cal B}(x) = \{y:x\in y\}$ for each $x$ (the notation commemorates Maurice Boffa's interest in this construction) and the axiom of union.

\end{description}

We consider notation for hierarchies of types.

\begin{description}

\item[Definition:]  For any function symbol $F$, we define $F^0(x) = x$ and $F^{n+1}(x) = F(F^n(x))$, for each numeral $n$.  Notice that these superscripts
are not variables in our language and will not be quantified over.  In particular, we now have definitions for $\tau^n(x)$ for any $x$.

\item[Observations and Definition:]  Notice that $x \in \tau(x)$ for any $x$, so $\tau^n(x) \in \tau^{n+1}(x)$.  Note further that if $\tau^2(x) = \tau^2(y)$, we have
$\tau(x) \sim_\tau \tau(y)$ and with $x \in \tau(x),y \in \tau(y)$ the axiom of levels gives us $x \sim_\tau y$, so $\tau(x)=\tau(y)$.  Thus we can define $\tau^{-1}$ by
$\tau^{-1}(\tau^2(x)) = \tau(x)$, with $\tau^{-1}$ defined only at types of types.  We can define $\tau^{-n}(x)$ as $(\tau^{-1})^n(x)$, when this is defined.

Note that for any $A$, $\tau^{-1}(\tau(A))$ is defined iff $\tau(A) = \tau^2(x)$ for some $x$ and in this case is $\tau(x)$, the type of elements of $A$ if $A$ is nonempty,
and the type of the same type as $A$ if $A$ is empty.  If $A$ is not of a type $\tau^2(x)$ this is undefined.

\item[Definition (common set builder notations):]  We define $\{x \in A:\phi\}$ for a set $A$ as $\{x \in \tau^{-1}(\tau(A)):x \in A \wedge \phi\}$.  We define $$\{F(x_1,\ldots,x_n) \in A:\phi\}$$ as $$\{y \in A:(\exists x_1,\ldots,x_n:y = F(x_1,\ldots,x_n) \wedge \phi)\},$$ where $F$ is any $n$-ary  function symbol (it may be a complex term construction).

The axiom scheme of separation of Zermelo, adorned with technicalities about type, asserting for each formula $\phi$ $$(\forall A:{\tt set}(A)\rightarrow (\exists B:{\tt set}(B) \wedge B \sim_\tau A \wedge (\forall x:x \in B \leftrightarrow x \in A \wedge \phi))),$$ is equivalent to the axiom of comprehension we have presented (in the presence of the other axioms).

\item[Definition (singletons):]  We define $\iota(x)$ or $\{x\}$ as $\{y \sim_\tau x:y=x\}$.

\item[Definition (unordered pair):]  If $x \sim_\tau y$, we define $\{x,y\}$ as $$\{z \sim_\tau x:z=x \vee z=y\}.$$  Note that the existence of the unordered pair of two objects is equivalent to the two objects being of the same type: $$x \sim_\tau y \leftrightarrow(\exists p:(\forall z:z \in p \leftrightarrow z=x \vee z=y)) $$  is a theorem of our system.

\end{description}

We know, because we have shown that $\sim_\tau$ is an equivalence relation, that if two types meet, they are the same.  We argue that the types in the hierarchies we have defined are distinct.  The argument is related to the paradox of Russell.

\begin{description}

\item[Theorem:]  $\tau(x) \neq \tau^2(x)$.

\item[Proof of Theorem:]  Suppose otherwise.  Then we can define $$R = \{y \in \tau(x):y \not\in y\},$$ and we will have $R \sim_\tau \tau(x)$ so $R \in \tau^2(x) = \tau(x)$.
Then $$R \in R \leftrightarrow R \in \tau(x) \wedge R \not\in R,$$ a contradiction.

\end{description}

This can be generalized.

\begin{description}

\item[Theorem:]  $\tau(x) \neq \tau^{n+2}(x)$.

\item[Proof of Theorem:]  Suppose otherwise.  Note that $\tau(\iota^n(x)) = \tau^{n+1}(x)$.  Then we can define $$R_n = \{\iota^n(y) \in \tau(x):\iota^n(y) \not\in y\},$$ and we will have $R_n \sim_\tau \tau^{n+1}(x)$ so $R_n \in \tau^{n+2}(x) = \tau(x)$ and also $\iota^n(R_n) \in \tau^{n+1}(\tau(x)) = \tau^{n+2}(x) = \tau(x)$.
Then $$\iota^n(R_n) \in R_n \leftrightarrow \iota^n(R_n) \in \tau(x) \wedge \iota^n(R_n) \not\in R_n,$$ a contradiction (since $\iota^n(R_n) \in \tau(x)$ is supposed true). 

\end{description}

This establishes that for any particular $x$, the sequence of types $\tau(x), \tau^2(x),\tau^3(x),\ldots$ are distinct and therefore disjoint.  This does have one corollary which may feel unusual.  The empty set $\emptyset_x \in \tau^2(x)$, and generally $\emptyset_{\tau^n(x)} \in \tau^{n+2}(x)$, so for distinct values of $n$ these empty sets are distinct.

\section{Typed language introduced}

We explain the type theory we use below using the unsorted language of the previous subsection.

We fix a type $\tau({\bf x})$ ({\bf x} being a constant object about which we have very little to say) and define ``type $n$" as $\tau^{n+1}({\bf x})$.  We adopt the convention that each variable $x$ in our language has a natural number type ${\tt type}(x)$,
and stands for an object in $\tau^{{\tt type}(x)+1}({\bf x})$.
We read any quantifier $(\forall x:\phi)$ or $(\exists x:\phi)$ as $(\forall x\in \tau^{{\tt type}(x)+1}({\bf x}):\phi)$ or $(\exists x\in \tau^{{\tt type}(x)+1}({\bf x}):\phi)$, respectively.
We further adopt the convention that we will only write $x=y$ when ${\tt type}(x)={\tt type}(y)$, and we will only write $x \in y$ when ${\tt type}+1 = {\tt type}(y)$
(notice that if we wrote an atomic sentence not satisfying the appropriate one of these conditions, it would be false).  Notice that in this stereotyped language we cannot even write down the definitions of the sets $R$ and $R_n$ appearing in the proofs that the types are disjoint.  We can write the definition of $\sim_\tau$, but as interpreted, any instance
of $x \sim_\tau y$ that we can write down consistent with our convention will be true, so this relation does not need to be mentioned.

We can then formulate typed versions of our axioms (which are presented below).  It might seem that we are restricting our means of expression by regimenting our
language in this way, but this is provably not the case.  Any instance of any of the axioms other than comprehension actually satisfies our typed variable conventions.
We argue that every instance of comprehension ``$\{x \in \tau^n({\bf x}):\phi\}$ exists" is a consequence of an instance of comprehension which can be expressed in our typed
language (so the full unsorted version of the comprehension axiom does not entail the existence of any sets, at least in types $\tau^n({\bf x})$, whose existence is not entailed by the typed axiom scheme).

Any sentence in which no variable of a type $\tau^n({\bf x})$ appears can be replaced with a truth value (easily represented as $u=u$ or its negation):  its truth value won't
depend on the value of the binding variable $x$.  Each quantified subformula $(\forall y:\psi)$ should be replaced with a conjuncton of versions in which $y$ is assigned a type:  all types should be used which are obtained from the type of a parameter or $\tau^n({\bf x})$ by applying $\tau^{\pm(i+1)}$ where $i$ is less than (say) the number of variables in the formula plus one.  Similarly, existentially quantified formulas should be replaced with disjunctions of  existential formulas restricted to types.  When this is done, every variable in the expanded formula will have a type:  it is possible that some negative indexed types will be conjured into being.  All atomic formulas which do not satisfy the type conventions can be replaced with $\neg u=u$.
Logical identities can be used to ensure that no variable not of a type $\tau^{\pm n}({\bf x})$ appears in a quantified formula restricted to a type  type $\tau^{\pm n}({\bf x})$,
and no variable of this form appears in a quantified formula restricted to a type not of this form:  this is done by using logical identities which pull formulas in which the bound variable does not occur out of a quantified formula.  Then every formula in which  no variable of a type $\tau^{\pm n}({\bf x})$ appears can be replaced with a truth value (either one, we do not care;
either truth value is supported by an instance of typed comprehension) and every formula in which variables of  type $\tau^{\pm n}({\bf x})$ appear will actually be well-typed.  This transforms our instance of general untyped comprehension into an instance of typed comprehension (or several instances, one of which gives the desired set).

The point of this is the only untyped comprehension axioms that do any work here are the definitions of $R$ and $R_n$, which manifest themselves only in our assurance that
badly typed atomic sentences are false.

It is worth noticing that the unsorted theory may have lots of types other than the types $\tau^{n+1}({\bf x})$ which we use in our typing scheme.  The normal expectation that we have types indexed by the natural numbers cannot be conveniently expressed in the unsorted language used here.

\section{Simple ideas of set theory in the language of the unsorted preamble}

In this section we develop some familiar ideas of set theory, in the unsorted language of section 9.1.  This section is here just to give a flavor of what a development without typed language of the same theory might look like.

We first develop the familiar list notation for finite sets.  Here are
the standard notations for one and two element sets.

\begin{description}

\item[List notation for sets:]  $\{x\}$ is defined as $\{y \sim_\tau x:  y=x\}$.
$\{x,y\}$ is defined, when $x \sim_\tau y$, as $\{z\sim_\tau x:  z=x\vee z=y\}$.

\end{description}

It is convenient to define Boolean union and intersection of sets
before giving the general definition of list notation.

\begin{description}

\item[Boolean union and intersection:] If $x$ and $y$ are sets of the same type, define
$x \cup y$ as $$\{z \in \tau^{-1}(\tau(x)): z \in x \vee z \in y\}$$ and $x \cap y$ as $$\{z \in \tau^{-1}(\tau(x)): z \in x \wedge z \in y\}.$$  Notice that though we may informally
think of $x \cup y$ as ``$x$ and $y$'', it is actually the case that
$x \cup y$ is associated with the logical connective $\vee$ and it is
$x \cap y$ that is associated with $\wedge$ in a logical sense.

The axiom of levels plays an important role here, ensuring that the fact that $x \sim_\tau y$ implies that elements of $x$ are of the same type as elements of $y$.

We also define $a^c$ (the complement of $a$) as $\{x \in \tau^{-1}(\tau(a)): x \not\in
a\}$ and $a-b$ (the set difference of $a$ and $b$) as $a \cap b^c$ (of course under the assumption $a \sim_\tau b$).

\item[recursive definition of list notation:] $\{x_1,x_2,\ldots,x_n\}$
is defined as $$\{x_1\} \cup \{x_2,\ldots,x_n\}.$$  Notice that the
definition of list notation for $n$ items presupposes the definition
of list notation for $n-1$ items: since we have a definition of list
notation for 1 and 2 items we have a basis for this recursion.

Note that all elements of a set defined by listing must be of the same
type, just as with any set.

\end{description}

There is one more very special case of finite sets which needs special
attention.

\begin{description}

\item[null set:]  We define $\emptyset_x$ as $\{y \in \tau(x):y \neq y\}$.  Notice that $\emptyset_x \in \tau^2(x)$.  It would
be natural to extend our typed language convention to allow $\emptyset$ to be used as a constant without the subscript in situations where its
type can be deduced from that of neighboring variables.  A statement such as $\emptyset \in \emptyset$ could be read as
$\emptyset_x \in \emptyset_{\tau(x)}$ (though strictly speaking probably one of the subscripts should be supplied).  In any event, all such statements are false.

\item[universe:] We define $V_x$ as $\{y \in \tau(x):y = y\} (= \tau(x))$.  Notice that $V_x \in \tau^2(x)$.  It would
be natural to extend our typed language convention to allow $V$ to be used as a constant without the subscript in situations where its
type can be deduced from that of neighboring variables.  A statement such as $V \in V$ could be read as
$V_x \in V_{\tau(x)}$ (though strictly speaking probably one of the subscripts should be supplied).  In any event, all such statements are true.

\end{description}

Of course we assume that the universal set is not finite, but we do
not know how to say this yet.

The combination of the empty set and list notation allows us to write
things like $\{\emptyset,\{\emptyset\}\}$, but not things like
$\{x,\{x\}\}$: the former expression is another pun, with empty sets
of different types appearing, and the latter expression is
undefined, because $x$ and $\{x\}$ have different types.  An expression like this can make sense in a more usual
untyped set theory (and in fact in the usual set theory the first
expression here is the most popular way to define the natural number 2, as we
will explain later).

Set builder notation can be generalized.  

\begin{description}

\item[Generalized set builder notation:] If we have a complex term
$t[x_1,\ldots,x_n]$ containing only the indicated variables, we define
$$\{t[x_1,\ldots,x_n] \in \tau(u):  A\}$$ as $\{y \in \tau(u):  (\exists x_1\ldots
x_n.y=t[x_1,\ldots,x_n] \wedge A)\}$ (where $y$ is a new variable).
We do know that this kind of very abstract definition is not really
intelligible in practice except by backward reference from examples,
and we will provide these!

\item[Examples:] $\{\{x\}\in \tau(u):x=x\}$ means, by the above convention,
$$\{z\in \tau(u) \mid (\exists z.z=\{x\}\wedge x=x)\}.$$  It is straightforward to
establish that this is the set of all sets with exactly one element,
and we will see below that we will call this the natural number 1.
The notation $\{\{x,y\}\in \tau(u): x \neq y\}$ expands out to $\{z\in \tau(u) \mid
(\exists xy.z = \{x,y\} \wedge x\neq y)\}$: this can be seen to be the
set of all sets with exactly two elements (belonging to $\tau^2(u)$), and we will identify this
set with the natural number 2 below.

\end{description}


We define some familiar relations on sets.

\begin{description}

\item[subset, superset:] We define $A \subseteq B$ as $${\tt
set}(A)\wedge{\tt set}(B) \wedge A \sim_\tau B \wedge (\forall x.x \in A \rightarrow x \in
B).$$  We define $A \supseteq B$ as $B \subseteq A$.

\item[Theorem:] For any set $A$, $A \subseteq A$.

\item[Theorem:]  For any sets $A,B$, $A \subseteq B \wedge B \subseteq A \rightarrow A=B$.

\item[Theorem:] For any sets $A,B,C$, if $A \subseteq B$ and $B
\subseteq C$ then $A \subseteq C$.

\item[Observation:] The theorems we have just noted will shortly be
seen to establish that the subset relation is a ``partial order''.

\item[Proof Strategy:] To show that $A \subseteq B$, where $A$ and $B$
are known to be sets of the same type (and so with elements of the same type), introduce an arbitrary object $x$ and assume
$x\in A$: show that it follows that $x\in B$.

If one has a hypothesis or previously proved statement $A \subseteq B$ and a statement $t \in A$, deduce $t \in B$.

Notice that the proof strategy given above for
proving $A = B$ is equivalent to first proving $A \subseteq B$, then
proving $B \subseteq A$.

\end{description}

The notions of element and subset can be confused, particularly
because mathematicians and math students have a bad habit of saying things like ``$A$ is in $B$'' or
``$A$ is contained in $B$'' both  for $A \in B$ and for $A \subseteq B$.  It
is useful to observe that elements are not ``parts'' of sets.  The
relation of part to whole is transitive: if $A$ is a part of $B$ and
$B$ is a part of $C$, then $A$ is a part of $C$.  The membership
``relation'' is not transitive in a quite severe sense: if $A \in B$
and $B \in C$, then $A \not\in C$, because $C \in \tau^2(B) = \tau^3(A)$, and any set to which $A$ belongs is an element of $\tau^2(A)$, which is disjoint from $\tau^3(A)$.   But the subset
relation is transitive: if $A \subseteq B$ and $B \subseteq C$, then
any element of $A$ is also an element of $B$, and so is in turn an
element of $C$, so $A \subseteq C$.  If a set can be said to have
parts, they will be its subsets, and its one-element sets $\{a\}$ for
$a \in A$ can be said to be its atomic parts.



We give a general format for introducing operations, and then
introduce an important operation.

\begin{description}

\item[Definable Operations:] For any formula $\phi[x,y]$ with the
property that $$(\forall xyz.\phi[x,y] \wedge \phi[x,z] \rightarrow
y=z)$$ we define $F_{\phi}(x)$ or $F_{\phi}`x$ as the unique $y$ (if
there is one) such that $\phi[x,y]$.  Note that we will not always
explicitly give a formula $\phi$ defining an operation, but it should
always be clear that such a formula could be given.  Note also that
there might be a type differential between $x$ and $F_{\phi}(x)$
depending on the structure of the formula $\phi[x,y]$:  we require that
$\tau(F_\phi(x))$ be expressible for any $x$ for which it is defined
in some form $\tau^m(\tau^{-n}(\tau(x)))$ for fixed $m,n$.

For any such definable operation $F(x)$, we define $F``x$ for any set
$x$ as $\{F(u) \in \tau^m(\tau^{-n}(\tau(x))):  u \in x\}$, where $\tau^m(\tau^{-n}(\tau(x)))$ is the type of values of $F$ with arguments taken from $x$: $F``x$ is called the (elementwise) {\em
image\/} of $x$ under the operation $F$.

We also support iteration of such operations:  $F^{\bf 0}(x)$ is defined as
$x$ and $F^{\bf n+1}(x)$ is defined as $F(F^{\bf n}(x))$.   The numerals here are in boldface to indicate that no reference to natural numbers as mathematical objects is intended.

\item[Power Set:] For any set $A$, we define ${\cal P}(A)$ as $\{B
\in \tau(A): B \subseteq A\}$.  The power set of $A$ is the set of all subsets
of $A$.  Notice that ${\cal P}(V_x)$ is the collection of all
sets in $\tau^2(x)$, and is not necessarily the universe $\tau^2(x)$,
which might also contain some atoms.

\item[Singleton:]  For any object $x$, we define $\iota(x)=\{x\}$.  The primary use
of this alternative notation for the singleton operation is to allow notations like $\iota^{\bf 3}(x)$ for $\{\{\{x\}\}\}$.

\item[Observation:] $\cal P$ is $F_{\phi}$ where $\phi[x,y]$ is the
formula $(\forall z.z \in y \leftrightarrow z \subseteq x)$ (or just $y = \{z \in \tau(x):  z \subseteq x\}$).  The operator $\iota$ is $F_{\phi}$ where $\phi$ is $(\forall z.z \in y \leftrightarrow z=x)$.

\end{description}

It is {\bf very important} to notice that ${\cal P}(x)$ is one type
higher than $x$ (belongs to $\tau^2(x)$ instead of $\tau(x)$), and similarly that $\iota(x)=\{x\}$ is one type higher than
$x$.


\chapter{Appendix:  Description of the logic of Marcel}

A {\em sequent\/} is a pair $(p,g)$ where $p$ is a finite sequence of formulas (the posits or premises in the sequent) and
$g$ is a finite sequence of formulas (the goals or conclusions in the sequent).  The usual notation for $(p,g)$ would look like this:
$$p_1, \ldots, p_n \vdash g_1,\ldots,g_m.$$  Notice that this looks slightly different from the treatment in chapter 5, where the paired objects are sets of formulas rather than finite sequences of formulas.  Sequences have convenient structure for manipulation by a computer.

We note the syntax of propositional logic in Marcel:  a proposition identifier is a string of lower case letters followed by a question mark.  The symbols
$\neg, \wedge, \vee, \rightarrow, \leftrightarrow$ are replaced by

\begin{verbatim}

~, &, V, ->, == [note that the V is capitalized]

\end{verbatim}

\noindent due to the limitations of the typewriter keyboard.  Order of operations is as in the text, but conjunction and disjunction group by default to the left rather than to the right, for practical reasons having to do with what happens when large conjunctions or disjunctions are unpacked using the Marcel logical rules.  It remains advantageous for implication to group to the right, and the biconditional groups to the right as well.

\newpage

Notice also that in this notation posits are on the {\em left\/} and goals are on the {\em right}.  This feature of terminology is preserved in Marcel though the Marcel display shows posits above and goals below:

\begin{verbatim}

1. p_1
.
.
.
n.   p_n

--------------------

1. g_1
.
.
.
m. g_m


\end{verbatim}

Rules that act on posits or the list of posits are called left rules and rules that act on goals or the list of goals are called right rules.

We say that a sequent $(p,g)$ is {\em valid} iff any assignment of meanings to non-logical symbols in formulas in the ranges of $p$ and $g$ which makes all the formulas in the range of $p$ true makes at least one of the formulas in the range of $g$ true.

\newpage

The setting of Marcel which I use in teaching maintains the illusion that there are zero or more premises and exactly one goal in a sequent, which conforms
to the way that arguments are usually presented.

The display looks like this:

\begin{verbatim}

1. p_1
.
.
.
n.   p_n

*2.  ~g_2
.
.
.
*m ~g_m

--------------------

1. g_1

\end{verbatim}

If $m=0$, so there are no goals, the format is

\begin{verbatim}

1. p_1
.
.
.
n.   p_n

--------------------

_|_

\end{verbatim}

Observe that $$p_1, \ldots, p_n \vdash g_1,\ldots,g_m$$  is valid if and only if $$p_1, \ldots, p_n, \neg g_2,\ldots,\neg g_m \vdash g_1$$ is valid:  to show that if all the premises $p_i$ are true than some goal $g_i$ is true is the same thing as to show that if all the premises $p_i$ are true and all the goals
$g_2, \ldots g_m$ are false, then the goal $g_1$ has to be true, and this appears to be the best presentation for students.

The order of posits and goals really does not matter, though they do have to be presented in some order (which is one reason that it is an advantage for Marcel to represent sequents using sequences).

The {\tt gl} (get left) and {\tt gr} (get right) commands allow the order of the posits and goals to be manipulated.  If the sequent being viewed is

$$p_1, \ldots, p_n \vdash g_1,\ldots,g_m,$$ application of {\tt gl(i)} will change it to $$p_i,\ldots,p_n,p_1,\ldots,p_{i-1} \vdash g_1,\ldots,g_m,$$
and application of {\tt gr(i)} will change it to $$p_1, \ldots, p_n \vdash g_i,\ldots,g_m,g_1,\ldots,g_{i-1}$$

In the one-conclusion format {\tt gr(i)} changes $$p_1,\ldots,p_n,\neg g_2,\ldots,\neg g_m \vdash g_1$$ to $$p_1,\ldots,p_n,\neg g_{i+1},\ldots,\neg g_m,\neg g_1,\ldots,\neg g_{i-1} \vdash g_i.$$

If the sequent viewed is $$p_1, \ldots, p_n \vdash g_1,\ldots,g_m,$$ and it happens that $p_1$ is the same formula as $g_1$ then the sequent is valid.
If the {\tt Done()} command is issued in this situation, Marcel records this sequent as proved and displays the next unproved sequent in the current proof (or declares the original theorem proved if no unproved sequents are left).

The previous command lets us finish things:  we ought to report how we can get started:  the command {\tt s('p')} will set up the sequent
$\vdash p$ to be proved valid, where $p$ is any formula.

The very powerful {\tt l()} (left) and {\tt r()} (right) commands act on the first posit (left) or the first goal (right) applying the appropriate logical transformation based on the form of the leading posit or goal (as appropriate), producing either one or two new sequents to prove valid.  Once these sequents are proved valid, the original sequent is recorded as proved valid.

\begin{description}

\item[conjunctions as posits:]  $$P \wedge Q,p_2,\ldots,p_n \vdash g_1,\ldots,g_m$$ is valid iff $$P, Q,p_2,\ldots,p_n \vdash g_1,\ldots,g_m$$ is valid. Thus, if the first posit in the sequent viewed is a conjunction, the effect of applying the {\tt l()} command will be to break it apart into two posits.

\item[conjunctions as goals:]  $$p_1,\ldots,p_n \vdash P \wedge Q,g_2,\ldots,g_m$$ is valid iff both $$p_1,\ldots,p_n \vdash P,g_2,\ldots,g_m$$ and $$p_1,\ldots,p_n \vdash Q,g_2,\ldots,g_m$$ are valid.  So if the first goal in the sequent (which in the one-conclusion format is the only goal below the line) is a conjunction, Marcel will first present the sequent with the conjunction replaced by its first part to prove, and then present the sequent with the conjunction replaced by the second part to prove:  the strategy for proving $P \wedge Q$ under given assumptions is to first prove $P$ with those assumptions, and then prove $Q$ with those assumptions.

\item[disjunctions as posits:]   $$P \vee Q,p_2,\ldots,p_n \vdash g_1,\ldots,g_m$$ is valid iff $$P,p_2,\ldots,p_n \vdash g_1,\ldots,g_m$$ is valid and $$Q,p_2,\ldots,p_n \vdash g_1,\ldots,g_m$$ is valid.  So if the first posit is a disjunction $P \vee Q$ and the {\tt l()} command is applied, Marcel first presents the sequent with $P \vee Q$ replaced by $P$ to be proved valid, then presents the sequent with $P \vee Q$ replaced by $Q$ to be proved valid.  This is exactly the strategy of proof by cases!

\item[disjunctions as goals:]  $$p_1,\ldots,p_n \vdash P \vee Q,g_2,\ldots g_m$$ is valid iff $$p_1,\ldots,p_n \vdash P,Q,g_2,\ldots g_m$$ is valid.  This is delightfully simple under the hood, but in the one-conclusion mode it looks slightly more complicated:  $$p_1,\ldots,p_n,\neg g_2,\ldots \neg g_m \vdash P \vee Q$$ is valid iff $$p_1,\ldots,p_n,\neg Q,\neg g_2,\ldots \neg g_m \vdash P$$ is valid.  Notice that this is one of the cases of alternative elimination.  The other case is readily recovered by issuing the command {\tt gr(2)} to make $Q$ the chosen conclusion instead of $P$.

\item[negations as posits:]  The sequent $$\neg P,p_2,\ldots,p_n\vdash g_1\ldots g_m$$ is valid iff $$p_2,\ldots,p_n\vdash P,g_1\ldots g_m $$ is valid.
Like any rule which adds or removes a posit, this has less obvious effects in one-conclusion mode.  The sequent $$\neg P,p_2,\ldots,p_n,\neg g_2,\ldots \neg g_m\vdash g_1$$ is valid iff $$p_2,\ldots,p_n,\neg g_1\ldots \neg g_m \vdash P$$ is valid.  The strategy presented is that one can prove a sequent with  a posit 
$\neg P$ by instead proving from the other original hypotheses and the negation of the original first goal that $P$ must be true.  

\item[negations as goals:]  The sequent $$p_1,\ldots,p_n \vdash \neg P,g_2,\ldots,g_m$$ is valid iff the sequent $$P,p_1,\ldots,p_n \vdash g_2,\ldots,g_m$$  is valid.  In the one-conclusion mode, this has interesting effects:  $$p_1,\ldots,p_n,\neg g_2,\ldots,\neg g_m \vdash \neg P$$ is valid iff the sequent $$P,p_1,\ldots,p_n,\neg g_3 ,\ldots,\neg g_m\vdash g_2$$  is valid.  Notice that when the goal $\neg P$ is removed, the former second goal becomes first goal, and in one-conclusion mode this looks nontrivial.  Of course, if there is no second goal, we will get the empty conclusion situation with the fake goal of $\perp$.  This is the standard strategy of negation introduction:  to prove that $\neg P$, assume $P$ and deduce a contradiction.  But the transformation in the one-conclusion mode makes it look like a contrapositive move:  to prove $\neg P$ from the assumption $\neg g_2$, assume $P$ and prove $g_2$.  If there is no $g_2$, we get the usual strategy of negation introduction.

\item[implications as goals:]  The sequent $$p_1,\ldots,p_n \vdash P \rightarrow Q,g_1,\ldots,g_m$$ is valid iff $$P, p_1,\ldots,p_n \vdash Q,g_1,\ldots,g_m$$ is valid.  This gives Marcel an action exactly similar to our rule of deduction.

\item[implications as posits:]  The sequent $$P \rightarrow Q,p_2,\ldots,p_n \vdash g_1,\ldots, g_m$$ is valid iff both of the sequents  $$p_2,\ldots,p_n \vdash P, g_1,\ldots, g_m$$ and $$Q,p_2,\ldots,p_n \vdash g_1,\ldots, g_m$$ are valid.  This looks a little different in one-conclusion mode:  
$$P \rightarrow Q,p_2,\ldots,p_n ,\neg g_2,\ldots, \neg g_m\vdash g_1$$ is valid iff both of the sequents  $$p_2,\ldots,p_n, \neg g_1,\ldots, \neg g_m \vdash P$$ and $$Q,p_2,\ldots,p_n,\neg g_2,\ldots, \neg g_m \vdash g_1$$ are valid.  This rule is always the hardest one to follow.  The problem is that we are used to using the posit $P \rightarrow Q$ together with $P$ in the rule of modus ponens, but Marcel prefers to act on a single posit.  The strategy implemented here is to attempt to show that $P$ follows from the other hypotheses, then that the original conclusion follows from $Q$ and the other hypotheses:  if this works, it does show that the original conclusion follows from the original hypotheses:  deduce $P$, apply modus ponens to get $Q$, then deduce the original goal.
The additional option is provided of swapping $P$ for the original goal in the first sequent:  in this case the original conclusion follows without using the posit $P \rightarrow Q$ at all.

\item[biconditionals as posits:]  $$P \leftrightarrow Q,p_2,\ldots,p_n \vdash g_1,\ldots,g_m$$ is valid iff $$P \rightarrow Q,Q\rightarrow P,p_2,\ldots,p_n \vdash g_1,\ldots,g_m$$ is valid.  We choose to just break a biconditional apart as a conjunction is broken apart:  the user can break apart whichever of the new implication posits they want to use.

\item[biconditionals as goals:]  $$p_1,\ldots,p_n\vdash P \leftrightarrow Q,g_1,\ldots,g_m$$ is valid iff $$P,p_1,\ldots,p_n\vdash Q,g_1,\ldots,g_m$$ is valid and $$Q,p_1,\ldots,p_n\vdash P,g_1,\ldots,g_m$$ is valid.  This gives precisely the behavior we expect.


\end{description}

This essentially completes the rules for propositional logic (Marcel supports a couple of other less-used operators).

We note the syntax of quantification for Marcel.  What we write $(\forall x.P[x])$, Marcel writes as {\tt (A x : P[x])} (of course $P[x]$ is not Marcel notation:  this stands in for the Marcel translation of $P[x]$).  What we write $(\exists x.P[x])$, Marcel writes as {\tt (E x : P[x])} (of course $P[x]$ is not Marcel notation:  this stands in for the Marcel translation of $P[x]$).   It is an interesting fact in the background that Marcel understands $(\forall x.P[x])$ as having the underlying form $\forall(\{x:P[x]\})$ and $(\exists x.P[x])$ as having the underlying form $\exists(\{x:P[x]\})$:  for Marcel, set-builder notation is more basic.  But this does not affect the way that quantifier rules are implemented.

We present the rules for handling quantified goals and posits.

\begin{description}

\item[universally quantified statements as goals:]

A sequent $$p_1,\ldots,p_n \vdash (\forall x:A[x]),g_2,\ldots,g_m$$ is valid iff $$p_1,\ldots,p_n \vdash A[a],g_2,\ldots,g_m$$ is valid, where $a$ is an atomic constant not appearing anywhere in the original sequent.  This precisely reproduces the rule of universal generalization.  Marcel actually generates the term $a$ by applying a fresh numerical index to the bound variable $x$ (producing a variable {\tt x\_n} with $n$ a fresh numerical index).

\item[universally quantified statements as posits:]

A sequent $$(\forall x:A[x]),p_2,\ldots,p_n \vdash g_1,\ldots,g_m$$ is valid iff $$A[t],(\forall x:A[x]),p_2,\ldots,p_n \vdash g_1,\ldots,g_m,$$ where $t$ is any term at all.
What Marcel actually does is provide in place of $t$ an ``instantiable" {\tt x\$n}, $n$ being a fresh index.  Marcel then allows the instantiable {\tt x\$n} to be replaced at any later point, throughout the entire current proof, with any term not containing any constant or instantiable with index $\geq n$.  The advantage of ``instantiables" is that it might become clear only later what the best value is to plug in for $x$, and further that the {\tt l()} command can be used for universal posits:  in old versions of Marcel, a different command was needed for universal posits and existential goals, which required the intended $t$ as an argument.  The fact that a copy of the universally quantified posit is preserved reflects the fact that more than one instance of the universally quantified posit might be wanted in a proof.  It also preserves the precise equivalence of the validity of the original sequent and the validity of the new sequent.

\item[existentially quantified statements as goals:]

A sequent  $$p_1,\ldots,p_n \vdash (\exists x:A[x]),g_2,\ldots,g_m$$ is valid iff $$p_1,\ldots,p_n \vdash A[t],(\exists x:A[x]),g_2,\ldots,g_m$$ is valid, where $t$ is any term.
Application of the {\tt r()} command introduces $t$ as an instantiable {\tt x\$n}:  this can later be assigned a value (throughout the current proof) as discussed above.  The idea is that we then can handle existential posits with the parameter-free {\tt r()} command like all other posits, and also that it may not become evident until later in the proof what the best witness is to choose.  This looks a little different in the one-conclusion format:  $$p_1,\ldots,p_n,\neg g_2,\ldots,\neg g_m \vdash (\exists x:A[x])$$ is valid iff $$p_1,\ldots,p_n,\neg(\exists x:A[x]),\neg g_2,\ldots,\neg g_m \vdash A[t]$$ is valid.  To understand the preservation of the original goal as an alternative goal, consider that if we choose the wrong witness by mistake we can move the existential goal back into the first goal slot and instantiate it again.  This can be important if different witnesses are to be chosen in different cases in an argument.

\item[existentially quantified statements as posits:]

A sequent $$(\exists x:A[x]),p_2,\ldots,p_n \vdash g_1,\ldots,g_n$$ is valid iff $$A[a],p_2,\ldots,p_n \vdash g_1,\ldots,g_n$$ is valid, where $a$ is a new atomic constant not appearing in the original sequent.  Marcel implements $a$ in the form ${\tt x\_n}$, where $n$ is a fresh index.  This implements the rule of witness introduction quite naturally.

\end{description}

The command that instantiates symbols ${\tt x\$n}$ takes two forms:  to replace ${\tt x\$n}$ with $t$, issue the command {\tt Inst('t','x\$n')}  or the alternative form
{\tt SU('x\$n := t')}.   The appearances of {\tt t} of course are to be replaced by a possibly quite complex expression.

\newpage

\printindex

\end{document}

Notes from my diary.

work on 502 text -- this is looking like a really good idea ;-)
   502 text is designed for teaching with Marcel and could reasonably
   contain theoretical stuff about NFU needed for Marcel (not  
   necessarily covered in this course).  Teaching idea is to
   use proofs in TST and development of mathematics in TST
   as a vehicle for teaching proof and formal mathematical
   construction.  At end we need to convert to Zermelo and
   finally briefly introduced the axiom of replacement and the stronger
   theory ZFC.  I also need to look over the M502 course
   description.

   reasons why one might use TST:  historically this was
   an early approach.  The proofs are less likely to be familiar
   The treatment of well-orderings is better (but the von Neumann
   ordinals (and cardinals) do need to be specifically introduced
   in the transition to ordinary set theory.  Another reason
   to use TST is that the logic of Marcel looks like TST
   (though it actually isn't).

   section 1:  proofs, formal logical notation.  Introduce
   definite description and Hilbert symbol early.  Hilbert symbol
   is important for term models.  Work order:  implement general
   binders in Marcel...

   section 2:  sets

      2a:  development of TST and implementation of some math
      in TST, with attention to writing proofs.
      ordered pairs, construction of functions and relations.
      construction of the natural numbers and arithmetic;
      give the Peano axioms and do some proofs;
      then give definition of the natural numbers in TST
      and prove the Peano axioms
      and also perhaps do natural proofs for arithmetic
      theorems using set theory definitions.
      [should Quine ordered pair be introduced?  after all,
      Marcel's pair is type level]
      construction of the reals and analysis;
      Equinumerousness
      and similarity of well-orderings; 
      transfinite induction and recursion;
      cardinal and ordinal numbers.

      Cantor's theorem?  Hartog's theorem?  aleph and beth numbers?
      Konig's Lemma?  basic transfinite arithmetic?
      Ramsey and Erd\"os Rado theorems?

      The Axiom of Choice:  show its equivalence to well-ordering
      theorem and Zorn's Lemma.

      2b.  development of ordinary set theory from TST.  von Neumann
      ordinals and cardinals.

      2c.  Higher powered stuff.  Axiom of replacement and ZFC.

   section 3:  formalized logic

      3a.  Formal proof in sequent calculus (as implemented in
      Marcel.  Introduce the single conclusion variant as well
      as the main version.

      3b.  Formal semantics.  Completeness and consistency
      for propositional logic.  Completeness theorem for first-order
      logic by building term models.  Mathematical representation
      of formulas and terms.  Formal definition of substitution?
      Compactness theorem; Lowenheim Skolem theorem (all focussed
      on term models).  Nonstandard models?

   section 4:  alternative set theories

      4a:  basic stuff about NFU, and discussion of its implementation
      in Marcel.  This fits in naturally due to the relation of the
      book to the software...

      4b.  positive set theory?  We know this can also be developed
      starting with TST.

Math 502 Logic and Set Theory (3-0-3)(S)(even-numbered years). This
course is structured as three 5-week components: formal logic, set
theory, and topics to be determined by the instructor. The logic
component will include: formalization of language and proof, the
completeness theorem, the Lowenheim-Skolem Theorem. The set theory
component will include: cardinality, Cantor's theorem, well-orderings,
ordinals, the transfinite recursion theorem, the Axiom of Choice and
its equivalents. PREREQ: Math 314.







