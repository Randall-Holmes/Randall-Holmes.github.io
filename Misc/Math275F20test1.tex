\documentclass[12pt]{article}

\usepackage{amssymb}

\title{Math 275 sections 1,5,6 Fall 2020:  Test I}

\author{Dr Holmes}

\date{Sept 18-20, 2020}

\begin{document}

\maketitle

Please work this exam out on paper, in private, without seeking advice from anyone but the instructor (from whom you may seek clarification of the meaning of a question or to whom you may report  a suspected error in a question).

Please return your paper to the instructor by email before 11:55 pm on Sunday the 20th of September.  You should scan it and send me the electronic file (pictures taken with your phone are acceptable if that is all you can do).  The file or files that you send me should have informative names:  I should be able to tell from the file name what your name is, that you are a student in Math 275 section x, and that this is a Test I paper.

\newpage

\begin{enumerate}

\item  Draw on your paper vectors {\bf a} and {\bf b} which are not parallel.  Then construct copies of ${\bf a+2b}$, ${\bf a-b}$, and ${\bf -\frac32 a}$ in convincing ways.  You may copy any vector (preserving magnitude and direction as best you can) but be sure to label copies of a vector so that I can tell what you intend.

\newpage

\item   Compute the scalar and vector projections of $\left<1,3,5\right>$ onto $\left<1,-1,2\right>$.

Express $\left<1,3,5\right>$ as the sum of a vector parallel to $\left<1,-1,2\right>$ and a vector orthogonal to $\left<1,-1,2\right>$.  Your work should include a check that the second variable is in fact orthogonal to $\left<1,-1,2     \right>$.

\newpage

\item     Coordinates of three points are given:  $A = (2,1,2);  B = (1,3,5);  C = (4,4,4)$.        

Determine the angle $\angle BAC$ in degrees with two decimal places of accuracy.

Compute the area of $\triangle ABC$.    

\newpage

\item  Let $A,B,C$ be the same points as in the previous problem.

Give scalar parametric equations and symmetric equations for line $AC$.

Determine an equation for the plane in which points $A,B,C$ lie, in the form $ax+by+cz = d$.

\newpage

\item  The line with symmetric equations $$\frac{x-5}2 = \frac{y-4}1 = \frac{z-7}3$$
intersects the line with symmetric equations $$\frac{x-6}1 = \frac{y}{-1}=\frac{z-7}1.$$

Write vector and scalar parametric equations for each line.

Find the point of intersection of the two lines.

\newpage

\item  The planes $x+y+z=6$ and $x-2y-2z = -9$ share the point (1,2,3).

Find vector parametric equations for the line of intersection of these two planes.

What is the angle between the two planes?

\newpage

\item   The equation ${\bf r}(t) = \left<3\cos(t),3\sin(t),4t\right>$ parameterizes a helix.

Determine the tangent vector to this helix at the point $(0,3,2\pi)$.

Determine vector parametric equations for the line tangent to this curve at $(0,3,2\pi)$.

Determine the arc length of the portion of this helix between $(3,0,0)$ and $(0,3,2\pi)$.

\newpage

\item  The acceleration of a particle at time $t$ is given by ${\bf a}(t) = \left<1,t,-t\right>$.

The velocity of the particle when $t=0$ is $\left<-1,0,1\right>$.

The position of the particle when $t=0$ is $\left<1,1,1\right>$.

Determine the position of the particle at any time $t$.  Determine its position when $t=2$.

\newpage





\end{enumerate}

\end{document}