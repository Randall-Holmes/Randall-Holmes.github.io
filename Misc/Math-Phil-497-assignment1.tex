\documentclass[12pt]{article}

\usepackage{frege}

\usepackage{yfonts}

\usepackage{amssymb}

\title{Math/Phil 497 Fall 2020 Assignment 1}

\author{Randall Holmes}

\begin{document}

\maketitle

This is the first assignment.  For greatest instructor happiness, please turn in your paper electronically by 11:55 pm on Wednesday Sept 9.  Any files you turn in to me should have long file names, including your last name, the name of the course, and the name of the assignment, so that when I look at a file I can tell where to put it without opening it.

\begin{enumerate}

\item  Rewrite the following expressions in standard logical notation in Frege's notation.  You are welcome to draw Frege notation by hand.  You might need to look at the notes
or recall things I say in the Sept 2 lecture to recall how to deal with ``and" and ``or" (or figure it out on your own, of course).

Please take advantage of obvious simplifications (double negations come to mind, for example).

\begin{enumerate}

\item $P \rightarrow \neg Q$

\item$(P \rightarrow (Q \vee R)$

\item $P \wedge Q \wedge R$

\item  $((P \rightarrow Q) \wedge (Q \rightarrow R)) \rightarrow (P \rightarrow R)$

\item $(\forall x:(\forall y: x = y \rightarrow y = x))$

\item $(\forall x \neq 0:(\exists y:x\cdot y = 1))$  You have to see how to expand the restricted quantifier $(\forall x \neq 0:$, of course.


\end{enumerate}



\item Rewrite the following expressions in Frege's notation into standard logical notation and then into mathematical English.  You have style choices in how to do it, but I will of course be most impressed by the simplest answers.

\begin{enumerate}

\item  $\Fcdt[\Fn]{\Fn P}{\Fn Q}$

\item  $\Fcdt[\Fq{a}]{\Fcdt[\Fq{b}\Fq{c}]{\F \textfrak{a}\cdot \textfrak {b} > \textfrak{a} \cdot \textfrak{c}}{\F \textfrak{b} < \textfrak{c}}}{\F \textfrak{a} < 0}$

\item $\Fcdt[\Fnq{a}]{\Fn \textfrak{a} > 3}{\F \textfrak{a} <5}$  The most natural answer to this is an existential statement.

\end{enumerate}

\item  Classify each of the following functions as first-level, second-level, or mixed.  Discuss each one briefly, enough that I can see why you gave the answer you did.
You may have further comments about what kinds of arguments or how many arguments a function takes;  say what you think is important.

If a function is a concept or relation, say so.

\begin{enumerate}

\item the function $x^2+1$

\item the function $x^2+y^2=0$

\item the function $\Fn x$

\item the function $\Fcdt[\Fq{a}]{\F g(\textfrak{a})}{\F f(\textfrak{a})}$  hints:  this is a function of $f$ and $g$.  What sort of thing is $f$?  What sort of thing is $g$?
What does the sentence actually say about $f$ and $g$?

\item the calculus function $\int_x^3\,g(t)\,dt$

\end{enumerate}

\end{enumerate}

\end{document}