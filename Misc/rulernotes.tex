\documentclass[12pt]{article}

\title{Dr Holmes's notes on the Ruler Postulate}

\author{Randall Holmes}

\usepackage{amsmath}
\usepackage{amssymb}

\newcommand\Line[1]{\overset{\leftrightarrow}{#1}}

\begin{document}

\maketitle

These are some notes on the axiomatics in Venema.  Venema has a wonderful book, but some things I think benefit from a different approach.

Primitive notions that we start with.  There are {\em points\/}.  There are {\em lines\/}, which are sets of points (a line is a set of points, but of course not all sets of points are lines).  We will refer to the set of all points as the plane.

We have a primitive $d$, the {\em distance function\/}, which sends any pair of points to a real number.  
Note that the distance from point $P$ to point $Q$, which we write $d(P,Q)$, is written just as $PQ$ in Venema, which we find potentially confusing and so avoid.

More primitive notions will be introduced later.

The first two axioms are

\begin{description}


\item[Existence Postulate:]  There are at least two distinct points.

\item[Incidence Postulate:]  For each pair of distinct points $A,B$ there is exactly one line $L$ such that
$A \in L$ and $B \in L$:  we denote this line by $\Line{AB}$.

\end{description}

The next axiom is best preceded by a definition.

\begin{description}


\item[Definition (coordinate function):]  Let $L$ be a line.  A {\em coordinate function for $L$\/} is defined as a function from $L$ to the set $\mathbb R$ of real numbers which is 

\begin{enumerate}\item one-to-one (for any points $P, Q$, $f(P) =f(Q) \rightarrow P=Q$), \item onto (for any real number $r$, there is a point $P\in L$ such that $f(P) = r$;  we can also write $P=f^{-1}(r)$), \item and distance preserving:  for any points $P,Q \in L$, $d(P,Q) = |f(P)-f(Q)|$.\end{enumerate}
\item[Ruler Postulate:]  For any line $L$, there is a coordinate function for $L$.

\item[Semi-Metric Theorem:]  For any points $A,B$, $d(A,B)\geq 0$, $d(A,B) = d(B,A)$, and $d(A,B)=0$ iff $A=B$.

\item[Proof of Semi-Metric Theorem:]

Let $A,B$ be points.   Define a point $P$ as $B$, in case $B$ is distinct from $A$, and otherwise as some point distinct from $A$ (there is such a point by the Existence Postulate).

Let $L$ be the line $\Line{AP}$.  Note that both $A$ and $B$ are on $L$, because $B$ is either $A$ or $P$.  Note the use of the Incidence Postulate.

Let $f$ be a coordinate function for $L$.

$d(A,B) = |f(A)-f(B)| = |f(B)-f(A)| = d(B,A)$.

$d(A,B)= |f(A)-f(B)| \geq 0$.  Further, $|f(A)-f(B)|=0$ if and only if $f(A)=f(B)$, and in turn this is true if and only if $A=B$, because $f$ is one-to-one.

\end{description}

As we note briefly under one of the headings, a coordinate function $f$ for $L$ has an inverse $f^{-1}$ such that for each real number $r$, $f^{-1}(r)$ is the unique point $P$ on $L$ such that $f(P)=r$.



The use of the Ruler Postulate depends on some facility with the notion of absolute value.

A line does not have a uniquely determined coordinate function.  We give a complete account of what coordinate functions a line has.

\begin{description}



\item [Theorem:]  Let $L$ be a line and let $f$ be a coordinate function for $L$.  For any real number $c$ and any $\sigma = \pm 1$, the function $g$ from $L$ to $\mathbb R$ defined by $g(P) = \sigma f(P) + c$ is a coordinate function.

\item[Proof:]  $g$ is one-to-one:  suppose $g(P) = g(Q)$.  It follows by definition of $g$ that $\sigma f(P) + c = \sigma f(Q) + c$, from which it follows by algebra that $f(P) = f(Q)$ from which it follows by the fact that $f$ is a coordinate function and so one-to-one that $P=Q$, so we have shown that $g$ is one-to-one.

$g$ is onto:  Let $r$ be a real number.  We want to find a point $P$ on $L$ such that $g(P)=r$.  That is, we want to find $P$ such that $\sigma f(P) + c = r$, for which we need $f(P) = \sigma(r-c)$.  So let $P = f^{-1}(\sigma(r-c))$.  $g(P) = g(f^{-1}(\sigma(r-c))) = \sigma(f(f^{-1}(\sigma(r-c))))+ c = \sigma(\sigma(r-c)) + c = (r-c)+c = r$.  Note the use of the fact that $\sigma=\pm 1$, so $\sigma^2=1$.

$g$ is distance preserving:  $|g(P)-g(Q)| = |(\sigma f(P) + c) - \sigma f(Q) +c| = |\sigma(f(P) - f(Q))| = |\sigma||f(P)-f(Q)| = |f(P)-f(Q)| = d(P,Q)$.  Notice the use of the fact that $|\sigma|=1$ and the fact that $f$ is a coordinate function and so distance preserving.

\item[Observation about absolute values:]  For any real number $x$, there is $\sigma = \pm 1$ such that $|x| = \sigma x$, and for any $\tau$,  if $\tau = \pm1$ and $\tau x \geq 0$, $\tau x = |x|$.

\item[Lemma:]  If $r,s,x,y$ are real numbers, $r \neq s$,  and $|x-r|=|y-r|$ and $|x-s| = |y-s|$ then $x=y$.  In a geometric manner, we can say that if $r$ and $s$ are distinct real numbers, and $x$ and $y$ have the same distances from $r$ and $s$ respectively, then $x=y$:  if we know the distance of a real number from both $r$ and $s$, we have exactly determined that number.

\item[Proof:]  Let $r \neq s$.  Let $|x-r|=|y-r| = d_1$ and let $|x-s|=|y-s|=d_2$.

For any $z$, $T$ and $d$, $|z-t|=d$ implies that there is $\sigma=\pm 1$ such that $z= t+\sigma d$.

It follows from this that there is $\sigma_1 = \pm 1$ such that $x=r+\sigma_1 d_1$, and if $y \neq x$, it follows
that $y = r-\sigma_1 d_1$.  Similarly, there is $\sigma_2 = \pm 1$ such that $x = s+ \sigma_2 d_2$ and
if $y \neq x$, it follows that $y = s-\sigma_2d_2$.

It then follows that $x+y = (r + \sigma_1 d_1) + (r -\sigma_1d_1) = 2r$ and $x+y = (s + \sigma_2 d_2) + (s -\sigma_2d_2) = 2s$, so $2r=2s$, so $r=s$, which is a contradiction, so our assumption that $y \neq x$ is shown to be false.

I enjoy the elimination of case analysis by the use of variables equal to 1 or $-1$ in this presentation.

\item[Corollary:] If $f$ and $g$ are coordinate functions for the same line $L$, and $P \neq Q$ are distinct points on $L$, and $f(P) = g(P)$ and $f(Q)=g(Q)$, we have $f=g$.  Coordinate functions need only agree at two distinct points to be known to be equal.

\item[Proof:]  Let $R$ be an arbitarily chosen point on $L$.  

We have $d(R,P) = |g(R)-g(P)|$ and  $d(R,P) = |f(R)-f(P)|$.
But also $d(R,P) = |g(R)-g(P)|=|g(R)-f(P)|$.

We have $d(R,Q) = |g(R)-g(Q)|$ and  $d(R,P) = |f(R)-f(Q)|$.
But also $d(R,Q) = |g(R)-g(Q)|=|g(R)-f(Q)|$.

Now apply the previous lemma with $x = f(R), y= g(R), r = f(P), s = f(Q)$ to conclude that $f(R)=g(R)$ for every $R \in L$, so $f=g$.

\item[Theorem:]  If $L$ is a line with coordinate function $f$, and we use $R$ as an independent variable ranging over $L$, every coordinate function $g$ is if the form $g(R) = c + sf(R)$ where $c$ is a real number and
$s=\pm 1$.

\item[Proof:]  Let $L$ be a line.  Let $f$ be a coordinate function for $L$.  Let $g$ be a coordinate function for $L$.

Let $P,Q$ be two distinct points on $L$.  Define $h$, a function from $L$ to the real numbers, by $h(R) = g(P)  + \frac{g(Q)-g(P)}{f(Q)-f(P)}(f(R)-f(P))$.

$h$ is a coordinate function because $h(R)=c + sf(R)$ where $c$ is a real number and
$s=\pm 1$. ($c$ being $g(P) - \frac{g(Q)-g(P)}{f(Q)-f(P)}(f(P))$, and $s$ being $\frac{g(Q)-g(P)}{f(Q)-f(P)}$)

$h(P) = g(P)  + \frac{g(Q)-g(P)}{f(Q)-f(P)}(f(P)-f(P)) = g(P)$

$h(Q) = g(P)  + \frac{g(Q)-g(P)}{f(Q)-f(P)}(f(Q)-f(P)) = g(P) + g(Q)-g(P) = g(Q)$

so by the previous corollary, $h$ is the same coordinate function as $g$, since they agree at two distinct points, and $h$ is of the form $h(R) = c + sf(R)$ where $c$ is a real number and
$s=\pm 1$, establishing that $g$ is of this form.

\end{description}

Now we introduce the notions of betweenness, segments, congruence, and rays.

\begin{description}

\item[Definition:]  We say that three points $A,B,C$ are {\em collinear\/} iff $A \neq B$, $A \neq C$, $B \neq C$, and there is a line $L$ such that $A \in L$, $B \in L$, and $C\in L$, i.e., the three points are distinct, and they all lie on the same line.

\item[Definition:]  Let $A,B,C$ be points.  We define $A*B*C$, read ``$B$ is between $A$ and $C$" as meaning ``$A, B,$ and $C$ are collinear and $$d(A,B) +d(B,C) = d(A,C).$$

\item[Theorem:]  Let $L$ be a line and let $A,B,C$ be three distinct points on $L$.  Let $f$ be a coordinate function for $L$.  Then $A*B*C$ holds if and only if
either $f(A) < f(B) < f(C)$ or $f(C)<f(B)<f(A)$.

\item[Proof:]  $d(A,B) = \sigma_1(f(B) - f(A))$, where $\sigma_1=\pm 1$ and $\sigma_1(f(B) - f(A)) = \sigma_1f(B) -\sigma_1f(A)>0$ (definition of absolute value).

 $d(B,C) = \sigma_2(f(C) - f(B))$, where $\sigma_2=\pm 1$ and $\sigma_2(f(C) - f(B)) = \sigma_2f(C) -\sigma_2f(B) >0$ (definition of absolute value).

There are two cases:  either $\sigma_1 = \sigma_2$ or $\sigma_1 \neq \sigma_2$.

\begin{description}

\item[Case 1 ($\sigma_1 = \sigma_2$):]If $\sigma_1=\sigma_2$, then $d(A,B)+d(B,C) =  \sigma_1f(B) -\sigma_1f(A) + \sigma_2f(C) -\sigma_2f(A) =  \sigma_1f(B) -\sigma_1f(A) + \sigma_1f(C) -\sigma_1f(B) = \sigma_1f(C) -\sigma_1f(A) = \sigma_1(f(C) - f(A)) = |f(C)-f(A)|$ (because $\sigma_1 = \pm 1$ and this is the sum of two nonnegative (in fact positive) quantities and so certainly nonnegative) $=d(A,C)$, so $A*B*C$ holds.

We also have $\sigma_1f(A) < \sigma_1f(B) < \sigma_1f(C)$, so either $f(A)<f(B)<f(C)$ or $f(C)>f(B)>f(A)$, so in this case we have $A*B*C$  if and only if
either $f(A) < f(B) < f(C)$ or $f(C)<f(B)<f(A)$, because both are true.

\item[Case 2 ($\sigma_1 \neq \sigma_2$):]  If $\sigma_1 \neq \sigma_2$ then $\sigma_2 = -\sigma_1$ and $d(A,B)+d(B,C) =  \sigma_1f(B) -\sigma_1f(A) + \sigma_2f(C) -\sigma_2f(A) =  \sigma_1f(B) -\sigma_1f(A) - \sigma_1f(C) +\sigma_1f(B) = 2\sigma_1f(B) - \sigma_1f(A) - \sigma_1f(C)$.  We show that this is greater than either $\sigma_1(f(C)-f(A))$ or
$-\sigma_1(f(C)-f(A))$ and so is greater than $d(A,C)$ (which is equal to whichever of these is positive).

$(2\sigma_1f(B) - \sigma_1f(A) - \sigma_1f(C)) - \sigma_1(f(C)-f(A)) = 2\sigma_1(B) - 2 \sigma_1(C) = 2\sigma_2(f(C)-f(B)) >0$

$(2\sigma_1f(B) - \sigma_1f(A) - \sigma_1f(C)) - (-\sigma_1(f(C)-f(A))) = 2\sigma_1(B) - 2 \sigma_1(A) = 2\sigma_1(f(B)-f(A)) >0$

This establishes the two inequalities.  So if $\sigma_1 \neq \sigma_2$, we have $d(A,B)+d(B,C)>d(A,C)$, so $A*B*C$ does not hold.

We cannot have $f(A) < f(B) < f(C)$ or $f(C)<f(B)<f(A)$ in this case, because either of these inequalities implies $\sigma_1 = \sigma_2$ and forces us into the other case.  So in this case we also have $A*B*C$  if and only if
we have either $f(A) < f(B) < f(C)$ or $f(C)<f(B)<f(A)$, because both are false.
\end{description}

\item[Theorem (properties of betweenness):]  From the previous theorem and basic properties of order, we get three essential properties of betweenness which can be stated without reference to numbers: 
\begin{enumerate}

\item  for any points $A,B,C$, $A*B*C$ if and only if $C*B*A$, 

\item  for any three distinct points $A,B,C$ lying on the same line, exactly one of $A*B*C$, $A*C*B$, and $B*A*C$ holds, 

\item and for any points $A,B,C,D$, if  $A*B*C$ and $A*C*D$ hold then $A*B*D$ holds.

\end{enumerate}

\item[Proof:]  If $A*B*C$ than $A,B,C$ are collinear and $d(A,B)+d(B,C) = d(A,C)$ [definition] so $C,B,A$ are collinear [symmetry of definition of collinear] and $d(C,B) + d(B,A) = d(C,A)$ (symmetry of distance and commutativity of addition] so $C*B*A$ [definition of betweenness].

If $A,B,C$ are distinct and lie on the same line, we can choose without loss of generality a coordinate function $f$ for the line
such that $f(A) < f(C)$ (otherwise we could replace it with $-f$), and then we have exactly one of $f(A)<f(B)<f(C)$, $f(A)<f(C)<f(B)$
and $f(B)<f(A)<f(C)$ by basic properties of order on the real line, so we have exactly one of $A*B*C$, $A*C*B$, and $B*A*C$ by the Betweenness Theorem.

If $A,B,C,D$ are points and $A*B*C$ and $A*C*D$, then all four points lie on the same line by the incidence axiom.  Choose a coordinate function $f$ for this line such that $f(A)<f(B)$.  We then have $f(A)<f(B)<f(C)$ ($A*B*C$, betweenness theorem and the given order relation) so $f(A)<f(C)$, so $f(A)<f(C)<f(D)$ ($A*C*D$, betweenness theorem, and the given order relation) so $f(A)<f(B)<f(D)$ (order properties of the reals), so $A*B*D$ (betweenness theorem).

\item[Definition:]  Let $A$ and $B$ be two distinct points.  The segment from $A$ to $B$, written $\overline{AB}$ is defined as $\{P: P = A \vee P = B \vee A * P * B\}$.

\item[Definition:]  Let $\overline{AB}$ be a segment.   We say that the length of $\overline{AB}$ is $d(A,B)$.  We say that segments $\overline{AB}$ and $\overline{CD}$ are {\em congruent\/}, written $\overline{AB} \cong \overline{CD}$ iff they have the same length, that is, $d(A,B) = d(C,D)$.

This definition requires verification.  We need to establish that if $\overline{AB} = \overline{CD}$, we must have $d(A,B) = d(C,D)$.  This is proved as the following:

\item[Lemma:]  If $\overline{AB} = \overline{CD}$, we must have $d(A,B) = d(C,D)$.  In fact, we must have either $A=C \wedge B=D$ or $A=D \wedge B=C$.

\item[Proof of Lemma:]  Let $A,B,C,D$ be points with $\overline{AB} = \overline{CD}$.  Let $f$ be a coordinate function for $\Line{AB}$.  We may suppose without loss of generality that $f(A) < f(B)$ (because otherwise we could use $-f$ instead).

Since $C \in \overline{CD} = \overline{AB}$ we have either $C=A$ or $C=B$ or $f(A)<f(C)<f(B)$ (by the betweenness theorem, with the alternative $f(B) < f(C) < f(A)$ ruled out because $f(A) < f(B)$).

Suppose for the sake of a contradiction that $f(A)<f(C)<f(B)$.  We then observe further that since $A$ and $B$ belong to $\overline{AB} = \overline{CD}$, we
have either $f(C) < f(A) < f(B) \leq f(D)$ or $f(D) \leq f(A) < f(B) <f(C)$, by the betweenness theorem and the known order fact $f(A) < f(B)$.  But both of these are incompatible with $f(A)<f(C)<f(B)$, so this is false.

It follows that $C=A$ or $C=B$.  If $C=A$, then $D=B$, because $C$ and $D$ must be distinct (because there is a segment between them).  If $C=B$ then $D=A$, for the same reason.  So $d(A,B) = d(C,D)$ or $d(D,C)$, but $d(D,C)=d(C,D)$, so $d(A,B) = d(C,D)$ holds in either case.

\item[Definition:]  If $A,B$ are distinct points, we define $\overrightarrow{AB}$, the ray from $A$ through $B$, as $\{P:P = A \vee A*P*B \vee P=B \vee A*B*P\}$.  We note that another possible definition is $\Line{AB} - \{P:P*A*B\}$.

\end{description}


\end{document}