\documentclass[12pt]{article}

\title{Test I, Math 502, Spring 2021}

\begin{document}

\maketitle

This takehome exam will be posted sometime on Thursday March 11 and will be due on Tuesday March 16th at 1155 pm.

Please do not consult any other human being except myself about this test.

Please consult me promptly if you think there is a typo or mathematical error in the test.

Applications of the logical rules should follow the pattern of examples in the text.

\newpage

\begin{enumerate}

\item  Write a proof of $$((P \rightarrow Q) \wedge (Q \rightarrow R)) \rightarrow (P \rightarrow R),$$ complete with line numbers, stated goals, and documentation of what rules are applied to get each line.

\newpage

\item  Write a proof of $$\neg(P \rightarrow Q) \leftrightarrow P \wedge \neg Q, $$ using just our official logical rules.

\newpage

\item  Verify the rule of destructive dilemma, with the same level of formality (line numbers, stated goals, state rules used to get each line).

$$\begin{array}{c}

P \rightarrow Q \\

R \rightarrow S \\

\neg Q \vee \neg S \\ \hline

\neg P \vee \neg R$$


\end{array}$$

Again, you may use rules but not theorems.

Please give two different proofs, one by cases on the third premise, and one by applying alternative elimination to the goal.

\newpage

\item  Prove $$(\forall x:P[x]) \wedge (\forall x:P[x] \rightarrow Q[x]) \rightarrow (\forall x:Q[x])$$ (same level of formality:  be sure to be especially careful to indicate all uses of quantifier rules).

\newpage

\item Prove $$(\forall x:P[x]) \wedge (\exists x:P[x] \rightarrow Q[x]) \rightarrow (\exists x:Q[x])$$  (same instructions)

\newpage

\item  In lecture I showed how to do problem 10 on homework 3.  In this problem perform the converse task.  Assume that we define the universal quantifier
$(\forall x:P[x])$ as $\neg (\exists x:\neg P[x])$.  Using only rules of propositial logic, the rules EI and EG for the existential quantifier, and this definition of the universal quantifier,
show the validity of the rules of universal instantiation and universal generalization.

\newpage

\item  Prove $A-(B \cup C) = (A-B)-C$.  You may be somewhat informal about propositional logic steps (feel free for example to use de Morgan's laws) but you should be explicit about uses of definitions of set operations and the strategy for proving that two sets are equal.

Recall that the definition of $D-E$ is $\{x:x \in D \wedge x \not\in E\}$:  I am reminding you of this definition because I do not think we have used it as much.

\newpage

\item  Prove the theorem $${\cal P}(A) \subseteq {\cal P}(B) \rightarrow A \subseteq B$$  (same instructions as previous question)

We did the converse of this in class.  This proof is different but not really much harder.  Hint:  if $a \in A$ then $\{a\} \in {\cal P}(A)$.

I'd like to see something as close as possible to a line-by-line proof using proof strategies for sets, which are given in the book, but I'll give substantial credit for less formal arguments which get the point.

\newpage

\item  Give an assignment of types to the constants, variables, and sets in the equation

$$1 = \{x\mid  (\exists y:(\forall u:u \in x \leftrightarrow u=y))\}$$

and in the equation

$$(A+1) = \{(p \cup \{q\})\mid p \in A \wedge q \not\in p\}$$  (in this one, be sure to assign a type to each variable, to each set expression in braces, and also to the expressions I have put in parentheses).

You can do this by making tables or just by writing numbers (neatly) in the appropriate places.

\newpage

\item  List the elements of the set $$\{\{p,q\}:p \in A \wedge q \in B\}$$  where $A = \{a,b,c\}$ and $B = \{c,d,e\}$.  $a,b,c,d,e$ are five distinct individuals.  Remember that $\{x,y\} = \{y,x\}$ and $\{x,x\}=\{x\}$.  How many elements does this set have?


\newpage

\end{enumerate}


\end{document}