\documentclass[12pt]{article}

\usepackage{amssymb}

\title{Math 189 Spring 2022 Homework 3 Solutions}

\begin{document}

\maketitle

\begin{description}

\item[Levin 1:]  a: 64 subsets, 

b: 4 of them have $\{2,3,5\}$ as a subset, 

c: $64-8 = 56$ of them contain at least one odd number (there are 8 sets just of even numbers, take those out). 

d: To count ones with exactly one even number, choose an even number (3 choices) and a set of odd numbers (8 choices):  there are 24 of these.

\item[2:]  a: There are $\left(\begin{array}{c}6\\4 \end{array}\right)$ sets of size 4, that is, 15.  

b: There are three subsets of size 4 containing $\{2,3,5\}$ as a subset.

c: All subsets of size 4 (all 15 of them) contain at least one odd number.

d: Three subsets of size 4 contain exactly one even number (choose one of the three evens to put in with the three odds).
\item[3:]  (accidentally left this out at first)

a:  $2^9 = 512$

b:  9 choose 5 = 126

c:  16

d:  256


\item[4:] a: $2^{9-3} = 64$.  

b:  

b:after the 101 we will have a string of length 6, weight 3, and there are $\left(\begin{array}{c}6\\3 \end{array}\right)= 20$ of these

c: $2^6$ start with $101$, $2^7$ end with 11, $2^4$ have both features, so $64+128-16 = 176$

d: Weight 5 starting with 101, there are 20.  Weight 5 ending with 11, there are $\left(\begin{array}{c}7\\3 \end{array}\right)$ of these, that is 35, and there are 4 strings with both properties (put a single 1 in one of the four remaining positions).
so the answer is $20+35-4=51$,

\item[6:]  386.  Add up $\left(\begin{array}{c}10\\n \end{array}\right)$ where $n$ ranges from six to ten.

\item[8:]  3640 = $\left(\begin{array}{c}15\\12 \end{array}\right)\cdot 2^3$

\item[12:]  there are 11 choose 3 = 165 possible pizzas.

Of these, 10 choose 3 = 120 have pineapple

and 10 choose 2 = 45 do have pineapple

and 120 + 45 = 165.

\item[Lovasz 4.7:]

Prove $\left(\begin{array}{c}n\\2 \end{array}\right) +\left(\begin{array}{c}n+1\\2 \end{array}\right)=n^2$ algebraically, then as a counting argument.

Algebraically:

$\left(\begin{array}{c}n\\2 \end{array}\right) +\left(\begin{array}{c}n+1\\2 \end{array}\right)=$

$\frac{n!}{2!(n-2)!} + \frac{(n+1)!}{2!((n+1)-2)!} = $

$\frac{(n-1)(n)}{2} + \frac{(n)(n+1)}{2} =$

$\frac{n^2-n}2 + \frac{n^2+n}2 = \frac{2n^2}2 = n^2$

Combinatorially, $n^2$ is the number of ordered pairs $(a,b)$ where $a, b$ belong to $\{1,\ldots, n\}$.

There are $\left(\begin{array}{c}n\\2 \end{array}\right)$ pairs $(a,b)$ with $a > b$.

This leaves the pairs $(a,b)$ with $a \leq b$.  These correspond exactly to pairs $(a,b+1)$ which are all the pairs of distinct increasing numbers from $\{1,\ldots,n+1\}$ and there are $\left(\begin{array}{c}n+1\\2 \end{array}\right)$ of these.  I thought that was a bit tricky!

There might be other solutions, and if students come up with others Ill add them here.

\end{description}

\end{document}