\documentclass[12pt]{article}

\title{Measures defined and their properties explored, and the measure(s) we are interested in (Lebesgue measure) introduced}

\author{Randall Holmes}

\usepackage{amssymb}

\begin{document}

\maketitle

A bit belatedly, notes on 2C and 2D.

In many cases, Ill refer you for the proof of an important result to Axler.  If I did something different, or if I regard the proof as particularly important, I might do it here.
My discussion of the uncountable of course is off the top of my own head, and this will appear here.

We define a {\em measure\/} as a function $\mu$ from a $\sigma$-algebra $S$ on a set $X$ to the nonnegative reals which sends the empty set to zero and is countably additive:
if $\{A_i\}$ is a sequence of pairwise subsets of $X$ which belong to $S$, $$\mu(\bigcup_{i=1}^{\infty} A_k) = \sum_{i=1}^{\infty}\mu(A_k).$$

I will usually but not always write $|A|$ instead of $\mu(A)$, risking confusion with outer measure, because in fact both of the measures we are actually interested in coincide with outer measure.

Here is a question to think about.  Suppose we just assumed that a measure is countably additive.  Could we use this to prove that the measure of the empty set is zero?

Notice that countable additivity implies finite additivity (and the special case of binary additivity, $|A \cup B| = |A| + |B|$ if $A,B$ are disjoint).  The idea is that one can fill out a finite sequence of pairwise disjoint sets to a countable sequence by filling in the missing terms with $\emptyset$.  Yes, the empty set is disjoint from itself!

I remind you that in the theorems that follow we are talking about a general measure space on a $\sigma$-algebra $S$ on a set $X$, and I am writing $|A|$ for $\mu(A)$, where $\mu$ is a general measure, not necessarily outer measure.

A measure respects the inclusion order:  if $D \subseteq E$, then $|E| = |D| + |E \setminus D| \geq |D|$.  Binary additivity is used:  certainly $D$ and $E \setminus D$ are disjoint.

A related result about set differences:  Suppose $D\subseteq E$, $D,E$ are in $S$ and $D$ has finite measure.  Then from $|D|+|E \setminus D| = |E|$ we can deduce
$|E \setminus D| = |E|-|D|$.  $D$ has to have finite measure because $\infty - \infty$ is not defined.

A measure is countably subadditive: $$\mu(\bigcup_{i=1}^{\infty} A_k) \leq \sum_{i=1}^{\infty}\mu(A_k),$$ for any sequence $\{A_i\}$ whose range is included in $S$.  The proof is on page 43 of Axler.  The idea is that $\bigcup A_k$ is covered by the sequence of pairwise disjoint sets $B_k = A_k \setminus \bigcup_{i=1}^{k-1} A_i$ (where we interpret $\bigcup_{i=1}^{k-1} A_i$ as the empty set when $k=1$).

The measure of the union of an increasing sequence of sets is the limit of their measures.  If $\{A_i\}$ is a sequence of subsets of $X$ in $S$ with $i < j \rightarrow A_i \subseteq A_j$ (that is what we mean by an increasing sequence of sets) then $|\bigcup_{i=1}^\infty A_i| = \lim_{i \rightarrow \infty} |A_i|$.  The proof is on pp. 43-4 of Axler.  It makes a nice use of properties of set differences established above and the infinite version of telescoping sums.

There is a similar theorem for intersections of decreasing sequences of sets, but it has an exception:   If $\{A_i\}$ is a sequence of subsets of $X$ in $S$ with $i < j \rightarrow A_j \subseteq A_i$ (that is what we mean by an decreasing sequence of sets) then $|\bigcap_{i=1}^\infty A_i| = \lim_{i \rightarrow \infty} |A_i|$, as long as $|A_1|<\infty$.  I believe the last condition could be weakened to $|A_2|<\infty$, as I commented in class when looking at the proof.  In fact, I think it can be weakened to the assertion that some $A_i$ has finite measure.  This is proved using de Morgan's laws to transfer our attention to the increasing sequence of sets $\{A_1 \setminus A_k\}$.  The exception arises because we cannot subtract infinity from infinity. and it isn't just a weakness in the proof:  there are counterexamples.  See the full proof
on p. 44 of Axler.

Suppose $D$ and $E$ are elements of $S$ and their intersection has finite measure.  Then $|D \cup E| = |D| + |E| - |D \cap E|$.   This is proved on p. 45 and I think might make a nice test question.  Any of the theorems here are probably eligible to be test questions.  And that closes my review of 2C.

2D has a lot of stuff in it!

In 2D our goal is to show that outer measure is a measure on the $\sigma$-algebra of Borel sets, and actually on the larger $\sigma$-algebra of Lebesgue measurable sets, so when we write $|A|$ we really mean the outer measure on a subset $A$ of the reals, and we do not yet know that this is a measure on the Borel sets
(to show that is our goal).

I am going to do something differently than Axler.  He introduces a $\sigma$-algebra in an offhand, throwaway way in the course of one of his proof, which we discover later is the $\sigma$-algebra of Lesbesgue measurable sets.  I am just going to define it up front.

\begin{description}

\item[Definition:]  Let $A$ be a subset of the reals.  We say that $A$ is {\em Lebesgue measurable\/} iff for every $\epsilon>0$, there is a closed set $F \in A$ such that $|A \setminus F|<\epsilon$.

\end{description}

We do not know yet that this is a $\sigma$-algebra.  We have to prove it.

This isn't the definition of Lebesgue measurable set that Axler gives, but he uses this definition first, and he shows it later to be equivalent to his usual definition.

Notice that every open interval is Lebesgue measurable -- if $A = (a,b)$ let $F = [a+\frac{\epsilon}4,b-\frac{\epsilon}4]\cap \mathbb R$ (the intersection with the reals fixes the case where one or both of the endpoints is infinite).  Thus, if we can show that the Lebesgue measurable sets make up a $\sigma$-algebra, we show that all Borel sets are Lebesgue measurable.

Suppose that $G$ is an open set and $A$ is any set of reals disjoint from $G$ at all.  We show that $|A \cup G| = |A|+|G|$.  (proof in Axler, pp. 47-48).

Suppose that $F$ is a closed set and $A$ is any set of reals disjoint from $F$ at all.  We show that $|A \cup f| = |A|+|F|$.  (proof in Axler, p. 48).

Now we prove that the Lebesgue measurable sets (defined as above) make up a $\sigma$-algebra.  Axler proves exactly the same thing on pp. 49-50, but he doesn't call the set $\cal L$ the set of Lebesgue measurable sets yet.  For the moment, refer to pp. 49-50 for the proof.  Notice the important point that if you show that a collection of subsets of $X$ is closed under countable intersection and complement relative to $X$, you have also shown that it is closed under countable unions:  so you can replace the countable union component of the proof that something is a $\sigma$-algebra with a countable intersection component.  Axler does this without comment in the proof he gives.

Once this is proved, we know that all Borel sets are Lebesgue measurable, because we showed above that open intervals are.

Theorem 2.66. Axler, p. 50, can be rephrased for free as, Suppose $A$ and $B$ are disjoint subsets of $\mathbb R$ and $B$ is Lebesgue measurable [Axler merely says that $B$ is Borel] .  Then $|A \cup B| = |A| + |B|$.

The reason is that the only property of $B$ which is used in the proof (on page 50) is that it is Lebesgue measurable.

Further, Axler's theorem 2.68 can be strengthened to,  Outer measure is a measure on Lebesgue measurable sets.  The reason is the same as with 2.66:  he does not use any property of the sequence of sets $B_i$ in the proof (on p. 51) other than the fact that the $B_i$'s are Lebesgue measurable (if we strengthen 2.66 as above to be a result about all Lebesgue measurable sets).

I have no objection to definition 2.70 of Lebesgue measurable sets in itself, but the fact that he defined the collection of Lebesgue measurable sets already (without admitting it) makes me think the definition he gives first should be the definition.  What do you think?

In any case the equivalences in 2.71 (which we proved in class) [statement p. 52, proof on p. 53] prove that his definition (item d) and my definition (item b) are equivalent.  

It is interesting that item b
(the definition I am using) does not depend on the notion of Borel set at all.  It depends only on the much simpler definition of closed set and the definition of outer measure.    I am actually not sure that anything in Axler's proof that the Lebesgue measurable sets make up a $\sigma$-algebra, slightly recast as I give it, actually depends on any result about Borel sets, or even on the definition of Borel set.  This supports a polemic for the proper domain of Lebesgue measure being the collection of Lebesgue measurable sets!

Theorem 2.72 is redundant.  He has already proved it but didn't admit it, in effect.

There follows the discussion of the Cantor set, the Cantor function, and additional discussion of the existence of uncountable sets.

The Cantor set is definable in two ways.  One, which I leave it to Axler to explain, is as the set of all numbers in $[0,1]$ which have base 3 expansions which contain only 0's and 2's.

I give my definition.  I first give a recursive definition of a sequence of sets of closed intervals.  Define $X_0$ as $\{[0,1]\}$.  Define $X_{k+1}$ as $$\{[a,a+\frac23(b-a)],[b-\frac13(b-a),b]:[a,b]\in X_k\}.$$.  This exactly captures the idea of punching out middle thirds of intervals over and over, but we haven't defined a set of real numbers yet.

It is easy to see (you can prove this by induction if you are finicky) that each $X_k$ has $2^k$ members, and each element of $X_k$ is a closed interval of length $\frac1{3^k}$.

Define $C_k$ as the union of the intervals in $X_k$.  Since each $X_k$ is a finite set of closed intervals, each $C_k$ is a closed set.  Define the Cantor set as $\bigcap_{i=1}^\infty C_k$.

$C$ is a closed set because it is an intersection of Borel sets.  The measure of $C_k$ is $\left(\frac23\right)^k$, so by the result on measures of decreasing sequences of sets (or directly from the definition of outer measure with only slight tweaking) the measure of $C$ is 0.

We prove that $C$ is an uncountable set.  We have a theorem that countable sets have outer measure zero.  Some of you formed the conjecture that the converse is true.  It is not.

From Axler's description of the Cantor set (using base 3 expansions) we notice that each element $r$ of $[0,1]$ is associated with either one or two elements of the Cantor set, obtained by
replacing each 1 in the binary expansion of $r$ with a 2 then reinterpreting the result as a base 3 expansion.  Some reals have two binary expansions, but in any case this shows there are at least
as many elements of the Cantor set as there are reals in $[0,1]$. and by reputation at least you know that $[0,1]$ is not a countable set.

I took an excursion into a discussion of infinite cardinality.

We say that sets $A$ and $B$ are the same size ($A \sim B$) just in case there is a bijection from $A$ to $B$ (a one to one and onto map).  This is the common sense definition for finite sets.
The added power of this definition is that we now have a notion of size of infinite sets.  It is reasonably easy to show that this is an equivalence relation.

It has the immediate strange consequence that a set can be the same size as a proper subset of itself.  Galileo (of all people) noticed this, pointing out in a quite modern way that
the sequence 1,2,3,4...  is in one to one correspondence with 1,4,9,16... the sequence of perfect squares.  But this contradicts a principle going back to Euclid, that the whole is greater than the part.

Another consequence noticed in the Middle Ages is that ordinary geometry gives us the example of an obvious one to one correspondence between the sets of points on the circumferences of circles with different radii and the same center.  This created a conflict in people's heads.  Now we would say, the larger circle has larger measure (a longer circumference) but has the same size as a set of points:  these are simply different notions of size.

Cantor was the first to take infinite sets seriously, and the first to notice the perhaps surprising fact that there are sets of different sizes.  The set of natural numbers is the same size as the set of integers and the set of rationals, but it is strictly smaller than the set of reals.   There are even larger sets.

I talked about this only a little, enough to firm up our understanding of why the Cantor set is uncountable.

Cantor's original proof that $[0,1]$ (or any closed interval in the reals) is uncountable has an analytic flavor different from the argument you may have seen using diagonalization on decimal expansions
(though it is actually basically that very argument in base 3).

Suppose that $[0,1]$ was countable.  Then we would have a sequence $\{r_i\}$ whose range was $[0,1]$.  We define a sequence of closed intervals recursively.  Let $I_1=[0,1]$.

Once we have defined $I_i = [a,b]$, we define $I_{i-1}$ as the first of the three intervals $[a,a+\frac13(b-a)],[a+\frac13(b-a),b-\frac13(b-a)],[b-\frac13(b-a),b]$ which does not contain $r_i$ (two of the intervals might contain it if it is on a boundary point, but not all three).  Observe that the sequence of intervals is nested, and the length of $I_i$ is $\frac1{3^{i-1}}$, so the length converge to 0.
By the Nested Interval Theorem (which I proved in class but won't review here) the intersection $\bigcap_{i-1}^\infty I_i$ contains a single point $c$, which is distinct from every $r_i$ by the construction, which is a contradiction, so in fact there can be no such sequence $r_i$ and the set $[0,1]$ is uncountable.  We actually already proved this when we proved that the measure of $[0,1]$ is 1, but that is an extremely strange proof.  Professionally, I would rather like to study how that proof works...

I give a couple of other results which might be useful for thinking about sizes of sets.\

The set of all bit sequences (sequences of 0's and 1's) is uncountable.  Suppose to the contrary that we had a sequence $E$ of all bit sequences.  Define a sequence $D$ by
$D_i = 1 - (E_i)_i$.  To read this, remember that any $E_i$ is a sequence of 0's and 1's so $(E_i)_j$ is the $j$th term of the $i$th sequence, a 0 or a  1.  Now $E$ is supposed
to enumerate all the sequences, so $D = E_k$ must be true for some $k$.  But then $D_k = 1 - (E_k)_k=1-D_k$, which is impossible.  So there can be no such $E$, and the collection of bit sequences is uncountable.

There is a one to one correspondence between elements of the Cantor set and bit sequences.  The quick way to see this is to take any bit sequence $B$ to $\sum_{i=1}^infty \frac2{3^i}B_i$:  convert the bit sequence into a base three expansion in 0's and 2's and you get an element of the Cantor set.  This is a one to one map, because if a number has two base 3 expansions, one of them contains ones.  So, again, the Cantor set is uncountable.

If $A$ is a set, define ${\cal P}(A)$ as the collection of all subsets of $A$.  We argue that the set ${\cal P}(A)$ is strictly larger than the set $A$ (which means there are lots of sizes of infinite sets).
Strictly, we are only going to argue here that they are not the same size.  But notice that $A$ is the same size as the subset of  ${\cal P}(A)$ consisting of one element subsets:  the map taking $a \in A$ to
$\{a\} \in {\cal P}(A)$ is an injection, and a bijection from $A$ to the set of all one element subsets of $A$.

Suppose that $A$ and ${\cal P}(A)$ were the same size.  There would be a bijection $f:A \rightarrow {\cal P}(A)$.  Define $R$ as $\{a \in A:a \not\in f(a)\}$.  $R \subseteq A$ so $R=f(r)$ for some $r \in A$.  Now $r \in R \leftrightarrow r\in A \wedge f \not\in f(r)$.  $r \in A$ is true, so $r \in R \leftrightarrow r \not\in f(r)$.  But $f(r) = R$, so $r \in R \leftrightarrow r \not\in R$, which is a contradiction.

The real use of this result for us is to observe that the collection of subsets of the reals, or subsets of $[0,1]$ is even bigger than the reals or $[0,1]$ (the set of all reals and $[0,1]$ are actually the same size).

It is a fairly sophisticated result that the set of Borel sets is the same size as the set of all reals (each Borel set can be cleverly coded by a single real number).  It is dead easy that there are more Lebesgue measurable sets than there are numbers in $[0,1]$:  the Cantor set is at least as large as $[0,1]$, and every subset of the Cantor set is a Lebesgue measurable set (of measure 0).

Now we discuss the Cantor function.  Here is a simple way to describe it:  there is a natural function $f$ from the Cantor set to $[0,1]$, sending each number in $C$,  which has a ternary expansion consisting entirely of 0's and 2's to the real number with the binary expansion obtained by replacing all the 2's with 1's.  This function is obviously increasing on the Cantor set, and it maps onto $[0,1]$.  It can clearly be filled in to be an increasing (not strictly increasing) function:  define $f(x)$ for $x$ not in the Cantor set as the supremum of $f(y)$ for $y<x$ where $y \in C$.  Since the range
of $f$ is all of $[0,1]$ and $C$ is a closed set, it actually must be true that the largest $y \in C$ less than $x$ must have $f(y)=f(x)$ and similarly the smallest $z>x$ in $C$ must have $f(z)=f(x)$:  we see that the function is constant on all the open intervals in the complement of $C$.

This function has been defined in a way which makes it clear that it is increasing.  An increasing function from $[0,1]$ onto $[0,1]$ must be continuous.  This is a truly weird function:  it increases
continuously from $f(0) = 0$ to $f(1)=1$, but on a subset of measure 1 of $[0,1]$ it is locally constant:  it does all of its increasing on the measure zero Cantor set.

A weird counterexample closes these notes.  Using $\Lambda$ for the Cantor function as Aczel does, consider a set $E$ which is not Lebesgue measurable.  The set $\Lambda^{-1}[E] \cap C$ is mapped onto $E$ by the continuous function $\Lambda$.  But $\Lambda^{-1}[E] \cap C$ is Lebesgue measurable, since it is a subset of $C$ and so is measure zero.  The image of a Lebesgue measurable set under a continuous function can fail to be Lebesgue measurable.

I asked in class whether the image of a Borel set under a continuous function has to be Borel.  We know this is true for inverse images, but they are much better behaved.





\end{document}