\documentclass[12pt]{article}

\usepackage{amssymb}

\title{A snippet about the proof of the Archimedean property}

\author{Randall Holmes}

\begin{document}

\maketitle

This is a snippet about the fact that I proved the Archimedean property in a way complementary to the way it is proved in the book.

We prove this:  ``Suppose that $\mathbb F$ is a complete ordered field.  Then for every $t \in \mathbb F$, there is a positive integer $n$ such that $t<n$".

The proof in the book:  suppose otherwise.  Then there is a $t$ which is larger than every positive integer, and so is an upper bound for the set $\mathbb Z^+$.

Thus by the Completeness Axiom there is a least upper bound $M$ for $\mathbb Z^+$.

Since $M$ is the least upper bound for $\mathbb Z^+$, it follows that $M-1$ is not an upper bound for $\mathbb Z^+$, so there is a positive integer $N>M-1$.

But then $N+1$ is a positive integer, and $N+1>M$, which contradicts the claim that $M$ is the least upper bound of $\mathbb Z^+$, because it shows that $M$ is not
an upper bound for $\mathbb Z^+$ at all.

The alternative proof I came up with on my feet:  consider the set $S = \{t \in {\mathbb F}:(\forall n \in \mathbb Z^+:t>n\}$.  This could be described as the set of strict upper bounds
for $\mathbb Z^+$.

Suppose the statement ``for every $t \in \mathbb F$, there is a positive integer $n$ such that $t<n$" is false.  Then $S$ is nonempty.  $S$ is certainly bounded below
(by any positive integer you like, such as 1) so it has a greatest lower bound (a corollary of the completeness property).

Let $M$ be the greatest lower bound of $S$.  Then $M+1$ is not a lower bound of $S$, so there is $t \in S$ which is less than $M+1$.  This implies that $t-1<M \not\in S$, from which it follows that there is an integer $n$ such that $t-1<n$.  But then $n+1$ is an integer and $t<n+1$, contradicting the assertion that $t \in S$.

If you look at it sideways, it is basically the same argument.

I hint that this is seriously relevant to problem 14b.  Explain.



\end{document}