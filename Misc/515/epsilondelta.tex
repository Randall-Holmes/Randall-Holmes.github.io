\documentclass[12pt]{article}

\title{A nonce note on $\delta$'s in the discussion of the square root of 2 in supplement A}

\author{Randall Holmes}

\usepackage{amssymb}

\begin{document}

\maketitle

\maketitle

This is a snippet of notes which Ill place in the schedule.  I'll also create a master file of notes at the top of the resource page, when there are enough snippets.

The supplement discusses the result that the set of rational numbers with square less than 2, $\{x \in \mathbb Q:x^2<2\}$ has no least upper bound in $\mathbb Q$.  The proof given in the supplement, p. 9, is correct but has a rhetorical defect which I felt and some of you asked about.

Let $S=\{x \in \mathbb Q:x^2<2\}$.  Notice that $S$ does have upper bounds (2, for example) so if it does not have a least upper bound it is a counterexample to the Completeness Axiom
for the system $\mathbb Q$.

Its worth noting that some sets (the set of all even integers, for example, or $\mathbb Q$ itself) do not have least upper bounds in $\mathbb Q$, but they are not counterexamples to the Completeness Axiom, because they do not have any upper bounds at all.

It is also worth noting that $S$ is also a subset of the reals, and it {\em does\/} have a least upper bound in $\mathbb R$ (because it is bounded above, and so has one by the Completeness Axiom, which does hold in $\mathbb R$) and moreover this least upper bound is $\sqrt 2$.  This last fact is not as obvious as it seems, and we will come back to verifying it.

Suppose for the sake of a contradiction that there is a least upper bound $b$ of $S$.  Notice that $b<2$ follows, since 2 is an upper bound of $S$ and not the least upper bound
(since 1.5 is also an upper bound for $S$, for example).

The argument of Pythagoras (given in the notes) establishes that there is no rational square root of 2.

Thus there are two possibilities:  either $b^2<2$ or $b^2>2$.  We show that both of these lead to contradiction.

Suppose $b^2<2$.  We want to find a positive $\delta$ such that $(b+\delta)^2<2$, so $b+\delta \in S$, so $b$ is not an upper bound of $S$ at all, a contradiction.

$(b+\delta)^2 = b^2+2b\delta + \delta^2$.  If we can make $2\delta + \delta^2<2-b^2$ we will have $(b+\delta)^2 = b^2+2\delta + \delta^2 < b^2+(2-b^2) <2$ which is what we want.

Notice that if any $\delta$ works, any smaller $\delta$ will also work, so we can assume $0<\delta<1$ without any loss of generality.

 $2b\delta+\delta^2 = (2b+\delta)\delta<5\delta$ (because $b<2, \delta<1$) so if we have $5\delta=2-b^2$ we will have $2b\delta+\delta^2 = (2b+\delta)\delta<5\delta=2-b^2$ which is what we want.  So choose $\delta = \frac{2-b^2}5$.

You can read the proof in the book or reconstruct the full proof from what I have written above.  But in this presentation, the question of where the 5 comes from (or more generally, where
$\frac{2-b^2}5$ comes from, which he seems to have pulled out of the air) is answered.

The part above we did in class.  Now Ill do the second part.  Suppose $b^2>2$.  If we can find positive $\delta$ such that $(b-\delta)^2>2$ and $b-\delta$ is positive (so $0<\delta<b$) then $b-\delta$ is also an upper bound for $S$,
and $b$ is not the {\em least\/} upper bound for $S$, a contradiction.  $b-\delta$ has to be positive for us to be able to conclude that $b-\delta$ is an upper bound for $S$:  if $b-\delta$ were negative, then it would be a lower bound for $S$, not an upper bound (a negative number whose square is greater than 2 is less than all elements of $S$).  This is where the bound
$0<\delta<b$ which was puzzling us in class came from.  Notice that if a given $\delta$ works, so does a smaller one, and in fact we {\em could\/} assume $0<\delta<1$ because it is easy to show
$b>1$ -- but further, if you look at the argument below, you would still end up with the denominator $2b$.

Now we have $(b-\delta)^2 = b^2 -2b\delta+\delta^2 = b^2 - (2b-\delta)\delta$.  If we can make $(2b-\delta)\delta < b^2-2$ then we have $(b-\delta)^2 = b^2 -2b\delta+\delta^2 = b^2 - (2b-\delta)\delta>b^2-(b^2-2)=2$ as required.  

We know that $2b-\delta<2b$, so if we have $b^2-2=2b\delta$ we have $(2b-\delta)\delta < 2b\delta = b^2-2$, so choose $\delta = \frac{b^2-2}{2b}$, as he does, and we are done.
But we can see now where his $\delta$ came from.  Again, the full proof of the contradiction for this case can be reconstructed by unwinding what we did above, or we can read from the supplement, but without the impression that the value of $\delta$ is just pulled out of the air.

The logic and the arithmetic of the second case are actually rather different from the first.  Note that in the second case, we are not showing that $b$ is not an upper bound but that
$b$ is not the {\em smallest\/} upper bound.  Also notice that picking a different upper bound for $\delta$ (such as 1) would not change the fact that the upper bound for
$2b-\delta$ which we want is $2b$.  We {\em could\/} use the upper bound 2 for $b$ and say $2b-\delta<4$, and come up with $\delta = \frac{b^2-2}4$, but there seems to be no particular advantage to this.

A further remark is that we can recycle the proof above to show that the least upper bound of $\{x \in \mathbb Q:x^2<2\}$ in $\mathbb R$, which does exist by the completeness axiom, is in fact the positive square root of 2.  The exact algebra above can be used to show that its square cannot be less than 2 (it would then not be an upper bound of the set) nor greater than 2 (it would then not be the {\em least\/} upper bound of the set.  This argument appears in the supplement with the same backward reference to the proof about the rationals which we restate above.



\end{document}