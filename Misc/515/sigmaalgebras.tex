\documentclass[12pt]{article}

\title{Notes on $\sigma$-algebras}

\author{Randall Holmes}

\usepackage{amssymb}

\begin{document}

\maketitle

These notes cover some discussion of open and closed sets from the supplement and the development of $\sigma$-algebras and Borel sets.

Coverage is in section 2B, pp. 25-35, and supplement part D on open and closed sets.

\section{Open and closed sets in $\mathbb R$ and $\mathbb R^n$}

A subset $A$ of the real line is {\em open} iff for every $x \in A$, there is an open interval $(x -\delta,x+\delta)$ which is a subset of $A$.

The definition is modified for $\mathbb R^n$ but the basic idea is similar. 

 Define $B_\delta(x)$ as the set of points in $\mathbb R^n$ at distance less than $\delta$ from $x$.

The notion of distance used can be taken here to be the usual $d((x_1,\ldots,x_n),(y_1,\ldots,y_n)) = \sqrt{\sum_{i=1}^n (x_i-y_i)^2}$.  Other definitions of distance are discussed in the supplement:  they give the same notion of open set.  Notable alternatives are $\sum_{i=1}^n (x_i-y_i)$ and ${\tt max}_{1 \leq i \leq n}(x_i-y_i)$.  Use of either of these notions of distance does not affect which sets are open.

A subset $A$ of $\mathbb R^n$ is {\em open} iff for every $x \in A$, there is an open ball $B_\delta(x)$ which is a subset of $A$.

Theorems about open sets:

\begin {description}

\item[Theorem:]  Arbitary unions of open sets are open:  if $\cal O$ is a collection of open sets, $\bigcup O$ is open.

\item[Proof:]  If $x \in \bigcup \cal O$, then for some $O \in \cal O$, $x \in O$, and some open ball $B_\delta(x) \subseteq O \subseteq \bigcup\cal O$.  Notice that this proof applies to $\mathbb R$ as well, since $B_\delta(x)$ in the context of $\mathbb R$ is simply $(x - \delta,x+\delta)$.

\item[Theorem:]  Finite intersections of open sets are open:  if $\cal F$ is a finite collection of open sets, $\bigcap F$ is finite.

\item[Proof:]  Let $\cal F$ be $\{O_1,\ldots O_n\}$.  If $x \in \bigcap \cal F$ then for each apppropriate $i$ we can choose $B_{\delta_i}(x) \subseteq O_i$.  Then
$B_{{\tt min}_{1 \leq i \leq n}(\delta_i)}(x) \subseteq \bigcap \cal F$.

\item[Observation:]  Notice that the intersection of an infinite family of open sets cannot be expected to be open.  For example $\bigcap_{\epsilon>0}(-\epsilon,1+\epsilon)$ is the closed interval $[0,1]$.

\end{description}

We proved the following interesting Theorem, which applies only to $\mathbb R$:

\begin{description}

\item[Theorem:]  A subset of $\mathbb R$ is open if and only if it is the union of a countable pairwise disjoint family of open intervals (recall that our open intervals do include
things like $(-\infty,a)$ and $(a,\infty)$.)

\item[Proof:]  Let $\{r_i\}$ be a sequence listing all the rational numbers.  Let $O$ be an open set.  Define $I_i$ for each $i$ as the union of all open intervals included in $O$ which contain $r_i$.

We consider the family of all nonempty $I_i$'s.  This family is obviously countable.  This family is pairwise disjoint:  if $r_j \in I_i$ then there is an interval $I$ included in $O$ which contains both $r_i$ and $r_j$.  For any interval $J$ included in $O$ and containing $r_j$,
$I \cup J$ is a larger interval included in $O$ which includes $r_i$:  thus everything in $I_j$ belongs to $I_i$ and symmetrically everything in $I_i$ belongs to $I_j$:  if two $I_i$'s meet, they are equal.  For any $x \in O$, there is an interval including $x$ included in $O$, which contains some rational number $r_i$, from which it follows that $x \in I_i$, so the union of the $I_i$'s is $O$.

\end{description}

I had less to say about closed sets.  A closed set is defined as the complement of an open set.  Arbitrary intersections of closed sets are closed;  finite unions of closed sets are closed.
These theorems follow in effect from de Morgan's laws applied to theorems about open sets. 

 An important fact about closed sets is that if $C$ is a closed set and $\{x_i\}$ is a convergent sequence
with each $x_i \in C$, then $\lim_{i\rightarrow \infty}x_i \in C$.

A quick argument for this (we argue in the context of the reals, but the proof in $\mathbb R^n$ would be much the same):  suppose that $\{x_i\}$ is a sequence and $\lim_{i \rightarrow \infty}=L$ exists and does not belong to $C$.  $\mathbb R \setminus C$ is open.  Thus there is $\epsilon>0$ such that $(L - \epsilon,L + \epsilon) \subseteq \mathbb R \setminus C$.  Now there is an $N$ such that for all $i>N$, $|x_i-L|<\epsilon$, so $x_i \not\in C$.  So we have shown the contrapositive:  if a sequence $\{x_i\}$ converges to something not in $C$, then not all terms of the sequence are in $C$.

\section{There are no measures on subsets of $\mathbb R$ which have all the properties we want.}

A list of desirable properties for a measure on subsets of the reals (a notion of total length of a set).  We will write $|A|$ for the measure of the set $A$.

\begin{description}

\item[totality:]  Every subset of the reals has a measure (a nonnegative real or $\infty$)

\item[length of intervals:]  The lengths of intervals are what we expect.

\item[translation invariance:]   For any set $A$ and real $r$, the measure of $A$ is equal to the measure of $r+A$:  moving a set should not change its measure.

\item[countable additivity:]   For any infinite sequence $\{A_i\}$ of subsets of the reals whose range is a pairwise disjoint (if $A_i$ meets $A_j$, then $i=j$), $|\bigcup_{i=1}^{\infty} A_i| = \sum_{i=1}^\infty |A_i|$

\end{description}

It is a sad truth that no measure can have these properties.  

In essence we have already proved this.  Axler gives proofs of a list of properties of a measure (on p. 25)  which follow from the properties stated above and claims (which you would have to check)
that these are the only properties of outer measure used in the proof that outer measure is not countably additive (in fact not even finitely additive).

The property which we give up is the first one, totality.  We will use a notion of measure on the reals which in fact is precisely outer measure -- restricted to ``well-behaved" subsets of the power set of the reals.

\section{$\sigma$-algebras}
\begin{description}
\item[Definition:]let $X$ be a set (in practice this will always be $\mathbb R$ or $\mathbb R^n$):  a $\sigma$-algebra on $X$ is a collection $\cal S$ of subsets of $X$ with desirable closure properties:

\begin{enumerate}
\item $\emptyset \in \cal S$

\item If $A \in \cal S$ then $X \setminus A \in \cal S$ ($\cal S$ is closed under complements)

\item  If $\{A_i\}$ is a sequence with each $A_i \in \cal S$, then $\bigcup_{i=1}^\infty A_i \in \cal S$ ($\cal S$ is closed under countable unions).

\end{enumerate}

We further say that a measurable space is a pair $(X,\cal S)$ where $\cal S$ is a $\sigma$-algebra on $X$, and in this situation we refer to elements of $\cal  S$ as $\cal S$-measurable sets.

\end{description}

It follows immediately from these properties that a $\sigma$-algebra is closed under countable intersections (you should be able to write out the argument for this yourself).

There are really only two $\sigma$-algebras we are interested in in this subject.  Wt have to prove a theorem in order to define the first one.

\begin{description}

\item[Theorem:]  Let $X$ be a set and let $A$ be a subset of ${\cal P}(X)$.  Then there is a smallest $\sigma$-algebra $\cal S$ which includes $A$ as a subset:  to be exact, there is a $\sigma$-algebra $\cal S$ which includes $A$ and further has the property that for any $\sigma$-algebra $\cal T$ which includes $A$, $\cal S \subseteq \cal T$.

\item[Proof:]  We define $\cal S$ as the intersection of all $\sigma$-algebras which include $A$ as a subset, and show that $\cal S$ thus defined is a $\sigma$-algebra (it obviously includes $A$).
It is useful to remark that there is at least one $\sigma$-algebra which includes $A$, namely, the set ${\cal P}(X)$.

We verify that $\emptyset \in \cal S$:  each $\sigma$-algebra $\cal T$ which includes $A$ as a subset contains $\emptyset$ as an element because it is a $\sigma$-algebra, so $\emptyset$ is in the intersection of all such $\sigma$-algebras, which is $\cal S$.

Suppose that $B \in \cal S$:  we want to show that $X \setminus B \in \cal S$.  To do this, we need to show that $X \setminus B$ belongs to every $\sigma$-algebra which includes $A$.
Let $\cal T$ be a $\sigma$-algebra which includes $A$.  $B \in \cal T$ because $B \in \cal S$.  Thus $X \setminus B \in \cal T$ because $\cal T$ is a $\sigma$-algebra.  Since $X \setminus B$ is seen to belong to every such $\cal T$, $X \setminus B \in \cal S$, the intersection of all such $\cal T$'s.

Suppose that a sequence $\{B_i\}$ has all $B_i \in \cal S$.  Our aim is to show that $\bigcup_{i=1}^\infty B_i \in \cal S$.  Let $\cal T$ be a $\sigma$-algebra including $A$.
Each $B_i \in \cal T$, because $\cal S$ is the intersection of all such $\cal T$'s.  Thus $\bigcup_{i=1}^\infty B_i \in \cal T$ because $\cal T$ is a $\sigma$-algebra.  We have shown that 
$\bigcup_{i=1}^\infty B_i \in \cal T$ for every such $\cal T$, and so $\bigcup_{i=1}^\infty B_i \in \cal S$.

We have completed the verification that $\cal S$ is a $\sigma$-algebra.



\end{description}

Now we can introduce the first of the two $\sigma$-algebras that specifically interest us.

\begin{description}

\item[Definition:]  We define $\cal B$, the collection of Borel sets, as the smallest $\sigma$-algebra on $\cal R$ which contains all open intervals.

\item[Comment:]  This is not the same as Axler's definition.  He defines $\cal B$ as the smallest $\sigma$-algebra which contains all open sets.  But this is the same thing.
Any $\sigma$-algebra which contains all open intervals contains all open sets because any open set is a countable union of open intervals, and $\sigma$-algebras are closed under countable unions.  Obviously, any $\sigma$-algebra which contains all open sets contains all open intervals.  Thus $\cal B$ as I define it and $\cal B$ as Axler defines it are defined as the intersection of exactly the same collection of $\sigma$-algebras.

\end{description}

\section{More to come, not done with this set of notes yet.}


\end{document}