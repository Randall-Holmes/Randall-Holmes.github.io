\documentclass[12pt]{article}

\title{Notes on $\sigma$-algebras}

\author{Randall Holmes}

\usepackage{amssymb}

\begin{document}

\maketitle

These notes cover some discussion of open and closed sets from the supplement and the development of $\sigma$-algebras and Borel sets.

Coverage is in section 2B, pp. 25-35, and supplement part D on open and closed sets.

\section{Open and closed sets in $\mathbb R$ and $\mathbb R^n$}

A subset $A$ of the real line is {\em open} iff for every $x \in A$, there is an open interval $(x -\delta,x+\delta)$ which is a subset of $A$.

The definition is modified for $\mathbb R^n$ but the basic idea is similar. 

 Define $B_\delta(x)$ as the set of points in $\mathbb R^n$ at distance less than $\delta$ from $x$.

The notion of distance used can be taken here to be the usual $d((x_1,\ldots,x_n),(y_1,\ldots,y_n)) = \sqrt{\sum_{i=1}^n (x_i-y_i)^2}$.  Other definitions of distance are discussed in the supplement:  they give the same notion of open set.  Notable alternatives are $\sum_{i=1}^n (x_i-y_i)$ and ${\tt max}_{1 \leq i \leq n}(x_i-y_i)$.  Use of either of these notions of distance does not affect which sets are open.

A subset $A$ of $\mathbb R^n$ is {\em open} iff for every $x \in A$, there is an open ball $B_\delta(x)$ which is a subset of $A$.

Theorems about open sets:

\begin {description}

\item[Theorem:]  Arbitary unions of open sets are open:  if $\cal O$ is a collection of open sets, $\bigcup O$ is open.

\item[Proof:]  If $x \in \bigcup \cal O$, then for some $O \in \cal O$, $x \in O$, and some open ball $B_\delta(x) \subseteq O \subseteq \bigcup\cal O$.  Notice that this proof applies to $\mathbb R$ as well, since $B_\delta(x)$ in the context of $\mathbb R$ is simply $(x - \delta,x+\delta)$.

\item[Theorem:]  Finite intersections of open sets are open:  if $\cal F$ is a finite collection of open sets, $\bigcap F$ is finite.

\item[Proof:]  Let $\cal F$ be $\{O_1,\ldots O_n\}$.  If $x \in \bigcap \cal F$ then for each apppropriate $i$ we can choose $B_{\delta_i}(x) \subseteq O_i$.  Then
$B_{{\tt min}_{1 \leq i \leq n}(\delta_i)}(x) \subseteq \bigcap \cal F$.

\item[Observation:]  Notice that the intersection of an infinite family of open sets cannot be expected to be open.  For example $\bigcap_{\epsilon>0}(-\epsilon,1+\epsilon)$ is the closed interval $[0,1]$.

\end{description}

We proved the following interesting Theorem, which applies only to $\mathbb R$:

\begin{description}

\item[Theorem:]  A subset of $\mathbb R$ is open if and only if it is the union of a countable pairwise disjoint family of open intervals (recall that our open intervals do include
things like $(-\infty,a)$ and $(a,\infty)$.)

\item[Proof:]  Let $\{r_i\}$ be a sequence listing all the rational numbers.  Let $O$ be an open set.  Define $I_i$ for each $i$ as the union of all open intervals included in $O$ which contain $r_i$.

We consider the family of all nonempty $I_i$'s.  This family is obviously countable.  This family is pairwise disjoint:  if $r_j \in I_i$ then there is an interval $I$ included in $O$ which contains both $r_i$ and $r_j$.  For any interval $J$ included in $O$ and containing $r_j$,
$I \cup J$ is a larger interval included in $O$ which includes $r_i$:  thus everything in $I_j$ belongs to $I_i$ and symmetrically everything in $I_i$ belongs to $I_j$:  if two $I_i$'s meet, they are equal.  For any $x \in O$, there is an interval including $x$ included in $O$, which contains some rational number $r_i$, from which it follows that $x \in I_i$, so the union of the $I_i$'s is $O$.

\end{description}

I had less to say about closed sets.  A closed set is defined as the complement of an open set.  Arbitrary intersections of closed sets are closed;  finite unions of closed sets are closed.
These theorems follow in effect from de Morgan's laws applied to theorems about open sets. 

 An important fact about closed sets is that if $C$ is a closed set and $\{x_i\}$ is a convergent sequence
with each $x_i \in C$, then $\lim_{i\rightarrow \infty}x_i \in C$.

A quick argument for this (we argue in the context of the reals, but the proof in $\mathbb R^n$ would be much the same):  suppose that $\{x_i\}$ is a sequence and $\lim_{i \rightarrow \infty}=L$ exists and does not belong to $C$.  $\mathbb R \setminus C$ is open.  Thus there is $\epsilon>0$ such that $(L - \epsilon,L + \epsilon) \subseteq \mathbb R \setminus C$.  Now there is an $N$ such that for all $i>N$, $|x_i-L|<\epsilon$, so $x_i \not\in C$.  So we have shown the contrapositive:  if a sequence $\{x_i\}$ converges to something not in $C$, then not all terms of the sequence are in $C$.

\section{There are no measures on subsets of $\mathbb R$ which have all the properties we want.}

A list of desirable properties for a measure on subsets of the reals (a notion of total length of a set).  We will write $|A|$ for the measure of the set $A$.

\begin{description}

\item[totality:]  Every subset of the reals has a measure (a nonnegative real or $\infty$)

\item[length of intervals:]  The lengths of intervals are what we expect.

\item[translation invariance:]   For any set $A$ and real $r$, the measure of $A$ is equal to the measure of $r+A$:  moving a set should not change its measure.

\item[countable additivity:]   For any infinite sequence $\{A_i\}$ of subsets of the reals whose range is a pairwise disjoint (if $A_i$ meets $A_j$, then $i=j$), $|\bigcup_{i=1}^{\infty} A_i| = \sum_{i=1}^\infty |A_i|$

\end{description}

It is a sad truth that no measure can have these properties.  

In essence we have already proved this.  Axler gives proofs of a list of properties of a measure (on p. 25)  which follow from the properties stated above and claims (which you would have to check)
that these are the only properties of outer measure used in the proof that outer measure is not countably additive (in fact not even finitely additive).

The property which we give up is the first one, totality.  We will use a notion of measure on the reals which in fact is precisely outer measure -- restricted to ``well-behaved" subsets of the power set of the reals.

\section{$\sigma$-algebras}
\begin{description}
\item[Definition:]let $X$ be a set (in practice this will always be $\mathbb R$ or $\mathbb R^n$):  a $\sigma$-algebra on $X$ is a collection $\cal S$ of subsets of $X$ with desirable closure properties:

\begin{enumerate}
\item $\emptyset \in \cal S$

\item If $A \in \cal S$ then $X \setminus A \in \cal S$ ($\cal S$ is closed under complements)

\item  If $\{A_i\}$ is a sequence with each $A_i \in \cal S$, then $\bigcup_{i=1}^\infty A_i \in \cal S$ ($\cal S$ is closed under countable unions).

\end{enumerate}

We further say that a measurable space is a pair $(X,\cal S)$ where $\cal S$ is a $\sigma$-algebra on $X$, and in this situation we refer to elements of $\cal  S$ as $\cal S$-measurable sets.

\end{description}

It follows immediately from these properties that a $\sigma$-algebra is closed under countable intersections (you should be able to write out the argument for this yourself).

There are really only two $\sigma$-algebras we are interested in in this subject.  Wt have to prove a theorem in order to define the first one.

\begin{description}

\item[Theorem:]  Let $X$ be a set and let $A$ be a subset of ${\cal P}(X)$.  Then there is a smallest $\sigma$-algebra $\cal S$ which includes $A$ as a subset:  to be exact, there is a $\sigma$-algebra $\cal S$ which includes $A$ and further has the property that for any $\sigma$-algebra $\cal T$ which includes $A$, $\cal S \subseteq \cal T$.

\item[Proof:]  We define $\cal S$ as the intersection of all $\sigma$-algebras which include $A$ as a subset, and show that $\cal S$ thus defined is a $\sigma$-algebra (it obviously includes $A$).
It is useful to remark that there is at least one $\sigma$-algebra which includes $A$, namely, the set ${\cal P}(X)$.

We verify that $\emptyset \in \cal S$:  each $\sigma$-algebra $\cal T$ which includes $A$ as a subset contains $\emptyset$ as an element because it is a $\sigma$-algebra, so $\emptyset$ is in the intersection of all such $\sigma$-algebras, which is $\cal S$.

Suppose that $B \in \cal S$:  we want to show that $X \setminus B \in \cal S$.  To do this, we need to show that $X \setminus B$ belongs to every $\sigma$-algebra which includes $A$.
Let $\cal T$ be a $\sigma$-algebra which includes $A$.  $B \in \cal T$ because $B \in \cal S$.  Thus $X \setminus B \in \cal T$ because $\cal T$ is a $\sigma$-algebra.  Since $X \setminus B$ is seen to belong to every such $\cal T$, $X \setminus B \in \cal S$, the intersection of all such $\cal T$'s.

Suppose that a sequence $\{B_i\}$ has all $B_i \in \cal S$.  Our aim is to show that $\bigcup_{i=1}^\infty B_i \in \cal S$.  Let $\cal T$ be a $\sigma$-algebra including $A$.
Each $B_i \in \cal T$, because $\cal S$ is the intersection of all such $\cal T$'s.  Thus $\bigcup_{i=1}^\infty B_i \in \cal T$ because $\cal T$ is a $\sigma$-algebra.  We have shown that 
$\bigcup_{i=1}^\infty B_i \in \cal T$ for every such $\cal T$, and so $\bigcup_{i=1}^\infty B_i \in \cal S$.

We have completed the verification that $\cal S$ is a $\sigma$-algebra.



\end{description}

Now we can introduce the first of the two $\sigma$-algebras that specifically interest us.

\begin{description}

\item[Definition:]  We define $\cal B$, the collection of Borel sets, as the smallest $\sigma$-algebra on $\cal R$ which contains all open intervals.

\item[Comment:]  This is not the same as Axler's definition.  He defines $\cal B$ as the smallest $\sigma$-algebra which contains all open sets.  But this is the same thing.
Any $\sigma$-algebra which contains all open intervals contains all open sets because any open set is a countable union of open intervals, and $\sigma$-algebras are closed under countable unions.  Obviously, any $\sigma$-algebra which contains all open sets contains all open intervals.  Thus $\cal B$ as I define it and $\cal B$ as Axler defines it are defined as the intersection of exactly the same collection of $\sigma$-algebras.

\end{description}

\section{More to come, not done with this set of notes yet.}

I don't find that I want to say much in my own words about the rest of section 2B.  My lecture hewed very close to Axler's actual language and his exact proofs.  I do have a couple of comments, and Ill summarize which proofs and definitions I lectured, but you should read them in Axler.

Note with regard to inverse images that I will try always to use the notation $f^{-1}[A] = \{x \in A:f(x) \in A\}$ rather than $f^{-1}(A)$:  the latter notation, while common, is ambiguous with the standard notation for application of a function.

Similarly, I will use the notation $f[A]$ instead of $f(A)$ for the image of a set $A$, $f(A) = \{y:(\exists x\in A:f(x)=y\}$.

It is important to be aware that while the inverse image has extremely nice algebraic properties, the image does not.

Let $f$ be a function from $X$ to $Y$.  Let $A$ be a subset of $Y$.  Let $\{A_i\}$ be a sequence of subsets of $Y$.

$f^{-1}[Y \setminus A] = X \setminus f^{-1}[A]$.

$f^{-1}[\bigcup_{i=1}^\infty A_i] = \bigcup_{i=1}^\infty f^{-1}[A_i]$

Because this works for complements and countable unions, it also works for countable intersections.

For the image, the situation is different.  The problem is that it is not true that for $B \subseteq X$, $f[X \setminus B] = Y \setminus f[B]$: in general, the problem is that something in $X \setminus B$ can perfectly well map to something in $f[B]$.  An easy counterexample is $X = \mathbb R; Y = \mathbb R; f(x) = x^2$.  It is very far from true that
$f[\mathbb R \setminus (0,\infty)] = \mathbb R \setminus f[\mathbb R \setminus (0,\infty)]$:  $f[(0,\infty)] = (0,\infty)$ and $f[\mathbb R \setminus (0,\infty)]=f[(-\infty,0]] = [0,\infty)$, which is very far from being the complement of the other image:  these sets have a very large intersection!

$f[\bigcup_{i=1}^\infty B_i] = \bigcup_{i=1}^\infty f[B_i]$ is true for a sequence of subsets $B_i$ of $X$, but the corresponding statement for intersections is not true:  the proof that works for inverse images breaks because we do not have the result for complements in the case of images, and in fact counterexamples are easy to construct.  Take any function with subsets $B$ and $C$ of its
domain with $b \in B, c \in C, f(b)=f(c)$.  Then $f[B \cap C]$ is empty but $f[B] \cap f[C]$ is not.

I lectured Axler's proof of 2.39 on Wednesday, but on Monday I at least hinted at a different approach.  I want to talk about that approach briefly, but Axler's proof is fine, and much  more accessible.  As I will repeat at the end of this discussion, this approach to 2.39 is not examinable.

Suppose $(X,S)$ is a measurable space and $f$ has the property that the inverse image of $(a,\infty)$ is in $S$ for every real number $a$.

Show that $f$ is measurable, i.e., that the inverse image of any Borel set $B$ is measurable.

First observe that the smallest $\sigma$-algebra containing every $(a,\infty)$ is the algebra of Borel sets.  To show this, it is enough to show that every
open interval can be built from intervals $(a,\infty)$ using complement and countable unions and intersections [though this proof of 2.39 is not examinable, the proof that  the smallest $\sigma$-algebra containing every $(a,\infty)$ is the algebra of Borel sets IS).

$(a,b) = (-\infty,b) \cap (a,\infty) = (\mathbb R \setminus [b,\infty)) \cap (a,\infty) = (\mathbb R \setminus \bigcap_{i=1}^\infty (b-\frac1i,\infty))\cap (a,\infty)$ verifies this.

Thus an arbitary Borel set $B$ can be construed as the value of an infinite expression in complements, countable unions and intersections, and intervals $(a,\infty)$.

We have to say exactly what we mean by this.  We simultaneously define infinite expressions of a complicated sort and a valuation function $v_G$ (depending on the choice of a function $G$ described below) sending them to
Borel sets.

\begin{description}

\item[variables:]  Think of a pair $(0,a)$, where $a$ is a real number, as a variable $x_a$, and $v_G((0,a))$ is defined as $G(a)$, where $G$ is a function given for assigning values to the variables.

\item[complements:]  If $E$ is an expression, think of $(1,E)$ as representing the complement of $E$, and define $v_G((1,E)) = \mathbb R \setminus v_G(E)$.

\item[countable unions:]  If $\{E_i\}$ is a sequence of expressions, think of $(2,E)$ as representing the countable union of the things represented by the expressions $E_i$
and define $v_G((2,E)) = \bigcup_{i=1}^\infty v_G(E_i)$.

\item[countable intersections:]  If $\{E_i\}$ is a sequence of expressions, think of $(3,E)$ as representing the countable union of the things represented by the expressions $E_i$
and define $v_G((3,E)) = \bigcap_{i=1}^\infty v_G(E_i)$.

\item[definition of the set of expressions:]  Define the set $\mathbb E$ of expressions as the intersection of all sets $C$ such that for each $r \in \mathbb R$,
$(0,r) \in C$, for each $E \in C$, $(1,E) \in C$, and for each sequence $\{E_i\}$ whose range is included in $C$, $(2,\{E_i\})$ and $(3,\{E_i\})$ belong to $C$.  A modest amount of technical set theory is needed to ensure that there are such sets $C$ and so the definition of $\mathbb E$ succeeds;  this is structured so that proofs by induction and definitions of functions
(such as $v_G$) by recursion on the structure of expressions will succeed.


\end{description}

With all this machinery, the outlines of a proof of 2.39 are reasonably visible.  Let the function $G_1$ be defined by $G_1(a) = (a,\infty)$.  The Borel set $B$ is
$v_{G_1}(E)$ for some infinite expression $E$, because it can be built from intervals in the range of $G_1$ by relative complements and countable unions.  Then define $G_2(a) = f^{-1}[G_1(a)]$:  this will be an element of $S$ in every case by hypothesis.  By the nice algebraic properties of inverse images, $f^{-1}[B] = f^{-1}[v_{G_1}[E]]$ for the infinite expression $E$ chosen above $= v_{G_2}[E]$, the result of replacing every variable representing an $(a,\infty)$ in $E$ with $f^{-1}[(a,\infty)]$, because inverse image commutes with all the constructions which build our expressions, and  $v_{G_2}[E]\in S$ is clear because the value of such an expression is computed from sets $f^{-1}[(a,\infty)]\in S$ by complements, countable unions and intersections and $S$ is a $\sigma$-algebra.  This is not examinable:  I'm really including it to indicate clearly why I backtracked and did Axler's proof.  The intuition of this argument is reasonably clear but the machinery to make it mathematically precise is massive (and second nature to me because I am a logician, which tempted me to think about it that way).

That said, what I lectured after the definition of Borel set was 2.31, 2.33 (I would do all these proofs yourself and make sure you understand them), 2.34, 2.35, 2.39, 2.40, 2.41, 2.43, 2.44, 2.46, which I found really cool,  and 2.48.

\end{document}