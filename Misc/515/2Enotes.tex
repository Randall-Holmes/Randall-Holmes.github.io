\documentclass[12pt]{article}

\usepackage{amssymb}

\title{Notes on section 2E in Axler}

\author{Randall Holmes}

\begin{document}

\maketitle

These are my notes on section 2E in Axler.  Since I had trouble with these myself, I am going to write out proofs of the main results in my own words.

Much of this section involves the interplay between two notions of convergence of a sequence of functions.

\begin{description}

\item[Definition (pointwise convergence):]  Let $X$ be a set, let $\{f_k\}$ be a sequence of functions from $X$ to $\mathbb R$.  We say that $\{f_k\}$ converges pointwise to
$f$ just in case ${\tt lim}_{k \rightarrow \infty}f_k(x) = f(x)$ for each $x \in X$.

We expand this definition.  For every $x \in X$, for every $\epsilon >0$, there is an $n \in \mathbb N$ such that for $k \geq n$, $|f_k(x) - f(x)|<\epsilon$.  Notice
that the point $n$ in the sequence where things fall within $\epsilon$ depends on $x$ as well as $\epsilon$.

\item[Definition (uniform convergence):]  Let $X$ be a set, let $\{f_k\}$ be a sequence of functions from $X$ to $\mathbb R$.  We say that $\{f_k\}$ converges uniformly to $f$ just in case for every $\epsilon>0$, there is $n \in \mathbb N$ such that for all $k \geq n$ and $x \in X$ we have $|f_k(x)-f(x)| < \epsilon$.

Clearly uniform convergence implies pointwise convergence.  It makes the stronger statement that in effect the $f_k$'s converge at each point in $X$ at the same rate (speaking a bit loosely:  we are not talking about derivatives here!):  the choice of $n$ to get within $\epsilon$ depends only on $\epsilon$, not on $x$.

\end{description}

You might want to take a look at the example of a sequence of functions converging pointwise but not uniformly on p. 62 of Axler.  One has a sequence of continuous functions with a steadily narrowing spike at a fixed $x$ value converging pointwise to a function with a jump discontuity at the fixed $x$ value.

\begin{description}

\item[Theorem:]  If a sequence $\{f_k\}$ of continuous functions from a subset $B$ of the reals to the reals converges uniformly to $f$, then $f$ is continous.

\item[Proof:]  We want to show that for each $x \in B$ and $\epsilon>0$, there is $\delta>0$ such that for any $y \in B$,  if $|x-y|<\delta$ then $f(x)-f(y)<\epsilon$.

Choose an $n$ such that for all $k \geq n$ and all $y\in B$, $|f_k(y) - f(y)|<\frac\epsilon3$ (by uniform convergence).

Choose a $\delta$ such that for all $y\in B$, if $|x-y|<\delta$ then $|f_n(x)-f_n(y)|<\frac\epsilon3$.

It then follows that for any $y \in B$, if $|x-y|<\delta$, we have $f(y)$ within $ \frac\epsilon3$ of $f_n(y)$, $f_n(y)$ within $ \frac\epsilon3$ of $f_n(x)$, and 
$f_n(x)$ within $ \frac\epsilon3$ of $f(x)$, from which it follows by the triangle inequality that $f(y)$ is within $\epsilon$ of $f(x)$, so we have shown that $f$ is continuous.

\end{description}

We now get the first exciting theorem.  Egoroff's theorem asserts that on any set $X$ supporting a measure which gives all of $X$ a finite measure,
if we have a pointwise convergent sequence of measurable functions $\{f_k\}$ converging to a function $f$, and choose an $\epsilon>0$, we can find an $E \subseteq X$ such that $X \setminus E$ has measure less than $\epsilon$ and $\{f_k\}$ converges uniformly to $f$ on $E$.


\end{document}