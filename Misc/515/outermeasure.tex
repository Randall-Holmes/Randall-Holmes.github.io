\documentclass[12pt]{article}

\title{Outer measure and its properties}

\usepackage{amssymb}

\author{Randall Holmes}

\begin{document}

\maketitle

\begin{description}

\item[introductory remarks; length of open intervals:]  Outer measure is a generalization of the notion of total length to sets of the reals more complicated than finite or countably infinite unions of intervals.  Any set of the form $(a,b) = \{x\in \mathbb R:a \leq x \leq b\}$ (where $a \leq b$)  is called an open interval:  we do not exclude
$a=b$ or $a$ or $b$ being $\pm \infty$.  The length $L((a,b))$ is defined as $b-a$:  note that for exotic intervals this may be 0 or $\infty$.

\item[total lengths of unions of intervals:]  A finite or countably infinite union of intervals $\{I_i\}$ which is pairwise disjoint has a natural length equal to $\sum_{i=1}^{\infty} L(I_i)$ (we can always assume the sequence of intervals $\{I_i\}$ is infinite by
defining $I_i= \emptyset$ for all unused indices).  A general union of intervals without the disjointness requirement we at least think of as having total length less than or equal to the sums of the lengths of the individual intervals (it is likely to have less total length because the intervals in the union may overlap).

\item[a remark about overlaps:]  It is worth noticing that for any general countable union of intervals, we cannot necessarily totally eliminate overlap between the intervals, but we can  modify the intervals in the union  in such a way as to make the total overlap less than any $\epsilon>0$ (without changing what set the union of the intervals is).  I may add a sketch of an argument for this statement here.
The brief idea is, take each interval $I_i$ and consider the result of deleting all the $I_j$'s for $j<i$ from it.  This will leave a finite set of closed and half-open intervals.  Fatten these intervals
to open intervals, adding a total length of less than $\frac\epsilon{2^i}$, and of course not going outside of $I_i$.  Replace $I_i$ with the finite set of open intervals thus obtained
and proceed to the next interval (of course using the new intervals, not the original $I_i$, in subsequent stages:  the fact that intervals are being replaced with finite sets of intervals
makes this tricky to write.  Proceeding through the entire interval, you get a sequence of intervals with the same union whose overlaps have total length $<\epsilon$.   This means
that our technique for estimating lengths of countable unions of intervals is actually precise.

\item[the definition of outer measure:]  Now we can define, for any set $A$, the {\em outer measure\/} of $A$, written $|A|$, as ${\tt inf}\{\sum_{i=1}^{\infty} L(I_i): \bigcup_{i=1}^{\infty} I_i \supseteq A\}$.  This uses
the intuitive idea of measure of a countable union of intervals outlined in the previous paragraph:  the idea is that surely the ``total length" of a set is less than the total length of a countable union of intervals which covers it, which is in turn less than the sum of the lengths of the intervals in the union:  we define the ``total length" of a general set, for which we adopt the technical term {\em outer measure\/}, as the greatest lower bound of these intuitive upper bounds on its total length.

\item[order preserving property of outer measure:]  If $A \subseteq B$ are subsets of the reals, then $|A| \leq |B|$.  This is direct:  if a sequence $\{I_i\}$ covers $B$ (this is a quick way of saying that $\bigcup_{i=1}^{\infty} I_i \supseteq B$) then it also covers $A$. by transitivity of the subset relation, and so $\sum_{i=1}^{\infty} L(I_i)$ is greater than or equal to $|A|$ because it belongs to the set of which $|A|$ is the greatest lower bound.

\item [Finite or countable sets of reals have measure zero:]   Let $X$ be a finite or infinite sequence of reals whose range is the given set $A$.  Let $$I_i = (X_i-\frac1{2^{i+1}},X_i-\frac1{2^{i+1}}),$$
or $\emptyset$ if $i$ is not in the range of $X$ (so we turn finite sequences of points into infinite sequences of intervals),  Now the measure of $A$ is less than or equal to the sum of the lengths of the intervals $I_i$, which is less than or equal to $\epsilon$, and $\epsilon$ can be chosen to be any positive real number, so $|A|=0$.

Notice that this has the rather shocking corollary that the measure of the set of all rational numbers is zero.

\item[Translation invariance:]  For any set $A \subseteq \mathbb R$ and $r \in \mathbb R$, define $r + A$ as $\{r+a:a \in A\}$.

Notice that $L(I) = L(r+I)$ is immediate, since $r+(a,b) = (r+a,r+b)$.

Suppose that a sequence $\{I_i\}$ of open intervals covers $A$.  We claim that the sequence $\{r+I_i\}$ covers $r+A$.  We show this:  let $y \in r+A$.  It follows that
$y = r+x$ for some $x \in A$.  $x \in I_i$ for some $i$, so $y \in r+I_i$.  Thus every point in $r+A$ is in one of the intervals $r+I_i$.

This means that $|r+A| \leq \sum_{i=1}^{\infty} L(r+I_i)$ for any $\{I_i\}$ with $\bigcup_{i=1}^{\infty} I_i \supseteq A$, so $|r+A| \leq \sum_{i=1}^{\infty}L(I_i)$ for any $\{I_i\}$ with $\bigcup_{i=1}^{\infty} I_i \supseteq A$ , so $|r+A| \leq {\tt inf}\{\sum_{i=1}^{\infty} L(I_i): \bigcup_{i=1}^{\infty} I_i \supseteq A\} = |A|$

This gives $|r+A| \leq |A|$.  Applying the same result, replacing $A$ with $r+A$ and $r$ with $-r$, we have $|(-r)+(r+A)|\leq |r+A|$, that is $|A| \leq |r+A|$, so in fact
$|A|=|r+A|$

This appears to be a common sense property of a notion of total length:  the length of a set should not change if you move it uniformly on the line.

\item[Countable subadditivity:]  We prove that for any sequence $\{A_i\}$ of subsets of the reals, $|\bigcup_{i=1}^\infty A_i| \leq \sum_{i=1}^\infty |A_i|.$

This is a common sense property:  the total length of a union of sets should be less than or equal to the sum of the total lengths of the sets (it can be expected to be strictly
less than this sum if some of the sets overlap).

Choose $\epsilon>0$.

Choose a double indexed sequence $\{I_{i,j}\}$ so that $A_i \subseteq \bigcup_{j=1}^\infty I_{i,j}$  and $\sum_{j=1}^\infty L(I_{i,j})<|A_i|+\frac\epsilon{2^i}$ for each $i$.  We can do this because of the definition of $|A_i|$ as a greatest lower bound.

Reorganize this sequence using a bijection $p$ from $\mathbb Z^+ \times \mathbb Z^+$ to $Z^+$,  into the sequence $J$ for which $J_{p(i,j)} = I_{i,j}$. 

Now observe that $\sum_{k=1}^\infty L(J_k) \leq \sum_{i=1}^\infty |A_i|+\epsilon$.  This means, since $J$ covers $\bigcup_{i=1}^\infty A_i$, that
$|\bigcup_{i=1}^\infty |A_i|| \leq \sum_{i=1}^\infty |A_i|+\epsilon$.  Since $\epsilon$ can be as small as you like, $|\bigcup_{i=1}^\infty |A_i|| \leq \sum_{i=1}^\infty |A_i|$.

\item[Closed intervals have the expected measure:]  We prove that $|[a,b]|=b-a$.

It is easy to see that $|[a,b]| \leq b-a$, because we can cover it by the single open interval $(a-\frac\epsilon2,b+\frac\epsilon2)$ for any $\epsilon>0$, so
$|[a,b]|\leq b-a+\epsilon$ for any $\epsilon>0$, so $|[a,b]|\leq b-a$.

The proof uses a very general theorem about open covers of a closed interval (or a closed bounded set generally) which we do want to know about, but for which we need only a special case here.

\begin{description}

\item[Lemma:]  For any sequence $\{I_i\}$ covering $[a,b]$, there is a finite subset $\{I_1,\ldots,I_n\}$ which covers $[a,b]$.

\item[Proof:]  Let $a \leq b$ and let $\{I_i\}$ be a sequence of open intervals covering $[a,b]$.

Define $S$ as the set of all $c \in [a,b]$ such that there is an $n$ such that $\bigcup_{i=1}^n I_i \supseteq [a,c]$.

$S$ is nonempty because it contains $a$.  Let $I_n$ be chosen to include $a$, and we have $\bigcup_{i=1}^n I_i \supseteq [a,a]$.

$S$ is bounded above by $b$.

Thus, by the completeness property of the reals, $S$ has a least upper bound.  Let's call it $c$.

Notice that $c>a$, because there are elements $d$ of $[a,b]$ greater than $A$ which belong to the interval $I_n$ including $a$ as an element, and so $\bigcup_{i=1}^n I_i \supseteq [a,d]$
so $a <d \leq c$

We have two possibilities:  $a<c<b$ and $c=b$.  The latter will be seen to be true:  we assume $a<c<b$ and now reason to a contradiction.

Suppose $a<c<b$.  Choose an interval $I_n$ including $c$, and choose an interval $(c-\epsilon, c+\epsilon)$ included in $I_n$,

Notice that $c-\frac\epsilon2 \in S$, so there is $m$ such that $\bigcup_{i=1}^m I_i \supseteq [a,c-\frac\epsilon2]$.  Adding the single interval $I_n$ (and intervals leading up to it if necessary) we can cover more.  In particular, $\bigcup_{i=1}^{{\tt max}(m,n)} I_i \supseteq [a,c+\frac\epsilon2]$ -- and this is impossible, because it shows that $c+\frac\epsilon2$ belongs to $S$, which is incompatible with $c$ being the least upper bound of $S$.

Thus $c=b$.  Choose any interval $I_n$ which includes $b$, and $\epsilon$ such that $(b-\epsilon,b+\epsilon) \subseteq I_n$.  $b-\frac\epsilon2 \in S$ so there is $m$ such that $\bigcup_{i=1}^m I_i \supseteq [a,b-\frac\epsilon2]$.   Then  $\bigcup_{i=1}^{{\tt max}(m,n)} I_i \supseteq [a,b]$, completing the proof.

\item[Lemma:]  For any sequence $\{I_i\}$, for all $n$, if $\bigcup_{i=1}^n I_i \supseteq [a,b]$ then $\sum_{i=1}^n L(I_i)$ (and so $\sum_{i=1}^\infty L(I_i)$) is greater than $b-a$.

\item[Proof:]  We prove this by induction on $n$.  If $n=1$, then $I_1 = (c,d) \supseteq [a,b]$, so $d-c >b-a$.  Basis case complete.

Suppose the result is true for $n=k$ and we have $\bigcup_{i=1}^{k+1} I_i \supseteq [a,b]$.  We can suppose without loss of generality that $I_{k+1}$ has $b$
as an element:  if it doesn't, we can exchange $I_{k+1}$ with a previous $I_i$ which does contain $b$.  Let $I_{k+1} = (c,d)$.  We know that $\bigcup_{i=1}^{k} I_i \supseteq [a,c]$
so by ind hyp $\sum_{i=1}^k L(I_i) >c-a$, so  $\sum_{i=1}^{k+1 }L(I_i) >(c-a)+(d-c)=d-a>b-a$.

\item[closing of proof:]  Now for any $\{I_i\}$ such that $\bigcup _{i=1}^\infty I_i \supseteq [a,b]$ we have $n$ such that $\bigcup _{i=1}^n I_i \supseteq [a,b]$
and $\sum_{i=1}^\infty L(I_i) \geq \sum_{i=1}^n L(I_i)>b-a$, so that $|[a,b]| \geq b-a$ by the definition of $|[a,b]|$ as a greatest lower bound.  We have seen above that
$|[a,b] \leq b-a$, so in fact $|[a,b]|=b-a$.


\end{description}

\item[comments:]  This is the last of the nice results about outer measure.  It was harder to prove than one might expect.

A corollary is that if $b-a>0$, the closed interval $[a,b]$ is an uncountable set.  This is not the proof you might have seen!  It follows from what we have done:
if $[a,b]$ were countable, its measure would be zero, and we have shown this not to be true.

\item[outer measure is not additive:]  We will show that there are disjoint sets $A$ and $B$ such that $|A \cup B| > |A|+|B|$.  This shows that outer measure does not have a property essential to our notions of length of a subset of the reals.

We define an equivalence relation on $[-1,1]$ by $a \sim b$ iff $a-b$ is rational.  We use the notation $[a]$ for the equivalence class of $a$ under this relation (Axler's notation is different, I do not know how to typeset it).  

We then select a set $V$ by choosing exactly one element from each set $[a]$.  This is a very strange move.

Let $\{r_k\}$ be a sequence enumerating the rationals in $[-2,2]$ (Axler was right about this, I didnt notice something).  Require further than if $k \neq l$, $r_k \neq r_l$.

Notice that the sets $r_k+V$ cover $[-1,1]$:  each $x \in [-1,1]$  has $x \sim y$ for exactly one $y \in V$, and $x-y$ is rational and in $[-2,2]$, so equal to some $r_k$
and $x = (x-y)+y = r_k+y \in r_k+V$.

For $k \neq l$, $r_k + V$ and $r_l +V$ are disjoint:  if we have $x,y \in V$ with $r_k + x = r_l + y$ then $x\neq y$ and $x-y$ is $r_k - t_l$ which is rational, and two distinct elements of $V$ will not differ by a rational number (we chose just one element from each equivalence class).

Further, $|r_k+V|=|V|$, by translation invariance.

Because $\bigcup_{k=1}^\infty r_k + V \supseteq [-1,1]$, we have $2 \leq |\bigcup_{k=1}^\infty r_k + V|$ by the order preserving property and measure of a bounded interval.

By countable additivity $2 \leq |\bigcup_{k=1}^\infty r_k + V| \leq \sum_{i=1}^\infty |r_k+V| = \sum_{i=1}^\infty |V|$, so $|V|>0$.

Side remark:  it is easy to make the measure of $V$ as close to zero as you want, because you can pick any open interval in $[-1,1]$ and choose all elements of $V$ from that interval.
I believe I can write an harder set theoretical proof that the measure of $V$ can be made as close to 2 as you want, by choosing elements of $V$ really maliciously.  Neither of these facts
matter here:  all we need to know is that the measure of $V$ must be positive.

Now comes the bad final result.  Each of the sets $r_k+V$ has the fixed positive measure $|V|$.  These sets are disjoint, and there are infinitely many of them, and each of them is a subset of $[-3,3]$.  Take a finite number $n$ greater than $\frac6{|V|}$ of these sets, and you have a family of $n$ disjoint sets each with measure $|V|$ whose total measure is greater
than 6, but the measure of the union of these disjoint sets is less than or equal to the measure of $[-3,3]$, of which it is a subset, so less than or equal to 6.

If it were true that the union of two disjoint sets always had measure equal to the sum of the measures of the disjoint sets, this would also be true for any finite number of sets.
We have just given an example where this fails to be true, so the principle for two sets has to be false.

\end{description}

\end{document}