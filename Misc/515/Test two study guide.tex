\documentclass[12pt]{article}

\usepackage{amssymb}

\title{Test II  study guide, Math 515, fall 2023}

\author{Randall Holmes}

\begin{document}

\maketitle

Every item on the test (except possibly one) will be either a theorem in the book or supplement or a homework problem in one of your assignments.

Moreover, every one of them will be taken from the list of items in this study guide.

The exam does not contain many items past 3A -- I will highlight these.

You are welcome to ask me questions about items appearing in this study guide, and I'll give not necessarily complete answers.  Any answers I give privately, as in office hours, I will post or discuss in class if I think they give significant advantages.

The exam has not been written yet.  It will consist of 10 questions, organized into pairs.  You need do only four of the pairs;  if you do all, your best work will count.  In each pair, you will get 70 percent of the credit for the one you do best on and 30 percent for the one you do worse on.  No changes in my plan for the exam will be made which make this study guide inaccurate, without specific warning.

I might in some cases give three questions in a ``pair";  in this case you only need to do two, and the scoring is as above using the two problems you do best on.

\begin{description}

\item[There will be a pair of questions from section 2C:]    You will be supplied with basic definitions and at my discretion with statements of facts you may use in your proofs.

The questions I ask will be taken from:  theorem 2.57; theorem 2.58; theorem 2.59; theorem 2.60, 2C problem 1, 2C problem 9, 2C problem 10.

\item[There will be a pair of questions from section 2D:]  You will be supplied with statements of theorems which I do not ask you to prove but which you will need to use.

These will be taken from, theorem 2.62, theorem 2.63 (in which case you would be given the statement of theorem 2.62), theorem 2.66 (which has prerequisites which would be supplied), 2.76b, 2D problem 5 (for which I would state prerequisites), 2D problem 17 (think about the characterization of numbers in the Cantor set using their ternary (base 3) expansion:  why can any general number whose ternary expansion is given be written as the average of two elements of the Cantor set?).

\item[There will be a pair of questions from 2E:]  These will be taken from theorem 2.84,  theorem 2.89,  2E problem 1, 2E problem 9.  I'm not happy with the plausible options in this section;  I might let a 2D problem slide in here, possibly as a third option.

\item[There will be a pair of questions from 3A:]   These will be taken from theorem 3.4, theorem 3.20, theorem 3.21, theorem 3.22, 3A exercise 1, 3A exercise 3, 3A exercise 8

\item[There will be an extra pair of questions:]  These will be taken from theorem 3.28, theorem 3.29, 3B problem 6 (this is actually a question about the difference between conditional and absolute convergence, a Calc II topic), theorem 4.1 from section 4A, and the theorem I did in class that for any set $A$ at all there is a Borel set $B$ such that $A \subseteq B$ and $|A|=|B|$.

\end{description}




\end{document}