\documentclass[12pt]{article}

\usepackage{amssymb}

\title{Midterm study guide, Math 515, fall 2023}

\author{Randall Holmes}

\begin{document}

\maketitle

Every item on the test is either a theorem in the book or supplement or a homework problem in one of your assignments.

Moreover, every one of them appears on the list of items in this study guide.

The exam does not contain any items past section 2B.

You are welcome to ask me questions about items appearing in this study guide, and I'll give not necessarily complete answers.  Any answers I give privately, as in office hours, I will post or discuss in class if I think they give significant advantages.

The exam has been written.  It consists of 8 questions, organized into pairs.  In each pair, you will get 70 percent of the credit for the one you do best on and 30 percent for the one you do worse on.  I might rethink and make changes in the exam, but no changes will be made which make this study guide inaccurate.

You will also receive a copy of the exam which you may take home, work on at your leisure {\bf by yourself -- any merest, slightest  hint of similarity between any two take home papers  will cause me to ignore all of them --}
and which I may use for partial credit on exam questions at my discretion.   You should expect that most of the grade on your exam will depend on your in-class work.

\begin{description}

\item[There will be a pair of questions from the supplement part A:]    Ther axioms of an ordered field will be supplied to you.  Study 0.2, 0.3 in the text.  Study 0.6, 0.8  in the text.  Study problems 1,9,13 from the homework.

Be aware that in problems in equational algebra you are allowed the axioms only;  no familiar facts about algebra should be used.

In problems about order, you should not use properties of order.  It is actually better to carefully phrase your work so that you never use $<$ or its relatives at all except where they actually appear in the problem statement, so that you use only the definition of $<$ and properties of the set $P$ in your reasoning.  In problems about order, you may use equational algebra in the style familiar to you.

\item[There will be a pair of questions from 1A:]  One of these will be exactly to prove theorem 1.11.  You will be supplied with the definition of Riemann integration and the statement of the theorem that a continuous function on a closed interval is uniformly continuous, which you are allowed to use in your proof.

The other question will be taken from 1.5, 1.8  in the text (which you need to know anyway whether one of these is itself a problem or not), problems 3,4,5,10.

\item[There will be a pair of questions from 2A:]  These will be taken from 2.4, 2.5, 2.7, 2.8 (all of which you need to be familiar with as you may need to use them), problems 1,6,10,13.

You will be supplied with the definition of outer measure and descriptions of basic results needed to prove the assigned problems.

\item[There will be a pair of questions from 2B:]   These will be taken from theorem 2.41, 2.43  in the text, or one or two of problems 1, 5, 10, 22.

\end{description}




\end{document}