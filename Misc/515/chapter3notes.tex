\documentclass[12pt]{article}

\usepackage{amssymb}

\title{Notes on Axler Chapter 3}

\author{Randall Holmes}

\begin{document}

\maketitle

\begin{description}

\item[Homework 9:]  3A exercises 1, 3, 6, 8, 13, 3B exercises 2, 3, 5, 6, 15

You still only have to do three of these, but unlike earlier sets, if you do more you will get more credit (think of it as...successful work on more than three problems might give you another good homework assignment grade).

Problem 6 in 3B is interesting because it shows that the mechanism of improper integrals does not always agree with Lebesgue integration.  Think carefully about the situations in which the Lebesgue integral is undefined and you might be able to see how to build an example like this.

I might possibly add some chapter 4 problems to the mix here after I prepare my notes.

This is due at the final examination period.

\item[Status of notes:]  I'm done with notes on 3A and will update tomorrow with notes on 3B;  I do not think there is anything in the problems I actually asked which depends on anything eccentric I did in section 3B.

\end{description}

We begin with the definition of our new kind of integration.  Throughout, we will be talking about a measure $S$ on a space $X$ with a measure $\mu$.  Our default assumption is that $X$ is a subset of the reals and $\mu$ is Lebesgue measure, and coincides with outer measure on sets in $S$ (which is the set of Lebesgue measurable sets:  for this reason, we write $|A|$ instead of $\mu(A)$ for the measure of a set $A$.

A partition $P$  is a finite set $\{A_1,\ldots,A_m\}$ of measurable sets whose union is $X$.

Let $f$ be a nonnegative measurable function.  We define the lower Lebesgue sum of $f$ over $P$ as $L(f,P)$ as $\sum_{i=1}^m |A_i|\inf_{A_i}f$.

This looks just like what we did before, except that $X$ is more general than $[a,b]$, which is used in the definition of the Riemann integral, and
the partition $P$ is a partition into arbitrary measurable sets instead of subintervals.

An additional difference is that we do not need to define a notion of upper sum.  We certainly could, but we have no need for it.

We then define the integral of $f$ over the measure $\mu$ by

$\int f d\mu = \sup\{L(f,P): P$ is a partition of $X\}$

It is important to note that we have only defined integrals of nonnegative functions $f:X \rightarrow [0,\infty]$.  We will fix this later.

It is a useful result that $\int \chi_E d \mu= |E|$ for any measurable set $E$.  The proof in Axler is fine for my purposes, theorem 3.4 p. 75

Example 3.5 makes the nice point that we now correctly integrate the characteristic function of the irrationals in $[0,1]$ and get 1, because we can
use the partition into the rational and irrational numbers in $[0,1]$, the mnimum of the characteristic function on the irrationals is 1 and the measure of the set of irrationals in $[0,1]$ is 1, so the integral in fact turns out to be one.  In contrast, the lower Riemann integral of the characteristic function of the irrationals on [0,1] is 0, because on any {\em interval\/} the minimum of this characteristic function is 0.

Axler theorem 3.7 shows that the integral of a simple function defined as $\sum_{i=1}^n c_i \chi_{E_i}$, where the $c_i$'s are nonnegative reals and the $E_i$'s are a pairwise disjoint collection of subsets of $X$, is $\sum_{i=1}^n c_i|E_i|$ as one would expect.  This is not the interesting case where the sets may overlap, and Axler's proof is good for my purposes.  Notice in the proof his use of changing the order of summation (proof of 3.7 on p. 76).

Theorem 3.8 is the obvious result that if $f(x) \leq g(x)$ for all $x$ then the integral of $f$ is less than or equal to the integral of $g$.  I have nothing special to say about the very easy proof, but you should read it (p. 77, proof of 3.8).

Theorem 3.10 (with a glance at 3.7) says that the integral of $f$ is the supremum of the integrals of all nonnegative simple measurable functions dominated by $f$.  What Axler says makes it look like a much more complicated statement.  I think my contribution is rephrasing it:  I am comfortable with his proof (proof of 3.10, p. 77)

3.11 is the first of a series of theorems of the general form $$\int (\lim_{k \rightarrow \infty}f_k(x)) d \mu = \lim_{k \rightarrow \infty} f_k(X) d\mu.$$  In general this statement (that limits and integrals can be interchanged) is false!  It holds only under special conditions.  In 3.11, the restrictive condition that makes this work (a quite strong one) is that
$\{f_k(x)\}_{k \in {\mathbb Z}^+}$ is an increasing sequence for each $x$ (thus {\em monotone\/} convergemce theorem).

I was very interested in this proof (I seem to recall lecturing it twice) and I advise you to study it, but I didn't restate it in any way when I lectured it:  see proof of 3.11 p. 78.

I had a lot of ``fun" with his theorem 3.13, p. 79, and incurred the obligation to write out my own proof of this statement.

I believe that this can be done by simply proving 3.15:  if $\{c_i\}_{1\leq i \leq n}$ is a finite sequence of nonnegative constants and $[{A_i}_{1 \leq i \leq n}$ is a finite sequence of subsets of $X$ then $\int \sum_{i=1}^n c_i\chi_{A_i} = \sum_{i=1}^n c_i|A_i|$.

I will prove this below but first I'll point out why this implies the weird statement 3.13.  Suppose we have constants $a_i$ and sets $A_i$ for
$1 \leq i \leq n$ and constants $b_i$ and sets $B_i$ for $1 \leq i \leq m$, and further we have $\sum_{i=1}^n a_i\chi_{A_i} = \sum_{i=1}^n b_i\chi_{B_i}$ (this is an assertion that two {\em functions\/} are the same).  Then $\int \sum_{i=1}^n a_i\chi_{A_i} = \int \sum_{i=1}^n b_i\chi_{B_i}$:  integrals of the same function are equal.  And by 3.15 this converts to $\sum_{i=1}^n a_i|A_i| = \sum_{i=1}^n b_i|B_i|$.

So the upshot is that I'm not sure what 3.13 is for.  Proving 3.15 shows that it is true, and I will now do that.

We suppose that $\{c_i\}_{1\leq i \leq n}$ is a finite sequence of nonnegative constants and $\{A_i\}_{1 \leq i \leq n}$ is a finite sequence of subsets of $X$.

We want to engineer a disjoint collection of sets which we can use to define the same function.  For any natural number $x$, let $d(i,n)$ be the
$i$th digit (0 or 1) in the base 2 expansion of $x$.  If $1 \leq x \leq 2^n$ then $x = \sum_{i=0}^n d(i,n)2^i$.

We define $V_i$ for $1 \leq i \leq 2^n$ as $$\{x \in X:(\forall j:(1 \leq j \leq n) \rightarrow x \in A_j \leftrightarrow d(j-1,i)=1\}.$$

The idea is that the $V_i$'s are the $2^n$ pairwise disjoint compartments in the Venn diagram formed by the $n$ sets $A_i$.

Now we compute.

$\int  \sum_{i=1}^nc_i\chi_{A_i} =$

$\int \sum_{i=1}^n c_i\chi_{\bigcup_{j=1}^{2^n} A_i \cap V_j} =$  [the characteristic function of a union of pairwise disjoint sets is the sum of the characteristic functions.  The collection of $A_i \cap V_j$'s is certainly pairwise disjoint though somehow it seems like too many sets (a lot of them are empty of course)!]

$\int \sum_{i=1}^n c_i \sum_{j=1}^{2^n}\chi_{A_i \cap V_j}=$

$\int \sum_{i=1}^n \sum_{j=1}^{2^n}c_i\chi_{A_i \cap V_j}=$ [if $d(i-1,j)$ is 0 we can just insert it here because $A_i \cap V_j$ is empty]

$\int \sum_{i=1}^n \sum_{j=1}^{2^n}c_id(i-1,j)\chi_{A_i \cap V_j}=$ [interchange the order of summation]

$\int \sum_{j=1}^{2^n} \sum_{i=1}^n c_id(i-1,j)\chi_{A_i \cap V_j}=$ [something tricky:  $c_id(i-1,j)\chi_{A_i \cap V_j}$ is always $c_id(i-1,j)\chi_{ V_j}$:  if $d_(i-1,j)=1$ we have $A_i \cap V_j=V_j$, and otherwise the whole thing is 0 either way.]

$\int \sum_{j=1}^{2^n} \sum_{i=1}^n c_id(i-1,j)\chi_{V_j}=$ [now the factor $\chi_{V_j}$ can be pulled out because it does not depend on $i$]

$\sum_{j=1}^{2^n} (\sum_{i=1}^n c_id(i-1,j))\chi_{V_j}=$ [by the theorem we already have on integration of simple functions]

$ \sum_{j=1}^{2^n} (\sum_{i=1}^n c_id(i-1,j))|{V_j}|=$ [the rest of the argument reverses the summation manipulations using measures instead of characteristic functions]

$ \sum_{j=1}^{2^n} \sum_{i=1}^n c_id(i-1,j)|V_j|=$

$ \sum_{j=1}^{2^n} \sum_{i=1}^n c_id(i-1,j)|A_i \cap V_j|=$

$ \sum_{i=1}^n \sum_{j=1}^{2^n}c_id(i-1,j)|A_i \cap V_j|=$

$ \sum_{i=1}^n \sum_{j=1}^{2^n}|A_i \cap V_j|=$

$\sum_{i=1}^n c_i|\bigcup_{j=1}^{2^n} A_i \cap V_j| =$

$\sum_{i=1}^nc_i|A_i|$

This is not the same as what I did in class;  there I was really working on trying to prove 3.13.  But I think I did appeal to the integration of simple functions in that argument, so really I was doing something very much like this, and it turns out that there is really no need for 3.13 as stated at all:  just prove 3.15, which is what we really want, by brute force, as I do here.

Theorem 3.16 on the additivity of integrals of nonnegative functions is then very straightforward to prove.  3.15 implies immediately that the integral is additive on simple functions.  We use earlier technology to present functions $f$ and $g$ as limits of increasing sequences $\{f_k\}$ and $\{g_k\}$ of simple functions,
and then the monotone convergence theorem and the ordinary limit addition theorem for sequences show the general additivity theorem:  see the bottom of p.80 
for the very brief proof that Axler gives.

The whole development so far has been for nonnegative functions.  For negative functions, we use a trick:  for a general function $f$, define $f^+$ as $\frac{|f|+f}2$ and $f^-$ as $\frac{|f|-f}2$.  $f^+(x) = f(x)$ if $x$ is positive and 0 otherwise;  $f^{-}(x) = -f(x)$ if $f(x)$ is negative and otherwise 0.  Both of these are nonnegative functions, and $f^+-f^-=f$, so it is natural to define $\int f d\mu$ as
$\int f^+ d\mu -\int f^- d\mu$, with the observation (which is the reason for the restriction to nonnegative functions so far) that this is undefined if both $\int f^+d\mu$ and $\int f^- d\mu$ are $\infty$.

Theorems 3.21, 3.22 and 3.23 are important and relatively easy to prove.  In 3.21, familiarizing yourself with manipulations of the functions with + and - superscripts is a good idea.  The second equation in the proof of 3.21 is a useful one;  I would generally familiarize yourself with that proof.


\end{document}