\documentclass[12pt]{article}

\title{Riemann integrals of continuous functions exist}

\author{Randall Holmes}

\usepackage{amssymb}

\begin{document}


\maketitle

In this snippet of notes, we will work toward proving that $\int_a^b f$ exists for any $f$ continuous on $[a,b]$ along with prerequisite and related results

We begin by proving that if a function $f$ is nondecreasing on $[a,b]$, $\int^a_b f$ exists.

Suppose that $f$ is nondecreasing on $[a,b]$.  (this means that for any $x,y \in [a,b]$ with $x<y$, $f(x) \leq f(y)$).

We will show that for any $\epsilon > 0$, there is a partition $P$ such that $U(f,P) - L(f,P)<\epsilon$.  It is a homework exercise in your current assignment that this is sufficient to
establish that $\int_a^b f$ exists.

Let $P$ be a partition $\{x_i\}_{0 \leq i \leq n}$ of $[a,b]$ such that there is a constant $\delta <\frac{\epsilon}{f(b)-f(a)}$ such that $x_i - x_{i-1} = \delta$ for each $i$ for which this is defined:  $P$ determines a subdivision
of $[a,b]$ into closed intervals all of the same length strictly less than $\epsilon$.

Now $$U(f,P) - L(f,P) = \sum_{i=1}^n (x_i-x_{i-1})(\sup_{[x_{i-1},x_i]}f -\inf_{[x_{i-1},x_i]}f)$$

$$= \sum_{i=1}^n \delta (f(x_i) -f(x_{i-1}))$$

[because the length of each interval in $P$ is $\delta$ and $\sup_{[x_{i-1}]}f = f(x_i)$ and $\inf_{[x_{i-1}]}f = f(x_{i-1})$ because
$f$ is nondecreasing ]

$$= \delta \sum_{i=1}^n (f(x_i) -f(x_{i-1})) = \delta(f(b)-f(a)) < \epsilon$$

[the second equation holds because $\sum_{i=1}^n (f(x_i) -f(x_{i-1}))$ is a telescoping sum]

And this completes the proof that $\int_a^b f$ exists, mod the homework assignment mentioned.

I strongly recommend and may assign proving the same result for nonincreasing functions $f$.

The proof that $\int_a^b f$ exists if $f$ is continuous on $[a,b]$ relies on the theorem that a function $f$ continuous on a closed
interval $[a,b]$ is uniformly continuous on $[a,b]$.  We first explain what this statement means, then use it to prove
that $\int_a^b f$ exists, then perhaps prove the prerequisite theorem.

That $f$ is continuous on a set $A$ means that for each $x \in A$, there is an $\epsilon>0$ such that for any $y \in A$,
if $|y-x|<\delta$ then $|f(y)-f(x)|<\epsilon$.  This follows from the usual definitions of limits and continuity which you should have known since undergraduate real anaysis if not since Calculus I.

That $f$ is uniformly continuous on a set $A$ means that for each $\epsilon >0$, there is $\delta>0$ such that for any $x,y \in A$,
if $|x-y|<\delta$ then $|f(x)-f(y)|<\epsilon$.

The second assertion is stronger:  it says that the tolerance of error $\delta$ you need such that if $y$ is that close to $x$,
$f(y)$ will be within $\epsilon$ of $f(x)$ does not depend on $x$:  the same tolerance works everywhere in the set $A$.

The prerequisite theorem is ``If $f$ is continuous on $[a,b]$, $f$ is uniformly continuous on $[a,b]$".  For the moment we assume
this and proceed to prove that $\int_a^b f$ exists.

Again, we will show that for any $\epsilon > 0$, there is a partition $P$ such that $U(f,P) - L(f,P)<\epsilon$.  It is a homework exercise in your current assignment that this is sufficient to
establish that $\int_a^b f$ exists.

Choose $\epsilon >0$ arbitrarily

Choose $\delta$ such that for any $x,y \in [a,b]$, if $|x-y|<\delta, |f(x)-f(y)|< \frac\epsilon{2(b-a)}$.

Let $P$ be the partition of $[a,b]$ determined by $\{x_i\}_{0 \leq i \leq n}$ subdividing the interval into closed intervals all with equal length $\delta$.

Now $$U(f,P) - L(f,P) = \sum_{i=1}^n (x_i-x_{i-1})(\sup_{[x_{i-1},x_i]}f -\inf_{[x_{i-1},x_i]}f)$$

$$= \sum_{i=1}^n \delta(\sup_{[x_{i-1},x_i]}f -\inf_{[x_{i-1},x_i]}f)$$

$$\leq \sum_{i=1}^n \delta \frac{\epsilon}{2(b-a)}$$

because $\sup_{[x_{i-1},x_i]}f -\inf_{[x_{i-1},x_i]}f \leq \frac{\epsilon}{2(b-a)}$ since the length of the interval is $\delta$
(any two points in the interval except $x_i$ and $x_{i-1}$ are at distance $<\delta$ and have values of $f$ differing by less than $\frac{\epsilon}{2(b-a)}$;  $x_i$ and $x_{i-1}$ are at distance exactly $\delta$ but continuity of $f$ lets us see that the values of $f$ at the endpoints might differ exactly by $\frac{\epsilon}{2(b-a)}$ but no more:  so the difference between the largest and smallest value
of the function on the interval is bounded above by $\frac{\epsilon}{2(b-a)}$ and $\sup_{[x_{i-1},x_i]}f -\inf_{[x_{i-1},x_i]}f$ is no greater than $\frac{\epsilon}{2(b-a)}$.

$$= \sum_{i=1}^n \frac{b-a}n \frac  {\epsilon}{2(b-a)} = \frac\epsilon2 < \epsilon$$

Note that $\delta=\frac{b-a}n$.

I'll lecture the proof that a continuous function on a closed interval is uniformly continuous on Sept 6;  notes on it will be added here eventually.



\end{document}