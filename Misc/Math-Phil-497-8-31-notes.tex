\documentclass[12pt]{article}

\usepackage{frege}

\usepackage{yfonts}

\usepackage{amssymb}

\title{Notes on class discussion of August 31}

\author{Randall Holmes}

\begin{document}

\maketitle

This was an interesting discussion.  We are getting into what is perhaps the earliest presentation of the modern system of propositional logic and logic of quantifiers, but with a philosophical view in the background rather different from common current views.

To write these notes, I am learning to use a LaTeX package {\tt frege.sty} that typesets Frege's notation!  

If anyone wants to learn to use this package, I have the documentation, and MikTeX, the version of LaTeX I use, downloaded the package automatically when I asked for it.

\begin{description}

\item[p. 142:]  Frege introduces the operation $\F[1]x$:  $\F[1] x$ is $x$ if $x$ is a proposition (either {\bf true} or {\bf false}:  I introduce these as notations for the True and the False), and {\bf false} otherwise.  An informal way to read $\F[1] x$ is just $x={\bf true}$.

This is needed because Frege views truth functions as objects, and he wants any object term to be able to go in any context appropriate for an object.  This is an example of what we would call ``type casting" in computer science:  if we write an object term which isn't {\bf true} or {\bf false} in a context which is appropriate for a proposition, it will act like {\bf false}.

So $\F[1] 4$ is a name for the False.

\item[p. 142:]  Frege introduces $\vdash$, the judgement stroke, to signal that a following proposition is actually being asserted.

$\F[1] 2+2=4$ is a name for {\bf true}.

$\Fa[1]2+2=4$ shows that we are not only naming a truth value but actually saying that $2+2=4$.

$\F[1]2+2=5$ is a name for {\bf false}.

$\Fa[1]2+2=5$ is something we will hopefully never write.

The judgment stroke seems to address our discomfort with the idea that sentences are being treated as nouns!

\item[p. 143:]  Frege introduces his logical negation operator.  $\Fn[1] x$ will denote the True if $x$ is the False and otherwise will denote the False.  We would like to represent
$\Fn[1] P$ as $\neg P$ or $\sim P$, the usual logical negation of $P$ and this is accurate for propositions.  An accurate informal representation of the meaning of 
$\Fn[1] x$ is $(\F[1] x) = {\tt false}$ or $(x = {\bf true}) = {\tt false}$, if one is comfortable with the idea that an equation is a term with a value.  An even pithier representation is
$x \neq {\tt true}$, but using inequality to define negation seems circular...

so for example we can say

$\Fan[1] 2^2=5$

to deny (correctly) that $2+2=5$, and we can correctly say

$\Fan 2$

because 2 is not the True.

$\Fn[1] 2^2=4$ 

is a name for the False, which might appear in some context, but we would certainly never say

$\Fan[1] 2^2 = 4$.

\item[p. 143:]  We might expect to see the binary operations such as conjunction, disjunction, implication, and the biconditional next...but we don't!  Frege presents the universal quantifier next.

Frege gives $x=x$ as an example of a function all of whose values are the True.  Given his qualms about variables, I would like to write this ${\bf X}={\bf X}$, where $\bf X$ is used
invariably as an argument place, never as an unknown number ($\bf X$ by itself is a name for the identity function $I(x)=x$, not an unknown quantity).

We would like to {\em say\/} that this is always true in a way that makes it clear that we are claiming that $x=x$ is true for all $x$, not for some unspecified particular value.

For this Frege introduces the notation

$\Fq[1.5]{a}\, \textfrak{a}=\textfrak{a}$


Frege insists on using different type faces for bound variables bound by quantifiers and by course-of-values  notation.  We would write this just $(\forall x:x=x)$.  It is safe to use
the same letters, but Frege was being very cautious, and it can be useful to be reminded forcibly that different letters are doing different things.

The rule is that $\Fq[1.5]{a}\, f(\textfrak{a})$ represents the True iff $f(x)$ represents the True for any object $x$ and otherwise represents the False.

It is worth noting that $\grave{\epsilon}(f(\epsilon)) = \grave{\epsilon}(\epsilon=\epsilon)$, or $$(x \mapsto f(x)) = (x \mapsto x=x)$$ is equivalent to $\Fq[1]{a}\, f(\textfrak{a})$ in Frege's logic:  it would not have been impossible for him to define universal quantification (over objects) in terms of courses of values.

It is also useful to observe that Frege's universal quantifier is unrestricted:  it ranges over all objects without exception.  We can use implication to get restricted quantifiers once we have it.

This can be combined with the judgment stroke:  to write 

$\Faq[1.5]{a} \,\textfrak{a}=\textfrak{a}$

is to assert the reflexive law of equality.

This is notationally unfamiliar but is simply the familiar operation of universal quantification.

It can be combined with the notation for negation as in

$$\Fanq[1.5]{a} \,\textfrak{a}^2=1$$

the assertion that not all numbers are square roots of one.

\item[p. 144:]  There is no need for the existential quantifier as an independent operation:

$$\Fanqn[1.5]{a} \, \textfrak{a}^2 = 1$$

asserts that it is not the case that for every $x$, $x$ is not a square root of one.  $\neg (\forall x:\neg x^2=1))$ is equivalent to $(\exists x:x^2=1)$.

He notes, importantly, that $$\Fanqn[1.5]{a} \, f(\textfrak{a})$$  can be understood as a {\em function\/} of a variable $f$ (I might write this  $$\Fanqn[1.5]{a} \, {\bf F}(\textfrak{a}),$$
the use of the boldface F making it clear that this is to be understood as an argument place not a unspecified function.  This function is an object we are familiar with:  it is simply the existential quantifier as an operation.  Frege calls this a second-level or second-order function:  an ordinary function of a single object variable we would call first-level or first-order;
all functions for Frege (at least in his main work) have objects as output, but second level ones are distinguished from first level ones by having first level functions instead of objects as input.

\item[p. 146:]  His discussion of functions of two arguments (I'll use the boldface letter $\bf Y$ for a second argument to distinguish it from an unknown) is interesting, and in some ways a departure from what he does elsewhere.  His phrasing suggests that he is inviting us to view $3>2$ as the application of the function ${\bf X}>2$ to 3, and then to view the function ${\bf X}>2$ as the application of a function ${\bf X}>{\bf Y}$ to 2.  He subsequently writes
values of functions $f$ of two arguments in the form $f(x,y)$, but this discussion suggests that he might think of this as $f(x)(y)$ [a maneuver known as ``currying"].  In his major formal work, he avoids talking about functions with function output altogether (all functions of whatever order have object output).  He does use currying to reduce talk of courses of values of functions with two variables to talk of courses of values of functions with one variable in his main work.

He introduces the term {\em relation\/} for functions of two arguments which have truth value output, which correlates with his use of the term {\em concept\/} for functions of one argument with truth-value output.

\item[p. 147:]  Now Frege introduces a binary propositional connective at last (and in fact he introduces only one).  This is implication.  What we would write as
$P \rightarrow Q$ (if $P$ and $Q$ are propositions) he writes $$\Fcdt[\F]{\F Q}{\F P}$$

There are two weird things to notice here.  The notation is two dimensional, and one needs to get used to the idea that what is first in an implication for us is below in this notation.

This can get complex.

$(P \rightarrow Q) \rightarrow (R \rightarrow S)$

becomes $\Fcdt[\F]{\Fcdt[\F]{\F S}{\F R}}{\Fcdt[\F]{\F Q}{\F P}}$

Isn't that delightful?

An extra point that I didn't make in class is that the other binary connectives we are used to all reduce for Frege to constructions using implication and negation.

$P \wedge Q$ is equivalent to $\neg(P \rightarrow \neg Q)$:

$$\Fcdt[\Fn]{\Fn Q}{\F P}$$

$P \vee Q$ is equivalent to $(\neg P) \rightarrow Q$:

$$\Fcdt[\F]{\F Q}{\Fn P}$$

We can build $P \leftrightarrow Q$ in this way, but notice that it is also neatly expressed by $(\F P) = (\F Q)$.

in all of these operations, a non-truth value will behave like the False.  The official definition of $$\Fcdt[\F]{\F Q}{\F P}$$ is that it is the False if $P$ is the True
and $Q$ is not the True, and otherwise it is the True.  Frege does not even mention implication when defining it.

\item[p. 147:]  He closes the paper with a discussion of other sorts of function which is philosophically important. He gives an example of a second order function which takes a first-order function with two arguments as input.  I'll typeset this function and discuss its meaning below.  From calculus he gives the example $\int_0^a\,f(x)$ of a mixed level function
which takes an object $a$ and a first level function $f$ as arguments.

I was asked in class about any relationship between Frege's levels and the difference between first and second order logic.  There is a relationship, but it is a bit indirect.
First order logic involves quantification over objects.  If one defines the universal quantifier of first-order logic in Frege's terms, it is a function which takes a first-level function
(a predicate of objects) as an argument.  If one defines the universal quantifier of second-order logic (which {\em does\/} occur in Frege's system) as a function in Frege's terms, 
it is a (third-order!) function which takes a second-order function (a predicate of first-order functions = predicates or relations on objects) as an argument.

\item[the p. 147 example and alternative meanings of the word ``function":]

For my sins, I will type set this example:

$$\Fcdt[\Fq{e}\Fq{d}]{\Fcdt[\Fq{a}]{\F \textfrak{d}=\textfrak{a}}{\F f(\textfrak{e},\textfrak{a})}}{\F f(\textfrak{e},\textfrak{d})} $$

As a teacher, I would much rather give a horrible example than be a horrible example, but the latter can also be useful.  When I first parsed this, I misread it as

$$(\forall e:(\forall d:(f(e,d) \rightarrow (\forall a:d=a \rightarrow f(e,a)))))$$

which is a direct consequence of basic properties of equality, so this would be true no matter what $f$ is.  But I fell into a trap:  I was reading one of the implications in the wrong order!

Actually, this says $$(\forall e:(\forall d:(f(e,d) \rightarrow (\forall a:f(e,a)\rightarrow d=a)))),$$

which expresses a nontrivial property of a relation:  if $f(e,d)$ is true and $f(e,a)$ is true, then $d=a$.

If we write this with relations between their arguments, as is more usual this says that if $e \, F\, d$ and $e\, F\, a$ are true, for any $e,d,a$, then $d=a$.  In a usual presentation of set theory, we would say that a relation with this property is a ``function", which is totally confusing in this context, because Frege is using the word ``function" in a quite different though not unrelated way.

This is best explained by talking about a specific example.  The square function $y=x^2$ can be viewed in a modern discrete math class as the relation $Q$ such that
$x \, Q \, y$ holds iff $y = x^2$.  This relation does have the property indicated above:  if $e \, Q\, d$ and $e \, Q a$, then by definition of $Q$, $d = e^2$ and
$a=e^2$ so $a=e$.

But Frege does not identify $Q$ with the function $f(x)=x^2$,  For him $Q$ is a function with two object inputs and a truth value output, and $f$ thus defined
is a function taking a single object input and giving a single object output.  This doesn't mean that Frege doesn't appreciate the importance of the kind of relations that we now identify with functions:  in fact, he defines an operation which I will anachronistically write $Q`x$ (this notation is due to Russell) which takes a functional relation $Q$ and an object $x$ as its two
arguments and returns the unique object $y$ such that $x\,Q\,y$ (actually I am cheating, I think:  I believe that Frege defines this in such a way that its arguments are both objects, the first being the course of values of $Q$ and the second being $x$).

The underlying point is that the notion of function that Frege is using and the one that we implicitly usually use in mathematical discourse are different in interesting and important ways.

\end{description}








\end{document}