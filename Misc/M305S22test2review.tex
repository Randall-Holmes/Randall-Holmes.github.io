\documentclass[12pt]{article}

\title{Math 305 Test II review sheet}

\author{Dr Holmes}

\begin{document}

\maketitle

The exam is on April 7.  It covers  Crisman 7.4, 7.5, Judson chapter 3, chapter 4, chapter 5, chapter 9, and section 10.1, also a fact in section 13.1.
In my notes, pp. 41-60.

There is plenty to do here, and it is rather miscellaneous in my mind.  There will not be more than 8-10 questions you need to do;  there might as in the last test be more questions with you having a choice of which ones to drop.

\begin{description}

\item[section 7.4-5 in Crisman:]  In the 7.7 exercises in Crisman, I suggest problem 9 and problem 11 as study questions.  I also like problem 8.

I could also ask for simplification of an exponential in a modulus $n$ where you can easily compute $\phi(n)$ (a small one, as the full method of computing $\phi(n)$ which I lectured on March 31 is not in your test coverage).

If I ask you about the Fermat and Wilson theorems (or the Legendre theorem) I will provide a statement of the theorem.

\item[chapter 3, Judson:]  

The formal definition of a group is examinable.

You should be ready to answer a question as to whether a familiar mathematical system, or a small finite one, is a group or not, and to say why not if it isn't.

Any of the Propositions 3.18-3.22 are legitimate targets for examination.

Be ready to prove one of 3.23 (1) and (2) by mathematical induction.  Notice that there are cases involved
as to whether $m$ and $n$ are positive or negative.  

I may ask you to identify subgroups of some small group.

In Judson 3.5 exercises, I specially recommend 2, 25 (an induction proof), 31, 33, 41 (just an example of showing that an explicitly given mathematical system is a group), 51.

\item[chapter 4, Judson:]

Be able to prove theorem 4.10 (proof uses number theory, the division algorithm).

Be able to use Theorem 4.13, if not prove it.

Questions recommended from 4.5 exercises:  5, 6, 8.  You should be able to look at a group $U(n)$ and determine whether it is cyclic, 24, 25, 26, 29, 37 (what else can you say about such a group?)

\item[chapter 5, Judson:]

Be able to convert from the array notation for permutations to cycle notation and back.  Be able to convert cycle notation to a product of transpositions and identify permutations as even or odd.

Be able to compute products of permutations given in either notation.

You should understand the notations $S_n$, $A_n$, $D_n$ for particular groups.

Recommended questions from 5.4 exercises:  1,2,3 (computation practice), 8, 9, 10.

Question for study:  is a permutation of odd order an odd permutation or an even permutation?  Or can it be either?

is a permutation of even order an odd permutation or an even permutation?  Or can it be either?

\item[chapter 6, Judson:]

Be able to compute cosets of  subgroup given to you in a group given to you.

Be able to prove lemma 1 and lemma 2 in my notes, p. 53.

Know Fermat's Theorem and Euler's Theorem and be able to prove them using Lagrange's Theorem.
The proof is easy of course, but conceptually important.

Recommended exercises from 6.5:

2, 5, 16 (you all had to look up this argument, I think, but having looked it up...you should know it), 17, 21.

\item[chapter 9, Judson:]

Show that the relation of isomorphism between groups is an equivalence relation.

Be able to describe isomorphisms between familiar groups or small finite groups presented to you, giving actual formulas or tables of values for the isomorphism function.  Be able to verify that a function from a group to a group that I give you
is an isomorphism.

Be able to recognize when groups cannot be isomorphic, for example when they have different numbers of elements of some order.

Be aware of Cayley's Theorem (every group of order $n$ is isomorphic to a subgroup of $S_n$).  You don't have to prove it but I may come up with a question whose answer requires you to use it.

Be able to give an operation table for a small product group.

Be able to use theorem 9.17.  Be aware of theorem 9.21:  it might be wise to know how to prove it.

I might ask you to verify that a given group is the internal direct product of two of its subgroups (also given).  If I do, I will state the conditions you need to prove.

Recommended questions in 9.4 exercises:  1, 3, 5, 12, 15, 16, 28, 31, 48, 50

\item[chapters 10 and 13:]

For chapter 10, the most I might ask you to do is, given a specific group and a specific subgroup of that group, determine
whether the subgroup is normal.  I might to so far as to say, show that the subgroup is normal, or show that it is not normal.

For chapter 13, I might ask you to describe the abelian groups of a particular size, up to isomorphism, using Theorem 13.10.


\end{description}

\end{document}