\documentclass[12pt]{article}
\usepackage{amsmath}
\usepackage{amssymb}

\newcommand\Line[1]{\overset{\leftrightarrow}{#1}}

\title{Math 311, Test I, Spring 2023}

\author{Randall Holmes}

\begin{document}

\maketitle

This is a take home exam.

The primary limitation is the obvious one, which of course it is hard for me to enforce directly.  Please consult no person other than myself
while doing this exam.  If I see suspiciously similar work on multiple student papers, I will have to consider more strenuous proctoring options (which will imply harder conditions while taking a test, and much less time) on subsequent exams.

Every proof on your paper should be presented in a line by line two column format:  in the first column a series of statements, in the second column the justifications for these statements.  I did an example of a proof in this format in class, and will do more exampls on Wednesday.

The exam is due at 11:55 pm  next Monday, to be turned in electronically (so class on Monday is also available for the resolution of issues if any arise).

\newpage

\begin{enumerate}
\item  logic (book)

If we consider a conditional statement $H \rightarrow C$ it is conventional to define the converse of $H \rightarrow C$ as $C \rightarrow H$ and
the inverse of $H \rightarrow C$ as $\neg H \rightarrow \neg C$,  Use truth tables to demonstrate that the converse and the inverse of a conditional are equivalent to each other (they are true in the same circumstances) and that the original conditional statement is not equivalent to its converse or inverse.

Your demonstration should be a truth table with some comments on specific rows and columns making the points you need to make.

\newpage

\item  logic (manual)

In the system of the manual of logical style, prove that $$((A \wedge B) \vee (B \wedge C)) \rightarrow B.$$  You will need to use proof by cases.

You can think of the system of the manual as practice for writing proofs in the two column format I am asking for here.

Use the manual of logical style for reference:  I did not include a digest of the rules here.  Use the version on the Math 311 page.

\newpage

\item  Suppose that we modify the axioms of incidence geometry to say this:

\begin{description}

\item[primitive notions:]  point, line, ``lies on" as usual.

\item[IA1:]  For each pair of distinct points $P, Q$, there is one and only one line $L$ such that $P$ lies on $L$ and $Q$ lies on $L$. (same as the usual IA1)

\item[IA2$^*$]  For each line $L$, there are three distinct points on $L$ (different from the usual IA2 in requiring three points instead of two).

\item[IA3:]  There are three distinct points which do not all lie on the same line (same as the usual IA3).

\end{description}

Prove using these axioms (and nothing else) that there are at least six points (it is possible to prove that there are seven, but I don't require this;  go ahead and do it if you can).

Steps in the proof should be justified by single axioms or by hypotheses introduced in the course of the proof or by previous lines in the proof, not, for example, by theorems you have already proved in incidence geometry, though it may be very useful to use proofs already done in class as models.  Hint:  start by proving that there are three distinct lines, in the same way that you have probably already done.

If you really want to use a theorem already proved in incidence geometry in class, you may use it if you include its proof (in two column format) on your paper (and give it a name so it can be properly referenced to justify lines in your main proof).

It is very proper that your proof be supported by a diagram, but the entire proof should appear in words.

\newpage

\item  If we have a space with a metric $d$ and distinct points $A, B$ define the circle with center $A$ and radius $\overline{AB}$ as the set of all points
$C$ such that $d(A,B)=d(A,C)$.  This is precisely the usual definition of a circle, except that the distance measurement is not assumed to be the usual one.

You know that in ordinary geometry (on the plane with the Euclidean metric) two distinct circles with nonempty intersection intersect in two points.  You don't have to prove this:  it is just background for this question.

For this exercise, our space is the usual plane and our metric is the taxicab metric:  $d((x,y),(z,w)) = |x-z| + |y-w|$.

Give an example, with a picture, of two circles with specific centers and radii (give actual coordinates for the points involved) which are distinct and intersect in exactly two points.

Give an example, with a picture, of two circles with specific centers and radii (give actual coordinates for the points involved) which are distinct and intersect in infinitely many points.

\newpage

\item  Prove (using the axioms of chapter 3, which I refer to as ``neutral geometry" for the moment) that for any distinct points $A,B$ that there are
points $C,D$ such that $A*C*B$ and $A*D*B$ and $d(A,C) = d(C,D) = d(D,B)$:  in other words, prove that a line segment can be trisected.  This is an application
of the Ruler Postulate very similar to the proof that there are midpoints.  It involves a little algebra!

When you have exhibited your points, please be sure that you actually verify that they have the stated properties.  Just saying what the points are will carry substantial but not complete credit.

It is true that the points $C,D$ are uniquely determined, but I do not require you to prove this.

\newpage

\item  Exercise 3.3.5 in the book:  suppose $\triangle ABC$ is a triangle and line $L$ is a line such that none of $A,B,C$ are on $L$,

Prove that line $L$ cannot intersect all three sides of $\triangle ABC$.

Can a line $M$ intersect all three sides of  $\triangle ABC$?

Read definition 3.3.11 of a triangle carefully.

\newpage

\item  Prove that if $L$ is a line and $f$ is a coordinate function for $L$ and $r$ is a real number, that the function $g$ from
$L$ to $\mathbb R$ defined by $g(P) = r-f(P)$ is a coordinate function for $L$.  Remember that the proof is divided into three parts.

\newpage


\item Prove that if $\overrightarrow{AB} = \overrightarrow{CD}$ then $A=C$ (Hint:  suppose otherwise and write out betweenness statements:  please try to be sure that you reason all the way to a contradiction, and do not just point out that something is intuitively impossible where this requires some proof;  notice that
$A,B,C,D$ all lie on the same line (you can prove this) and it will be useful to fix a coordinate function $f$ for that line and use theorem 3.2.17 [which you are allowed to cite, you don't have to prove it from the axioms!]).

The Ruler Placement Postulate also could be useful.

\newpage

\item Prove  proposition 3.3.4, p. 47 in the book,  using Axiom 3.3.2 and Definition 3.3.3.  Notice that the two parts of the proposition are if and only if statements, so there are four conditional statements to prove.  This should be boring because of all the detail you have to write out, not especially tricky.

\newpage

\end{enumerate}

\section{Venema's Axioms, with some definitions and theorems}

2013 exam:  For rules of logic for problem 2, use the logical style manual (the version on the Math 311 page).

2013 exam:   I provide the following because I do give somewhat different statement of some axioms and definitions than the book does.  They are equivalent, and the book is a quite adequate reference. 

The official theory has as its primitive notions points, lines (which are sets of points) and the notion of distance:  $d(P,Q)$ is a real number for points $P,Q$ (Venema himself writes
$PQ$ instead of $d(P,Q)$), and half-planes, which are special sets of points described in the Plane Separation Axiom.

\begin{description}
\item[Existence Axiom:]  There are at least two points.

\item[Incidence Axiom:]  For any pair of distinct points $P$, $Q$, there is exactly one line $L$ such that $P \in L$ and $Q \in L$.  This line is called $\Line{PQ}$.

\item[Ruler Postulate:]  For each lne $L$, there is a function $f$ from $L$ to $\mathbb R$ which is one-to-one and onto (a bijection) and satisfies $|f(P)-f(Q)| = d(P,Q)$ for any $P,Q \in L$.
Such a function $f$ is called a coordinate function for the line $L$.

\item[Plane Separation Axiom:]  With each line $L$ we can associate two sets $H_1$ and $H_2$ (the sides of the line) such that $H_1 \cap H_2 = H_1 \cap L = H_2 \cap L = \emptyset$
and $H_1 \cup H_2 \cup L$ is the entire plane (the three sets are a partition of the plane) {\bf and} if $P,Q$ are both in $H_1$ then $\overline{PQ} \subseteq H_1$ and if $P,Q$ are both in $H_2$ then $\overline{PQ} \subseteq H_2$ [in these cases we say that $P$ and $Q$ are on the same side of the line] {\bf and} if $P$ is in $H_1$ and $Q$ is in $H_2$, then $\overline{PQ}$ meets $L$ [in this case we say that $P$ and $Q$ (or $Q$ and $P$) are on opposite sides of the line].  The sets $H_1$ and $H_2$ are called {\em half-planes\/}.


\end{description}

We give some definitions.

\begin{description}

\item[lies on:]  A point $P$ lies on a line $L$ iff $P \in L$.

\item[parallel:]  Lines $L,M$ are parallel iff there is no point which lies both on $L$ and on $M$.

\item[collinear:]  Points $A,B,C$ are collinear iff they are distinct and there is a line $L$ such that $A,B,C$ all lie on $L$.

\item[betweenness:]  $A*B*C$ ($B$ is between $A$ and $C$) iff $A,B,C$ are collinear (and so distinct) and $d(A,B)+d(B,C) = d(A,C)$.

\item[segment:]  $\overline{AB}$ is defined as $\{P:P=A \vee P=B \vee A*P*B\}$, where $A,B$ are distinct points.

\item[ray:]  $\overrightarrow{AB}$ is defined as $\{P: P = A \vee P=B \vee A*P*B \vee A*B*P\}$, where $A,B$ are distinct points.

\item[congruence:]  $\overline{AB} \cong \overline{CD}$ is defined as holding iff $d(A,B)=d(C,D)$.


\end{description}

\end{document}

