\documentclass[12pt]{article}

\title{Math 275, Fall 2020 Practice Exam}

\author{Dr Holmes}

\begin{document}

\maketitle

This might be longer than your actual test;  the problems, mostly taken from old tests of mine, are similar to what you will see.

Where approximate calculations with your calculator are needed, please give answers to two decimal places.

\newpage

\begin{enumerate}

\item  Arrows representing two vectors {\bf a} and {\bf b} are shown.  Draw arrows representing {\bf a-b}, {\bf a+2b}, and $-\frac12 {\bf a}$, in each case drawing additional copies of {\bf a} and/or {\bf b} as appropriate to make it clear what you are doing.

Draw your own non-parallel vectors {\bf a} and {\bf b} on this practice paper.

\newpage

\item  Determine the scalar component of $\left<1,-2,1\right>$ in the direction of $\left<1,1,2\right>$, and determine the vector projection of 
$\left<1,-2,1\right>$ onto $\left<1,1,2\right>$.  Hint:  this will be a vector parallel to $\left<1,1,2\right>$!

For a couple of points, use your work above to express $\left<1,-2,1\right>$ as the sum of a vector parallel to $\left<1,1,2\right>$ and a vector
perpendicular to $\left<1,1,2\right>$.  There is an advantage to this:  it should give you an easy check of your work.

\newpage

\item  Determine the measure in degrees of the angle $\angle BAC$ (the angle at the vertex $A$) in the triangle with vertices $A=(1,1,1), B=(2,3,4),  C=(5,-3,1)$ using an appropriate vector operation.

Determine the area of the triangle using another appropriate vector operation (remember that it's a triangle).

\newpage

\item  Give a vector parametric equation for the line $AB$ in the previous problem.

Give an equation for the plane passing through the three points of the previous problem in the form $ax+by+cz=d$;  you might want to check that the three points satisfy the equation.
Hint:  you should have done more than half the  work needed for this already in the previous problem!

\newpage

\item  Write scalar parametric equations for the line $L_1$ with vector parametric equation $\left<1,1,1\right>+t\left<2,-1,3\right>$ and the line
$L_2$ with vector parametric equation $\left<4,-3,4\right>+s\left<1,2,3\right>$.

These two lines intersect.  Find the point of intersection (this does not mean, just find the appropriate values of $s$ and $t$!)

\newpage

\item  The point (1,1,1) lies on the plane $x+2y+3z=6$ and also on the plane $x-y+z=1$.  Find vector parametric equations for the line in which these planes intersect.

Determine the angle at which the two planes intersect.

\newpage

 \item Find the tangent vector to the curve parameterized by $\left<t^2,t^3,t^4\right>$ at the point $(4,-8,16)$.  (Hint:  what is the value of $t$ at this point?).

For a small bit of additional credit, give a vector parameterization of the tangent line to that curve at that point.

\newpage

\item  The acceleration vector of a particle at time $t$ is $\left<1,t\right>$.  Its velocity vector at time 0 is $\left<1,-1\right>$.  Its position at time 0 is $\left<2,3\right>$.  Find the position and velocity vectors for the particle at time $t$.
\newpage
\item  Set up and evaluate the integral for the length of the helix $${\bf r}(t) =\left<\cos(t),\sin(t),2t\right>$$ from the point (1,0,0)  to the point $(0,1,\pi)$.  If you set it up correctly, it will be very easy to evaluate!  Hint:  you need to identify the appropriate values of $t$ to serve as bounds of the integral.

\newpage

\item Write parametric equations for the tangent line to the  helix of the previous problem at the point  $(0,1,\pi)$.


\newpage
\item   Write parametric equations for a helix of radius 2 around the $z$-axis which rises 4 units in the course of three complete rotations.

\newpage






\end{enumerate}


\end{document}