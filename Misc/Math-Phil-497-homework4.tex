\documentclass[12pt]{article}

\title{Math/Phil 497 Homework 4}

\author{Randall Holmes}

\usepackage{amssymb}

\begin{document}

\maketitle

This problem set is motivated by the fancy proofs we have done in the class sessions of Week 8, but in no way requires full understanding of those proofs.  Doing these
exercises might help you to understand some details of these proofs.

\begin{enumerate}

\item  In the last line on page 11 of the Post notes (2),

``$\vdash (\sim p \vee q) \supset (q \vee \sim p) \supset ((p \supset (\sim p \vee q)) \supset (p \supset (q \vee \sim p)))$  rule II applied to variant 6 of axiom 5"

give the details of the application of rule II (explain what substitutions are made).  Of course you will need to find variant 6 of axiom 5 (on page 10).

\item  On page 11 of the Post notes, I make a remark about the substitution I describe (that I carry them out simultaneously rather than one after the other).  What would  go wrong if I carried out one of the substitutions, then the second, then the third?

``(2)  $[[q \supset r] \supset ([p \supset q] \supset  [p \supset r])] \supset ([p \supset q] \supset  ([q \supset r] \supset [p \supset r]))$  variant 8 of axiom IV with rule II
applied, replacing $p$ with $q \supset r$, $q$ with $p \supset q$, and $r$ with $p \supset r$ (note that these substitutions are being carried out simultaneously, not one after the other)."

I'll award extra credit if you can write down the expression which would result if you did carry out one, then the second, then the third, pointing out that it is different from the result I want.  This expression is rather long ;-)  I corrected the Post notes so that the variant of axiom 4 really is numbered {8} instead of (10).

\item Compute the rank of the expression $\neg q \vee (p \vee q)$.

Compute the rank of the expression $p \equiv q$ (this is a trick question:  you have to expand the definition into terms of just $\sim$ and $\vee$:  the expression you get by doing this actually appears in Post's paper.

Don't just write a number:  give the analysis, with ranks for all the component terms.

Rank is defined in the Post notes on pp. 13 and 14.

\item  Give a proof of the rule of hypothesis reordering described on page 14 of the Post notes, from Post's axioms and rules (and lines cited from my notes).  You may use my proof of transitivity of implication as a model for what a proof of a rule looks like,
but note that this proof is easier, as it is really a retread of variant (8) of axiom (4).

For additional credit you can give the very similar derivation of the rule of hypothesis collapse from Lemma 2 (you may cite Lemma 2 as a line:  you do not need to repeat its verification!)


\item  Suppose that $f(p_1,p_2,p_3)$ is given as $\sim (\sim(p_1 \vee p_2)) \vee p_3)$.

In a simple algebraic manner (you do not need to give derivations in Post's system:  this is more like Boolean algebra) show step by step how to transform
this term first to one in which negation is only applied to single letters, and then to one which is in disjunctive normal form (a disjunction of conjunctions of letters and negated letters)
and then to a disjunction of conjunctions of letters and negated letters in which each letter or its negation appears just once in each conjunction.

I want to see step by step calculations using the kinds of steps I mention in the notes on Post's proof (deMorgan laws, double negation, reordering and regrouping of conjunction
and/or disjunction, distributivity of logical multiplication over logical addition are examples:  this list is not quite exhaustive.  It is worth noting the theorem $p \equiv (p.q \vee p.\sim q)$ needed to add missing letters to conjunctions.

You could literally do this in Boolean algebra notation if you like;  you can also do it in propositional logic notation.

I'll do a different example similar to this one on request in lecture.

\end{enumerate}


\end{document}