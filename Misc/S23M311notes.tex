\documentclass[12pt]{article}

\usepackage{amsmath}
\usepackage{amssymb}

\newcommand\Line[1]{\overset{\leftrightarrow}{#1}}

\title{Geometry Notes, Spring 2023}

\author{Dr. Holmes}

\date{The version for Spring 2023 is starting out as a copy of the version of Spring 2019, but it won't stay that way.  These notes will not be relevant at the very beginning of the course.  Dates stated in them may be from 2019.}

\begin{document}


\maketitle

\tableofcontents
\vspace{.125 in}

This will be a place where I'll make additional remarks about the axiomatics of the geometry we are working on.  In particular, I will say things about alternatives to the Ruler Postulate here.

{\bf Students, please note that points can be earned by finding errors in this document.  Yes, proofreading counts.}

\section{The main development:  axioms and definitions as in Venema}

\subsection{Main development, section 3.2, notes mostly from February 5}

As of section 3.2 we have the primitive notions {\em point\/}, {\em line\/}, and {\em distance\/}.

We have some type information about these:  points are bare objects.  Lines are sets of points.  The distance $d(P,Q)$ between points $P,Q$ is a real number [Venema writes $PQ$ for the distance from $P$ to $Q$, which I do  not like.]



We have an implicit understanding of sets.  We assume that for any property of points we can state there is a set of points with exactly the points with that property as its elements.
We assume that sets with the same elements are equal.

We give some definitions.

\begin{description}

\item[lies on:]  A point $P$ lies on a line $L$ iff $P \in L$.

\item[plane:]  The plane is the set of all points.

\item[parallel:]  Lines $L,M$ are {\em parallel\/} iff there is no point $P$ which lies on both $L$ and $M$.

\item[external point:]  A point $P$ is an external point for a line $L$ iff $P \not\in L$.

\item[collinear:]  Three distinct points $A,B,C$ are collinear if there is a line $L$ on which all three points lie.


\end{description}

His axioms are 

\begin{description}

\item[Existence Axiom:]  There are at least two distinct points.

\item[Incidence Axiom:]  For any two distinct points $A,B$ there is a unique line $L$ such that $A \in L$ and $B \in L$.  We use the notation $\Line{AB}$ for this line $L$.

\item[Ruler Postulate:]  For any line $L$, there is a bijection (one-to-one and onto function) from $L$ to $\mathbb R$ with the property that for any $P,Q$ lying on $L$
we have $d(P,Q)=|f(P)-f(Q)|$.  We call such a function $f$ a {\em coordinate function\/} for $L$.  It should be noted that a line has many coordinate functions.

\end{description}

Some of the material that he puts in the axioms I put in the typing information about the primitive notions above.   I phrase the ruler postulate explicitly in terms of coordinate functions, which makes it easier to state compactly.

\begin{description}

\item[Theorem 3.1.7:]  Let $L$ and $M$ be two distinct, non-parallel lines.  Then there is a unique point $P$ such that $P \in L$ and $P\in  M$.

\item[Proof:]   Let $L$ and $M$ be arbitrarily chosen lines.  Assume that $L\neq M$ and that $L$ and $M$ are not parallel.  By the definition of {\em parallel\/}, if $L$ and $M$ are not parallel there must be at least one point $P$ which is on both $L$ and $M$.  


Suppose there was another point $Q$ such that $Q$ lay on $L$ and $Q$ lay on $M$.  Then by the uniqueness part of the Incidence Axiom we have $L=\Line{PQ}$ and $M = \Line{PQ}$, since $\Line{PQ}$ is th eonly line on which $P$ and $Q$ both lie.  But then $L=M$, a contradiction, so our assumption that there was a second point $Q$ on both $L$ and $M$ must be false.

So we have shown that there is one and only one point $P$ lying on both $L$ and $M$, for any distinct lines $L,M$ which are not parallel.

\end{description}

We prove some relatively easy consequences of the Ruler Postulate.

\begin{description}

\item[Theorem 3.2.7:]  For any points $P,Q$ (distinct or not) we have $d(P,Q)=d(Q,P)$, $d(P,Q) \geq 0$, $d(P,Q)=0 \leftrightarrow P=Q$ (so $d(P,P)=0$ and if $d(P,Q)=0$, it follows that $P=Q$).

\item[Proof:]  It is important to notice the importance of using the Incidence Axiom and even the Existence Axiom as well as the Ruler Postulate here.

If $P=Q$ it is obvious that $d(P,Q)=d(Q,P)$.  Suppose that $P$ and $Q$ are distinct.  In this case, $P$ and $Q$ both lie on the line $\Line{PQ}$.  Let $f$ be a coordinate function for $\Line{PQ}$.  Then $d(P,Q)= |f(P)-f(Q)| = |f(Q)-f(P)|$ (by properties of absolute value) $=d(Q,P)$.

If $P=Q$, let $X$ and $Y$ be two distinct points (which we can choose by the Existence Axiom).  Let $Z$ be defined as $Y$ if $P=Q=X$ and otherwise as $X$;  then $Z \neq P$.  Let $f$ be a coordinate function for the line $\Line{PZ}$:  then $d(P,Q) = |f(P)-f(Q)| = 0$ (recalling that $P=Q$).  If $P$ and $Q$ are distinct, $d(P,Q)=|f(P)-f(Q)| \geq 0$ (properties of absolute valuye).  In either case, we have shown $d(P,Q) \geq 0$.

We have just shown in the previous proof that $d(P,Q)=0$ if $P=Q$ (that is, $d(P,P)=0$).  Suppose that $P$ and $Q$ are distinct.  Then let $f$ be a coordinate function
for $\Line{PQ}$ and we have $d(P,Q)=|f(P)-f(Q)|$.  Because $f$ is one-to-one and $P\neq Q$, we have $f(P) \neq f(Q)$, so $d(P,Q) = |f(P)-f(Q)|>0$.  Thus it follows
that if $d(P,Q)=0$, it cannot be the case that $P$ and $Q$ are distinct.

\end{description}

The properties in Theorem 3,2,7 make $d$ a {\em semi-metric\/} on the plane, though Venema says just ``metric".  A metric would satisfy the Triangle Inequality,
$d(P,R) \leq d(P,Q)+d(Q,R)$, which we will be able to prove when we have more axioms.

We introduce definitions (found in section 3.2) for notions which we take as primitive in the next section.   We should also prove the axioms in the next section as theorems of the main develiopment, to show that the alternative approach in that section cannot prove anything that Venema cannot prove.

\begin{description}

\item[betweenness:]  If $P,Q,R$ are any points, we  say that $Q$ is between $P$ and $R$ (written $P * Q * R$ iff $P,Q,R$ are collinear (not that this implies that they are three distinct points) and $d(P,Q)+d(Q,R)=d(P,R)$.

\item[segment:]  The segment $\overline{AB}$ is defined as $\{A,B\} \cup \{P:A*P*B\}$.

\item[congruence:]  The relation $\overline{AB} \cong \overline{CD}$ is defined as holding iff $d(A,B)=d(C,D)$. 

\end{description}

There is a question as to the validity of the definition of congruence.  Can we be sure that if $\overline{AB} = \overline{A'B'}$, then $d(A,B)=d(A'B')$?  If this were not true, then the congruence or non-congruence of two segments might depend on exactly how we represented them, and so would not be a genuine relation between the segments.

We resolve this question with a 

\begin{description}

\item[Theorem:]  If $\overline{AB} = \overline{CD}$ then either $A=C$ and $B=D$ or $A=D$ and $B=C$.

\item[Proof:]  Suppose $\overline{AB} = \overline{CD}$.  Let $f$ be a coordinate function for $\Line{AB}=\Line{CD}$.  For any point $X$ in $\overline{AB}$ we have
$d(A,X) + d(X,B) = d(A,B)$ (this is clear when $A*X*B$;  it is easy to see that it is also true if $X=A$ or $X=B$).  This means that we have $|f(A)-f(X)| + |f(X)-f(B)| = |f(A)-f(B)|$.
This is clearly true if $f(A) \leq f(X) \leq f(B)$ or $f(A) \geq f(X) \geq f(B)$.  Straightforward calculations show that if $f(X)$ is not in the closed interval with endpoints
$f(A)$ and $f(B)$ then this inequality cannot hold.  Now if either $f(C)$ or $f(D)$ were strictly between $f(A)$ and $f(B)$, we could show that both $f(A)$ and $f(B)$ had to be strictly between $f(C)$ and $f(D)$ (because it is equally true that any $f(X)$ for $X$ in the segment must be in the interval with endpoints $f(C)$ and $f(D)$).  Thus we either have 
$f(A)=f(C)$ and $f(B)=f(D)$ or $f(A)=f(D)$ and $f(B)=f(C)$, so either $A=C$ and $B=D$ or $A=D$ and $B=C$.

\end{description}



\subsection{Main Development, the Ruler Placement Postulate}

Venema has the following 

\begin{description}

\item[Theorem (Ruler Placement Postulate):]
For each pair of distinct points $P$ and $Q$, there is a unique coordinate function $f$ for the line $\Line{PQ}$ such that $f(P)=0$ and
$f(Q)>0$.

\end{description}

Let $P$ and $Q$ be arbitrarily chosen distinct points.

Let $g$ be an arbitrarily chosen coordinate function for $\Line{PQ}$.

We cannot have $g(P)=g(Q)$, because $g$ is one-to-one, so we either have $g(P)>g(Q)$ or $g(Q)>g(P)$.

In the case where $g(Q)>g(P)$, we define $f(R)$ for any point $R \in \Line{PQ}$ as $g(R)-g(P)$.   We immediately have $f(P)=0$ and $f(Q)>0$. 

$f$ is one-to-one:  if $f(R)=f(S)$  then $g(R)-g(P)=g(S)-g(P)$ so $g(R)=g(S)$ (algebra) so $R=S$ ($g$ is one-to-one).

$f$ is onto:  choose a real number $r$.  We want to find $R$ so that $f(R)=r$, that is $g(R)-g(P)=r$, that is $g(R) = r+g(P)$.  There is such an $R$
because $g$ is onto.

$f$ has the correct relation to distance:  $|f(R)-f(S)| = |(g(R)-g(P))-(g(S)-g(P))| = |g(R)-g(S)| = d(R,S)$.  The last equation appeals to the fact that $g$ is a coordinate function.

In the case where $g(P)>g(Q)$, we define $f(R)$ as $g(P)-g(R)$.  We leave it as an exercise (*) for the reader to fill in the very similar proof that this gives a coordinate function with the desired properties.  You will need to use the fact that $|x-y|=|y-x|$.

This much I did in class.   We need to prove that there is exactly one coordinate function satisfying these conditions.  I'm going to prove this by proving a lemma.

\begin{description}

\item[Lemma:]  Let $f$ and $g$ be coordinate functions for a line $L$.  Let $P,Q \in L$ be distinct points.  Suppose $f(P)= g(P)$ and
$f(Q)=g(Q)$.   Then $f=g$.

\item[Proof of Lemma:] It is sufficient to show that for any $R \in L$, $f(R)=g(R)$.

We can assume without loss of generality that $f(P) < f(Q)$ (otherwise, we could switch the identities of $P$ and $Q$).

There are then three cases, $f(P) < f(Q) <f(R)$, $f(P) < f(R) <f(Q)$, and $f(R) < f(P) <f(Q)$.

If $f(P)<f(Q)<f(R)$, notice that $d(Q,R) = |f(Q)-f(R)| = f(R)-f(Q)= (f(R)-f(P)) - (f(Q)-f(P)) = |f(R)-f(P)| - |f(Q)-f(P)| = d(P,R)-d(P,Q))$.

If $f(P) < f(Q) <f(R)$, we have $d(P,R) = |f(P)-f(R)| = f(R)-f(P)$, and so $f(R) = f(P)+d(P,R)$. 

Now we attempt to compute $g(R)$.  Either we have $g(R)>g(P)$ or $g(R)<g(P)$.  If we have $g(R)>g(P)$, we determine that $g(R) = g(P) + d(P,R) = f(P)+d(P,R)$ exactly as above, so we have $g(R)=f(R)$.

If we have $g(R)<g(P)$, then we have $g(R) < g(P) < g(Q)$, so $d(Q,R) = |g(Q)-g(R)| = g(Q)-g(R) = (g(Q)-g(P))+(g(P)-g(R)) = |g(Q)-g(P)|+|g(P)-g(R)| = d(P,R)+d(P,Q)$.

But we cannot have $d(P,R)+d(P,Q) = d(Q,R)=d(P,R)-d(P,Q)$, because $P,Q$ are distinct, so $d(P,Q)$ is nonzero.  So this case is impossible:
we must have $g(R)>g(P)$, and we have shown in that case that $g(R)=f(R)$.

The proofs for the other two cases, $f(P) < f(R) <f(Q)$, and $f(R) < f(P) <f(Q)$, are similar.  We leave it as an exercise (**) for the reader to write them out.

\end{description}

The Lemma does the trick of proving the RPP because if we have two coordinate functions satisfying the RPP, they have the same values
0 and $d(P,Q)$ at $P$ and $Q$ respectively:  this is true because if $f$ satisfies the conditions required by the RPP, $d(P,Q) = |f(P)-f(Q)| = |0-f(Q)| = |f(Q)| = f(Q)$, recalling at crucial points that $f(P)=0$ and $f(Q)>0$ are in the conditions required by the RPP.  




\subsection{Main development, further work in section 3.2}

\begin{description}

\item[Betweenness theorem for points (3.2.17):]  If $A,B,C$ are collinear, and $f$ is a coordinate function for $\Line{AB}$ then $A*B*C$ holds iff either $f(A)<f(B)<f(C)$ or $f(C)<f(B)<f(A)$.

\item[Proof:]  Note that this is a biconditional statement, so proofs are needed in both directions.

Let $A,B,C$ be arbitrarily chosen collinear.

Suppose that $f(A)<f(B)<f(C)$ or $f(C)<f(B)<f(A)$.  Our aim is to show that $A*B*C$, that is, that $A,B,C$ and $d(A,C)=d(A,B)+d(B,C)$.

if $f(A)<f(B)<f(C)$, $d(A,B)+d(B,C)$ = $|f(A)-f(B)| + |f(B)-f(C)|$ = $(f(B)-f(A)) + (f(C)-f(B))$ = $f(C)-f(A)$ = $|f(C)-f(A)|$ = $d(C,A)$.

if $f(C)<f(B)<f(A)$, $d(A,B)+d(B,C)$ = $|f(A)-f(B)| + |f(B)-f(C)|$ = $(f(A)-f(B)) + (f(B)-f(C))$ = $f(A)-f(C)$ = $|f(C)-f(A)|$ = $d(C,A)$.

Note the use of the definition of a coordinate function to get to and from absolute differences, then the use of order facts to convert absolute differences to simple differences.

This establishes one direction. 

Now we assume that $A*B*C$ and our aim is to show  $f(A)<f(B)<f(C)$ or $f(C)<f(B)<f(A)$.  Our strategy is to exclude the other four alternatives (certainly
all of $f(A), f(B), f(C)$ are distinct, because $f$ is a bijection).

so we show that

\begin{enumerate}

\item $f(A)<f(C)<f(B)$ leads to contradiction:  Suppose $f(A)<f(C)<f(B)$.  Then $d(A,B)+d(B,C)$= $|f(A)-f(B)| + |f(B)-f(C)|$ = $(f(B)-f(A)) + (f(B)-f(C))$ = $(f(C)-f(A)) + 2(f(B)-f(C))$ = $d(A,C)+2d(B,C)>d(A,C)$  This shows
that $A*B*C$ is false, so this alternative is excluded.

\item $f(B)<f(C)<f(A)$ leads to contradiction:  As an exercise, write the other three subproofs.

\item $f(C)<f(A)<f(B)$ leads to contradiction:  left as an exercise.

\item $f(B)<f(A)<f(C)$ leads to contradiction:  left as an exercise.

\end{enumerate}

\item[Venema's proof of the second implication:]  Much slicker than mine!  He uses the lemma ``if $|x|+|y| = |x+y|$ then either $x$ and $y$ are both positive or $x$ and $y$ are both negative, which I will discuss just below.

His argument goes as follows:  $d(A,C) = d(A,B)+d(B,C)$ is known.  Thus $|f(A)-f(C)| =|f(A)-f(B)| + |f(B)-f(C)|$.  But then we have $|f(A)-f(B)+(f(B)-f(C))| = |f(A)-f(C)| =|f(A)-f(B)| + |f(B)-f(C)|$ so by the lemma to be proved just below we have $f(A)-f(B)$ and $f(B)-f(C)$ either both positive or both negative, so we have either $f(A)<f(B)<f(C)$ or $f(C)<f(B)<f(A)$.

\item[Venema's absolute value lemma (ex. 14 in section 3.2):]  If $|x| + |y| = |x+y|$ then either $x$ and $y$ are both positive or $x$ and $y$ are both negative.

Suppose $x$ and $y$ have different signs.  Without loss of generality we can suppose that $x$ is positive and $y$ is negative (because the situation is symmetrical).  We then have $|x+y| = ||x|-|y||$, which is equal either to
$|x|-|y|$ or to $|y|-|x|$, whichever is larger, but both are smaller than $|x|+|y|$. Notice that if $x$ were negative and $y$ were positive, we would have $|x+y| = ||y|-|x|| = ||x|-|y||$ and then the argument goes the same way.

A second very slick way to show this is $|x|+|y| = |x+y| \rightarrow (|x|+|y|)^2 = (|x+y|)^2 \rightarrow x^2+2|x||y| +y^2 = x^2 + 2xy +y^2$, so $|x||y| = xy$, which is true iff $x$ and $y$ have the same sign.

Students had the problem with this lemma that they were trying to prove the much easier converse.

\end{description}

\subsection{Main development, proof of problems 12c and 13}

We begin by proving the lemma about absolute values that I was in need of.

\begin{description}

\item[Lemma:]  Suppose that $a<b$  and $x,y$ are nonzero real numbers, and that $|x-a|=|y-a|$ and $|x-b|=|y-b|$.  Then $x=y$.  In other words, a position on the number line is exactly determined by its distances from two fixed distinct points.

\item[Proof:]  We show that we can compute $x$ from the values $|x-a|$ and $|y-b|$.  This establishes that $y=x$ because it can be computed in the same way.

Our claim is that if $|x-a|+|b-a| = |x-b|$, then $x = a-|x-a|$, and otherwise $x = a+|x-a|$.  Notice if this is correct, it determines $x$ exactly using just $|x-a|$ and $|x-b|$.

We show that this result is correct in each of the two  possible situations, $x<a$, $a<x$.  ($x=a$ is direct).

In case 1, $x<a$, $|x-a|+|b-a| = (a-x)+(b-a) = B-x = |x-b|$, and $a-|x-a| = a - (a-x) = x$.

In case 2,  $a<x$, $|x-a|+|b-a| = (x-a)+(b-a) = x +b - 2a > x+b-2b = x-b = |x-b|$, and $a+|x-a| = a+(x-a)=x$.

In each case, the statement ``if $|x-a|+|b-a| = |x-b|$, then $x = a-|x-a|$, and otherwise $x = a+|x-a|$" is true.  This is enough to show that
if $|x-a|=|y-a|$ and $|x-b|=|y-b|$, we have $x=y$.


\item[Another Proof:]  Here is another argument for the preceding Lemma.  It is incredibly slick but somehow not very enlightening.  Recall to begin with that $|x|=|y|$ iff $x^2=y^2$.

Suppose that  $a \neq b$ and $|x-a|=|y-a|$ and $|x-b|=|y-b|$.

It follows that $(x-a)^2=(y-a)^2$ and $(x-b)^2=(y-b)^2$.

Thus $x^2 -2ax + a^2 = y^2-2ay+a^2$ and $x^2 -2bx + b^2 = y^2-2by+b^2$.

Thus $x^2-2ax = y^2-2ay$ and $x^2 -2bx =y^2-2by$.

Thius $x^2-y^2 = 2a(x-y)$ and $x^2-y^2 = 2b(x-y)$.

Thus $2a(x-y) = 2b(x-y)$, and since $a \neq b$, this is only possible if $x-y=0$, that is, $x=y$.

\item[A purely geometric statement of this result (suitable for the alternative approach):]  If $X$ and $Y$ are each collinear with $A$ and $B$ ($A$ and $B$ distinct) and $\overline{XA} \cong \overline{YA}$ and $\overline{XB} \cong \overline{YB}$, then $X=Y$.




\item[Problem 12c:]  If $f$ and $h$ are coordinate functions for line $L$, then there is a number $c$ such tht for every point $P \in L$, $h(P) = c+f(P)$ or $h(P) = c-f(P)$.

\item[Proof:]  There is a point $Q$ such that $f(Q) = 0$.  Let $c=h(Q)$.

There is a point $R$ such that $f(Q)=1$.  We know that $|h(Q)-h(R)| = d(P,Q) = |f(Q)-f(R)| = 1$, so $|h(R)-c|=1$, so there are two cases, $h(R)=c+1$ and $h(R)=c-1$.

We claim that if $h(R)=c+1$, we have $h(P) = c+f(P)$ for all $P$.

$|h(P)-h(Q)|$ and $|h(P)-h(R)|$ are $|h(P)-c|$ and $|h(P)-(c+1)|$, and are also respectively $d(P,Q)$ and $d(P,R)$.  Now $|(f(P)+c)-c| = |f(P)| = |f(P)-f(Q)| = d(P,Q)$ and $|(f(P)+c)-(c+1)|$ = $|f(P)-1|$ = $|f(P)-f(R)|$ = $d(P,R)$.
Thus by the lemma we have $f(P)+c = h(P)$,  because these two numbers have the same absolute differences from $c$ and $c+1$.

We claim that if $h(R)=c-1$, we have $h(P) = c-f(P)$ for all $P$

This can be filled in as an exercise.


\item[Lemma:]  For any $d>0$ and any point $P$ on a line $L$, there are exactly two points $Q$  on $L$ such that $d(P,Q)=d$, and they are at distance $2d$ from each other.  This lemma is essentially the Diameter Axiom which we propose in the alternative development.

Let $f$ be a coordinate function on $L$.  If $d(P,Q)= d$, then $|f(P)-f(Q)|=d$, that is, either $f(Q) = f(P)+d$ or $f(Q) = f(P)-d$.
Define $Q_1$ as $f^{-1}(f(P)+d)$ and $Q_2$ as $f^{-1}(f(P)-d)$.  We see that $d(P,Q_1)=d(P,Q_2)=d$, and these are the only points at distance $d$
from $P$, and also $d(Q_1,Q_2)=2d$.

\item[Problem 13:]  Suppose that $h$ is a function from a line $L$ to the reals such that $|h(P)-h(Q)| = d(P,Q)$ for every $P,Q \in L$.  Then $h$ is a coordinate function.

\item[Proof:]  The function $h$ will be seen to be a coordinate function if it is one to one and onto.

$h(P) = h(Q) \rightarrow |h(P)-h(Q)|=0 \rightarrow d(P,Q)=0 \rightarrow P=Q$, so $h$ is one to one.

It remains to show that $h$ is onto. 

Let $r$ be an arbitrarily chosen real number.  Our aim is to find a point $P$ such that $h(P)=r$.

We begin by choosing an arbitrary point $Q$ on $L$.  Let $h(Q)=s$.  If $r=s$, we are done.  Otherwise, we consider the two points $R_1$ and $R_2$ on $L$ at distance $|r-s|$ from $Q$.  The two values
$h(R_1)$ and $h(R_2)$ will be the values $s+|r-s|$ and $s-|r-s|$ (we don't know which is which), and again these two values will be $s+(r-s)=r$ and $s-(r-s) = 2s-r$  (again, we don't know which is which).
So one of $h(R_1)$ and $h(R_2)$ is equal to $r$, so $h$ is onto.

This completes the proof that $h$ is a coordinate function.




\end{description}

\subsection{Plane separation (both approaches), including the Crossbar Theorem}  Here we introduce a new primitive idea, the {\em half-plane\/}.  This new notion is governed by a new axiom.

\begin{description}

\item[Plane Separation Postulate:]  For each line $L$, there are two sets $H^L_1$ and $H^L_2$, called the half planes determined by $L$,  such that 

\begin{enumerate}

\item Each point in the plane is in exactly one of the sets $L, H^L_1$, and $H^L_2$.

\item For any points $A,B$ not on $L$, the segment $\overline{AB}$ meets $L$ if and only if either $A \in H^L_1$ and $B \in H^L_2$ or vice versa (that is, if and only if
$A$ and $B$ lie in different half-planes determined by $L$).

\end{enumerate}

\end{description}

Notice that this implies that if $A$ and $B$ are both in $H^L_1$ that $\overline{AB}$ is a subset of $H^L_1$.  Since $A$ and $B$ are both in the same half-plane determined by $L$,
$\overline{AB}$ cannot meet $L$.  Suppoe some $C \in \overline{AB}$ was in $H^L_2$:  it would follow that $\overline{AC}$, a subset of $\overline{AB}$, would have to meet $L$,
and we have already seen that this is impossible.  So $\overline{AB}$ is a subset of $H^L_1$.  The same is true of $H^L_2$.

\begin{description}

\item[Definition:]  A subset $S$ of the plane is {\em convex\/} iff for any $A,B \in S$ we have $\overline{AB} \subseteq S$.  We have just shown that half-planes are convex.

\item[Definition:]  We say that points $A,B$ external to a line $L$ are on the same side of $L$ iff $A$ and $B$ both belong to $H^L_1$  or $A$ and $B$ both belong to $H^L_2$.
Otherwise we say that $A,B$ (still external to $L$!) are on opposite sides of $L$.  The PSP says that $\overline{AB}$ meets $L$ iff $A,B$ are on opposite sides of $L$.

\end{description}

We develop some interesting consequences of plane separation.

\begin{description}

\item[Theorem (Ray theorem):]  If $L$ is a line and $A$ is on $L$ and $P$ is not on $L$, then every point $Q \neq A$ on $\overrightarrow{AP}$ is on the same side of $L$ as $P$.

\item[Proof:]  Suppose otherwise.  $Q$ is not on $L$, because this would force $P$ on $L$ by incidence.  Suppose for the sake of a contradiction that $P$ and $Q$ are on opposite sides of $L$.  Then $\overline{PQ}$ meets $L$, and it can only meet $L$ at $A$, so $P*A*Q$.  But $Q \in \overrightarrow{AP}$ and $Q \neq A$ implies that we must have
$Q=P$, or $A*P*Q$, or $A*Q*P$, each of which is incompatible with $P*A*Q$.  So $P$ and $Q$ must be on the same side of $L$.

\end{description}

We define an important geometric idea.

\begin{description}

\item[Definition (angle):]  An {\em angle\/} is a set of the form $\overrightarrow{AB} \cup \overrightarrow{AC}$, where $A,B,C$ are noncollinear.  We write this $\angle BAC$.  Venema's definition is slightly different, as he allows zero angles but (like us) forbids straight angles.  We think the approach which disallows both is simpler, and moreover we seem to recall that Euclid does not allow zero angles either.

\item[Definition (interior of an angle):]  A point $D$ is said to be in the interior of $\angle BAC$ iff $D$ is on the same side of $\Line{AB}$ as $C$ and is on the same side of $\Line{AC}$ as $B$.  We think that this is a beautiful definition.

\item[Definition (betweenness for rays):]  We say that $\overrightarrow{AD}$ is between $\overrightarrow{AB}$ and $\overrightarrow{AC}$ iff $D$ is in the interior of $\angle BAC$.

\item[Question:]  Why does this definition not depend on the choice of the point $D$ in the ray $\overrightarrow{AD}$?  A different point might determine the same ray:  why would it have the same property?

\end{description}

We prove a theorem relating betweenness of points and betweenness of rays sensibly.  Venema does not give this theorem a name.

\begin{description}

\item[Theorem 3.3.10:]  Let $A,B,C$ be noncollinear points and let $D$ be on $\Line{BC}$.  Then $\overrightarrow{AD}$ is between $\overrightarrow{AB}$ and $\overrightarrow{AC}$ if and only if $D$ is between $B$ and $C$.

\item[Proof:]  The proof has two parts.

Assume $D$ is between $B$ and $C$:  we need to show that $\overrightarrow{AD}$ is between $\overrightarrow{AB}$ and $\overrightarrow{AC}$, that is, we need to show that $D$ is in the interior of $\angle BAC$, that is, we need to show that $D$ is on the same side of $\Line{AB}$ as $C$ and $D$ is on the same side of $\Line{AC}$ as $B$.  We just had a cascade of definitions.  $B$ is not on $\Line{AC}$ and $D$ is on $\overrightarrow{CB}$, so $B$ and $D$ are on the same side of $\Line{AC}$ by the Ray Theorem.   $C$ is not on $\Line{AB}$ and $D$ is on $\overrightarrow{BC}$, so $C$ and $D$ are on the same side of $\Line{AB}$ by the Ray Theorem.   This is what we needed to show:   $\overrightarrow{AD}$ is between $\overrightarrow{AB}$ and $\overrightarrow{AC}$.

Now assume that  $\overrightarrow{AD}$ is between $\overrightarrow{AB}$ and $\overrightarrow{AC}$.  We want to show that $D$ is between $B$ and $C$.  Since
$D$ is on the same side of $\Line{AB}$ as $C$, $\overline{CD}$ does not meet $\Line{AB}$, so we cannot have $C*B*D$.  Since
$D$ is on the same side of $\Line{AC}$ as $B$, $\overline{BD}$ does not meet $\Line{AC}$, so we cannot have $B*C*D$.  $B,C,D$ are three distinct points on the same
line:  one of them must be between the other two.  The only possibility left is $B*D*C$ (remember that there are three possibilities, not six:  $X*Y*Z$ says exactly the same thing as $Z*Y*X$).

This completes the proof of the theorem.

\end{description}

The following is often taken as an axiom missed by Euclid (thus its name) but we can prove it from PSP:

\begin{description}

\item[Pasch's Axiom:]  Let $\triangle ABC$ be a triangle and let $L$ be a line such that none of $A,B,C$ lies on $L$.
If $L$ meets $\overline{AB}$ then $L$ meets either $\overline{AC}$ or $\overline{BC}$.  (note that $L$ cannot meet all three sides because this would make $A,B,C$ collinear).

\item[Proof:]  Because $L$ meets $\overline{AB}$, $A,B$ are on opposite sides of $L$.  $C$ is either on the same side of $L$ as $A$, and so on the opposite side of $L$ from $B$, in which case $L$ meets $\overline{BC}$, or $C$ is on the same side of $L$ as $B$, and so on the opposite side of $L$ from $A$, in which case $L$ meets $\overline{AC}$

\end{description}

Pasch's Axiom and the Ray Theorem are good examples of theorems which are important not because they have hard proofs (they are both fairly easy to prove) but because they are constantly useful.

I am going to ignore the order of the text and prove the Crossbar Theorem here in the PSP section, because it in fact does not depend on later axioms.

\begin{description}

\item[Lemma (Z-theorem):]  Let $A$ and $B$ be distinct points on line $L$ and let $C,D$ be on opposite sides of $L$.
Then no point lies on both $\overrightarrow{AC}$ and $\overrightarrow{BD}$.

\item[Proof:]  Suppose a point $E$ lay both on $\overrightarrow{AC}$ and $\overrightarrow{BD}$.

If $E$ were on $L$ it would have to be $A$, since $A$ is the only point on $\overrightarrow{AC}$ which is on $L$,
and it would have to be $B$, because $B$ is the only point on $\overrightarrow{BD}$ which is on $L$, and this is absurd, because $A \neq B$.

If $E$ is not on $L$, the ray theorem tells us (because it is on $\overrightarrow{AC}$ and not equal to $A$) that it is on the same side of $L$ as $C$.  Also the ray theorem tells us (because it is on $\overrightarrow{BD}$ and not equal to $B$) that it is on the same side of $L$ as $D$.  But this is absurd, because $C$ and $D$ are on opposite sides of $L$.

The proof is complete.

\item[Crossbar Theorem:]  If $\triangle{ABC}$ is a triangle and $D$ is a point in the interior of $\angle BAC$, then there is a point $G$ which lies on both $\overrightarrow{AD}$ and $\overline{BC}$.

\item  This is utterly obvious visually and rather tricky to prove.  Venema says that it is quite reasonable for readers of the book to take it as an axiom and skip ahead.  But we love proofs (don't we?  of course we do!).

I'm going to insert my observation from my approach in class that if we know that $\overline{AD}$ intersects $\Line{BC}$ at a point $G$, then we know that $G$ is on $\overline{BC}$ by Theorem 3.3.10.  But there doesn't seem to be an easy way to show that this ray intersects this line at all.

Here is the actual proof.

Choose a point $E$ such that $E*A*B$ and a point $F$ such that $F*A*D$.

Let's give line $\Line{AD}$ the name $L$.  Line $L$ is distinct from $\Line{AB}$, so neither $B$ nor $E$ is on $L$, so Pasch's Axiom tells us that $L$ meets either $\overline{BC}$ (in which case we would need to show further that $\overrightarrow{AF}$, the ray opposite to $\overrightarrow{AD}$ in $L$, does not meet $\overline{BC}$ to show that $\overrightarrow{AD}$ must meet $\overline{BC}$) or line $\overline{EC}$.

$L = \overrightarrow{AF} \cup \overrightarrow{AD}$ meets $\overline{EC} \cup \overline{BC}$.  There are four possible ways this can happen.  We rule out three of them using applications of the Z-theorem;  the fourth, the only remaining possibility, is the conclusion of the theorem which is thus proved.

\begin{description}

\item[$\overline{AF}$ meets $\overline{EC}$:]  $F$ and $D$ are on opposite sides of line $\Line{AB}$.  $C$ and $D$ are on the same side of $\Line{AB}$, because of the angle interior fact about $D$.  Thus $F$ and $C$ are on opposite sides of $\Line{AB}$ and this case is impossible because $\overrightarrow{AF}$ and $\overrightarrow{EC}$ do not meet by the Z-theorem.

\item[$\overline{AF}$ meets $\overline{BC}$:]  $F$ and $D$ are on opposite sides of line $\Line{AC}$.  $B$ and $D$ are on the same side of $\Line{AC}$, because of the angle interior fact about $D$.  Thus $F$ and $B$ are on opposite sides of $\Line{AC}$ and this case is impossible because $\overrightarrow{AF}$ and $\overrightarrow{CB}$ do not meet by the Z-theorem.

\item[$\overline{AD}$ meets $\overline{EC}$:]  $E$ is on the opposite sice of $\Line{AC}$ from $B$.  $B$ and $D$ are on the same side of $\Line{AC}$ by the angle interior fact about $D$.  So $E$ and $D$ are on opposite sides of $\Line{AC}$, and this case is impossible because $\overrightarrow{AD}$ and $\overrightarrow{CE}$ do not meet by the Z-theorem.

\item[$\overline{AD}$ meets $\overline{BC}$:]  Since the other three possibilities have been shown to lead to absurdity, this alternative must hold, and this completes the proof of the Crossbar Theorem.
\end{description}

\end{description}

Again ignoring the order in the book, I am proving Lemma 3.4.4 (used in the Betweenness Theorem for Rays) in this section, because it does not use any axiom later than PSP.

\begin{description}

\item[Lemma 3.4.4:]  If $A,B,C,D$ are four distinct points with $C,D$ on the same side of $\Line{AB}$ and $D$ is not on $\overrightarrow{AC}$, then either $C$ is in the interior of $\angle BAD$ or $D$ is in the interior of $\angle BAC$.

\item[Proof:]  We will prove this by supposing that $C$ is not in the interior of $\angle BAD$ and arguing to the conclusion that $D$ is in the interior of $\angle BAC$.

Since $C$ is not in the interior of $\angle BAD$ and $C$ is on the same side of $\Line{AB}$ as $D$, it must be the case
that $C$ is on the opposite side of $\Line{AD}$ from $B$.  Thus $\Line{AD}$ intersects $\overline{BC}$ at a point $E$, which must be on the same side of $\Line{AB}$ as $C$ and $D$ (ray theorem because it is on $\overrightarrow{BC}$),
and ray $\overrightarrow{AE}=\overrightarrow{AD}$ is between rays $\overrightarrow{AB}$ and $\overrightarrow{AC}$ by Lemma 3.3.10,
that is, $D$ is in the interior of $\angle{BAC}$.

\end{description}

Here is a lemma used in the proof below of the Linear Pair Theorem, which again I place in this section because it depends purely on plane separation issues.

\begin{description}

\item[Lemma 3.5.7:]  If $C * A * B$ and $D$ is in the interior of $\angle BAE$, then $E$ is in the interior of
$\angle DAC$.

\item[Proof:]  Let $A,B,C,D,E$ be chosen so that $C * A * B$ and $D$ is in the interior of $\angle BAE$.

To show that $E$ is in the interior of
$\angle DAC$ we need to show two things:  (1) $E$ is on the same side of $\Line{AC}$ as $D$ [which we know already because $D$ is in the interior of $\angle BAE$, so $D$ is on the same side of line $\Line{AB} = \Line{AC}$ as $E$, as desired], and (2) $C$ is on the same side of line $\Line{AD}$ as $E$, which requires more work.

It is obvious that $C$ is on the opposite side of $\Line{AD}$ from $B$, since $C*A*B$ and $B,C$ are not on $\Line{AD}$.  Since $D$ is in the interior of $\angle{BAE}$, the Crossbar Theorem tells us that $\overrightarrow{AD}$ meets $\overline{BE}$ at a point $F$, which cannot be $B$ or $E$.  It thus follows that $B$ and $E$ are on opposite sides of line
$\Line{AD}$, so $C$ and $E$ are on the same side of $\Line{AD}$ [since they are both on the opposite side from $D$], which means we have shown (2) and completed the proof of the lemma.

\end{description}

\subsection{The Protractor Postulate and developments up to the Linear Pair Theorem}

We introduce the new primitive notion of angle measure:  if $\angle BAC$ is an angle then $\mu(\angle BAC)$ is a real number.

\begin{description}

\item[Protractor Postulate:]  We have three parts rather than four because we do not allow zero angles.

\begin{enumerate}

\item For any angle $\angle BAC$, $0<\mu(\angle BAC)<180$.  Recall that unlike Venema we do not allow zero angles, and like Venema we do not allow straight angles.

\item (Angle Construction Postulate):  For any real number $r \in (0,180)$, any line $\Line{AB}$ and half-plane $H$ determined by this line, there is a uniquely determined  ray $\overrightarrow{AC}$ with $C \in H$ and $\mu(\angle BAC)=r$.  Note that the point $C$ is {\bf not} uniquely determined, just the ray.

\item (Angle Addition Postulate):  If $D$ is in the interior of $\angle BAC$, then $\mu(\angle DAB) + \mu(\angle DAC) = \mu(\angle BAC)$.

\end{enumerate}

Where I didn't say things the same way as Venema, it is worthwhile to work out why I have said the same thing, or what feature of my approach caused me to say something different or omit something he said.  There is a section below in the alternative approach explaining how to do this without numbers.



\item[Betweenness theorem for rays (3.4.5):]  Let $A,B,C,D$ be four points with $C,D$ on the same side of
$\Line{AB}$.  Then $\mu(\angle BAD) < \mu(\angle BAC)$ if and only if $\overrightarrow {AD}$ is between
$\overrightarrow{AB}$ and $\overrightarrow{AC}$.

\item[Proof:]  This is a proof of a biconditional, so it falls into two parts.

If $\overrightarrow {AD}$ is between
$\overrightarrow{AB}$ and $\overrightarrow{AC}$ then by definition $D$ is in the interior of $\angle BAC$ so
we have by the Angle Addition Postulate that $\mu(\angle BAD) + \mu(\angle CAD) = \mu(\angle BAC)$, so, 
since $\mu(\angle CAD)$ is positive, $\mu(\angle BAD) < \mu(\angle BAC)$.

Now we take the implication in the other direction, but we prove the contrapositive.  Suppose that $\overrightarrow {AD}$ is not between
$\overrightarrow{AB}$ and $\overrightarrow{AC}$;  our aim is to prove that $\mu(\angle BAD) \geq \mu(\angle BAC)$
$D$ is either on $\Line{AC}$ or it is on the opposite side of $\Line{AC}$ from $B$ (otherwise (PSP) it would be on the same side of $\Line{AC}$ and we would have the betweenness fact we are denying).

If $D$ is on $\Line{AC}$, then $\mu(\angle BAD) = \mu(\angle BAC)$ and we have our desired conclusion.

If $D$ is on the opposite side of $\Line{AC}$ from $B$, then $C$ is in the interior of $\angle BAD$ (lemma 3.4.4), and we argue
exactly as in the first part that $\mu(\angle BAC) < \mu(\angle BAD)$, so we are done:  in either case in the second part we have $\mu(\angle BAD) \geq \mu(\angle BAC)$.


\end{description}

Now we develop the proof of the Linear Pair Theorem, which in effect is an extension of the Angle Addition Postulate to the case of ``straight angles".

\begin{description}

\item[Definition (linear pair):]  A pair of angles $\angle{BAD}$ and $\angle{CAD}$ is a {\em linear pair\/} iff $C*A*B$.

\item[Linear Pair Theorem:]  If $\angle{BAD}$ and $\angle{CAD}$ make up a linear pair, then $\mu(\angle{BAD}) + \mu(\angle{CAD})= 180$.

\item[Proof of the Linear Pair Theorem:]  Let $A,B,C,D$ be chosen so that $\angle{BAD}$ and $\angle{CAD}$ make up a linear pair.

Define $\alpha$ as $\mu(\angle{BAD})$ and $\beta$ be defined as $\mu(\angle{CAD})$.       

There are three cases, by the trichotomy axiom of order on the real numbers.  Either $\alpha+\beta<180$, or $\alpha+\beta>180$, or $\alpha+\beta=180$.  We will show that the first two cases leas to absurdity, establishing that the last one, the desired conclusion of the theorem, must be true.

\begin{description}

\item[Case 1 ($\alpha+\beta<180$):]  Since $\alpha+\beta<180$, we can use the angle construction postulate to find a unique $E$ on the same side of $\Line{ABC}$ as $D$ with
$\mu(\angle{BAE})=\alpha+\beta$.

By the Betweenness Theorem for Rays, $\overrightarrow{AD}$ is between $\overrightarrow{AB}$ and $\overrightarrow{AE}$, because the measure $\alpha+\beta$ of $\angle BAE$ is greater than the measure $\alpha$ of $\angle BAD$, that is, $D$ is in the interior of $\angle EAB$.  It follows by Lemma 3.5.7 that $E$ is in the interior of $\angle CAD$.   We know that $\mu(\angle {EAD}) + \mu(\angle {BAD}) [=\alpha] = \mu{\angle EAB} = \alpha+\beta$, so $\mu(\angle {EAD}) = \beta$.  But $E$  and $C$ are on the same side of $\Line{AD}$, because $E$ is in the interior of $\angle CAD$.  By the uniqueness clause of the ACP there can be only one ray $\overrightarrow{AF}$ where $F$ is on the same side of $\Line{AD}$ as $C$ and $E$ and which satisfies $\mu(\angle FAD)=\beta$.  But we have shown that these statements are true with both $C$ and $E$ replacing $F$, which is a contradiction:  clearly $\overrightarrow{AC}$ is not equal to $\overrightarrow{AE}$, because $A,B,C$ are collinear and $\angle EAB$ is an angle

\item [Case 2 ($\alpha+\beta>180$):]  Let $\overrightarrow{AE}$ be chosen so that $E$ is on the same side of $\overline{ABC}$ as $D$ and $\mu(\angle EAB) = \alpha+\beta-180$.
Notice that $\alpha+\beta-180 <\alpha+180-180 = \alpha$, so by the Betweenness Theorem for Rays, $E$ is in the interior of $\angle DAB$ (because $\angle {DAB}$ has larger measure than $\angle{EAB}$).  We use the AAP:  $\mu(\angle DAE ) + \mu(\angle EAB) = \mu(\angle DAB)$ so $\mu(\angle DAE) + \alpha+\beta -180 = \alpha$, so 
$\mu(\angle DAE) = 180-\beta$.  Now by Lemma 3.5.7, $\overrightarrow{AD}$ is between $\overrightarrow{AC}$ and $\overrightarrow{AE}$.  We then have by
AAP that $\mu(\angle CAD) + \mu(\angle DAE) = \mu(\angle CAE)$.  But this is absurd because $\mu(\angle CAD) + \mu(\angle DAE)=\beta + (180-\beta) =180$.

\item[Case 3: ($\alpha+\beta=180$):]  Since the other two cases are excluded, this case must be true and the theorem is proved.

\end{description}                                

\end{description}

\subsection{Side-Angle-Side Postulate and Isosceles Triangles}

We first define triangles and congruence of triangles.

\begin{description}

\item[Definition (triangle):]  If $A,B,C$ are noncollinear, $\triangle ABC$ is defined as $\overline{AB} \cup \overline{BC} \cup \overline{AC}$.

\item[Definition (triangle congruence):]  If $A,B,C$ are noncollinear and $A',B',C'$ are noncollinear, we say that $\triangle ABC \cong \triangle A'B'C'$ iff the following six things are true:

\begin{enumerate}

\item $\overline{AB} \cong \overline{A'B'}$

\item $\overline{AC} \cong \overline{A'C'}$

\item $\overline{BC} \cong \overline{B'C}'$

\item $\angle BAC \cong \angle B'A'C'$

\item $\angle ACB\cong \angle A'C'B'$

\item $\angle CBA \cong \angle C'B'A'$

\end{enumerate}

Note that this is not really a statement about triangles but about six points, the vertices of two triangles with a fixed order on each.  To really make it a statement about triangles, we would have to have a different idea of what a triangle is, including a preferred order on the vertices.

\end{description}

We then have the last axiom of neutral geometry (only the Parallel Postulate remains!)

\begin{description}

\item[SAS Postulate:]  Let $A,B,C$ be noncollinear and let $A',B',C'$ be noncollinear.  Suppose that $\overline{AB} \cong \overline{A'B'}$, $\overline{AC} \cong \overline{A'C'}$, and $\angle BAC \cong \angle B'A'C'$.
It follows that $\triangle ABC \cong \triangle A'B'C'$.

\end{description}

As a first application, we discuss the well-known theorem that the base angles of an isosceles triangle are equal.

\begin{description}


\item[Definition:]  A triangle is isosceles iff at least two of its sides are congruent to one another.

\item[Definition:]  A pair of angles in a triangle are a pair of base angles  if the third angle in the triangle is between two congruent sides of the triangle.

\item[Theorem:]  If $\triangle ABC$ is a triangle and $\overline{AB} \cong \overline{AC}$, then $\angle ABC \cong \angle ACB$:  that is, any pair of base angles in an isosceles triangle are congruent.

\item[First Proof:]  This is said by Venema to be a common high school textbook proof.

Let $\triangle ABC$ be a isosceles triangle and let $\angle ABC$ and $\angle ACB$ be a pair of base angles for this triangle (that is, let $\overline {AB} \cong \overline {AC}$).

Let $\overrightarrow{AD}$ be an angle bisector of $\angle BAC$.  By the Crossbar Theorem, $\overrightarrow {AD}$ intersects $\overline{BC}$ at a point $M$.

Now observe that

\begin{enumerate}

\item $\overline{AB} \cong \overline{AC}$ initial hypothesis

\item $\overline {AM} \cong \overline{AM}$ congruence is an equivalence relation

\item $\angle MAB \cong \angle MAC$ $\overrightarrow{AM} = \overrightarrow{AD}$ is an angle bisector of $\angle BAC$.


\end{enumerate}

It follows by SAS that $\triangle ABM \cong \triangle ACM$, from which it further follows by the definition of triangle congruence that $\angle  ABC = \angle ABM \cong \angle ACM = \angle ACB$, and the theorem is proved.

\item[Second Proof:]  This is essentially  Euclid's proof in the Elements.

Let $\triangle ABC$ be isosceles with $\overline{AB} \cong \overline{AC}$:  our aim is to show the base angles $\angle ABC \cong \angle ACB$.

Choose $D$ such that $A*B*D$.

Choose $E$ such that $A*C*E$ and $\overline{BD} \cong \overline{CE}$.

Our first aim is to show $\triangle ABE \cong \triangle ACD$.

We have 

\begin{enumerate}

\item $\overline{AB} \cong \overline{AC}$ initial hypothesis

\item $\overline{AD} \cong \overline{AE}$ by segment addition:  $A*B*D$ and $A*C*E$ and $\overline{AB} \cong \overline{AC}$ by initial hypothesis and $\overline{BD} \cong \overline{CE}$ by construction, so $\overline{AD} \cong \overline{AE}$.

\item $\angle BAE \cong \angle CAD$ because they are the same angle and congruence of angles is an equivalence relation.

\end{enumerate}

so by SAS $\triangle ABE \cong \triangle ACD$.

Now we prove that $\triangle DBC \cong \triangle ECB$.

\begin{enumerate}

\item $\overline{DB} \cong \overline{EC}$ by construction.

\item $\overline{DC} \cong \overline{EB}$ because $\triangle ABE \cong \triangle ACD$.

\item $\angle BDC \cong \angle CEB$ because $\triangle ABE \cong \triangle ACD$ so $\angle CEB = \angle AEB \cong \angle ADC = \angle BDC$

\end{enumerate}

so by SAS  $\triangle DBC \cong \triangle ECB$.

Thus we have $\angle CBD \cong \angle BCE$.

By the Linear Pair Theorem, $\mu(\angle ABC) + \mu(\angle CBD) = 180$ and $\mu(\angle ACB) + \mu(\angle BCE) = 180$.

Then $\mu(\angle ABC) = 180- \mu(\angle CBD)  = 180 - \mu(\angle BCE) = \mu(\angle ACB)$, establishing the theorem.

I don't think the Linear Pair Theorem (supplements of congruent angles are congruent) is what Euclid used.  One can also argue
that $\angle ABE \cong \angle ACD$ by the first congruence, $\angle ECA \cong \angle DBC$ by the second congruence, and ``subtract equals from
equals" to get the result.

That was painful.

\item[Third Proof:]  This proof is shorter but confusing.  It is not a modern invention:  it was known to Pappus of Alexandria.

Let $\triangle ABC$ be isosceles with $\overline{AB} \cong \overline{AC}$:  our aim is to show the base angles $\angle ABC \cong \angle ACB$.

we have

\begin{enumerate}

\item $\overline{AB} \cong \overline{AC}$ by initial hypothesis

\item  $\overline{AC} \cong \overline{AB}$ by initial hypothesis

\item  $\angle BAC \cong \angle CAB$   because congruence of angles is an equivalence relation

\end{enumerate}

so by SAS $\triangle ABC \cong \triangle ACB$ so $\angle ABC \cong \angle ACB$ by the definition of triangle congruence.

In both of the other versions we go to great lengths and use additional assumptions not really needed to prove this theorem to avoid congruences between the same triangles ordered differently.

\end{description}

\subsection{Angle bisectors and perpendiculars}

We define angle bisectors and prove their existence and uniqueness.

\begin{description}

\item[Definition:]  An angle bisector of an angle $\angle BAC$ is a ray $\overrightarrow{AD}$ such that $D$ is in the interior of $\angle BAC$ and $\angle DAB \cong \angle DAC$.

\item[Theorem:]  An angle $\angle BAC$ has one and only one angle bisector.

\item[Proof of theorem:]  By the angle construction postulate (part of the Protractor Postulate) there is exactly one ray $\overrightarrow{AD}$ such that $\mu(\angle DAB) = \frac12(\mu(\angle BAC))$.  By the Betweenness Theorem for Rays (compare angle measures) $D$ is in the interior of $\angle BAC$.  Thus by the angle addition
postulate, $\mu(\angle DAC) + \mu(\angle DAB) = \mu(\angle BAC)$, so by subtraction $\mu(\angle DAB)$ is also $\frac 12(\mu(\angle BAC))$, so we have shown that
$\overrightarrow{AD}$ is in the interior of the original angle and the angles it forms with the two sides of the original angle are equal so we have shown existence of an angle bisector.

Now we need to show uniqueness.  Suppose that $\overrightarrow{AE}$ is an angle bisector for $\angle BAC$:  what we know from the definition is that $E$ is in the interior
of $\angle BAC$ and $\angle{EAB} \cong \angle{EAC}$.  Now by angle addition we know that $\mu(\angle EAB)) + \mu(\angle EAC) = \mu(\angle BAC)$, from which algebra
tells us that $\mu(\angle EAB) = \mu(\angle EAC) = \frac 12(\mu(\angle BAC))$.  Now we have that $E$ is on the same side of $\Line{AB}$ as $D$ because they are both in the interior of $\angle BAC$ and so on the same side of $\Line{AB}$ as $C$,  and the angles $\angle EAB$ and $\angle DAB$ have the same measure, so by the uniqueness part of the angle construction postulate, $\overrightarrow{AD} = \overrightarrow{AE}$, completing the proof that there is exactly one bisector of the original angle.

\end{description}

Now for some results about perpendiculars.

\begin{description}

\item[Definition:]  Lines $L$ and $M$ are perpendicular if they meet at $P$ and there are $Q$ on $L$ and $R$ on $M$ such that the measure of $\angle QPR$ is 90.

\item[Lemma:]  If $L$ and $M$ are perpendicualr lines and $L$ is their point of intersection, and $A$ on $L$ and $B$ on $M$ are distinct from $P$, then the measure of $\angle APB$ is 90.  On other words, all four of the angles formed by a pair of perpendicular lines are right angles.

\item[Proof of Lemma:]  By the definition of perpendicular lines, we can choose points $Q,R$ on $L,M$ respectively such that $\angle QPR$ is a right angle.

Either $\overrightarrow {PA} = \overrightarrow{ PQ}$ or not.

Suppose $\overrightarrow{ PA} = \overrightarrow {PQ}$.

Either $\overrightarrow {PB} = \overrightarrow {PR}$ or not.

If so then $\angle APB = \angle QPR$ and so is a right angle.

Otherwise we have $B*P*R$, so $\angle APB = \angle QPB$ and $\angle QPR$ form a linear pair, so by the Linear Pair Theorem $\mu(\angle QPB)+ \mu(\angle QPR) = 180$
so $\mu(\angle APB) = \mu(\angle QPB) = 90$ and we are done with the second subcase of the first case.

Suppose  $\overrightarrow {PA} \neq \overrightarrow {PQ}$.

Then we have $A*P*Q$.

Either $\overrightarrow {PB} = \overrightarrow {PR}$ or not.

If $\overrightarrow{ PB} = \overrightarrow{ PR}$ then we have $\angle APB = \angle APR$ and $\angle QPR$ forming a linear pair, and by the LPT the measures of these two angles add to 180, so the measure of $\angle APB = \angle APR$ is 90, completing the proof in this case.

Otherwise we have $B*P*R$.  We argue in this case that $\angle QPR$ and $\angle APR$ form a linear pair, so their measures add to 90, so the measure of
$\angle APR$ is 90, and further $\angle APR$ and $\angle APB$ form a linear pair, so their measures add to 90, so the measure of $\angle APB$ is 90 as well.

In each of the four possible situations, the result follows, so it always holds.

\item[Theorem:]  Let $L$ be a line and let $P$ be a point on $L$.  Then there is a unique line $M$ such that $P$ is on $M$ and $M$ is perpendicular to $L$.

\item[Proof of Theorem:]  Let $Q$ be a point on $L$ other than $P$.  Let $H$ be a side of $L$.  By the ACP there is a unique ray $\overrightarrow{PR}$ with $R \in H$ and $\angle QPR$ a right angle.  Note that $\Line{PR}$ has $P$ on it and is perpendicular to $L$.

Now suppose that line $N$ is perpendicular to $L$ and $P$ is on $N$.  Let $S$ be a point on $N$ on the same side of $L$ as $R$ (if we chose a point $T$ on $N$ other than $P$ and it turned out to be on the other side of $L$ from $R$, choosing $S$ so that $S*P*T$ would give $S$ on $N$ on the same side of $L$ as $R$:  so we certainly can do this).   By the previous lemma, $\angle SPQ$ is a right angle;  then by the uniqueness part of the ACP, we have $\overrightarrow{PR}=\overrightarrow{PS}$, from which it follows that
$M = \Line{PR} = \Line{PS} = N$, so there is exactly one perpendicular to $L$ through $P$.

\end{description}


\subsection{The Exterior Angle Theorem and existence and  uniqueness of perpendiculars dropped to a line}

We define exterior angles of a triangle and remote interior angles for an exterior angle of a triangle.

\begin{description}

\item[Definition:]  Let $\triangle ABC$ be a triangle.   Suppos $B*C*D$.  We say that $\angle ACD$ is an exterior angle of the triangle, and that $\angle BAC$ and $\angle ABC$ are the remote interior angles of the triangle for this exterior angle.

\item[Exterior Angle Theorem:]  For any triangle $\triangle ABC$ and $D$ with $B*C*D$, we have $\mu(\angle ACD) > \mu(\angle BAC)$ and $\mu(\angle ACD) > \mu(\angle ABC)$:  any exterior angle of a triangle is strictly greater in measure that its remote interior angles.

\item[Proof of the Exterior Angle Theorem:]

Let $E$ be the midpoint of $\overline{BC}$ (Midpoint Theorem).  Let $F$ be chosen so that $B*C*F$ and $\overline{BC} = \overline{CF}$ (Point Construction  Postulate).

$\angle AEB$ is congruent to $\angle CEF$ by the Vertical Angle Theorem (for which I expect you to be able to write your own proof using the LPT).

Now we have $\triangle AEB \cong \triangle  CEF$ by SAS, since $\overline{AE} \cong \overline {CE}$ and $\overline{EB} \cong \overline{EF}$ by the way they are constructed,
and $\angle AEB \cong \angle CEF$

Thus $\angle FCE \cong \angle BAE = \angle BAC$.

Now $E$ is on the same side of $\Line{BC} = \Line{CD}$ as $A$ by the Ray Theorem and $F$ is on the same side of the same line again by the Ray Theorem,
so $F$ is on the same side of $\Line{CD}$ as $A$.  $F$ is on the same side of $\Line{AC}$ as $D$ because both are on opposite sides of $\Line{AC}$ from $B$, because segments
$\overline{BF}$ and $\overline{BD}$ each meet that line (if either point were on $\Line{AC}$ we would be able to show that $B$ was on $\Line{AC}$ using I.P.; you might want to fill in the details).  Thus $F$ is in the interior of $\angle{ACD}$, so $\mu(\angle BAC) = \mu(\angle FCE) < \mu(\angle FCE) + \mu(\angle FCD)$, and $\mu(\angle FCE) + \mu(\angle FCD) = \mu(\angle ECD) = \mu(\angle ACD)$ by the angle addition postulate.

Now to show that $\angle{ABC}$ is smaller in measure than $\angle{ACD}$, choose $G$ so that $A*C*G$.  By the VAT, $\angle BCG \cong \angle ACD$.  Now use the same argument as above with $B, A, C, G$ replacing $A, B, C, D$ to show  that $\angle{ABC}$ is smaller in measure than $\angle{BCG}$, which has the same measure as $\angle ACD$.

I checked the letters in this argument against a diagram...can you draw it?

\end{description}

Now we improve our result above on perpendiculars.

\begin{description}

\item[Theorem:]  Let $P$ be a point and $L$ be a line.  There is exactly one line $M$ such that $P$ is on $M$ and $M$ is perpendicular to $L$.

\item[Proof of theorem:]  Either $P$ is on $L$ or not.  If $P$ is on $L$, we have already shown this above, so we assume that $P$ is not on $L$.

Let $Q$ be a point on $L$.  Let $R$ be another point on $L$.  If $\angle PQR$ has measure 90, we have already constructed a perpendicular $\Line{PQ}$ to $L$.

So suppose $\angle PQR$ does not have measure 90.  By the ACP, there is a unique ray $\overrightarrow{QT}$ with $T$ on the opposite side of $L$ from $P$ such that $\angle PQR \cong \angle TQR$.  By PCP, there is $S$ on $\overrightarrow {QT}$ with $\overline{QS} \cong \overline{QP}$.

The segment $\overline{PS}$ meets $L$ at some point $X$ (opposite sides!).  $Q$ is not $X$, as in this case we would have $\angle PQR$ and $\angle SQR$ a linear pair
and so adding to 180, and so both of measure 90 contrary to hypothesis.

We have $\triangle PQX \cong \triangle SQX$ by SAS: $\overline{PQ} \cong \overline{SQ}$, $\overline{QR} =\overline{QR}$.  We have only to show that
$\angle PQX \cong \angle SQX$.    We either have $X$ on $\overrightarrow{PR}$, in which case $\angle PQX = \angle PQR \cong \angle TQR = \angle SQR = \angle SQX$,
or we have $X*Q*R$ in which case $\angle PQX, \angle SQX$ form linear pairs with $\angle PQR$, $\angle SQR$ and so have the same measure and are congruent.

It then follows that $\angle QXP \cong \angle QXS$.  But these two angles make up a linear pair, so must have measures adding to 180, so each of them has measure 90,
and $\Line{PS}$ is perpendicular to $L$.

Now suppose that there were two distinct perpendiculars $M,N$ to $L$ with $P$ on $M$ and $P$ on $N$.  The points $Q$ at which $L$ and $M$ intersect and $R$ at which $L$ and $N$ intersect must be distinct (why?), and $\triangle PQR$ is a triangle.  Choose $S$ such that $Q*R*S$.  $\angle PRS$ and $\angle PQS$ are both right angles because they are angles
between perpendicular lines (lemma above).  But $\angle PQS$ is a remote interior angle for the exterior angle $\angle PRS$ of $\triangle PQR$, and both are supposed to have the same measure, 90, which contradicts the Exterior Angle Theorem.

Again, draw the picture to check my work.

\end{description}

\subsection{Triangle congruence conditions}

\begin{description}

\item[Angle-Side-Angle Theorem:]  If $\triangle ABC$ and $\triangle DEF$ are triangles and $\overline AB \cong \overline DE$ and $\angle ABC \cong \angle DEF$.
and $\angle BAC \cong \angle EDF$, then $\triangle ABC \cong \triangle DEF$.

\item[Proof:]  There is a point $C'$ on $\overrightarrow{AC}$ such that $AC' \cong DF$ (PCP).  By SAS, $\triangle ABC' \cong \triangle DEF$, so $\angle ABC' \cong \angle DEF$.
Since $\angle ABC \cong \angle DEF$ (hypothesis) and $C$ and $C'$ are on the same side of $\Line{AB}$ (Ray Theorem), we have $\overrightarrow{BC} = \overrightarrow{BC'}$ (ACP uniqueness).  But $\Line{AC}$ can intersect $\Line{AB}$ at only one point, so $C=C'$ and so $\triangle ABC = \triangle ABC' \cong \triangle DEF$.


\item[Converse to the Isosceles Triangle Theorem:]  If $\triangle ABC$ is a triangle and $\angle ABC \cong \angle ACB$, then $\overline {AB} \cong \overline{AC}$:  if a triangle contains a pair of congruent angles, then the triangle is isosceles and the pair of congruent angles is a pair of base angles.

\item[Proof:]  We show that $\triangle ABC$ is congruent to $\triangle ACB$ by ASA:  certainly $\overline{BC} \cong \overline{CB}$, and by hypothesis $\angle ABC \cong \angle ACB$
and equivalently $\angle ACB \cong \angle ABC$.  This is what is needed to show  $\triangle ABC$ is congruent to $\triangle ACB$ by ASA, and from the triangle congruence it
follows that $\overline{AB} \cong \overline{AC}$.

\item[Angle-Angle-Side Theorem:]

An exercise on your review sheet:  I'll prove it here before the final.

\item[Hypotenuse-Leg Theorem:]

An exercise on your review sheet:  I'll prove it here before the final.


\item[Theorem 4.2.6:] If $\triangle{ABC}$ is a triangle and $\overline{AB} \cong \overline{DE}$ and $H$ is a half plane determined by $\Line{AB}$, there is a unique point
$F$ such that $F \in H$ and $\triangle{ABC}\cong \triangle{DEF}$

\item[Proof (spoiler, this is on your review sheet):]  By the ACP there is one and only one ray $\overrightarrow{DG}$ such that $G \in H$ and $\angle BAC \cong \angle EDG$.  By the PCP there
is only one point $F$ on $\overrightarrow{DG}$ (in $H$ by the Ray Theorem)  such that $\overline{AC} \cong \overline{DF}$.  Then $\triangle{ABC}\cong \triangle{DEF}$ by SAS.

Now suppose that $K \in H$ and $\triangle{ABC} \cong \triangle DEK$.  We then have $\angle BAC \cong \angle EDK$, so by uniqueness part of ACP we have
$\overrightarrow{DK} = \overrightarrow{DG} = \overrightarrow{DF}$, and we have $\overline{DK} \cong \overline{AC}$, whence we have $K=C$ ($\overline{DK} \cong \overline{AC} \overline{DF}$, and $F$ and $K$ are both in $H$, so are on the same side of $\Line{AB}$, so $\overrightarrow{DF} = \overrightarrow{DK}$, and the equality of $F$ and $K$ follows from the uniqueness part of PCP).

\item[Side-Side-Side Theorem:]  If $\triangle ABC$ is a triangle and $\triangle{DEF}$ is a triangle and $\overline{AB} \cong \overline{DE}$ and $\overline{AC} \cong \overline{DF}$ and $\overline{BC} \cong \overline{EF}$, then $\triangle ABC \cong \triangle{DEF}$.

By theorem 4.2.6, there is a unique triangle $\triangle ABG$ with $G$ in the half plane determined by $\Line{AB}$ which does not contain $C$ (i.e., on the other side of $\Line{AB}$ from $C$) and which is congruent to $\triangle DEF$.

By PSP, the segment $\overline{CG}$ intersects $\Line{AB}$ at a unique point $H$.

We either have $H=A, H=B, H*A*B, A*H*B, A*B*H$.  The roles of $A,B$ are completely symmetric, so we really need to consider only three cases (and swap $A,B$ in the argument suitably to get the other two):

\begin{description}

\item[$H=A$:]  We have $\overline{CB} \cong \overline{FE} \cong \overline{GB}$ by triangle congruence facts, so we have $\angle ACB = \angle GCB \cong \angle CGB =  \angle AGB$ by the Isosceles Triangle Theorem, and we have $\overline{AC} \cong \overline{DF} \cong \overline{AG}$ by triangle congruence facts, so we have $\triangle ABC \cong \triangle{ABG}$ by SAS, so we have $\triangle ABC \cong \triangle DEF$ (because congruence of triangles is clearly transitive).

\item[$A*H*B$:]  $\overline{AC}\cong \overline{AG}$ and $\overline{CB} \cong \overline{GB}$ just as above.  By the Isosceles Triangle Theorem we then have
$\angle ACG \cong \angle AGC$ and $\angle GCB \cong \angle CGB$.   Because $H$ is in the interior of $\angle{ACB}$ (theorem 3.3.10) and so $G$ is as well (Ray Theorem), and for exactly the same reasons $G$ is in the interior of $\angle AGB$,  it follows by AAP that
$\mu(\angle ACB) = \mu(\angle ACG) + \mu(\angle GCB) = \mu(\angle AGC) + \mu(\angle CGB)  = \mu(\angle AGB)$, whence $\triangle{ABC} \cong \triangle{AGB} \cong \triangle DEF$ by SAS.

\item[$H*A*B$:]  Fill in this case yourself.  (Or look it up in the book).  Draw the picture and notice that the argument will be exactly the same as in the previous case
except that you are subtracting angles known to be congruent by isosceles triangles instead of adding them.


\end{description}



\end{description}


\section{The alternative development, without distance or the Ruler Postulate}

\subsection{Alternative approach, segment and congruence axioms, definitions and lemmas (to be discussed)}

The point of my alternative approach is that it is odd to import the whole theory of the real numbers into our geometry, and it is not necessary.

In my alternative approach I would eliminate {\em distance\/} as a primitive, and instead use {\em segments} and {\em congruence\/} as primitives:  for each pair of distinct points $A,B$ we have the segment $\overline{AB}$, a set of points, and congruence is a relation between segments, written $\overline{AB} \cong \overline{CD}$.

In my alternative approach we drop distance as a primitive notion and drop the Ruler Postulate and instead add the following axioms.

\begin{description}

\item[lies on:]  A point $P$ lies on the segment $\overline{AB}$ iff $P \in \overline{AB}$.

\item[betweenness:]  $C$ is between $A$ and $B$ (written $A*C*B$) iff $C$ lies on $\overline{AB}$ but is distinct from $A$ and $B$.

\item[Basic segment axiom:]  $\{A,B\} \subseteq \overline{AB} =\overline{BA} \subseteq \Line{AB}$.  Notice that $$A*B*C \leftrightarrow C*B*A$$ is a consequence.

\item[Trichotomy:]  For any distinct and collinear points $A,B,C$, exactly one of $A*B*C$, $A*C*B$, $B*A*C$ holds.

\item[Segment partition:]  For distinct $A,B,C$, $A*B*C$ iff both $\overline{AB} \cap \overline{BC} = \{B\}$ and $\overline{AB} \cup \overline{BC} = \overline{AC}$.

\item[Basic congruence axiom:]  Congruence is an equivalence relation.

\item[Point placement axiom:]  For any segments $\overline{AB}$ and $\overline{CD}$, there is exactly one point $E$ such that $\overline{AE} \cong \overline{CD}$ and $E*A*B$.

This is the reverse of the usual point placement postulate (which is proved as the second point placement lemma below) but it is cleaner to state and has the same effect.

\item[Segment addition axiom:]  If $A*C*B$ and $A'*C'*B'$ and $\overline{AC} \cong \overline{A'C'}$ and $\overline{CB} \cong \overline{C'B'}$ then $\overline{AB} \cong \overline{A'B'}$. 

\item[Midpoint lemma:]  For any points $A,B$. there is a point $C$ such that $A*C*B$ and $\overline{AC} \cong \overline{CB}$ (we will be able to prove this using further axioms we introduce later, but for the moment we treat it as an axiom).


\end{description}

Our segment and congruence axioms are not as strong as the Ruler Postulate, but they are more clearly geometrical in nature.

We prove some useful lemmas.

\begin{description}

\item[definition of ray:]  For distinct points $A,B$,the ray  $\overrightarrow{AB}$ is defined as $$\{P:P=A \vee A*P*B \wedge P=B \wedge A*B*P\}.$$  A point lies on the ray iff it is a member of the ray as a set.

\item[segment partition corollary:]  If $A*B*C$ then $\overline{AB} \subseteq \overline{AC}$ and $C \not\in \overline{AB}$.

\item[Proof:]  This follows directly from the fact that $\overline{AC} = \overline{AB} \cup \overline{BC}$ (so $\overline{AB} \subseteq \overline{AC}$) while
$C \in \overline{AB}$ is ruled out by the fact that $B$ is the only common member of $\overline{AB}$ and $\overline{AC}$.

\item[Comparability lemma:]  If $\overline{AB}$ and $\overline{AC}$ are both subsets of $\overline{AD}$, then one of $\overline{AB}$ and $\overline{AC}$ is a subset of the other.

\item[Proof:]  This is obvious if $B=C$, or if either $B$ or $C$ is equal to $D$, so we may suppose $B,C,D$ distinct.

We have $A*B*D$, so we have $\overline{AB} \cap \overline{BD} = \{B\}$ and $\overline{AB} \cup \overline{BD} = \overline{AD}$.  Thus $C$ lies
either on $\overline{AB}$, in which case $\overline{AC} \subseteq \overline{AB}$, and we are done, or on $\overline{BD}$.  Similarly $B$ lies on $\overline{AC}$, in which case we are done, or on $\overline{CD}$.  But the trichotomy axiom  forbids us to have both $C$ on $\overline{BD}$ and $B$ on $\overline{CD}$, so we must have one of the two cases which make our conclusion true.

\item[Second point placement lemma:]  If $\overline{AB}$ and $\overline{CD}$ are segments, there is a unique point $E$ lying  on $\overrightarrow{AB}$ such that $\overline{AE} \cong \overline{CD}$.  This is actually the usual point placement theorem!

\item[Proof:]  By the point placement axiom, there is a unique point $F$ such that $\overline{AF} \cong \overline{CD}$ and $F*A*B$.  By a second application of the
point placement axiom, there is a unique point $E$ such that $\overline{AE} \cong \overline{CD}$ and $E*A*F$.

We claim that $E$ is the point we are looking for.

We have that $\overline{AE} \cong \overline{CD}$ by construction.  We need to verify that $E$ lies on $\overrightarrow{AB}$ and that $E$ is the only point satisfying these two conditions.
If $E$ were not on  $\overrightarrow{AB}$, the definition of the ray and the trichotomy axiom  would ensure $E*A*B$, but then the point placement lemma would ensure $E=F$,
and $E*A*F$ ensures that $E$ is distinct from $F$.  So $E$ lies on $\overrightarrow{AB}$.

Now suppose that a point $G$ lay on $\overrightarrow{AB}$ and we had $\overline{AG} \cong \overline{CD}$.  Notice that if we can show $F*A*G$ we have shown $G=E$ by the uniqueness conditions of the point placement axiom.

Suppose that we don't have $F*A*G$ for the sake of a contradiction.  Then we must have either $F*G*A$ or $A*F*G$.

Suppose we have $F*G*A$.  We also have either $A*G*B$ or $B=G$ or $A*B*G$.  $B=G$ is impossible because $F*B*A$ is false.  $A*G*B$ is impossible because, since
$F*A*B$, $\overline{AF}$ and $\overline{AB}$ cannot share the point $G$ distinct from $F$.  $A*B*G$ is impossible because we would then have $\overline{AB} \subseteq \overline{AG} \subseteq \overline{AF}$, and so $A*B*F$, contradicting $F*A*B$.

Suppose we have $A*F*G$.  We also have either $A*G*B$ or $B=G$ or $A*B*G$.  $B=G$ is impossible because $A*F*B$ is false.  $A*G*B$ is impossible because  $\overline{AF} \subseteq \overline{AG} \subseteq \overline{AB}$ would follow, so we would have $A*F*B$, which is ruled out since we have $F*A*B$.   If we have $A*B*G$ then we have
$\overline{AF}$  and $\overline{AB}$ both subsets of $\overline{AG}$, so one of them is a subset of the other by the comparability lemma, so we either have
$A*F*B$ or $A*B*F$, both of which are impossible because $F*A*B$.

\end{description}

We prove a useful fact about rays.

\begin{description}

\item[ray lemma:]  If $A*B*C$ then $\overrightarrow{AB} = \overrightarrow{AC}$.

\item[proof of ray lemma:]  We show that any $D \in \overrightarrow{AB}$ lies on $\overrightarrow{AC}$

We either have $D=A$, for which the conclusion is obvious, or $D=B$, for which the conclusion follows from $A*B*C$, or $A*D*B$, in which case the conclusion follows
because $\overline{AD} \subseteq \overline{AB} \subseteq \overline{AC}$, so we have $A*D*C$, or we have $A*B*D$.  We are given $A*B*D$ and aim to show $A*C*D$ or $A*D*C$.  We show this by ruling out
the third alternative $D*A*C$:  $D*A*C$ and $A*B*D$ and $A*B*C$ lead to contradiction because $D*A*C$ rules out $\overline{AD}$ and $\overline{AC}$ having the common point $B$.

We show that any $D \in \overrightarrow{AC}$ lies on $\overrightarrow{AB}$

We must have $D=A$ (the conclusion follows), $D=C$ (the conclusion follows because $A*B*C$), $A*C*D$, from which $A*B*D$ follows because $\overline{AB} \subseteq \overline{AC} \subseteq \overline{AD}$ (so $A*B*D$ follows), or $A*D*C$.  In this last case, we want to deduce either $A*D*B$ or $A*B*D$ (or $B=D$, from which the conclusion follows immediately).  $A*B*C$ and $A*D*C$ imply by compatibility that one of $\overline{AD}$ and $\overline{AB}$ is a subset of the other, so $A*D*B$ or $A*B*D$ or $B=D$.


\item[second ray lemma:] if $B*A*C$ on $\Line{AD}$ then exactly one of $B,C$ lies on $\overrightarrow{AD}$.    


\item[proof of second ray lemma:]  We either have $D*A*B$ or $A*D*B$ or $A*B*D$.  In the second and third cases, we get $\overrightarrow{AB} = \overrightarrow{AD}$, from which $B$ is on $\overrightarrow{AD}$ and $C$ is not.   So we can suppose $D*A*B$ in what follows, because other cases have been handled.  It would be sufficient to show
$C*D*A$ or $D*C*A$ because either would show $\overrightarrow{AD}=\overrightarrow{AC}$ and we would proceed as above to show $C$ is on the ray of interest and $B$ is not.  So we need to rule out $D*A*C$.
Given $D*A*B$ and $D*A*C$ we deduce from the ray lemma that $\overrightarrow{DA} = \overrightarrow{DB} = \overrightarrow {DC}$.  Thus we have either $D*B*C$ or $D*C*B$.  But from the first we get $A*B*C$
and from the second we get $A*C*B$ (by seeing that relevant segments can only intersect at a single point), and this is a contradiction.

\end{description}

We prove a segment subtraction lemma to match our segment addition axiom.

\begin{description}

\item[segment subtraction lemma:]  Suppose that $A*B*C$ and $A'*B'*C'$ and $\overline{AC} \cong \overline{A'C'}$ and $\overline{AB} \cong \overline{A'B'}$.  Then $\overline{BC} \cong \overline{B'C'}$.  

\item[Proof:]  By the second point placement lemma, there is a uniquely determined point $C''$ lying on $\overrightarrow{B'C'}$ such
that $\overline{BC} \cong \overline {B'C''}$.  We have $A*B*C$, we claim that $A'*B'*C''$, and we have $\overline{AB} \cong \overline{A'B'}$ and $\overline{BC} \cong \overline {B'C''}$ so by the segment addition axiom we have $\overline{AC} \cong \overline{A'C''}$, and, if we know that $C''$ lies on $\overrightarrow{A'C'}$, it follows that $C'=C''$ and so
 $\overline{BC} \cong \overline {B'C''}=\overline{B'C'}$ as desired.

We verify that $A'*B'*C''$:  we have $A'*B'*C'$ and one of $B'*C''*C'$, $C'=C''$ (which requires no work), or $B'*C'*C''$.  In the first case, $B'*C''*C'$, since $\overline{A'B'}$ meets
$\overline{B'C'}$ in a single point $B'$, it also meets the subset $B'C''$ in that single point, so we have $A'*B'*C''$.  The second case is trivial.  The third case seems harder.
If we have $A'*C''*B'$ and $B'*C''*C'$, this contradicts the fact that $A'B'$ and $B'C'$ meet in only one point.  If we have $C''*A'*B'$ and $B'*C'*C''$ then we have one of $A'B'$ and
$B'C'$ a subset of the other by the comparability lemma, which is again impossible.  So we must have $A'*B'*C''$ in the third case as well.

We verify that $C''$ lies on $\overrightarrow{A'C'}$.  We have $C''*B'*A'$ so $C''$ lies on $\overrightarrow{A'B'}$, which is the same ray as $\overrightarrow{A'C'}$, because
$A'*B'*C'$ (ray lemma).

This is philosophically interesting as it raises the question of whether Euclid needs both an addition axiom and a subtraction axiom.  We don't, but we do make use of the point placement postulate to prove the subtraction lemma from the addition axiom.  If Euclid were to prove point placement, he might use subtraction!  In fact, in our analysis of Euclid's Prop II, in which we do not assume point placement because that would amount to assuming what we are trying to prove, we use subtraction as an independent postulate, in effect proving point placement using subtraction.

\end{description}





\subsection{Alternative Development, the Rational Plane, the Circle Crossing Axiom, and Euclid's Propositions I and II}

It is an embarrassing fact that the normal plane with points restricted to those with rational coordinates is a model of Euclid's explicitly stated axioms in which
Euclid's Proposition I is false and the argument he gives for Proposition II does not work (although Proposition II is nonetheless correct in the rational plane).

The axiom usually given to correct this is very sophisticated (a continuity axiom equivalent to the Least Upper Bound Axiom in real analysis).  We give a more elementary
additional axiom which fills in the gaps in the proofs of Euclid's first two propositions.

\begin{description}

\item[shorter, longer, closer, farther:]  We say that $\overline{AB}$ is shorter than $\overline{CD}$ iff the unique $E$ lying on $\overrightarrow{CD}$ such that $\overline{AB} \cong \overline{CE}$  satisfies $C*E*D$.   We say that  $\overline{AB}$ is longer than $\overline{CD}$ iff $\overline{CD}$ is shorter than $\overline{AB}$.  We say that $A$ is closer to $B$ than $C$ is to $D$ iff either $A=B$ and $C \neq D$ or if $\overline{AB}$ is shorter than $\overline{CD}$.  We say that $A$ is farther from $B$ than $C$ is from $D$ if $C$ is closer to $D$ than $A$ is to $B$.

\item[Lemmas:] There are common sense lemmas to be proved about these relations.

\item[Definition (circle):]  The circle with center at $A$ and radius $\overline{AB}$ is the set of all points $C$ such that $\overline{AB} \cong \overline{AC}$.  We say that a point lies on a circle iff it is an element of the circle {\em qua} set.

\item[Definition (interior, exterior):]  The interior of the circle with center at $A$ and radius $\overline{AB}$ is the set of all points $C$ such that $A$ is closer to $C$ than it is to $B$.  The exterior of the circle with center at $A$ and radius $B$ is the set of all points $C$ such that $A$ is farther from $C$ than it is from $B$.

\item[Circle Crossing Axiom:]  A circle  which contains points in both the interior and the exterior of a circle also contains at least one point on the circle.

\end{description}

We are going to use a weaker point placement axiom (relating lines and circles) and a strong addition and subtraction axiom.  The reason for this is that Euclid's Prop II
follows trivially from our point placement axiom;  proving Prop II will enable us to show that our point placement axiom follows from these weaker assumptions, which seem to be
closer to what Euclid is implicitly assuming in his argument.

\begin{description}

\item[Diameter Axiom:]  If a circle has center $A$ and $A$ lies on a line $L$ then there are exactly two points $B,C$ which lie on the circle and lie on the line, and $B*A*C$.

\item[Segment addition and subtraction axiom:]  If $A*B*C$ and $A'*B'*C'$ and any two of $\overline{AB} \cong \overline{A'B'}$, $\overline{AC} \cong \overline{A'C'}$, $\overline{BC} \cong \overline{B'C'}$ hold, the third one holds.


\item[Euclid, Proposition I:]  To construct an equilateral triangle with base a given segment $\overline{AB}$ (I think this is what he is actually saying).

\item[Proof:]  Let $\overline{AB}$ be a segment.  We consider the circle with center $A$ and radius $\overline{AB}$ (circle 1) and the circle with center $B$ and radius $\overline{BA}$ (circle 2).
There is a uniquely determined point $D$ such that $\overline{BD} \cong \overline{AB}$ and $A*B*D$.  This follows from the point placement axiom.

The point $D$ is on circle 2 by the definition of circle.

The point $A$ is on circle 2 by the definition of circle and the fact that congruence is an equivalence relation.

The point $A$ is in the interior of circle 1 by the definitions of ``interior" and ``closer than".

We need to show that $D$ is in the exterior of circle 1.  Thus we need to show that $A$ is farther from $D$ than it is from $B$, that is that $A$ is closer to $B$ than it is to $D$,
that is that the unique point $E$ on $\overrightarrow{AD}$ such that $\overline{AE} \cong \overline{AB}$ is on $\overline{AD}$, which is true  because $E=B$ and $A*B*D$.

It then follows by the Circle Crossing Axiom that circle 2 contains a point $C$ which lies on circle 1.  We then have $\overline{AC} \cong \overline{AB}$ and $\overline{BC} \cong \overline{BA}= \overline{AB}$, because the point $C$ lies on indicated circles, and $\overline{AC}\cong \overline{BC}$ by the fact that congruence is an equivalence relation.

It is worthy of note that $C$ cannot be on $\Line{AB}$, because the only points $E$ on $\Line{AB}$ which have $\overline{BE} \cong \overline{BA}$ are $A$ and $D$, and $C$ is not $A$ (because $\overline{AC}$ is a segment) and not $D$ (because $\overline{AC} \not\cong \overline{AD}$).  This deserves a separate writeup, as it establishes that we have a point not on a given line.

\item[Discussion of Euclid, Proposition II:]  Euclid's Proposition II says that given any point $A$ and any segment $\overline{BC}$ one can construct a segment $\overline{AD} \cong \overline{BC}$.  

We are here going to use weaker alternative axioms, as mentioned above, as this actually follows trivially from the second point placement lemma in our original alternative approach.


In the rational plane (the counterexample interpretation which shows that there is a problem with the proof of Prop I), Euclid's Prop II is true, but the Diameter Axiom, which is essential to the proof of Prop II, is false:
for example, the unit circle in the rational plane does not intersect the line $y=x$ which passes through its center.

The interest relative to Euclid is that his argment cannot work in the rational plane, so he must be using some unstated assumption.  

Here is the proof, lifted from the online document.

Let A be the given point, and BC the given straight line. 

It is required to place a straight line equal to the given straight line BC with one end at the point A.
Post. 1, I.1.

Join the straight line AB from the point A to the point B, and construct the equilateral triangle DAB on it.
Post.2
Post.3  (This uses the proof of Prop I, so for us uses the Circle Crossing Axiom).

Produce the straight lines AE and BF in a straight line with DA and DB. Describe the circle CGH with center B and radius BC, and again, describe the circle GKL with center D and radius DG.
I.Def.15  (Producing the lines is unproblematic.  What is problematic is why he knows that the points G and  L [whose important properties are really only specified by the diagram] exist.)

The Diameter Axiom gives us both of these points.  The circle with center $B$ and radius $\overline{BC}$ intersects $\Line{DB}$ in two points $G,H$ with $G*B*H$, by the Diameter Axiom.  One of these
is on $\overrightarrow{DB}$ and one is not:  we call the first one $G$.

Since the point B is the center of the circle CGH, therefore BC equals BG. Again, since the point D is the center of the circle GKL, therefore DL equals DG.
C.N.3

The Diameter Axiom gives us the point $L$, because the circle with center $D$ and radiius $\overline{DG}$ will intersect the line $\Line{DA}$ in exactly two points $K$ and $L$, with $K*D*L$, and we let $L$
be the one of these two points which lies on $\overrightarrow{DA}$.

And in these DA equals DB, therefore the remainder AL equals the remainder BG.
C.N.1

This application of our segment addition and subtraction axiom goes through without difficulty.  There will however be two different cases.  Euclid in his diagram assumes $D*B*G$ but there is another case
where $D*G*B$ (and so $D*L*A$ instead of $D*A*L$).

But BC was also proved equal to BG, therefore each of the straight lines AL and BC equals BG. And things which equal the same thing also equal one another, therefore AL also equals BC.

Therefore the straight line AL equal to the given straight line BC has been placed with one end at the given point A.

We can then prove something stronger.  

Suppose that we further have another point $E$.  We can argue by another application of the Diameter Axiom that there are exactly two points $X,Y$ such that $X,Y$ lie on $\Line{AE}$ and  on the circle with center $A$ and
radius $\overline{AL}$, with $X*A*Y$.  One of these (say $X$) will lie on $\overrightarrow{AE}$ and one will not.  So we have shown the point construction postulate:  for each $\overrightarrow{AE}$ and segment $\overline{BC}$, there is exactly one point $X$ on $\overrightarrow{AE}$ such that $\overline{AX} \cong \overline{BC}$.

\end{description}

\subsection{The alternative approach to congruence, construction, and addition of angles}

Just as we avoided use of real numbers as measures of line segments, we avoid their use as measures of angles, to keep things geometric.

We introduce a primitive notion of congruence of angles: $\angle BAC \cong \angle EDF$ is a sentence.

We define angles as above, and we use all definitions and lemmas from the plane separation postulate section above, which is shared by both approaches.

\begin{description}

\item[basic angle congruence axiom:]  Congruence of angles is an equivalence relation.

\item[Angle Construction Postulate:]  For any line $\Line{AB}$, half-plane $H$ determined by $\Line{AB}$,  and angle $\angle EDF$, there is a unique ray $\overrightarrow{AC}$ such that $C \in H$ and $\angle BAC \cong \angle EDF$.  Note that $C$ is not uniquely determined, just the ray.

\item[Angle Addition Postulate:]  For any angles $\angle BAC$ and $\angle EDF$ and points $G,H$ with $G$ in the interior of $\angle BAC$ and $H$ in the interior of $\angle EDF$, if $\angle GAB \cong \angle HDE$ and $\angle GAC \cong \angle HDF$, then $\angle BAC \cong \angle EDF$.


\end{description}

\end{document}