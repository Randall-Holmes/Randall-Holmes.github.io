\documentclass{slides}

\title{8/25/2021 Math 189 lecture slides}

\author{Randall Holmes}

\date{Wednesday Aug 25}

\begin{document}

\begin{slide}

\maketitle

\end{slide}

\begin{slide}

I don't usually make slides of all my lectures, but...I am trying an experiment. 

Since I have a drawing tablet, I can write on my slides as I lecture...we shall see how this innovation goes.

\end{slide}

\begin{slide}

{\Large Propositions}

A proposition is a statement which is either true or false.  If the proposition is true, its truth value may be written T or 1.
If the proposition is false, its truth value may be written F or 0.  

The operations on propositions that we consider take nothing into account about the proposition(s) involved than their truth values.  This is {\bf not} always true of the English or other natural language constructions which we might use to translate these logical operations, and we will point out specific examples of this.

We do not require that we actually {bf know} or can even easily determine the truth value of a proposition.  We do require that the definition of a proposition be precise (that its truth value, even if we cannot determine it, is not simply a matter of opinion).

\end{slide}

\begin{slide}

{\Large Atomic propositions}

We use variables $p,q,r,s,t\ldots$ to represent propositions whose internal structure we are not concerned with.   

This is a temporary usage in propositional logic:  more typical propositions are things like $3>2$, $x=y$, ``17 is prime", which have such features as subjects, objects, and verbs (they really do!).

It's not a typical use of variables either:  these usually stand for mathematical objects, such as numbers.

\end{slide}

\begin{slide}

{\Large Logical operations and compound propositions}

More complex propositions are built in propositional logic using a suite of operations.  We call a proposition which contains one or more of these logical operations a compound proposition;  the only non-compound propositions we have presented so far are variables, but something like $3=2$ is also not a compound proposition.



\end{slide}

\begin{slide}

These operations have truth value outputs which depend only on the truth values of their inputs.

The simplest example is {\em negation\/}:  $\neg p$ (which in typewriter notation might be written $\sim p$)
is true if $p$ is false, and false if $p$ is true.

This is our only unary operation (operation with one input).  It can conveniently be presented using a truth table:

$\begin{array}{c|c}
p & \neg p \\ \hline

T & F \\

F & T \\

\end{array}$


\end{slide}

\begin{slide}

We have a number of binary operations, written 

\begin{enumerate}

\item $p \wedge q$:  we pronounce this ``$p$ and $q$" and typewriter notation for it might be $p \& q$.

\item $p \vee q$:  we pronounce this ``$p$ or $q$" (but the meaning of this is more precise than the meaning of English ``or" as we will see).

\item $p \rightarrow q$:  we pronounce this ``if $p$, then $q$", and in various other ways, and its exact meaning is definitely somewhat surprising to an English speaker, because it completely ignores nuance in the meaning of this construction to make it depend only on truth values of the inputs.



\item $p \leftrightarrow q$:  we pronounce this ``$p$ if and only if $q$, which we may abbreviate ``$p$ iff $q$".

\item $p \oplus q$:  we might pronounce this ``$p$ exor $q$" if we ever do:  this operation expresses another meaning of English ``or":  we do not use it much if at all, but we present it for clarity.

\end{enumerate}

Each of the binary operations has a truth table.  They all have a common format:

$$\begin{array}{cc|c}

p & q & p \_\_ q \\ \hline
T & T & \\
T & F & \\
F & T & \\
F & F & \\
\end{array}$$

Challenge question:  how many possible binary propositional logic operations are there?

\end{slide}

\begin{slide}

{\Large And}

``$p$ and $q$" is true if $p$ is true and $q$ is true, and otherwise is false.

$$\begin{array}{cc|c}

p & q & p \wedge q \\ \hline
T & T & T\\
T & F & F\\
F & T & F\\
F & F & F\\
\end{array}$$

Some other English constructions, such as ``$p$, but $q$", have the same truth table as ``$p$ and $q$ but to our mind do not mean the same thing: there is a comment here that $q$ is surprising, which is not captured by the truth table.  But if they appear in a mathematical context, this is how we make their meaning precise.

\end{slide}

\begin{slide}

Notice that the ``and" in ``John and Mary are tall" cannot be translated to $\wedge$ until the sentence is expanded to
``John is tall and Mary is tall",  because $\wedge$ only connects sentences.  Similarly, the ``and" in ``Tim sings and dances" cannot translate to $\wedge$ until the sentence is expanded to$\ldots$

The ``and" in ``John and Mary carried the log" probably does not translate to $\wedge$ at all, even though the grammar of the English sentence is identical.  ``12 and 55 are relatively prime" is a mathematical example of a use of ``and" which does not translate to $\wedge$.


\end{slide}

\begin{slide}

{\Large Or}

We dictate that in mathematical language ``$p$ or $q$" is false if both $p$ and $q$ are false, and otherwise is true.
This is not the only meaning of English ``or":  this is the inclusive sense of ``or" which lawyers express as and/or.

$$\begin{array}{cc|c}

p & q & p \vee q \\ \hline
T & T & T\\
T & F & T\\
F & T & T\\
F & F & F\\
\end{array}$$

The other common sense of English ``$p$ or $q$" is an assertion which is true if one of $p$ or $q$ is true and the other is not, the exclusive sense of ``or".  We give this its own symbol $p \oplus q$, which we might read ``$p$ exor $q$" (this is really written xor but we suggest how to pronounce it by the way we write it).

$$\begin{array}{cc|c}

p & q & p \oplus q \\ \hline
T & T & F\\
T & F & T\\
F & T & T\\
F & F & F\\
\end{array}$$

We will not make much if any use of this operation.  It does have its uses in computer science.


\end{slide}

\begin{slide}

{\Large If}

I'll add slides for this operation later.  It isn't defined until section 1.3.

I do provide a truth table to fill in which we might get to as an interactive activity.

$$\begin{array}{cc|c}

p & q & p \rightarrow q \\ \hline
T & T & \\
T & F & \\
F & T & \\
F & F & \\
\end{array}$$

Let $p$ represent ``You will clean up your room today" and $q$ represent ``We will go out for pizza".
$p \rightarrow q$ is something mother might say to son.  In which rows of the table above is she telling the truth?

\end{slide}

\begin{slide}

{\Large More complicated propositions and truth tables}

A proposition using two or more operations will have a more complicated table.

As I work on these notes in my office before class....I am running out of time!   The last item I'll put in here is the format for the truth table for a statement involving three letters.

$$\begin{array}{ccc|c}

p & q & r & ??? \\ \hline

T & T & T & \\

T & T & F & \\

T & F & T & \\

T & F & F & \\

F & T & T & \\

F & T & F & \\

F & F & T & \\

F & F & F & \\

\end{array}$$

I'll probably add more material to these slides, related to things I say in lecture today, before I post them.



\end{slide}

\end{document}