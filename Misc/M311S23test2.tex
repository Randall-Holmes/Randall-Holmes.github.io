\documentclass[12pt]{article}

\title{Math 311. Spring 2023 Test II}

\author{Randall Holmes}

\begin{document}

\maketitle

This is a take home examination to be distributed on April 12 and due by April 19th at 1155 pm.  If you send me a draft of your exam paper by 1155 pm on April 16th (that is a hard deadline), I will return it with brief comments before midnight on the 17th.  The draft may include questions and expressions of puzzlement.

The basic rule is to consult no person other than myself.

If you have difficulty understand directions in a question, please ask me.  If you suspect that there is a typo in the statement of a question, please contact me promptly!

You may assume at the outset that the question you do worst on will be dropped;  I may make further adjustments to grading depending on class performance.

\begin{enumerate}

\item  Demonstrate using the Ruler Postulate  that given a point $A$ on a line $L$ and a positive real number $r$ that there are two points $B,C$ on $L$ such that
$d(B,A) = d(C,A) = r$, and $B*A*C$, and for any point $D$ on $L$ such that $d(D,A)=r$, either $D=B$ or $D=C$.

Your use of each postulate and definition should be accompanied by all supporting details.  It is easy to see that the theorem stated here is true, and I of course will understand vague hints, but I will not accept them for full credit in this problem.  This is true in general on this test, but particularly emphasized on this problem.

\item  Let $A,B$ be distinct points on a line $L$.   Demonstrate that there is a point $C$ on $L$ such that $d(A,C) = 2d(B,C)$ ($C$ is twice as far from $B$ as it is from $A$).  Demonstrate that there are exactly two such points.

Hint:  you might want to work with a coordinate function $f$ for $L$ such that $f(A)=0$ and $f(B)=d(A,B)$, which is provided by the Ruler Placement Postulate.

Extra credit:  Let $n$ be a positive real number not equal to 1.  Let $A,B,L$ be as above.  Show that there are exactly two points $C$ such that $\frac{d(A,C)}{d(B,C)} = n$

Hint (equally applicable to the first part):  use algebra to find out what values of $f$ at these points should be.  There are two possible situations, basically because for any points $X,Y$ on $L$, $d(X,Y)$ can be either $f(X)-f(Y)$ or $f(Y)-f(X)$.

\item  Demonstrate using the Plane Separation Postulate, the Incidence Postulate, and the Ruler Postulate that there are infinitely many distinct lines.

Extra credit: demonstrate that for any point $A$ and positive real number $r$ there are infinitely many distinct points $B$ such that $d(A,B)= r$ (this uses the previous proof, another application of the Ruler Postulate, and a little extra thought).

\item Prove that if one interior angle of a triangle is right or obtuse, then both the other interior angles are acute.  This was one of your homework problems.  I expect you to identify correctly the two major named theorems in the book that you need to use and explain with complete supporting details how they are used to prove this result.

\item  Prove Theorem 4.2.6 on p. 76.  The proof is reasonably straightforward and direct, and I intend to give a strong hint in lecture, but your assignment is to write the proof in full detail,
explaining each application of a postulate or theorem in detail.  You are allowed to use postulates and theorems which appear earlier than theorem 4.2.6 in the book (other than the parallel postulate, of course).

\item  Complete the proof of the exterior angle theorem:  prove $\mu(\angle DCA) > \mu(\angle ABC)$ by completely expanding the hint at the end of the proof on p. 71:  what you write should be at the same level of detail as the earlier part of the proof, and supported by a diagram.  This should be straightforward if you can follow the earlier part of the proof.

\end{enumerate}

\end{document}