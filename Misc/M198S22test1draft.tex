\documentclass[12pt]{article}

\usepackage{amssymb}

\usepackage{comment}

\title{Math 189, Fall 2022, Test I}

\author{Randall Holmes}

\begin{document}

\maketitle

The test will begin at 130 pm and officially end at 245 pm.

You are allowed your test paper, your writing instrument, and a non-graphing calculator.  You will need the calculator;
make sure you have a non-graphing scientific calculator for the exam.

There is no use of your book or notes allowed, nor is there a sheet of formulas.  If I think a specific fomula should be supplied, I will supply it with the problem.  For the most part, you are expected to know the formulas you are working with.


This is a practice version of the exam, but in broad terms the actual exam will be similar.
\newpage
\begin{enumerate}

\item  sets (0.3)

\begin{enumerate}


\item  Fill in each sentence with $\in$ or = in such a way as to make it true.  If both work, say both, if neither work, say neither.

\begin{enumerate}

\item  $\{1,2\} {\tt \_\_\_} \{1,2,\{1,2\}\}$

\item  $\emptyset {\tt \_\_\_}  \{1,2,3\}$

\item  $\{3\}  {\tt \_\_\_}  \mathbb N$

\item  $1 {\tt \_\_\_}  \{\{1,2\}\}$

\end{enumerate}

\item  Let $A$ be the set $\{3,4,5,6,7,8,9,10\}$

Give a definition of $A$ in the form $\{x \in \mathbb N:\ldots\}$

Give the set $\{x \in A:\frac x2 \in A\}$ in list notation.

\item  List the elements of $\{1,2\} \times \{a,b,c\}$

\end{enumerate}

\newpage

\item  functions...combine with counting, ask about how many injective functions, how many functions, how many increasing or nondecreasing functions

Let $A = \{1,2,3\}$ and $B = \{3,4,5,6\}$.

\begin{enumerate}

\item How many functions from $A$ to $B$ are there?

\item Give one of these functions in list notation as a set of ordered pairs.

\item How many injective functions from $A$ to $B$ are there?

\item How many injective functions from $B$ to $A$ are there?

\item How many nondecreasing functions from $A$ to $B$ are there (hint:  this is a stars and bars question)
\end{enumerate}
\newpage

\item  simple additive and multiplicative principle word problems


 

\begin{enumerate}
\item Lauren has four blouses, six skirts and ten one-piece dresses she can wear.  How many outfits can she make?

\item  Suppose in addition that she has three patterns of striped socks she can wear with a short skirt.  Two of her six skirts are short, and four of her one-piece dresses have short skirts.  How many outfits does she have now (with a long skirt she will always wear plain white socks).

\end{enumerate}



\newpage

\item  In a sophomore class of 21 students at a small school, every student takes at least one of English, Math, French.  13 take English, 13 take Math and
10 take French.  7 take English and Math, 7 take Math and French, and 4 take English and French.  How many brave students are taking all three subjects?

\newpage

\item  combinations and permutations

\begin{enumerate}

\item  A committee with 20 members needs to choose a Rules Committee with 5 members, which will have a secretary and a chair.  How many ways are there to do this?

\item  Suppose that the committee has 13 men and 7 women, but rules require gender balance in subcommittees.  To accommodate this, the Rules Committee is revised to have six members (three men and three women), still needing a secretary and a chair.  How many ways are there to choose the subcommittee and assign its officers?

\item  Tell a story about a committee choosing a subcommittee and officers which gives a combinatorial proof that \newline (30)(29)(28 choose 4) = (30 choose 6)(6)(5).
Also give an algebraic demonstration of this fact (hint:  it's fraction cancellation!)

\end{enumerate}

\newpage

\item  Each of these questions about $k$ choices from $n$ alternatives is answered in a different way, because of different combinations of conditions:  in some we are allowed to repeat choices, and in some we are not;  in some the order in which we make choice matters and in others it does not.  Briefly answer each question, and include calculations and brief explanation of the conditions which apply.

\begin{enumerate}

\item  How many ``words" (they don't need to be in the dictionary or even pronouncable) can you make with 7 letters in your Scrabble tray?  Assume that you have at least 7 of each letter.

\item  You go to the florist and order a bunch of a dozen roses.  There are pink, white and red roses.  How many bunches of a dozen are possible?

\item   Suppose you have one decal with each of the digits from 0 to 9, and you want to put 7 of these decals on your computer to make an adhoc ID number for it.  How many numbers are possible?

\item   A committee with 10 members wants to choose a three member executive committee.  In how many ways can this be done?

\end{enumerate}

\newpage

\item  sequence notation and definitions, summations perhaps, computing functions defined by recurrence relations

The sequence of Lucas numbers is defined by $L_0 = 1, L_1, = 3, L_{n+2} = L_n + L_{n+1}$

Determine the Lucas numbers up to $L_8$.

A sequence $S_n$ is defined by $S_n = \sum_{i=0}^n L_i$.  Compute the $S_i$'s up to $S_8$.

\newpage

\item  arithmetic and geometric sequences

Two sequences are given, one arithmetic and one geometric.  The indexing in both cases starts with 0.

Write a formula for the $n$ term of each sequence.

Compute the sum of the first 100 terms of each sequence (obviously, not by actually adding them up).

3,6,12,24$\ldots$ is the first sequence.

1,5,9,13$\ldots$ is the second sequence.

\newpage

\item  polynomial fitting or exponential recurrence relations (choose)

Do one of the following.  If you do both you can earn extra credit.

\begin{enumerate}

\item 5,4,8,26.61,120$\ldots$ (a sequence with indexing starting at 0) is defined by a polynomial.  

Use the method of finite differences to determine its degree.

Find a formula for the polynomial.

\item  Determine a closed form for the sequence defined by $a_0 =1, a_1=6, a_{n+2}=6a_{n+1}-8a_n$.  Show work developing this from a characteristic polynomial.

\end{enumerate}

\newpage

\item   Do both parts.  The part on which you do better will count 70 percent and the part you do worse on 30 percent.

\begin{enumerate}

\item  Prove that the sum of the first $n$ odd numbers is $n^2$.  State the theorem using summation notation, then prove it by mathematical induction.
Be sure to clearly identify the basis step, the induction hypothesis, the induction goal, and show where the induction hypothesis is used in the proof of the induction goal.


\item  Prove using mathematical induction that $n^3 + 2n$ is divisible by 3 for each natural number $n$.

\end{enumerate}

\newpage



\end{enumerate}

\end{document}