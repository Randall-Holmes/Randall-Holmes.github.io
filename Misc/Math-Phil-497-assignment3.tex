\documentclass[12pt]{article}

\title{Math/Phil 497 Assignment 3}

\author{Randall Holmes}

\begin{document}

\maketitle

\begin{enumerate}

\item Verify that $\sim(\sim(\sim p \vee q) \vee \sim(\sim q \vee p))$ is a suitable definition for $p \leftrightarrow q$ (if and only if:  Post's notation for this will be seen to be $p \equiv q$)
by presenting a truth table for it, which should include a truth table for each of its subexpressions (so that I can see that the truth table has been worked out by computation from the structure of the given term).  You may use T and F or + and $-$ in your truth tables, depending on your taste.

\item Write down an expression in familiar logical notation which has the following truth table

$$\begin{array}{c|c|c||c}

+ & + & + &  - \\
+ & + & - &  + \\
+ & - & + &  - \\
+ & - & - &  - \\
- & + & + &  + \\
- & + & - &  - \\
- & - & + &  - \\
- & - & - &  - \\

\end{array}$$

using whatever logical operations are convenient.  Then rewrite it using only $\sim$ and $\vee$.



\item In the notes, I gave a derivation of $\vdash\, \sim p \vee p$ in Post's system.

Give a derivation of the usual form of excluded middle, $\vdash p \vee \sim p$ in Post's system.  You may use the fact that $\vdash\, \sim p \vee p$ has already been derived.
This is not long:  you need to use an axiom, make a substitution into it (rule II) then apply rule III to get the desired result.  Your first line can be $\vdash\, \sim p \vee p$ (previously proved).

\item  I propose two logical equivalences:  $$(P \vee Q)  \rightarrow R \equiv (P \rightarrow R) \wedge (Q \rightarrow R)$$ and $$(P \vee Q)  \rightarrow R \equiv (P \rightarrow R) \vee (Q \rightarrow R)$$

Write each of these as an equation in the notation of Boolean algebra (you may need to consult the notes on this notation).  You may use the same letters (you don't need to follow the odd procedure I followed in lecture of replacing $P,Q,R$ with $x,y,z$ as I went).

One of them is correct and one of them is incorrect.  Determine which one is correct and which one is wrong.  You may use a Boolean algebra calculation to do this, or you may
use a truth table.

\item An exploration

Define $P | Q$ (neither $P$ nor $Q$) as having this truth table:

$$\begin{array}{c|c||c}

P & Q & P|Q \\

+ & + & - \\

+ & - & - \\

- & + & - \\
- & - & + \\

\end{array}$$

Write an expression using just $P$ and $|$ which is equivalent to $\sim P$ (hint:  you may use $P$ more than once).

Write an expression using just $P, Q$ and $|$ which is equivalent to $P \vee Q$.

Write an expression using just $P, Q$ and $|$ which is equivalent to $P \wedge Q$.

Explain why this implies that for any truth table we can write an expression in just variables and $|$ with that truth table.

Write an expression for $P \leftrightarrow Q$ using just $P,Q$ and $|$.

\end{enumerate}

\end{document}