\documentclass[12pt]{article}

\title{Math 387 Spring 2025 Test I review sheet}

\author{Randall Holmes}

\usepackage{amssymb}

\begin{document}

\maketitle

In this document, I review the homework assignments and specify problems which are similar to intended test questions.
This may be used as instruction for myself to construct a sample test to be posted later.

The exam will probably have about 8 questions.  The questions will be of equal weight;  my favorite way to adjust a test on which measures of central tendency are low is to reduce the weight of the question you do worst on (but by default I won't do this:  a low class average or median, or sometimes bad performance on a single question, might trigger this).

Where hard items mentioned below are concerned, be aware that you may be given a choice of challenging problems to do.

\begin{description}

\item[Homework 1:]  All questions on homework 1 are legitimate targets.  Counting word problems that you might have seen in Math 189 will probably make up a large part of the test.  Function counting problems are definitely in the zone for the test (counting functions generally, injections or surjections).  The question about unordered versus ordered pairs is a little out of the zone:  I am not planning to ask questions about set theory as such.

\item[Homework 2:]  Again, all questions on homework 2 are legitimate targets.   Question 5 is a reminder that various ways to understand binomial coefficients, including the formula in terms of factorials, are part of the exam content.  Question 6 is a direct exercise in reading the triangle.   The others are natural 189 style counting problems, perhaps harder than the ones in homework 1.

\item[Homework 3:]   Problems 47 and 48 (basic lattice path questions) which were not on this homework, are both good test questions.  49,50,51 are good study questions.  I may ask about diagonal lattice paths or Catalan numbers, you should be ready for this.  You should know how to compute Catalan numbers and situations where they may be applied.  I think problem 52 is too hard for a test.  55,56 are good questions:  you should be able to recognize an application of the binomial theorem, which is what is happening in this problems.  I also like problem 58 for that and other reasons.  I like problem 57 which I didnt assign.

\item[Homework 4:]  Problem 52 I think is a bit much for a test question.  65 and 66 make up a legitimate basis for a hard question...but if I literally ask that you've seen it already, so it is not so hard if you are ready.  I'll sketch for you a quick way to answer problem 65 which one of your colleagues gave...my solution is rather lengthy for a test response.  Problems 1,3,5,7 in the chapter one supplement are all legitimate.  The parenthesis balancing question is important as an example of the use of the Catalan numbers.

\item[Homework 5:]  The proofs in Homework or something similar are reasonable models for test questions.

The proofs in section 5 of the class notes are similarly possible test targets.  The final proof of the formula for the sum of an arithmetic sequence is probably too much, but the Lemmas used to prove it are fair game.  The proof of problem 75 I regard as something you should be able to follow but with too much abstract machinery for a test question.  Problems 76 and 78, proved in the notes, are fair game.

\item[Other proofs:]  The proof by induction that $_nC_r+_nC_{r-1} = _{n+1}C_r$, either from the definition of the binomial coefficients as sets, or from the factorial formula (I'll tell you which I am asking for) is fair game.

The proof by induction of the binomial theorem is not fair game.  But you should know the statement of the binomial theorem and be able to use it.

You should know the pigeonhole principle and be able to use it in an argument.  The proof by induction of the pigeonhole principle, as it appears on page 7 of the notes is hard but fair game.

You should know the definition of the Ramsey numbers $R(m,n)$ and you should know (though I don't promise either that I will ask this or I won't) the argument I gave that $R(m-1,n)+R(m,n-1)\leq R(m,n)$, which I will review on Wednesday.   I might ask you a question involving {\em using\/} this identity rather than proving it (along with the basic facts
that $R(1,n) = R(n,1) = 1$ and $R(2,2)=2$).  One could use these facts to prove that all the Ramsey numbers actually exist (by induction) or one could verify $R(3,3) \leq 6$.

\end{description}

\end{document}


