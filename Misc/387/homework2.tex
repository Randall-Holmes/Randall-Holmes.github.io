\documentclass[12pt]{article}

\usepackage{amssymb}

\title{Math 387 Homework 2}

\author{Randall Holmes}

\begin{document}

\maketitle

This is a rather strangely assorted set of problems.  I hope you will have fun with them.

They are due Friday the 31st January 2025.

\begin{enumerate}

\item  A bridge club has $4n$ members.  In how many ways can the group be divided into $n$ tables of 4 to play bridge?

In how many ways can the bridge players be divided into tables of four with the added information as to who is partners with whom in each table?

\item  In how many ways can we hand out 10 pieces of fruit to 5 children if each is to get exactly two pieces?

In how many ways can 10 pieces of fruit be portioned into 5 goodie bags of two pieces of fruit?  (this is not the same problem, and its answer is different!)

The numerical answers may be absurdly large:  computations using arithmetic operations and binomial coefficients are acceptable as answers.

\item  How many surjections are there from a set of 5 elements to a set of two elements?  How many surjections are there from a set of 5 elements to a set of three elements?

\item   In how many ways can ten ping pong balls be painted, the alternatives for each ball being white, red, green, and blue?  In how many ways can this be done if the number of green balls is to be even?

\item  Give two different arguments  that $\left(\begin{array}{c}n \\n-k\end{array}\right)$ is equal to $\left(\begin{array}{c}n \\k\end{array}\right)$.  There is a computation argument, and there is an argument straight from the definition of binomial cooefficients in terms of subset counting.

\item  How many subsets does a set of size 12 have?  How many of them are of sizes divisible by three?  [This is a problem for which Pascal' Triangle is a tool].

\end{enumerate}

\end{document}