\documentclass[12pt]{article}

\usepackage{amssymb}

\title{Math 387, Spring 2025, Class Notes}

\author{Randall Holmes}

\begin{document}

\maketitle

\tableofcontents

\newpage

\section{Jan 15 2025:}  We did administrative stuff then read problems 1-7 in the Bogart book.

I'm going to write out my official definition of what a function is.  We will talk about motivation for this later.

Let $S$ and $T$ be sets.

We say that $R$ is a relation from $S$ to $T$ iff $R$ is a triple $(S,T,G)$, where $G \subseteq S \times T$.  $G$ is called the graph of $R$, and we say that $x \, R \, y$ is true iff $(x,y) \in G$.

We say that $f$ is a function from $S$ to $T$ (written $f:S \rightarrow T$) just in case $f$ is a relation $(S,T,G)$ from
$S$ to $T$ and for each $x \in S$ there is exactly one $y$ in $T$ such that $(x,y) \in G$.  We define $f(x)$, for each $x \in S$, aas the unique $y$ such that $(x,y) \in G$.

An alternative approach is to identify relations and functions with their graphs (so they are just subsets of $S \times T$), but this requires care in expression which most undergraduate textbooks don't bother with.  We will discuss in detail what the problems are with the approach the book takes.

\end{document}