\documentclass[12pt]{article}

\usepackage{amssymb}

\title{Test I Math 287 Spring 2022}

\author{Dr Holmes}

\date{February 17, 2022}

\begin{document}

\maketitle

There are eight questions on this exam.  Questions 1 and 2 form a pair, questions 3 and 4 form a pair and questions 5 and 6 form a pair.  Your grade on a pair of questions will be seventy percent the grade on the one you do better on and thirty percent the grade on the one you do worse on.  Questions 7 and 8 are free standing.

You are allowed your test paper and your writing instrument.  There is no use for a calculator on this exam.

A reference sheet with axioms, propositions and definitions you need is the last page of the exam.  You may tear it off for reference.

You will be handed a second copy of this test when it is distributed in class.  You may complete as much of it at home as you wish, consulting no human being other than myself, but with free use of other resources, and submit what you do electronically by 11:55 pm on the 19th.  No submission of the take-home copy is required.  If you do submit a take-home, I may use it for partial credit decisions.

\newpage

\begin{enumerate}

\item (paired with 1) The FOIL identity you learned in school is $$(a+b)\cdot(c+d) = (a\cdot c + a \cdot d) + (b \cdot c + a \cdot d)$$ (First, Outer, Inner, Last).  We supply the parentheses for precision.  

Use the axioms (parts of Axiom 1.1, listed in the attachments to the paper, which you may tear off for reference) to give a detailed step by step proof of FOIL.  

Each step should be justified by a single axiom.  

You may use references to parts of the axiom using the exact phrases I give, and be aware that the phrase distributive law refers to exactly the form in the axioms:  you need to change things to apply it on the other side.

$(a+b)\cdot(c+d) = (c+d) \cdot(a+b)$  comm *

$= (c+d)a + (c+d)b$  dist (distributivity has to be $x(y+z) = xy+xz$)

$=a(c+d) + b(c+d)$  comm * (in two places)

$=(ac +ad) + (bc + bd)$  dist (in two places)

If one applies distributivity to the original form, before commutativity, one gets a form in which the terms have to be reordered using commutativity and associativity of addition.

\newpage

\item  (paired with 2) Prove $a \cdot 0 = 0$ using Proposition 1.9 and the axioms from chapter 1 in the reference sheet.

Each step should use one axiom or the proposition.

The application of 1.9 will take something of the form $X + a0 = X +0$ to the desired $a0 = 0$.

The right $X$ is $a0$.  Here is the actual proof:

(1) $a0 + a0= a(0+0)$  dist

$=a0$

(2) $a0 + 0 = a0$  identity property of addition

(3)  $a0 + a0 = a0 +0$  (1), (2) things equal to the same thing are equal to each other

$a0 =0$ prop 1.9 applied to (3)

\newpage

\item (paired with 4)  Prove using the definition of divisibility (on the reference sheet) and algebra (you may be more informal about the algebra) that if $d|a$ and $d|b$, it follows that $d|(a+b)$.

Your proof will start:  Let $a,b,d$ be integers and assume that $d|a$ and $d|b$...because $d|a$, there is an integer $x$ such that $a=d\cdot x$...carry on from there.

Let $a,b,d$ be integers and assume that $d|a$ and $d|b$.  

Because $d|a$ there is an integer $x$ such that $a=dx$. (def of divisibility)

Because $d|b$ there is an integer $y$ such that $b=dy$. (def of divisibility)

To show $d|(a+b)$ we need to find an integer $z$ such that $a+b=dz$.

Now $a+b = dx+dy = d(x+y)$:  setting $z=x+y$ we have found an integer $z$ such that $dz=a+b$, so $d|(a+b)$.

\newpage

\item (paired with 3)  Prove, using the axioms for {\bf N} (the set of positive integers:  axiom 2.1 on the reference sheet) and the definition of $<$ given on the reference sheet and algebra of equations with addition, subtraction and multiplication (about which you may be informal but be quite formal about applying the axioms for the positive integers (referencing the correct part of axiom 2.1) and the definition of $<$) that if $x<y$ and $0<z$, $x\cdot z <y\cdot z$.

Your proof will begin ``Let $x,y,z$ be integers.  Suppose that $x<y$ and $0<z$.  It follows that $y-x \in {\bf N}$, by the definition of $<$  and...(carry on from there).

Let $x,y,z$ be integers.  Suppose that $x<y$ and $0<z$. 

It follows that $y-x \in {\bf N}$ and $z-0 \in {\bf N}$ (and so since $z=z-0$, $z \in {\bf N})$.

It then follows by Ax 2.1 b that $(y-x)z \in {\bf N}$.

By algebra, $yz -xz \in {\bf N}$, since $yz-xz = (y-x)z$.

By def $<$, $xz<yz$ follows.

\newpage

\item (paired with 6)  

Prove by induction that the sum of the first $n$ integers is $\frac{n(n+1)}2$:  in symbols $\sum_{i=1}^n i = \frac{n(n+1)}2$.

Basis:  $\sum_{i=1}^1 i = 1 = \frac{1(1+1)}2$. check

Induction Step:  Let $k$ be chosen arbitrarily.

Assume (ind hyp) that $\sum_{i=1}^k i = \frac{k(k+1)}2$

Goal:  show that $\sum_{i=1}^{k+1} i = \frac{(k+1)((k+1)+1)}2 = \frac{(k+1)(k+2)}2$

Proof of induction step:  $\sum_{i=1}^{k+1} i = \sum_{i=1}^{k} i +(k+1)$  recursive definition of summation

$= \frac{k(k+1)}2 + (k+1)$  ind hyp

$= \frac{k(k+1)+ 2(k+1)}2$ common denominator

$= \frac{(k+2)(k+1)}2$  dist

$= \frac{(k+1)(k+2)}2$  comm *

\newpage

\item (paired with 5)   

A sequence $a_1$ is defined recursively:  $a_1=1$, $a_{k+1} = 2a_k+1$.  

Compute the first six terms of this sequence.  $1, 3, 7, 15, 31, 63$

Prove by induction that for each natural number $n$, $a_n = 2^n-1$.

Basis:  $a_1 = 1 = 2^1-1$ check

Induction step:  Fix a natural number $k$, chosen arbitrarily.

Assume (ind hyp)  $a_k = 2^k -1$

Goal: $a_{k+1} = 2^{k+1} -1$

Proof of goal:  $a_{k+1} = 2a_k +1$  def of sequence a

$= 2(2^k-1)+1$ ind hyp

$= 2^{k+1}-2+1 = 2^{k+1}-1$, which is what we want.

\newpage

\item (unpaired)

\begin{enumerate}

\item  Write the negation of the sentence ``I like coffee and I don't like tea" in natural English (the negation moved all the way in and applied to the verb).  answer:  ``I don't like coffee {\bf or} I like tea"

\item  Write the negation of the sentence in logical notation $$(\exists x \in {\bf N}:(\forall y \in {\bf N}: x \geq y)),$$ in a form which doesn't involve negation at all (move the negation all the way to the right and replace the order relation with its negation).

answer:  $$(\forall x \in {\bf N}:(\exists y \in {\bf N}: x < y)),$$
which is all you needed to write, but here is step by step justification:

$$\neg (\exists x \in {\bf N}:(\forall y \in {\bf N}: x \geq y))$$ is equivalent to

$$(\forall x \in {\bf N}:\neg (\forall y \in {\bf N}: x \geq y))$$ which is equivalent to

$$(\forall x \in {\bf N}:(\exists y \in {\bf N}: \neg x \geq y))$$ which is equivalent to

$$(\forall x \in {\bf N}:(\exists y \in {\bf N}: x < y))$$

\item  Say in English what the sentences  $$(\exists x \in {\bf N}:(\forall y \in {\bf N}: x >  y))$$  and $$(\forall y \in {\bf N}:(\exists x \in {\bf N}: x >  y))$$ mean.  Which one is true?

The first sentence says that there is a fixed natural number $x$ which is greater than every natural number (this would include itself!).  This is false.

The second sentence says that for any natural number, there is a greater natural number.  This is true.


\end{enumerate}

\newpage

\item  (unpaired) 

 State the Binomial Theorem using notation for binomial coefficients and summation notation.

State it for the exponent 4 and write the sum out in full, eliminating the summation notation and evaluating all the binomial coefficients (in other words, expand out $(x+y)^4$ using the theorem).

The theorem:  for any natural number $n$, $(x+y)^n = \sum_{i=0}^k \left(\begin{array}{c} n \\ i \end{array}\right)x^{n-i}y^i$

For $n=4$,  $(x+y)^4 = \sum_{i=0}^4 \left(\begin{array}{c} n \\ i \end{array}\right)x^{n-i}y^i$

$= \left(\begin{array}{c} 4 \\ 0 \end{array}\right)x^{4}+\left(\begin{array}{c} 4 \\ 1 \end{array}\right)x^{3}y+\left(\begin{array}{c} 4 \\ 2 \end{array}\right)x^{2}y^2+\left(\begin{array}{c} 4 \\ 3 \end{array}\right)xy^3+\left(\begin{array}{c} 4 \\ 4\end{array}\right)y^4$

$=x^4 +4xy^3+6x^2y^2+4xy^3+y^4$

I didnt require the step by step development but I was pleased to see it on many papers.


\newpage

\section{Reference sheet}


\begin{description}
\item[Axiom 1.1.] If m, n, and p are integers, then
\begin{enumerate}
\item m + n = n + m . (commutativity of addition)
\item  (m + n) + p = m + (n + p) . (associativity of addition)
\item  m · (n + p) = m · n + m · p . (distributivity)
\item m · n = n · m . (commutativity of multiplication)
\item  (m · n) · p = m · (n · p) . (associativity of multiplication)
\end{enumerate}
\item[Axiom 1.2.] There exists an integer 0 such that whenever m $\in$ Z, m + 0 = m.
(identity element for addition)
\item[Axiom 1.3.] There exists an integer 1 such that 1 $\neq$  0 and whenever m $\in$ Z, m · 1 = m .
(identity element for multiplication)
\item[Axiom 1.4.] For each m $\in$ Z, there exists an integer, denoted by $-$m , such that m +
($-$m) = 0. (additive inverse)
\item[Axiom 1.5.] Let m , n, and p be integers. If m · n = m · p and m $\neq$ 0, then n = p. (cancellation).
\item[Proposition 1.9.] Let m, n, and p be integers. If m + n = m + p, then n = p
\item[Axiom 2.1.] There exists a subset N $\subseteq$ Z with the following properties:
\begin{enumerate}
\item If m, n $\in$ N then m + n $\in$ N.
\item  If m, n $\in$  N then mn $\in$ N.
\item  0 $\not\in$ N.
\item  For every m $\in$ Z, we have m $\in$ N or m = 0 or $-$m $\in$ N.
\end{enumerate}
\item[Definition:]The statements m $<$ n (m is less than n) and n $>$ m (n is greater than
m) both mean that
n $-$ m $\in$ N .
\item[Definition:]  When m and n are integers, we say m is divisible by n (or alternatively, n divides m) Do not confuse this with the
notations n
m and n/m for
fractions.
if there exists j $\in$ Z such that m = jn. We use the notation n $|$ m.

\end{description}

\end{enumerate}

\end{document}




\end{document}