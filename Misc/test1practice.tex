\documentclass[12pt]{article}

\usepackage{amssymb}

\title{Study Guide and Test Practice for Test I, Feb 17}

\author{Dr Holmes}

\begin{document}

\maketitle

This test is going to be rather menacing, being made entirely of proofs.

There will be 8 questions, organized into pairs.  The question you do better on in a pair will count 70 percent
and the question you do worse on 30 percent of the total score for the two questions.

There might be an unpaired extra question or two of an easy (computational or test of memory) nature.

What follows is a set of proof exercises.  Some of them may be similar or identical  to questions which will appear on the exam.
Pay attention to questions which tell you what you are allowed to use.  When you are required to use a limited set of axioms (or definitions) these axioms and definitions will appear on your test paper.

I may at my discretion do a limited number of these things in class.  I do intend to lecture.  I am offering an extra office hour 3-4pm today, Monday the 14th.

I remind you that you will get two copies of the exam, one to do in class and one which you may take home and work entirely or in part, turning your paper in electronically by 1155 pm on February 19th.

\begin{enumerate}

\item  Do a couple of parts of Prop 1.11, doing every step using a single application of a single axiom.

\item  Prove prop 1.13.

\item  Prove prop 1.9.  Use prop 1.9 and axioms to prove that for any integer $x$, $x \cdot 0 = 0$.  I {\bf will} ask this, have it ready.


\item  Prove that for any integers $d,a,b$, if $d|a$ and $d|b$ then $d|a+b$ and $d|ab$.

\item  Using Axiom 2.1 (properties of the set of positive integers) and the definition of $<$ in terms of $\mathbb N$. prove
props 2.2. 2,4, 2.8, all parts of 2.7.  all parts of 2.12.  You can use the axioms from chapter 1 implicitly, just say algebra.  But you may not use familiar properties of order unless you have proved them from the given axiom and definition.

Prove Prop 2.10 using the same resources.  Prove prop 2.14.

\item  Things like any part of prop 2.18 are targets.

\item  I may ask something like Project 3.1, 3.2 testing reading of quantifier notation.

Something like project 3.7 (negations of logically connected and quantified statements) can be expected.

\item  I might ask a computational question about recursively defined sequences:  just ask you to compute the first few terms of a sequence.
But it might be paired with instructions to prove something about that sequence by induction.

For example, define $a_1 = 1, a_{k+1} = 2a_k+1$.  Compute the first six terms of this sequence.  Then prove that $a_n = 2^n-1$ for each natural number $n$, by induction.

\item  Be familiar with the recursive definitions of exponentiation and factorial.  Be ready to prove something like Prop 4.6 iii or another familiar property of exponentiation from algebra of addition and multiplication and the recursive definition.  $(ab)^k = a^k\cdot b^k$ is another good law to prove.

Be able to do things like prop. 4.7.

\item  Be able to do induction proofs involving summations like 4.11, 4.13.

\item  Be ready to write the proof of 4.16ii.  I will ask this.

\item  About the binimial theorem, some relaxed things, compute some binomial coeffients (showing hand calculations) and apply the theorem to some power of a sum.
Corollary 4.22.

I will ask you to write down the Binomial Theorem, exactly as it appears in the book.  Obviously this wont count on your take home paper.

\end{enumerate}


\end{document}