\documentclass[12pt]{article}

\title{Homework 10, Math 189, Fall 2023}

\author{Randall Holmes}

\begin{document}

\maketitle

This is homework on section 2.3 and 2.4 in Levin.  It is due next Monday rather than Friday (I am writing late on Thursday the 17th).   Preparing the test in the other class is eating my time.

\begin{enumerate}

\item   Consider the sequence $0, 6, 24, 60, 120, 210\ldots$  The first term is $a_1$.

Determine its sequences of first, second, third, fourth differences until you get a constant sequence.

Use polynomial fitting to determine a closed form for terms of the sequence, showing all work.   You might be able to determine the formula by guessing, but this carries very limited credit.
You may use Levin's method or my binomial coefficient method;  in either case you need to exhibit the answer as a polynomial (so more calculations will be needed if you use the binomial coefficient method, to simplify the answer).

\item   A simple exercise:  compute eight terms of the sequence $b_n = n^4-n^2+2n$ (starting at $n=0$) and compute difference sequences to the point where you get a constant sequence.  You should be able to predict where this will happen.  

Then write an expression for the sequence using the method involving binomial coefficients that I displayed in class (I'll exhibit this in office hours on request and in class on Friday to anyone who didn't see it) and do calculations verifying that it is the same sequence.

\item  Levin, section 2.3, problem 12.

\item  Consider the sequence $a_0=1; a_1 = 3;a_{n+2} = 7a_{n+1}-10a_n$.

Compute the first eight terms of this sequence.  Find a closed form for $a_n$ using the characteristic equation technique.


\item  Levin, section 2.4 problem 11:  hint for the recurrence relation:  its straightforward to determine $a_1$ and $a_2$.  Determine each subsequent $a_n$ in terms of earlier $a_m$'s by considering two cases:  we are placing the tiles left to right, there are two cases depending on whether the last one placed is one by one or one by two.  It might help to actually figure out some $a_n$'s by playing with suitable slips of paper...

\item   Levin, section 2.4, problem 13.

\end{enumerate}

\end{document}