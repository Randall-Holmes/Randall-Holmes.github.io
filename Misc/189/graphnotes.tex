\documentclass[12pt]{article}

\title{Graph theory notes for Math 189, Fall 2023}

\author{Dr Holmes}

\begin{document}

\maketitle

Updated again 10/28/2022:  I left out the handshake lemma.  I also clarified the definition of a cycle and added the theorem that a finite graph must contain two vertices of the same degree.

updated 11/29/2023 just to say that this is for the fall 2023 class.

Much of this is actually in Levin, but I want to write things out in more or less the way I presented them, and there are some things I did which he does not do.

\begin{description}

\item[Definition:]   A {\em graph\/} is an ordered pair $(V,E)$, where $V$ is a set and $E$ is a collection of two element subsets of $V$.

If $G=(V,E)$ is a graph, we say that elements of $V$ are {\em vertices\/} of (or in) $G$ and elements of $E$ are {\em edges\/} of (or in) $G$.

If $a$ is a vertex and $e$ is an edge in $G$, we say that $a$ is incident on $e$ and that $e$ is incident on $a$ (in $G$) just in case $a \in e$.

If $a$ and $b$ are vertices in $G$, we say that $a$ and $b$ are {\em adjacent\/} in $G$ just in case $\{a,b\}$ is an edge in $G$.

If $e$ and $f$ are edges in $G$, we say that $e$ and $f$ are adjacent just in case they are distinct and share an element.

\item[Observation:]  We work with pictures of graphs, in which vertices are represented by dots and edges are represented by curves drawn between the dots.

We say that a graph is {\em planar\/} just in case it has a picture (drawn on an ordinary plane surface) in which the curves representing edges do not cross each other.  A planar graph is likely to have pictures in which the edges do cross each other:  we would be likely to call these non-planar pictures of a planar graph.  A graph is non-planar if {\em all\/} pictures of that graph are non-planar.  We will show later in the course that certain graphs are non-planar.

\item[Related ideas we won't usually talk about:]  $G=(V,E)$ is a directed graph if each element of $E$ is an ordered pair of elements of $V$ (edges can be pictured as arrows from one vertex to another).

$G=(V,E)$ is a multigraph if each element of $E$ is a three-element set two elements of which belong to $V$ (the third element serves as a label, making it possible to draw many edges between the same two vertices).

Either graphs or multigraphs can have the definition modified to allow loops (edges from a vertex to itself):  edges are sets of no more than two vertices, or no more than two vertices plus a non-vertex for multigraphs.  Directed graphs naturally allow loops.

\item[Identity of graphs:]  Two graphs are {\em equal\/} if they are literally the same ordered pair of sets.

Note that one can draw multiple different looking pictures of the same graph:  a kind of question I might ask is, are these two pictures of graphs pictures of equal graphs (or of the same graph, same question).

Graphs $G_1 = (V_1,E_1)$ and $G_2=(V_2,E_2)$ are {\em isomorphic\/} iff there is a bijection $f:V_1 \rightarrow V_2$ (a one-to-one function from $V_1$ onto $V_2$) such that for every $a,b \in V_1$, $\{a,b\} \in E_1$ if and only if $\{f(a),f(b)\} \in E_2$.

If two graphs are isomorphic, they have the same structure.  A picture of one can be turned into a picture of the other simply by relabelling the vertices (the process of relabelling is recorded in the function $f$).

$G_1$ and $G_2$ might be the same and there might be an isomorphism from $G_1$ to $G_2$ other than the identity map on the vertices of the graph (the graph might have some kind of symmetry).

You may be asked to find isomorphisms from one graph to another or argue that there cannot be one.

Notice that isomorphic graphs must have the same number of vertices.

\item[Degree of a vertex in a graph;  degree sequence:]  The degree of a vertex $a$ in $G$ is the number of edges incident in $a$ (or the number of vertices adjacent to $a$).

The degree sequence of a graph is the list of degrees of its vertices in nonincreasing order.

Notice that isomorphic graphs clearly must have the same degree sequence.

The converse is not true:  try to draw a pair of graphs which are not isomorphic and have the same degree sequence.

\item[Theorem (handshake lemma):]  The sum of the degrees of the vertices in a finite graph is twice the number of edges, and so must be even.

\item[Proof:]  Each edge $\{a,b\}$ in a graph contributes one to the degree of $a$ and one to the degree of $b$.
The result is then seen to be obvious.

\item[Corollary:]  The number of vertices of odd degree in a finite graph must be even.

\item[Theorem:]  In any finite graph with at least two vertices, there must be two vertices of the same degree.

\item[Proof:]  Suppose the graph has $k$ vertices.  The degrees which are possible in the graph are
$0,1,2,\ldots,k-1$, a total of $k$ degrees.  But in fact, there cannot be a vertex of degree $k-1$ and a vertex of degree 0,
because a vertex of degree $k-1$ is connected to every other vertex, so none of the vertices can be of degree 0.
So there are actually $k-1$ possible degrees to assign to the $k$ vertices, either $\{0,\ldots,k-2\}$ or
$\{1,\ldots,k-1\}$.  Thus two vertices must be assigned the same degree.

How does this argument fail when $k=1$?

This result is somehow really spooky:  it is not clear what force is making you have two vertices of the same degree when you are actually drawing small graphs.  It is also false for infinite graphs:  it is not difficult to draw an infinite graph in which all vertices have different degrees.

\item[Walks:]  Let $G=(V,E)$ be a graph and let $a$ and $b$ be vertices in $G$.  A walk from $a$ to $b$ of length $n-1$ is a function $w$ from $\{1,\ldots,n\}$ to $V$, a finite sequence of vertices, in which $w(1)$, which we usually write $w_1$, is $a$ and $w(n)$, which we usually write $w_n$, is $b$, and each pair $\{w_i,w_{i+1}\}$ ($1 \leq i <n$)belongs to $E$.  

We say that a walk $w$ from $a$ to $b$ is a path if and only if for all $i,j$, if $w_i = w_j$ then $\{i,j\}=\{1,n\}$:  the only way vertices can be repeated in the graph is if the walk starts and ends at the same vertex.

A walk visits $c \in V$ just in case there is $i$ such that $w_i = c$, and it traverses edge $\{c,d\}$ just in case there is $i$ such that $w_i = c$ and $w_{i+1}=d$.

A path which has its beginning and end the same and which has length at least 3 we call a {\em cycle\/}.  We also call a graph a cycle if there is a cycle in it which visits all of its vertices and traverses all of its edges.  Notice that for any vertex $a$ the path of length 0 which begins (and ends) at $a$ is not a cycle, and for any edge $\{a,b\}$ the paths of length 2 which we might write $a,b,a$ and $b,a,b$ have the same beginning and end but are not cycles.  In terms of pictures, the length 0 and 2 paths with the same beginning and end do not enclose space.

Notice that for
any vertex $a$ there is a walk of length 0 from $a$ to $a$, namely the function from $\{1\}$ to $V$ sending 1 to $a$.

\item[Theorem:]  If there is a walk from $a$ to $b$ in $G$, then there is a path from $a$ to $b$ in $G$.

\item[Proof:]  If there is a walk from $a$ to $b$ in $G$, there must be a walk $w$ of shortest length $n$ from $a$ to $b$.

Now suppose that $i < j$ and $w_i = w_j$.  We define $w'_k$  for $k \in \{1,n-(j-i)\}$ as  $w_k$ if $k \leq i$ and as $w_{k+(j-i)}$ otherwise.  It is easy to verify that $w'$ is a walk from $a$ to $b$ shorter than $w$, which is impossible.  The definition just formalizes the idea of skipping everything in $w$ between the identical $w_i$ and $w_j$.

\item[Definition (connected graph, 1):]  A graph $G$ is {\em connected\/} just in case for each pair of vertices $a,b$
there is a walk from $a$ to $b$ in $G$ (equivalently, by previous theorem, just in case for each pair of vertices $a,b$ there is a path from $a$ to $b$ in $G$).

\item[Definition (connected graph, 2)]:  A graph $G=(V,E)$ is {\em disconnected\/} just in case there
are disjoint nonempty sets $A,B$ such that $A \cup B = V$ and for every $e \in E$, either $e \subseteq A$ or $e \subseteq B$ (i.e, there is no edge from a vertex in $A$ to a vertex in $B$).  A graph $G$ is connected just in case it is not disconnected.

\item[Temporary rule:]  We cannot define the same concept in two different ways without proving that the definitions are equivalent!  For the moment, we call the concept defined by the first definition connected$_1$ and the concept defined by the second definition connected$_2$.

This state of suspense will not last long.

\item[Theorem:]  For any graph $G$, $G$ is connected$_1$ if and only if it is connected$_2$ (so we can say just ``connected").

\item[Proof:]  The proof falls into two parts, as is natural for the proof of an if and only if statement.

First we show that if $G$ is connected$_1$, it cannot be disconnected, and so must be connected$_2$.

We do this by introducing a graph $G$ arbitrarily, assuming that $G$ is connected$_1$, and further assuming that it is disconnected, and showing that a contradiction follows (remember that this is not an immediate contradiction, because disconnected means ``not connected$_2$").

Because we assume that $G=(V,E)$ is disconnected, we can choose disjoint nonempty sets $A,B$ such that
$A \cup B = V$ and there is no edge from an element of $A$ to an element of $B$.  Choose $a \in A$ and $b \in B$ (we can do this because $A,B$ are not empty).  

Since $G$ is connected$_1$, we can choose a walk $w$ from $a$ to $b$.  Notice that $w_1 \in A$ and $w_n$ is in $B$.

We use the well-ordering principle for a snappy proof here.  Let $S$ be the set of all $j$ such that
$w_j \in B$.  $S$ is nonempty since it contains $n$ and bounded below (by 1) so it has a smallest element $s$.
$s>1$ since $w_1 \not\in B$, so $w_{s-1}$ is defined.  Now there is an edge from $w_{s-1}$ to $w_s$.
$w_s \in B$ because $s \in S$,  $s-1 <s $ is not in $S$ because $s$ is the smallest element of $S$.  So $w_{s-1} \not\in B$.  But this means $w_{s-1} \in A$.  But this is a contradiction:  we cannot have an edge from an element $w_{s-1}$ of $A$ to an element $w_s$ of $B$.

Now we get to the second part of the proof:  we need to show that if a graph $G$ is connected$_2$ it must be connected$_1$.

We introduce a graph $G$ arbitrarily and suppose that $G$ is connected$_2$.  We suppose for the sake of argument that it is not connected$_1$ and show that a contradiction follows.

Since $G=(V,E)$ is not connected$_1$, we can choose vertices $a,b \in G$ such that there is no walk from $a$ to $b$.

Now we define $A$ as the set of all vertices $c$ in $G$ such that there is a walk from $a$ to $c$.  We define $B$ as
$V-A$.  Note that $A$ is nonempty because $a \in A$, $B$ is nonempty because $b \in B$ and $A \cup B$ is clearly $V$.

We claim that there cannot be a path from any $c \in A$ to any $d \in B$, which shows that $G$ is disconnected, which is a contradiction.  Let $c \in A$, $d \in B$, and $w$ be a walk from $c$ to $d$.  There is a smallest $i$ such
that $w_i \in B$ by the well-ordering principle, and $i>1$ because $w_1 = c \not\in B$.  Thus $w_{i-1}$ exists,
and $w_{i-1} \in A$ because $i-1$ is less than the smallest index $i$ of a vertex visited by the walk which is in $B$.
Thus there is a walk $w'$ from $a$ to $w_{i-1} = w'_j$.  But then $w' \cup \{(j+1,w_i)\}$ is a walk from $a$ to $w_i$
(just tack $w_i$ onto the walk $w'$) so $w_i \in A$, and this contradicts the choice of $w_i$ as the first vertex visited by the walk which is in $B$ and so not in $A$.

\end{description}

Now we talk about trees.

\begin{description}

\item[Definition:]  A tree is a graph which is nonempty and contains no cycles.

\item[Theorem:]  A graph $G$  is a tree if and only if for any pair of vertices $a,b$ in $G$ there is exactly one path from $a$ to $b$,

\item[Proof:]  We show that if $G$ is any graph and there is more than one path from $a$ to $b$, vertices in $G$,
then there is a cycle in $G$, and so $G$ is not a tree.

Suppose $u$ and $v$ are distinct paths from $a$ to $b$, $u_1 = v_1 = a;  u_m = v_n = b$. Let $i$ be the smallest integer such that
$u_i \neq v_i$ (there must be such an integer because the paths are different).  Now find the first $u_j$ with $j>i-1$ such that $u_j = v_k$ for some $k$.  There is one because $u_m = v_n = b$.  Observe that $u_j$ (and so $v_k$)
is different from all $u_s$ with $s<i$, because this is a path, so in fact $k\geq i$.   We define $w_I$ as $w_{i-2+I}$ for
$I \leq j-(i-2)$ (in other words, follow $u$ from $u_{i-1}$ to $u_j$) then $w_I = v_{k-(I-(j-(i-2)))}$ for $j-(i-2)<I<(j-(i-2))+(k-(i-2))$ [the exact indexing here might be off]:  the idea is to have a sequence of the $u_s$'s starting at $u_{i-1}$ and ending at $u_j$, then a sequence of $v_s$'s starting at $v_k$ and going down to $v_{i-1}=u_{i-1}$.  This is a cycle.

Now if $G$ is a tree, there is a walk and thus a path from a given $a$ to a given $b$ because a tree is connected, and only one such path because if there were two such paths there would a cycle in $G$.

Suppose $G$ is a graph in which there is a unique path from any given vertex $a$ to any other given vertex $b$.  $G$ is then connected because there is a path, and so a walk, from each vertex to each other and it contains no cycle because for any two distinct vertices in a cycle (and any cycle has at least 3 vertices and so does contain two distinct vertices) there are obviously two paths from each to the other (going in opposite directions around the cycle).

The theorem is proved.

\item[Theorem:]  A tree with at least two vertices, and finitely many vertices, has at least two vertices of degree one.

The book doesn't note the requirement that the graph be finite.

\item[Proof:]  Since the tree has finitely many vertices, there is a longest path in the tree.  Let $p$ be this 
path, from $a=p_1$ to $b=p_n$.  We claim that both $a$ and $b$ are of degree one.

Suppose $b$ is of degree greater than one.  Then there is an edge from $b$ to $p_{n-1}$ and also an edge from
$b$ to some other vertex $c$.  $c$ cannot be any $p_i$, because this would create a cycle.  But this means that
the path $p \cup \{(n+1,c)\}$, tagging $c$ onto the end of $p$ after $b$, is a path longer than $p$, which is impossible.

A very similar argument, tacking on another vertex at the beginning of the path $p$, shows that $a$ cannot be of degree greater than 1.

\item[Theorem:]  A tree $G = (V,E)$ which is finite and has at least one vertex satisfies $|E| = |V|-1$.

\item[Proof:]  We prove this by induction on $|V|$.  If $|V|=1$, the graph has one vertex and no edges, so $|E| = 0 = 1-1 = |V|-1$,

Now suppose that the theorem is true for all graphs with $k$ vertices.  Let $G$ be a graph with $k+1$ vertices.
$G$ has at least one vertex of degree 1, which we call $v_0$, which belongs to a single edge $\{v_0,v_1\}$.

Consider $G- = (V - \{v_0\},E - \{\{v_0,v_1\}\})$.  This is a graph:  we removed a single vertex and also removed all edges containing it.

It is a tree.  If it contained a cycle, $G$ would contain a cycle, and $G$ is a tree.  The unique path in $G$ from any element $a$ to any element $b$ of $G-$ will not go through $v_0$, because $v_0$ can only appear in a path in $G$ as an endpoint.  So the path in $G$ from $a$ to $b$ is also a path in $G-$, so $G-$ is connected (because this works for any pair of vertices there), so $G-$ is a tree.

Now $G-$ has $k$ vertices, so by ind hyp it contains $k-1$ edges.  That means that $G$ contains $k+1$ vertices and $k$ edges, and the theorem is proved ($|E| = k = (k+1)-1 = |V|-1$).

\item[Definition:]  If $G=(V,E)$ is a connected graph, we say that $T = (V,E')$ is a spanning tree fo $G$ iff
$E' \subseteq E$.  A spanning tree for $G$ has the same vertices as $G$ and all of its edges are edges in $G$.

\item[Theorem:]  Any connected graph has a spanning tree.  

\item[Proof:]  It is obvious that a graph which is not connected cannot have a spanning tree.

We argue by induction on the number of edges that a connected graph must have a spanning tree.

A connected graph with 2 or fewer edges cannot contain a cycle (which must contain at least three edges)
so it is its own spanning tree.

Suppose all graphs with $k$ edges which are connected have spanning trees.  Let $G$ be a graph with $k+1$ edges.
If it does not contain a cycle, it is a tree and is its own spanning tree.  If it does contain a cycle, choose one edge from the cycle to delete.   If a walk from vertex $a$ to vertex $b$ went through the deleted edge, there is still a walk from $a$ to $b$, because the deleted edge can be replaced by the rest of the cycle.  So the graph with the edge deleted is a connected graph with $k$ edges, by ind hyp has a spanning tree, which will also be a spanning tree for $G$.

Notice that it follows from this theorem that a connected graph with $k>0$ vertices must have at least $k-1$ edges,
because deleting some number of them gives a tree with $k$ vertices which has exactly $k-1$ edges.

\item[Theorem (Euler's formula):]  Let $G$ be a connected planar graph with $v\geq 1$ vertices and $e$ edges, which divides the plane into $f$ regions (``faces").  Then $v-e+f=2$.

\item[Proof (by induction on the number of edges):]  If the graph has 0 edges, then $v=1, e=0, f=1$ gives $v-e+f = 1-0+1 = 2$.

Now suppose that the result is true for all connected planar graphs with $k$ edges (ind hyp).  Let $G$ be a connected planar graph with $k+1$ edges. 

$G$ either contains a cycle or it doesn't.

If $G$ does not contain a cycle, it is a tree.  It thus has a vertex of degree 1.  Drop this vertex and the edge containing it from the graph to contain a graph $G^-$.  We have shown in earlier proofs that dropping a vertex of degree 1 and its associated edge from a connected graph gives a connected graph.  $G^-$ satisfies the theorem $v-e+f=2$, where $v,e,f$ are counting vertices edges and faces in $G^-$, because it has $k$ edges (using ind hyp).  $G$ has $v+1$ vertices, $e+1$ edges (we know this is actually $k+1$ but that doesn't play a role in the argument at this point) and divides the graph into $f$ regions (dropping a vertex of degree 1 and its associated edge will not change the number of regions, because the edge is just a spur sticking into a region),
and $(v+1)-(e+1)+f = v-e+f = 2$, so $G$ satisfies the theorem.

If $G$ does contain a cycle, choose an edge in a cycle and delete it to obtain a graph $G^-$ with $k$ edges.  $G^-$ is connected because in a walk from a vertex $a$ to a vertex $b$ in $G^-$, the deleted edge can be replaced with the rest of the cycle it was removed from.  Thus $G^-$ satisfies the theorem $v-e+f=2$ (where again $v,e,f$ count vertices edges and faces in $G^-$).  Now $G$ has $v$ vertices, $e-1$ edges, and $f-1$ faces, because the deleted edge linked two vertices on the boundary of a face of $G^-$, which is thus divided into two faces in $G$.  We have $v -(e+1)+(f+1) = v-e+f =2$, so $G$ satisfies the theorem.

We have thus shown that Euler's formula works by mathematical induction.  Our arguments about the effects of deletion of edges on faces are common sense:  actually proving these statements as (roughly speaking) Calculus III exercises is quite hard (in fact, far beyond the level of Calculus III!).  [It is fairly easy though certainly technical to show that a vertex of degree 1 and its associated edge removed from a graph does not affect the face count;  what is perfectly obvious but quite nasty to prove is that removing an edge between two points on the boundary of a face removes just one face and not more than one.  You don't need to worry about this, but it is worth mentioning].

\item[$K_5$ is not planar:]  A complete graph with 5 vertices has 10 edges (5 choose 2).  If it is planar, a planar picture must satisfy $5-10+f = 2$, thus $f=7$, and so divides the plane into 7 regions.

The total number of sides of the 7 regions is at least $7\cdot 3 = 21$.  But the total number of sides must be exactly twice the number of edges, that is, 20, which is a contradiction.  (Notice that where we have edges sticking into a region but not dividing it, we still count the two sides as separate).

\item[$K_{3,3}$ is not planar:]  (so the gas water electricity problem has no solution):  $K_{3,3}$ has 6 vertices and 9 edges.  We notice that there are no triangles in a hypothetical planar picture, because there are no 3-cycles in the graph.

We have $6-9+f = 2$, so there must be 5 regions.  There must be at least $5 \cdot 4=20$ total sides of regions (because none of them are triangles, each has at leat 4 sides).  But also there must be exactly $9 \cdot 2 = 18$ total sides, twice the number of edges, and this is impossible.

\item[The five regular polyhedra:]  For now, I refer you to Levin for the proof that there are only five regular polyhedra.  If I am energetic, I may write my own version here later.

\item[Important Theorem:]  In any planar graph, there is at least one vertex of degree less than or equal to 5 (there cannot be a planar graph in which all vertices have degree $\geq 6)$.

\item[Proof:]  This is currently a homework problem.  I will post a proof here (using Euler's formula) after that homework is due.

\end{description}

\end{document}