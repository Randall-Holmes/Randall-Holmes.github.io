\documentclass[12pt]{article}

\title{Homework 9, Math 189, Fall 2023}

\author{Dr Holmes}

\usepackage{amssymb}

\begin{document}

\maketitle

This homework set is due on Monday, Oct 16.

\begin{enumerate}

\item  Find a closed form for each of the following sequences.  The first term is $a_1$ in each case.

\begin{enumerate}


\item  $2,5,8,11,14,\ldots$

\item  $0,3,8,15,24,\ldots$

\item  $4,10, 18, 28, 40,\ldots$

\item  $2,7,24,77,238,\ldots$

\end{enumerate}

\item  Levin, 2.1 exercises, 10

\item  Levin, 2.1 exercises 11  To verify that a recurisive definition of a sequence is valid, what you have to do is show that it has the right initial values and satisfies the recurrence relation.

\item  Find the sum of the first 100 terms of the arithmetic sequence starting $3,7,11,15,\ldots$ using the methods of this section (i.e., not by brute force)

What is the 100th term of this sequence?  (the first is 3).

Write this sum in summation notation (with $\Sigma$, without dots).

\item  Find the sum of the first 20 terms of the geometric sequence starting $1, 1.1, 1.21, 1.331\ldots$, using the methods of this section (i.e, not by brute force)

What is the 100th term of this sequence?  (the first is 3).

Write this sum in summation notation (with $\Sigma$, without dots).

\item  Levin, 2.2 exercises, problem 13

\end{enumerate}


\end{document}