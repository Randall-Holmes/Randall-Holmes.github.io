\documentclass[12pt]{article}

\usepackage{amssymb}



\title{Practice Exam I, Math 189, Fall 2023}

\author{Randall Holmes}

\begin{document}

\maketitle

This is a practice exam paper.  It should be comparable in coverage to the actual exam that will be given on Friday October 6.  It 's longer, but this is not a vice in a practice exam.  The number of questions is probably about right, but there are probably more parts.

You may well be able to find answers to some of these questions in my handouts or in the book.  You will not benefit from the practice exam if you do not attempt it on your own before looking for answers.

If you think you have identified an error in this paper, please contact me immediately.

Solutions will be posted sometime on Thursday.

\newpage

\begin{enumerate}


\item Do two of the three parts.  If you do all three, you may earn extra credit.  The formal rules from the manual of logical style are summarized in an appendix to the test, which you may tear off for convenience.

I believe all three parts have answers in the manual.  Don't look for them:  attempt the problems on your own using the appendix to this test paper for reference.

\begin{enumerate}

\item  Set up and fill out a truth table verifying the validity of the rule of modus tollens, $$\begin{array}{c} A \rightarrow B \\ \neg B \\ \hline \neg A\end{array}$$

Provide an explanation in English which makes it clear that you understand why your table verifies this.


\newpage  

\item Prove $$(A \rightarrow (B \wedge C)) \rightarrow (A \rightarrow C)$$ using the rules in the manual.

\newpage  

\item Verify the validity of the rule of constructive dilemma,  $$\begin{array}{c} P \vee Q \\ P \rightarrow R \\ Q \rightarrow S \\ \hline R \vee S \end{array}$$ with a proof in the style of the manual.
This question is NOT asking for a truth table.

\end{enumerate}

\newpage

\item  If $x$ and $y$ are integers, we say that $y$ is divisible by $x$, written $x|y$ just in case there is an integer $k$ such that $xk=y$.

Write out a proof of the statement ``If $x,y,d$ are integers and $d|x$ and $d|y$, then $d|(x-y)$", in the style of the third homework.  You do not need to (but are certainly allowed to)
write quantifiers explicitly.  Make sure that you mention when quantities you introduce are integers, and verify that quantities that you compute are integers.  Your work should not mention the concept of division at all.

\newpage

\item  I'm conscious that in this problem I am asking you to do things different from the homework.  You should be able to do them, with thought.

\begin{enumerate}
\item  Use Venn diagrams to verify the identity $$A \setminus (B \cup C) = (A \setminus B) \cap (A \setminus C).$$

Do this by drawing a Venn diagram showing how each side of the equation is computed (provide a key to any shadings you use and outlining the set which is the result of the calculation):
the outlined part of the diagram for each side should be the same.

\newpage

\item  Fill in the appropriate relation between the indicated items ($\in$, $\subseteq$, both or neither)

We officially assume here that numbers are not sets.

\begin{enumerate}

\item $\emptyset \verb ___ \{1,2,3\}$

\item $\{1,2\} \verb ___ \{2,\{1,2\},1\}$

\item $1 \verb ___ \{\{1,2\}\}$

\item  $3 \verb ___ \mathbb N$

\end{enumerate}

\end{enumerate}

\newpage

\item In each part, I give the domain, codomain, and graph of a relation (the graph in two line notation).  In each case, tell me whether what you are given is a function (and explain, mentioning specific domain and/or codomain elements), and if it is a function, whether it is an injection  (and explain, mentioning specific domain and/or codomain elements), and whether it is a surjection  (and explain, mentioning specific domain and/or codomain elements).  If it has an inverse function, write a representation of the inverse in the same notation, with the domain elements in the top row in standard order.

\begin{enumerate}

\item  domain $\{1,2,3\}$, codomain $\{a,b\}$, graph $\left(\begin{array}{cccc}1 & 1 & 2 & 3 \\ b & a & a & b\end{array}\right)$

\vspace{1 in}

\item   domain $\{1,2,3\}$, codomain $\{a,b,c\}$, graph $\left(\begin{array}{cccc}1 & 2 & 3 \\ a & a & b\end{array}\right)$

\vspace{1 in}

\item  domain $\{1,2,3\}$, codomain $\{a,b,c\}$, graph $\left(\begin{array}{cccc}1 & 2 & 3 \\ c & a & b\end{array}\right)$

\end{enumerate}

\newpage

\item 1.1  A certain state has license plates consisting of three letters followed by three digits.  In each part, give a numerical answer and the supporting calculation.

\begin{enumerate}

\item  How many plates are possible if no additional restrictions are imposed?

\item  How many plates are possible if no letter or digit can appear more than once on the plate?

\item  How many plates contain at least one A (with no other restrictions)?

\item  How many plates are possible if no letter or digit is immediately followed by the same letter or digit?

\end{enumerate}

\newpage

\item Do all parts.

\begin{enumerate}

\item  Write out the ninth row of Pascal's triangle without constructing the rows above it.  Your paper should show work supporting this using only multiplication and division.

\item  If a committee has twenty members, 11 men and 9 women, how many ways are there to appoint a subcommittee with nine members which is gender balanced (as much as possible for an odd number of members:  the numbers of men and women in the subcommittee should not differ by more than one).  Hint:  there are two cases, so there will be some addition.

\item How many anagrams are there of the word TRATTORIA?


\end{enumerate}

\newpage

\item Four problems are presented.  They are all of the form ``make $k$ choices from $n$ alternatives":  in some cases order matters, in some it does not, in some cases
repetitions are allowed, in some cases they are not.  For each problem, do the (short) calculation of the answer and identify which kind of problem it is (order matters or not, repetitions allowed or not).

\begin{enumerate}

\item  How many ways can you form a ``word" of seven letters from the 26-letter alphabet?

\item  You bought 10 books at the library book sale, all different, but only seven of them will fit on the shelf at the head of your bed.  In how many ways can you put seven of the books on the shelf?

\item  If you have a large supply of pennies, nickels, dimes and quarters in a jar, reach in and pull out a handful of ten, how many possible outcomes are there?

\item  A convention of 100 members is to choose a three member executive committee.  In how many ways can they do this?

\end{enumerate}

\newpage

\item  Two parts:

\begin{enumerate}
\item The 24 students at my favorite exclusive prep school are each sweating through English, Russian, and/or Math.

10 students take English, 15 take Russian, and 13 (the best!) take Math

5 take English and Russian, 3 take English and Math, 8 take Russian and Math

How many students are suffering through all three?

\item Write down an expression for the number of elements of the union of four sets $A,B,C,D$ in terms of the sizes of the sets and their various intersections (to be lectured on Monday the 2nd)

\end{enumerate}

\end{enumerate}

\newpage

\section{Proof strategies from the manual of logical style}

\subsection{Conjunction}

In this section we give rules for handling ``and''.  These are so simple that we barely notice that they exist!

\subsection{Proving a conjunction}

To prove a statement of the form $A \wedge B$, first prove $A$, then prove
$B$.

This strategy can actually be presented as a rule of inference:

$$\begin{array}{l} A \\ B \\ \hline A \wedge B \end{array}$$

If we have hypotheses $A$ and $B$, we can draw the conclusion $A \wedge B$:  so a strategy for proving $A \wedge B$ is to first prove $A$ then prove $B$.  This gives a proof in two parts, but notice that there are no assumptions being introduced in the two parts:  they are not separate cases.

This rule is called ``conjunction".

\subsubsection{Using a conjunction}

If we are entitled to assume $A \wedge B$, we are further entitled to assume $A$ and $B$.  This can be summarized in two rules of inference:

$$\begin{array}{l} A \wedge B \\ \hline A \end{array}$$

$$\begin{array}{l} A \wedge B \\ \hline B \end{array}$$

This has the same flavor as the rule for proving a conjunction:  a conjunction just breaks apart into its component parts.

This rule is called ``simplification".

\subsection{Implication}

In this section we give rules for implication.  There is a single basic rule for implication in each subsection, and then some derived rules which also involve negation, based on the equivalence of an implication with its contrapositive.  These are called derived rules because they can actually be justified in terms of the basic rules.  We like the derived rules, though, because they allow us to write proofs more compactly.

\subsubsection{Proving an implication}

\begin{description}

\item[The basic strategy for proving an implication:]  To prove $A \rightarrow B$, add $A$ to your list of assumptions and prove $B$; if you can do this, $A \rightarrow B$ follows without the additional assumption.

Stylistically, we indent the part of the proof consisting of statements depending on the additional assumption $A$:  once we are done proving $B$ under the assumption and thus proving $A \rightarrow B$ without the assumption, we discard the assumption and thus no longer regard the indented group of lines as proved.

This rule is called ``deduction".

\item[The indirect strategy for proving an implication:]  To prove $A \rightarrow B$, add $\neg B$ as a new assumption and prove $\neg A$:  if you can do this, $A \rightarrow B$ follows without the additional assumption.  Notice that this amounts to proving $\neg B \rightarrow \neg A$ using the basic strategy, which is why it works.

This rule is called ``proof by contrapositive" or ``indirect proof".

\end{description}

\subsubsection{Using an implication}

\begin{description}

\item[modus ponens:]  If you are entitled to assume $A$ and you are entitled to assume $A \rightarrow B$, then 
you are also entitled to assume $B$.  This can be written as a rule of inference:

$$\begin{array}{l} A \\A \rightarrow  B \\ \hline B \end{array}$$

\item[when you just have an implication:]  If you are entitled to assume $A \rightarrow B$, you may at any time adopt $A$ as a new goal, for the sake of proving $B$, and as soon as you have proved it, you also are entitled to assume $B$.  Notice that no assumptions are introduced by this strategy.  This proof strategy is just a restatement of the rule of {\em modus ponens\/} which can be used to suggest the way to proceed when we have an implication without its hypothesis.

\item[modus tollens:]  If you are entitled to assume $\neg B$ and you are entitled to assume $A \rightarrow B$, then 
you are also entitled to assume $\neg A$.  This can be written as a rule of inference:

$$\begin{array}{l} A \rightarrow  B \\ \neg B \\ \hline \neg A \end{array}$$

Notice that if we replace $A \rightarrow B$ with the equivalent contrapositive $\neg B \rightarrow \neg A$, then this becomes an example of modus ponens.  This is why it works.

\item[when you just have an implication:]  If you are entitled to assume $A \rightarrow B$, you may at any time adopt $\neg B$ as a new goal, for the sake of proving $\neg A$, and as soon as you have proved it, you also are entitled to assume $\neg A$.  Notice that no assumptions are introduced by this strategy.  This proof strategy is just a restatement of the rule of {\em modus tollens\/} which can be used to suggest the way to proceed when we have an implication without its hypothesis.

\end{description}

\subsection{Absurdity}

The symbol $\perp$ represents a convenient fixed false statement.   The point of having this symbol is that it makes the rules for negation much cleaner.

\subsubsection{Proving the absurd}

We certainly hope we never do this except under assumptions!  If we are entitled to assume $A$ and we are entitled to assume $\neg A$, then we are entitled to assume $\perp$.  Oops!  This rule is called {\em contradiction\/}.

$$\begin{array}{r} A \\ \neg A \\ \hline \perp \end{array}$$

\subsubsection{Using the absurd}

We hope we never really get to use it, but it is very useful.  If we are entitled to assume $\perp$, we are further entitled to assume $A$ (no matter what $A$ is).  From a false statement, anything follows.  We can see that this is valid by considering the truth table for implication.

This rule is called ``absurdity elimination" or ``ex falso".

\subsection{Negation}

The rules involving just negation are stated here.  We have already seen derived rules of implication using negation, and we will see derived rules of disjunction using negation below.

\subsubsection{Proving a negation}

\begin{description}

\item[direct proof of a negation (basic):]  To prove $\neg A$, add $A$ as an assumption and prove $\perp$.  If you complete this proof of $\perp$ with the additional assumption, you are entitled to conclude $\neg A$ without the additional assumption (which of course you now want to drop like a hot potato!).  This is the direct proof of a negative statement:  proof by contradiction, which we describe next, is subtly different.

Call this rule ``negation introduction".  You won't be marked off if you call it ``reductio ad absurdum", but it is not quite the same thing.

\item[proof by contradiction (derived):]  To prove a statement $A$ of any logical form at all, assume $\neg A$ and prove $\perp$.
If you can prove this under the additional assumption, then you can conclude $A$ under no additional assumptions.  Notice that the proof by contradiction of $A$ is a direct proof of the statement $\neg\neg A$, which we know is logically equivalent to $A$; this is why this strategy works.

Call this rule ``reductio ad absurdum".

\end{description}

\subsubsection{Using a negation:}

\begin{description}

\item[double negation (basic):]  If you are entitled to assume $\neg\neg A$, you are entitled to assume $A$.  Call this rule ``double negation elimination".

\item[contradiction (basic):]  This is the same as the rule of contradiction stated above under proving the absurd:
if you are entitled to assume $A$ and you are entitled to assume $\neg A$, you are also entitled to assume $\perp$.  You also feel deeply queasy.

$$\begin{array}{r} A \\ \neg A \\ \hline \perp \end{array}$$

\item[if you have just a negation:] If you are entitled to assume $\neg A$, consider adopting $A$ as a new goal:  the point of this is that from $\neg A$ and $A$ you would then be able to deduce $\perp$ from which you could further deduce whatever goal $C$ you are currently working on.  This is especially appealing as soon as the current goal to be proved becomes $\perp$, as the rule of contradiction is the only way there is to prove $\perp$.

\end{description}

\subsection{Disjunction}

In this section, we give basic rules for disjunction which do not involve negation, and derived rules which do.  The derived rules can be said to be the default strategies for proving a disjunction, but they {\em can\/} be justified using the seemingly very weak basic rules (which are also very important rules, but often used in a ``forward" way as rules of inference).   The basic strategy for using an implication (proof by cases) is of course very often used and very important.  The derived rules in this section are justified by the logical equivalence of $P \vee Q$ with both $\neg P \rightarrow Q$ and $\neg Q \rightarrow P$:  if they look to you like rules of implication, that is because somewhere underneath they are.

\subsubsection{Proving a disjunction}

\begin{description}

\item[the basic rule for proving a disjunction (two forms):]  To prove $A \vee B$, prove $A$.   Alternatively, to prove $A \vee B$, prove $B$.
You do {\em not\/} need to prove both (you should not expect to be able to!)

This can also be presented as a rule of inference, called {\em addition\/}, which comes in two different versions.

$$\begin{array}{r} A \\ \hline A \vee B \end{array}$$

$$\begin{array}{r} B \\ \hline  A \vee B \\ \end{array}$$

\item[the default rule for proving a disjunction (derived, two forms):]   To prove $A \vee B$, assume $\neg B$ and attempt to prove $A$.  If $A$ follows with the additional assumption, $A \vee B$ follows without it.  

Alternatively (do not do both!):  To prove $A \vee B$, assume $\neg A$ and attempt to prove $B$.  If $B$ follows with the additional assumption, $A \vee B$ follows without it.

Notice that the proofs obtained by these two methods are proofs of $\neg B \rightarrow A$ and $\neg A \rightarrow B$ respectively, and both of these are logically equivalent to $A \vee B$.  This is why the rule works.  Showing that this rule can be derived from the basic rules for disjunction is moderately hard.

Call both of these rules  ``alternative elimination".


\end{description}


\subsubsection{Using a disjunction}

\begin{description}

\item[proof by cases (basic):]  If you are entitled to assume $A \vee B$ and you are trying to prove $C$, first assume $A$ and prove $C$ (case 1);
then assume $B$ and attempt to prove $C$ (case 2).  

Notice that the two parts are proofs of $A \rightarrow C$ and $B \rightarrow C$,
and notice that $(A \rightarrow C) \wedge (B \rightarrow C)$ is logically equivalent to $(A \vee B) \rightarrow C$ (this can be verified using a truth table).

This strategy is very important in practice.


\item[disjunctive syllogism (derived, various forms):]  If you are entitled to assume $A \vee B$ and you are also entitled to assume $\neg B$, you are further entitled to assume $A$.  Notice that replacing $A \vee B$ with the equivalent $\neg B \rightarrow A$ turns this into an example of modus ponens.


If you are entitled to assume $A \vee B$ and you are also entitled to assume $\neg A$, you are further entitled to assume $B$.  Notice that replacing $A \vee B$ with the equivalent $\neg A \rightarrow B$ turns this into an example of modus ponens.

Combining this with double negation gives further forms:  from $B$ and $A \vee \neg B$ deduce $A$, for example.

Disjunctive syllogism in rule format:

$$\begin{array}{r}  A \vee B \\ \neg B \\ \hline A \end{array}$$

$$\begin{array}{r}  A \vee B \\ \neg A \\ \hline B \end{array}$$

Some other closely related forms which we also call ``disjunctive syllogism":

$$\begin{array}{r}  A \vee \neg B \\ B \\ \hline A \end{array}$$

$$\begin{array}{r}  \neg A \vee B \\ A \\ \hline B \end{array}$$

\end{description}

\subsection{Biconditional}

Some of the rules for the biconditional are derived from the definition of $A \leftrightarrow B$ as $(A \rightarrow B) \wedge (B \rightarrow A)$.  There is a further very powerful rule allowing us to use biconditionals to justify replacements of one expression by another.

\subsubsection{Proving biconditionals}

\begin{description}

\item[the basic strategy for proving a biconditional:]  To prove $A \leftrightarrow B$, first assume $A$ and prove $B$; then (finished with the first assumption) assume $B$ and prove $A$.  Notice that the first part is a proof of $A \rightarrow B$ and the second part is a proof of $B \rightarrow A$.

Call this rule ``biconditional introduction".

\item[derived forms:]  Replace one or both of the component proofs of implications with the contrapositive forms.  For example one could first
assume $A$ and prove $B$, then assume $\neg A$ and prove $\neg B$ (changing part 2 to the contrapositive form).

\end{description}

\subsubsection{Using biconditionals}  The rules are all variations of modus ponens and modus tollens.   Call them biconditional modus ponens (bimp)
or biconditional modus tollens (bimt) as appropriate.

If you are entitled to assume $A$ and $A \leftrightarrow B$, you are entitled to assume $B$.

If you are entitled to assume $B$ and $A \leftrightarrow B$, you are entitled to assume $A$.

If you are entitled to assume $\neg A$ and $A \leftrightarrow B$, you are entitled to assume $\neg B$.

If you are entitled to assume $\neg B$ and $A \leftrightarrow B$, you are entitled to assume $\neg A$.

These all follow quite directly using modus ponens and modus tollens and one of these rules:

If you are entitled to assume $A \leftrightarrow B$, you are entitled to assume $A \rightarrow B$.

If you are entitled to assume $A \leftrightarrow B$, you are entitled to assume $B \rightarrow A$.

The validity of these rules is evident from the definition of a biconditional as a conjunction.


\end{document}