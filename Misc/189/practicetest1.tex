\documentclass[12pt]{article}

\usepackage{amssymb}



\title{Practice Exam I, Math 189, Fall 2023}

\author{Randall Holmes}

\begin{document}

\maketitle

This is a practice exam paper.  It should be comparable in coverage to the actual exam that will be given on Friday October 6.  It 's longer, but this is not a vice in a practice exam.  The number of questions is probably about right, but there are probably more parts.

You may well be able to find answers to some of these questions in my handouts or in the book.  You will not benefit from the practice exam if you do not attempt it on your own before looking for answers.

If you think you have identified an error in this paper, please contact me immediately.

Solutions will be posted sometime on Thursday.

\newpage

\begin{enumerate}


\item Do two of the three parts.  If you do all three, you may earn extra credit.  The formal rules from the manual of logical style are summarized in an appendix to the test, which you may tear off for convenience.

I believe all three parts have answers in the manual.  Don't look for them:  attempt the problems on your own using the appendix to this test paper for reference.

\begin{enumerate}

\item  Set up and fill out a truth table verifying the validity of the rule of modus tollens, $$\begin{array}{c} A \rightarrow B \\ \neg B \\ \hline \neg A\end{array}$$

Provide an explanation in English which makes it clear that you understand why your table verifies this.


\newpage  

\item Prove $$(A \rightarrow (B \wedge C)) \rightarrow (A \rightarrow C)$$ using the rules in the manual.

\newpage  

\item Verify the validity of the rule of constructive dilemma,  $$\begin{array}{c} P \vee Q \\ P \rightarrow R \\ Q \rightarrow S \\ \hline R \vee S \end{array}$$ with a proof in the style of the manual.
This question is NOT asking for a truth table.

\end{enumerate}

\newpage

\item  If $x$ and $y$ are integers, we say that $y$ is divisible by $x$, written $x|y$ just in case there is an integer $k$ such that $xk=y$.

Write out a proof of the statement ``If $x,y,d$ are integers and $d|x$ and $d|y$, then $d|(x-y)$", in the style of the third homework.  You do not need to (but are certainly allowed to)
write quantifiers explicitly.  Make sure that you mention when quantities you introduce are integers, and verify that quantities that you compute are integers.  Your work should not mention the concept of division at all.

\newpage

\item  I'm conscious that in this problem I am asking you to do things different from the homework.  You should be able to do them, with thought.

\begin{enumerate}
\item  Use Venn diagrams to verify the identity $$A \setminus (B \cup C) = (A \setminus B) \cap (A \setminus C).$$

Do this by drawing a Venn diagram showing how each side of the equation is computed (provide a key to any shadings you use and outlining the set which is the result of the calculation):
the outlined part of the diagram for each side should be the same.

\newpage

\item  Fill in the appropriate relation between the indicated items ($\in$, $\subseteq$, both or neither)

We officially assume here that numbers are not sets.

\begin{enumerate}

\item $\emptyset \verb ___ \{1,2,3\}$

\item $\{1,2\} \verb ___ \{2,\{1,2\},1\}$

\item $1 \verb ___ \{\{1,2\}\}$

\item  $3 \verb ___ \mathbb N$

\end{enumerate}

\end{enumerate}

\newpage

\item In each part, I give the domain, codomain, and graph of a relation (the graph in two line notation).  In each case, tell me whether what you are given is a function (and explain, mentioning specific domain and/or codomain elements), and if it is a function, whether it is an injection  (and explain, mentioning specific domain and/or codomain elements), and whether it is a surjection  (and explain, mentioning specific domain and/or codomain elements).  If it has an inverse function, write a representation of the inverse in the same notation, with the domain elements in the top row in standard order.

\begin{enumerate}

\item  domain $\{1,2,3\}$, codomain $\{a,b\}$, graph $\left(\begin{array}{cccc}1 & 1 & 2 & 3 \\ b & a & a & b\end{array}\right)$

\vspace{1 in}

\item   domain $\{1,2,3\}$, codomain $\{a,b,c\}$, graph $\left(\begin{array}{cccc}1 & 2 & 3 \\ a & a & b\end{array}\right)$

\vspace{1 in}

\item  domain $\{1,2,3\}$, codomain $\{a,b,c\}$, graph $\left(\begin{array}{cccc}1 & 2 & 3 \\ c & a & b\end{array}\right)$

\end{enumerate}

\newpage

\item 1.1  A certain state has license plates consisting of three letters followed by three digits.  In each part, give a numerical answer and the supporting calculation.

\begin{enumerate}

\item  How many plates are possible if no additional restrictions are imposed?

\item  How many plates are possible if no letter or digit can appear more than once on the plate?

\item  How many plates contain at least one A (with no other restrictions)?

\item  How many plates are possible if no letter or digit is immediately followed by the same letter or digit?

\end{enumerate}

\newpage

\item Do all parts.

\begin{enumerate}

\item  Write out the ninth row of Pascal's triangle without constructing the rows above it.  Your paper should show work supporting this using only multiplication and division.

\item  If a committee has twenty members, 11 men and 9 women, how many ways are there to appoint a subcommittee with nine members which is gender balanced (as much as possible for an odd number of members:  the numbers of men and women in the subcommittee should not differ by more than one).  Hint:  there are two cases, so there will be some addition.

\item How many anagrams are there of the word TRATTORIA?


\end{enumerate}

\newpage

\item Four problems are presented.  They are all of the form ``make $k$ choices from $n$ alternatives":  in some cases order matters, in some it does not, in some cases
repetitions are allowed, in some cases they are not.  For each problem, do the (short) calculation of the answer and identify which kind of problem it is (order matters or not, repetitions allowed or not).

\begin{enumerate}

\item  How many ways can you form a ``word" of seven letters from the 26-letter alphabet?

\item  You bought 10 books at the library book sale, all different, but only seven of them will fit on the shelf at the head of your bed.  In how many ways can you put seven of the books on the shelf?

\item  If you have a large supply of pennies, nickels, dimes and quarters in a jar, reach in and pull out a handful of ten, how many possible outcomes are there?

\item  A convention of 100 members is to choose a three member executive committee.  In how many ways can they do this?

\end{enumerate}

\newpage

\item  Two parts:

\begin{enumerate}
\item The 24 students at my favorite exclusive prep school are each sweating through English, Russian, and/or Math.

10 students take English, 15 take Russian, and 13 (the best!) take Math

5 take English and Russian, 3 take English and Math, 8 take Russian and Math

How many students are suffering through all three?

\item Write down an expression for the number of elements of the union of four sets $A,B,C,D$ in terms of the sizes of the sets and their various intersections (to be lectured on Monday the 2nd)

\end{enumerate}

\end{enumerate}




\end{document}