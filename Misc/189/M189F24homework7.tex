\documentclass[12pt]{article}

\title{Math 189, Fall 2024, Homework 7}

\author{Randall Holmes}

\usepackage{amssymb}

\begin{document}

\maketitle

This homework set is due on Wednesday October 16.

\begin{enumerate}

\item   Consider the sequence 1,4,7,10,13$\ldots$.

Give a closed form definition for this sequence (assuming the indexing starts at 0).

Give a recursive definition of the sequence (assuming the indexing starts at 0).

Write these two definitions again assuming the indexing starts at 1.

\item   Do Levin section 2.1 problem 9. 

\item   Do Levin 2.1 problem 10.

\item  Write 25+36+49+64+81+100 in summation notation in the natural way.

Write out $\sum_{i=3}^8 i^3+i^2$ as a sum.

\item  Do 2.1 problem 18 in Levin.


\item  How many terms are there in the arithmetic sequence $$2,5,8,\ldots,152?$$

Compute  $$2+5+8+\ldots+152$$ using our methods for summing arithmetic sequences.


\item  Express the decimal number 3.3333333 as the sum of a geometric sequence in the natural way.  What is the first term of this sequence?  What is the common ratio of this sequence?  Write the sequence in summation notation.


\item  Verify the calculus formula $1+x+x^2+\ldots = \frac{1}{1-x} (|x|<1)$ using the formula for the sum of a geometric sequence.  Start by computing the sum
$1+x+x^2+\ldots+x^n$ using the formula, then explain what the fact that $|x|<1$ does for you (what happens when $n$ gets large?)


\item  Do Levin, 2.2, problem 14


\end{enumerate}


\end{document}