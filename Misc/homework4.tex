\documentclass[12pt]{article}

\title{Homework 4:  Counting Word Problems}

\author{Randall Holmes}

\begin{document}

\maketitle

Please use the space I give you on each page to write out your thinking in English, or at the very least write computations which make it clear how you came up with the answer.  Some English combined with appropriate structured computations is best.

These are due on Wednesday the 14th, to give you plenty of chances to ask questions.

\newpage

\begin{enumerate}

\item (modelled on a problem we did in class)

Alma needs to take two classes to complete her degree.  Classes are offered in morning and afternoon slots.  She is interested in a math course, a business course, and an engineering drawing course.

There are two sections of math, three of business, and five of engineering drawing in the morning.

There is one section of math, five sections of business, and three sections of engineering drawing in the afternoon.

In how many ways can she fill out her schedule to take two different courses next term and finish her degree?

\newpage

\item (modelled on a problem we did in class)

How many four digit numbers contain both a 1 and a 2, and neither a 0 or a 9?

\newpage

\item  A committee with 17 members has 10 male members and 7 female members.  They are going to select an Equity Committee, which must be ``gender balanced":  the number of men and the number of woman will either be the same or differ by one (so that a committee with an odd number of members can be gender balanced).

In how many ways can the Equity Committee be chosen if it is to have four members?  (hint:  use binomial coefficients and the multiplication principle).

In how many ways can the Equity Committee be chosen if it is to have five members (hint: you will need to use the addition principle as well in this case).

\newpage

\item  An organization has 8 math majors, 12 computer science majors, and 6 science majors.

In how many ways can the organization choose a president, vice president, and secretary in each of the following situations?:

\begin{enumerate}
\item   The three officers must all be of the same major.

\item   At least one of the officers must be a science major.

\item  Either the president or the vice president must be a math major.

\end{enumerate}

For some of these, I would think about the complement (think of the number of things for which the condition doesn't hold).

\newpage

\item  We want to count the number of five card hands (considered as five cards drawn {\em in order\/} from a standard deck of cards) in which the first card has the same rank as the last (the first and last card differ only in suit).

The book suggest the following algorithm:  50 choices for the first card, 49 for the second, 48 for the third, 47 for the fourth,
and 3 for the fifth (since it must be of the same rank as the first card), so $(50)(49)(48)(47)(3)$ ways.  This is wrong:
explain why.

Compute the number of such five card hands correctly.

\newpage

\item  In how many ways can four married couples be seated at a round table in such a way that spouses sit together?

Notice that the round table means that we do not have a special starting point, as we would if they were standing in a line.  How do we adjust for this?

\newpage

\item  Your friend spins a spinner with three colors (green, red, blue) and you record the results of six consecutive spins.

How many possible results are there?

How many possible results are there with exactly two of each color?

\newpage

\item  In how many ways can a group of 15 children divide themselves into three teams of five, the Lions, the Tigers, and the Bears?

In how many ways can a group of 15 children divide themselves into three teams of five?  This is not the same question.

\newpage

\item  License plates for a certain state have three letters followed by four digits.

How many plates are possible with no restrictions?

How many plates are possible if no letter can be repeated?

How many plates are possible if no letter or digit can be immediately followed by the same letter or digit?

How many plates contain exactly one 8?

How many plates contain no X's?

\newpage

\item  A combination lock has five digits on it, 1 to 5.  A combination consists of three items in order, each item being either a digit
or two digits pressed at the same time.  No digit will appear more than once in a combination.  How many possible combinations for the lock are there?

This problem will combine use of binomial coefficients, the multiplication principle and the addition principle.

\end{enumerate}

\end{document}