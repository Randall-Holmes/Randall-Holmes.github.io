\documentclass[12pt]{article}

\title{Math 287 Spring 2023 Test II}

\author{Dr Holmes}

\begin{document}

\maketitle

This is a takehome exam.  It is due at 11:55 pm on Thursday, April  20th.

Please do not consult anyone but the instructor as you do these problems.

Don't hesitate to ask me if you have questions about the instructions on any of these problems.  Do contact me quickly if you suspect an error or typo in the instructions.

\newpage

\begin{enumerate}

\item  Consider the following rule:

$$\begin{array}{c}

P \vee \neg Q \\

P \rightarrow R \\ \hline

\neg Q \vee R

\end{array}$$

Show that this rule is valid in three different ways:

Set up a truth table demonstrating that it is valid.  Essential to this is stating accurately what the properties of rows and/or columns in the truth
table are which illustrate that the rule is valid.

Prove it in the system of the manual of logical style.

Prove it in Marcel (you can turn in the demonstration in either of the formats allowed in the lab).

\newpage

\item  Use the table method taught in lecture to compute ${\tt gcd}(1180,514)$ and at the same time to find integers $x$ and $y$ such that
$1180x + 514y = {\tt gcd}(1180,514)$.

Your work should make it clear that you know what $x$ is, what $y$ is, and what the gcd is.

\newpage

\item  This question is about the Fibonacci sequence 1,1,2,3,5,8,13$\ldots$

Choose two successive Fibonacci numbers $F_n$ and $F_{n+1}$ for $n$ moderately large (numbers past the end of the part of the sequence shown above) and use the extended Euclidean algorithm to compute $x$ and $y$ such that ${\tt gcd}(F_n,F_{n+1}) = F_nx + F_{n+1}y$.

Then do this again for a larger $n$ (for a larger pair of successive Fibonacci numbers).

Describe any patterns that you notice (there will be some points reserved for actually describing some patterns, though most credit will reside in doing two EEA computations).  You may be able to conjecture one or two theorems about the Fibonacci sequence after doing your calculations.

Extra credit if you can give a formal proof by induction (or using the well-ordering principle) that for any $n$,  ${\tt gcd}(F_n,F_{n+1})=1$.  There are ways of doing this by talking about the EEA computation, but it can be done directly in fairly simple ways as well (hint:  one of the lemmas we used to prove that our algorithm for proving the Euclidean algorithm correct can be used).

\newpage

\item  Construct the addition and multiplication tables for mod 11 arithmetic and mod 9 arithmetic.

For one of these systems, you can also construct a table of multiplicative inverses  of all nonzero remainders.  For the other, you can't:  support this by describing a number in that system which is not 0 and does not have a multiplicative inverse
(and say why it doesn't have one, using facts about the table).

\newpage

\item  Compute the multiplicative inverse of 19 in mod 211 arithmetic.

Show and explain supporting calculations (explain how the EEA formula $211x+19y = {\tt gcd}(211,19)$ shows that a certain number is the multiplicative inverse of 19; this is a brief remark!)

Include a check that the product of 19 and the number you compute actually is congruent to 1 mod 211.

\newpage

\item  Show full calculations to find $37^{500}$ in mod 1000 arithmetic.

Show full calculations to find $12^{51}$ in mod 15 arithmetic.

Show full calculations in the style I used in lecture.  The spreadsheet which I will share with you will allow you to check your work, but it does not display the
actual calculations supporting the steps in a style I can read, and that is expected (watch the videos to see what I did).

\newpage

\item  Chinese remainder theorem:

Find the general solution to the system of equations

$x {\tt mod} 15 = 14$

$x {\tt mod} 37 = 12$

and identify the smallest nonnegative solution.

You should do this using calculations in modular arithmetic as demonstrated in class.

\newpage

\end{enumerate}

\end{document}