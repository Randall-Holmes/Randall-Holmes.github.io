\documentclass[12pt]{article}

\title{Math 275 Practice Test II, Fall 2020}

\author{Dr. Holmes}

\begin{document}

\maketitle

This is a somewhat edited version of parts of old tests.  It is much longer than a single test, which will have no more than a question or two per section.  
You may hand it in for a homework check (I don't expect to mark it);  it is really provided as a study resource.

Solutions to the Practice Test will be posted at about the same time as the Test itself.  I will not be unwilling to give away answers from this review, particularly since it has so many questions.

\newpage

\begin{enumerate}

\item 14.3 partial derivatives

Let $f(x,y) = x^2y^3 + xy^4$.  Compute the first partials of this function (two functions) and the second partials of this function (four functions).  

What fact about your second partials is an example of Clairaut's thoerem?

\newpage

\item	14.3 

Let $h(x, y) = x^2y + x^3 + y^2$. Compute and clearly label the first  and second partial derivatives of $h(x, y)$. Show that the mixed partials are equal (your work should reveal that you know what this means).

\newpage

\item 14.4 

Give an equation for the tangent plane to $z = x^2y+y$ at the point $(1, 2, 4)$.

\newpage

\item 14.4
 
Give an approximation to	$\sqrt{(3.9)(25.1)}$ by 14.4 methods.
Hint:  you will be using the linearization of the function $k(x, y) =
\sqrt{xy}$...at what nice point?


\newpage

\item  14.4 tangent planes and linearization

Give the equation for the tangent plane to $z=\frac{x^2}y$ at $(1,2,.5)$.

\vspace{1.5 in}

Give the linear function $L(x,y)$ which best approximates $\frac{x^2}y$ when $x$ is close to 1 and $y$ is close to 2.  You might notice that this is almost the same question.


\vspace{1.5 in}

Give the linear approximation to $\frac{1.02^2}{1.97}$ using this function.

\newpage

\item  14.6 gradient and directional derivative

Compute the gradient of the function $g(x,y) = x^2y-xy^2$ at the point $(1,3)$.  Compute the directional derivative of this function in the direction of the vector $\left<3,4\right>$.

\vspace{1.5 in}

In what direction does this function increase most rapidly at $(1,3)$?  What is the rate of change?

\newpage

\item  14.5 chain rule

We consider the usual formulas $x=r\cos(\theta)$ and $y=r\sin(\theta)$ used in the definition of polar coordinates (but this question is not really about polar coordinates).

Let $f(x,y) = x^2y$.  

Set up the instances of the chain rule required to compute $\frac{\partial f}{\partial r}$ and $\frac{\partial f}{\partial \theta}$.

Compute  $\frac{\partial f}{\partial r}$ and $\frac{\partial f}{\partial \theta}$.  You may leave your answer in terms of all four letters, $x,y,r,\theta$.

\newpage


\item  14.7  critical points and local extrema

Find all critical points of the function $f(x,y) = x^3+y^3-3x-27y$ and classify them as local maxima, local minima, or saddle points using the Second Derivative Test.


\newpage

\item  14.8  Lagrange multipliers

Maximize and minimize $f(x,y) = xe^y$ subject to the constraint $$x^2+y^2=2.$$  A large part of the credit will be for setting it up correctly, but the algebra is actually not bad at all.

\newpage

\item  15.2

  Sketch the region of integration of the iterated integral $\int_0^2\int_{x^2}^4 \, y \,dy\,dx$.

\vspace{1.5 in}

Change the order of integration and set up the integral as $\int_{??}^{??} \int_{??}^{??}\,y\,dx\,dy$.

\vspace{1.5 in}

Evaluate both integrals and check that the same value is obtained

\newpage

\item 15.2

A thin plate occupying the triangular region with corners at (-1,0), (1,0), and (0,2) has density $y^2$ at each point $(x,y)$ in this region.

Determine the mass of this object.  This will require you to set up and compute a suitable integral (I tell you for free that this is the same problem
as finding the volume  under $z=y^2$ over the given triangle in the $xy$ plane:  the book and Dr Kaiser may have discussed examples of this kind, but I have not).

\newpage

\item  Chain Rule  14.5

Let $y = u^2 + sin(uv)$, where $u= st + e^t$ and $v = \sqrt{1+s+t^2}$.

Write out the form of the Chain Rule needed to compute the partial derivative of $y$ with respect to $s$.

Compute the partial derivative of $y$ with respect to $s$.  You may
leave $u$'s and $v$'s in your answer.

\newpage 

\item 14.6 

Compute the gradient of the function $f(x,y,z) = xy^2z^3$.  

Compute the directional derivative of the function $f(x,y,z) =
xy^2z^3$ at (1,1,1) in the direction of the vector ${\bf i} + 2{\bf j}
+ 2{\bf k}$.  You should set up your calculation as the dot product of
a value of the gradient of $f$ with an appropriate unit vector.

Determine the equation of the tangent plane to the graph of
$xy^2z^3=1$ at (1,1,1).

\newpage 

\item 14.7

 Find all critical points of $f(x,y)=-1+2xy-2y-2x^2+4x-y^2$ and classify them
as local maxima, local minima, or saddle points.

\newpage

\item 14.8

Use the method of Lagrange multipliers to find the closest point
to the origin on the plane $x+2y+3z=2$.  Hint: find the critical point
for $x^2+y^2+z^2$ (the square of the distance from the origin) subject
to the constraint given by the equation of the plane.  There is only
one critical point, and you do not have to verify that it is actually
a minimum (this is obvious from geometry).

\newpage 

\item  15.2

Find the volume of the solid bounded below by the triangle in the $xy$
plane with corners (0,0,0), (0,1,0), and (1,1,1) (and on the sides by
the vertical planes through the sides of this triangle), and above by
$z=xy^2$.  Set up the double integral for this problem in both orders
of integration; you only need to evaluate one of the two forms,
though.

\newpage 

\item  15.1  

Evaluate $$\int_0^1\int_1^2\,xy\,dy\,dx.$$   Reverse the order of integration and evaluate it again.

\newpage 

\item  15.2

Set up and evaluate the integral of the function $f(x,y) = xy^2$ over the region bounded by the $x$-axis, the curve $y=x^2$, and the line $x=2$.

\newpage

\item  15.2 


Evaluate the iterated integral $$\int_0^1\int_{2x}^1\,x^2y\,dy\,dx.$$. 15.2

Draw the region of integration.

Set up the integral in the form $$\int_{??}^{??}\int_{??}^{??}\,x^2y\,dx\,dy,$$ with the order of integration reversed, and evaluate it again.

\newpage

\item 14.7 or 14.8.  Im not sure I agree with the advice I gave ;-)

Show that the rectangular box (with both top and bottom) with fixed volume 27 cubic centimeters and smallest possible surface area is a cube.
$V=xyz$.  $A = 2xy+2xz+2yz$.  You do not need to give any argument that your critical point is actually a minimum; you just need to do the calculations to find the critical point.   I suggest 14.7 methods;  14.8 methods are possible in principle but look too complicated.

\newpage

\newpage 

\item 14.7  

 Find and classify the critical points of $f(x,y)=x^3+y^3-3xy$ and classify them as relative maxima, relative minima, or saddle points.

\newpage

\item  14.8 

Use the method of Lagrange multipliers to find the maximum and minimum values of the function $f(x,y)=x-y$ on the circle $x^2+y^2=1$.   Why do you know that the function takes on an absolute maximum and an absolute minimum value somewhere on this curve?  You do need to show full work with Lagrange multipliers to get credit; it is not hard at all to see what the answer is and where it occurs!

\newpage











\end{enumerate}



\end{document}