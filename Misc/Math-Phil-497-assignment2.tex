\documentclass[12pt]{article}

\title{Math/Phil 497 Homework 2}

\author{Randall Holmes}

\begin{document}

\maketitle

Much of this is done in the notes or in Zermelo's paper.  The benefit to be derived is to write things out in your own words.  You may use whatever mix of words and mathematical or logical notations you are most comfortable with, but be clear.

This is due at 11:55 pm on Monday Sept 28 (officially;  I am pretty flexible). 

\begin{enumerate}

\item  Write out in your own words, using as much or as little mathematical and logical notation as you think appropriate, a proof of Zermelo's theorem of section 10,
that for every set $M$ there is a subset $M_0$ of $M$ which does not belong to $M$.  I strongly suggest that you sit and look at a piece of paper and write this without consulting Zermelo's paper or my notes.

\item  Given sets $A$ and $B$, use Zermelo's axioms to prove the existence of the set $A \cap B = \{x:x \in A \wedge x \in B\}$ and the existence of the set $A \cup B = \{x:x \in A \vee x \in B\}$.  Be certain to clearly indicate which axioms are being used in each case.  The second one is harder than the first:  in plain English, give some discussion of why you might think this is true.

\item I give you for free that Zermelo defines 0 as $\emptyset$, 1 as $\{0\}$, and 2 as $\{1\}$.  Prove the existence of the set $\{0,1,2\}$ from Zermelo's axioms.  Notice that building a set with three elements is a bit trickier than building a set with two elements (which is direct from an axiom).  You  may not use the result of exercise 2 in proving this:  I want you to do it from the axioms.

\item  Frege's definition of the number 1 might be implemented in naive set theory by defining 1 as $\{x:(\exists y:(\forall z:z\in z \leftrightarrow z = y))\}$.  A perhaps more informative way of putting this is $1 = \{\{y\}: y=y\}$, the set of all singleton sets.  Use Zermelo's axioms to prove that the set of all singleton sets does not exist.  Hint:  assume the existence of the set of all singleton sets and show that from its existence follows the existence of a set containing every object (which axiom do you use to do this;  and how do you know that the resulting set contains every object?)  You know already that there cannot be a set containing every object (exercise 1).

\item Who says pre-moderns didn't know about set theory?  

Galileo, I seem to recall, observed that the set of counting numbers 1,2,3,$\ldots$ and the set of perfect squares 1,4,9,$\ldots$ are the same size, because the relation between a natural number $a$ and its square $a^2$  gives a one-to-one correspondence between the two sets.  This made him extremely uncomfortable:  there is a  statement of traditional mathematics (one of Euclid's axioms) which this violates.  Track down the sentence from Euclid. Make your own comments about this situation.

Do a little research, or do this on your own if you know how to do it:  describe a one-to-one correspondence between the natural numbers and the even natural numbers (this is pretty direct from Euclid's example) and a one-to-one correspondence between the natural numbers and the odd numbers.  Does this cause some discomfort?

Describe a one-to-one correspondence between the natural numbers and the integers.

Describe a one-to-one correspondence between the natural numbers and pairs of natural numbers, from which you can get a one-to-one correspondence between the natural numbers
and the positive rational numbers, if you think about it a little.

\end{enumerate}

\end{document}