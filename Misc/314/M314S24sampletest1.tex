\documentclass[12pt]{article}

\title{Math 314 Spring 2024 Sample Test I}

\author{Randall Holmes}

\usepackage{amssymb}

\begin{document}

\maketitle

This paper should have the same look and feel as your actual exam.

There will be 8 problems, organized in groups of two.  In each group, your score will be 70 percent based on the problem you do better on, and 30 percent on the other.
The weight of the pair of problems on which you personally do worst could be reduced:  this is one of my favorite ways to fix a bad grade distribution.

It will be a closed book, closed notes exam.  There will be an appendix supplying access to axioms and logical rules for sections that need them.

Full solutions to the practice test will be supplied some time before the exam.

\begin{enumerate}

\section{First Pair}

\item[]This pair of problems is on truth table reasoning.

\item  Using truth tables, verify that a conditional statement (an implication) $P \rightarrow Q$  is logically equivalent to its contrapositive and not logically equivalent to its converse and inverse (and that the converse and inverse are equivalent to each other).  You should know what these words mean.

Highlight relevant columns in the truth table and say in English what facts about them support your statements of equivalence or inequivalence.

\newpage

\item  I present two logical arguments.  One is valid and one is not.  Give truth table verification that one argument is valid and the other isn't, and explain in English what facts about the truth tables make the arguments valid or invalid.

Your tables should have labelled columns for the premises and conclusion of the arguments being analyzed, and the rows should be numbered because you will want to talk about them in your explanations.

$$\begin{array}{c}

P \rightarrow Q \\

\neg Q \\ \hline

\neg P\end{array}$$

The argument above is called {\em modus tollens\/}

$$\begin{array}{c}

P \rightarrow Q \\

\neg P \\ \hline

\neg Q\end{array}$$

The argument above is called {\em denying the antecedent\/}

\newpage

\section{Second Pair}

This pair of problems is on the formal rules for propositional logic (natural deduction)

\item  Using natural deduction, prove the theorem $$((P \rightarrow Q) \wedge (Q \rightarrow R)) \rightarrow (P \rightarrow R)$$

\newpage

\item  Using natural deduction, verify the rule of {\em destructive dilemma\/}

$$\begin{array}{c}

P \rightarrow Q\\

R \rightarrow S\\

\neg Q \vee \neg S \\ \hline

\neg P \vee \neg R


\end{array}
$$

There are two different ways to prove this, one using proof by cases on the third premise, and one using alternative elimination on the conclusion.
Giving both of these proofs could carry extra credit.

\newpage


\section{Third Pair}

This pair of problems is on somewhat informal proofs about parity and divisibility, giving us a chance to think about quantifiers without being completely abstract.

\item   Prove that the product of an odd integer and an even integer is even.   Start by rephrasing the statement in a way that makes it clear
that it is a universally quantified implication.

\newpage

\item  Show that for any integers $d,m,n$, if $d|m$ and $d|n$, then $d|(m-n)$.

\newpage

\section{Fourth Pair}

This pair of problems is about formal arithmetic (which we are talking about Wednesday and Monday)

\item  Prove using the axioms of formal arithmetic (which you can read from the notes for the practice test, but they will be supplied with your exam paper)
that $1+1 = 2$, where $1$ is defined as $S(0)$ and $2$ is defined as $S(1)$ [and so as $S(S(0))$.  We will do similar things on Wednesday and Monday before the exam; you should have orientation for this after the lecture on the 21st.

\newpage

\item   Prove using the axioms of formal arithmetic that for any natural numbers $m$ and $n$, $S(m)+n = S(m+n)$.  This is a proof by induction;  you will see me write it because it is a lemma in the proof of commutativity of addition.

\newpage


\end{enumerate}


\end{document}