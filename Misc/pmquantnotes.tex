\documentclass[12pt]{article}

\title{Notes on quantifier logic in PM}

\author{Randall Holmes}

\begin{document}

\maketitle

By PM, we mean the {\em Principia Mathematica\/} of Russell and Whitehead, the book from which the set of phone pictures is taken.  This dates from 1910-13.

We extract the axioms and primitive notions of the section 10 approach to quantifier logic in PM.

These include the primitive notions and axioms of propositional logic in PM, which are exactly the same as those of Post's paper, which are implemented in our 
computer program which you use in Homework 5.

We add some parentheses which to our mind improve clarity.  Where the authors write $\phi x$, $\phi \hat{x}$, we write $\phi(x)$, $\phi(\hat{x})$, supplying the parentheses which normally surround the inputs of functions in notations for function application.  Throughout, we will use parentheses instead of the dots used in PM, though we think the dots are interesting.

The new primitive notion added in section 9 (and also in section 10) is the universal quantifier $(x)(\phi(x))$ for any propositional function $\phi(\hat{x})$.  What Russell means by a propositional function is not utterly clear:  he seems to regard them as dependent on the text of the formula $\phi(x)$ in some way (he is not convinced that if $(x)(\phi(x) \equiv \psi(x))$ we necessarily have
$\phi(\hat{x}) = \psi(\hat{x})$.  But he does seem to say that $\phi(\hat{x}) = \phi(\hat{y})$:  changing the dummy variable used as the argument of a propositional function does not
change what the propositional function is.

In the section 9 approach the existental quantifier $(\exists x)(\phi(x))$ is a separate primitive notion.  If we talk about the section 9 approach at all, we may look at why
the authors wand to do this.  In the section 10 approach that we discuss here, $(\exists x)(\phi(x))$ is defined as $\sim (x)(\sim \phi(x))$.

We present the axioms and rules stated in PM section 10 (and one in 11).

\begin{description}

\item[10.1]  $\vdash (x)(\phi(x)) \supset \phi(y)$  In combination with modus ponens, this gives the familiar rule of universal instantiation.

\item[10.11] If we have $\vdash \phi(y)$, where $y$ is a variable (not a constant), then we have $\vdash (x)(\phi(x))$.  This is the rule of universal generalization.

\item[10.12]  $\vdash (x)(p \vee \phi(x)) \equiv (p \vee (x)(\phi(x)))$

\item[10.121]  If $x$ is a variable and $a$ is any constant of the same type as $x$, $\phi(x)$ is significant if and only if $\phi(a)$ is significant.

\item[10.122:]  If $a$ is a constant of the same type as the variable $x$, then $\phi(a)$ is a proposition iff $\phi(\hat{x})$ is a propositional function.



\end{description}

We define Russell's chapter 11 notation $(x,y)(\phi(x,y))$ as $(x)((y)(\phi(x,y))$.

His axiom 11.07 seems to be redundant.  Its use is to reverse the order of universal quantifiers and we can do this in any case.

Suppose we have $\vdash (x)((y)(\phi(x,y)))$.

By two applications of 10.1 and modus ponens, we have $\vdash \phi(x,y)$.

From this we get $\vdash (x)(\phi(x,y))$ by 10.11.

From this we get $\vdash (y)((x)(\phi(x)))$ by 10.11.

We do not see the need for 11.07.

We list the statements about types which PM has as axioms but we will not make much use of them (or give much of an explanation of them) until we actually have some discussion of the theory of types.

We analyze some sample theorems.




\end{document}