\documentclass[12pt]{article}

\usepackage{amssymb}

\title{Homework 7 with study guide}

\author{Dr Holmes}

\begin{document}

\maketitle

I was skipping around in the notes, so I will present this assignment with some narrative of what part of the text to read for each exercise.

\begin{description}

\item[reading for problems 1 and 2:]  read section 2.10.

\item[problem 1:]  Do exercise 1 in section 2.10.1.

\item[problem 2:]  $\{\{a,b,c\},\{d,e,f\},\{g,h,i\}$, where $a,b,c,d,e,f,g,h,i$ are distinct objects, is a partition.  What set is it a partition of?  Present a choice set for this partition.  Present another one.  How many are there?


\item[reading for problems 3,4,5:]  Read section 2.11.

\item[Problem 3:]  Prove that equinumerousness is an equivalence relation.  Be more explicit than I am in my hint of the proof on p. 136:  talk about specific sets and bijections and how you obtain new ones from old ones in detail.  But hasically it is all in the hint.

\item[Problem 4:]  Do problem 2.11.1 from the notes.

\item[Problem 5:]  Prove the commutativity of addition $\kappa + \lambda = \lambda +\kappa$ for cardinals in the same style as the examples on pp. 142-3.  Feel free to use 0 and 1 as the two distinct objects in the concrete definition of addition.


\item[reading for problem 6:]  read section 2.12.

\item[problem 6:]   Verify the distributivity of multiplication over addition for fractions.  I did a similar exercise in the video lecture.  Your calculations can be essentially algebraic (this is not a fancy logic proof with line justifications), but you might want to notice how much machinery from the theory of natural numbers you end up having to use for this basic principle for the theory of positive rationals.

\item[problem 7:]  Describe the construction of the greatest lower bound of a nonempty set of magnitudes as an operation on sets.  You don't need to prove this, just state it.


\item[reading for problems 6 and 7:]  read section 2.13.

\item[problem 8:] 2.13.1 problem 1

\item[problem 9:]   2.13.1 problem 3 (I actually indicated the answer to this in the video lecture!)

\item[reading for problem 10:]  read the proof of the Cantor-Schroder-Bernstein theorem in section 2.17.

\item[problem 10:]  The intervals $[0,2]$ and $[0,2)$ in the real numbers are of the same cardinality.  Prove this by exhibiting a one-to-one map from each set into the other
and applying the theorem.  Hint:  these will just be linear functions.

Then explicitly describe a bijection from $[0,2]$ to $[0,2)$.  The one you get from the theorem would be ugly:  I'll give credit if you can explicitly describe it but that is not the way to go.
Hint: make clever use of the infinite set $\{0,\frac12, \frac 23, \frac 45\ldots\}$:  defining your bijection cleverly on this set lets you get rid of or add the solitary number 2 which appears at the top of one of the sets.

\end{description}


\end{document}