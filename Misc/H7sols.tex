\documentclass[12pt]{article}

\title{Math 189, Fall 2022, Homework 7 Solutions}

\author{Randall Holmes}

\begin{document}

\maketitle

Again, I am only going to grade selected problems, which I will mark with a star.

\begin{description}
\item[Homework 7 on sections 2.3 and 2.4 (you are free to use my technique using binomial coefficients to solve 2.3 problems):]

\item[ section 2.3]

\item[ problems 2$^*$,]  The difference sequences

1,1,1

1,2,3,4

 1,2,4,7, 11
  
1,2, 4, 8, 15, 26

(n choose 0) + (n choose 1) + (n choose 2) + (n choose 3)

$\frac{n^3}6 +\frac{5n}6 +1$

\item[3,]

1,1

3,4,5

3,6,10,15

2,5,11,21,36

2(n choose 0) + 3(n choose 1) + 3(n choose 2) + n choose 3

$\frac{n^3}6 +n^2 +\frac{11n}6 +2$

\item[4 (I know I did 4 in class! I want to see full calculations, using either the book's method or mine, justifying your answers);]

1,1,1

3,4,5,6

3,6,10,15,21

3,6,12,22,37,58

3(n choose 0) + 3(n choose 1) + 3(n choose 2) + 1(n choose 3)

$\frac{n^3}6 + n^2 + \frac{11n}6 +3$

\item[ 9$^*$,]

$(n^2+3n+4) - ((n-1)^2 + 3(n-1) +4)$ simplifies to $2n+2$, so the difference sequence is arithmetic.

\item[ 12,]

The difference sequence is 1,3,6,10,15 and the first term  is 1

so the sequence is

1,4,10,20,35

method of differences:

1,1

2,3,4,5

1,3,6,10,15

0,1,4,10,20,35

This turns out to be $\frac{n(n+1)(n+2)}6$ which you may notice is \newline ($(n+2)$ choose 3) [so it is found in the triangle]

indexing needs to start at 0, so I added the obvious n=0 value

There is a way to fix it if you use just the sequence of values given, ask me if you are interested.

(n choose 1) + 2(n choose 2) +(n choose 3)

There are 680 cannonballs if the pyramid has 15 layers.



\item[ and section 2.4 problems ]

\item[5,]  The characteristic polynomial is $r^2-3-4 = (x-4)(x+1)$ so we will have $a_n = A4^n+B(-1)^n$.

so $A+B = 2 = a_0$ and $4A-B = 3 = a_1$

Adding these equations, $5A=5$ so $A=1$, so $B=(2-A)=1$.

The solution is $a_n = 4^n + (-1)^n$.

\item[ 6$^*$, ]   Like 5, except that

$A+B=5$ and $4A-B=8$.

adding these equations we get $5A = 13$, $A = \frac{13}5$ and $B = 5-A = \frac {12}5$

so $a_n = \frac{13}5\cdot 4^n +\frac{12}5 \cdot (-1)^n$.

\item[8,]  We are given that $r^n = \alpha r^{n-1} + \beta^{n-2}$ for any $n$ and also that
$q^n = \alpha q^{n-1} + \beta^{n-2}$ for any $n$:  this is what it means for these sequences to satisfy this recurrence relation.

We want to show that $a_n = cr^n + d q^n$ satisfies the same relation $a_n = \alpha a_{n-1}+\beta a_{n-2}$.

The verifying calculation:

$\alpha a_{n-1}+\beta a_{n-2} = \alpha( cr^{n-1} + d q^{n-1}) + \beta( cr^{n-2} + d q^{n-2}) = c(\alpha r^{n-1} + \beta r^{n-2}) + d(\alpha q^{n-1} + \beta q^{n-2}) = cr^n + dq^n = a_n$.

\item[ 9,]  I didn't intend to assign 9, and I'm not marking it...

\item[ 10$^*$, ]  $a_n = 4a_{n-1} + 5a_{n-2}$:  one of 4 length 1 tiles followed by n-1 tiles + one of 5 length 2 tiles followed by n-2 tiles.

$a_1 = 4$, $a_2 = 4\cdot 4 + 5 = 21$

4,21,104,521,2604,13021

characteristic polynomial $r^2-4r-5 = (r-5)(r+1)$

so $a_n = A5^n + B(-1)^n$ 

where  5A-B = 4, 25A+B = 21

so 30A = 25, A = $\frac 56$, B = 5A-4 = $\frac 16$

so $a_n = \frac 565^n + \frac16 (-1)^n$.



\item[13 (again, the book has answers: for full credit you need to show convincing work).]


The characteristic polynomial is $r^2-4r+4 = (r-2)^2$.  This gives a possible solution $2^n$ and the weird
alternative $n2^n$.

In general, we have $a_n = A2^n + Bn2^n$

suppose $a_0 = 1 =  A2^0 + B02^0 = 1$.  Then $A=1$.

This applies to both sets of initial values.

If $a_1 = A2^1 + B12^1 = 2$  then $2+2B = 2$ so $B = 0$

and $a_n = 2^n$.

If $a_1 = A2^1 + B12^1 = 8$  then $2+2B = 8$ so $B = 3$

and $a_n = 2^n+3n2^n$.


\end{description}


\end{document}