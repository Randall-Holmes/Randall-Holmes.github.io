\documentclass[12pt]{article}

\usepackage{amssymb}

\title{Math 275 Test III and Final Exam, Fall 2020}

\author{Dr Holmes}

\begin{document}

\maketitle

This take-home exam will be distributed sometime on Sunday 12/13/2020 and due at 11:55 PM  on Thursday, 12/17/2020.  Please e-mail me your exam as you have done with previous exams.

You are allowed free use of your book.  Please do not consult any person other than myself, online or offline.

You will get two grades, a Test III grade based on the Test III part and a final exam grade which will be one-third the Test III grade and two thirds the grade on the Cumulative Part.

Please be sure to communicate with me as soon as possible if you suspect typos or errors in setup.  Also, be careful about similarities with the fall 2018 test.  Some questions are the same as in that test;  some are similar but have different numbers or other details.

Have a lovely holiday season!

\newpage

\section{Cumulative Part}

\begin{enumerate}

\item  Find an  equation of the plane passing through the points (1,2,3),  (2,3,4), and (3,6,9), in the form $ax+by+cz=d$.

\newpage

\item  Find an equation for the line which is the intersection of the planes $x+2y+3z=6$ and $x+y+z=3$.  Hint:  take a cross product to find a vector parallel to this line.  If you are alert you may notice a point which is on both planes without computation (certainly you can find such a point by computation).


\vspace{3 in}

What is the angle between these two planes?

\newpage

\item  Find scalar parametric equations and symmetric equations for the tangent line to the twisted cubic $\left<t,t^2,t^3\right>$ at the point (2,4,8).

\vspace{3 in}

Find an equation for the plane in which both the velocity and the acceleration vectors at time $t=2$ for a particle whose position at time $t$ is $(t,t^2,t^3)$ lie.  (This question is related to things in chapter 13 which we did not discuss, but doesn't require you to do anything we did not cover).



\newpage

\item  Find the  critical points of the function $$f(x,y)=2x^2+y^4+4xy.$$  Use the second derivative test to classify them as local maxima, local minima and saddle points.

Hint:  two of the three critical points are (0,0) and (1,$-1$).  You do need to show work deriving all three.

\newpage

\item  Set up and evaluate $\int\int_D \, xy^2 \, dA$ as an iterated integral, where $D$ is the triangle bounded by
$x=0$, $y=2$, and $y=2x$ (picture provided).  Set up as an iterated integral using  both orders of integration ($dxdy$ and $dydx$)   Evaluate one of these iterated integrals.

\newpage


\end{enumerate}


\section{Test IV}


\begin{enumerate}

\item   Determine the volume of the cone whose lower boundary is $z=\sqrt{x^2+y^2}$ and whose upper boundary is $z=2$, using triple integration with cylindrical coordinates.  (The function integrated is simply 1;  the shadow of the three dimensional region of integration on the $xy$ plane  is of course $x^2+y^2 \leq 4$).  The setup of a triple integral using cylindrical coordinates is worth a lot of the credit, but I do want you to evaluate it.

\newpage

\item  Compute $\int_C \frac xy \, ds$ (a scalar line integral) where $C$ is parameterized by $t^3{\bf i} + t^4{\bf j}$, $0 \leq t \leq 1$.  I guarantee, the integral comes out easy if you set it up correctly.  In any event, you will get most of the credit if you set it up correctly.

\newpage

\item  Compute the vector line integral $\int_C 2ydx + xdy$, where $C$ is the circle $x^2+y^2=1$ traversed counterclockwise from (1,0) back to (1,0), in two different ways:  

\begin{enumerate}

\item compute it directly, 

\vspace{2.5 in}

\item and set up an equivalent double integral using Green's Theorem and evaluate it (using high school geometry).


\end{enumerate}

\newpage

\item  The field ${\bf g}(x,y,z)=y^2{\bf i} +(2xy + z^2){\bf j}+2yz{\bf k}$ is conservative.   Verify this using a suitable test (using the curl operation).

\vspace{1 in}

Find a function $f(x,y)$ such that $\nabla f = {\bf g}$.

\vspace{3 in}

Then compute $\int_C {\bf g \cdot dr}$ (a vector line integral) where $C$ is parameterized by $\left<t,\ln(t),e^t\right>$ where $1 \leq t \leq 2$.  Hint:  you don't have to do any integrals involving logarithms or exponentials to do this.  In fact, you don't have to do any integration at all.  The corresponding question on the sample test had an error in it...did you spot it?  The answer does involve $e$ and logarithms and looks mildly complicated...please give it in exact form (you may evaluate it with a calculator too).

\newpage

\item  Do one of the two  parts.  If you do both, your best work will count.

\begin{enumerate}

\item  Compute the surface area of the slanted part of a cone defined by $z=\sqrt{x^2+y^2}, 0 \leq z \leq 3$, by setting up and evaluating a scalar surface integral.  Since you are just computing an area, the function being integrated is 1.   A parameterization of this surface is given by
$\left<v\cos(u),v\sin(u),v\right>; 0 \leq u \leq 2\pi; 0 \leq v \leq 3$.

\newpage


\item  Compute the vector surface integral of the field $\left<x+2y,y+2z,z+2x\right>$ over the closed surface $x^2+y^2+z^2=4$ with the standard outward orientation, using the Divergence Theorem.  By the theorem, the surface integral is equal to a certain triple integral.  Describe it (being sure to say what the region of integration is).  Then evaluate it using high school geometry.

\end{enumerate}

\newpage




\end{enumerate}

\end{document}