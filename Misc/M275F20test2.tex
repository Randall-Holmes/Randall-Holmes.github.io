\documentclass[12pt]{article}

\title{Math 275 Test II, Fall 2020}

\author{Dr. Holmes}

\begin{document}

\maketitle

Please do this exam without consulting anyone else but the instructor.  It is due electronically at 11:55 pm on Sunday, 18 Oct, 2020:  email it to me.  You may work on a printed copy of the test paper or on your own paper.

\newpage

\begin{enumerate}

\item 14.3 partial derivatives

Let $f(x,y) = xy^4 + x^3y^2$.  Compute the first partials of this function (two functions) and the second partials of this function (four functions).    Clearly label each first and second partial so that I know that you know which one is which.

What fact about your second partials is an example of Clairaut's thoerem?

\newpage

\item  14.4 tangent planes and linearization

Give the equation for the tangent plane to $z=xy^3$ at $(1,2,8)$.

\vspace{1.5 in}

Give the linear function $L(x,y)$ which best approximates $xy^3$ when $x$ is close to 1 and $y$ is close to 2.  You might notice that this is almost the same question.


\vspace{1.5 in}

Give the linear approximation to $(1.01)(1.99)^3$ using this linear function.

\newpage

\item  Chain Rule  14.5

Let $y = u^2 + 2uv +vw^3$ where $u=s+t$, $v=st^2$,  $w=s^2-t^2$

Write out the form of the Chain Rule needed to compute the partial derivative of $y$ with respect to $t$.

Compute the partial derivative of $y$ with respect to $t$.  You may
leave $u$'s, $v$'s, and $w$'s in your answer.

\newpage

\item  14.6 gradient and directional derivative

Compute the gradient of the function $g(x,y) = x^3y-xy^3$ at the point $(1,2)$.  Compute the directional derivative of this function in the direction of the vector $\left<5,12\right>$.

\vspace{1.5 in}

In what direction does this function increase most rapidly at $(1,2)$?  What is the rate of change?

\newpage


\item  14.7  critical points and local extrema

Find all critical points of the function $f(x,y) = x^3-y^3-12x+3y$ and classify them as local maxima, local minima, or saddle points using the Second Derivative Test.

\newpage

\item 14.8

Use the method of Lagrange multipliers to find the closest point
to the origin on the plane $x+4y+9z=13$.  Hint: find the critical point
for $x^2+y^2+z^2$ (the square of the distance from the origin) subject
to the constraint given by the equation of the plane.  There is only
one critical point, and you do not have to verify that it is actually
a minimum (this is obvious from geometry).

\newpage 

\item  15.1  

Evaluate $$\int_0^1\int_1^2\,x+y\,dy\,dx.$$   Reverse the order of integration and evaluate it again.

\newpage 


\newpage

\item  15.2

  Sketch the region of integration of the iterated integral $\int_0^2\int_{x^3}^8 \, x \,dy\,dx$.

\vspace{1.5 in}

Change the order of integration and set up the integral as $\int_{??}^{??} \int_{??}^{??}\,x\,dx\,dy$.

\vspace{1.5 in}

Evaluate both integrals and check that the same value is obtained

\newpage











\newpage











\end{enumerate}



\end{document}