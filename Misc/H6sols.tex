\documentclass[12pt]{article}

\title{Homework 6, Math 189, 2022}

\author{Randall Holmes}

\begin{document}
\maketitle

I supply solutions to all problems, but I am only marking every other problem.  I have to catch up with my backlog.

\begin{description}

\item[Levin 2.1:]  

\item[ 2,]  a:  $3+2^n$; b: $n^2-1$; c: $(n+2)(n+3)$; d: $n^2+\frac{n^2+n}2$:  square numbers plus triangular numbers equal house numbers.

\item[4, ]  a:  $2n-1$:  notice that the notation tells you that indexing starts with 1.

b:  the sequence is 1,4,9,$\ldots$.  It is confusing that he suggests starting indexing of $b_n$ with $n=2$, then
actually gives it with indexing starting at 1 (and the first partial sum just the first term).

\item[5,]  a:  0,1,2,4,7,12,20, 33

b: compare with 1,2,3,5,8,13,21,34,$\ldots$, that is $F_{n+2}$.

we get the sequence of partial sums by subtracting one from this, so $F_{n+2}-1$.

\item[10, ]  $$7a_{n-1}-10a_{n-2}= 7(2^{n-1} -5^{n-1})-10(2^{n-2}-5^{n-2}) $$ $$= 14\cdot2^{n-2} - 35\cdot5^{n-2}-10\cdot2^{n-2}+10\cdot5^{n-2} = 4\cdot2^{n-2} - 25\cdot5^{n-2} = 2^n - 5^n = a_n$$

$a_0 = 0;  a_1= -3$

\item[15,]  $R_0 = 1$; $R_1=2$.  When you draw the $(n+1)$st line, it intersects each of the other $n$ lines in a point:
$n$ points determine $n+1$ line segments.  Each of these segments divide one of the $R_n$ regions in two,
so $R_{n+1}=R_n+(n+1)$.  $R_2 = 4$. $R_3=7$, $R_4 = 11$ follow this pattern.

 \item[19; ] $t_1=1;t_2=2;t_3=3$  in general, $t_{n} = t_{n-2} + t_{n-1}$:  think about how the top right domino is placed.
if it is vertical what remains is to tile a $2 \times (n-1)$;  if it is horizontal, what remains is to place a domino under it
then tile a $2 \times (n-2)$.

\item[Levin 2.2:]

\item[ 2,]  32, $8+6n$, 30,500 = $\frac{a_0+a_99}2\cdot 100 = \frac{8+(8+(6)(99))}2\cdot 100$

\item[ 4,]  $n+2$, since indexing has to start with $-1$.  $6n+1$ is the second to last term.  The sum of all the terms is
$\frac{1 + (6n+7)}2 \cdot (n+2) = \frac{(6n+8)(n+2)}2$.

Note that the sum of an arithmetic sequence is the average of the first term and the last term multiplied by the number of terms.  This is how I am generating my calculations here.

\item[ 7,]   This is $a+ar+ar^2+\ldots+ar^n = \frac{a -ar^{n+1}}{1-r}$ where $a=1, r=-\frac 23, n=30$.

This is $\frac{1+\frac23^{31}}{1+\frac 23} = \frac35(1-(\frac 23)^{31})$

\item[ 14, ]  1,5,13,25  This is enough to see that it is neither arithmetic or geometric:  1+4 = 5, but 5+4 is not 13, and
$1 \cdot 5 = 5; 5 \cdot 5 \neq 13$

it is 1, 1+4, 1+4+8, 1+4+8+12, 1+4+8+12+16, which is 1 + a sum of multiples of 4, which is not
the sequence of partial sums of any arithmetic or geometric sequence; it differs by one from such a sum.

it is the partial sum of a sequence starting 1,4,8,12  which again can be seen to be neither arithmetic or geometric
just by looking at these terms.  However, the terms of this sequence turn out to go up by 4 at each subsequent step.

so it is $1 + 4\frac{n(n-1)}2= 1+2n(n-1) = 1+2n^2-2n$

This can be seen to work by realizing that each diagonal square can be reduced to the previous one by removing the middle row and the row below it;  then notice that the sum of the middle row and the row below it goes up by 4 at each step.

It can also be seen, by thinking of the diagonal square as a sum of two triangles (vertically, one on top of the other)and using the result about the sum'of the first n odd numbers being $n^2$, that the $n$th diagonal square number is $n^2 + (n-1)^2$:  I'll be happy to illustrate this in class if you dont see it.  $n^2+ (n-1)^2  = n^2 + n^2 -2n+1 = 2n^2-2n+1$.

\item[15. ]  a:  $4^6$ contain no numerals.

$3\cdot 4^5$ contain one numeral.

$3^2 \cdot 4^4$ contain two numerals, etc.

This is a geometric sequence with $a=4^6$ and $r=\frac 34$.

The answer is then $\sum_n=0^6 4^6\cdot (\frac34)^n$, and this, by results in this section, is
$\frac{4^6 - 4^6\cdot (\frac 34)^7}{1-\frac 34}$ and this simplifies straightforwardly to $4^7-3^7$.

\end{description}




\end{document}