\documentclass[12pt]{article}

\title{Math 414/514 Spring 2024 Sample Test I}

\author{Randall Holmes}

\usepackage{amssymb}

\begin{document}

\maketitle

This document should have the look and feel of your actual exam.  There are 8 questions, organized into pairs.

In each pair, you get 70 percent credit for the problem you do better on and 30 percent for the other.

Some reference information (axioms, theorems and definitions for reference) may be supplied on the actual test paper; for the practice test you can consult your book, but the actual test will be closed book, closed notes.

Problems on this exam should not be surprises:  they should be or be very similar to things I have done in class or that you have been assigned in homework.  Of course, they may be things
that you chose not to do in homework...

The problems I choose for this practice test should give you a general idea of my thinking about the actual exam.   Some of the questions on the actual exam may be exactly the same:  don't assume that because I asked something on the practice exam I won't ask that exact question on the exam itself.

\newpage

\begin{enumerate}

\section{First Pair}

\item[]   abstract stuff about sets

\item   Prove that $A \setminus (B \cup C) = (A \setminus B) \cap (A \setminus C)$.

Recall that our strategy for showing that two sets are equal is to postulate an arbitrary element of the first, and show that it must belong to the second, then postulate an arbitrary element of the second and show that it belongs to the first.

Suppose $x \in A-(B \cup C)$:  

it follows that $x \in A$ and $x \not\in B \cup C$

it follows that $x \not\in A$ and ($x \not\in B$ and $x \not\in C$)

so ($x \not\in A$ and $x \not\in B$) and ($x \not\in A$ and $x \not\in C$)

so $x \in A-B$ and $x \in A-C$

and so $x \in (A-B) \cap (A-C)$

Other direction:

Suppose $x \in (A-B) \cap (A-C)$

this implies $x \in A-B$ and $x \in A-C$

and so ($x \in A$ and $x \not\in B$) and ($x \in A$ and $x \not\in C$)

and so $x \in A$ and $x \not\in B$ and $x \not\in C$

so $x \in A$ and not($x \in B$ or $x \in C$)

so $x \in A$ and not ($x \in B \cup C$) so $x \in A$ and $x \not\in B \cup C$ so $x \in A-(B \cup C)$

We have shown that each of these sets is a subset of the other so they are equal.

\newpage

\item   Prove that for any function $f$ and sets $C$ and $D$, $f[C \cup D] = f[C] \cup f[D]$

Recall that $f[C]$ is defined as $\{f(c):c \in C\}$.  I prefer brackets to avoid even the possiblity of confusion between values of a function and elementwise images of a set under a function.

Suppose $x \in f[C \cup D]$.  It follows that there is $y$ such that $y \in C \cup D$ and $x=f(y)$. So $y \in C$ or $y \in D$, so
either $y \in C$ and $x=f(y)$, in which case $x \in f[C]$ or $y \in D$ and $x=f(y)$, in which case $x\in f[D]$, so either $x \in f[C]$ or $x \in f[D]$ so $x \in f[C] \cup f[D]$. 

The other direction:  suppose $x \in f[C] \cup f[D]$.  Then either case 1: $x \in f[C]$ or case 2: $x \in f[D]$.

in case 1, there is $y$ such that $x=f(y)$ and $y \in C$, in which case $x=f(y)$ and $y \in C\cup D$, so $x \in f[C \cup D]$.

in case 2, there is $y$ such that $x=f(y)$ and $y \in D$, in which case $x=f(y)$ and $y \in C\cup D$, so $x \in f[C \cup D]$.

In either case, $x \in f[C\cup D]$.

Thus we have shown that each of these sets is a subset of the other, so they are equal.

\newpage

\section{Second Pair}

mathematical induction

\item   Prove by mathematical induction that $\sum_{i=0}^n ar^i = \frac{a - ar^{n+1}}{1-r}$

Prove by induction on $n \geq 0$.

Basis ($n=0$):  $\sum_{i=0}^0 ar^i = a = \frac{a(1-r)}{1-r} = \frac{a-ar^1}{1-r}$

Induction step:  let $k$ be an arbitrarily chosen integer

Assume (ind hyp) that $\sum_{i=0}^k ar^i = \frac{a - ar^{k+1}}{1-r}$

Goal:  $\sum_{i=0}^{k+1} ar^i = \frac{a - ar^{k+2}}{1-r}$

Proof:   $\sum_{i=0}^{k+1} ar^i = (\sum_{i=0}^{k} ar^i)+ar^{k+1} = \verb|[ind hyp]| \frac{a - ar^{k+1}}{1-r} + ar^{k+1}=$

$\frac{(a-ar^{k+1}) + (ar^{k+1} - ar^{k+2})}{1-r} = \frac{a - ar^{k+2}}{1-r}$

\newpage

\item  Prove by mathematical induction that a set of natural numbers which is nonempty must have a smallest element:  hint, prove by strong induction
that a set of natural numbers without a smallest element must be empty.

Suppose that $A$ is a set of natural numbers with no smallest element.

We prove by induction that for any natural number $n$, $n \not\in A$ ($A$ is empty).

We actually prove the stronger assertion that for all $k \leq n$, $n \not\in A$.

Basis ($n=1$):  the aim is to prove that for all $k \leq 1$, $k \not\in A$

Proof of basis:  if $k \leq 1$ is a natural number, then $k=1$.  If $1 \in A$ then 1 would be the smallest element of $A$, a contradiction.
So we have shown that if $k \leq 1$, it follows that $k \not\in A$.

Induction step:  let $m$ be an arbitrarily chosen integer.  Suppose (ind hyp) that $(\forall k \leq m:k \not\in A)$

Goal:  $(\forall k \leq m+1:k \not\in A)$

Proof of induction step:  If $k \leq m+1$ then either $k \leq m$ or $k=m+1$.  If $k \leq m$, then $k \not\in A$ by ind hyp.  If $m+1$ were in $A$ it would follow that it was the smallest element of $A$ [which does not have a smallest element] because we know that all $k <m+1$ are not in $A$.  So $m+1$ is not in $A$ either,
and the induction step is proved.

Since we have shown that a set of natural numbers with no smallest element is empty, we have shown the contrapositive, that a nonempty set of natural numbers has a smallest element.

\newpage

\section{Third Pair}

\item  Prove from the axioms given in the book that if $0<x<y$ then $x^2<y^2$.  You may be a little informal about equational algebra but your order reasoning must be directly from the axioms he actually gives for order (definitions 1.1.1 and 1.1.7).

If $x<y$ and $z>0$, then $(y-x)>0$ and $z>0$, so $(y-x)z>0$ so $yz-xz>0$ so $yz>xz$ so $xz<yz$.

Now suppose $0<x<y$.  It follows by applying the rule above that $xx<yx$ and also that $xy<yy$, so we have $x^2<xy<y^2$,
and $x^2<y^2$ by transitivity.

The handwritten solution doesn't include the first line here.

\newpage

\item 

Let $A \subseteq B$ be nonempty sets of real numbers.  Suppose that $B$ is bounded above and below.  

Argue that $A$ must be bounded above and below.

Show that ${\tt inf}(B) \leq {\tt inf}(A) \leq {\tt sup}(A) \leq {\tt sup}(B)$

This has to be proved from basic properties of order and sets and the definitions of inf and sup.

${\tt inf}(B)$ is a lower bound for $B$.  Suppose $a \in A$.  It follows that $a \in B$, from which it follows that ${\tt inf}(B)\leq a$, so ${\tt inf}(B) \leq {\tt inf}(A)$, since ${\tt inf}(A)$ is the largest lower bound of $A$.

There is an element $a$ of $A$;  ${\tt inf}(A)\leq a$ because ${\tt inf}(A)$ is a lower bound of $A$; $a \leq {\tt sup}(A)$ because ${\tt sup}(A)$ is an upper bound for $A$.  Thus ${\tt inf}(A) \leq {\tt sup}(A)$ by transitivity.

${\tt sup}(B)$ is an upper bound for $B$.  Suppose $a \in A$.  It follows that $a \in B$, from which it follows that ${\tt sup}(B)\geq a$, so ${\tt sup}(B) \leq {\tt sup}(A)$, since ${\tt sup}(A)$ is the smallest upper bound of $A$.  This establishes the third inequality ${\tt sup}(A) \leq {\tt sup}(B)$.

All three inequalities have  been established.

\newpage

\section{Fourth Pair}

\item  Show that $|x-y|<\epsilon$ if and only if $x-\epsilon < y < x+\epsilon$

This should not need anything but the definition of absolute value and the most basic properties of order.  You need to prove an implication in one direction then the other.  You might have a use for proof by cases.

Assume that $|x-y|<\epsilon$.  It follows that $0<\epsilon$.  Further, it follows that $x-y < \epsilon$, because if $x-y >0$ it follows that
$x-y = |x-y| < \epsilon$ and if $x-y \leq 0$ we have $x-y \leq 0 <\epsilon$.  $y-x <\epsilon$ follows by an exactly symmetrical argument.

Since $x-y <\epsilon$ it follows that $x-\epsilon < y$ (add $y-\epsilon$ to both sides) and if $y-x <\epsilon$ it follows that $y<x+\epsilon$ so we
have $x-\epsilon < y < x+\epsilon$.

The other direction:  assume that $x-\epsilon < y < x+\epsilon$:  it follows that $x-y <\epsilon$ (add $y - \epsilon$ to both sides of the first inequality)
and $y-x <\epsilon$ (subtract $x$ from both sides of the second inequality) and so $|x-y|<\epsilon$ since $|x-y|$ is equal to one of these two numbers.

\newpage

\item  Prove that if $f:D \rightarrow \mathbb R$ and $g:D \rightarrow \mathbb R$ are bounded functions that
${\tt inf}_{x \in D}[f(x)+g(x)] \geq {\tt inf}_{x \in D}f(x) + {\tt inf}_{x \in D}g(x)$

For every $x \in D$, $f(x) \geq {\tt inf}_{x \in D}f(x)$, 

and for every $x \in D$, $g(x) \geq {\tt inf}_{x \in D}g(x)$.  

Thus for every
$x \in D$, $f(x) + g(x) \geq {\tt inf}_{x \in D}f(x)+{\tt inf}_{x \in D}g(x)$, so ${\tt inf}_{x \in D}f(x)+{\tt inf}_{x \in D}g(x)$ is a lower bound of
$\{f(x)+g(x):x \in D\}$, so ${\tt inf}_{x \in D}[f(x)+g(x)] \geq {\tt inf}_{x \in D}f(x) + {\tt inf}_{x \in D}g(x)$ 

because ${\tt inf}_{x \in D}f(x) + {\tt inf}_{x \in D}g(x)$ is the greatest lower bound of the set $\{f(x)+g(x):x \in D\}$.


\newpage

\end{enumerate}

\end{document}