\documentclass[12pt]{article}

\title{Math 414/514 Spring 2024 Sample Test I}

\author{Randall Holmes}

\usepackage{amssymb}

\begin{document}

\maketitle

This document should have the look and feel of your actual exam.  There are 8 questions, organized into pairs.

In each pair, you get 70 percent credit for the problem you do better on and 30 percent for the other.

Some reference information (axioms, theorems and definitions for reference) may be supplied on the actual test paper; for the practice test you can consult your book, but the actual test will be closed book, closed notes.

Problems on this exam should not be surprises:  they should be or be very similar to things I have done in class or that you have been assigned in homework.  Of course, they may be things
that you chose not to do in homework...

The problems I choose for this practice test should give you a general idea of my thinking about the actual exam.   Some of the questions on the actual exam may be exactly the same:  don't assume that because I asked something on the practice exam I won't ask that exact question on the exam itself.

\newpage

\begin{enumerate}

\section{First Pair}

\item[]   abstract stuff about sets

\item   Prove that $A \setminus (B \cup C) = (A \setminus B) \cap (A \setminus C)$.

Recall that our strategy for showing that two sets are equal is to postulate an arbitrary element of the first, and show that it must belong to the second, then postulate an arbitrary element of the second and show that it belongs to the first.

\newpage

\item   Prove that for any function $f$ and sets $C$ and $D$, $f[C \cup D] = f[C] \cup f[D]$

Recall that $f[C]$ is defined as $\{f(c):c \in C\}$.  I prefer brackets to avoid even the possiblity of confusion between values of a function and elementwise images of a set under a function.

\newpage

\section{Second Pair}

mathematical induction

\item   Prove by mathematical induction that $\sum_{i=0}^n ar^i = \frac{a - ar^{n+1}}{1-r}$

\newpage

\item  Prove by mathematical induction that a set of natural numbers which is nonempty must have a smallest element:  hint, prove by strong induction
that a set of natural numbers without a smallest element must be empty.

\newpage

\section{Third Pair}

\item  Prove from the axioms given in the book that if $0<x<y$ then $x^2<y^2$.  You may be a little informal about equational algebra but your order reasoning must be directly from the axioms he actually gives for order (definitions 1.1.1 and 1.1.7).

\newpage

\item 

Let $A \subseteq B$ be nonempty sets of real numbers.  Suppose that $B$ is bounded above and below.  

Argue that $A$ must be bounded above and below.

Show that ${\tt inf}(B) \leq {\tt inf}(A) \leq {\tt sup}(A) \leq {\tt sup}(B)$

This has to be proved from basic properties of order and sets and the definitions of inf and sup.

\newpage

\section{Fourth Pair}

\item  Show that $|x-y|<\epsilon$ if and only if $x-\epsilon < y < x+\epsilon$

This should not need anything but the definition of absolute value and the most basic properties of order.  You need to prove an implication in one direction then the other.  You might have a use for proof by cases.

\newpage

\item  Prove that if $f:D \rightarrow \mathbb R$ and $g:D \rightarrow \mathbb R$ are bounded functions that
${\tt inf}_{x \in D}[f(x)+g(x)] \geq {\tt inf}_{x \in D}f(x) + {\tt inf}_{x \in D}g(x)$

\newpage

\end{enumerate}

\end{document}