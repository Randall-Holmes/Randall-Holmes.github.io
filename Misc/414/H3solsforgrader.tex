\documentclass[12pt]{article}

\title{Homework 3 Solutions}

\author{Dr Holmes}

\usepackage{amssymb}

\begin{document}

\maketitle

They are supposed to do six of these, best six count if they do more.

\begin{description}

\item [1.1.1,] 

If $x<0$ and $y<z$, then $xy>xz$

They are supposed to prove this from the given properties of order.  Look at the actual axioms when deciding whether what they do makes sense.

$y<z$, so $y+(-y) < z+(-y)$ (1.17i), that is, $0 < z+(-y)$, and $z+(-y)>0$.

$x<0$ so $x+(-x) < 0+(-x)$ (1.17i) so $0<-x$ so $-x>0$.

It follows that $(-x)(z+(-y))>0$ (1.17ii) and $(-x)(z+(-y)) = xy-xz>0$ so $(xy -xz) + xz > 0+xz$ that is, $xy>xz$.

Feel free to ask me if you see other patterns of reasoning how much I would award for them.  Reasoning from given sets of axioms
about things they know perfectly well is always tricky to mark.


\item[1.1.3,]

Suppose $0<x<y$.  Show that $x^2<y^2$.

We have $y>0$, so $y+x>0+x$ (add $x$ to both sides) so $y+x>0$ (transitivity).

We have $x<y$ so $y>x$ so $y+(-x)>x+(-x) = 0$.

Thus we have $y+x$ and $y-x$ positive, so $(y+x)(y-x)>0$, so $y^2-x^2>0$, so (add $x^2$ to both sides) $y^2>x^2$ so $x^2<y^2$.

Notice in 1.1.1 and 1.1.3 I am using equational algebra pretty freely but I stick to the exact basic properties of order that the section gives, which
are in definition 1.1.1 and definition 1.1.7.


 \item[1.1.4 (tricky, I may offer advice if you ask: this needs to be proved from the definitions, not by pictures or hand waving),]

I did this more concretely than the book asks for, talking about sets of real numbers where they are talking about an abstract ordered set
$S$ in which all the specified sups and infs exist.  The argument is the same.

Let $A$ be a nonempty subset of $B$, a nonempty set of reals which is bounded above and below.   Show that
${\tt inf}(B) \leq {\tt inf}(A) \leq {\tt sup}(A) \leq {\tt sup}(B)$.

let $m$ be a lower bound for $B$ and let $M$ be an upper bound for $B$:  we are given that there are such numbers.

For any element $a$ of $A$, $a \in B$ because $A \subseteq B$, so $m \leq a$.  Thus $m$ is a lower bound for $A$, so $m \leq {\tt inf}(A)$ because
${\tt inf}(A)$ is the greatest lower bound of $A$ (which exists by the completeness property).  This is true for {\em any\/} lower bound of $B$, and ${\tt inf}(B)$ is a lower bound of $B$, existing by the completeness property, so
we have ${\tt inf}(B) \leq {\tt inf}(A)$, the first inequality to be proved.

$A$ is nonempty:  let $a$ be an element of $A$.  ${\tt inf}(A)$ exists by the completeness property and is a lower bound for $A$, so
${\tt inf}(A)\leq a$.   ${\tt sup}(A)$ exists by the completeness property and is an upper bound for $A$, so
$a \leq {\tt sup}(A)$.  By transitivity, ${\tt inf}(A) \leq {\tt sup}(A)$, the second inequality to be proved.

For any element $a$ of $A$, $a \in B$ because $A \subseteq B$, sp $M \geq a$.  Thus $M$ is an upper bound for $A$, so $M \geq {\tt sup}(A)$ because
${\tt sup}(A)$ is the least upper bound of $A$ (which exists by the completeness property).  This is true for {\em any\/} upper bound of $B$, and ${\tt sup}(B)$ is an upper bound of $B$, existing by the completeness property, so
we have ${\tt sup}(B) \geq {\tt sup}(A)$, so ${\tt sup}(A) \leq {\tt sup}(B)$, the third inequality to be proved.

\item[ 1.1.5 (I've commented on this),]

Suppose that $S$ is an ordered set, $A \subseteq S$, and $b \in A$ is an upper bound for $A$.  Show that $b = \sup(A)$.

We suppose that $b$ is an upper bound for $A$.  Suppose $c < b$:  $c$ cannot be an upper bound for $A$ because we would have to have
$b \leq c$, contrary to hypothesis about $c$.  So $b$ is the least upper bound for $A$, so $b = \sup(A)$ by definition of sup.

\item[ 1.1.8,]  They should come up with the addition and multiplication tables of mod 3 arithmetic.

In the addition table, all sums involving 0 are handled by the identity property.  $1+1=1$ cannot be true, because adding $-1$ to both sides
would give $1=0$.  So $1+1=2$.  $1$ has to have an additive inverse, it isn't 1 itself, so it is 2, so $1+2=2+1=0$.  $2+2 = 2 + (1+1) = (2+1)+1 = 0+1 = 1$.

In the multiplication table, every entry except $2 \cdot 2$ follows from the identity or zero properties of multiplication.  2 has to have a multiplicative inverse and it cannot be 0 or 1, so it must be 2 itself, so $2 \cdot 2 = 1$.

I'd be interested to see reasoning, but if they get mod 3 arithmetic give them the benefit of the doubt.

It can't be an ordered field:  in any ordered field, we have $0<1$.  Add 1 to both sides and we get $1<2$.  Add 1 to both sides and we get $2<0$.  Transitivity gives $1<0$ among other absurdities.

\item[ 1.1.9,]  Let $S$ be an ordered set and suppose $A \subset S$ and $\sup(A)$ exists.

Suppose $B \subseteq A$ and for any $x \in A$ there is $y \in B$ such that $y>x$.

Show that $\sup(B)$ exists and $\sup(B)=\sup(A)$.

Proof:  for any $b \in B$, we have $b \in A$, so $b \leq \sup(A)$:  so $\sup(A)$ is an upper bound for $B$.

Suppose that $c < \sup(A)$ is an upper bound for $B$.  Since $c < \sup(A)$ we have that $c$ is not an upper bound for $A$,
so for some $a \in A$ we have $c <a$, and by hypotheses there is $d \in B$ such that $a <d$, so since $c <d \in B$ $c$ is not an upper bound for
$B$, so in fact $\sup(A)$ is the smallest upper bound for $B$, so $\sup(A)=\sup(B)$.

It is crucial to use the fact that $c<\sup(A)$ implies that something in $A$ is above $c$;  certainly you can't assume $c \in A$, which I can imagine a student doing.

\item[ 1.1.11,]

Suppose $x \leq y$ and $z \leq w$ and deduce $x +z \leq y+w$.

Proof: $x \leq y$ implies (1) $x+z \leq y+z$ by adding $z$ to both sides (which works for equations and less than statements).
$z \leq w$ implies (2) $y+z \leq y+w$  by adding $y$ to both sides.  $x+z \leq y+w$ follows by transitivity of $\leq$ (which they may assume but which is easy:
$a \leq b$ and $b \leq c$ obviously imply $a \leq c$ if $a=b$ or $b=c$, and otherwise apply transitivity of $<$).

if $x < y$ and $z \leq w$ show that $x+z < y+w$:  if $z=w$ then $x<y$ implies $x+z < y+z$ which implies $x+z < y+w$ by substitution.

if $z<w$ then $x<y$ implies $x+z<y+z$, $z<w$ implies $y+z<y+w$, and $x+z<y+w$ follows by transitivity.

\item[ 1.2.1,]  Prove that if $t>0$ there is an $n \in \mathbb N$ such that $\frac 1{n^2}<t$.

This is equivalent to $\frac 1t < n^2$.  Note that $\frac 1t < n$ will work because $n \leq n^2$ will be true for $n \in \mathbb N$.

So, let $t >0$ be chosen arbitrarily.  $\frac 1t>0$ follows.  By the Archimedean property there is $n \in \mathbb N$ such that
$\frac 1t<n$.  We then have $\frac 1t < n \leq n^2$, and multiplying through by $\frac t{n^2}$ we get $\frac 1{n^2}<t$.

\item[1.2.2,]  Prove that if $t \geq 0$ there is $n \in \mathbb N$ such that $n-1 \leq t <n$.

Consider the set $S$ of all natural numbers greater than $t$.  It is nonempty by the Archimedean property (if $t>0$; if $t=0$ it is clearly nonempty).
and so has a smallest element $n$ by the well-ordering property of $\mathbb N$.  $n>t$ by the definition of $S$.
$n-1 \not\in S$ by choice of $n$ as the smallest element of $S$, so $n-1 \leq t$.  So $n-1 \leq t  <n$.

 \item [1.2.13 (hint: binomial theorem), ]

I don't think my binomial theorem hint works.  The problem is that the conditions allow $x<0$.

Prove it by induction on $n$.

$1 + x \leq 1 + x$, basis.

Suppose $1+kx \leq (1+x)^k$. [ind hyp]

So $(1+kx)(1+x) \leq (1+x)^{k+1}$ [this uses the fact that $1+x\geq 0$).

So $1+(k+1)x + kx^2 \leq (1+x)^{k+1}$

and certainly $1+(k+1)x \leq 1+(k+1)x + kx^2 \leq (1+x)^{k+1}$.

\item[1.2.15]

Show that for any real number $y$, the supremum of the set of rationals less than $y$ is $y$:

Suppose otherwise.  Let $z<y$ be the supremum of the set of rational numbers less than $y$.  Then the interval $(z,y)$ is of positive length and contains no rational number.   This contradicts theorem 1.2.4ii.

By the way, the set of rational numbers less than $y$ needs to be seen to be nonempty to do this (though I wouldnt fault a student for not noticing:
this is easy, as there is a natural number $N$ greater than $|y|$ (Archimedean property) and certainly $-N<-|y| \leq y$.

Define a Dedekind cut as a set of rationals which is downward closed as a subset of the rationals, bounded above, and has no largest element [this is the same as what Lebl says]:
show that if $D$ is a Dedekind cut, then $D = \{x \in \mathbb Q:x<y\}$ for some real number $y$:  proof:  let $y = \sup(D)$.  Obviously
any element of $D$ is a rational less than $y$ [since $D$ has no largest element it cannot have its sup $y$ as an element].  We also need to show that any rational less than $y$ is in $D$:  if $r<y$ is rational, then there is an element $s$ of $D$ with $r<s<D$ (because $r$ cannot be an upper bound for $D$) and
$D$ is a downward closed subset of the rationals, so $r \in D$ because $r$ is a rational less than $s$.

Show that there is a bijection between $\mathbb R$ and Dedekind cuts:  this is really just a remark after the previous part is proved:  the correspondence between
reals $y$ and the sets $\{x \in \mathbb Q:x<y\}$, which are the Dedekind cuts, clearly is a one to one correspondence.

\end{description}


\end{document}