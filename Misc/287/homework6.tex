\documentclass[12pt]{article}

\title{Math 287 Fall 2024 Homework 6}

\title{Randall Holmes}

\begin{document}

\maketitle

Some, perhaps all of these proof exercises are inspired by the proof of the Binomial Theorem, which is itself a bit too involved to be examinable as a whole.  Some of these I have done, or sketched the proofs of, while lecturing the Binomial Theorem:  you should then do better at writing them up yourselves.

\begin{enumerate}

\item  Prove 4.16 part ii.  This is a very likely test question, it is very much to your advantage to nail it down.

\item  Where $a$ and $b$ are constants, prove that $\sum_{i=a}^b x_i = x_a + \sum_{i=a+1}^b x_i$.  I haven't quite fully stated the problem:  be sure to state additional conditions needed for this to make sense, and also be sure to state the exact subset of the integers over which the induction is carried out.

\item  Prove proposition 4.17.  We proved a special case of this ($r=1$) in the Binomial Theorem proof.

\item  Given the definition of $\frac ab$ (where $a$ is an integer and $b$ is a nonzero integer) as the unique $x$ such 
that $ax=b$, if there is one, prove that $\frac{ac}{bc} = \frac ab$ if $\frac ab$ is defined and $c \neq 0$.  Also prove
that $\frac ac + \frac bc = \frac {a+b}c$ if the two fractions on the left are defined.  Finally, prove
that $\frac ab + \frac cd = \frac {ad+bc}{bd}$ if the two fractions on the left are defined.  In all of these proofs, make sure you point out reasons why things are nonzero, when this is required, and reasons why things are unique, when this is required.

\item   Prove 4.16 part i.  This wasn't used in the proof of the Binomial Theorem, at least as I presented it (he did use it:  I used the definition of summation and the result of problem 2 in this set instead).  You might be able to use the result of problem 2 to prove it.

\end{enumerate}

\end{document}