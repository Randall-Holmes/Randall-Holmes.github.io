\documentclass[12pt]{article}

\title{Graph theory notes for Math 189, Fall 2022}

\author{Dr Holmes}

\begin{document}

\maketitle

Much of this is actually in Levin, but I want to write things out in more or less the way I presented them, and there are some things I did which he does not do.

\begin{description}

\item[Definition:]   A {\em graph\/} is an ordered pair $(V,E)$, where $V$ is a set and $E$ is a collection of two element subsets of $V$.

If $G=(V,E)$ is a graph, we say that elements of $V$ are {\em vertices\/} of (or in) $G$ and elements of $E$ are {\em edges\/} of (or in) $G$.

If $a$ is a vertex and $e$ is an edge in $G$, we say that $a$ is incident on $e$ and that $e$ is incident on $a$ (in $G$) just in case $a \in e$.

If $a$ and $b$ are vertices in $G$, we say that $a$ and $b$ are {\em adjacent\/} in $G$ just in case $\{a,b\}$ is an edge in $G$.

If $e$ and $f$ are edges in $G$, we say that $e$ and $f$ are adjacent just in case they are distinct and share an element.

\item[Observation:]  We work with pictures of graphs, in which vertices are represented by dots and edges are represented by curves drawn between the dots.

We say that a graph is {\em planar\/} just in case it has a picture (drawn on an ordinary plane surface) in which the curves representing edges do not cross each other.  A planar graph is likely to have pictures in which the edges do cross each other:  we would be likely to call these non-planar pictures of a planar graph.  A graph is non-planar if {\em all\/} pictures of that graph are non-planar.  We will show later in the course that certain graphs are non-planar.

\item[Related ideas we won't usually talk about:]  $G=(V,E)$ is a directed graph if each element of $E$ is an ordered pair of elements of $V$ (edges can be pictured as arrows from one vertex to another).

$G=(V,E)$ is a multigraph if each element of $E$ is a three-element set two elements of which belong to $V$ (the third element serves as a label, making it possible to draw many edges between the same two vertices).

Either graphs or multigraphs can have the definition modified to allow loops (edges from a vertex to itself):  edges are sets of no more than two vertices, or no more than two vertices plus a non-vertex for multigraphs.  Directed graphs naturally allow loops.

\item[Identity of graphs:]  Two graphs are {\em equal\/} if they are literally the same ordered pair of sets.

Note that one can draw multiple different looking pictures of the same graph:  a kind of question I might ask is, are these two pictures of graphs pictures of equal graphs (or of the same graph, same question).

Graphs $G_1 = (V_1,E_1)$ and $G_2=(V_2,E_2)$ are {\em isomorphic\/} iff there is a bijection $f:V_1 \rightarrow V_2$ (a one-to-one function from $V_1$ onto $V_2$) such that for every $a,b \in V_1$, $\{a,b\} \in E_1$ if and only if $\{f(a),f(b)\} \in E_2$.

If two graphs are isomorphic, they have the same structure.  A picture of one can be turned into a picture of the other simply by relabelling the vertices (the process of relabelling is recorded in the function $f$).

$G_1$ and $G_2$ might be the same and there might be an isomorphism from $G_1$ to $G_2$ other than the identity map on the vertices of the graph (the graph might have some kind of symmetry).

You may be asked to find isomorphisms from one graph to another or argue that there cannot be one.

Notice that isomorphic graphs must have the same number of vertices.

\item[Degree of a vertex in a graph;  degree sequence:]  The degree of a vertex $a$ in $G$ is the number of edges incident in $a$ (or the number of vertices adjacent to $a$).

The degree sequence of a graph is the list of degrees of its vertices in nonincreasing order.

Notice that isomorphic graphs clearly must have the same degree sequence.

The converse is not true:  try to draw a pair of graphs which are not isomorphic and have the same degree sequence.

\item[Walks:]  Let $G=(V,E)$ be a graph and let $a$ and $b$ be vertices in $G$.  A walk from $a$ to $b$ of length $n-1$ is a function $w$ from $\{1,\ldots,n\}$ to $V$, a finite sequence of vertices, in which $w(1)$, which we usually write $w_1$, is $a$ and $w(n)$, which we usually write $w_n$, is $b$, and each pair $\{w_i,w_{i+1}\}$ ($1 \leq i <n$)belongs to $E$.  

We say that a walk $w$ from $a$ to $b$ is a path if and only if for all $i,j$, if $w_i = w_j$ then $\{i,j\}=\{1,n\}$:  the only way vertices can be repeated in the graph is if the walk starts and ends at the same vertex.

A walk visits $c \in V$ just in case there is $i$ such that $w_i = c$, and it traverses edge $\{c,d\}$ just in case there is $i$ such that $w_i = c$ and $w_{i+1}=d$.

A path which has its beginning and end the same we call a {\em cycle\/}.  We also call a graph a cycle if there is a cycle in it which visits all of its vertices and traverses all of its edges.

Notice that for
any vertex $a$ there is a walk of length 0 from $a$ to $a$, namely the function from $\{1\}$ to $V$ sending 1 to $a$.

\item[Theorem:]  If there is a walk from $a$ to $b$ in $G$, then there is a path from $a$ to $b$ in $G$.

\item[Proof:]  If there is a walk from $a$ to $b$ in $G$, there must be a walk $w$ of shortest length $n$ from $a$ to $b$.

Now suppose that $i < j$ and $w_i = w_j$.  We define $w'_k$  for $k \in \{1,n-(j-i)\}$ as  $w_k$ if $k \leq i$ and as $w_{k+(j-i)}$ otherwise.  It is easy to verify that $w'$ is a walk from $a$ to $b$ shorter than $w$, which is impossible.  The definition just formalizes the idea of skipping everything in $w$ between the identical $w_i$ and $w_j$.

\item[Definition (connected graph, 1):]  A graph $G$ is {\em connected\/} just in case for each pair of vertices $a,b$
there is a walk from $a$ to $b$ in $G$ (equivalently, by previous theorem, just in case for each pair of vertices $a,b$ there is a path from $a$ to $b$ in $G$).

\item[Definition (connected graph, 2)]:  A graph $G=(V,E)$ is {\em disconnected\/} just in case there
are disjoint nonempty sets $A,B$ such that $A \cup B = V$ and for every $e \in E$, either $e \subseteq A$ or $e \subseteq B$ (i.e, there is no edge from a vertex in $A$ to a vertex in $B$).  A graph $G$ is connected just in case it is not disconnected.

\item[Temporary rule:]  We cannot define the same concept in two different ways without proving that the definitions are equivalent!  For the moment, we call the concept defined by the first definition connected$_1$ and the concept defined by the second definition connected$_2$.

This state of suspense will not last long.

\item[Theorem:]  For any graph $G$, $G$ is connected$_1$ if and only if it is connected$_2$ (so we can say just ``connected").

\item[Proof:]  The proof falls into two parts, as is natural for the proof of an if and only if statement.

First we show that if $G$ is connected$_1$, it cannot be disconnected, and so must be connected$_2$.

We do this by introducing a graph $G$ arbitrarily, assuming that $G$ is connected$_1$, and further assuming that it is disconnected, and showing that a contradiction follows (remember that this is not an immediate contradiction, because disconnected means ``not connected$_2$").

Because we assume that $G=(V,E)$ is disconnected, we can choose disjoint nonempty sets $A,B$ such that
$A \cup B = V$ and there is no edge from an element of $A$ to an element of $B$.  Choose $a \in A$ and $b \in B$ (we can do this because $A,B$ are not empty).  

Since $G$ is connected$_1$, we can choose a walk $w$ from $a$ to $b$.  Notice that $w_1 \in A$ and $w_n$ is in $B$.

We use the well-ordering principle for a snappy proof here.  Let $S$ be the set of all $j$ such that
$w_j \in B$.  $S$ is nonempty since it contains $n$ and bounded below (by 1) so it has a smallest element $s$.
$s>1$ since $w_1 \not\in B$, so $w_{s-1}$ is defined.  Now there is an edge from $w_{s-1}$ to $w_s$.
$w_s \in B$ because $s \in S$,  $s-1 <s $ is not in $S$ because $s$ is the smallest element of $S$.  So $w_{s-1} \not\in B$.  But this means $w_{s-1} \in A$.  But this is a contradiction:  we cannot have an edge from an element $w_{s-1}$ of $A$ to an element $w_s$ of $B$.

Now we get to the second part of the proof:  we need to show that if a graph $G$ is connected$_2$ it must be connected$_1$.

We introduce a graph $G$ arbitrarily and suppose that $G$ is connected$_2$.  We suppose for the sake of argument that it is not connected$_1$ and show that a contradiction follows.

Since $G=(V,E)$ is not connected$_1$, we can choose vertices $a,b \in G$ such that there is no walk from $a$ to $b$.

Now we define $A$ as the set of all vertices $c$ in $G$ such that there is a walk from $a$ to $c$.  We define $B$ as
$V-A$.  Note that $A$ is nonempty because $a \in A$, $B$ is nonempty because $b \in B$ and $A \cup B$ is clearly $V$.

We claim that there cannot be a path from any $c \in A$ to any $d \in B$, which shows that $G$ is disconnected, which is a contradiction.  Let $c \in A$, $d \in B$, and $w$ be a walk from $c$ to $d$.  There is a smallest $i$ such
that $w_i \in B$ by the well-ordering principle, and $i>1$ because $w_1 = c \not\in B$.  Thus $w_{i-1}$ exists,
and $w_{i-1} \in A$ because $i-1$ is less than the smallest index $i$ of a vertex visited by the walk which is in $B$.
Thus there is a walk $w'$ from $a$ to $w_{i-1} = w'_j$.  But then $w' \cup \{(j+1,w_i)\}$ is a walk from $a$ to $w_i$
(just tack $w_i$ onto the walk $w'$) so $w_i \in A$, and this contradicts the choice of $w_i$ as the first vertex visited by the walk which is in $B$ and so not in $A$.

\end{description}



\end{document}