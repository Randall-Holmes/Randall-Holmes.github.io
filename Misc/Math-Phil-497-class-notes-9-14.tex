\documentclass[12pt]{article}

\usepackage{amssymb}

\title{Notes on M497 lecture, 9/14/2020}

\author{Randall Holmes}

\begin{document}

\maketitle

\section{Postulating mathematical objects as abstractions}

There are two ways that mathematical objects may be defined by abstraction.

Functions in the naive sense can be defined by abstraction from expressions.  When we write $f(x) = x^2+x+1$, for example, we are implicitly defining $f$ as abstracted from the expression $x^2+x+1$.  For any expression $E$, define $E[a/x]$ as the result of replacing the variable $x$ with the term $a$ in $E$.  The naive axiom of function abstraction asserts
that for any expression $E$ there is a function $(x \mapsto E)$ such that $(x \mapsto E)(a) = E[a/x]$.   For example, $(x \mapsto x^2+x+1)(2+2) = [x^2+x+1][2+2/x] = (2+2)^2 + (2+2)+1$.

Mathematical objects can also be described by abstraction from sentences.  This is perhaps more familiar:  for any sentence $P$, $a \in \{x :P\}$ holds iff $P[a/x]$.  Abstract objects associated with open sentences (which are often called propositional functions in our sources) can be thought of as collections, sets or classes in the most general sense.

When one takes functions as primitive, there is a natural way to represent sets, as functions whose values are truth values (propositional functions, in fact).  So for Frege
there is a function ${\bf X}^2 <4$ which takes each $x$ for which $x^2<4$ to the True, and other objects to the False.  In Frege's system the notation $(x \mapsto x^2 <4)$ is reserved for the course of values of this function, a notion which is found only in Frege's system.

When one takes sets as primitive (once one knows how to define ordered pairs) there is a natural way to implement functions:  the function we define by $f(x)=x^2+x+1$ is implemented as the ser of pairs $\{(x,y):y = x^2+x+1\}$.

The relationship between sets and their graphs may be interesting even in a context where functions are the primitive notion:  remember that in the Frege paper he presented the 
definition of a relation being functional in the sense that if $x \,R\,y$ and $x \,R\,z$ then $y=z$.  In set theory, such a relation would {\em be\/} a function.  But in Frege's system the
relation $x\,R\,y$ defined by $y=x^2+x+1$ is quite a different function from $f(x)=x^2+x+1$;  the first takes two object arguments and returns a truth value;  the second takes an object argument (a number) and returns a number.  There is a relationship:  if we define $x\,R\,y$ as $y=x^2+x+1$ and define $f(x)$ as $x^2+x+1$ then it is true that
$f(x) = \backslash(y \mapsto x \,R\,y)$ in Frege's system.  But this does not make $f$ and $R$ the same.   We might refer to $R$ as a functional relation in this context.

The association of a function with a set can happen even in systems with set foundations.  A common construction is the characteristic function.  For example, a set $A \subseteq {\mathbb R}$
has a characteristic function $\chi_A:R \mapsto \{0,1\}$ sending each element of $A$ to 1 and each element of ${\mathbb R}-A$ to 0.  This function is nothing like the set $A$:  it is a set of ordered pairs, containing $(a,1)$ for each $a \in A$ and $(r,0)$ for each real number $r$ not in $A$.  But the intellectual relationship between characteristic functions and propositional functions in a primitive sense is there.

Both forms of abstraction lead to error if they are taken to be unrestricted.

If we can define $\{x:P(x)\}$ for any expression $P(x)$, then we can consider $R= \{x:x \not\in x\}$, the Russell class.  For any $a$, $a \in \{x:x \notin x\}$ iff $a \not\in a$;  let us incestuously consider $a=R$:  we find that $R \in R \leftrightarrow R \not\in R$, which is a disaster.

The same sort of bad thing happens if we allow unrestricted formation of functions.  Suppose that we define $f(x)$ as 0 in case either $x$ is not a function defined at $x$  or $x(x) \not\in {\mathbb R}$;  otherwise define $f(x) = x(x)+1$.  The idea of applying a function to itself should make you queasy, and with good reason.  Notice that $f(x)$ is a number for any $x$ and so
for $f$, so $f(f) = f(f)+1$, but there is no real number which is the result of adding 1 to itself.

\section{Things Frege got right}

Frege's presentation of first-order logic, in spite of his weird notation, is the ancestor of ours.  Russell popularized his logic with a better notation, and it is still used with only minor changes, and interesting results were proved about it by G\"odel and others.

A tragic point is that Frege understood the importance of distinguishing between different types of object perfectly well.  When Russell first communicated his paradox, he talked about defining a propositional function $R(\phi) = \neg \phi(\phi)$, and Frege pointed out that one cannot apply a function to itself.  Unfortunately, Frege did postulate a correspondence between functions
of a single object argument with object values on the one hand and objects on the other, each function being associated with its course of values.  This correspondence enabled the translation of Russell's paradox into Frege's system.

Frege's discussion of logical types of function is actually quite modern.  He distingushes between objects, functions from objects to objects (first-level functions), functions taking two object arguments to an object (another kind of first-level function), second order functions taking first level functions of a single object argument to objects, and so forth.

The common characteristic of all his types is that each type of function has a list of simpler types for its arguments and a simpler type as its output:  so no function can be applied to itself,
because its arguments must belong to a type simpler than its own type.

An example of a second level function in Frege's system is the universal quantifier. $(\forall x:P(x))$ has output an object (a truth value) and input the function $P({\bf X})$, a function from objects to objects (in fact to truth values).  The universal quantifier over functions (which you may recall we needed) is a third level function constant in Frege's system for similar reasons.

Type discipline of this kind can be used to avoid the paradoxes, as Russell showed.  Frege has the type discipline but makes the impossible assumption that there are as many objects as there are functions (though in fact his system {\em can \/} be repaired by restricting the formation of functions and concepts cleverly).

\section{Starting to read Zermelo}

I note things which I find interesting under the first ten numbered points of the Zermelo paper. You do have the assignment of reading the paper for next time and seeing what you can make of it.

\begin{enumerate}

\item In section 1 he introduces the domain {\bf B} of all objects, some of which are sets, and notes the relation of identity $a=b$.

\item In section 2 he introduces the primitive relation $a \in b$.  He asserts that any object $b$ which has an element $a$ is a set (and that we will discover that there is a unique set with no elements later).  He leaves open the possibility that there are other objects which are not sets.

\item In section 3 he defines the usual subset relation $M \subseteq N$ as holding if $M$ and $N$ are sets and every element of $M$ is also an element of $N$.  He also defines disjointness.

\item In section 4 he introduces the idea of a propositional function $P(x)$.  This should seem familiar.  What he has to say about propositional functions being ``definite" may be mysterious, but he does say that statements about membership and inclusion are definite.

Then, still in section 4, he gives the first two axioms.  Axiom I says that any two sets with the same elements are equal.  Notice that this means that there can be no more
than one set with no elements, but it does not rule out the possibility that there might be many distinct non-sets, all with no elements.  Axiom II says that there is an empty set 0.
for any object $a$ a set $\{a\}$ with only $a$ as an element, and for any two objects $a,b$ a set $\{a,b\}$ with only $a,b$ as elements.  By Axiom I these sets are unique.

\item In section 5 he discussed the uniqueness of the objects introduced by Axiom II.  One might think about whether what he says about definiteness is informative or useful.
I suppose that in number 4 he did not postulate that $a=b$ was definite, but he has postulated that all statements $a \in b$ are definite, and $a \in \{b\}$ is equivalent to $a=b$.

\item In section 6 he notes that the empty set is a subset of every set.  Then he makes a rather important remark:  he refers to all subsets of $A$ other than 0 and $A$ itself as
``parts" of $A$.  It is an important point that the relation of element to set is not like the relation of part to whole, but the relation of a subset to the set it is included in is:  the inclusion relation is transitive, for example, where the membership relation is not.

Still in section 6, he introduces Axiom III, which is extremely important (it is the local instance of an abstraction principle, in this case from sentences, restricted so as to avoid problems).
He says that whenever we are given a set $M$ and a propositional function $P(x)$ definite on elements of $M$, we get a set (which I will write in modern notation) $\{x \in M:P(x)\}$ (he writes $M_P$) such that \newline for any $a$, $a \in \{x \in M:P(x)\}$ iff $a \in M$ and $P(a)$.  We collect the elements with property $P$, but only those which lie in the previously given set $M$.  We will see below how this protects us from paradox.

\item  In section 7, he discusses the relative complement:  if $M_1 \subseteq M$ (and even if this is not true) we can define $M-M_1$ as $\{x \in M:x \not\in M_1\}$.

\item In section 8, he discusses for any sets $M,N$ the existence of the set he writes $[M,N]$, which we would write $M \cap N$, defined as $\{x \in M:x \in N\}$ (or vice versa).

\item In section 9, he notes the possibility of defining intersections $[M_1,M_2,M_3,\ldots,M_n]$ of more than two sets in the obvious way, and also of defining for any nonempty set
$T$ whose elements are sets the intersection which we would write $\bigcap T = \{x \in t_0:(\forall t \in T:x \in t)\}$, where $t_0$ is any element of $T$.  What goes wrong if we try
to define the intersection $\bigcap 0$?

\item In section 10, we make a practical application of the Russell argument!  In fact, Zermelo uses the construction in this proof later in the paper for mathematical purposes.

\begin{description}

\item[Theorem:]  For any set $M$, there is a set $M_0 \subseteq M$ such that $M_0 \not\in M$.

\item[Proof:]  We define $M_0$ (appealing to Axiom III) as $\{x \in M:x \not\in x\}$.  Zermelo notes that Axiom III can be applied because $x \in x$ is definitely true or false for
each $x \in M$.  He notes specifically that his axioms do not exclude $x \in x$ from being true for some $x$.

Obviously $M_0 \subseteq M$.  Suppose for the sake of a contradiction that $M_0 \in M$.  Note first that for any $a$, $a \in M_0 = \{x \in M:x \not\in x\}$ if and only if $a \in M$ and $a \not\in a$.  Now take $a=M_0$:  $M_0 \in M_0$ iff $M_0 \in M$ (which we have supposed true) and $M_0 \not\in M_0$.  But this is a contradiction:  it follows then that our hypothesis $M_0 \in M$ must be false.

\end{description}

Zermelo notes that this proof shows that his domain $\bf B$ cannot be a set.  The question might arise then what exactly $\bf B$ is, and this has been pursued by later students of the subject.

\end{enumerate}



\end{document}

