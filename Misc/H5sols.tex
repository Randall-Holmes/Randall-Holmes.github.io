\documentclass[12pt]{article}

\title{Homework 5 Solutions, Math 189}

\author{Randall Holmes}

\begin{document}
\maketitle

Homework 5, due Friday Sept 16: Levin section 1.5, probs 1, 2, 4, 6, 7, 10, 11; section 1.6 probs 2, 3, 5, 8

I notice that I didn't directly lecture central issues in 1.6 (an oversight, my fault), and I'll accommodate for this in marking.  I'll decide on the exact accommodation when I see the papers.

\begin{description}

\item[Levin 1.5, 1:]  a:  (9 choose 5) = 126

b:  ($n+k-1$ choose $k$) with $n=10, k=5$, that is, (14 choose 5) = 2002.  His answer of 14 choose 9 is equivalent.

\item[2:]   In both parts, we are talking about seven possible digits.

a:  (7 choose 5) = 21:  choose five distinct digits and write them in increasing order.

b:  ($n+k-1$ choose $k$) with $n=7, k=5$, that is (11 choose 5) = 462.

\item[4:]  a:  ($n+k-1$ choose $k$) with $n=5, k=14$, or (18 choose 14) [equivalently, ($n+k-1$ choose $n-1$) with $n=5,k=14$, or (18 choose 4), that is 3060 by both methods.

b:  same as above but with $k=9$ instead of $k=14$ (5 of the choices are made for you) so
(13 choose 9) or (13 choose 4), i.e. 715.

\item[6:]  ($n+k-1$ choose $k$) with $n=6, k=5$, so (10 choose 5) = 252.

\item[7:]  The answer he intends is that there are ($n+k-1$ choose $k$) possible distributions with $n=4, k=7$ which is (10 choose 7) = 120, so a 1 in 120 chance.  I'll accept this but I think it is wrong:  the probabilities of the different distributions are not equal and so I think a better than 1 in 120 chance of being right can be arrranged with a suitable strategy.

\item[10:]  a:  (9 choose 5) = 126

b: (9+5-1 choose 5) = 1287.

\item[11:]  a:  (20 choose 4) = 4845

b:  $(20)(19)(18)(17)$ = 116280

c:  (20+4-1 choose 4)  = 8855

d:  $20^4$ = 160,000

\item[Levin 1.6: 2]  a:  (7+16-1 choose 16)

b:  (7 + 9 -1 choose 9):  seven of the choices are fixed

c:  This is rather tricky and I don't think I lectured an example!

For each of the 7 flavors, there are (7 + 11 -1 choose 11) = 74613 ways to choose at least 5 of that flavor.

You want to take all of these out, but then there is some overcounting.

7 times (7 + 11 -1 choose 11) counts the ways to have at least 5 of a flavor, but ways to have
5 of two flavors are counted twice, and ways to have 5 of three flavors are counted three times.

For each of the (7 choose 2) pairs of flavors, there are (7+6-1 choose 6) ways to choose at least 5 of those two flavors.

(7 times (7 + 11 -1 choose 11) ) minus ((7 choose 2) times  (7+6-1 choose 6))  counts each way of having at least 5 of just one flavor once, each way of having at least 5 of exactly two flavors once, and each way of having at least 5 of three flavors zero times (we started with it counted three times, then took it away three times).

For each of the (7 choose 3) choices of three flavors, there are (7+1-1 choose 1) ways of choosing at least 5 of those three flavors (one item left over to pick).

There are  (7 times (7 + 11 -1 choose 11) ) minus ((7 choose 2) times  (7+6-1 choose 6)) plus (7 choose 3) times (7+1-1 choose 1) = 30,345 ways to choose 5 or more of one of the flavors, so the answer is 74613-30345=44268.

I intended to lecture this and didnt:  I'll mark part c as EC unless performance is really good.

\item[3:]  Same strategy as part c of the previous problem:  

You start with (14+5-1 choose 14) ways to place the balls.  You need to exclude the ways of placing 7 of a color in a bin.

There are (7+5-1 choose 7) ways to put 14 balls in the 5 bins with at least 7 in a particular bin.

So there are 5 times (7+5-1 choose 7)  ways to put at least 7 balls in one of the bins...counting ways to put 7 balls in two of the bins twice.

For each of the (5 choose 2) choices of two bins there is just one way to put at least 7 balls in those two bins.

So the answer is (18 choose 4) minus [5*(11 choose 7) minus (5 choose 2)] = 1420.

EC for this unless performance is really good.


\item[5:]  The answer is (13 choose 7):  why use PIE at all?  [the answer of course is to illustrate that it works, in a situation where we can see the answer in another easier way].

\item[8:]  Choose one of 5 elements to fix.  Then we need to choose a map on the four remaining elements which doesn't fix any of the values.  So the answer is five times the number of permutations of 4 objects which don't fix any of them.

Now, there are 24 one to one maps from a set with 4 elements to itself.

For each of the 4 elements, there are two permutations which fix exactly that element (rotate the three elements one way or the other).  For each of the six pairs of two of the four elements, there is one permutation which fixes exactly those two elements.  There are no permutations which fix exactly three of the four elements:  any such permutation would fix the other element as well.  There is one permutation which fixes all four elements.  So there are 24-(2)(4)-(6)(1)-1 = 10 pemutations of 4 elements which don't fix any of them, and so there are (5)(10) = 50 permutations of a five element set which fix exactly one of its elements.

I'm treating this as EC, I didn't lecture anything like it.

\end{description}

\end{document}