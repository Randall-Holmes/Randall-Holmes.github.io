\documentclass[12pt]{article}

\title{Homework 8 solutions, Math 189, Fall 2022}

\author{Randall Holmes}

\begin{document}

\maketitle

I'm planning to check off this assignment for full credit if you turned it in, but I am making solutions available
so you can see how you did.

I am planning to revisit induction proofs and do harder ones as part of the logic and formal proof unit, which I think I will do last this time.  So you will see more of this.  You will also see induction used as a tool in the number theory and graph theory sections of the course, and questions about such induction proofs may be on the next exam.


\begin{description}
\item[section 2.5 problems 2,]

The statement you want to prove true for all $n\geq 0$ is $\sum_{i=0}^n 2^i = 2^{i+1}-1$

Basis (n=0):  $\sum_{i=0}^0 2^i = 2^{0+1}-1$ is the statement to be proved in the basis step.

$\sum_{i=0}^0 2^i$

$ = 2^0$ basic property of summations

$ = 1$

$ = 2-1$

$= 2^{0+1}-1$

Induction step:  Let $k\geq 0$ be chosen arbitrarily.

Suppose that $\sum_{i=0}^k 2^i = 2^{k+1}-1$ (inductive hypothesis = ind hyp)

Our goal is to show that $\sum_{i=0}^{k+1} 2^i = 2^{k+2}-1$ follows.

$\sum_{i=0}^{k+1} 2^i$

$=[\sum_{i=0}^{k} 2^i] + 2^{k+1}$  pull last term out of a summation:  a basic property of summations

$=2^{k+1}-1 + 2^{k+1}$  IND HYP (fireworks go off)

$= 2^{k+2}-1$ algebra

\item[ 5,]  Prove that $7^{n}-1$ is a multiple of 6 for all $n$

Basis (n=0):  $7^0-1 = 0$ is indeed a multiple of 6.

Induction step:  Let $k \geq 0$ be chosen arbitrarily.

Assume that $7^k-1$ is divisible by 6.

Goal:  show that $7^{k+1}-1$ is divisible by 6.

Proof of induction step (this involves pulling a rabbit out of a hat in a way which you should have encountered in other proofs):

$7^{k+1}-1$

$= (7^{k+1}-7^k) + (7^k - 1)$  add and subtract the same thing

$ = 6\cdot 7^k + (7^k-1)$:  this is the sum of two terms, the first, $6\cdot 7^k$, obviously divisible by 6
and the second, $(7^k - 1)$, divisible by six by IND HYP (fireworks go off);  the sum of two numbers divisible by 6 is divisible by 6, so we have proved the induction goal.

\item[6,]  Prove that $2^n < n!$ for all $n \geq 4$

Basis (n=4):  $2^4 = 16 < 24 = 4!$.  Check.

Induction step:  Let $k \geq 4$ be chosen arbitrarily.

Suppose that $2^k < k!$.  IND HYP

Our goal is $2^{k+1} < (k+1)!$.

Multiply both sides of the ind hyp (this is where it is used) by 2 to get $2^{k+1}< k!(2)$.

Since $k\geq 4$, we have $2 < k+1$, so $k!(2) < k!(k+1) = (k+1)!$.

So we have shown $2^{k+1} < k!(2) < (k+1)!$, and the proof is complete by transitivity of $<$.



\item[10 (we did it in class, no reason you shouldn't do your own writeup though),]

Prove that the sum of the first $n$ squares is $\frac{n(n+1)(2n+1)}6$, that is

$\sum_{i=1}^n i^2 = \frac{n(n+1)(2n+1)}6$ is true for $n \geq 1$.

Basis (n=1):  $\sum_{i=1}^1 i^2 = 1^2$ (by basic property of summations) $= 1 = \frac{1(1+1)(2\cdot 1 +1)}6$.  Check.

Induction step: Let $k \geq 1$ be chosen arbitrarily.

Suppose that $\sum_{i=1}^k i^2 = \frac{k(k+1)(2k+1)}6$

The goal is to prove that $\sum_{i=1}^{k+1} i^2 = \frac{(k+1)(k+2)(2k+3)}6$ follows.

The proof of the induction goal:  

$\sum_{i=1}^{k+1} i^2$

$ = [\sum_{i=1}^{k} i^2] + (k+1)^2$  pull last term out of a summation

$= \frac{k(k+1)(2k+1)}6 + (k+1)^2$  IND HYP (20 gun salute)

$ = \frac{k)(k+1)(2k+1) + 6 (k+1)^2}6$

$ = \frac{(k+1)(k(2k+1) + 6(k+1))}6$

$ = \frac{(k+1)(2k^2+7k+6)}6$

$= \frac{(k+1)(k+2)(2k+3}6$, establishing the induction goal.

\item[ 17,]  A number is even iff it is two times an integer.

Prove by induction that for any $n \geq 0$, $n^2+n$ is even.

Basis (n=0):  $0^2+0=0$ is even.  Check.

Induction step:  let $k \geq 0$ be chosen arbitrarily.

Assume (IND HYP) that $k^2+k$ is even.

The induction goal is to show that $(k+1)^2 + (k +1)$ is even.

Proof of the induction goal:  Since $k^2+k$ is even (IND HYP) there is an integer $m$ such that $k^2+k = 2m$.

$(k+1)^2 + (k+1) = k^2+2k+1+k+1 = (k^2+k)+(2k+2) = 2m + 2(k+1) = 2(m+k+1)$ which is even because $m+k+1$ is an integer.

\item[ 23,]   Show that the sum of the row of Pascal's triangle with second number $n$ is $2^n$.

The statement to be proved is that $\sum_{i=0}^n (n$ choose $i) = 2^n$

The hint tells you to use the identity (n choose k-1) + (n choose k) = (n+1 choose k).

I also use some clever manipulations of summations which I would like you to understand.

Basis:  $\sum_{i=0}^0 (0$ choose $i)$ = (0 choose 0) = 1 = $2^0$, check

Induction step:  Let $k \geq 0$ be chosen arbitrarily.

Assume $\sum_{i=0}^k (k$ choose $i) = 2^k $  (IND HYP)

Goal:  $\sum_{i=0}^{k+1} (k+1$ choose $i) = 2^{k+1} $

Proof:  I will work from right to left.

$2^{k+1} = 2^k + 2^k = \sum_{i=0}^k (k$ choose $i) + \sum_{i=0}^k (k$ choose $i)$ by IND HYP (gong is sounded),
which is equal to  \newline $\sum_{i=0}^k (k$ choose $i) + \sum_{i=1}^{k+1} (k$ choose $i-1)$ by changing the indexing in the second copy of the sum.  Pull out the first term from the left summation and the last term from the right summation to get
\newline $(k$ choose $0) + \sum_{i=1}^k (k$ choose $i) + \sum_{i=1}^{k} (k$ choose $i-1)+ (k$ choose $k)$.

The middle two sums now have the same range and can be added term by term.

$(k$ choose $0) +[\sum_{i=1}^k ((k$ choose $i) + (k$ choose $i-1))] + (k$ choose $k)$, and we apply the identity

$(k$ choose $0) +[\sum_{i=1}^k (k+1$ choose $i)] + (k$ choose $k)$

and observe that we can replace (k choose 0) with (k+1 choose 0) (both are 1) and similarly replace (k choose k) with 
(k+1 choose k+1) and see that we actually have $\sum_{i=0}^{k+1} (k+1$ choose $i) $.




\item[29 (read the hint).]  This problem is visual and I'll do it in class on request.

\end{description}


\end{document}