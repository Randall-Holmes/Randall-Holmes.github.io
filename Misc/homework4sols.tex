\documentclass[12pt]{article}

\title{Homework 4:  Counting Word Problems}

\author{Randall Holmes}

\begin{document}

\maketitle

Please use the space I give you on each page to write out your thinking in English, or at the very least write computations which make it clear how you came up with the answer.  Some English combined with appropriate structured computations is best.

These are due on Wednesday the 14th, to give you plenty of chances to ask questions.

\newpage

\begin{enumerate}

\item (modelled on a problem we did in class)

Alma needs to take two classes to complete her degree.  Classes are offered in morning and afternoon slots.  She is interested in a math course, a business course, and an engineering drawing course.

There are two sections of math, three of business, and five of engineering drawing in the morning.

There is one section of math, five sections of business, and three sections of engineering drawing in the afternoon.

In how many ways can she fill out her schedule to take two different courses next term and finish her degree?

{\bf Solution:}  There are three ways to choose the morning class, either math, business, or drawing.

If the morning class is math, she has $(2)(5+3)$ ways to complete her schedule (two math sections in the morning,
5+3 non-math sections in the afternoon).

If the morning class is business, she has $(3)(1+3)$ ways to complete her schedule (three business sections in the morning,
1+3 non-business sections in the afternoon).

If the morning class is drawing, she has $(5)(1+5)$ ways to complete her schedule (five drawing sections in the morning,
1+5non-drawing sections in the afternoon).

So the total is $2(5+3) + 3(1+3) + 5(1+5)= 58$ ways to complete her degree next term.

\newpage

\item (modelled on a problem we did in class)

How many four digit numbers contain both a 1 and a 2, and neither a 0 or a 9?

There are 8 digits which are not 0 or 9.  There are $8^4$ sequences of four digits containing no 0 or 9.

There are $7^4$ sequences of four digits containing no 1's and $7^4$ sequences of four digitis containing no 2's:
if we take these out, $8^4-2\cdot 7^4$ this is not quite right because we have taken out each of the $6^4$ sequences which contain neither a 1 nor a 2, twice.  So add them back in: $8^4-2\cdot 7^4+6^4=590$ such numbers.

Other correct methods of solution exist:  if students come up with another one, I'll try to include it here.

\newpage

\item  A committee with 17 members has 10 male members and 7 female members.  They are going to select an Equity Committee, which must be ``gender balanced":  the number of men and the number of woman will either be the same or differ by one (so that a committee with an odd number of members can be gender balanced).

In how many ways can the Equity Committee be chosen if it is to have four members?  (hint:  use binomial coefficients and the multiplication principle).

(10 choose 2) times (7 choose 2) = 945

In how many ways can the Equity Committee be chosen if it is to have five members (hint: you will need to use the addition principle as well in this case).

((10 choose 3) times (7 choose 2)) plus ((10 choose 2) times (7 choose 3)) = 4095.

\newpage

\item  An organization has 8 math majors, 12 computer science majors, and 6 science majors.

In how many ways can the organization choose a president, vice president, and secretary in each of the following situations?:

\begin{enumerate}
\item   The three officers must all be of the same major.

\item   At least one of the officers must be a science major.

\item  Either the president or the vice president must be a math major.

\end{enumerate}

For some of these, I would think about the complement (think of the number of things for which the condition doesn't hold).

a:  (8)(7)(6) + (12)(11)(10) + (6)(5)(4) = 1776

b:  There are 26 students in all, so (26)(25)(24) = 15600 ways to choose officers with no restriction.  There are 14 students who are not science majors so there are (14)(13(12) = 2184 ways to choose officers excluding the science majors.
So there are 15600 - 2184 = 13416 ways to choose officers in such a way as to include at least one science major.

c.  There are 18 students who are not math majors.  (18)(17)(24) counts the number of ways to choose a non math major president and vice president, so there are (26)(25)(24) - (18)(17)(24) = 8256 ways to choose the officers meeting these conditions.  

If we approach it positively, there are (8)(25)(24) ways to choose a math major as president, and also
(8)(25)(24) ways to choose a math major as VP (choose VP first).  But beware, if we just add these numbers
we are counting solutions in which both the president and the vice president are math majors (there are (8)(7)(24) of these) twice.  (8)(25)(24) + (8)(25)(24) - (8)(7)(24) gives the same answer 8256.

\newpage

\item  We want to count the number of five card hands (considered as five cards drawn {\em in order\/} from a standard deck of cards) in which the first card has the same rank as the last (the first and last card differ only in suit).

The book suggest the following algorithm:  50 choices for the first card, 49 for the second, 48 for the third, 47 for the fourth,
and 3 for the fifth (since it must be of the same rank as the first card), so $(50)(49)(48)(47)(3)$ ways.  This is wrong:
explain why.

{\bf Solution:}  An easy way to see that it is wrong is to observe that if we choose an ace as our first card, then choose aces for the second third and fourth cards, we have {\bf no} choices remaining for the last card.  The starting number is also strange (and might be my mistake, I might have miscopied the problem).

Compute the number of such five card hands correctly.

{\bf Solution:}  We need to choose the first card dealt (52 choices) then choose the last card dealt (3 choices) then we have 50 choices for the second card dealt, 49 for the fourth, and 48 for the fifth, numerical answer 18,345,600.

\newpage

\item  In how many ways can four married couples be seated at a round table in such a way that spouses sit together?

Notice that the round table means that we do not have a special starting point, as we would if they were standing in a line.  How do we adjust for this?

Seat the first couple (4 choices) in one of two ways, the second couple (3 choices) in one of two ways, the third couple (2 choices) in one of two ways and the last couple in one of two ways, for 192 different linear arrangements of four married couples.  We then divide by four, because any circular arrangement of the four couples corresponds to exactly four linear arrangements, starting with one of the couples and proceeding clockwise:  the answer is 48.

You could also divide by four in effect by dictating that we choose one of the four couples to start with (we can, because the table is round and there is no specified starting point.  Then there are two ways to seat the designated couple, three ways to choose the couple to their right and two ways to seat them, two ways to choose the couple next to the right and two ways to seat them, and two ways to seat the last couple:  (3)(2)(2)(2)(2) = 48.

\newpage

\item  Your friend spins a spinner with three colors (green, red, blue) and you record the results of six consecutive spins.

How many possible results are there?  {\bf Solution:}  $3^6 = 729$

How many possible results are there with exactly two of each color?  {\bf Solution:}  (6 choose 2) times (4 choose 2) times (2 choose 2):
first choose where the greens happen, then where the reds happen, then you know where the blues happen.  The numerical answer is 90.

\newpage

\item  In how many ways can a group of 15 children divide themselves into three teams of five, the Lions, the Tigers, and the Bears?

(15 choose 5) times (10 choose 5) times (5 choose 5) = 756,756

In how many ways can a group of 15 children divide themselves into three teams of five?  This is not the same question.

With any division of the children into three teams of five, there are six different ways to assign the three names to the teams.
So there are $\frac {756756}6$ ways to divide them into three teams, since there are six times as many ways to divide them into named teams:  the numerical answer is 126,126.

\newpage

\item  License plates for a certain state have three letters followed by four digits.

How many plates are possible with no restrictions?  $26^310^4=175,760,000$

How many plates are possible if no letter can be repeated?  $26\cdot25\cdot24\cdot 10^4 = 156,000,000$

How many plates are possible if no letter or digit can be immediately followed by the same letter or digit?
$25\cdot25^2\cdot 10 \cdot 9^3=118,462,500$

How many plates contain exactly one 8?  $26^3*4*9^3=51,251,616$:  choose the letters, then choose the position of the 8, then make one of nine choices for the digit in each of the remaining three spaces.

How many plates contain no X's?  $25^310^4 = 156,250,000$

\newpage

\item  A combination lock has five digits on it, 1 to 5.  A combination consists of three items in order, each item being either a digit
or two digits pressed at the same time.  No digit will appear more than once in a combination.  How many possible combinations for the lock are there?

This problem will combine use of binomial coefficients, the multiplication principle and the addition principle.

For each pattern of number of keys pressed, handle this separately:

if 1 key is pressed each time, (5)(4)(3) = 60 combinations

2-1-1  (5 choose 2) times 3 times 2, again 60

There are 60 1-2-1 and 60 1-1-2 combinations as well, by the same argument (choose the two digits together
first in your counting argument and you get the same story)

2-2-1 combinations (5 choose 2) times (3 choose 2) times 1 = 30

there are 30 2-1-2 combinations and 30 1-2-2 combinations by the same reasoning

There are no 2-2-2 combinations

So there are 60 + (3)(60) + (3)(30) = 330 possible combinations.

\end{enumerate}

\end{document}