\documentclass[12pt]{article}

\usepackage{amssymb}

\title{Class Lecture Notes for Math 305, Spring 2022}

\author{Dr Holmes}

\begin{document}

\maketitle

These are notes on what I say in class in Math 305.

\tableofcontents


\section{Tuesday, January 11, 2022}

Administrative preliminaries.


I discussed the definitions of ${\mathbb Z}$, ${\mathbb N}$, ${\mathbb Z}^+$: 

${\mathbb Z} = \{\ldots,-2,-1,0,1,2,\ldots\}$, the set of integers;

${\mathbb N}= \{0,1,2,\ldots\}$, the set of natural numbers (there is no general agreement in mathematical literature as to whether 0 is a natural number, but this book includes it), or non-negative integers;

${\mathbb Z}^+ = \{1,2,3,\ldots\}$, the set of positive integers.  In all of these, the use of dots is really cheating:  giving a rigorous definition of these sets is rather difficult, and we appeal instead to your pre-formal understanding of these concepts.

I stated a set of axioms for the integers which I will include here (based on the axioms in the Math 287 book with two alternative approaches to order).

We begin with a set of purely algebraic axioms.  Our variables range over the set $\mathbb Z$ of integers;  we assume special integers 0 and 1 and primitive operations or addition (+) multiplication ($\cdot$) and additive inverse ($-$, used as a prefix unary operator).

\begin{description}

\item[commutative laws:]  For any $x,y \in {\mathbb Z}$, $x+y=y+x$ and $x \cdot y = y \cdot x$.

\item[associative laws:]  For any $x,y,z \in {\mathbb Z}$, $(x+y)+z = x+(y+z)$ and $(x \cdot y)\cdot z = x \cdot(y\cdot z)$.  I should add that we are only allowed to write things like $x+y+z$ or $x \cdot y \cdot z$ because we know these operations are associative.  In proofs in section 1.2 you should write parentheses, and explicitly use the associative laws to move them.

You {\em may\/} use standard order of operations and read $x \cdot y + z$ as meaning $(x \cdot y) + z$ wihtout writing out the parentheses (multiplication binds more tightly than addition, unary minus binds more tightly than either).

\item[distributive law:]  For any $x,y,z \in {\mathbb Z}$, $x\cdot(y+z) = x \cdot y + x \cdot z$.

\item[identity laws:]  For any $x \in {\mathbb Z}$, $x+0=x$ and $x \cdot 1 = x$.  $0 \neq 1$.

\item[multiplicative cancellation:]  For any $x,y,z \in {\mathbb Z}$, if $x \neq 0$ and $x\cdot y = x\cdot z$, then $y=z$.
This amounts to the ability to divide both sides of an equation by the same thing, but we do not have a full division operation in the integers as we do in the rationals or reals.

\end{description}

This is not a full axiomatization of the integers.  Of course, systems like the rationals and the reals which extend the integers
satisfy these axioms, but there are also systems (even ones familiar to you) which satisfy these axioms and are quite different from the integers.  Arithmetic mod $p$ where $p$ is prime satisfies these axioms, and the domain of ``numbers" in mod $p$ arithmetic is finite (the remainders $0,1,\ldots p-1$ mod $p$).

Additional axioms appropriate for the integers which rule out the system described being modular arithmetic are axioms of order.  We present these (just for fun) in two different ways.

We can axiomatize order by introducing the set of positive integers ${\bf Z}^+$ as a primitive notion, providing some of its properties as axioms, and using it to define order relations.

\begin{enumerate}

\item $0 \not\in {\mathbb Z}^+$.

\item  For each $m \in {\mathbb Z}$ with $m \neq 0$ either $m \in {\mathbb Z}^+$ or  $-m \in {\mathbb Z}^+$.

\item For each $m,n \in {\mathbb Z}^+$, we have $m+n \in {\mathbb Z}^+$ and $m\cdot n \in {\mathbb Z}^+$

\item Define $m<n$ as $n+(-m) \in {\mathbb Z}^+$.

\end{enumerate}

This is a very elegant set of axioms, and it should be straightforward for you to see that they are true in the familiar system of integers, but it may be less obvious that they are enough.  This might be a homework exercise.

Here is a more familiar set of axioms for order.  They do follow as consequences of the algebraic and positive integer axioms if we define $<$ as above, but for this approach we ``forget" about ${\bf Z}^+$ and take $<$ as a primitive relation (and we define ${\mathbb Z}^+$ in terms of $<$).

\begin{description}

\item[transitivity:]  For any $m,n,p \in {\mathbb Z}$,  if $m<n$ and $n<p$ then $m<p$.

\item[trichotomy:]  For any $m,n \in {\mathbb Z}$, exactly one of $m<n,m=n,n<m$ is true.

\item[additive monotonicity:]  For any $m,n,p \in {\mathbb Z}$, if $m<n$ then $m+p < n+p$.

\item[multiplicative monotonicity:]  For any $m,n,p \in {\mathbb Z}$, if $p  > 0$ and $m<n$, then $m\cdot p <n\cdot p$.  Our axioms are enough to show that the right things happen if $p$ is zero or negative (that might be a homework exercise).

\item  [definition of positive integers:]  We define ${\mathbb Z}^+$ as $\{x \in {\mathbb Z}:0<x\}$.

\end{description}

I stated the Well-Ordering Principle and proved two sample theorems, ``each positive integer is either even or odd", and ``there is no integer strictly between 0 and 1".

If $S$ is a set of integers, $x$ is a smallest element of $S$ iff $x \in S$ and $(\forall y \in S:x \leq y)$.  You could try proving that a nonempty set with a smallest element has just one smallest element.

\begin{description}

\item[Well-Ordering Principle:]  Any nonempty set $S$ of positive integers has a smallest element.

\end{description}

I proved a couple of sample theorems using the Well-Ordering Principle in class.  Proofs using this principle are usually indirect (proofs by contradiction);  pay attention to the logical structure of what I say.

\begin{description}

\item[Definition:]  An integer $m$ is even iff there is an integer $x$ such that $m=2\cdot x$.  

An integer $m$ is odd iff there is an integer $x$ such that $m=2\cdot x +1$.

\item[Theorem:]  Each positive integer is either even or odd.

\item[Proof:]  Suppose otherwise, so there are integers which are neither even nor odd.  Let $S$ be the set of all integers which are neither even nor odd.  By our assumption, it is nonempty, so by the Well-Ordering Principle it has a smallest element $w$.  This number $w$ will be the smallest integer which is neither even nor odd.

The integer $w$ is not 1, because $1 = 2\cdot 0 +1$ is odd.

So $w-1$ is a positive integer, and because it is less than $w$ it must be either even or odd.

If $w-1=2\cdot x$ is even, then $w=2\cdot x +1$ is odd.

If $w-1=2\cdot x +1$ is odd, then $w = 2 \cdot x +2 = 2 \cdot(x+1)$ is even.

In either case, we get that $w$ is either odd or even, which is a contradiction, so there can be no such $w$ and the theorem must be true.

\item[Observation:]  At a crucial point in the argument above, I cheated (or at least I appealed to your intuition), and the fact is used is important and should be proved.  How do I know that if $w\neq 1$ is a positive integer that $w-1$ is a positive integer?
If we have $w>1$, we do have $w-1>0$.  We need to rule out the possibility that $0<w<1$ (which, since we know what the integers are, is hard to even take into account).

\item[Theorem:]  There is no integer $x$ such that $0<x<1$.

\item[Proof:]  If there is such an integer than the set $S = \{x \in {\mathbb Z}:0<x<1\}$ is nonempty and so by the Well-Ordering Principle has a smallest element $w$.

So we have $0<w<1$.  By multiplicative monotonicity (because $w>0$) we have $0<w^2<w$ and of course
we then have $0<w^2<w<1$.  Using transitivity we see that $0<w^2<1$ and $w^2<w$, so $w^2$ belongs to the set
$S$ but is smaller than $w$, which is a contradiction.


\end{description}

\newpage

\section{Homework 1}

This is being assigned on January 13 and is due January 20.

\begin{enumerate}

\item  Prove all parts of proposition 1.2.8 in Crisman on properties of divisibility (this is on page 4 of Crisman).

\item  Prove by mathematical induction that for every $n \in {\mathbb Z}^+$, $3|(n^3+5n)$.

\item  Prove by mathematial induction that the sum of the first $n$ odd numbers is $n^2$.  Make appropriate use of summation notation.

\item Use the first set of order axioms in these notes (in which the set of positive integers is primitive) along with the algebra axioms to prove at least two of the axioms in the second set of order axioms, in which the less-than relation is primitive (you will need to use the definition of the less-than relation given with the first set of axioms).  Your algebra may be somewhat informal:  your use of order axioms should be careful and explicit.  Extra credit will be rewarded for proving more of the order axioms in the second set.

\end{enumerate}

\newpage

\section{Thursday, January 13, 2022}

We begin with the principle of mathematical induction.

Mathematical induction can be presented as a proof strategy.

\begin{description}

\item[Goal:]  Prove $(\forall n \in {\mathbb Z}^+:P(n))$

\item[basis step:]  Prove $P(1)$

\item[induction step:]

\begin{description}

\item

\item[induction hypothesis:]  


Choose a natural number $k$ arbitrarily.  \newline Assume $P(k)$.

\item[induction goal:]  Prove $P(k+1)$ (under the assumption that $P(k)$ is true).

\end{description}

If you succeed in proving the induction goal, assuming the induction hypothesis, you have proved $(\forall k \in {\mathbb Z}^+:P(k) \rightarrow P(k+1))$.



\end{description}
If you complete both steps, you can conclude $(\forall n \in {\mathbb Z}^+:P(n))$
 by mathematical induction.

One can prove theorems by mathematical induction on ${\mathbb N}$ instead of ${\mathbb Z}^+$:  in this case the basis step is to prove $P(0)$.

We give an example (which also illustrates nice tools for working with summation notation).

\begin{description}
\item[Theorem:]  The sum of the first $n$ squares of positive integers is $\frac{n(n+1)(2n+1)}6$, that is, $$\sum_{i=1}^ni^2 =\frac{n(n+1)(2n+1)}6$$

\item[Proof:]

We prove this by mathematical induction on $n$.

\begin{description}

\item[basis step:]  Prove $\sum_{i=1}^1i^2 = \frac{(1)(1+1)(2\cdot 1+1)}6$

$\sum_{i=1}^1i^2 = 1^2=1= \frac{(1)(1+1)(2\cdot 1+1)}6$ (check)

\item[induction step:]  Choose a positive integer $k$ arbitarily.\newline
Assume (ind hyp) that $\sum_{i=1}^ki^2 = \frac{(k)(k+1)(2k+1)}6$

The induction goal is to prove  $\sum_{i=1}^{k+1}i^2 = \frac{(k+1)(k+2)(2k+3)}6$

Notice that in formulating the induction goal we allowed ourselves to do a little obvious algebra after replacing $k$ with $k+1$.

$\sum_{i=1}^{k+1}i^2  = [\sum_{i=1}^{k}i^2 ]+ (k+1)^2$ (pulling out the last term) $= \frac{(k)(k+1)(2k+1)}6+(k+1)^2$ (by ind hyp: {\bf ALWAYS highlight the use of the inductive hypothesis in any proof by induction}) $= \frac{(k+1)(k)(2k+1) + 6(k+1)(k+1)}6 = \frac{(k+1)((2k^2+k)+(6k+6))}6 = \frac {(k+1)(2k^2+7k+6)}6 = \frac {(k+1)(k+2)(2k+3))}6$ (check)                                                                                                                                                                                                                                                                                                                                                                                                                                                                                                                                                                    


\end{description}

The proof by induction is complete.

\end{description}

I was impressed with success in Math 189 last term at teaching this general approach to proofs of statement involving summations, avoiding dots, which can cause various confusions.

Next, I lectured the equivalence of math induction to the well-ordering principle.

First assume the well-ordering principle and show that math induction follows:

\begin{description}

\item[Given:]

\begin{description}
\item
\item[1.]  $P(1)$

\item[2.] $(\forall k \in {\mathbb Z}^+:P(k) \rightarrow P(k+1))$

\item[3.] the well-ordering principle

\end{description}

\item[Show:]  $(\forall n \in {\mathbb Z}^+:P(n))$

\item[note:]  Convince yourself that if we complete this plan we really have shown that WOP does the work of math induction.

\item[Proof:]  Suppose for the sake of a contradiction that $\neg (\forall n \in {\mathbb Z}^+:P(n))$, so there is some
$x$ a positive integer such that $\neg P(x)$.  Let $S$ be the set of all $x$ such that $\neg P(x)$, which we see is nonempty and so by WOP has a smallest element which we will call $w$.

$w$ is not 1, because we have assumed $P(1)$.  Thus $w>1$ (here we are using the result proved above using WOP that there is no integer strictly between 0 and 1).  Thus $w-1>0$ is a positive integer.  Thus we have $P(w-1)$, because $w$ is the smallest positive integer such that $\neg P(w)$.
But plugging $w-1$ in for $k$ in $(\forall k \in {\mathbb Z}^+:P(k) \rightarrow P(k+1))$ gives $P(w-1) \rightarrow P(w)$, and so since we have
$P(w-1)$ and $P(w-1) \rightarrow P(w)$ we have (by the rule of modus ponens) $P(w)$, but this is a contradiction.

Thus our assumption that $\neg (\forall n \in {\mathbb Z}^+:P(n))$ is incorrect, and we have $(\forall n \in {\mathbb Z}^+:P(n))$


\end{description}

This completes the proof that the well-ordering principle implies the principle of mathematical induction.

Now we argue that the principle of mathematical induction implies the well-ordering principle.

\begin{description}

\item[Given:]

\begin{description}
\item
\item[1.]  $S$ is a nonempty set of positive integers

\item[2.]  the principle of math induction

\end{description}

\item[Show:]  $S$ has a smallest element.

\item[note:]  Convince yourself that this proof plan really does show that the principle of math induction does the work of the well ordering principle, if we can carry it out.

\item[Proof:]  Assume for the sake of a contradiction that $S$ has no smallest element.  We will prove by induction that $S$ is empty, completing the desired contradiction.

We do not prove by induction that for every $n$, $n \not\in S$:  we prove the stronger statement that for every $n$, $(\forall m \in {\mathbb Z}^+: m \leq n \rightarrow m \not\in S)$:   not only is $n$ not in $S$, but no smaller positive integer is in $S$.
We will describe the strategy of strong induction of which this is an example in the last section of the notes for today.

The basis step for the induction is to show $(\forall m \in {\mathbb Z}^+:m \leq 1 \rightarrow m \not\in S)$:  the only positive integer less than or equal to 1 is 1 itself, so all we have
to show is $1 \not\in S$, and this follows from the assumption that $S$ has no smallest element:  if it contained 1, 1 would be its smallest element.

Choose an arbitrary positive integer $k$.  Assume $(\forall m \in {\mathbb Z}^+: m \leq k \rightarrow m \not\in S)$ as our induction hypothesis.  Our induction goal is to show that $(\forall m \in {\mathbb Z}^+: m \leq k+1 \rightarrow m \not\in S)$  If $m$ is an integer $\leq k+1$, it is either
less than $k$ or equal to $k$, in which cases the induction hypothesis tells us that $m \not\in S$, or (final case to be checked) $m>k$.
Now, because there is no integer strictly between 0 and 1, there is also no integer strictly between $k$ and $k+1$ (we could subtract $k$ from it to get between 0 and 1).  Thus, since $m>k$ and $m\leq k+1$, $m$ is simply $k+1$.  We can conclude $k+1 \not\in S$, because if it were in $S$ it would be the smallest element of $S$, since we have shown that nothing less than $k+1$ can belong to $S$.

So we have shown by induction that for every positive integer $n$, $(\forall m \in {\mathbb Z}^+: m \leq n \rightarrow m \not\in S)$, but this immediately implies that for every positive integer $n$, $n \not\in S$, so $S$ is empty, which is a contradiction.

This means that our assumption that $S$ has no smallest element must be false:  it follows from the statements given that $S$ has a smallest element.


\end{description}

This completes the proof that the well-ordering principle follows from the principle of mathematical induction. 

The final topic of this lecture was the method of strong induction.  This is a version of mathematical induction with a stronger hypothesis which is sometimes useful.  We will state it and prove a theorem as an example.  We state but do not prove (it might be fairly easy to see from the proof of equivalence of ordinary math induction and the well-ordering principle) that strong induction is in fact precisely equivalent in strength to ordinary induction.
But it is sometimes much more convenient.

We state strong induction as a strategy of proof.

\begin{description}
\item[Goal:]  Prove $(\forall n \in {\mathbb Z}^+:P(n))$

\item[basis step:]  Prove $P(1)$

\item[induction step:]
\begin{description}

\item
\item[induction hypothesis:]  Let $k$ be an arbitrarily chosen positive integer.  Assume $(\forall m \in {\mathbb Z}^+:m \leq k \rightarrow P(m))$:  instead of assuming just $P(k)$ we assume $P(1), P(2),\ldots,P(k)$.  This is a stronger hypothesis, and this is why we call this method strong induction.

\item[induction goal:]  Prove $P(k+1)$ under the assumption of the inductive hypothesis.

\end{description}

If you succeed in completing the basis and induction steps, you have proved $(\forall n \in {\mathbb Z}^+:P(n))$ by strong induction.

\end{description}

Here is an important example.  (I am not for the moment trying to expound this in terms of product notation as I suggested in class;  I might do it later, but my brain is tired after writing these notes).

\begin{description}

\item[Theorem:]  Each integer $\geq 2$ is a prime or a finite product of primes.

\item[Proof:]

we prove this by strong induction.

The basis step requires us to prove that 2 is a prime or a finite product of primes.  2 is a prime (check).

We choose an arbitrary positive integer $k\geq 2$.  The induction hypothesis will be that for every positive integer $m \leq k$, $m$ is a prime or a product of primes.

The induction goal is to prove that $k+1$ is a prime or a finite product of primes.

By the law of excluded middle either $k+1$ is a prime (in which case we are done, as it is then a prime or a finite product of primes)
or it is composite, in which case there are $a,b$ such that $2 \leq a,b \leq k$ and $ab=k+1$.  Now by inductive hypothesis, each of $a,b$ is either a prime or a finite product of primes, so $ab$ is a finite product of primes.  And this completes the proof of the theorem by strong induction.

\end{description}

\section{Tuesday, January 19, 2022}

Today I talked about the Division Algorithm and the Euclidean Algorithm (plain and extended).  I talked about this off the top of my head, and I owe you a discussion of what this material looks like in Crisman and how it might differ from what I say.

\begin{description}

\item[Theorem (division algorithm):]  For each $a \in {\mathbb Z}$ and $b \in {\mathbb Z}^+$, there are unique determined integers $q$ and $r$ such that $a=bq+r$ and $0 \leq r <b$.

Of course ``$q$" and ``$r$" are hints:  we give these variables these names because they suggest {\em quotient\/} and {\bf remainder\/}.

We prove the theorem using the Well-Ordering Theorem (and it is a positive result, we are not arguing by contradiction!)

\item[Proof:]

Define $S$ as the set $\{a-bq:q \in {\mathbb Z} \wedge a-bq\geq 0\}$.  This is the set of candidates for the remainder $r$, as it were.

It is a set of nonnegative integers, so if it is nonempty it has a least element.

If $a\geq 0$, let $q=0$ and we see that $a-bq=a\geq 0$, so $a \in S$ and $S$ is nonempty.

If $a <0$ let $q=a$ and we see that $a-bq = a-ba=a(1-b)$.  $a$ is negative and $1-b$ is nonpositive (since $b$ is positive), so $a(1-b)$ is nonnegative, and so belongs to $S$, so $S$ is nonempty.

Define $r$ as the smallest element of $S$.  There is a unique $q$ such that $r=a-bq$, so $a=bq+r$.

All that remains is to show $0 \leq r<b$.  We know that $r \geq 0$ because $r \in S$.  Notice that $a-b(q+1)$ must be negative, because
if it were nonnegative it would be an element of $S$ smaller than $r=a-bq$.

$a-b(q+1) = a-bq-b=r-b$ so we have $r-b<0$ so $r<b$ completing the proof.

% typo point for Bridger Lenz

We still need to prove that $q$ and $r$ are uniquely determined.  Suppose that $a=bq-r = bQ-R$ and $0\leq r < R <b$.

Observe that $R-r = b(Q-q)$.  Now $R-r<b$, and the only way for $b(Q-q)<b$ to be true is $Q-q=0$, so $Q=q$.  Then $r=a-bq = a-bQ = R$.

\item[Definition:]  For $a \in {\mathbb Z}$ and $b \in {\mathbb Z}^+$ define $a {\tt div} b$ and $a {\tt mod} B$ as the unique $q$ and $r$ whose existence is proved by the division algorithm.

\item[Observations:]  Be careful with negative values of $a$.  Notice that while $100 \,{\tt div}\, 3 = 33$ and $100\, {\tt mod}\, 3 = 1$, it turns out that 
$100\, {\tt div}\, 3 = -34$ and $100 \,{\tt mod} \,3 = 2$.

It might not be obvious that we can compute {\tt mod} with a simple calculator.  But we can.  For positive $a$, $a {\tt div} b$ is easy to compute, by computing $\frac ab$ in floating point then dropping what is after the decimal point.  Then $a\, {\tt mod}\, b=a - b(a\,{\tt div}\,b)$.


\end{description}

Now we prove the Euclidean Algorithm theorem, indicating the procedure for computing the greatest common divisor of two integers, and the extended Euclidean Algorithm theorem which shows that the gcd of two integers is an integer linear combination of those two integers.

\begin{description}

\item[Definition:]  Recall that for integers $a,d$, $d|a$ means that there is an integer $x$ such that $dx=a$.  We say that $d$ is a divisor of $x$.

\item[Definition:]  $d$ is a {\em common divisor\/} of $a$ and $b$ iff $d|a$ and $d|b$.

\item[Lemma:]  For any $a,b$ which are not both zero, there is a greatest common divisor of $a$ and $b$.

\item[Proof:]  If $a$ is not zero, every divisor of $a$ is $\leq |a|$.  Thus if we do not have $a=b=0$, we have an upper bound on common divisors of $a$ and $b$.

Any nonempty set $S$ of integers which has an upper bound $B$ has a greatest element:  this follows from the W.O.P:  the set $S'=\{B-x:x \in S\}$ is a set of nonnegative integers so has a smallest element $B-x$ and this $x$ will be the largest element of $S$.

It follows that the set of common divisors of $a$ and $b$ has a largest element, unless $a=b=0$, in which case all integers fall in the set of common divisors.

\item[Definition:]  Except in the case $a=b=0$, we define ${\tt gcd}(a,b)$, for integers $a,b$ as the greatest common divisor of$a$ and $b$.

\item[Lemma:]   ${\tt gcd}(a,b) = {\tt gcd}(|a|,|b|)$.  This justifies restricting our attention for the rest of this discussion to nonnegative $a,b$.

\item[Lemma:]  ${\tt gcd}(a,0)=a$ if $a>0$.  Obvious.

\item[Lemma:] ${\tt gcd}(a,b) = {\tt gcd}(b,a\,{\tt mod}\,b)$ if $a>b>0$.

\item[Proof of Lemma:]  Let $a>b>0$.  Let $q=a\,{\tt div}\,b$ and let $r=a\,{\tt mod}\,b$.

Since $r=a-bq$, any common divisor of $a,b$ is also a divisor of $r$ and so a common divisor of $b,r$.

Since $a=bq+r$, any common divisor of $b,r$ is also a divisor of $a$, and so a common divisor of $a,b$.

It follows that ${\tt gcd}(a,b)$ and ${\tt gcd}(b,a\,{\tt mod}\,b)$ are respectively the greatest element of one and the same set, so they are equal.

\item[Euclidean Algorithm:]

Let $a>b\geq 0$.  Define a finite sequence $E$ by $E_1=a, E_2=b$ and $E_{i+2}= E_i {\tt mod} E_{i+1}$ if this is nonzero, and otherwise is undefined.

It is straightforward to see that this is a strictly decreasing sequence of positive integers, and so it must end:  if it were infinite, its range would be a set of positive integers with no smallest elements.

Notice that it is straightforward by the previous Lemma and induction that ${\tt gcd}(E_i,E_{i+1}) = {\tt gcd}(E_1,E_2)$ for each $i$ for which these terms are defined.  If $E_{i+1}$ is the last term, it goes evenly into $E_i$ (that is how the sequence stops) and so $E_{i+1}={\tt gcd}(E_i,E_{i+1}) = {\tt gcd}(E_1,E_2)= {\tt gcd}(a,b)$.  So if one computes this sequence by repeated application of the mod operation, the sequence ends with the greatest common divisor of the two numbers with which you start.

\item[Extended Euclidean Algorithm:]  For any $a>b\geq0$ integers, there are integers $x,y$ such that $ax+by={\tt gcd}(a,b)$ [these integers $x,y$ are not unique, but the procedure we describe will give specifix $x,y$ that work).

\item[Proof:]  Let $a>b>0$.  Compute the sequence $E$ just as above.  

Notice that $E_{i+2}= E_i - (E_i {\tt div}E_{i+1})E_{i+1}$.

Compute two new sequences

$X_1=1, X_2=0, X_{i+2}= X_i - (E_i {\tt div}E_{i+1})X_{i+1}$.
$Y_1=0, Y_2=1, X_{i+2}= Y_i - (E_i {\tt div}E_{i+1})Y_{i+1}$.

Prove by induction that for each $i$ for which the terms of the sequences are defined, $E_i = aX_i + bY_i$:

This is obvious for $i=1,2$:

$aX_1+bY_1 = a1+b0 = a = E_1$.
$aX_2+bY_2 = a0+b1 = a = E_1$.

Suppose it works for $i$ and $i+1$:  then it works for $i+2$:

$aX_{i+2} +bY_{i+2} = a(X_i - (E_i {\tt div}E_{i+1})X_{i+1}) + b(Y_i - (E_i {\tt div}E_{i+1})Y_{i+1}) = (aX_i+bY_i) - (E_i {\tt div}E_{i+1})(aX_{i+1}+bY_{i+1}) =[{\tt ind-hyp}] E_i - (E_i {\tt div}E_{i+1})E_{i+1} = E_{i+2}$.

So if $E_i$ is the last term of the sequence, we have ${\tt gcd}(a,b)=E_i = aX_i +bY_i$.


\end{description}
We will spend time in class examining these formal proofs (I didn't give proofs in the first lecture, just built tables, but in fact I am saying basically the same thing).

I set up the main in-class example:  compute ${\tt gcd}(1024,137)$ (other knowledge tells us this will be 1) and find $x,y$ so that
1024$x$ + 137$y$ = ${\tt gcd}(1034,137)$, which we will find is 1.

$$\begin{array}{c|c|c|c}

&x & y& q \\

1024 & 1&0 &\\

 137 & 0 & 1 &\\

 65 & 1 & -7 & 7\\

7 &-2 & 15& 2 \\

 2 &19 & -142& 9 \\

 1 &-59 & 441 & 3
\end{array}$$

The first column is the sequence $E$, the second the sequence $X$, the third the sequence $Y$.  The fourth column contains the quotients used.

The final result is that ${\tt gcd}(1024,137)= 1 = (-59)(1024)+(441)(137)$.

I provide a spreadsheet you can use to do these calculations, but you do need to know how to do them by hand with the assistance of a calculator.

\section{Thursday, January 21, 2022}

My apologies to the class for the initial disruption.  I'm working with my technically minded son on getting a scheme for delivering Zoom sessions that will be useful to students out of class:  and yes, he got the web cam software to work in a flash.

I talked through some topics in Crisman related to the Tuesday lecture.

I spent some time discussing the more formal way I presented the extended Euclidean algorithm in the notes:  I said the same thing in the Tuesday lecture, but I did not define the sequences used in the formal presentation.

\begin{description}

\item[Theorem:]  If ${\tt gcd}(a,b)$ is defined, then ${\tt gcd}(a,b)$ is the smallest positive integer which can be written in the form $ax+by$ where $x$ and $y$ are integers (i.e, as an integer linear combination of $a$ and $b$.)

\item[Proof:]  By the extended Euclidean algorithm theorem, ${\tt gcd}(a,b)$ can be written in this form.

Now suppose that $w=ax+by$ for integers $x$ and $y$, and $w$ is positive.  ${\tt gcd}(a,b)|a$ and ${\tt gcd}(a,b)|b$, so ${\tt gcd}(a,b)|ax$ and ${\tt gcd}(a,b)|by$ so ${\tt gcd}(a,b)|ax+by=w$.  A positive multiple of any integer $z$ must be $\geq z$ (can you prove this?) so ${\tt gcd}(a,b)\leq w$, so ${\tt gcd}(a,b)$ is the smallest positive number which can be expressed in the form $ax+by$.

\item[Definition:]  The theorem that ${\tt gcd}(a,b)=ax+by$ for some integers $x,y$ is called the Bezout identity.  I learned this preparing for this class!

\item[Observation:]  The $x,y$ in the Bezout identity are not unique (though there is a specific one we find with the extended Euclidean algorithm (EEA)).  Suppose I want $ax+by = a(x+u) + b(y-v)$.  For this to be true, it is sufficient for $au=bv$ to hold.  $u=b$ and $v-a$ will work, giving
$a(x+b) + b(y-a)$ with the same value as $ax+by$.  For this to work, it is sufficient for $au=bv$ to be a common multiple of $a$ and $b$ and this may be less than the product $ab$:  $b$ is not necessarily the smallest number that can be added to $x$ here, nor $a$ the smallest number that can be subtracted from $y$.

\item[Definition:]  We say that $a$ and $b$ are {\em relatively prime\/} iff ${\tt gcd}(a,b)=1$.  This is a familiar concept, but it probably wasn't defined in this exact way when you first encountered it.

\item[Theorem (Euclid's lemma):]  If $p$ is a prime and $p|ab$ then either $p|a$ or $p|b$.

Study this proof.  You might be asked to write it.

Either $p|a$, in which case we are done, or $p\not|a$.  

So the rest of the argument is in the case $p\not|a$:  we need to show that in this case $p|b$.

Because $p$ is prime, ${\tt gcd}(p,a)=1$, so there are integers $x,y$ such that $px+ay=1$.

So $b=1b=(px+ay)b = pxb + ayb$.  $pxb$ is obviously divisible by $p$,  $ayb$ is divisible by $p$ because $ab$ is divisible by $p$.
Thus $b=pxb +ayb$ is divisible by $p$, which is what we needed to show.

\item[Prop 2.4.9 part 1:]  Suppose ${\tt gcd}(ab) = 1$.  If $a|c$ and $b|c$ then $ab|c$ (this theorem shows that in this case $ab$ is the least common multiple of $a$ and $b$).

\item[Proof:]  Suppose ${\tt gcd}(ab) = 1$.  Suppose $a|c$ and $b|c$.  Our goal is to show $ab|c$.

Since $a|c$ we have $x$ such that $ax=c$.  Since $b|c$ we have $y$ such that $by=c$.  Since ${\tt gcd}(ab) = 1$ we have $u$ and $v$ such that $au+bv=1$.  Now $c=1c=(au+bv)c=auc + bvc= auby + bvax = ab(uy+vx)$ which is divisible by $ab$ by inspection, so $ab|c$, which is what we needed to prove.

\item[Prop. 2.4.9 part 2:]  Suppose ${\tt gcd}(ab) = 1$.  Suppose $a|bc$.  Then $a|c$.

Since $a|bc$ we have $bc=xa$ for some $x$.  Since ${\tt gcd}(ab) = 1$ we have $ay+bz=1$ for some $y,z$.  Thus $c=1c = (ay+bz)c=ayc+bzc=ayc +bxa=a(cy+bx)$ which is divisible by $a$ by inspection, so $a|c$, which is what we need to show.

\end{description}

I was setting out to prove Prop 3.7.1 at the end of class.  These notes may contain a proof of that theorem before Tuesday's class:  keep an eye on them.

\newpage

\section{Homework 2, assigned 1/21/2022, due 1/27/2022}

In section 2.5 in Crisman, problems 3, 5 (you may use my spreadsheet, but say how you did it), 6, 7, 8 (with no more than a simple calculator;  of course I cannot stop you from using the spreadsheet to check, but you do need to know how to do this by hand for in-class tests), 10 (same remark as on 8), 15 (coprime is another word for ``relatively prime"), 17, 20.






\end{document}