\documentclass[12pt]{article}

\title{A brief and no doubt prejudiced history}

\author{Randall Holmes}

\begin{document}

\maketitle

In these notes I'll list people or schools of thought, with dates, that I think are interesting in the context of history and philosophy of modern logic.

\begin{description}

\item[Aristotle (384–322 BCE):]  A Greek philosopher, and the first to study formal logic.  Aristotelean logic is limited, a fact which was known to those who followed him in classical times
but less perceived in the Middle Ages and early modern times.

\item[Euclid (323–283 BCE):]  Inventor or at least popularizer of the axiomatic method.  Following Euclid, rigorous reasoning from first principles became traditional in geometry
(but not so much in arithmetic or later in algebra).  His reasoning in his famous work has gaps in it, which were recognized in classical times, but his work was treated more like holy writ in the Middle Ages and early modern times.

\item[the Stoic logicians (classical times):]  Apparently formulated the modern definition of implication (and some others).

\item[problems with the calculus (and the arithmetic side of things generally):]  The calculus invented by Newton and Leibniz was in practical terms very useful but its logical foundations were extremely dubious (a fact pointed out by Bishop Berkeley, accurately).  One of the roots of modern developments in logic was the attempt to put first the calculus, then arithmetic itself, on rigorous logical foundations, which had never been done by the ancient Greeks.  An example of the unsatisfactory statement of proof in arithmetic is that the uniqueness of factorization into primes, which Euclid certainly knew and supporting lemmas for which (and  partial versions of which) appear in the Elements, does not appear to have received a rigorous proof until the 19th century.

\item[problems with geometry:]  There was a general sense that the parallel postulate of Euclid was too complicated to be a primitive assumption.  There were attempts to prove it.
In the 19th century, Gauss, Lobachevsky, and Bolyai took the opposite path of developing geometries in which the parallel postulate was false and showing that they were coherent.
This caused deep disturbance, and helped motivate investigations into firming up the axiomatic method.

\item[George Boole (1815-1864):]  One of the first workers in modern logic.  He formulated an algebra of classes, which is a precursor of modern algebras of classes and propositional logic.
But his algebra of classes is not quite the same as the modern Boolean algebra of classes, and there were interesting developments not commonly now remembered between Boole's work and modern formulations of ``Boolean algebra".  I only realized that when reading for this class!

\item[Richard Dedekind (1831-1916):]  Dedekind proposed the modern rigorous definition of the real numbers (using sets of rational numbers).  He proposed one of the modern definitions of infinite sets.  Dedekind proposed an axiomatization of the natural numbers using just 0 and the successor, which was superseded by the simpler formulation of Peano.

\item[Guiseppe Peano (1858-1932):]  Peano is important for us mainly because of his formalization of arithmetic, which we have described in class.  In the course of this work, he made major progress in logical notation.  He did not develop the quantifier (Pierce and Frege did that in full generality) but he did use the notation $\phi(x) \supset_x \psi(x)$ for $$(\forall x:\phi(x) \rightarrow \psi(x)).$$

\item[categoricity results for mathematical systems:]  Dedekind and Peano showed that if one uses second order logic (if one allows quantification over predicates of numbers studied as well as the numbers studied) then one can give axiom sets which completely describe the natural numbers and the real numbers.  It turns out that this does not settle all questions about these theories, because it is provable that we cannot completely formalize methods of proof for second order logic as we can for first order logic:  this was not known until the work of G\"odel.

\item[Gottlob Frege (1848-1925):]  The founder of modern logic of propositions and quantifiers.  His logic was second order and included higher order assumptions which led to inconsistency,
but his logical work without Axiom V is a modern development of first or second-order logic, disguised by notation which is amazingly strange to our eyes.  His work was popularized with better notation by Russell in Principia Mathematica.  His main work was an attempt to found arithmetic on logic:  this formalization is basically successful, in spite of the inconsistency in his logic, and in fact his logic can be repaired in a way which preserves his work in arithmetic with minimal modifications.

\item[Zermelo set theory, 1908:]  We have looked at Zermelo's 1908 paper in which foundations of mathematics in set theory are presented in a way which has survived, though
Zermelo's work is somewhat odd in style for us because he is not aware of the possibility of defining the ordered pair in set theory.

\item[Principia Mathematica, Russell and Whitehead, 1911-13:]  In this vast, complicated, and lately little-read work Bertrand Russell and Alfred North Whitehead attempted to present
foundations of logic and mathematics in such a way as to avoid paradoxes discovered by Burali-Forti, Russell himself, and others.  The system of PM is a type theory as opposed to a set theory, the subject of a recent lecture.

The type theory of PM is much more complex that what is now often called the type theory of Russell because Russell did not know how to define ordered pairs as sets.

It has other oddities (culminating in the famous [in limited circles] Axiom of Reducibility), having to do with the issue of ``impredicative definitions", which I may try to talk about.

\item[Norbert Wiener, 1914:]  Wiener presented the first definition of an ordered pair usable in set theory and type theory.  The one we use now is simpler and was proposed by Kuratowski in 1920.

\item[Emil Post, 1921:]  A paper we have been reading:  propositional logic proved to be complete.  This paper pays serious attention to the difference between reasoning in a formal system and reasoning about what can be proved in that formal system, a theme not found in PM, but found in later work of Hilbert and G\"odel.

\item[David Hilbert and his program:]  David Hilbert is important in our outline of history for two reasons.  He gave a complete axiomatization of geometry, filling in the holes in Euclid.
He had a program whose aim was to avert problems with the foundations of mathematics by proving that commonly used axiomatic systems were consistent (reasoning about methods of proof in this systems rather than inside the systems themselves).  This led to great advances in the formalization of logic.

\item[G\"odel 1930, 1931:]  In 1930, G\"odel proved that first order logic is complete.  A proposition of a first-order theory is provable iff it is true in all models of that theory.  Further,
(not necessarily a result of G\"odel himself) any consistent theory with an infinite model has a countably infinite model (one which can be placed in one to one correspondence with the natural numbers).

In 1931, G\"odel proved that any first order theory strong enough to interpret ordinary arithmetic is incomplete:  there is some statement which is consistent with the theory while its negation is also consistent with the theory.  

These results establish that there cannot be a complete proof technique for second-order logic, since there is a formulation of second-order logic which completely determines the natural numbers.  If there were a complete formalization of second order logic, the theory of the natural numbers using this formulation would be within the scope of G\"odel's incompleteness theorem.

A corollary of G\"odel's incompleteness theorem turns out to be that no theory strong enough to interpret ordinary arithmetic can prove its own consistency.  So Hilbert's program was impossible.


\end{description}


\end{document}