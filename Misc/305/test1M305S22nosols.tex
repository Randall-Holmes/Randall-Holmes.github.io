\documentclass[12pt]{article}

\usepackage{amssymb}

\title{Test I Math 305 Spring 2022 (Solutions omitted, provided for Spring 2025 Test I review)}

\author{Dr Holmes}

\date{February 7, 2022}



\begin{document}

\maketitle

The exam begins officially at 12 noon and ends at 1:15 pm.  What actually happens at 1:15 is that I give a five minute warning.

There are 10 numbered questions on this exam.  Complete eight of them.  If you do more than eight of them, you will receive credit for the best 8 of them, and some extra credit is possible if the dropped questions have significant value.

You may use a scientific calculator without graphing capability.  You are strongly encouraged to use my table format for computing the extended Euclidean algorithm.

You may bring a single sheet of standard sized notebook paper to the exam, with whatever you like written or printed on it.

You may not use anything on the exam but your test paper, your writing instrument, your non-graphing calculator and your single sheet of notebook paper.

[I have taught this class in its present format exactly once:  this is the first test that I gave on that earlier occasion.
Our coverage hasn't been precisely the same, and over the course of the week I may have remarks about what other sorts of things I might ask about.  In general, anything I asked about in one of the homework sets is fair game, but I'll look through the materials and be more specific about other possible things I might ask about.  I may post solutions sets for the homework assignments this term, if I have them in presentable form.

The version with solutions will be posted later, probably early next week.]

\newpage

\begin{enumerate}

\item  

Prove using the well-ordering principle and standard algebra, including basic properties of order, that
there is no integer between 0 and 1.



\newpage

\item

Prove by mathematical induction that $\sum_{i=1}^n (2i-1) = n^2$.


\newpage

\item

Find ${\tt gcd}(37,25)$ and find integers $x$ and $y$ such that $37x+25y = {\tt gcd}(37,25)$. {\bf using the extended Euclidean algorithm}.  No substitutes accepted.

Please make it clear that you know what ${\tt gcd}(37,25)$  is and what $x$ and $y$ are.

You are {\bf required} to present a calculation using a general procedure recognizable as the extended Euclidean algorithm to find these answers:  preferably, use the table layout I use in class and in the notes.  Of course you know what the gcd is and you might be able to find $x$ and $y$ by trial and error.  This will not be good for much if any credit.


\newpage

\item  Prove Euclid's lemma using the Bezout identity.



\newpage

\item   Write three distinct primitive Pythagorean triples.


\newpage

\item  Explain (it is a fact about a suitable system of modular arithmetic) why any integer of the form $4k+3$ must have a prime factor of the form $4k+3$.  Be sure that your explanation mentions facts about primes as well as facts about modular arithmetic.


\newpage

\item   Build the multiplication table of mod 7 arithmetic.  Make a table of the multiplicative inverses of the nonzero residues in mod 7 arithmetic.



\newpage

\item   Compute $17^{375} {\tt mod} 100$ by the method of repeated squaring (this really means by the method of repeated squaring;  other methods can be used as a check, but if you know another one you do need to do the official calculation).  [as I post this, we haven't talked about this topic yet, but it might happen]



\newpage

\item Do both parts.  There is a connection. [we haven't talked about this yet, but we will in time for test coverage]
\begin{enumerate}

\item  Compute the multiplicative inverse of 25 in mod 137 arithmetic.



\item  Find all solutions to $$100x \equiv_{548} 16.$$  List the residues mod 548 which are solutions.



\end{enumerate}

\newpage

\item  Solve the system of simultaneous equations [I don't think we are likely to get to this before the curtain for test material]

$$x \equiv_{17} 4$$

$$x \equiv_{25}  11$$

State the smallest solution.  State another solution.




\newpage

\end{enumerate}


\end{document}