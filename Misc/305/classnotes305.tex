\documentclass[12pt]{article}

\usepackage{amssymb}

\title{Class Lecture Notes for Math 305, Spring 2022}

\author{Dr Holmes}

\begin{document}

\maketitle

These are notes on what I say in class in Math 305.  They were originally written in the Spring 2022 semester, and I will update them as we go through the Spring 2025 semester.  I reserve the right to make substantial changes of plan, but the existence of the notes does encourage me to proceed in more or less the way they indicate.  You will be able to tell if a section has been updated by the year mentioned in its title.

\tableofcontents


\section{Tuesday, January 11, 2022}

Administrative preliminaries.


I discussed the definitions of ${\mathbb Z}$, ${\mathbb N}$, ${\mathbb Z}^+$: 

${\mathbb Z} = \{\ldots,-2,-1,0,1,2,\ldots\}$, the set of integers;

${\mathbb N}= \{0,1,2,\ldots\}$, the set of natural numbers (there is no general agreement in mathematical literature as to whether 0 is a natural number, but this book includes it), or non-negative integers;

${\mathbb Z}^+ = \{1,2,3,\ldots\}$, the set of positive integers.  In all of these, the use of dots is really cheating:  giving a rigorous definition of these sets is rather difficult, and we appeal instead to your pre-formal understanding of these concepts.

I stated a set of axioms for the integers which I will include here (based on the axioms in the Math 287 book with two alternative approaches to order).

We begin with a set of purely algebraic axioms.  Our variables range over the set $\mathbb Z$ of integers;  we assume special integers 0 and 1 and primitive operations or addition (+) multiplication ($\cdot$) and additive inverse ($-$, used as a prefix unary operator).

\begin{description}

\item[commutative laws:]  For any $x,y \in {\mathbb Z}$, $x+y=y+x$ and $x \cdot y = y \cdot x$.

\item[associative laws:]  For any $x,y,z \in {\mathbb Z}$, $(x+y)+z = x+(y+z)$ and $(x \cdot y)\cdot z = x \cdot(y\cdot z)$.  I should add that we are only allowed to write things like $x+y+z$ or $x \cdot y \cdot z$ because we know these operations are associative.  In proofs in section 1.2 you should write parentheses, and explicitly use the associative laws to move them.

You {\em may\/} use standard order of operations and read $x \cdot y + z$ as meaning $(x \cdot y) + z$ wihtout writing out the parentheses (multiplication binds more tightly than addition, unary minus binds more tightly than either).

\item[distributive law:]  For any $x,y,z \in {\mathbb Z}$, $x\cdot(y+z) = x \cdot y + x \cdot z$.

\item[identity laws:]  For any $x \in {\mathbb Z}$, $x+0=x$ and $x \cdot 1 = x$.  $0 \neq 1$.

\item[multiplicative cancellation:]  For any $x,y,z \in {\mathbb Z}$, if $x \neq 0$ and $x\cdot y = x\cdot z$, then $y=z$.
This amounts to the ability to divide both sides of an equation by the same thing, but we do not have a full division operation in the integers as we do in the rationals or reals.

\end{description}

This is not a full axiomatization of the integers.  Of course, systems like the rationals and the reals which extend the integers
satisfy these axioms, but there are also systems (even ones familiar to you) which satisfy these axioms and are quite different from the integers.  Arithmetic mod $p$ where $p$ is prime satisfies these axioms, and the domain of ``numbers" in mod $p$ arithmetic is finite (the remainders $0,1,\ldots p-1$ mod $p$).

Additional axioms appropriate for the integers which rule out the system described being modular arithmetic are axioms of order.  We present these (just for fun) in two different ways.

We can axiomatize order by introducing the set of positive integers ${\bf Z}^+$ as a primitive notion, providing some of its properties as axioms, and using it to define order relations.

\begin{enumerate}

\item $0 \not\in {\mathbb Z}^+$.

\item  For each $m \in {\mathbb Z}$ with $m \neq 0$ either $m \in {\mathbb Z}^+$ or  $-m \in {\mathbb Z}^+$.

\item For each $m,n \in {\mathbb Z}^+$, we have $m+n \in {\mathbb Z}^+$ and $m\cdot n \in {\mathbb Z}^+$

\item Define $m<n$ as $n+(-m) \in {\mathbb Z}^+$.

\end{enumerate}

This is a very elegant set of axioms, and it should be straightforward for you to see that they are true in the familiar system of integers, but it may be less obvious that they are enough.  This might be a homework exercise.

Here is a more familiar set of axioms for order.  They do follow as consequences of the algebraic and positive integer axioms if we define $<$ as above, but for this approach we ``forget" about ${\bf Z}^+$ and take $<$ as a primitive relation (and we define ${\mathbb Z}^+$ in terms of $<$).

\begin{description}

\item[transitivity:]  For any $m,n,p \in {\mathbb Z}$,  if $m<n$ and $n<p$ then $m<p$.

\item[trichotomy:]  For any $m,n \in {\mathbb Z}$, exactly one of $m<n,m=n,n<m$ is true.

\item[additive monotonicity:]  For any $m,n,p \in {\mathbb Z}$, if $m<n$ then $m+p < n+p$.

\item[multiplicative monotonicity:]  For any $m,n,p \in {\mathbb Z}$, if $p  > 0$ and $m<n$, then $m\cdot p <n\cdot p$.  Our axioms are enough to show that the right things happen if $p$ is zero or negative (that might be a homework exercise).

\item  [definition of positive integers:]  We define ${\mathbb Z}^+$ as $\{x \in {\mathbb Z}:0<x\}$.

\end{description}

I stated the Well-Ordering Principle and proved two sample theorems, ``each positive integer is either even or odd", and ``there is no integer strictly between 0 and 1".

If $S$ is a set of integers, $x$ is a smallest element of $S$ iff $x \in S$ and $(\forall y \in S:x \leq y)$.  You could try proving that a nonempty set with a smallest element has just one smallest element.

\begin{description}

\item[Well-Ordering Principle:]  Any nonempty set $S$ of positive integers has a smallest element.

\end{description}

I proved a couple of sample theorems using the Well-Ordering Principle in class.  Proofs using this principle are usually indirect (proofs by contradiction);  pay attention to the logical structure of what I say.

\begin{description}

\item[Definition:]  An integer $m$ is even iff there is an integer $x$ such that $m=2\cdot x$.  

An integer $m$ is odd iff there is an integer $x$ such that $m=2\cdot x +1$.

\item[Theorem:]  Each positive integer is either even or odd.

\item[Proof:]  Suppose otherwise, so there are integers which are neither even nor odd.  Let $S$ be the set of all integers which are neither even nor odd.  By our assumption, it is nonempty, so by the Well-Ordering Principle it has a smallest element $w$.  This number $w$ will be the smallest integer which is neither even nor odd.

The integer $w$ is not 1, because $1 = 2\cdot 0 +1$ is odd.

So $w-1$ is a positive integer, and because it is less than $w$ it must be either even or odd.

If $w-1=2\cdot x$ is even, then $w=2\cdot x +1$ is odd.

If $w-1=2\cdot x +1$ is odd, then $w = 2 \cdot x +2 = 2 \cdot(x+1)$ is even.

In either case, we get that $w$ is either odd or even, which is a contradiction, so there can be no such $w$ and the theorem must be true.

\item[Observation:]  At a crucial point in the argument above, I cheated (or at least I appealed to your intuition), and the fact is used is important and should be proved.  How do I know that if $w\neq 1$ is a positive integer that $w-1$ is a positive integer?
If we have $w>1$, we do have $w-1>0$.  We need to rule out the possibility that $0<w<1$ (which, since we know what the integers are, is hard to even take into account).

\item[Theorem:]  There is no integer $x$ such that $0<x<1$.

\item[Proof:]  If there is such an integer than the set $S = \{x \in {\mathbb Z}:0<x<1\}$ is nonempty and so by the Well-Ordering Principle has a smallest element $w$.

So we have $0<w<1$.  By multiplicative monotonicity (because $w>0$) we have $0<w^2<w$ and of course
we then have $0<w^2<w<1$.  Using transitivity we see that $0<w^2<1$ and $w^2<w$, so $w^2$ belongs to the set
$S$ but is smaller than $w$, which is a contradiction.


\end{description}

\newpage

\section{Homework 1}

This is being assigned on January 13 and is due January 20.

\begin{enumerate}

\item  Prove all parts of proposition 1.2.8 in Crisman on properties of divisibility (this is on page 4 of Crisman).

\item  Prove by mathematical induction that for every $n \in {\mathbb Z}^+$, $3|(n^3+5n)$.

\item  Prove by mathematial induction that the sum of the first $n$ odd numbers is $n^2$.  Make appropriate use of summation notation.

\item Use the first set of order axioms in these notes (in which the set of positive integers is primitive) along with the algebra axioms to prove at least two of the axioms in the second set of order axioms, in which the less-than relation is primitive (you will need to use the definition of the less-than relation given with the first set of axioms).  Your algebra may be somewhat informal:  your use of order axioms should be careful and explicit.  Extra credit will be rewarded for proving more of the order axioms in the second set.

\end{enumerate}

\newpage

\section{Thursday, January 13, 2022}

We begin with the principle of mathematical induction.

Mathematical induction can be presented as a proof strategy.

\begin{description}

\item[Goal:]  Prove $(\forall n \in {\mathbb Z}^+:P(n))$

\item[basis step:]  Prove $P(1)$

\item[induction step:]

\begin{description}

\item

\item[induction hypothesis:]  


Choose a natural number $k$ arbitrarily.  \newline Assume $P(k)$.

\item[induction goal:]  Prove $P(k+1)$ (under the assumption that $P(k)$ is true).

\end{description}

If you succeed in proving the induction goal, assuming the induction hypothesis, you have proved $(\forall k \in {\mathbb Z}^+:P(k) \rightarrow P(k+1))$.



\end{description}
If you complete both steps, you can conclude $(\forall n \in {\mathbb Z}^+:P(n))$
 by mathematical induction.

One can prove theorems by mathematical induction on ${\mathbb N}$ instead of ${\mathbb Z}^+$:  in this case the basis step is to prove $P(0)$.

We give an example (which also illustrates nice tools for working with summation notation).

\begin{description}
\item[Theorem:]  The sum of the first $n$ squares of positive integers is $\frac{n(n+1)(2n+1)}6$, that is, $$\sum_{i=1}^ni^2 =\frac{n(n+1)(2n+1)}6$$

\item[Proof:]

We prove this by mathematical induction on $n$.

\begin{description}

\item[basis step:]  Prove $\sum_{i=1}^1i^2 = \frac{(1)(1+1)(2\cdot 1+1)}6$

$\sum_{i=1}^1i^2 = 1^2=1= \frac{(1)(1+1)(2\cdot 1+1)}6$ (check)

\item[induction step:]  Choose a positive integer $k$ arbitarily.\newline
Assume (ind hyp) that $\sum_{i=1}^ki^2 = \frac{(k)(k+1)(2k+1)}6$

The induction goal is to prove  $\sum_{i=1}^{k+1}i^2 = \frac{(k+1)(k+2)(2k+3)}6$

Notice that in formulating the induction goal we allowed ourselves to do a little obvious algebra after replacing $k$ with $k+1$.

$\sum_{i=1}^{k+1}i^2  = [\sum_{i=1}^{k}i^2 ]+ (k+1)^2$ (pulling out the last term) $= \frac{(k)(k+1)(2k+1)}6+(k+1)^2$ (by ind hyp: {\bf ALWAYS highlight the use of the inductive hypothesis in any proof by induction}) $= \frac{(k+1)(k)(2k+1) + 6(k+1)(k+1)}6 = \frac{(k+1)((2k^2+k)+(6k+6))}6 = \frac {(k+1)(2k^2+7k+6)}6 = \frac {(k+1)(k+2)(2k+3))}6$ (check)                                                                                                                                                                                                                                                                                                                                                                                                                                                                                                                                                                    


\end{description}

The proof by induction is complete.

\end{description}

I was impressed with success in Math 189 last term at teaching this general approach to proofs of statement involving summations, avoiding dots, which can cause various confusions.

Next, I lectured the equivalence of math induction to the well-ordering principle.

First assume the well-ordering principle and show that math induction follows:

\begin{description}

\item[Given:]

\begin{description}
\item
\item[1.]  $P(1)$

\item[2.] $(\forall k \in {\mathbb Z}^+:P(k) \rightarrow P(k+1))$

\item[3.] the well-ordering principle

\end{description}

\item[Show:]  $(\forall n \in {\mathbb Z}^+:P(n))$

\item[note:]  Convince yourself that if we complete this plan we really have shown that WOP does the work of math induction.

\item[Proof:]  Suppose for the sake of a contradiction that $\neg (\forall n \in {\mathbb Z}^+:P(n))$, so there is some
$x$ a positive integer such that $\neg P(x)$.  Let $S$ be the set of all $x$ such that $\neg P(x)$, which we see is nonempty and so by WOP has a smallest element which we will call $w$.

$w$ is not 1, because we have assumed $P(1)$.  Thus $w>1$ (here we are using the result proved above using WOP that there is no integer strictly between 0 and 1).  Thus $w-1>0$ is a positive integer.  Thus we have $P(w-1)$, because $w$ is the smallest positive integer such that $\neg P(w)$.
But plugging $w-1$ in for $k$ in $(\forall k \in {\mathbb Z}^+:P(k) \rightarrow P(k+1))$ gives $P(w-1) \rightarrow P(w)$, and so since we have
$P(w-1)$ and $P(w-1) \rightarrow P(w)$ we have (by the rule of modus ponens) $P(w)$, but this is a contradiction.

Thus our assumption that $\neg (\forall n \in {\mathbb Z}^+:P(n))$ is incorrect, and we have $(\forall n \in {\mathbb Z}^+:P(n))$


\end{description}

This completes the proof that the well-ordering principle implies the principle of mathematical induction.

Now we argue that the principle of mathematical induction implies the well-ordering principle.

\begin{description}

\item[Given:]

\begin{description}
\item
\item[1.]  $S$ is a nonempty set of positive integers

\item[2.]  the principle of math induction

\end{description}

\item[Show:]  $S$ has a smallest element.

\item[note:]  Convince yourself that this proof plan really does show that the principle of math induction does the work of the well ordering principle, if we can carry it out.

\item[Proof:]  Assume for the sake of a contradiction that $S$ has no smallest element.  We will prove by induction that $S$ is empty, completing the desired contradiction.

We do not prove by induction that for every $n$, $n \not\in S$:  we prove the stronger statement that for every $n$, $(\forall m \in {\mathbb Z}^+: m \leq n \rightarrow m \not\in S)$:   not only is $n$ not in $S$, but no smaller positive integer is in $S$.
We will describe the strategy of strong induction of which this is an example in the last section of the notes for today.

The basis step for the induction is to show $(\forall m \in {\mathbb Z}^+:m \leq 1 \rightarrow m \not\in S)$:  the only positive integer less than or equal to 1 is 1 itself, so all we have
to show is $1 \not\in S$, and this follows from the assumption that $S$ has no smallest element:  if it contained 1, 1 would be its smallest element.

Choose an arbitrary positive integer $k$.  Assume $(\forall m \in {\mathbb Z}^+: m \leq k \rightarrow m \not\in S)$ as our induction hypothesis.  Our induction goal is to show that $(\forall m \in {\mathbb Z}^+: m \leq k+1 \rightarrow m \not\in S)$  If $m$ is an integer $\leq k+1$, it is either
less than $k$ or equal to $k$, in which cases the induction hypothesis tells us that $m \not\in S$, or (final case to be checked) $m>k$.
Now, because there is no integer strictly between 0 and 1, there is also no integer strictly between $k$ and $k+1$ (we could subtract $k$ from it to get between 0 and 1).  Thus, since $m>k$ and $m\leq k+1$, $m$ is simply $k+1$.  We can conclude $k+1 \not\in S$, because if it were in $S$ it would be the smallest element of $S$, since we have shown that nothing less than $k+1$ can belong to $S$.

So we have shown by induction that for every positive integer $n$, $(\forall m \in {\mathbb Z}^+: m \leq n \rightarrow m \not\in S)$, but this immediately implies that for every positive integer $n$, $n \not\in S$, so $S$ is empty, which is a contradiction.

This means that our assumption that $S$ has no smallest element must be false:  it follows from the statements given that $S$ has a smallest element.


\end{description}

This completes the proof that the well-ordering principle follows from the principle of mathematical induction. 

The final topic of this lecture was the method of strong induction.  This is a version of mathematical induction with a stronger hypothesis which is sometimes useful.  We will state it and prove a theorem as an example.  We state but do not prove (it might be fairly easy to see from the proof of equivalence of ordinary math induction and the well-ordering principle) that strong induction is in fact precisely equivalent in strength to ordinary induction.
But it is sometimes much more convenient.

We state strong induction as a strategy of proof.

\begin{description}
\item[Goal:]  Prove $(\forall n \in {\mathbb Z}^+:P(n))$

\item[basis step:]  Prove $P(1)$

\item[induction step:]
\begin{description}

\item
\item[induction hypothesis:]  Let $k$ be an arbitrarily chosen positive integer.  Assume $(\forall m \in {\mathbb Z}^+:m \leq k \rightarrow P(m))$:  instead of assuming just $P(k)$ we assume $P(1), P(2),\ldots,P(k)$.  This is a stronger hypothesis, and this is why we call this method strong induction.

\item[induction goal:]  Prove $P(k+1)$ under the assumption of the inductive hypothesis.

\end{description}

If you succeed in completing the basis and induction steps, you have proved $(\forall n \in {\mathbb Z}^+:P(n))$ by strong induction.

\end{description}

Here is an important example.  (I am not for the moment trying to expound this in terms of product notation as I suggested in class;  I might do it later, but my brain is tired after writing these notes).

\begin{description}

\item[Theorem:]  Each integer $\geq 2$ is a prime or a finite product of primes.

\item[Proof:]

we prove this by strong induction.

The basis step requires us to prove that 2 is a prime or a finite product of primes.  2 is a prime (check).

We choose an arbitrary positive integer $k\geq 2$.  The induction hypothesis will be that for every positive integer $m \leq k$, $m$ is a prime or a product of primes.

The induction goal is to prove that $k+1$ is a prime or a finite product of primes.

By the law of excluded middle either $k+1$ is a prime (in which case we are done, as it is then a prime or a finite product of primes)
or it is composite, in which case there are $a,b$ such that $2 \leq a,b \leq k$ and $ab=k+1$.  Now by inductive hypothesis, each of $a,b$ is either a prime or a finite product of primes, so $ab$ is a finite product of primes.  And this completes the proof of the theorem by strong induction.

\end{description}

\section{Tuesday, January 19, 2022}

Today I talked about the Division Algorithm and the Euclidean Algorithm (plain and extended).  I talked about this off the top of my head, and I owe you a discussion of what this material looks like in Crisman and how it might differ from what I say.

\begin{description}

\item[Theorem (division algorithm):]  For each $a \in {\mathbb Z}$ and $b \in {\mathbb Z}^+$, there are unique determined integers $q$ and $r$ such that $a=bq+r$ and $0 \leq r <b$.

Of course ``$q$" and ``$r$" are hints:  we give these variables these names because they suggest {\em quotient\/} and {\bf remainder\/}.

We prove the theorem using the Well-Ordering Theorem (and it is a positive result, we are not arguing by contradiction!)

\item[Proof:]

Define $S$ as the set $\{a-bq:q \in {\mathbb Z} \wedge a-bq\geq 0\}$.  This is the set of candidates for the remainder $r$, as it were.

It is a set of nonnegative integers, so if it is nonempty it has a least element.

If $a\geq 0$, let $q=0$ and we see that $a-bq=a\geq 0$, so $a \in S$ and $S$ is nonempty.

If $a <0$ let $q=a$ and we see that $a-bq = a-ba=a(1-b)$.  $a$ is negative and $1-b$ is nonpositive (since $b$ is positive), so $a(1-b)$ is nonnegative, and so belongs to $S$, so $S$ is nonempty.

Define $r$ as the smallest element of $S$.  There is a unique $q$ such that $r=a-bq$, so $a=bq+r$.

All that remains is to show $0 \leq r<b$.  We know that $r \geq 0$ because $r \in S$.  Notice that $a-b(q+1)$ must be negative, because
if it were nonnegative it would be an element of $S$ smaller than $r=a-bq$.

$a-b(q+1) = a-bq-b=r-b$ so we have $r-b<0$ so $r<b$ completing the proof.

% typo point for Bridger Lenz

We still need to prove that $q$ and $r$ are uniquely determined.  Suppose that $a=bq-r = bQ-R$ and $0\leq r < R <b$.

Observe that $R-r = b(Q-q)$.  Now $R-r<b$, and the only way for $b(Q-q)<b$ to be true is $Q-q=0$, so $Q=q$.  Then $r=a-bq = a-bQ = R$.

\item[Definition:]  For $a \in {\mathbb Z}$ and $b \in {\mathbb Z}^+$ define $a {\tt div} b$ and $a \,{\tt mod}\, B$ as the unique $q$ and $r$ whose existence is proved by the division algorithm.

\item[Observations:]  Be careful with negative values of $a$.  Notice that while $100 \,{\tt div}\, 3 = 33$ and $100\, \,{\tt mod}\,\, 3 = 1$, it turns out that 
$100\, {\tt div}\, 3 = -34$ and $100 \,\,{\tt mod}\, \,3 = 2$.

It might not be obvious that we can compute \,{\tt mod}\, with a simple calculator.  But we can.  For positive $a$, $a {\tt div} b$ is easy to compute, by computing $\frac ab$ in floating point then dropping what is after the decimal point.  Then $a\, \,{\tt mod}\,\, b=a - b(a\,{\tt div}\,b)$.


\end{description}

Now we prove the Euclidean Algorithm theorem, indicating the procedure for computing the greatest common divisor of two integers, and the extended Euclidean Algorithm theorem which shows that the gcd of two integers is an integer linear combination of those two integers.

\begin{description}

\item[Definition:]  Recall that for integers $a,d$, $d|a$ means that there is an integer $x$ such that $dx=a$.  We say that $d$ is a divisor of $x$.

\item[Definition:]  $d$ is a {\em common divisor\/} of $a$ and $b$ iff $d|a$ and $d|b$.

\item[Lemma:]  For any $a,b$ which are not both zero, there is a greatest common divisor of $a$ and $b$.

\item[Proof:]  If $a$ is not zero, every divisor of $a$ is $\leq |a|$.  Thus if we do not have $a=b=0$, we have an upper bound on common divisors of $a$ and $b$.

Any nonempty set $S$ of integers which has an upper bound $B$ has a greatest element:  this follows from the W.O.P:  the set $S'=\{B-x:x \in S\}$ is a set of nonnegative integers so has a smallest element $B-x$ and this $x$ will be the largest element of $S$.

It follows that the set of common divisors of $a$ and $b$ has a largest element, unless $a=b=0$, in which case all integers fall in the set of common divisors.

\item[Definition:]  Except in the case $a=b=0$, we define ${\tt gcd}(a,b)$, for integers $a,b$ as the greatest common divisor of$a$ and $b$.

\item[Lemma:]   ${\tt gcd}(a,b) = {\tt gcd}(|a|,|b|)$.  This justifies restricting our attention for the rest of this discussion to nonnegative $a,b$.

\item[Lemma:]  ${\tt gcd}(a,0)=a$ if $a>0$.  Obvious.

\item[Lemma:] ${\tt gcd}(a,b) = {\tt gcd}(b,a\,\,{\tt mod}\,\,b)$ if $a>b>0$.

\item[Proof of Lemma:]  Let $a>b>0$.  Let $q=a\,{\tt div}\,b$ and let $r=a\,\,{\tt mod}\,\,b$.

Since $r=a-bq$, any common divisor of $a,b$ is also a divisor of $r$ and so a common divisor of $b,r$.

Since $a=bq+r$, any common divisor of $b,r$ is also a divisor of $a$, and so a common divisor of $a,b$.

It follows that ${\tt gcd}(a,b)$ and ${\tt gcd}(b,a\,\,{\tt mod}\,\,b)$ are respectively the greatest element of one and the same set, so they are equal.

\item[Euclidean Algorithm:]

Let $a>b\geq 0$.  Define a finite sequence $E$ by $E_1=a, E_2=b$ and $E_{i+2}= E_i \,{\tt mod}\, E_{i+1}$ if this is nonzero, and otherwise is undefined.

It is straightforward to see that this is a strictly decreasing sequence of positive integers, and so it must end:  if it were infinite, its range would be a set of positive integers with no smallest elements.

Notice that it is straightforward by the previous Lemma and induction that ${\tt gcd}(E_i,E_{i+1}) = {\tt gcd}(E_1,E_2)$ for each $i$ for which these terms are defined.  If $E_{i+1}$ is the last term, it goes evenly into $E_i$ (that is how the sequence stops) and so $E_{i+1}={\tt gcd}(E_i,E_{i+1}) = {\tt gcd}(E_1,E_2)= {\tt gcd}(a,b)$.  So if one computes this sequence by repeated application of the mod operation, the sequence ends with the greatest common divisor of the two numbers with which you start.

\item[Extended Euclidean Algorithm:]  For any $a>b\geq0$ integers, there are integers $x,y$ such that $ax+by={\tt gcd}(a,b)$ [these integers $x,y$ are not unique, but the procedure we describe will give specifix $x,y$ that work).

\item[Proof:]  Let $a>b>0$.  Compute the sequence $E$ just as above.  

Notice that $E_{i+2}= E_i - (E_i {\tt div}E_{i+1})E_{i+1}$.

Compute two new sequences

$X_1=1, X_2=0, X_{i+2}= X_i - (E_i {\tt div}E_{i+1})X_{i+1}$.
$Y_1=0, Y_2=1, X_{i+2}= Y_i - (E_i {\tt div}E_{i+1})Y_{i+1}$.

Prove by induction that for each $i$ for which the terms of the sequences are defined, $E_i = aX_i + bY_i$:

This is obvious for $i=1,2$:

$aX_1+bY_1 = a1+b0 = a = E_1$.
$aX_2+bY_2 = a0+b1 = a = E_1$.

Suppose it works for $i$ and $i+1$:  then it works for $i+2$:

$aX_{i+2} +bY_{i+2} = a(X_i - (E_i {\tt div}E_{i+1})X_{i+1}) + b(Y_i - (E_i {\tt div}E_{i+1})Y_{i+1}) = (aX_i+bY_i) - (E_i {\tt div}E_{i+1})(aX_{i+1}+bY_{i+1}) =[{\tt ind-hyp}] E_i - (E_i {\tt div}E_{i+1})E_{i+1} = E_{i+2}$.

So if $E_i$ is the last term of the sequence, we have ${\tt gcd}(a,b)=E_i = aX_i +bY_i$.


\end{description}
We will spend time in class examining these formal proofs (I didn't give proofs in the first lecture, just built tables, but in fact I am saying basically the same thing).

I set up the main in-class example:  compute ${\tt gcd}(1024,137)$ (other knowledge tells us this will be 1) and find $x,y$ so that
1024$x$ + 137$y$ = ${\tt gcd}(1034,137)$, which we will find is 1.

$$\begin{array}{c|c|c|c}

&x & y& q \\

1024 & 1&0 &\\

 137 & 0 & 1 &\\

 65 & 1 & -7 & 7\\

7 &-2 & 15& 2 \\

 2 &19 & -142& 9 \\

 1 &-59 & 441 & 3
\end{array}$$

The first column is the sequence $E$, the second the sequence $X$, the third the sequence $Y$.  The fourth column contains the quotients used.

The final result is that ${\tt gcd}(1024,137)= 1 = (-59)(1024)+(441)(137)$.

I provide a spreadsheet you can use to do these calculations, but you do need to know how to do them by hand with the assistance of a calculator.

\section{Thursday, January 21, 2022}

My apologies to the class for the initial disruption.  I'm working with my technically minded son on getting a scheme for delivering Zoom sessions that will be useful to students out of class:  and yes, he got the web cam software to work in a flash.

I talked through some topics in Crisman related to the Tuesday lecture.

I spent some time discussing the more formal way I presented the extended Euclidean algorithm in the notes:  I said the same thing in the Tuesday lecture, but I did not define the sequences used in the formal presentation.

\begin{description}

\item[Theorem:]  If ${\tt gcd}(a,b)$ is defined, then ${\tt gcd}(a,b)$ is the smallest positive integer which can be written in the form $ax+by$ where $x$ and $y$ are integers (i.e, as an integer linear combination of $a$ and $b$.)

\item[Proof:]  By the extended Euclidean algorithm theorem, ${\tt gcd}(a,b)$ can be written in this form.

Now suppose that $w=ax+by$ for integers $x$ and $y$, and $w$ is positive.  ${\tt gcd}(a,b)|a$ and ${\tt gcd}(a,b)|b$, so ${\tt gcd}(a,b)|ax$ and ${\tt gcd}(a,b)|by$ so ${\tt gcd}(a,b)|ax+by=w$.  A positive multiple of any integer $z$ must be $\geq z$ (can you prove this?) so ${\tt gcd}(a,b)\leq w$, so ${\tt gcd}(a,b)$ is the smallest positive number which can be expressed in the form $ax+by$.

\item[Definition:]  The theorem that ${\tt gcd}(a,b)=ax+by$ for some integers $x,y$ is called the Bezout identity.  I learned this preparing for this class!

\item[Observation:]  The $x,y$ in the Bezout identity are not unique (though there is a specific one we find with the extended Euclidean algorithm (EEA)).  Suppose I want $ax+by = a(x+u) + b(y-v)$.  For this to be true, it is sufficient for $au=bv$ to hold.  $u=b$ and $v-a$ will work, giving
$a(x+b) + b(y-a)$ with the same value as $ax+by$.  For this to work, it is sufficient for $au=bv$ to be a common multiple of $a$ and $b$ and this may be less than the product $ab$:  $b$ is not necessarily the smallest number that can be added to $x$ here, nor $a$ the smallest number that can be subtracted from $y$.

\item[Definition:]  We say that $a$ and $b$ are {\em relatively prime\/} iff ${\tt gcd}(a,b)=1$.  This is a familiar concept, but it probably wasn't defined in this exact way when you first encountered it.

\item[Theorem (Euclid's lemma):]  If $p$ is a prime and $p|ab$ then either $p|a$ or $p|b$.

Study this proof.  You might be asked to write it.

Either $p|a$, in which case we are done, or $p\not|a$.  

So the rest of the argument is in the case $p\not|a$:  we need to show that in this case $p|b$.

Because $p$ is prime, ${\tt gcd}(p,a)=1$, so there are integers $x,y$ such that $px+ay=1$.

So $b=1b=(px+ay)b = pxb + ayb$.  $pxb$ is obviously divisible by $p$,  $ayb$ is divisible by $p$ because $ab$ is divisible by $p$.
Thus $b=pxb +ayb$ is divisible by $p$, which is what we needed to show.

\item[Prop 2.4.9 part 1:]  Suppose ${\tt gcd}(ab) = 1$.  If $a|c$ and $b|c$ then $ab|c$ (this theorem shows that in this case $ab$ is the least common multiple of $a$ and $b$).

\item[Proof:]  Suppose ${\tt gcd}(ab) = 1$.  Suppose $a|c$ and $b|c$.  Our goal is to show $ab|c$.

Since $a|c$ we have $x$ such that $ax=c$.  Since $b|c$ we have $y$ such that $by=c$.  Since ${\tt gcd}(ab) = 1$ we have $u$ and $v$ such that $au+bv=1$.  Now $c=1c=(au+bv)c=auc + bvc= auby + bvax = ab(uy+vx)$ which is divisible by $ab$ by inspection, so $ab|c$, which is what we needed to prove.

\item[Prop. 2.4.9 part 2:]  Suppose ${\tt gcd}(ab) = 1$.  Suppose $a|bc$.  Then $a|c$.

Since $a|bc$ we have $bc=xa$ for some $x$.  Since ${\tt gcd}(ab) = 1$ we have $ay+bz=1$ for some $y,z$.  Thus $c=1c = (ay+bz)c=ayc+bzc=ayc +bxa=a(cy+bx)$ which is divisible by $a$ by inspection, so $a|c$, which is what we need to show.

\end{description}

I was setting out to prove Prop 3.7.1 at the end of class.  These notes may contain a proof of that theorem before Tuesday's class:  keep an eye on them.

\newpage

\section{Homework 2, assigned 1/21/2022, due 1/27/2022}

In section 2.5 in Crisman, problems 3, 5 (you may use my spreadsheet, but say how you did it), 6, 7, 8 (with no more than a simple calculator;  of course I cannot stop you from using the spreadsheet to check, but you do need to know how to do this by hand for in-class tests), 10 (same remark as on 8), 15 (coprime is another word for ``relatively prime"), 17, 20.

\section{A Problem Solved:  Pythagorean Triples}

\begin{description}

\item[Definition:]   A {\em Pythagorean triple\/} is a triple of natural numbers $a,b,c$ such that $a^2+b^2=c^2$.

\item[Geometric Motivation:]   For any Pythagorean triple $a,b,c$ , there is a right triangle with legs $a,b$  and hypotenuse $c$.  The 3,4,5 Pythagorean triple can be used as a practical method to form a right angle.

\item[Example:]  $3^2+4^2=5^2$

\item[Definition:]  A {\em primitive Pythagorean triple\/} is a Pythagorean triple with no common factors other than 1.

\item[Motivation:]  If $a,b,c$ are a Pythagorean triple and $d \neq 1$ is a common factor of $a,b,c$,
so $a'd=a, b'd=b, c'd=c$, then $(a'd)^2 + (b'd)^2 = (c'd)^2$ implies $a'^2+b'^2=c'^2$ (divide both sides by
$d^2$).   If we further let $d$ be the greatest common divisor of $a,b,c$, then $a',b',c'$ will be a primitive Pythagorean triple.   So if we know all the primitive triples, we can obtain all the triples by multiplying by constants.

\item[Lemma:]  In a primitive Pythagorean triple $a,b,c$, the numbers $a,b$ will neither both be odd nor both be even.

\item[Proof:]  if $a,b$ were both even, then $a^2+b^2+c^2$ would be even, so $c$ would be even and
$a,b,c$ would not be a primitive triple.

If $a,b$ were both odd, then $a=2x+1$, $b=2y+1$, and since $a^2+b^2=c^2$ would be even, $c^2$ and so $c$
are even, so we can set $c=2z$.   Now $a^2+b^2=(2x+1)^2+(2y+1)^2 = 4x^2+4x+4y^2+4y+2$ is not divisible by 4, while $(2z)^2= 4z^2$ is divisible by 4.  But these two quantities are supposed to be equal.  So this situation is impossible.

\item[Observations:]  Let $a,b,c$ be a primitive Pythagorean triple.   We may safely assume that $a$ is odd and $b$ is even (if not we could switch them), and $c$ is thus odd.

Since $a^2+b^2=c^2$ we have $a^2=c^2-b^2=(c+b)(c-b)$.

$c+b$ and $c-b$ have no common factors.  Both are odd numbers.   If $d$ were a prime factor of both,
$d$ would be odd and $d$ would also be a factor of $2c$ (their sum) and $2b$ (their difference) and so would be
a factor of both $c$ and $b$ which is impossible as we have a primitive triple.

$a^2$ is a perfect square, so every prime in its factorization has an even exponent.  Any prime which goes
into $a^2$ goes into only one of $c+b$ and $c-b$, and in fact we can see that the exponent of each such prime must be the same as its exponent in the expansion of $a$, and so even.  And so $c+b$ and $c-b$ are perfect squares.

Set $c+b = s^2$ and $c-b=t^2$.   Notice that $s$ and $t$ have no common prime factors, as any common prime factor of these would be a common factor of $b$ and $c$ by reasoning already given.

Algebra gives $c=\frac{s^2+t^2}2$ and $b=\frac{s^2-t^2}2$.  $a^2=(c+b)(c-b)=s^2t^2$ so $a=st$.

\item[Theorem:]   Every primitive Pythagorean triple is of the form $st$, $\frac{s^2-t^2}2$, $\frac{s^2+t^2}2$,
where ${\tt gcd}(s,t)=1$ and $s,t$ are both odd..   Moreover, all such triples are primitive Pythagorean triples.

\item[Proof:]   The first sentence has been shown to be true in the observations above.  The second sentence
requires slightly more work.

That for any $s,t$ at all  ($s>t$, both odd or both even) $st$, $\frac{s^2-t^2}2$, $\frac{s^2+t^2}2$ is a Pythagorean triple is just algebra.

What remains is to shown that if ${\tt gcd}(s,t)=1$, then this triple is primitive.   It is enough to show
that $\frac{s^2-t^2}2$, $\frac{s^2+t^2}2$ have no common factors.   Any prime common factor of these two
numbers would be a prime factor of $s^2$, the sum of these two numbers, and $t^2$, their absolute difference.
But any prime which goes into $s^2$ and $t^2$ also goes into $s,t$ (by the lemma on prime factorizations proved earlier), and $s,t$ have no common prime factor.

\end{description}

To my mind, this is an example of the fact that proofs in number theory are often rather odd and indirect.  Others might not think so.

These notes are taken from a context where the usual results about prime factorizations were assumed.  We will justify our appeals
to prime factorizations using two specific facts, Prop 3.7.1 and 7.7.2 from Crisman.  Notes on my proofs of these results will appear
here later.

\begin{description}

\item[Prop. 3.7.1:]  If $a^2 | b^2$ then $a|b$.  

\item[Proof:]  It isn't clear to me that my proof above uses this, but it is easy enough to prove.

We remark first that it is enough to prove this result when ${\tt gcd}(a,b)=1$:  assume the theorem in this special case, let $a,b$ be general integers, and assume $a^2 | b^2$.  Then if $d={\tt gcd}(a,b)$, we have $a=a'd$ and $b=b'd$ (because $d$ goes into $a,b$) and we have ${\tt gcd}(a',b')=1$, because if $a'$ and $b'$ had a nontrivial common factor $k$, $kd>d$ would go into both $a'd=a$, and $b'd=b$, and $d$ is the greatest common divisor of $a$ and $b$.  So we have $a'^2d^2 | b'^2d^2$, from which we have $a'^2|b'^2$, from which we have $a'|b'$ by the special case of the Theorem, from which we have $a'd=a|b = b'd$.

Now we prove the special case.  Suppose that $a^2|b^2$ and ${\tt gcd}(a,b)=1$.  Then for suitable $x,y$ we have $ax+by=1$ so we have
$b = b1 = b(ax+by) = abx + b^2y$.  $a$ goes into $abx$ by inspection and it goes into $a^2$ which goes into $b^2y$ by hypothesis.

\item[Prop. 3.7.2:]  For any integers $a,b,c$, if ${\tt gcd}(a,b)=1$ and $ab=c^2$ then $a$ and $b$ are perfect squares.

\item[Proof:]  We expect, in fact that $a = {\bf gcd}(a,c)^2$.

Certainly ${\bf gcd}(a,c)^2 | c^2$.  Equally clearly, ${\bf gcd}(a,c)^2$ is relatively prime to $b$, so it goes into $a$ by theorems already shown, since $ab=c^2$ and $a,b$ have no common factors.  Thus ${\bf gcd}(a,c)^2 | a$.

Now show that $a|{\tt gcd}(a,c)=(ax+cy)^2$ for suitable $x,y$, $= a^2x^2 + 2acy +c^2y$, and $a$ goes into the first two terms by inspection and the last because $c^2=ab$.  So $a|{\bf gcd}(a,c)^2$.

Two positive integers which go into each other are equal.

I am proud of this proof, it is much better than the one in Crisman!

\end{description}

\section{Everything you might want to know about primes$\ldots$well, on day one:  lecture of 1/27/2022}

We discuss the important notions of prime and composite number.

\begin{description}

\item[Definition:]  A prime number is a positive integer with exactly two positive integer divisors.

\item[Observations:]  Every positive integer $n$ has 1 and $n$ as divisors.  If $n=1$, this fails to meet our definition, so 1
is not prime.  If $n>1$, then $n$ has at least two positive integer divisors, and will be prime just in case it has no others.
So the definition given is equivalent to ``$n$ is prime iff $n>1$ and has no factors other than 1 and itself."

\item[Definition:]  A positive integer $n$ is composite iff there are integer $a,b$ with $1<a \leq b < n$ and $ab=n$.
Notice that 1 is not composite.

\item[Theorem:]  Every natural number $n \geq 2$ can be expressed as a prime or a finite product of primes.

\item[Proof:]  Use the Well-Ordering Principle, and argue by contradiction.

If there is an integer $w \geq 2$ which is neither a prime nor a finite product of primes, then there is a smallest one, because the set of such integers would be a nonempty set of positive integers, and so have a smallest element.

Suppose that the Theorem is false, so this $w$ exists.

$w$ is not 1 and is not prime, so there are $a,b$ with $1<a\leq b <w$ and $w=ab$.

Since $a<w$ $b<w$ and $a$ and $b$ are both $\geq 2$ they are each either primes or finite products of primes.
But then $ab=w$ is a finite product of primes, which is a contradiction.

\item[Corollary:]  An immediate consequence is that any integer greater than one has at least one prime divisor.

\item[Theorem (Euclid?):]  There are infinitely many prime numbers.

\item[Proof:]  Suppose otherwise.  Then there is a finite list $p_1,\ldots,p_n$ containing all primes.  Define $P$ as
$\prod_{i=1}^n\,p_i$.  The integer $P+1$ is greater than 1, so it has a prime factor $q$.  There must be $j$ such that $q=p_j$.
Now $q|P$ because $P$ is the product of all primes, and $q|(P+1)$ by choice of $q$, so $q|(P+1)-P=1$, and $q|1$ is absurd.

\item[comments:]This proof is so well-known and (relatively) simple that some have proposed that every educated person should know it.

Here is a subtler related result.

\item[Theorem:]  There are infinitely many primes $p$ such that $p \,\,{\tt mod}\, \,4=3$.

\item[Comments:]  Obviously there are no primes $p$ such that $p \,\,{\tt mod}\, \,4=4$, and only one (2) such that $p \,\,{\tt mod}\, \,4=2$.
If $p$ is an odd prime, it will either be of the form $4k+1$ or the form $4k+3$.  It would seem natural that there are infinitely many primes of both kinds:  this is much easier to prove for 3 than for 1.

\item[Lemma:]  If $a \,{\tt mod}\, 4 =1$ and $b \,{\tt mod}\, 4 = 1$ then $ab \,{\tt mod}\, 4 = 1$.

$(4x+1)(4y+1) = 16xy + 4x+4y+1 = 4(4xy+x+y)+1$.  The Division Algorithm theorem tells us that the remainder is uniquely determined.

If $a \,{\tt mod}\, 4 =1$ and $b \,{\tt mod}\, 4 = 3$ then $ab \,{\tt mod}\, 4 = 3$.

$(4x+1)(4y+3) = 16xy + 12x +4y+3 = 4(4xy+3x+y)+3$.  The Division Algorithm theorem tells us that the remainder is uniquely determined.

If $a \,{\tt mod}\, 4 =3$ and $b \,{\tt mod}\, 4 = 3$ then $ab \,{\tt mod}\, 4 = 1$.

$(4x+3)(4y+3) = 16xy + 12x+12y +9 = 4(4xy +3x+3y+2)+1$.  The Division Algorithm theorem tells us that the remainder is uniquely determined.

\item[Corollary:]  Any integer of the form $4k+3$ must have a prime divisor of the form $4k+3$.

\item[Proof:]  Suppose otherwise.  Then the integer in question would have a prime factorization in which every prime was of the form $4y+1$, and a product of numbers of this form is of the form $4k+1$, not $4k+3$.

\item[Proof of the Main Theorem:]  Suppose that there are only finitely many primes $p_1,\ldots,p_n$ of the form $4k+3$.

Define $P$ as $\prod_{i=1}^n \, p_i$.

Either $P\,{\tt mod}\,4=1$ or $P{\tt mod 4}=3$.

In the first case $(P+2)\,{\tt mod}\,4 = 3$ so $P+2$ has a prime factor $q$ of the form $4x+3$, which goes into $P$ and $P+2$, so $q|2$, which is absurd, since $q$ is an odd prime.

In the second case, $(P+4)\,{\tt mod}\,4=3$ so $P+4$ has a prime factor $q$ of the form $4x+3$, which goes into $P$ and into $P+4$, and so goes into 4, which is absurd.

This can be proved, as a student noted, without cases.  Notice that $4P-1 = 4(P-1)+3$ is of the form $4k+3$, so has a prime factor $q$ of the form $4x+3$, and we observe that $q|4P$ and $q|(4P-1)$ so $q|1$, which is absurd.

\item[Theorem:]  Each positive integer can be expressed in exactly one way as the product of a nondecreasing sequence of primes.

\item[Proof:]  The statement in terms of a nondecreasing sequence is meant to tell us what is meant by uniqueness of factorization:  applications of the associative and commutative laws of multiplication do not give different factorizations.

We prove this by contradiction using the Well-Ordering Principle.

Suppose there is some $w = \prod_{i=1}^n p_i = \prod_{i=1}^m q_i$ where $p$ and $q$ are different finite nondecreasing sequences of primes.
Then there is a smallest such $w$ by the W.O.P.

We argue that $p_1$ cannot be one of the $q_i$'s, say $q_j$.  If it were, then $\frac w{p_1} = \prod_{i=2}^n p_i = \prod_{i=1 \wedge i\neq j}^m q_i$  would be both less than $w$ and would have two different prime factorizations, which is a contradiction.  (Removing the same term from two nondecreasing sequences of integers which are distinct must give distinct sequences; otherwise, adding the same term back, necessarily in the same position because the order determines it, would give the same sequence).

Now we prove using Euclid's Lemma and induction, that $p_1 | \prod_{i=k}^m q_i$ for all $k$ for which this makes sense.

Basis:  $p_1 | \prod_{i=1}^m q_i = w$.

Induction step:  Suppose $p_1 | \prod_{i=k}^m q_i$.  $\prod_{i=k}^m q_i= (q_k)^z\prod_{i=k+z}^m q_i$ for some $z$ with $q_{k+z}\neq q_k$ (I overlooked this in class, but so did everyone else) [or $\prod_{i=k}^m q_i= (q_k)^z$, in which case we have immediately that $p_1\neq q_k$ does not go into it, contradicting the inductive hypothesis].  These two numbers,  $(q_k)^z, \prod_{i=k+z}^m q_i$ are relatively prime, and $p_1$ goes into their product, so by Euclid's Lemma $p_1$ goes into one of the factors.  But it does not go into $(q_k)^z$, so it must go into $\prod_{i=k+z}^m q_i$,
and so it goes into $\prod_{i=k+1}^m q_i$, which can differ only in having more factors.

This completes the embedded induction proof.

So set $k=m$ and we have $p_1 | \prod_{i=m}^m q_i = q_m$, which is absurd. And this completes the proof.


\newpage

\end{description}


\section{Homework 3, posted 1/28/2022, due one week from 1/27/2022}

Homework 3: Use the results proved in class to describe at least five distinct primitive Pythagorean triples; do problems 14, 18, 19 on p. 34 in Crisman; as an extension of problem 19 look for patterns as to which numbers in a PPT can be divisible by 5 (for this, at least display some results of investigation; I'll be impressed if you can prove something). On p. 75 do problems 2,5,10,12,13. I like problem 20 on the next page; I'm not requiring it but Ill award EC if you do it.

\newpage

\section{Modular arithmetic lectured, Feb 1 and 3}

We begin by defining the congruence relation.

\begin{description}

\item[Definition (congruence mod m):]  Let $m>1$ and let $x,y$ be integers.  We say $x \equiv y \,\,{\tt mod}\, \, m$, or compactly
$x \equiv_m y$, just in case $m|(x-y)$.

\item[Theorem:]  $x \equiv y \,\,{\tt mod}\, \, m$ if and only if $x \,{\tt mod}\, m = y \,{\tt mod}\, m$.

\item[Proof:]  I suggest that you try to write the proof.  It follows from the Division Algorithm theorem.  You have to show the implication in both directions.

\item[Theorem:]  $\equiv_m$ is an equivalence relation.

\item[Proof:]  We need to prove that this relation is reflexive, symmetric, and transitive.

Let $m>0$.  Let $x,y,z$ be arbitrarily chosen integers.

We want to show $x \equiv_m x$.  This means $m|(x-x)$ which is equivalent to $m|0$, which is true.

We want to show that if $x \equiv_m y$ then $y \equiv_m x$.  Assume that $x \equiv_m y$.  This means $m|(x-y)$ and thus
for some integer $k$, $x-y=km$.  But then $y-x=(-k)m$, so $m|(y-x)$, so $y \equiv_m x$. 

We want to show that if $x \equiv_m y$  and $y \equiv_m z$, $x \equiv_m z$ follows.  Suppose $x \equiv_m y$  and $y \equiv_m z$.
Then $m|(x-y)$ and $m|(y-z)$.  It follows that $m|((x-y)+(y-z))$ and $(x-y)+(y-z)=x-z$ so $m|(x-z)$ so $x \equiv_m z$.

The proof is complete.

\item[Theorem:]  $\equiv_m$ respects addition and multiplication in the sense that if \newline $x\equiv_m x'$ and $y\equiv_m y'$ we have
$x+y \equiv_m x'+y'$ and $x\cdot y \equiv_m x' \cdot y'$.

\item[Proof:]  Suppose $x\equiv_m x'$ and $y\equiv_m y'$.  This is equivalent to there being integers $u$ and $v$ such that $x'=x+um$ and $y' = y+vm$.

Then $x'+y' = (x+um)+(y+vm) = (x+y) + m(u+v)$, so $x+y \equiv_m x'+y'$.

and $x'\cdot y' = (x+um)(y+vm) = x\cdot y + m(xv + yu + uvm)$, so $x\cdot y \equiv_m x' \cdot y'$.

This allows us to make addition and multiplication tables for mod $m$ arithmetic, with just the finite system of ``numbers" from 0 to $m-1$.

The interpretation of these ``numbers" admits two possibilities:  we can interpret them as congruence classes of integers, that is,
equivalence classes under $\equiv_m$, or as the remainders on division by $m$.  Either approach works.  The numbers may be called residues, if we think of them as remainders, or residue classes, if we think of them as equivalence classes.

For mod 4 arithmetic, we have

$$\begin{array}{c|cccc}

+ & 0& 1 &2& 3 \\ \hline

0 & 0& 1 &2& 3 \\
1 & 1& 2 &3& 0 \\
2& 2& 3 &0 & 1 \\
3& 3& 0 &1 & 2 \\

\end{array}$$

as the addition table.  Notice that each number has an additive inverse.  This is not surprising as the original system of integers on which this is based has additive inverses.  In general, the addition inverse of $a$ mod $m$ is $m-a$. For example the additive inverse of 3 mod 10 is 7.

For mod 4 arithmetic, we have 

$$\begin{array}{c|ccccc}

* & 0& 1 &2& 3 \\ \hline

0 & 0& 0 &0& 0 \\
1 & 0& 1 &2& 3 \\
2& 0& 2 &0 & 2 \\
3& 0& 3 &2 & 1 \\

\end{array}$$

Notice that the facts about multiplication of numbers of the forms $4k+1$ and $4k+3$ which we used in theorems proved earlier are encoded in this table.

Notice that we do not have multiplicative inverses for all nonzero numbers in this system:  we have $x$ such that $3x=1$ and $x$ such that $1x=1$
but no $x$ such that $2x=1$.  We do not have multiplicative inverses in the integers, so this is not surprising.

If we look at the multiplication table for mod 5 arithmetic, something unexpected happens.

$$\begin{array}{c|ccccc}

* & 0& 1 &2& 3 & 4\\ \hline

0 & 0& 0 &0& 0 &0 \\
1 & 0& 1 &2& 3 & 4\\
2& 0& 2 &4 & 1 & 3 \\
3& 0& 3 &1 & 4 & 2\\

4 & 0 & 4 & 3 & 2 & 1

\end{array}$$



\end{description}

Each nonzero residue has a multiplicative inverse:  the recprocal of 1 is 1, of 2 is 3, of 3 is 2, of 4 is 4.

This is surprising:  the system ends up looking more like the rationals than like the integers.

There is a theorem of course

\begin{description}

\item[Theorem:]  For each residue $a$ in mod $m$ arithmetic, there is an $x$ such that $ax \equiv_m 1$ if and only if
${\tt gcd}(a,m)=1$.

\item[Proof:]  If $ax \equiv_m 1$ then $ax-1$ is divisible by $m$, so any common divisor of $ax$ and $m$ would also be a divisor of 1,
so certainly $a$ and $m$ have no nontrivial common factors.

If ${\tt gcd}(a,m)=1$ then there are integers $x$ and $y$ such that $ax+my=1$, so for this $x$, $ax \equiv_m 1$.

\end{description}

This doesnt quite say that if ${\tt gcd}(a,m)=1$ implies that $a$ has a multiplicative inverse mod m.  The proof of this is completed
by the following observation:

\begin{description}

\item[Theorem:]  For each residue $a$ in mod $m$ arithmetic, if ${\tt gcd}(a,m)=1$ and $ax \equiv_m ay$ (and so in particular if $ax=ay=1$, which implies ${\tt gcd}(a.m)=1$) then $x \equiv_m y$.

\item[Proof:]  If ${\tt gcd}(a,m)=1$ and $ax \equiv_m ay$, then $m|a(x-y)$, and then by theorems proved above, $m|(x-y)$, since $m$ is relatively prime to $a$, and so $x \equiv_m y$.

\end{description}

This has the incidental effect that the multiplicative inverse of $a$ in mod m arithmetic, if it exists, is unique (up to congruence mod m).  More generally, it is a version of the cancellation property of multiplication.

\begin{description}

\item[Definition:]  We say that $a^{-1} \,\,{\tt mod}\, \,m$ is the unique remainder mod m such that $ax \equiv_m 1$, if it exists.

\item[Observation:]  If $p$ is prime, $a^{-1}\,\,{\tt mod}\,\,p$ is defined for each $a$ such that $a \not\equiv_p 0$.  That is, modular arithmetic mod $p$ satisfies the multiplicative inverse property.

\item[Proof:]  We have shown above that $a^{-1}\,\,{\tt mod}\,\,m$ is defined iff $a$ and $m$ are relatively prime.  A prime $p$ is relatively prime to any $a$ unless $p|a$, that is, $a \equiv_p 0$.

\end{description}

And this is surprising.  This means in effect that division is defined in the mod p integers, whereas it is not defined in the integers as usually understood.

In general, it should be easy to convince yourself that for any m, mod m arithmetic inherits from the integers the commutative, associative and distributive laws, the identity laws. and additive inverses.  It does not inherit the zero factor property:  you might want to work out
why the fact ``if $ab=0$ then $a=0$ or $b=0$" which is true in the integers does not carry over to mod m arithmetic unless m is prime.
And the fact that the multiplicative inverse property characteristic of the rationals holds in mod p arithmetic is a surprise.

What mod m arithmetic does not have which distinguishes it from the arithmetic of the integers is order properties.

\begin{description}

\item[Example:]  Compute $12^{-1} \,{\tt mod}\, 137$.

Use the usual Euclidean algorithm calculation (my table) to find $x$ and $y$ so that $137x + 12y = 1$, so $12y \equiv_{137} 1$.

We get $(137)(5) + 12(-57) =1$, so $y$ is to be $-57$...but the additive inverse of 57 in mod 137 arithmetic is 137-57 = 80, the answer.
You will need to make this last move about half the time.

It is wise to check that (80)(12) \,{\tt mod}\, 137 is indeed 1.

\end{description}

\subsection{Exponentiation}

It is not the case that exponentiation respects congruence mod m.  That is, it is not true in general that if $x \equiv_m y$ and
$r \equiv_m s$ that $x^r \equiv_m y^s$.  It {\em is} true that if $x \equiv_m y$ then $x^r \equiv_m y^r$ :  this is true by repeated application of the fact that congruence respects multiplication.

Nonetheless, the pattern continues that we can efficiently compute congruence facts about large numbers by ignoring everything about them but their remainder mod m.

\begin{description}

\item[Algorithm (modular exponentiation):]  To compute $x^r\,\,{\tt mod}\, m$, first compute $x^{r {\tt div} 2}\,\,{\tt mod}\,\,m$.  Then
if $r \,{\tt mod}\,2=0$, $x^r\,\,{\tt mod}\, m = (x^{r {\tt div} 2})^2\,\,{\tt mod}\,\,m$ and if $r \,{\tt mod}\,2=1$, $x^r\,\,{\tt mod}\, m = ((x^{r {\tt div} 2})^2\cdot x)\,\,{\tt mod}\,\,m$, in either case a small multiplication problem mod m.

This is a recursive computation:  at the basis, note that we can certainly compute $x^1 \,\,{\tt mod}\,\, m$.

\end{description}

In practice, I execute the algorithm by making a list of exponents obtained by starting with $r$ and successively dividing by 2, throwing away remainders, then computing the powers from 1 upward.

\begin{description}

\item[Example:]  Compute $32^{1153}\,\,{\tt mod}\, 100$.  100 is a convenient modulus just because it is easy to take remainders on division by 100.

$$\begin{array}{c | c}

1153 & 76^2\cdot 32 = 184832 \equiv 32\\

576 & 76^2 = 5776 \equiv 76\\

288 & 76^2 = 5776 \equiv 76\\

144 &  76^2 = 5776 \equiv 76\\

72 & 76^2 = 5776 \equiv 76\\

36 &  24^2  =576 \equiv 76\\

18 & 32^2 = 1024 \equiv 24\\

9 &  76^2\cdot 32 = 184832 \equiv 32\\

4 & 24^2 = 576 \equiv 76\\

2 &  32^2 = 1024 \equiv 24\\

1 &  32\\

\end{array}$$

\end{description}

This turned out to be a rather special example, but the pattern should be clear enough.  This is known as the method of repeated squaring (with addition of an extra copy of the base at odd exponents).  The number of multiplications is roughly proportional to the log base 2 of the exponent, so this will handle very large exponents with computer support (hundreds of digits are no challenge).

I'll add more notes here about Fermat's Little Theorem, $a^{p-1} \equiv_p 1$ for $p$ prime, and the way it allows much simpler computation of exponentials.  I do not think it figures in your homework.


\subsection{The Linear Congruence Theorem}

In this section we look at the precise conditions under which a linear congruence
$$ax \equiv b \,\,{\tt mod}\,\, m$$ has a solution and how many solutions it has.

First of all, we can apply the results about multiplicative inverses in moduli to solve a special case.

\begin{description}

\item[Theorem:]  If ${\tt gcd}(a,m)=1$ then $$ax \equiv b \,\,{\tt mod}\,\, m$$ has exactly one solution $x$.

\item[Proof:]   ${\tt gcd}(a,m)=1$ then $a^{-1}\,\,{\tt mod}\,\,m$ exists, which we will just write $a^{-1}$.

$ax\equiv_mb$ implies $a^{-1}ax\equiv_ma^{-1}b$ which implies $x\equiv_ma^{-1}b$.  And further, $a(a^{-1}b) \equiv_m b$ is true.  So there is exactly one solution for $x$, that is
$(a^{-1}\, \,{\tt mod}\,\, m)\cdot b$.

\end{description}

We now consider the general situation.

\begin{description}

\item[Convention:]  In everything that follows until the main theorem is proved, let $d={\tt gcd}(a,m)$.

\item[Lemma:]  If $d \not|b$, then there is no solution $x$ for $ax \equiv b \,\,{\tt mod}\,\, m$.

\item[Proof:]  If $ax \equiv_m b$ then $b = ax+km$ for some integer $k$ and so since $d|a$ and $d|m$ we have $d|b$.  We have proved $ax \equiv_m b \rightarrow d|b$, from which the contrapositive $d\not|b \rightarrow ax \not\equiv_m b$ follows.

\item[Lemma:]  If $d|b$ then there is at least one solution to $ax \equiv b \,\,{\tt mod}\,\, m$. 

\item[Proof:]  If $d|b$ then all of $\frac ad, \frac md, and \frac bd$ are integers, and ${\tt gcd}(\frac ad,\frac md)=1$, so there is a unique solution $x$ to
$\frac adx \equiv \frac bd \,\,{\tt mod}\,\, \frac md$.  For this $x$, we have $\frac adx + k\frac md = \frac bd$ for some integer $k$, so we have
$ax+km=b$, so $ax \equiv_m b$.

\item[Main Theorem:]  Let $m>0$.  Let $a,b$ be integers.  Let $d={\tt gcd}(a,b)$.  If $d \not|b$ then $ax \equiv_m b$ has no solutions.
Otherwise this equation has $d$ solutions.

\item[Proof:]  Let $x$ be the unique solution to $\frac adx \equiv \frac bd \,\,{\tt mod}\,\, \frac md$.  We know that $ax \equiv_m b$.
Suppose $ay \equiv_m b$.  It follows that for some $k$, $ay + km = b$.  Thus $\frac ady + k\frac md = \frac bd$ so $\frac ady + k\frac md = \frac bd$,
so $\frac ad y \equiv_m \frac bd - k\frac md$.  This calculation is reversible:  a solution of this equation for any $k$ is a solution of the original equation.
For each $k$, this equation has exactly one solution, because $\frac ad$ is relatively prime to $m$.  And $k\frac md \equiv_m k'\frac md$ just in case
$k \equiv k' \,\,{\tt mod}\,\, d$, so there are in effect $d$ possible values for $k$, and so $d$ solutions to the original equation up to congruence.

\end{description}

I was having difficulty with details of this last proof under the influence of muscle relaxant on Thursday;  I will be lecturing it and giving numerical examples for computation on Tuesday.



\newpage


\section{Abstract algebra definitions lectured, Feb 3}

I have nothing particular of my own to say about these definitions yet.  I may soon revisit this and write more notes.

The definition of group on pp. 33-4 was lectured, and some theorems on p 36 (Props 3.17-18, uniqueness of the identity and the inverse).  Examples given after the definition may be instructive.

The definition of ring on p. 191 was lectured.  The definitions of extensions of this notion listed at the bottom are important.  Notice that
the integers are a ring but not a field (because not a division ring) and the integers are an integral domain (because they do satisfy the zero factor theorem).
Notice that mod p arithmetic gives a field, surprisingly, but that mod m arithmetic where m is composite gives a commutative ring which is not an integral domain (the zero factor theorem fails).

I will be happy to take questions about the questions from Judson which I ask, which are occasions for mathematical exploration.

I also may well settle down later and expand this section.  I'm tired out from all the stuff I wrote about modular arithmetic!

\section{Homework 4, due Feb 10}

Homework 4:  p.47 Crisman, 2, 3, 7, 12, 18, 19, p. 62 6, 10, 11, 13 (you can ask for others from 8-13), Judson, p. 40 problem 2, Judson, p. 205 problem 1 (do at least four parts)

\section{Notes on the Linear Congruence Theorem and the Chinese Remainder Theorem, Feb 8 and Feb 10 2022}

\subsection{What are the numbers of modular arithmetic?}

First, I am going to chat a little about what the objects are in mod m arithmetic.

Mod $m$ arithmetic is a finite system, with objects we usually refer to as $0,1,\ldots,m-1$.  There is some creative ambiguity in what these objects actually are.

They could be viewed as the remainders on division by $m$ (which are also called residues mod $m$).  In this case, when we compute
$p+q$ in mod $m$ arithmetic, we are computing $(p+q) \,{\tt mod}\, m$ in the ordinary sense, in order to be sure our answer is a remainder, and similarly for subtraction, additive inverse and multiplication (but {\bf not} for multiplicative inverse or division).

They could be viewed as equivalence classes of integers under the relation of congruence mod $m$.  In this case, we would understand $p$ and $q$ in mod $m$ arithmetic as shorthand for $\{p+km:k \in {\mathbb Z}\}$  and $\{q+km:k \in {\mathbb Z}\}$ (where $p$ and $q$ are the integers of the  same name)
and when we compute $p+q$ in mod $m$ arithmetic, we are adding elements of the sets:  $\{(p+km) + (q+k'm):k, k' \in {\mathbb Z}\}$ is exactly the same set
as $\{((p+q) \,{\tt mod}\,m)+km:k \in {\mathbb Z}\}$ (writing this out shows me what a complicated idea it is that I am asking you to accept!)

In general, which approach we are using does not make any difference.  When we say that an integer $x$ is to be ``identified" with $p$ in the sense of modular arithmetic
we can be taken as saying either that $x \,{\tt mod}\,m = p$ (if $p$ is understood as a remainder on division by $m$) or that $x \in p$ if $p$ is understood as the equivalence class $\{p+km:k \in {\mathbb Z}\}$ where $p$ is the integer of the same name).  The operations of modular arithmetic behave in the same way under either understanding.
\newpage
\subsection{The Linear Congruence Theorem, again}

I recapped the proof of this via a series of lemmas.  Some of these lemmas are theorems in their own right.

\begin{description}

\item[Lemma 1:]  Let $m>0$.  Let $a$ be a residue mod $m$ (recall that this just means, a remainder mod $m$, so $0\leq a <m$).  Suppose that ${\tt gcd}(a,m)=1$.  Then there is a unique residue $b$ mod $m$ such that $ab \equiv_m 1$.

\item[Proof:]  Because ${\tt gcd}(a,m)=1$, there are integers $x,y$ such that $ax+my=1$.  Notice that $ax \equiv_m 1$ follows.  So $x\,{\tt mod}\,m =b$ gives a residue $b$ such that $ab \equiv_m 1$ as desired.

But we need to show that there is only one.  Suppose that $ax\equiv_m ay \equiv_m 1$, so $x\,{\tt mod}\,m$ and $y\,{\tt mod}\,m$ are both candidates to be the $b$ we are looking for.  Then $m|(ax-ay)$ so $m|a(x-y)$ so by Euclid's Lemma $m|(x-y)$ so $x \equiv_m y$ so $x \,{\tt mod}\,m = y\,{\tt mod}\, m$:  there is only one residue $b$ with the desired property.

\item[Definition:]  Let $m>0$.  Let $a$ be a residue mod $m$.  Define $a^{-1}\,{\tt mod}\,m$ as the unique residue $b$ such that $ab \equiv_m 1$. 

\item[Obervation:]  It isn't part of the theorem stated above, but it is worth observing that if ${\tt gcd}(a,m)=d>1$ then there can be no $b$ such that $ab \equiv_m 1$.
We would have $ax+my=1$ for some integers $x,y$ and then $ax+my=1$ divisible by $d$, which is absurd.  So $a^{-1}\,{\tt mod}\,m$ is defined if and only if $a$ and $m$ are relatively prime.

\item[Lemma 2:]  Let $m>0$.  Let $a$ be a residue mod $m$ which is relatively prime to $m$.  Let $b$ be a residue mod $m$.  Then there is exactly one residue $x$ mod $m$ such that $ax \equiv_m b$.

\item[Proof:]  Since $a$ is relatively prime to $m$, $a^{-1}\,{\tt mod}\,m$ exists;  we will write it just $a^{-1}$.  So $a(a^{-1}b) \equiv_m  (aa^{-1})b \equiv_m 1b \equiv_m b$,
so $x=a^{-1}b\,{\tt mod}\,m$ is a solution of $ax \equiv_m b$.

Now suppose that $x$ is any solution of $ax \equiv_m b$.  It follows that $a^{-1}(ax) \equiv_m a^{-1}b$, and $a^{-1}(ax) \equiv_m (a^{-1}a)x \equiv_m 1x \equiv_m x$,
so $x \equiv_m a^{-1}b$, and we see that $a^{-1}b\,{\tt mod}\,m$ is the only residue which can be a solution to the equation (there are many integers $x$ such that $x \equiv_m a^{-1}b$, but of these only $a^{-1}b\,{\tt mod}\,m$ is a remainder on division by $m$).

\item[Lemma 3:]  Let $m>0$.  Let $a$ and $b$ be residues mod $m$ and let $d = {\tt gcd}(a,m)$.  Then $ax \equiv_m b$ has a solution $x$ if and only if
$d|b$.

\item[Proof:]  Suppose $d|b$.  We have $a=a'd$ and $b=b'd$ and $m=m'd$ with ${\tt gcd}(a',m')=1$ (if $a'$ and $m'$ had a nontrivial common factor $e$, verify that $de>d$ would be a common factor of $a$ and $m$).  Thus by Lemma 2 there is a unique residue $x$ mod $m'$ such that $a'x\equiv_{m'} b'$.  For some integer $k$,
$a'x +km' = b'$ by this last congruence.  But then $a'dx + km'd = b'd$, that is, $ax+km = b$, so $ax \equiv_m b$.  So if $d|b$ there is a solution to the congruence.

Suppose there is a solution $x$ to $ax \equiv_m b$.  Then $ax+km = b$ for some integer $k$.  $d|ax$ because $d|a$, and $d|km$ because $d|m$, so $d|ax+km$, so $d|b$.

So we have shown that ${\tt gcd}(a,m)|b$ if and only if there is a solution to $ax \equiv_m b$.

\item[Linear Congruence Theorem:]  Let $m>0$.  Let $a$ and $b$ be residues mod $m$ and let $d = {\tt gcd}(a,m)$.  Then there are exactly $d = {\tt gcd}(a,b)$ solutions to $ax \equiv_m b\,{\tt mod}\, m$.

Let $a'd = a, b'd = b, m'd=m$, as in the previous problem.  Let $x$ be the unique solution to $a'x \equiv_{m'} b'$.

Any solution to $ax \equiv_m b$, say $y$,  has $a(x-y) = a'd(x-y)$ divisible by $m=m'd$, so $a'(x-y)$ divisible by $m'$.  This means that $x \equiv_{m'} y$.

Any $y$ of the form $x+m'k$ actually is a solution:  we have $ax+um = b$ for some $u$, so we have $b=a(x+m'k)-am'k+um = a(x+m'k)-a'mk+um [{\tt note}\, am'=a'm]\equiv_m a(x+m'k)$.

Any $y$ of the form $x+m'k$ has to be congruent mod $m$ to some $x+m'k$ with $0 \leq k < d$:  any larger or smaller number of the form $x+m'k$ we can convert to this form by adding or subtracting some multiple of $m'd = m$.  And no two of these numbers are congruent mod $m$, because any two of them differ by less than $m'd=m$.  So we have exactly $d$ solutions, the remainders $x+m'k\,{\tt mod}\,m$ for $k$ ranging from 0 to $d-1$.

\end{description}

We give three examples.

\begin{enumerate}

\item Solve the linear congruence $$16x \equiv_{120} = 20$$

No solution, because the gcd of 16 and 120 is 8, which does not go into 20.

\item Solve the linear congruence $$77x \equiv_{120}n= 20$$

Use the Euclidean spreadsheet, or, better, build the table by hand, to find that $(-34)(120) + (53)(77) = 1$, so $(53)(77) \equiv_{120} 1$, so $77^{-1}{\tt mod}120 = 53$.

Without any calculation, Lemma 1 or the full congruence theorem tells us there is one solution up to congruence mod 120.

Now compute the solution:  multiply both sides of the congruence by 53 to get

$$77x \equiv_{120}n= 20$$
implies
$$(53)(77)x \equiv_{120}n= (20)(53)$$
which implies
$$x \equiv_{120} (20)(53){\tt mod}120 = 100$$

Lets check:  indeed $(77)(100){\tt mod} 120 = 20$.

\item Solve the linear congruence $$65x \equiv_{120} 20$$

Notice that the gcd of 65 and 120 is 5, so we expect to find 5 solutions.

We begin by dividing through by the gcd.

so we are solving $13x \equiv_{24} 4$

The spreadsheet (or the manual table computation which I strongly suggest to you) will give $(6)(24) + (-11)(13) = 1$
so $(-11)(13) \equiv_{24} 1$, so $24-11 = 13 = 13^{-1}{\tt mod} 24$ (a numerical coincidence, 13 is its own inverse mod 24).

Thus $$(13(13)x \equiv_{24} (13)(4) = 4$$

[multiplication by 13 didn't do much, this is a weird example]

so the only solution to $13x \equiv_{24} 4$ up to congruence mod 24 is 4 (check:  $(13)(4) {\tt mod} 24 = 4$.

So the solutions to the original congruence are of the form $4+24k$, and these numbers
are $4, 28, 52, 76, 100$ (we stop before 120 because we are looking for residues).  We suggest carrying out the check
for two of these as verification.




\end{enumerate}

\newpage
\subsection{The Chinese Remainder Theorem}

The Chinese Remainder Theorem allows us to solve simultaneous equations of the form $x \equiv_{m_1} a_1; x \equiv_{m_2} a_2;\ldots;x \equiv_{m_k}=a_k$ as long as for any $1 \leq i<j \leq k$ we have ${\tt gcd}(m_i,m_j)=1$.

We indicate how to solve this when there are two equations.

Suppose ${\tt gcd}(m,n)=1$.  We want to find an $x$ such that $x \equiv_m a$ and $x \equiv_n b$.  We show that we can find such an $x$, and moreover that
the solution is unique up to congruence mod $mn$.

We will mix things up a little and show the uniqueness first.  Suppose that $x \equiv_m a$ and $x\equiv_n b$ and $y \equiv_m a$ and $y\equiv_n b$ ($x$ and $y$ are both solutions to the system of equations).  Then $m|(x-y)$ and $n|(x-y)$.  But this implies, by a result already shown, that since $m,n$ are relatively prime, $mn|x-y$, so
$x \equiv_{mn}y$.  It should be clear that if $x \equiv_m a$ and $x \equiv_n a$ and $y \equiv_{mn} x$, then also $y \equiv_m a$ and $y \equiv_n b$.  So the solution set we are looking for, if it exists, will simply be a congruence class mod $mn$ (or a remainder mod $mn$ if we think of it in that style).

Now we argue that there is a solution.  Since $x \equiv_m a$, we have $x=a+km$ for some integer $k$.  Thus, if $x$ is a solution we must
have $a+km \equiv_n b$.  This gives us $km \equiv_n b-a$.  This gives us the solution $k = (b-a)(m^{-1} {\tt mod} n)$.

We plug this back into our equation for $x$ to get $x = a+(b-a)(m^{-1}{\tt mod}n)m$.  Clearly $x \equiv_m a$, because $(b-a)(m^{-1}{\tt mod}n)m$ is a multiple of $m$.
Further, $x \equiv_n b$ because $(m^{-1}{\tt mod}n)m \equiv_n 1$ so $x = a+(b-a)(m^{-1}{\tt mod}n)m \equiv_n a+(b-a)(1) = b$.

We actually compute solutions by using the extended Euclidean algorithm to compute multiplicative inverses.

Solve the simultaneous equations 

$$x \equiv_{11} 3$$

$$x \equiv_{18} 4$$

for $x$.

It is strategically better to use the larger of the two moduli for the first step.

$x=4+18k$ for some integer $k$

so $4+18k \equiv_{11} 3$

so $18k \equiv_{11} -1 \equiv_{11} 10$

Now we need $18^{-1}{\tt mod}11$ which simplified immediately to $7^{-1}{\tt mod} 11$

$(2)(11)+(-3)(7)=1$ (use the spreadsheet, but really you should do this manually for test practice) so $7^{-1}{\tt mod}11 = -3 {\tt mod}11 = 8$.

Now we multiply both sides of the last congruence by 8 to get $x \equiv_{11} (8)(7)x \equiv_{11} (8)(100) \equiv_{11} 80 {\tt mod} 11 = 3$

so $k=3$ works

so $x = 4+(18)(3) =58$

Let's check

$58 {\tt mod} 11 =3$

$58{\tt mod} 18 = 4$

as desired

The solution is unique up to congruence mod (11)(18) = 198 so the general solution is $58 + 198k$.

Another solution would be 58+198 = 256.  Feel free to check.

\section{Homework 5, due Feb 17 (accepted late within reason)}

Homework 5 (officially due on the 17th as usual, but likely to be accepted late within reason): Crisman p. 61 4 (I really like this; it might be very easy for you, or not), 14, 16, 17, 18, 19, 20, 21 (the rest being pleasantly computational)

\section{Week 7 notes:  a coming attraction, I hope to have them up by the 27th}

\subsection{More about modular arithmetic, theorems of Legendre, Wilson, and Fermat}

Before the test, we proved that in any prime modulus, a polynomial of degree $d$ has no more than $d$ roots (Legendre's theorem).

A polynomial of degree $d$ is a function $P(x) = \Sigma_{i=0}^da_ix^i$ (where we stipulate that $x^0=1$ for all $x$, including 0, and we provide that $a_d \not\equiv_p  0$).

At the basis, we have already shown that $ax \equiv_p b$ has exactly one solution if $a \not\equiv_p 0$, the solution being $x = a^{-1}_p b$  (we adopt the abbreviation $a^{-1}_p$ for $a^{-1}{\tt mod} p$).  This shows the result for degree 1 polynomials.

Suppose we have shown the result for all polynomials of degree $\leq d$.  Let $P(x)$ be a polynomial of degree $d+1$.   If $P(x)$ has no roots at all, we have that
$P(x)$ has $\leq d+1$ roots, and we are done.  So suppose that $P(r)\equiv_p 0$ for some root r.  For any root $x$ of $P$ we would have $P(x)= P(r)$ so
$\prod_{i=1}^{d+1} a_i(x^i-r^i) = 0$.  Now for any $i>0$, $x^i - r^i = (x-r)\prod_{j=0}^{i-1}x^jr^{(i-1)-j}$.  For $i=0$ the term $a_i(x^i-r^i) = 0$.  So, we can express
$P(x)-P(r)$ as $(x-r)Q(x)$, where the degree of $x$ is less than or equal to $d$, by factoring $x-r$ out of each term as indicated.   Any root of $P(x)=0$ is also a root
of $P(x)-P(r) = (x-r)Q(x) = 0$, and so by the Zero Factor Theorem (which does hold in prime moduli) must either be equal to $r$ or a solution of $Q(x)=0$, and by ind hyp
$Q(x)=0$ has no more than $d$ solutions, so $P(x)=0$ has no more than $d+1$ solutions.

After the test, I discussed Wilson's theorem and Fermat's little theorem in modular arithmetic, then switched gears to abstract algebra.

\begin{description}

\item[Wilson's Theorem:]  If $p$ is prime, $(p-1)! \equiv_p -1$.

\item[Proof:]  Suppose that $p$ is prime.  

$(p-1)! = 1\cdot \Pi_{i=2}^{p-2} i\cdot(p-1)$

$p-1 \equiv_p -1$ of course.

The idea of the proof is that $\Pi_{i=2}^{p-2}i$ can be reorganized in a way which makes it clear that it is congruent to 1.  Each $a$ between $2$ and $p-1$ has $a^{-1}{\tt mod}p$ defined and also between 2 and $p-2$ inclusive.  Moreover, for each such $a$, $a \neq a^{-1}{\tt mod} p$, because if $a=a^{-1}{\tt mod} p$, it follows that $a^2 \equiv_p 1$, and when $p$ is prime the only solutions to this equation are congruent to 1 or $p-1$.  So $\Pi_{i=2}^{p-2}i$ can be reorganized into a product of pairs of numbers which multiply to values congruent to 1 mod $p$, and so the entire product is $\Pi_{i=2}^{p-2}$ congruent to 1 mod $p$.  

$\Pi_{i=2}^{p-2}i = \Pi_{2 \leq i \leq p-1, (i^{-1}{\tt mod}p) >i}i\cdot (i^{-1}{\tt mod}p) \equiv_p \Pi_{2 \leq i \leq p-1, (i^{-1}{\tt mod}p) >i}1 =1$

Just for laughs, I provide actual summation notation for this argument.  It is tricky.

So $(p-1)! = 1\cdot \Pi_{i=2}^{p-2} i\cdot(p-1) \equiv_p 1\cdot 1\cdot -1 = -1$



\item[Inverse of Wilson's Theorem:]  If $m=4, (m-1)! \equiv_m 2$;  for composite $m>4$, $(m-1)! \equiv_m 0$.

\item[Proof:]  That $(4-1)! = 6 \equiv_4 2$ is immediate.

If $m>4$ is composite, then $m = ab$ for some $2 \leq a,b \leq m-1$.  If $a \neq b$, then both $a$ and $b$ appear as factors in $(m-1)! = \Pi_{i=1}^{m-1} i$, and so $ab=m | (m-1)!$ so this is congruent to 0 mod $m$.
If $a$ and $b$ cannot be chosen to be distinct, then $m = p^2$ for some prime $p$ greater than 2, and so both $p$ and $2p$ appear as factors in $(m-1)! = \Pi_{i=1}^{m-1} i$,
so $2p^2 = 2m | (m-1)!$ which is again congruent to 0 mod $m$.

\end{description}

This seems to give us a fine test for primality:  unfortunately, there is no easy way to compute $(m-1)! {\tt mod} m$ for very large $m$ that we know of.

\begin{description}

\item[Fermat's little theorem:]  If $p$ is prime and $p \not| a$, $a^{p-1}\equiv_p 1$.

\item[Proof:]  The product of all nonzero residues mod $p$ is $(p-1)!$.  

The product of all numbers $ai$ where $i$ is a nonzero residue mod $p$ is
$a^{p-1}(p-1)!$.  

The second product is congruent mod $p$ to the product of all numbers $ai {\tt mod} p$, where $i$ is a nonzero residue mod $p$.  

But the product of all numbers $ai {\tt mod} p$ for $i$ a nonzero residue mod $p$ is also $(p-1)!$, because each nonzero residue $i$ mod $p$ is of the form $aj {\tt mod}p$ where $j$ is the nonzero residue $(a^{-1}_pi ){\tt mod} p$:  the third product is the product of exactly the same numbers as in the first product, in a different order.  So $a^{p-1}(p-1)! \equiv_p (p-1)!$, from which it follows that
$a^{p-1}\equiv_p 1$ (because $(p-1)!$ is certainly not congruent to 0 mod $p$, so it has a multiplicative inverse;  not to mention that we know from the previous theorem that it is in fact congruent to $-1$;  so it can be cancelled)

\item[Corollary:]  For any prime $p$ and any nonnegative integers $a,b$, $$a^b \equiv_p (a {\tt mod} p)^{(b{\tt mod}(p-1))}.$$

\end{description}

This gives a much simpler method of computing large powers in arithmetic mod $p$ where $p$ is prime than repeated squaring.

\subsection{Initial steps in abstract algebra}

I may do infill in these notes as we go forward (add more detail to text I have already written).

\begin{description}

\item[Definition:]  A {\em group} is a set $G$ equipped with an operation $\circ:(G \times G) \rightarrow G$ with the following properties:

\begin{description}

\item[associativity:]  For any $a,b,c \in G$, $(a \circ b) \circ c = a \circ (b \circ c)$

\item[identity:]  There is an element $e$ of $G$ such that for any $a \in G$, $e \circ a = a \circ e = a$.  Such an element $e$ is called an identity for the group.

\item[inverse:]  For any identity $e$ of the group and for any element $a$ of $G$, there is an element $b$ of the group such that $a \circ b = b \circ a = e$.

\end{description}


\item[Theorem:]  The identity element in a group is unique.

\item[Proof:]  Suppose that for all $a \in G$, $e \circ a = a \circ e = a$, and also for all $a \in G$, $e' \circ a = a \circ e' = a$.  It follows that $e \circ e'$ is equal both to $e$ and to $e'$, and so $e=e'$.

\item[Theorem:]  For each $a \in G$ there is exactly one $b \in G$ such that $a \circ b = b \circ a = e$.

\item[Proof:]  Suppose $a \circ b = b \circ a = e$ and $a \circ c = c \circ a = e$.  It follows that

$b = b \circ e$  identity

$= b \circ (a \circ c)$ hypothesis

$= (b \circ a) \circ c$  associativity

$= e \circ c$ hypothesis

$=c$  identity

So there is only one inverse of $a$.

\item[Definition:]  For each $a \in G$ we define $a ^{-1}$ as the unique inverse of $a$.

\item[Abelian groups:]  Notice that the group operation is not assumed to be commutative.  If a group $G$ in addition satisifies the property ``for all $a,b \in G$, $a \circ b = b \circ a$, we say that the group is {\em abelian\/}.

\item[Examples:]  The integers or the rationals or the reals with addition as the operation make up a group.  Why do the natural numbers not make up a group?

The nonzero rationals or reals with multiplication as the operation make up a group.

The residues mod $n$ (or the congruence classes mod $n$) make up a group, usually called ${\mathbb Z}_n$, with addition (mod $n$) as the operation.

The residues mod $n$ {\em which are relatively prime to $n$\/} make up a group, which we call $U_n$, with multiplication (mod $n$) as the operation.

These are all abelian groups.

The symmetries of a triangle, discussed at length in the book, make up a group, which is our first example of a nonabelian group.

The permutations of an $n$ element set (with composition as the operation) make up a group called $S_n$ which has $n!$ elements and is nonabelian.

Matrices with nonzero determinants make up a nonabelian group (under matrix multiplication).



\item[Notational conventions:]  We may use the usual notation for multiplication for a group, writing $ab$ instead of $a \circ b$, and possibly writing 1 instead of $e$ for the identity.

We may use additive notation, usually for an abelian group, writing $a+b$ for $a \circ b$, writing 0 for the identity and $-a$ for the inverse.

\item[Solving equations:]  For any $a,b \in G$, the equations $ax=b$ and $xa=b$ have unique solutions.

\item[Proof:]  We prove only the first statement.  Notice that $a(a^{-1}b) = (aa^{-1})b = eb = b$, so $x = a^{-1}b$ is a solution.  Now suppose that $ax=b$.  It follows that $a^{-1}(ax) = a^{-1}b$ and $a^{-1}(ax) = (a^{-1}a)x = ex = x$, so i$x = a^{-1}b$, which we see is the only solution.

\item[Cancellation:]  for any $a,b,c \in G$, if $ac=bc$ then $a=b$, and if $ca = cb$, then $a=b$.

\item[Proof:]  In short, multiply both sides by the inverse.

\item[Definition (powers or multiples);]  Define $g^n$ (or $ng$ if using additive notation) so that $g^0 = e$, and for positive integers $n$, $g^{n+1} = g^ng, g^{-n} = (g^n)^{-1}$.

\item[Theorems:]  Certain familiar properties of exponents hold.  We can prove $g^{m+n} = g^mg^n$ and $(g^m)^n = g^{mn}$ for $g$ any group element and $m,n$ any integers.

One thing we absolutely cannot count on is $g^nh^n = (gh)^n$:  this only works if the group is abelian.

We suggest as an exercise if you haven't been assigned it already to show that if this identity holds for all $g,h,n$ that the group is actually abelian. Hint:  it is enough for $(ab)^2 = a^2b^2$ to hold for all $a,b$:  use cancellations to show that this implies $ab=ba$.

A theorem which we proved in class is that $(ab)^{-1} = b^{-1}a^{-1}$ ({\em not\/} $a^{-1}b^{-1}$)

\item[Definition (subgroup):]  If $G$ is a group and $H$ is a subset of $G$, we say that $H$ is a subgroup of $G$ if $H$ with the restriction of the same operation used on $G$ is a group.

\item[Theorem:]  Let $G$ be a group with operation $\circ$.  $H$ with the restriction of the operation $\circ \cap (H \times H)$ is a group iff for each $a,b \in h$, $a\circ b \in h$, and the identity $e \in H$,
and for each $a \in H$, $a^{-1} \in H$.

\item[Theorem:]  Let $G$ be a group with operation $\circ$.  $H$ with the restriction of the operation $\circ \cap (H \times H)$ is a group iff $H$ is nonempty and for any $a,b \in H$, we have $ab^{-1} \in H$.

\item[Definition and Theorem:]  Let $G$ be a group.  Define the cyclic subgroup generated by $a$ as $\{a^n:n \in {\mathbb Z}\}$.  The cyclic subgroup is a subgrup of $G$ (this is the theorem part).  The size of the cyclic subgroup generated by $a$ we call the {\em order\/} of $a$ (this is either a positive integer or infinite).

\item[Theorem:]  Cyclic subgroups are abelian groups.

\item[Definition:]  If $G$ is a group and $a \in G$ and the cyclic subgroup generated by $a$ is all of $G$, we say that $G$ is a cyclic group and that $a$ is a generator of $G$.

\end{description}

\newpage

\section{Homework 6}

Homework 6:  Crisman, 7,7 exercises starting p. 89, 7, 9;  Judson, 3.5 exercises starting p. 40, problems 5, 6 (remember, this is residues mod 12 that are relatively prime to 12 under multiplication), 7, 28, 31 (think about uniqueness of inverses), 32 (prove the contrapositive), 4.5 exercises starting p. 55, problem 1abc, 3 bg, 5.

\section{March 1:  Cyclic Groups and the Circle Group in the complex numbers}

Let $G$ be a group (use multiplicative notation) and $a$ an element.  

We define the cyclic subgroup of $G$ generated by $a$ as $\{a^k:k \in {\mathbb Z}$.

We show first that the cyclic subgroup generated by $a$ actually is a subgroup of $G$.

To show that it is a subgroup we need to show that it is closed under the operation:  this follows from the theorem $a^k \cdot a^l = a^{k+1}$:  if $x$ and $y$ are in the cyclic subgroup, then for some integers $k,l$, $x=a^k$, $y=a^l$, and so $x\cdot y = a^{k+l}$, and so $x\cdot y$ is a ``power: of $a$ and belongs to the cyclic subgroup.   We need to show that it contains the identity:  $a^0 = e$.  We need to show that if $x$ is in the subgroup, so is $x^{-1}$:  If $x=a^k$, $x^{-1} = a^{-k}$ and so is also in the subgroup.

We argue that the cyclic subgroup generated by $a$ is a subset of every subgroup of $G$ which contains $a$.  The lemma needed for this is that if $a \in H$, a subgroup of $G$, it follows that $a^k \in H$ for each integer $k$.  You should be able to prove this lemma (by induction for positive integers and negative integers separately, with attention to how it is defined for negative integers).  I rewrote my formal definition of powers in groups above to facilitate this.

We say that the order of a group is simply the size of the group (the cardinality of the set of group elements).  We say that the order of an element of a group is the order of the cyclic subgroup it generates.  We say that a group $G$ is a cyclic group iff it has an element $g$ such that the cyclic subgroup of $G$ generated by $g$ is $g$ itself.  We then call $g$ a generator of $G$.  If $G$ is a finite group, $g \in G$ is a generator of $g$ iff the order of $g$ in $G$ is $|G|$.  This isn't true for infinite groups:  the cyclic subgroup of the integers generated by 2 is the same size as the set of integers, but it is not the whole set, so 2 is not a generator of the integers.


Every cyclic group is abelian.  This is because $a^k \cdot a^l = a^{k+l} = a^{l+k} = a^l \cdot a^k$.  (I do suggest proving the exponent properties of powers by induction from the definition on your own).

This means immediately that not all groups are cyclic.  In particular, no nonabelian group can be cyclic.  We have already seen an example $U(8)$ of an abelian group which is not cyclic.

Every subgroup of a cyclic group is cyclic.  Let $G$ be a cyclic group with generator $g$.  Let $H$ be a subgroup of $G$.  If $H$ contains no element of $G$ except the identity, $H$ is cyclic.  If $H$ contains an element of $G$ which is not the identity, there is a $g^k \in H$ with $k \neq 0$.  $g^{-k}$ is also in $H$, and one of $k$ and $-k$ is positive, so there is a smallest positive $n$ such that $g^n \in H$.  Now for any $g^k \in H$, $k = nq+r$ for some integer $q$ and nonnegative integer $r<n$.  $g^k = g^{nq+r} = (g^n)^qg^t \in H$.  But also $g^{nq}$, and so its inverse, are in $H$, because $H$ includes the cyclic subgroup generated by $g^n$, so $g^r \in H$, so $r=0$ (no $g^r$ for $0<r<n$ belongs to $H$), so in fact $H$ is exactly the cyclic subgroup.  generated by $g^n$.   Notice here and in following results that number theory is allowing us to prove theorems in algebra.

The elements of a cyclic group of order $n$ with generator $a$ are exactly the $a^k$ with $0 \leq k \leq n-1$.  It cannot be the case that
all $a^k$ with $0 \leq k \leq n$ are distinct, because the group has order $n$ and this would give $n+1$ distinct elements.  So we must have
$0 \leq k < l \leq n$ such that $a^k = a^l$.  Notice that $a^k = a^l$ implies that $a^{l-k} = e$, and of course $l-k \leq n$,  We show that in fact $l-k = n$, and so $0 = k$ and $l=n$ (and all of  $0 \leq k \leq n-1$ are distinct, and so make up the entire group).  For any $a^m$, there are $q$ and $r$ such that $0 \leq r <l-k$ such that $a^m = a^{(l-k)q +r} = a^r$ (because $a^{l-k}=e)$.  But this means that every element of the group is of the form
$a^r$ for some $r$ with $0 \leq r <l-k$ so there are no more than $l-k$ elements of the group, so in fact $l-k=n$, so $l=n$ and $k=0$.

It follows that if $a^k = e$, we can argue that $a^k = a^{qn+r} = a^r$ for some $q$ and some $r$ with $0 \leq r <n$, (by the division algorithm) and of these $a^e=e$ only if $r=0$. so we have shown that $a^k = e$ exactly if $n|k$.

This is not how I did this in class on March 1 (you may recall that I was feeling strange at the time), but this does it neatly.  Notice that I have in effect argued that every cyclic group of order $n$ has the same structure as the modular addition group ${\mathbb Z}_n$.

This also gives a quick proof of a result the book appears to prove in a different way:  if $G$ is of (finite) order $n$, with $a$ as a generator, the order of $a^k$ is
$\frac n{{\tt gcd}(n,k)}$.  The elements of the subgroup are exactly the elements $a^{kx + ny}$ with $0 \leq kx+ny <n$, by what we have already shown.
Every element of the cyclic subgroup is of the form $a^{kx}$, of course, and equal to $a^{kx+ny}$ for every $y$, and one of the values $kx+ny$ will be in the range from 0 to $n-1$ inclusive by division algorithm.  So a generator of this group will be $a^d$ where $d$ is the smallest possible positive
$kx+ny$, that is, ${\tt gcd}(n,k)$, so the elements of $G$ belonging to the subgroup are exactly the multiples of ${\tt gcd}(n,k)$ in $[0,n-1]$, and there are $\frac n{{\tt gcd}(n,k)}$ of these.

It is not difficult to show that the nonzero complex numbers with multiplication are a group.  You can look at the discussion in the chapter for details.

What really interests us is the subgroup of ${\mathbb C}^*$ consisting of $a+bi$ with $a^2+b^2=1$.

Each of these numbers is of the form $\cos(\theta) + i\sin(\theta)$ for some $\theta$ (and then for every $\theta+2k\pi$, since sine and cosine are periodic.
This is straightforward from trigonometry.

You can verify using addition identities for sine and cosine that $(\cos(\theta) + i\sin(\theta))(\cos(\phi) + i\sin(\phi)) = \cos(\theta+\phi) + i\sin(\theta+\phi)$.  If you do not know this already, please take the time to verify it.  It is enormously useful in calculus and differential equations.

This group is called the circle group.  It is not a cyclic group, for an interesting technical reason:  the cyclic subgroup generated by any element of any grup at all can be at most the size of the set of integers, and the set of real numbers in the interval $[0,2\pi)$, which is the same size as the circle group, is larger than the set of integers.

However, it contains subgroups isomorphic to every cyclic group.  $(\cos(\theta) + i\sin(\theta))^n = \cos(n\theta) + i\sin(n\theta)$ for each $n \in \mathbb Z$.  It follows that $(\cos(\frac{2\pi}n) + i\sin(\frac{2\pi}n))^n = 1$, and a little thought shows that this actually is of order $n$ (no smaller power will be 1).   $(\cos(\frac{m2\pi}n) + i\sin(\frac{m2\pi}n))^n = 1$ will be true, and the order of this element will be ${\tt gcd}(m,n)$.  There are $n$ distinct $n$th roots of 1 in the circle group, the values of  $(\cos(\frac{m2\pi}n) + i\sin(\frac{m2\pi}n))^n$  for each $0 \leq m < n$;  of these,
the primitive $n$th roots of unity, the ones which are not also $k$th roots of unity for some positive $k<n$, are exactly those with ${\tt gcd}(m,n)=1$.  This should give you enough information to count the primitive $n$th roots of unity for small $n$.

Now suppose that $\cos(\theta) + i\sin(\theta)^n = 1$ for some integer $n$.  We argue that $\theta$ must be a rational multiple of $\frac{2\pi}n$.  If $\theta$ is negative, we can replace $\theta$ with positive $-\theta$.  We claim that in fact $\theta$ must be an integer multiple of $\frac{2\pi}n$.  If it isn't we can choose the largest $m$ which leaves $\theta - \frac{m2\pi}n$ positive (which leaves it less than $\frac{2\pi}n$).  $(\cos(\theta -\frac{m2\pi}n) + i\sin(\theta - \frac{m2\pi}n))^n$ would still be 1, but it is quite clear from the geometry of the unit circle that this cannot be the case.

This implies that for every irrational $r$, the order of $\cos(2\pi r) + i\sin(2\pi r))$ is infinite, so the cyclic group that it generates has the same structure as the group of integers under addition.  So every cyclic group has the same structure as some subgroup of the circle group:  it is a kind of universal object for cyclic groups.

\section{March 3: Permutation Groups}

We consider groups of permutations of finite sets.  We note that since we are really only interested in the formal structure of the groups,
we can replace consideration of a general group of permutations of a set $A$ of size $n$ with permutations of the set $\{1,\ldots,n\}$.

The permutation group $S_n$ has as its elements the bijections from $\{1,\ldots,n\}$ to $\{1,\ldots,n\}$, and composition of functions as its operation.
You are all aware that the composition of two bijections from $\{1,\ldots,n\}$ to $\{1,\ldots,n\}$ is a bijection from $\{1,\ldots,n\}$ to $\{1,\ldots,n\}$, that  the identity map on this set is a bijection of this kind, and that such bijections have inverse functions which are bijections of this kind, and the composition of a bijection and its inverse is the identity.

You also know that the group $S_n$ has $n!$ elements.  These are finite groups but rapidly become very large.

We note that any group of size $n$ is isomorphic to a subgroup of $S_n$ (which is much larger!).  We prove this, more economically than we did in class, but the idea is that same.  Let $G$ be a group with $n$ elements:  provide a bijection $g$ from $\{1,\ldots,n\}$ to $G$ (so the elements of $G$ are
exactly $g(0), g(1),\ldots,g(n)$).  Notice that for each element $a$ of $G$, $g^{-1}(a)$ is a number in $\{1,\ldots,n\}$. We define a map taking elements of $G$ to permutations in $S_n$:  if $a \in G$, we define $I(a)$ as the permutation which takes each $j \in \{1,\ldots,n\}$ to $g^{-1}(a \cdot g(j))$, where $\cdot$ is the group operation.  We compute $I(a) \circ I(b)$ of $j \in \{1,\ldots,n\}$ to show that we have an isomorphism.

$$(I(a) \circ I(b))(j) = I(a)(I(b)(j)) = $$ $$ I(a)(g^{-1}(b \cdot g(j))) = g^{-1}(a \cdot g(g^{-1}(b \cdot g(j)))) = g^{-1}(a \cdot (b \cdot g(j))) = g^{-1}((a \cdot b) \cdot g(j)) = I(a \cdot b)(j)$$

so $I(a) \cdot I(b) = I(a \cdot b)$

This shows (with obvious remarks about identity and inverses) that the permutations $I(a)$. which are basically correlated with columns in the multiplication table of $G$, make up a subgroup of $S_n$ with the same structure as $G$.

Now we talk about notations for permutation group calculations.

A standard notation for an element $f$ of $S_n$ is

$$\left(\begin{array}{ccc}

1 & \ldots & n \\
f(1) & \ldots & f(n)\end{array}\right)$$

with the elements of $\{1,\ldots,n\}$ listed above and the value of $f$ at each element listed below it.

In computing the composition of two permutations written in this notation (or any notation), one must remember that they have to be permutations of the same set, and that the second one is evaluated first at each number:  $(f\circ g)(x)$ is $f(g(x))$, you apply $g$ then you apply $f$, which may unfortunately seem backward.

Extended examples I leave to the book, or will produce on demand in class.  Typesetting this notation is nasty.

An alternative, more compact notation for permutations is cycle notation.  Where $a_1,\ldots,a_k$ are distinct elements of $\{1,\ldots,n\}$, we define the
$(a_1,\ldots,a_k)$ (I write commas here but their use is optional) to denote the permutation which sends $a_k$ to $a_1$ and each other $a_i$ to $a_{i+1}$, and fixes all elements of $\{1,\ldots,n\}$ which are not $a_i$'s.

A first remark is that disjoint cycle notations (where no number in one cycle appears in the other) commute.  This is straightforward to see:  applying the first cycle then the second or the second cycle then the first, one does the same thing, moving the numbers around the two cycles.  The change in order makes no difference because applying each cycle does not change the numbers in the other cycle.

Any permutation can be written as a composition of disjoint cycles.  Let $f$ be a permutation in $S_n$.  The first cycle in the composition will have
$a_i = f^i(1)$, where we define $f^0(x) = x$, $f({n+1}(x)= f(f^n(X))$.  The second cycle in the composition (if we didnt discover that $f$ is a cycle at the first step) is defined thus:  let $j$ be the first number in $\{1,\ldots,n\}$ which is not in the first cyle and define the second cycle by $b_i = f^i(j)$.   Continue until there are no numbers in $\{1,\ldots,n\}$ not included in cycles.  In a certain sense the deconposition into cycles is unique, mod the facts that we can rotate the number in each cycle ((1423) is the same as (4231), for example) and we can write the disjoint cycles in any order.

Computing where a permutation written in cycle notation sends each number is straightforward, and it is a skill you need to compute productes of permutations expressed in cycle notation, in cycle notation.  Remember always that in a composition you apply the second function, then the first.

A transposition is a cycle of length 2.  Previous results and the fact that $(a_1,\ldots,a_k) = (a_1a_k)(a_1a_{k-1})\ldots(a_1a_2)$ (check examples to see that this works and how it works) show that every permutation can be written as a product (composition) of transpositions.  This product is not unique.  But there is a Big Theorem about this.  We say that a permutation in $S_n$ is even if it can be written as a product of an even number of transpositions, and odd if it can be written as a product of an odd number of transpositions.  The theorem is that every transposition is either odd or even, and no transposition is both. [we remark that in a product of transpositions in which some transposition occurs more than once, we count each occurrence as a separate transposition for purposes of the definition of odd and even transposition].

The book proves this by giving a proof (not familiar to me until I lectured it yesterday) that the identity permutation is even, and not odd (so if it is written as a product of transpositions, the number of transpositions in the product must be even),

Given this result, it is straightforward to prove that no transposition can be both odd and even.  Suppose $f$ is both odd and even.  Then we can write $f$ in a form $f'$, a product of an even number of transpositions, and in a form $f''$, a product of an odd number of transpositions.  Writing $f''$ in reverse order gives an expressiom $f'''$ for $f^{-1}$ (each transposition is its own inverse, and multiplying inverses in the reverse order gives the inverse of a product).  Now $f'f'''$ would be a product of an odd number of transpositions giving the identity, contradicting the theorem.

I will give a proof of the theorem that the identity is not an odd permutation later when I find one that I really like, or when I make better friends with the one in the book.

This theorem implies that for each $n$, the collection of even permutations in $S_n$ makes up a group, called the alternating group $A_n$ of index $n$.



\section{Homework 7}

Homework 7: 4.5 exercises (starting on p. 55): 4adf, 11, 12, 13 (I dont think the conjecture is easy), 14 (this bears on a question asked in class), 23*, 24; 5.4 exercises starting on p. 71: 1ab, 2abcd, 3abc, 8, 6* starred questions are hard and optional. 

\section{Lagrange's Theorem}

\begin{description}

\item[Definition:]  Let $G$ be  a group.  Let $H$ be a subgroup of $G$.  Let $a \in G$.  We define $aH$, the left coset of $H$ with representative $a$, as $\{ah:h \in H\}$ and $Ha$, the right coset of $H$ with representative $a$,
as $\{ha:h \in H\}$.

\item[Lemma 1:]  The left cosets of a subgroup $H$ of a group $G$ are all of the same cardinality.

\item[Proof:]  Let $a,b \in G$.  A bijection from $aH$ to $bH$ is defined by $(x \in aH\mapsto ba^{-1}x)$.  Clearly if $x$ is in $aH$, $ba^{-1}x \in bH$:  this is a function from $aH$ to $bH$.  Equally clearly (cancellation property)
if $ba^{-1}x = ba^{-1}y$ then $x=y$:  the function is one to one.  Finally, if $bh \in bH$, then $bh = ba^{-1}ah$ is the image under this function of the element $ah$ of $aH$:  this function is one to one and onto $bH$, so $aH$ and $bH$ are the same size.

\item[Lemma 2:]  The collection of left cosets of a subgroup $H$ of a group $G$ make up a partition of $G$.

The cosets are subsets of $G$, and each coset $aH$ is nonempty because it contains $a=ae$ and $e \in H$.

We need to show that if two cosets meet, they are equal.  Suppose $c \in aH \cap bH$.  Then $c=ah_1$ and $c = bh_2$ for elements
$h_1,h_2$ of $H$.  It follows that $ah_1h_2^{-1} = b$ and $bh_2h_1^{-1}=a$, from which it follows that any $ah_3 \in aH$ is equal to $bh_2h_1^{-1}h_3$ and so is in $bH$,
and any $bh_3 \in aH$ is equal to $bah_1h_2^{-1}h_3$ and so is in $aH$, so these sets are equal.

We need to show that every element of $G$ is in a coset:  but we have already shown this above, as we know $a \in aH$.

\item[Definition:]  We define the index $[G:H]$ of a subgroup as the number of cosets of $H$.

\item[Theorem (Lagrange):]  Let $G$ be a group with a subgroup $H$.  Then $|G| = |H|[G:H]$.  If $|G|$ is finite then $|H|$ must be a divisor of $|G|$.

\item[Proof:]  Each left coset of $H$ is the same size and $H$ is the union of all the (pairwise disjoint) cosets, so the size of $G$ is the size of $H$ times the number $[G:H]$ of cosets, even if some of
these sizes are infinite.  If $|G|$ is finite then certainly $|H|$ and $[G:H]$ are finite and from $|G|=|H|[G:H]$ it follows that $|H|$ is a divisor of $|G|$.

\item[Remark:]  It is worth noticing that the left and right cosets of a subgroup may not be the same sets (if the group is nonabelian) but the number of left cosets and the number of right cosets will be the same.
Everything I have proved about left cosets can also be proved about right cosets (they are all the same size and they form a partition).

\item[Corollaries:]  It follows that if $G$ is a group of finite order $n$, and $d$ is the order of an element of $G$, then $d|n$.

\item[Definition:]  The Euler $\phi$ function is defined for each integer $n\geq 1$.  $\phi(n) = |\{m \in {\mathbb Z}:0 <m\leq n \wedge {\tt gcd}(m,n) = 1\}|$.

\item[Theorem (Euler):]  If ${\tt gcd}(m,n)=1$, then $m^{\phi(n)} \equiv_n 1$.

\item[Proof:]  The group $U(n)$ has $\phi(n)$ elements, so the order of each of its elements is a divisor of $\phi(n)$,
so for each $m \in U(n)$ we have $m^{\phi(n)}=1$ in the group $U(n)$ from which it follows that $m^{\phi(n)} \equiv_n 1$ in the integers.
Now for any $m \in {\mathbb Z}$ with ${\tt gcd}(m,n)=1$, we have $m {\tt mod} n \in U(n)$, ans $m^{\phi(n)} \equiv_n (m{\tt mod}n)^{\phi(n)} \equiv_n 1$.

\item[Fermat's Little Theorem:]  if $p$ is prime and $p|a$ then $a^{p-1}\equiv_p 1$.

\item[Proof:]  This follows immediately from Euler's Theorem and the fact that $\phi(p) = p-1$.

\end{description}


\section{Isomorphisms and Direct Products}

\begin{description}

\item[Definition:]  Let $G$ and $H$ be two groups, with operations $\circ_G$ and $\circ_H$.  We say that a function $I:G \rightarrow H$ if it is a bijection
from $G$ to $H$ (one-to-one and onto $H$) and it satisfies $I(a \circ_G b) = I(a) \circ_H I(b)$ for every $a,b \in G$.  If there is an isomorphism from $G$ to $H$
we say that $G$ and $H$ are isomorphic.

\item[Theorem:]  Isomorphism is an equivalence relation on groups.

\item[Observation:]  If two groups are isomorphic, they have the same abstract structure as groups.  In one mood, we may say that there are infinitely
many groups of order 4, because we can build such a group with any collection of four objects.  But in another mood, we say that there are only two groups of
order 4, because any group of order 4 is isomorphic to either ${\mathbb Z}_4$ or $U(8)$.  Proving this might take a little thought.  Certainly these two groups are not isomorphic to each other, because one contains an element of order 4
and one does not:  it should be easy to see that if $G$ is a group of order $n$ with an element of order $d$, and $G$ is isomorphic to $H$, then $H$ is a group of order $n$ (because an isomorphism is a bijection) and contains an element of order $d$ (the image of the element of order $d$ 
in $G$ under the isomorphism).

\item[Theorem (already proved):]  A cyclic group of order $n$ is isomorphic to ${\mathbb Z}_n$.

\item[Proof:]  I did say I already proved it, though I did not use the word isomorphic.  I showed that a cyclic group $G$ of order $n$ with generator $a$ has as its elements
exactly the powers $a^k$ for $0 \leq k \leq n-1$.  It is then evident that the map $(k \mapsto a^k$ from ${\mathbb Z}_n$ to $G$ is an isomorphism.  It is clearly a bijection,
and $a^{k+l {\tt mod}n} = a^{k+l} = a^ka^l$ establishes the essential property of an isomorphism.

\item[Theorem (already proved):]  For any group $G$ of order $n$, $S_n$ has a subgroup isomorphic to $G$.

\item[Proof:]  This proof appears in detail above.

\item[Definition:]  The direct product of groups $G$ and $H$ has as its underlying set the Cartesian product $G \times H = \{(g,h\}:g \in G \wedge h \in H\}$ and its operation is
defined by $(g,h) \circ_{G \times H} (g',h') = (g \circ_G g',h \circ_H h')$.  Of course, we will not be so careful to tag each multiplication with the name of its group in the usual applications.

\item[Theorem:]  If $g$ is of order $m$ in $G$ and $h$ is of order $n$ in $H$, then the order of $(g,h)$ in $G \times H$ is the least common multiple of $m$ and $n$.

\item[Theorem:]  Any group of prime order $p$ is cyclic, and so isomorphic to ${\mathbb Z}_p$.

\item[Proof:]  By Lagrange's Theorem, a group of prime order $p$ can only contain elements of order 1 and $p$, and it can have only one element of order 1,
so it must contain an element of order $p$, a generator, and so it is cyclic.

\item[Theorem:]  ${\mathbb Z}_m \times {\mathbb Z}_n$ is isomorphic to ${\mathbb Z}_{mn}$ if and only if $m$ and $n$ are relatively prime.

\item[Proof:]  Suppose $m$ and $n$ are relatively prime. Choose a generator $g$ of $G$ and a generator $h$ of $H$.  The order of $(g,h)$ will be the least common multiple of $m$ and $n$,
which is $mn$, and so $(g,h)$ is a generator, and so the group is cyclic and isomorphic to ${\mathbb Z}_{mn}$.

Suppose that ${\mathbb Z}_m \times {\mathbb Z}_n$ is isomorphic to ${\mathbb Z}_{mn}$.  Then the product group contains an element $(g,h)$ of order $mn$.  The order of $g$ must be $m$
and the order of $h$ must be $n$, or the order of $(g,h)$ will be less than $mn$.  But even if the order of $g$ is $m$ and the order of $h$ is $n$, the order of $(g,h)$ will be the least common multiple of $m$ and $n$, which can only be equal to $mn$ if $m$ and $n$ are
relatively prime.

\end{description}

\section{Homework 8}

Homework 8: 6.5 exercises starting on p. 78: 1, 3/4 are a single problem, in effect (index of a subgroup is the number of cosets); 5acdh, 8, 16 (hard? Im not sure), 23*, Chapter 9 exercises starting p. 121 (sorry left that out): 2, 3, 5, 9 (this may be hard or not: the question is whether you see the isomorphism), 11, 16. The * on problem 23 means hard and optional.

\section{March 15 lecture:  how many groups of order 4?  Internal direct products?}

I started the lecture with the question, how many groups of order 4 are there?

Of course, one answer is, infinitely many, because we could take the group table for a single order four group (say ${\tt mathbb Z}_4$ and replace the four elements of the group with any arbitrary natural numbers, for example and get a different concrete group.

The answer we are looking for is not the number of concrete groups, but the number of equivalence classes of groups under the relation of being isomorphic.  We will show that there are two of them.

We could do this by exhaustive search of all possible group multiplication tables on four fixed objects.  Ick.  We do something more intelligent.

Lagrange's theorem tells us that the only possible orders for an element of a four element group are 1,2, and 4.
There will only be one order 1 element, the identity.  

If there is an order 4 element, then the group is cyclic,
and so by things we have proved, isomorphic to ${\mathbb Z}_4$, the addition group mod 4.  We have found one of our isomorphism classes of groups mod 4.  

So  suppose there is no element of order 4.  It follows that every element $a$ of the group satisfies $a^2 = e$,
and from a homework problem we have that this implies that the group must be abelian.   One way to see this is that if $a$ and $b$ are any two elements, we have $(ab)^2 = abab= e = a^2b^2 = aabb$, and right and left cancellation on
$abab=aabb$ give $ba=ab$, but $a,b$ were any elements of the group at all.

Now consider a group of order 4.  Let 1 be the identity and let $a,b$ be two distinct ones of the order 2 elements.  $ab$ cannot be equal to 1, $a$, or $b$ (verify this) and so it must be the fourth element.  And we have enough information now to complete the operation table, using the facts that $a^2-b^2=e$ and $ab=ba$ (I'll ask you to finish all these details as a homework problem), so in fact there is only one group of order 4 which is not cyclic, up to isomorphism, which we have already seen:  $U(8)$ is of this kind.

Then I looked at but did not completely answer the question of what groups of order 6 there can be.

A group of order 6 can have order 1,2,3, or 6 elements.  If it has an order 6 element we know what it is (it is the mod 6 addition group).

In analysis of the other case, I got as far as pointing out that a group of order 6 must have an order 3 element.
If it didn't, it would have elements of orders 1 and 2 only, so it would be abelian.  Let $a$ and $b$ be two distinct order two elements.  Then $ab$ is another order 2 element.  Now there has to be another order 2 element $c$.  Then it is straightforward to determine that $a,b,ab,c,ac,bc,abc$ are all distinct and actually to compute all their products, and we see that a group with only order 1 and 2 elements has at least 8 elements if it has at least 5 elements, and certainly will not have six elements.

Then I discussed the theorem on internal direct products:  a group with certain special subgroups will be isomorphic to the direct product of those subgroups.

Let $G$ be a group with subgroups $H,K$ such that 

\begin{enumerate}

\item  $G = \{hk:h \in H \wedge k \in K\}$:  every element of $G$ is the product of an element of $H$ and an element of $K$.

\item $H \cap K = \{e\}$:  the only common element of the subgroups $H$ and $K$ is the identity element of $G$.

\item $(\forall h \in H:\forall k \in K:hk=kh)$:  elements of $H$ and $K$ commute with each other.  Notice that
this does not say that $G$ is abelian!

\end{enumerate}

We claim that these conditions imply that the function $I$ from $H \times K$ to $G$ defined by $I(h,k) = hk$ is
an isomorphism.

We need to show that the map $I$ is a bijction.  It is onto $G$ because we know that every element $g$ is $hk$ for
some $h \in H$ and $k \in K$, and so is $I(h,k) = g$ for that choice of $h$ and $k$.

Now we show that it is one to one.  If $I(h,k)=I(h',k')$ then $hk = h'k'$ so $hk(k')^{-1} = h'$ so $k(k')^{-1}= h^{-1}h'$ and since $k(k')^{-1}= h^{-1}h'$ belongs to both $H$ and $K$ it is the identity, so in fact $k(k')^{-1}= h^{-1}h'= e$ from which it follows that $h=h'$ and $k=k'$ so $(h,k)=(h',k')$ so $I$ is one-to-one (since this is shown for {\em any\/} $(h,k)$ and $(h',k')$.

We also need to verify $I((h,k)\cdot(h',k')) = I(h,k)\cdot I(h,k)$.  You should identify the multiplication operation on each side of that equation for yourself.  We give the computation:  make sure you follow it.

$I((h,k)\cdot(h',k')) = I((hh',kk')) = hh'kk' = hkh'k' = I(h,k)\cdot I(h',k')$.  I'm thinking of a homework problem of annotating each of the equations in this chain with its justification.

\section{March 17 lecture:  cosets in direct products, defining factor groups, then normal subgroups (the reverse of the usual procedure, but perhaps better motivated}

I thought about my unsatisfactory attempt to introduce normal subgroups and factor groups at the end of the March 15 lecture and came up with my own approach to presenting this material, in a somewhat different order.  The lecture fell into three parts:  I first analyzed cosets in direct products and pointed out that in a certain sense they formed a group.  Then I defined factor groups and used a definition of factor groups to motivate a definition of normal subgroups and a verification of their properties.  The definitions are not the same ones given in the book, but it should be clear that they are equivalent.  Then I gave a brief overview of the result characterizing finite abelian groups as direct products of cyclic groups, and briefly presented a proof of one of the lemmas.

First topic:  if  $G$ and $H$ are groups, the direct product group $G \times H$ has special subgroups $G \times \{e_H\}$ and $\{e_G\} \times H$.  You can verify that these subgroups satisfy the conditions of the internal direct product theorem (homework problem);  what I do here with these subgroups is a little different.

Let $(a,b)$ be an element of $G \times H$.  The left coset $(a,b)(G \times \{e_H\})$ is equal to the set of all
$(ag,b)$ for $g \in G$, which is exactly $G \times \{b\}$.  Similarly the left cosets of $\{e_G\} \times H$ are sets
of the form $\{a\} \times H$ for an $a \in G$.

A natural candidate for a group operation on a set of group elements is (where $A,B \subseteq G$) $AB = \{ab:a \in A \wedge b \in B\}$.

Now obvserve that $(a,b)(G \times \{e_H\}) \cdot (a',b')(G \times \{e_H\}) = (G \times \{b\})\cdot (G \times \{b'\}) = G \times \{bb'\}$, which is itself a left coset (of $(aa',bb')$ for example).  The left cosets of $(G \times \{e_H\})$ make up a group under the indicated multiplication operation, which is actually isomorphic to $H$ (via the map sending $b \in H$ to $G \times \{b\}$.  Similar remarks apply to the left cosets of $\{e_G\} \times H$.

Now I abstract from this situation.

\begin{description}
\item[Definition:]  Let $G$ be a group and let $H$ be a subgroup of $G$.  Let the factor group $G/H$ be defined
as the set of left cosets of $H$ in $G$ with the multiplication operation on sets introduced above {\em if this is a group\/}.

\item[Observation:]  There is no particular reason to expect that $G/H$ thus defined actually is a group;  specific consequences about $H$ follow from it being a group, which will be the basis for our definition of the notion of normal subgroup.

\item[Observation:]  I showed above that $(G \times H)/(G \times \{e_H\})$ exists and is isomorphic to $H$.

\item[Discussion:]  We discuss the properties of $H$ which are consequences of the existence of $G/H$ as a group.
To begin with, observe that if $G/H$ is a group, it follows that $(aH)(bH)$ is a left coset of $H$ for each $a,b \in G$, and in fact it must be $(ab)H$.

This means that for any $a,b \in G$ and $h_1, h_2 \in H$ there is $h_3$ such that $ah_1bh_2 = abh_3$.  Now suppose that $a$ and $h_2$ are specified as $e$.  For any $h_1 \in H$ there is $h_3 \in H$ such that
$h_1b = bh_3$, from which it follows that $b^{-1}h_1b$ is in $H$ for any $b \in G$ (and, replacing $b$ with $a^{-1}$, $ah_1a^{-1} \in H$ for any $h_1 ]in H$, that is, $aHa^{-1} \subseteq H$.

We then observe that this condition implies that $aHa^{-1} = H$, by using the calculation $H = a(a^{-1}Ha)a^{-1} \subseteq aHa^{-1}$.

Further, not only does $G/H$ being a group imply that for every $a \in G$, $aHa^{-1} = H$, but the converse holds as well.

Suppose that $aHa^{-1} = H$ for every $a \in G$.  It follows that $aHbH = abHb^{-1}bH = abHH = abH$, so
the left cosets are closed under set multiplication.   This is enough to establish that the cosets make up a group:  that associativity and the existence of identity and inverse for set multiplication on the left cosets follow from associativity and identity and inverses for the underlying group is straightforward to establish.

\item[Definition:]  We say that a subgroup $H$ of a group $G$ is normal iff for every $a \in G$, $aHa^{-1} = H$.
This is the exact condition required for $G/H$ to be a group.

\item[verification of Judson's definition:]  We show that $H$ is normal if and only if its left and right cosets are the same sets.

If $aHa^{-1}=H$ for every $a \in G$, then $aH = (aHa^{-1})a = Ha$.

If $aH = Ha$ for every $a\in G$, then $aHa^{-1} = Haa^{-1} = H$ for every $a \in G$.


\end{description}

\section{Homework 9}

problems starting on p. 121:  12 (hint:  think of orders of elements of $S_4$ and $D_{12}$);  15, 30, 34 (the thing here is to make sure you know what the things are that you have to prove;  I do not think that any of them are difficult to prove);  41;  problems starting p. 130:  2, 4, 5, 13ac.   

``$I((h,k)\cdot(h',k')) = I((hh',kk')) = hh'kk' = hkh'k' = I(h,k)\cdot I(h',k')$.  I'm thinking of a homework problem of annotating each of the equations in this chain with its justification.":  I quote from the proof of the internal direct product theorem above.  Supply justifications for each of the steps in this calculation.

Verify that the special subgroups $G \times \{e_H\}$ and $\{e_G\} \times H$ satisfy the conditions on subgroups $H$ and $K$ expressed in the statement of the internal direct product theorem.

\section{Computation of the Euler $\phi$ function}

This is a section taken from my advanced number theory notes, with some minor revisions.  It was lectured straight from this text.

In any of the notes I am adding from the advanced number theory notes (or any notes I write!) feel very free to ask for clarification or point out possible errors.

The formal definition of the Euler $\phi$ function is

\begin{description}

\item[Definition:]  For any natural number $m$, $$\phi(m)=|\{a : 0<a<m \wedge {\tt gcd}(a,m)=1\}|,$$ the size of the set of remainders mod $m$ which are relatively prime to $m$.

\end{description}

It is possible to compute the value of $\phi(m)$ given the prime factorization of $\phi$.   The reason for this is that it is easy to compute $\phi(p^k)$ and $\phi$ has a number-theoretic property which is important enough to have a name:

\begin{description}

\item[Definition:]   A function $f$ with domain the natural numbers is {\em multiplicative\/} iff $f(mn)=f(m)f(n)$ whenever ${\tt gcd}(m,n)=1$.

\item[Lemma 1:]   $\phi(p^k)= p^k-p^{k-1}$ for any prime $p$ and natural number $k$.

\item[Proof:]   The remainders mod $p^k$ which are not relatively prime to $p^k$ are the ones which are divisible by $p$, and there are $p^{k-1}$ of these, so there
are $p^k-p^{k-1}$ remainders mod $p^k$ which are relatively prime to $p^k$.

\item[Lemma 2:]  $\phi$ is multiplicative.

\item[Proof]:   Let ${\tt gcd}(m,n)=1$.   We show the result by exhibiting a set of size $\phi(m,n)$ and a set of size $\phi(m)\phi(n)$ and a bijection between them.

The set of size $\phi(mn)$ is the obvious one, the set of remainders mod $mn$ which are relatively prime to $mn$.

The set of size $\phi(m)\phi(n)$ is also the obvious one:  it is the set of all pairs $(b,c)$ where $b$ is a remainder mod $m$ which is relatively prime to $m$ (there are $\phi(m)$ of these) and $c$ is a remainder mod $n$ which is relatively prime to $n$ (there are $\phi(n)$ of these).  This set is of size  $\phi(m)\phi(n)$ because it is the cartesian product of a set of size $\phi(m)$ and a set of size $\phi(n)$.

The function which we claim is a bijection is the map which sends a remainder $a$ in the first set to the pair $(a{\tt mod}m,a{\tt mod}n)$.   Notice that if $a{\tt mod}m$ had a common factor with $m$, $a$ would have a common factor with $m$, and so $a$ would have a common factor with $mn$, which is impossible, and similarly $a{\tt mod}n$ cannot have a common factor with $n$:  if $a$ is in the first set, $F(a)$ is in the second set.

Now we argue that $F$ is one-to-one:   suppose $F(a)=F(a')$:  we need to show that $a=a'$ follows.   Since $F(a)=F(a')$ we have $a{\tt mod}m=a'{\tt mod}m$ and $a{\tt mod}n=a'{\tt mod}n$, so $a-a'$ is divisible by both $m$ and $n$.
But since $m$ and $n$ are relatively prime, this implies that $a-a'$ is divisible by $mn$, and since it is less than $mn$ in absolute value it must be 0, so $a=a'$.

Now we argue that $F$ is onto:   Given $(b,c)$ in the second set, we want to find $a$ in the first set such that $F(a)=(b,c)$, that is, $a{\tt mod}m=b$ and $a{\tt mod}n=c$.   We show that this pair of simultaneous equations has a solution.
Because $a{\tt mod}m=b$, we can write $b+mk=a$.   We then need to solve $b+mk{\tt mod}n=c$, for which it is sufficient to solve $mk\equiv c-b\,{\tt mod}\,n$.   Let $m^{-1}{\tt mod}n$ denote the multiplicative inverse of $m$ in mod $n$ arithmetic (which exists because $m$ and $n$ are relatively prime).  We can then solve for $k$ as $(m^{-1}{\tt mod}n)(c-b)$ and for $x$ as $b+m(m^{-1}{\tt mod}n)(c-b)$, which is clearly congruent to $b$ mod $m$ and to $b+1(c-b)=c$ mod $n$.

The result that $F$ is a bijection is usually called the Chinese Remainder Theorem, and we state it it separately (we have already proved it earlier):

\newpage

\item[Chinese Remainder Theorem:]   If ${\tt gcd}(m,n)=1$, then any system of equations

$$x\equiv a\,{\tt mod}\,m$$

$$x \equiv b\,{\tt mod}\,n$$

has a unique solution up to congruence mod $mn$.

\end{description}

It should now be clear that we can compute $\phi(m)$ for any $m$ directly from the prime factorization of $m$.   Note the method for computing $\phi(m)$ which you are supposed to verify in problem 11.3:
if $p_1,\ldots,p_n$ are the primes going into $m$, $\phi(m)=m(1-\frac1{p_1})(1-\frac1{p_2})\ldots(1-\frac1{p_n})$.

\section{The Primitive Roots Theorem (groups $U(p)$ have generators).}

Again, this is a section from my advanced number theory notes, with some editing to connect it with our work in abstract algebra.

We interest ourselves in solutions to equations $a^n \equiv 1 {\tt mod} p$.

\begin{description}

\item[Definition:]  For any $a$, $m$ with ${\tt gcd}(a,p)=1$, we define the order of $a$ in mod $p$ arithmetic (which we will write $o(a)$, leaving
$p$ understood) as the smallest positive $k$ such that $a^k\equiv 1{\tt mod}p$.

Notice that this is the same thing as the order of $a$ in the group $U(p)$.

\item[Observation:]  There is such an $k$:  it is clear that for some two numbers $m>n$ we have $a^m\equiv a^n{\tt mod}p$ because there are only finitely many
possible remainders mod $p$, and then we have $a^m\equiv a^n(a^{m-n})\equiv a^n$, from which we have $a^{m-n}\equiv 1$ by multiplying both sides by the mod $p$ reciprocal of $a^n$.

\item[Lemma:]  If $a^n=1$ then $o(a)|n$.

\item[Proof of Lemma:]  By the division algorithm, $n=o(a)q+r$ for some $q$ and some $r<o(a)$.  Thus $a^n=a^{o(a)q+r}\equiv (a^{o(a)})^qa^r\equiv a^r$.
Since $a^r\equiv 1$, we must have $r=0$, because if $0<r<o(a)$ $a^r$ cannot be 1 by definition of $o(a)$.

\item[Corollary:]  By Fermat's Little Theorem, $o(a)|p-1$.

\item[Definition:]   For each $d$, define $\psi(d)$ as the number of remainders $a$ mod $p$ which are of order $d$.  We know that $\psi(d)=0$ if $d$ is not a divisor of
$p-1$.

\item[Observation:]   For any $n$, the number of roots of $x^n-1 \equiv 0{\tt mod}p$ among the remainders mod $p$ is the sum of $\psi(d)$ over all divisors of $n$.
This is true because any root of $x^n-1 \equiv 0$ has $x^n\equiv 1{\tt mod}p$ whence its degree goes into $n$.

\item[Lemma:]  For any $n|p-1$, $x^n-1 \equiv 0{\tt mod}p$  has exactly $n$ roots.

\item[Proof of Lemma:] $x^{p-1}-1 \equiv 0$ has exactly $p-1$ roots by Fermat's Little Theorem.   If $p-1=mn$ then $x^{p-1}-1=x^{mn}-1=(x^n-1)(x^{n(m-1)}+x^{n(m-2)}+\ldots+x^{n2}+x^n)$.   Since we are in arithmetic in a prime modulus,
we have the Zero Property, and any root of $x^{p-1}-1$ in mod $p$ arithmetic is a root of one of the two factors.  The first factor has no more than $n$ roots by our Polynomial Roots theorem for prime moduli; the second has no more
than $n(m-1)=nm-n=(p-1)-n$ roots.  But the total number of roots of the two factors must be exactly $p-1$, which is only possible of the first term has $n$ roots, the second has $(p-1)-n$, and they share no roots.

\item[Corollary of the Lemma:]  The sum for all $d|n$ of $\psi(d)$ is $n$, for any $n|(p-1)$.

\item[Observation:]   The assertion $\sum_{d|n}f(d)=n$ is actually a recursive definition of the function $f$ (if we assume it true for all $n$).   To see this, rewrite it as
$f(n)=n-\sum_{d|n \wedge d \neq n}f(d)$.  Notice that this allows us to evaluate $f(1)$ as $1-\sum_{d \in \emptyset}f(d)=1-0=1$:  this is the hardest case.   In every other case,
we can clearly compute $f(n)$ once we know all smaller values of $f$.   The condition above shows us that $\psi$ coincides with this function $f$ on all divisors of $p-1$.
So what is the function $f$?

\item[Theorem:]  $\sum_{d|n}\phi(d)=n$, where $\phi$ is the Euler $\phi$ function ($\phi(n)$ is the size of the collection of natural numbers in the interval $[1,n]$ which are relatively prime to $n$).

$\phi(1)=1$ and the sum over all the divisors $d$ of 1 if $\phi(d)$ is just $\phi(1)=1$.

For any prime $p$, the sum over all divisors $d$ of $p^k$ of $\phi(d)$ is $\phi(1) + \phi(p) + \phi(p^2)+\phi(p^3)+\ldots+\phi(p^k)=1+(p-1)+(p^2-p)+(p^3-p^2)+\ldots+(p^k-p^{k-1})$, which is a telescoping sum:  everything cancels but the final $p^k$ so the result holds for $p^k$.

If ${\tt gcd}(m,n)=1$, the  product of the sum over all divisors $d$ of $m$ of $\phi(d)$ and the sum over all divisors $d$ of $n$ of $\phi(d)$ is the sum over all divisors $d$ of $mn$ of $\phi(d)$:

$$\sum_{d|mn}\phi(d) =(1)  \sum_{d_1|m,d_2|n}\phi(d_1d_2) $$ $$=(2) \sum_{d_1|m,d_2|n}\phi(d_1)\phi(d_2) =(3) (\sum_{d_1|m}\phi(d_1))\cdot(\sum_{d_2|n}\phi(d_2))$$

Thus if we have $\sum_{d_1|m}\phi(d_1))=m$ and $\sum_{d_2|n}\phi(d_2)=n$ we have $\sum_{d|mn}\phi(d) =mn$.

Equation 1 holds because each $d|mn$ can be expressed as a product of $d_1|m$ and $d_2|n$ in exactly one way (I suggest writing out a proof of this).   Equation 2 holds because the Euler function itself
is multiplicative.   Equation 3 holds by many applications of the distributive law.

Anything which is true for 1, for powers of primes, and is true for the product of a pair of relatively prime numbers if it is true for the two numbers is true for all numbers (by the prime factorization theorem).

\item[Theorem:]   For each $d|p-1$, $\psi(d)=\phi(d)$.

\item[Proof of Theorem:]   This follows from the previous theorem and the preceding Observation directly.

\item[Definition:]  $a$ is a {\em primitive root\/} for $p$ iff the order of $a$ in mod $p$ arithmetic is $p-1$.   Notice that this implies that all nonzero remainders mod $p$ are powers of $a$ in mod $p$ arithmetic.

\item[Corollary (Primitive Root Theorem):]  There is a primitive root for every prime $p$.  In fact, there are $\phi(p-1)$ primitive roots for $p$.

\end{description}
\section{The RSA and El Gamal Cryptsystems}

This is a selection of material from my advanced number theory notes, again, the actual source material for my lectures.  If anything is obscure because I wrote it for a somewhat different audience, ask me!

We have seen above how we can very efficiently compute $a^k\,{\tt mod}\,m$ for any reasonable $a,k,m$ using repeated squaring.

We now look into finding $k$th roots, that is, solving equations $x^k=a\,{\tt mod}\,m$.   We will be able to do this in many interesting cases, but we will find that conditions are not ideal.

Suppose that ${\tt gcd}(a,m)=1$ and ${\tt gcd}(k,\phi(m))=1$.   We can find $s$ such that $sk {\tt mod} \phi(m)=1$ (multiplicative inverses exist in any modular arithmetic for numbers relatively prime to the modulus).
We claim that $a^s$ is a solution to our equation, a $k$th root of $a$ mod $m$.   For $(a^s)^k \,{\tt mod}\,m=a^{sk} \, {\tt mod} m = a^1 \,{\tt mod}\, m = a \,{\tt mod}\, m$.   The fact that ${\gcd}(a,m)=1$
 is used to ensure that $sk\equiv 1\,{\tt mod}\,\phi(m)$ implies $a^{sk} \equiv a^1\,{\tt mod}\, m$ by Euler's Theorem.

Efficient computation of these roots is blocked by the fact that $\phi(m)$ is not easy to find for large numbers, as prime factorizations are not easy to find.


We now describe the RSA algorithm.

The public information which you distribute for RSA encryption is a large number $N$ and an exponent $r$.   Individual blocks of a message to you are to be encrypted thus $M'=M^r\,{\tt mod}\,N$.   This can be computed by repeated squaring.

The secret information you have is this.   You know that $N=pq$, where $p,q$ are two large primes known only to you.   Thus $\phi(N)=(p-1)(q-1)$ is again known only to you.  You know that ${\tt gcd}(r,\phi(N))=1$.
You compute $s$ such that $rs=1$ mod $(p-1)(q-1)$ (multiplicative inverse property in modular arithmetic) and you can then decrypt messages because $M'^s\,{\tt mod}\,N = (M^r)^s \,{\tt mod}\,N=M^{rs}\,{\tt mod}\,N= M\,{\tt mod}\,N$ by Euler's theorem as above.

As long as it is fairly easy to find huge primes and very hard to factor a large number which is a product of two primes, this gives a secure method of communication.

I showed you computer tools to use to generate examples.

The El Gamal cryptosystem is an application of the primitive roots theorem.

Although for any odd prime $p$ there are $\phi(p-1)$ generators (a lot of them) it is actually rather hard to identify a generator for a general prime $p$.

The characteristic a generator needs to have is that $g^d \not\equiv_p 1$ for each proper divisor $d$ of $p-1$.  This is straightforward to check...if you know the factorization of $p-1$, and finding that is in general quite a hard problem.

However, it is easy to find a generator for an odd prime $p$ of the form $2q+1$ where $q$ is prime.  Such primes $p$ are called {\em safe primes\/}, and they are fairly easy to find computationally (in the vicinity of a large number $N$, about in in every $\ln(N)^2$ numbers will be a safe prime, or so we guess on the assumption that primes are distributed fairly randomly).

If $p$ is a safe prime, then $p-1=2q$.  A number $g$  between 2 and $p-2$ (inclusive) will be a generator iff $g^q \not\equiv_p 1$ (it will be $-1$, can you see why?).  $p-1$ is ruled out because it is the only residue mod $p$ which is of order 2.

An application of safe primes and the ability to compute generators for them is the El Gamal public key cryptosystem, which we now describe.

Let $p$ be a large safe prime (my son tells me that 700 digits is the current state of the art:  I think my Python functions might take a while to find one).

Let $g$ be a generator for $p$ (identified using the procedure above:  choose a random $g$ in $[2,p-2]$ and compute $g^q$ mod $p$ (recalling that $q=\frac{p-1}2$ is also prime):  you have a 50 percent chance of getting 1 ($g$ is not a generator) or -1 ($g$ is a generator!).

Choose a random exponent $x$ less than $p-1$.  $x$ is your private key.  Compute $y=g^x {\tt mod} p$ and supply $p, g$, and $y$ to the world as your public key.



If someone wishes to send you a message $M$, they compute a random number $r<p-1$ (which they use only once!) and send you the pair $(My^r {\tt mod} p,g^r{\tt mod}p)$.  Notice that all the ingredients used are things you have made known to the world.

The insertion of random trash into the encryption process (in the form of the exponent $r$) is something you don't see in the RSA scheme (though in fact it is often simply added by force);  it has the nice feature that if you encrypt the same text more than once, the ciphertext will look completely different.  One needs to never use the same exponent $r$ more than once:  from two ciphertexts known to have the same $r$ value, it is possible to extract some information about the ciphertexts.

When you receive a message $(M_1,M_2)$, you compute $(M_1)((M_2)^{x})^{-1}{\tt mod}p$, and lo and behold, this is $M$.

We verify this.  $(My^r)((g^r)^{x})^{-1}{\tt mod}p$ = $(M((g^x)^r)((g^r)^x)^{-1}{\tt mod}p$ = $M{\tt mod} p$ (presumably $M <p$ so this is simply $M$).  It is worth noting
that we can compute $(M_2^x)^{-1}$ by computing $M_2^{p-1-x}$:  it can all be done with modular exponentiation without a separate step of computing a modular inverse.  You should be able to verify this with some reasoning using Fermat's Little Theorem.

The reason that we believe this to be secure is that we believe that the {\em discrete logarithm problem\/} (finding $x$ given $g^x {\tt mod} p$) is hard for large $p$.  This is not a theorem:  it is what is currently believed.

It is known that if there is a fast method for decrypting RSA, there is also a fast method for decrypting El Gamal, and vice versa;  though the two systems seem quite different, their level of difficulty is closely related.

Sometimes this weekend (writing on 4/15/2022) I will post some examples.

\section{Homework 10, posted 4/15/2022 and due 4/21/2022}

This homework is due Thursday as usual.

\begin{enumerate}

\item  Compute $\phi(n)$ for each $n$ in the interval from 30 to 40 inclusive.  Of course, do not do this by listing all the numbers relatively prime to $n$!

Compute the billionth power of 43 in each of these moduli.  You should not be using repeated squaring to do this!

\item  Find a generator for $U(29)$ and list all its powers.

Use this list of powers to list {\em all\/} the generators.

On request I will do this for a different prime.

\item My RSA public key has $N=55, r=3$.

Encrypt the message 42 to me.

My secret, which you can't possibly guess, is that $N=(5)(11)$.

Determine my decryption exponent $s$.

Carry out the calculation I will do to decrypt your message.

(The numbers here are wonderfully small;  of course the cryptographic security is zip!)

\item  I supply full information for a tiny El Gamal key, $p=47, g=31, y = 9, x=44$.

Encrypt the universal message 42 to me, and decrypt it.  Show all work by hand (with calculator support).

Contrive your own method to choose the random exponent $r$ (dice?  The Python random function?).  Do highlight for me what your value of $r$ is.

\item Prepare your own RSA and El Gamal full keys (and of course keep the information so you can decrypt messages I send you).
Send me your public keys in email, I will send you a message using each key in email, and you decrypt it and tell me what I said, again in email.  These should use large numbers:  use the Python functions.  Sometime this weekend I will definitely add more comments to the Python file to make it easier to identify the right functions;  I may (not promising this) upgrade the text coding so that you can use messages not made entirely of lower case letters.  I have historically been lax about marking this kind of exercise:  this time it is a numbered exercise in this homework, and it will be marked.

\item Read the notes on the RSA algorithm and notice that we introduce it as a special case of taking $k$th roots.  Why is it impossible to take {\em square roots\/} using this method?

\end{enumerate}

\section{Carmichael numbers and the Rabin-Miller test}

The Fermat test for compositeness is this:  Let $n$ be a composite number.  Let $2 \leq a \leq p-2$.  If $a^{n-1} \not\equiv_n 1$, then $n$ is composite.

Notice that this tells us nothing about what the factors of $n$ are!

To check whether $n$ is prime, the strategy would be to choose a lot of values of $a$ to test (at random) since the test for each $a$ is fairly cheap.

Unfortunately, there are Carmichael numbers, which can give a decent impression of being prime using this test.

A Carmichael number is an odd composite $n$ such that for every $a$ with ${\tt gcd}(a,n)=1$, $a^{n-1}=1$.

If a Carmichael number has a small prime factor, the Fermat test as described above will find it.  But if it has only very large factors, so that almost all natural numbers less than it are relatively prime to it, the Fermat test won't be likely to catch it.

The smallest example of a Carmichael number is $561 = (3)(11)(17)$.  Notice that 560 is divisible by (3-1), (11-1), and (17-1).
This is what makes it a Carmichael number.

For any $a$ relatively prime to 561

$a^2 \equiv_3 1$

$a^{10} \equiv_{11} 1$

and $a^{16}\equiv_{17} 1$,

because $a$ will be relatively prime to each of 3,11,17, and these things follow by Fermat's little theorem.

But then 

$a^{560} \equiv_3 1$

$a^{560} \equiv_{11} 1$

and $a^{560} \equiv_{17} 1$,

because 560 is a multiple of 2,10, and 16.

Now by the Chinese Remainder Theorem there is only one solution to 

$x \equiv_3 1$

$x \equiv_{11} 1$

and $x \equiv_{17} 1$,

mod (3)(11)(17) = 561, which is obviously 1, so $a^{560} \equiv_{561} 1$.

Exactly the same argument shows that any $n = p_1p_2\ldots p_m$, a product of distinct primes, with
$(p_i-1)|(n-1)$ for each $n$, is a Carmichael number.  You should be able to write out this proof on your own,
using the proof for 561 as a model.

This is called Korselt's Criterion, and it is an exact description of all Carmichael numbers;  the additional piece of information for which I do not suggest a proof here (it is in the advanced number theory notes!) is that for any $n$ and prime $p$, if $p^2 |n$, $n$ is not a Carmichael number.

I proved in class, and I am assigning as a homework exercise, that if $6n+1$, $12n+1$, and $18n+1$ are all primes, then
$(6n+1)(12n+1)(18n+1)$ is a Carmichael number by Korselt's Criterion.

I have expressed doubt that it is necessary to do this in practice (because Carmichael numbers are quite rare) but it is usual to use a somewhat stronger test for compositeness, the Rabin Miller test.

\begin{description}

\item[Rabin-Miller Primality Test:]  If $n$ is an odd number, $n-1$ can be expressed as $2^kq$, where $q$ is odd, in only one way.

Let $a$ be a residue mod $n$.  If $a^q \not\equiv 1$ and $a^{2^iq} \not\equiv_n -1$ for $0 \leq i <k$, then $n$ is composite.  We say in this case that $a$ fails the Rabin-Miller test for $n$.

\item[Proof:]  Suppose that the conditions of the test hold and $n$ is prime.

If $n$ is prime, then $a^{2^kq} = a^{n-1} \equiv_n 1$ by Fermat's little theorem.

$a^q \not\equiv 1$ so there must be a last $i<k$ such that $a^{2^iq} \not\equiv_n 1$.  This means that
$a^{2^{i+1}q} = (a^{2^iq})^2 \equiv_n 1$.  But this implies that $a^{2^iq}\equiv_n -1$, because if $n$ is prime
the equation $x^2 \equiv_n 1$ has only two solutions, which is a contradiction to the stated conditions.

\end{description}

If $n$ is composite, it is provable that at least three-quarters of the residues mod $n$ will fail the Rabin-Miller test and show that $n$ is composite.  So if we choose 50 random $a$ and they do not fail the test, the chance that $n$ is actually prime is at most 1 in $4^{50}$, not a kind of random event we will encounter in practice.

If $n$ is composite and $a$ does not fail the test, we call $a$ a Rabin-Miller misleader.

\section{Quadratic residues;  there are infinitely many primes of the form $4n+1$;  finding square roots in prime moduli}

Let $p$ be an odd prime number, used as modulus throughout this section.

The integers in $[1,p-1]$ fall into two classes of the same size, the set $QR = \{x^2 {\tt mod} p:p \not\in x\}$, called the set of quadratic residues, and the set $NR = \{x {\tt mod} p:p \not| x  \wedge x \not\in QR\}$.

QR contains $x^2$ for each $x \in [1,p-1]$.  $x^2 \equiv_p y^2$ iff $(x-y)(x+y)\equiv_p 0$, and we know by Euclid's lemma that this implies either $p|x-y$, so $x \equiv_p y$, or $p|x+y$, so $x \equiv_p -y$.  There is no $x$ in $[1,p-1]$ so that
$x=-x$, so there are exactly two elements of $\{(x,x^2{\tt mod}p):x \in [1,p-1]\}$ with each second component, so the set of second components (the set of quadratic residues) has exactly $\frac{p-1}2$ elements.

NR contains all the elements of $[1,p-1]$ which are integers and not in QR, so it also has $\frac{p-1}2$ elements.

We give a simple criterion for membership in QR.

$x^\frac{p-1}2 \equiv 1$ iff $x \in QR$

Proof:  if $x \in QR$ then $x=y^2$ for some $y$ and $x^\frac{p-1}2 = y^{p-1} \equiv_p 1$ by Fermat's Little Theorem.

This means that we have identified $\frac{p-1}2$ distinct roots of $x^\frac{p-1}2 -1\equiv_p 0$.  It cannot have any more roots than that, so all $x \in NR$ have $x^\frac{p-1}2 \not\equiv_p 1$.  The only other possible value for $x^\frac{p-1}2$ is
$-1$, since it has to be a square root of 1.

Lemma:  $p-1 \in QR$ iff $p \equiv_4 1$

Proof:  if $p=4n+1$ then $\frac{p-1}2 = 2n$ is even.  Thus $(p-1)^\frac{p-1}2 \equiv (-1)^{2n} = 1$.  Thus $p-1 \in QR$.

If $p=4n+3$ then $\frac{p-1}2 = 2n+1$ is odd.  Thus $(p-1)^\frac{p-1}2 \equiv (-1)^{2n+1} = -1$.  Thus $p-1 \in NR$.

Theorem:  There are infinitely many primes of the form 4n+1.

Suppose there are only finitely many such primes, $p_1,\ldots,p_n$.

Let $P= \prod_{1 \leq i \leq n}p_i$.

Now consider $(2P)^2 +1$.  Let $q$ be any prime factor of this.  Observe that $(2P)^2 \equiv_q -1$, so $-1$ is a QR mod $q$, so $q$ is a prime of the form $4n+1$.  $q|p$ because $q$ has to be one of the $p_i$'s so $q|(2P)^2+1$ and $q|(2P)^2$, so $q|1$ which is absurd.

Solving quadratic equations in modular arithmetic is a goal of the course.  Solving them in general moduli is a hard problem.
But there is an algorithm for computing square roots mod $p$, which relies on what we have done so far in this lecture.

First, there is a fairly easy way to find square roots if $p = 4n+3$.  In this case we look at $x^\frac{p+1}4$.  Note that
because $p=4n+3$, $p+1$ is divisible by 4.  If in fact $x=y^2$, then $x^\frac{p-1}2=1$ and $(x^\frac{p+1}4)^2= x^\frac{p+1}2 = x^\frac{p-1}2x =x$, so $x^\frac{p+1}4$ is either $y$ or $-y$.  If $(x^\frac{p+1}4)^2 \neq x$ then $x \in NR$.

There is a general algorithm.  It looks like a cousin of the Rabin-Miller algorithm, though it is solving a different problem!

let $p-1=2^kq$ with $q$ odd.

Try $R=x^\frac{q+1}2$.

$R^2 = x^{q}x$ so if $x^q=1$, we have found a square root $R$ of $x$.


Let $t=x^q$.  Notice that $t^{2^{k-1}}= x^{q2^{k-1}} = x^\frac{p-1}2$ if $x$ is a QR.

and we have $R^2=xt$ and $t$ is a $2^{k-1}$ root of 1.

Now we show that if we have $R^2 = xt$ where $t$ is a $2^i$ root of 1, we can get
$R'^2 = xt'$ where $t'$ is a $2^{i-1}$ root.  By repeating this, we can get to $i=0, t=1$, and then our new $R$ will be a square root of $x$.


we can check whether $t$ is already a $2^{i-1}$ root.  If it is not, it is a $2^{i-1}$ root of $-1$.

If it is not, we multiply $R$ by a factor $b$ to be determined.  Our aim for $b$ is that $b^2$ is a $2^{i-1}$ root of -1,
so that  $(Rb)^2 = xtb^2$ will give us what we need.  So we need $b$ to be a $2^{i}$ root of $-1$.  How do we find
such a thing?  Any element $z$ of NR (about 50 percent of residues, find one by random choice enough times) will
satisfy $z^{2^{k-1}q} = -1$, so $z^q$ is a $2^{k-1}$ root of -1, and $z^{q2^{(k-1)-(i)}}$ is a $2^{i}$th root of $-1$.

This describes an actual algorithm for computing square roots in prime moduli.

Computing square roots in composite moduli seems to depend on knowing their prime factorizations (and using the Chinese Remainder Theorem).  There are cryptosystems based on the difficulty of finding square roots in composite moduli.

As for solving quadratic equations in general, notice that the quadratic formula makes perfect sense in an odd prime modulus, as long as the leading coefficient has a multiplicative inverse.

I'm going to write a computer program implementing this and fill in the details here of the example I failed to complete in class (find the square root of 28 in mod 137 arithmetic).


\section{Homework 11}

I'll be happy to answer questions in this set in full on Thursday.

\begin{description}

\item  Write out the proof that if $n=p_1p_2\ldots p_n$, a product of distinct primes, and $(p_i-1)|(n-1)$ for each $i$, then
$n$ is a Carmichael number.  All the details are in the proof that 561 is a Carmichael number;  write it out in full abstraction.

\item  Prove that if $6n+1$, $12n+1$, and $18n+1$ are all primes, that $(6n+1)(12n+1)(18n+1)$ is a Carmichael number (using the result stated in problem 1).  Find two Carmichael numbers of this kind.

\item  Find two Rabin Miller misleaders for 49.  Show the calculations (of course, you may use my spreadsheet or my Python package to find them).  Find at least one misleader for 121:  there seem to be several.

\item  Determine whether 3 is a square in mod 137 arithmetic, and whether 103 is a square in mod 137 arithmetic, by carrying out a single modular exponentiation (not by looking for a square root).

\item  Find the square roots of 103 in mod 211 arithmetic using the method described in the notes (noting that 211 is of the form $4n+3$).  Show calculations to indicate that you understand the method.

Solve the quadratic equation $x^2-3x+8=0$ in mod 211 arithmetic, using the quadratic formula as usual.  You will need to use the same method as in the first part of this problem to find the square root.  Find both solutions.

\item  Find the four square roots of 15 in mod 187 = (11)(17) arithmetic.  Do this by  finding the two square roots of 15 (really 4) in mod 11 arithmetic (easy), then find the two square roots of 15 mod 17 (you can do this by trial and error), then applying the Chinese Remainder Theorem.



\item  I might add a question when I have completed the computer implementation of the general algorithm for finding square roots in prime moduli.  Watch this space.



\end{description}


\end{document}