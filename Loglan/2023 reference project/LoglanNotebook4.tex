\documentclass[12pt]{article}

\usepackage{comment}

\title{Notebook 4:  A complete account of the Loglan language as curated by Randall Holmes and others starting in 2013 (draft in progress, see version notes)
}

\author{Randall Holmes (and maybe others)}

\begin{document}

\maketitle

\tableofcontents

\newpage

\subsection{Version Notes}

\begin{description}

\item[5/10/2023:]  Starting.

\end{description}

\newpage

\section{Introduction}

I am sitting down in 2023, ten years after I (and others) started working on the project of wholesale revision of the Loglan language as it was handed down to us.  I am setting out to write a self-contained account of where the language is and both say and show why it is interesting to study and perhaps use this intellectual construction.

Loglan is an artificial language, originally conceived by James Cooke Brown in 1955.  It was designed for a specific purpose, which to my knowledge has never been carried out, and which is not really part of my aims for the language (if I have any beyond contemplation).  The intention was to support an experimental test of the Sapir-Whorf hypothesis, that the language that a human being speaks constrains the way that they think.

The way that this purpose drove the design of the language is described by Brown (for example in the 1975 version of Loglan I) and in fact makes good sense.  In order to be useful for an experimental test of the Sapir-Whorf hypothesis, the language needed to be small and easily learned (so that experimental subjects could learn it and try thinking in it) and extremely strange in some respect which could be engineered by the language designers (so that a Whorfian effect could be expected if there are such effects).

Brown claimed to have experimental evidence that the language was easily learned.  I cannot evaluate this after the lapse of time and death of witnesses to these experiments.  I can report that I have rather slowly learned the language:  some aspects of it I think are indeed fairly easy to learn, and others are deep and complicated.

The way in which Brown chose to make it strange was to make it extremely logical.  This again made sense, as progress in mathematical logic and computer programming meant that there was a good understanding of what a logical language might be like.  The language now exists, and I have some rather dry remarks about the exact senses in which it is logical (though it should be clear because I am writing this and directing the project that I think the project has value and interest).

First of all, there is a pun inherent in the claim that Loglan is logical.  It is logical in three different ways which are not necessarily related to each other.

It is to some extent designed to implement the machinery of first order predicate logic in a spoken language.  As we will detail in one of our essays, it does not do this in any way that a logician would have chosen to do, but it does do it to a considerable extent.  We will note briefly here, as an example (and a source of difficulty here and there) the structure of the Loglan sentence.  An atomic sentence has the form $P(x_1,\ldots,x_n)$ in logic, where the $P$ is the predicate (the verb, in this context) and the $x_i$'s are the arguments (noun phrases playing the grammatical roles of subject and objects).  In Loglan grammar, the default grammatical structure for such a sentence is SVO (a very common natural language approach), and the parse is $(x_1(P(x_2,\ldots,x_n)))$.  This gives the first argument (the subject) a very special role
(implicit in all Loglan work, more explicitly visible in our grammar) and further has the weird effect that the predicate (the verb) has the second and subsequent arguments (the objects) bound to it more tightly than the subject is bound to it.  In logical notation, the roles of the arguments are exactly coordinate;  not so in Loglan.  This is an example of the way Loglan differs from logical notation;  it isn't a criticism of the language as such.  It is also an example of something so deeply embedded in the design of the language as it has come down to us that it would be hard to change it without ripping everything up and starting afresh;  as I will say many times, my role is to curate an existing language, not to redesign it to make a better one.

The language is unambiguously parsable by computer.  This is a characteristic of logical notations, but has nothing essential to do with the first sense in which Loglan is logical.  It is, further, unambiguously parsable by computer (and one hopes by the human ear and brain) in two quite different senses.  Its phonetics and lexicography are unambiguous:  in principle samples of the the written and spoken forms of Loglan, if resolved successfully into phonemes and pauses,  should be unambiguously resolvable into ``words" in a general sense (with information about the lexical classes of these ``words").  Its grammar is unambiguous:  a written or spoken sample of Loglan should parse in just one way.

Loglan as it was handed down to us came with a parser (LIP) which would parse sentences using a BNF grammar which was provably unambiguous.  The software used cues in words to classify them phonetically, but it did not implement the official phonetic structure of the language fully and moreover attempts to test the phonetics carefully revealed bugs.  The sense in which the BNF grammar was unambiguous was qualified.  A method of proving an LALR(1) grammar unambiguous was used.
But Loglan grammar is not LALR(1):  a preprocessing step was needed.  Both the failure to fully implement the phonetics and the use of the preprocessor introduced ambiguity into the language (and it was unsatisfactory that strenuous tests of some language features led to parser crashes).

Loglan was actively developed from 1975 to 1983 (partly supported by a government grant).  This grant was not renewed and the project fell on hard times.  It was also decided to completely remodel the phonetics of the 1975 version, which occasioned considerable work.   In 1987, Notebook 3, a full account of all aspects of the language, was released, and in 1989 a new version of the book Loglan I was released, giving a baseline for ``the language as handed down to us" (along with computer based dictionaries:  no new paper dictionary was released along with the 1989 Loglan 1).  Sometime between 1983 and the release of Notebook 3, there was a schism in the project, and a separate, closely related but not mutually intelligible language came into being, which is called Lojban.  I have no comments to make about the language split;  I may occasionally have things to say about Lojban as a related language with related problems and opportunities.  The Loglan Institute is currently on good terms with the Logical Language Group.

The sources on which my work (with some collaboration from others) is based are Loglan 1,4,5, the Loglan book and dictionaries released in 1975, Notebook 3, the complete language description released in 1987, the trial.85 formal grammar on which LIP was built, the 1989 version of Loglan I, and the computer dictionaries (online versions of Loglan 4 and 5).  I have all issues of the Loglanist periodical which was produced at intervals from the late 1970s until 1983, and I have most issues of the periodical Lognet from the 80s and 90s, and Appendix H, which details changes to the Loglan described in 1989 which were made officially by the Loglan Academy of that time.  I also have Alex Leith's novella  {\em A Visit to Loglandia\/} and some other sample Loglan text.

I do not have a lot of things which a historian of the language would like to have.  I have very little documentary evidence of what went on from 1983 to 1989.  I would like to gather such documentation if anyone has it.  As I said above, I am not interested in the politics of the language split.  I am deeply interested in any documentation of the process of language building which still exists.  My interest in this is historical rather than linguistic:  I view Notebook 3 as the base language description from which the versions of 1989 and 2013+ spring, with the 1975 releases and the Loglanist volumes as partial history of the background.

James Cooke Brown died in 2000 and Alex Leith, who succeeded him as leader of the Loglan Institute, died shortly afterward, to be succeeded by Robert McIvor, the main architect of the Loglan formal grammar.  In 2008, McIvor tapped me as ``CEO" of the Loglan Institute (Brown's choice of title, not mine).

I had been the house logician in the 1990's but resigned (without acrimony) when Brown declined advice of mine about the role of sets in the language.  I met Brown in person once and I corresponded with McIvor on many occasions.  Unfortunately, a lot of relevant email disappeared in the collapse of Netscape (along with much of my early professional email).

In 2008 I became leader of the language (I certainly don't regard myself as owner:  I don't think a language can be owned), without really being able to speak it.
I did have a good understanding of the outlines of the formal grammar and implementation of logic, hampered by the unsatisfactory opacity of LIP.  The old parser
had unreadable internals (except for the BNF grammar) and its output was unreadable (human beings cannot make good use of parses which require matching many levels of nested parentheses).

From 2008 to the present I have met almost every week with a few people in the virtual world Second Life to talk in and about Loglan.  In this way, I have acquired some ability to speak the language, mostly in text rather than voice.

In 2013, I heard a rumor that someone had used PEG (Parsing Expression Grammar) machinery to parse Lojban from the level of letters up.  I decided to do this for Loglan.  This required me to develop what I think was the first formal parser for TLI Loglan phonetics (I think something at least approximating this had already been done in Lojban), which at the same time incorporated the grammar of Loglan as expressed in trial.85 (wth some amendments).

This gave us a language definition which was for the first time fully accessible and testable at all levels.

This enabled me to address some major problems on the periphery of the language definition which were already recognized as problems in the 1990's.  Notable instances are the problem of serial names and the problem of acronyms.  The original Loglan solution for strong quotation was not implementable in BNF or PEG;  I installed a different solution (which owes something in its details to an earlier analysis of Linnaean names in the 90s by other workers).  There was a systematic problem with tense-suffixed logical connectives (APA words) which required substantial work.  A full solution to the problem of the left boundaries of names was installed, which could only be tested because the new grammar software handles phonetic, lexical, and grammatical levels.

The main testbed for the grammar software has been parsing the entire text of Alex Leith's novel (twice),
revising it as necessary for the new parser to work.  There are many changes in the text, mostly minor, and the whole process gives evidence for the claim that 2013 Loglan, though different, would be intelligible to a speaker of 1989 Loglan if there were such a being.  And that is my aim:  I am engaged in making an existing language work, not in redesigning it to be a better one.   2013 Loglan has some profound differences in deep structure from 1989 Loglan, but the intention is to keep existing text stable, and the differences would only come into play with much more complex and extensive writing in the language.

\newpage

\section{Part I:  Essays}

\newpage

\section{Part II:  Commentary on the PEG}


\end{document}