\documentclass[12pt]{book}

\usepackage{verbatim}

\usepackage{listings}

\title{A New Systematic Grammar of TLI Loglan}

\author{Randall Holmes}

\begin{document}

\maketitle

\chapter{Introduction}

This is a new grammar of TLI Loglan.   The intention is to present a coherent picture of my provisional adjustments of the language.  The organizational principle is that I follow the structure of the Parsing  Expression Grammar (PEG) which is used to generate the computer grammar.

There will be observations on points of difference with earlier versions of the language as necessary.

This document speaks authoritatively, but not all these proposals have been approved by the TLI Loglan community.  I am writing this in hope of providing support for a consensus that this is the way to proceed.  The membership is welcome to offer criticisms, whether general or of particular points.

The reader can safely ignore comments in footnotes unless already proficient in Loglan or interested in the history of the language.

This document now contains the current text of the file \begin{center}
\verb draft-grammar-with-comments-alternative,  \end{center}
  with word wrap and line numbers.  This means that if you want to compile this file yourself, you need not only its source but the current PEG source.

\chapter{Phonetics and Orthography}

In this chapter, we discuss the rules for writing and pronouncing Loglan, and the way in which a stream of speech sounds or letters is formally resolved into Loglan words (with the proviso that grammar rather than phonetics dictates word boundaries between structure words).

\section{Phonetic and orthographic components}


\subsection{Loglan letters and punctuation}

The letters of the Loglan alphabet are the 23 letters of the Roman alphabet excluding {\bf q, w, x}. 

 The foreign letters {\bf q, w, x} can only occur in ``alien text" embedded in Loglan.\footnote{These letters were originally not included in Loglan, then they were added with strange pronunciations in the 1980's and 90's, then they were largely eliminated from the dictionary in the late 90's;  after 2013, we proceeded to eliminate them completely again.   Names for these letters (usable as pronouns) will be presented later.}

The vowels are {\bf a, e, i, o, u, y}.  The first five are the regular vowels.

The consonants are   {\bf b, c, d, f, g, h, j, k, l, m, n, p, r, s, t, v, z}

The Loglan name of a vowel V has two forms (legacy and modern\footnote{The modern forms were suggested by us after 2013, but we have fully accommodated the phonetics to the original forms.}):  legacy uppercase is V{\bf ma} and lowercase is V{\bf fi}, and modern uppercase is {\bf zi}V{\bf ma} and lowercase {\bf zi}V.  The legacy forms are fully supported, but they are phonetically irregular as Loglan words, and there are contexts where the modern forms must be used.

The Loglan name of a consonant C is C{\bf ai} (uppercase) or C{\bf ei} (lowercase).

There are some other series of letter names to be introduced below.  The primary function of these words is not phonetic, but as variables (pronouns), as will be explained later.

The junctures, indicating syllable breaks or stress are {\bf -, ', *}.  The hyphen {\bf -} is a simple syllable break.\footnote{The use of the hyphen to abbreviate the phonetic hyphen {\bf y} found in earlier sources is not accepted here;  in general, we do not pronounce punctuation.}  {\bf '} is a syllable stress marker, which may appear in place of (not in addition to) a hyphen after a stressed syllable, or in final position in a word after a stressed syllable.  {\bf *} is a symbol for emphatic syllable stress, with the same grammar as {\bf '}.  A juncture is never followed by another juncture;  a hyphen can be followed only by a letter (the hyphen, unlike the stress marks, never appears in final position).

We define a phonetic block as a sequence of  letters and junctures (referred to collectively as ``characters"), with the junctures giving information about syllable breaks and stress.  A phonetic block is always intended to be pronounced without pause.

The terminal punctuation marks of Loglan are {\bf .:?!;\#}.  

The comma {\bf ,} is an especially important punctuation mark, with the phonetic meaning of a pause in the flow of speech.  A comma is always followed by whitespace followed by a phonetic block or alien text (ignoring any initial parentheses or double quotes appearing before the block or alien text).
A phonetic pause is always denoted either by whitespace or a comma followed by whitespace;  in some contexts the comma-marked pause is mandatory.  Whitespace may or may not denote a pause;  there are some contexts where whitespace must denote a pause.\footnote{Uses of a close comma without a following space in earlier versions of Loglan are entirely replaced with uses of the hyphen as a syllable break.}

The double hyphen {\bf --} is an independent punctuation mark, not a syllable break:  it represents a pause, probably longer than the pause represented by a comma, and in some cases may be used in place of a comma where a pause is required.  The ellipsis $\ldots$ is an independent punctuation mark, not terminal punctuation, similarly representing a pause and occasionally usable in place of a comma.

Parentheses and double quotes may enclose Loglan or alien text  under some circumstances described in the grammar.  These are generally ignored for phonetic purposes.  They are not pronounced:  punctuation marks are never intended to be pronounced in the version of Loglan described here, though they may dictate pauses.

The letters have lowercase forms and uppercase forms and there is a capitalization rule applying to phonetic blocks.  The formal capitalization rule is quite complex:  the basic idea is that an uppercase letter will not appear immediately following a lowercase letter unless it is the first letter of a phonetic copy of a letter name (class {\tt TAI0} to be discussed below; the letter names given above are words of this class), and a vowel may appear capitalized after {\bf z} (for malicious reasons to be explained later).    There is no restriction on capitalization resuming after a juncture.  This allows the usual sort of capitalization, and also allows all-caps, and where junctures are present individual blocks of letters may be capitalized in different styles independently.
The special treatment of letter names will be motivated in examples when these words appear.\footnote{The approach to punctuation which we have taken has been driven partially by our own design decisions and partly by the punctuation and capitalization practices in the Visit to Loglandia.}

\subsection{Pronunciation of Loglan letters}

The regular vowels have typical continental European (not English!) pronunciation.  The irregular vowel {\bf y} may be pronounced with the indistinct schwa sound.  Unstressed regular vowels do {\bf not} become schwa (English and Russian speakers note!)  A more distinct pronunciation for {\bf y} (the vowel in English ``look"\footnote{A suggestion of John Cowan.} or the vowel written bI in Cyrillic) might be preferred.

The pronunciations of {\bf b, d, f, g,  k, l, m,  p, r, s, t, v, z} require no special comment (except to note that {\bf g} is always hard).

The letters {\bf c, j} have unusual pronunciations, {\bf c} being English ``sh" in ``ship", and {\bf j} being the voiced version of the same sound, the ``z" in ``azure".  English
``ch" and ``j" are the diphthongs {\bf tc} and {\bf dj} respectively.

The letter {\bf h} has the typical English pronunciation, except in syllable final position, where it has the sound of ``ch" in Scottish ``loch".  The latter pronunciation is actually permitted in other positions as well.\footnote{Syllable-final {\bf h} did not exist in previous versions of Loglan.  The pronunciation of {\bf h} which is mandatory in final position might be preferred by speakers of some languages in other positions.}

The letter {\bf n} is pronounced as usual in English, and (as usual in English as well!) is pronounced as the ``ng" in ``sing" when it occurs before {\bf g}, {\bf k}, or the alternative pronunciation of {\bf h}.

The vowels {\bf i} and {\bf u} are sometimes pronounced as English ``y" and ``w". as will be explained below.

The consonants {\bf l, m, n, r} are sometimes syllabic (``vocalic").  When they are used syllabically, they are always doubled.\footnote{The rule that syllabic continuants must be doubled forces changes of spelling in names in legacy Loglan text in many cases.  Syllabic consonants in borrowed predicates were already doubled, and Brown suggested in Loglan 1 that this might be a good rule to adopt in general.}

We neither reject nor support complex alternative schemes of pronunciation of the language outlined in older sources;  the alternatives we propose here seem sufficient.

\subsection{Alien text}

We describe the rules for embedding alien text in Loglan.\footnote{The model for these rules is actually the final state of the rule for Linnaean names with {\bf lao} (now foreign names in general) given in the late 90's.  We require that the occurrences of {\bf y} there suggested merely for speech also be expressed in writing.  The use of double quotes is a novelty but seems natural.  The strong quotation scheme of 1989 Loglan is abandoned, essentially by giving {\bf lie} the same phonetic grammar as {\bf lao}.}  For such text, rules of pronunciation are not supplied by Loglan.  It is required that alien text (however it is pronounced)
be preceded by a pause (regarded as part of the alien text) and followed by a pause or end of text or speech (not regarded as part of the alien text):  a pause may be expressed either by whitespace or by a comma or terminal punctuation (terminal punctuation only after the alien text, of course)  followed by whitespace, and end of text or speech simply by end of text or by terminal punctuation.  The body of the alien text between the initial pause and the final pause or end may be text not containing double quotes enclosed in double quotes, or it may consist of one or more blocks of text excluding commas, spaces and terminal punctuation marks, separated by the word {\bf y}, which must be preceded and followed by pauses in speech, which are independently expressible by whitespace or by a comma-marked pause.  Examples are {\bf``War and Peace"} and {\bf War y and y Peace}.   When alien text is enclosed in quotes, occurrences of {\bf y} between pause-separated components of the alien text may be omitted in writing but must appear in speech:  the two examples are pronounced in the same way.  Some contexts require double quoted alien text in writing.

Alien text is always preceded by one of the alien text markers {\bf hoi, hue, lie, lao, lio, sao, sue}, whose grammar and semantics will be discussed below.  Alien text marked with
{\bf hoi} or {\bf hue} must be double-quoted.   The parser identifies blocks of alien text by looking for these markers (some of the markers have multiple functions and will not always be followed by alien text).

Examples of alien text in Loglan utterances will appear when we discuss the grammatical constructions that use them.

\subsection{Vocalic diphthongs}

In this section, we describe two-letter forms which may appear as the ``vowel" component of a syllable.

The consonants {\bf l, m, n, r} we call {\em continuants}.  A doubled continuant {\bf ll, mm, nn, rr} represents a syllabic continuant, which may serve as the vocalic component of a syllable.
A syllabic continuant may not be followed  or preceded by another instance of the same continuant without an intervening pause in speech.

We now consider how to pronounce sequences of regular vowels not separated by junctures.  The issue is how to resolve such a sequence into syllables.
The irregular vowel {\bf y} is usually a single syllable, except for occurrences of syllables {\bf iy} and {\bf uy} in rare structure words, which will be discussed later.

We first consider sequences of two regular vowels.  Some of these sequences are mandatory monosyllables, some are optional monosyllables, and some cannot be read as monosyllables.

There are four mandatory diphthongs {\bf ai, ei, oi, ao}.  The diphthong {\bf ao} has the irregular pronunciation of ``ow"  in English ``cow".   The pairs of letters {\bf ai, ei, oi} are not mandatory diphthongs when followed by {\bf i} without an intervening juncture:  {\bf aii} is grouped {\bf a-ii}.

The forms {\bf a-i, e-i, o-i, a-o} (the hyphens may be replaced with other junctures), called broken monosyllables, can only occur in names (we do not regard {\bf a-ii} and its kin as containing broken monosyllables, but here we are talking about more than two letters).  One may write any other pair of regular vowels separated by a juncture, with the effect of enforcing the two syllable pronunciation (where it is optional).

There are six optional diphthongs, made up of {\bf i} or {\bf u} followed by a regular vowel.  (forms {\bf iy} and {\bf uy} also occur in special contexts, to be discussed later).  If these are pronounced as a single syllable, initial {\bf i} is pronounced as English ``y" and initial {\bf u} is pronounced as English ``w".  The disyllable pronunciation can be compelled by writing
a syllable break.   Monosyllabic {\bf iu} cannot be followed without a juncture by {\bf u} and monosyllabic {\bf ui} cannot be followed without a juncture by {\bf i} (so, for example, if
{\bf iuu} is encountered it will be read {\bf i-uu}).  As a rule, the speaker has a choice when presented with an optional monosyllable of pronouncing it as one or two syllables; sometimes the context forces one of the pronunciations.

Other pairs of adjacent vowels are pronounced as two separate syllables;  the use of a glottal stop to separate the components of a disyllabic vowel pair is permitted, but not expressed in the orthography.  The glottal stop is {\bf not} allowed as an allophone of the pause phoneme;  all required pauses must be distinct, if sometimes brief.\footnote{Previous versions of Loglan do not allow the glottal stop to appear medially in disyllables (we allow it but also allow the traditional Loglan pronunciation, a smooth glide from one vowel to the other);  previous versions of Loglan allowed the glottal stop as an allophone of pause, and we do not.  Lojban uses the h sound medially in disyllables, which would be allowed for a Loglan speaker who chose always to use the alternative pronunciation of {\bf h}.}

A pair of identical adjacent vowels not pronounced as a monosyllable has the characteristic that one of the vowels must be stressed and the other unstressed.  This always holds for
{\bf aa, ee, oo} and sometimes holds for {\bf ii}, {\bf uu} (special rules stated above are designed to encourage pronunciation of the latter two pairs as monosyllables whenever possible!)

For a three-vowel sequence appearing in a predicate or name word, the general rule is that formation of monosyllabic
{\bf ii} or {\bf uu} is the highest priority (so in {\bf aii}, forming {\bf ii} wins over forming {\bf ai}, which in this context is not a mandatory monosyllable anyway, producing {\bf a-ii}), followed by formation of a mandatory monosyllable (recalling that {\bf i}-final mandatory monosyllables are not followed by {\bf i}; {\bf aoi} is grouped {\bf ao-i} and is not considered to contain a broken monosyllable):  e.g.,  {\bf aiu} is grouped {\bf ai-u}), followed by formation of an optional monosyllable (which is often an optional preference for the speaker;  the parser does exercise this preference).  An extreme example of speaker freedom is {\bf iue}, which the parser will resolve into two syllables {\bf iu-e} (choosing to group the first two when both pairs have the same precedence)  but which the speaker can resolve into two or three syllables in any of the three possible ways.

We present a  formal rule\footnote{The formal rule for reading long sequences of vowels in names appearing in Loglan 1 is incredible, as it requires indefinite lookahead;  of course it was also really intended only for use with three or perhaps four vowels.} for reading the next syllable from a sequence of  regular vowels of any length written without junctures, which is used in resolving predicates and names into syllables.  A mandatory diphthong is read as the first syllable if it is present (recalling that if the pair of vowels ends in {\bf i} and is followed by another {\bf i} it is not a mandatory diphthong); a single vowel is read if it is not initial in a mandatory diphthong and the next two vowels form a mandatory diphthong;  if neither of the previous two cases holds an optional diphthong is read by preference by the parser (though a disyllabic reading is permitted); as the final option a single vowel is chosen, subject to the rule that {\bf i} or {\bf u} (when not part of a diphthong) cannot be followed by an intervening juncture and a consonantal occurrence of the same vowel (this situation will cause parse failure).  The process of resolution of the first syllable from a  stream of vowels is repeated until the stream of vowels is completely resolved into syllables.
This rule may look forbidding, but it should be noted that sequences of four or more vowels are quite rare in Loglan predicates or names, so the two and three vowel accounts will usually be quite enough.

There is a separate rule, used in resolving certain structure words,  in which a sequence of vowels of even length is parsed into vowel pairs, each of which is read as monosyllable or disyllable as the rules require or permit.  There is a further special rule for certain structure words with three-vowel sequences, which does not conform with the rule stated above for resolving vowel sequences in predicates and names, which will be stated when these structure words are described.

When the vowel component of a syllable is read, this will be either a syllabic continuant, or {\bf y}, or a vowel or vowel diphthong chosen using the appropriate one of the rules above.

One should note that the rules presented here are not of interest to readers and writers, speakers and listeners, very directly;  but they are certainly of interest to word makers, and might briefly be of interest to a dictionary reader encountering a word for the first time.  Such phonetic rules exist in natural languages, whose speakers are not necessarily even aware of them;  one could imagine that the native Loglander, though her speech will conform perfectly to the rules stated above, will not know much about them unless she is a grammarian!

\subsection{Consonant grouping}

There are different rules for syllable-initial and syllable-final consonant grouping.  It is worth noting that consonant grouping only occurs in regular Loglan text in predicates and names.   Syllables with final consonants also occur only in predicates and names.

These are governed by two sets of phonetic rules.  There is a list of permitted initial pairs of consonants\footnote{The initial pairs are {\bf bl}  {\bf br}  {\bf ck}  {\bf cl}  {\bf cm}  {\bf cn}   {\bf cp}  {\bf cr}   {\bf ct}   {\bf dj}   {\bf dr}   {\bf dz}  {\bf fl}  {\bf fr}   {\bf gl}   {\bf gr}   {\bf jm}   {\bf kl}  {\bf kr}   {\bf mr}   {\bf pl}  {\bf pr}   {\bf sk}  {\bf sl}   {\bf sm}  {\bf sn}  {\bf sp}   {\bf sr}  {\bf st}  {\bf sv} {\bf tc}  {\bf tr}  {\bf ts}  {\bf vl}  {\bf vr} {\bf  zb}  {\bf  zl}  {\bf zv}}.  The initial group of consonants in a syllable consists of a single consonant, or a permissible initial pair of consonants, or a triple of consonants in which each adjacent pair of consonants is an initial pair.\footnote{The rule for initial consonant groups appears in Notebook 3.}  We refer to a pair of consonants which would be a permissible initial pair if an intervening juncture were removed as a ``broken initial pair".

There is a list of forbidden medial pairs\footnote{The impermissible medial pairs consist of all doubled consonants, any pair beginning with {\bf h}, any pair both of which are taken from {\bf cjsz}, {\bf fv}, {\bf kg}, {\bf pb}, {\bf td}, any of ({\bf fkpt}) followed by either of ({\bf jz}), {\bf bj}, and {\bf sb}.}
and a list of forbidden medial triples\footnote{{\bf cdz}, {\bf cvl}, {\bf ndj}, {\bf ndz}, {\bf dcm}, {\bf dct}, {\bf dts}, {\bf pdz}, {\bf gts}, {\bf gzb}, {\bf svl}, {\bf jdj}, {\bf jtc}, {\bf jts}, {\bf jvr}, {\bf tvl}, {\bf kdz}, {\bf vts}, and {\bf mzb}}. 
 These cannot occur even if broken by a juncture.  

There can be one or two final consonants in
a syllable, which cannot be part of a forbidden medial pair or triple whether together (if there are two of them) or combined with consonants taken from the beginning of the following syllable.  A pair of final consonants cannot be a non-continuant followed by a continuant (this appears to be pronounceable only as a separate syllable).  A final consonant cannot be followed by a regular vowel or a syllabic continuant, even with an intervening juncture (in other words, such a consonant should be read as part of the following syllable).\footnote{The rules forbidding final consonants from participating in illegal medial pairs or triples are found in our sources.  The rule forbidding a pair of final consonants from being a non-continuant followed by a continuant seems quite natural but is ours;  no word was proposed that violated it, in any case.  Other rules that we state depend on a precise definition of the syllable, which appears nowhere in Loglan sources, although the notion of syllable is important in the definition of borrowed predicates in Notebook 3.}

A new (2/13/2021) rule forbids a pair of final consonants to consist of one of {\bf ptksfh} and one of {\bf bdgzv}, in either order.  This forbids a voiced and an unvoiced consonant to occur together in a pair of final consonants, if neither is a continuant.

A consonant in either of these sorts of groups which is a continuant cannot be adjacent to another copy of the same continuant, within or without the cluster, even if separated by a juncture.
An initial consonant triple cannot be followed by a syllabic continuant at all.

\subsection{The Loglan syllable}

A Loglan syllable consists of three parts. 

 There is an optional initial group of one, two or three consonants governed by rules stated in the previous subsection.  

This is followed by the mandatory vocalic component of the syllable, which is either a pair of identical continuants, a single regular vowel, a vowel diphthong, or {\bf y} ({\bf iy}  or {\bf uy} occur only in syllables (C){\bf iy} and (C){\bf uy} which are directly allowed as units in structure words but not supported in the formal syllable definition).

This is followed optionally by one or two final consonants, for which rules are stated above, with the additional remark that unless the syllable is of the shape CVC with the vowel regular,   no final consonant in the syllable (neither of them, if there are two)  may be readable as standing at the beginning of a following syllable (in other words, except in the case of CVC syllables, the automatic placement of syllable breaks where an explicit juncture is not present  is as early as possible;  but a CVC syllable is preferred to a CV syllable where possible).  Explicit junctures will override the preferred syllable breaks, but there are subtle rules about where explicit junctures can be placed:  sometimes they will simply cause parse errors.\footnote{The subtleties have to do with the fact that a borrowed predicate cannot resolve into djifoa (see below for these terms);  an apparently legal borrowing predicate written with explicit junctures will be rejected if moving some of the junctures would create a legal complex predicate.  These are issues which mostly affect the word designer.  If you are trying to write a complex predicate from the dictionary with explicit syllable breaks, make sure that the breaks you supply conform with djifoa boundaries and these issues will not arise.}

It is worth noting that previous versions of Loglan had no official formal definition of the syllable, though the syllable did play a role in the definition of some word classes.\footnote{The lack of felt need for a formal definition of the syllable  may have come from the fact that structure words and complex predicates resolve into units which are not themselves necessarily syllables, but which are expected to conform with syllable boundaries;  it is with the introduction of borrowed predicates that a precise notion of the syllable became essential to someone who wanted to parse words, and once this notion was in hand, it became natural to require that names (which were just consonant final strings of phonemes in earlier versions of Loglan) be resolvable into syllables as well.  The accuracy of our implementation can be gauged by the fact that almost all words in the dictionary parsed correctly when we ran a test, and the ones which did not parse had recognizable errors which needed to be fixed.  It should be noted that we cannot have three final consonants in a syllable, and this is not uncommon in names.  This can usually be fixed by doubling a continuant, as in {\bf Hollmz}, {\bf Marrks}, but some names may be found to be definitely foreign.}

\subsection{Pauses and whitespace:  general principles}

A pause is always expressed as either a comma followed by whitespace (which must be followed by a phonetic block) or simply whitespace, which must be followed by a phonetic block.
The former is always a pause;  the latter may sometimes not be a pause.

Whitespace at the beginning or the end of alien text must represent an actual pause.

Whitespace after a consonant and/or before a vowel must represent an actual pause.

Names are the only consonant-final words in regular Loglan text, and they must be followed by comma-marked pauses, terminal punctuation, or end of text, or by whitespace followed by another name word or the structure word {\bf ci}.

Logical connective words of class A, some but not all of which are vowel-initial, and sentence connectives of classes I and ICA must be preceded by comma-marked pauses.  The APA and IPA logical and sentence connectives, to be discussed below, and the ICA and ICAPA sentence connectives must be followed either by the suffix {\bf fi} or a comma-marked pause.  The issues in this paragraph are handled entirely in the grammar section.

Words quoted with {\bf liu} must be followed by a comma-marked pause (or terminal punctuation or end of text).

If the final syllable of a structure word is stressed and it is followed by a predicate, it must be followed by a comma-marked pause.  This rule is of course only enforced in our orthography if we actually write explicit stress.\footnote{This rule goes back to the beginnings of Loglan, but as no earlier parser had explicit indications of stress, there was never any occasion for an earlier parser to enforce it.}

In general, certain lexicographic issues tend to force explicit comma-marked pauses.  If a pause in a sequence of structure word syllables breaks a word, it must be explicitly comma-marked as a rule, since if it were written as mere whitespace, not pausing would cause a different interpretation of the utterance.   There will be a discussion of multi-syllable structure words in the lexicography section which lays out the situations under which this issue can occur.\footnote{In Lojban, apparently all structure word syllables are separate words, but this is not the case in Loglan.}

The ``false name marker" problem creates further need of explicit pauses, which will be discussed below.

This version of Loglan supports a form of orthography known as ``phonetic transcript" in which no whitespace appears but comma-marked pauses.  This means that we require that
in every place where we can or must pause, it must be possible to replace whitespace with a comma-marked pause.  It is mostly but not entirely true that every place whitespace is written
is a place where one {\em can\/} pause:  it is possible to create situations with the APA and IPA connectives in their legacy form where a whitespace that one can write cannot represent a pause, and there is a rule that one should not pause after the structure word {\bf ci} before a consonant unless the pause is comma-marked.  This whitespace can, however, be omitted.  Whitespace which does not represent pauses can always be omitted, though in the case of whitespace after predicates, this may require the writer to insert explicit indications of stress so that the reader can tell where the predicate ends.  Whitespace which cannot be omitted can always be replaced with an explicit comma-marked pause.

Because we have phonetic transcript, we do not need a special notation for expressing pronunciation.\footnote{Brown's phonetic notation in the sources is {\em ad hoc} and reveals such things as very inconsistent notions about syllable breaks.}

\section{Phonetic word forms}

\subsection{The four forms, and general principles}

There are four basic word forms in Loglan: 

\begin{enumerate}

\item Items of alien text (with their preceding alien text markers), already described above.

\item Phonetic names (name words accompanied with their required preceding pauses or name marker words with intervening optional pause).

\item  Structure words

\item Predicates, further subdivided into complexes and borrowings.

\end{enumerate}

These classes of words have general characteristics which allow us to distinguish them  We leave aside the case of alien text which we have already analyzed.

\begin{enumerate}

\item Name words are the only consonant-final words in Loglan (other than alien text).  They are thus followed by pauses in speech (and usually by explicit pauses in writing).  This makes the right boundary of a name word easy to recognize.  One must also pause at the beginning of a name word, unless it is preceded by one of a limited class of name markers.  There are few contexts in which a name word can appear without an immediately  preceding name marker word, and if a name word happens to include a phonetic copy of a name marker word (a ``false name marker") it {\em must\/} be immediately preceded by a name marker word (an intervening pause being permitted).  Where a name marker word occurs which is not immediately followed by a name word but followed by a name word starting later, a comma-marked explicit pause (or terminal punctuation) must appear somewhere between the name marker word not serving as such and the following name word:  this prevents pronunciation of the text in a way which causes everything between the name marker word and the end of the later name word to be construed as a single longer name word.

\item Predicates end with a regular vowel (so they are not names), are penultimately stressed (with qualifications to be stated later);  this allows the right boundary of a predicate word to be recognized in speech, or in phonetic transcript), and contain adjacent consonants (in some cases the pair of consonants may be separated by {\bf y}).  The left boundary of a predicate is determined
by the fact that it must begin CC or (C)V$^n$C({\bf y})C.  in the latter case with some conditions ensuring that the (C)V$^n$ cannot be construed as a structure word.  Predicates
can more rarely begin (CVV{\bf y})$^n$((C)V$^m$)CC.

\item Structure words (Loglan {\bf cmapua}) are not names or predicates (actually some are semantically names or predicates, but this is a matter for the grammar).  In addition, we specify that they resolve into phonetic units of the shapes V, VV, CV, CVV (where the VV may be a monosyllable or a disyllable, and {\bf iy}, {\bf uy} are permitted), and the rare Cvv-V, where the vv is a monosyllable (mandatory or optional, but in any case pronounced as such).
Further, a V unit may only occur initially, and any structure word which contains a VV unit consists entirely of VV units (except that we allow words of the shapes
{\bf no}-VV and VV-{\bf noi}).  A sequence of VV units is resolved into syllables by pronouncing each unit as one or two syllables as the grammar requires or permits.  Note that the unit cmapua are not necessarily syllables, but their boundaries are syllable boundaries in a structure word.  Where a structure word is followed by a predicate beginning with CC,
stressing its last cmapua unit might create the possibility of reading the last cmapua unit and the first syllable of the intended predicate word as a predicate:  to avert this, we require that a finally stressed structure word must be separated from a following predicate word (not just a CC-initial one) by a comma-marked pause.

\end{enumerate}

It should be noted that the classes of words here should be qualified as phonetic names, phonetic predicates, and phonetic structure words, as there are cases where ``words" which are phonetically of one of these shapes are used in a way associated with one of the others.

\subsection{Phonetic Names}

We distinguish between a name word, such as {\bf Djan}, and a phonetic name, such as {\bf la Djan Braon}, which comes equipped with the name marker word or initial pause that a name word requires in its context, and may contain more than one name word after the name marker.

A name word is a phonetic block which resolves into syllables, the last of which ends in a consonant (possibly with a final stress).

A possible name word is a name word, or a name word modified by insertion of whitespace at junctures preceded by a vowel and succeeded by a consonant (so that the whitespace does not necessarily represent a pause).

A marked name is a name marker word followed by a consonant initial name word, possibly with intervening whitespace between the two.

A falsely marked name is a name word with a proper final segment which is a marked name:  that is, it is a name word with a false name marker in it.  Notice that a phonetic
occurrence of a name marker word is not a false name marker unless what follows it is a consonant-initial name word.\footnote{In early versions of Loglan, falsely marked names were simply forbidden, but {\bf la} is very common.   Later, they were admitted and some effort was made to avoid problems with them.  The idea that a falsely marked name must be marked appeared in the context of implementation of serial names (falsely marked names in a serial name had to be marked with {\bf ci}; we required after 2013 that predicate components of serial names be marked
with {\bf ci} as well to avoid the need for two pause phonemes to avoid confusion of serial names with sentences.)  We extended the idea that falsely marked names must
be marked to all contexts, and in addition reduced the distribution of unmarked names to very few contexts by forbidding unmarked vocatives.}

The name marker words are {\bf la, hoi, hue, ci, liu, gao, mue}.  A subtle point is that {\bf ci} is only a name marker when followed by a pause (an explicit comma-marked pause
or whitespace followed by a vowel):  this allows us to avoid difficult-to-predict needs for pauses after the many uses of {\bf ci}.  It does mean that when whitespace is written after {\bf ci} before a consonant, we presume that the speaker does not pause.

A phonetic name (including its name marker or preceding pause if there is one) is of one of the following kinds:

\begin{enumerate}

\item a marked name as described above (a name marker followed by possible whitespace followed by a consonant-initial name word).

\item  a vowel initial name word which is not a falsely marked name, or  a comma-marked pause followed by a name word which is not a falsely marked name.

\item a name marker followed by optional whitespace or explicit pause followed by a name word, with the additional proviso that the optional whitespace or pause must be present if the name word is vowel-initial.

\end{enumerate}

To any of these, a series of name words marked with {\bf ci} and unmarked name words  which are not falsely marked, may be appended as part of the phonetic name,
so {\bf la Djan Braon} is a phonetic name, and so is {\bf la Pierr ci, Laplas}.  In the last example one pauses both before and after {\bf ci};  the second comma must be written, and the use of {\bf ci} is necessary because {\bf Laplas} is a falsely marked name.

It is then required that this be followed either by an explicit pause, terminal punctuation, end of text, or whitespace followed by {\bf ci} followed by a predicate of class {\tt predunit}, a peek forward at the grammar.  Note that the following explicit pause or punctuation or {\bf ci} phrase is not part of the phonetic name:  this is information about the context in which a phonetic name can appear.

Names may contain explicit junctures, including ones which form broken monosyllables, and junctures may be required features of name words:  {\bf Lo-is} and {\bf Lois} are different names.\footnote{This goes back to previous versions of Loglan, but we use hyphens instead of close commas.}

\subsection{Phonetic structure words (cmapua)}

Phonetic structure words are sequences of cmapua units as sketched above;  we give more details.

Cmapua units are of the shapes V, VV, CV, CVV, Cvv-V, where vv stands for a monosyllable and VV (in VV and CVV units) includes {\bf iy}, {\bf uy}.  {\bf y} is also accepted as a V unit.\footnote{The practical reason for allowing {\bf y} to occur above is to support names of the letter {\bf y}, legacy {\bf yfi} and modern {\bf ziy} (pronounced ``zyuh"!).  It seemed more principled to install general phonetic conditions that allowed these forms than to allow them individually by fiat.}

A phonetic structure word is a string of cmapua units.  A cmapua unit not of VV form cannot be followed by a vowel, even with an intervening juncture:  this helps to enforce
the condition that vowel-initial words must be preceded by pauses in speech, represented at least by whitespace. 

Each cmapua unit is restricted by lookahead tests for other classes.  A cmapua unit cannot be an alien text marker actually followed by alien text.  A cmapua unit cannot be an occurrence of {\bf li} or {\bf kie} which actually stands at the beginning of a quotation or parenthetical free modifier (for the uses of these words, see the grammar section).  A cmapua unit cannot be a name marker followed with optional pause by a possible name word (this excludes both
actual phonetic names and strings which could be misread as phonetic names by ignoring instances of whitespace one of which should be made an explicit pause).\footnote{This is our definitive solution to the false name marker problem.  Difficulties created by the markers other than {\bf ci} should generally be easy to anticipate, by following style rules such as ``always pause after a predicate name".  The word {\bf ci} presented special difficulties as a name marker because it has a wide variety of uses some of which have nothing to do with names.  Viewing it as a name marker only when followed by a pause seems to be the final refinement of our solution.}   A cmapua unit
cannot stand at the beginning of a legal predicate (the parser  does a lookahead test which identifies strings which can only be predicates if they are grammatical and not possible phonetic names;  we describe this test below at the beginning of the discussion of predicates).  

A cmapua unit cannot be stressed and then followed by optional whitespace and the start of a consonant-initial predicate (as detected by the test above).

We then provide a phonetic test for the logical and sentence connective classes which must be preceded by a pause.  A phonetic connective starts possibly with whitespace followed by possibly by an occurrence of 
{\bf no} (not starting a predicate) followed definitely by a regular V syllable or {\bf ha}\footnote{Note that we are ruling here that {\bf ha} and its derivatives must be preceded by a pause.}, {\bf nuu} (not starting a predicate), not followed by a vowel, and not followed by {\bf fi}, {\bf ma}, or {\bf zi}, which would make a V unit into a legacy letteral (none of these starting a predicate).

We can now describe a phonetic structure word.  It takes one of five forms.

\begin{enumerate}

\item a VV unit (here and in all clauses here including {\bf iy}, {\bf uy}) followed by {\bf noi} ({\bf noi} not starting a predicate).

\item {\bf no} (not starting a predicate) followed by a VV unit

\item a sequence of VV units

\item a regular or irregular V unit

\item an optional regular or irregular V unit followed by a sequence of one or more consonant-initial cmapua units

\end{enumerate}

A cmapua unit absorbs a following juncture.

Each cmapua unit is blocked from being followed by a vowel without intervening whitespace or by optional whitespace then a phonetic connective;  this forces explicit pauses before the logical and sentence connectives.

The phonetic structure words defined here have boundaries dictated entirely by phonetic convenience;  the actual boundaries of cmapua words in the proper sense are dictated by rules stated in the lexicography chapter.  Some words which appear in other structures, such as the name markers, alien text markers, and {\bf y}, and some others, are from a lexicographic standpoint structure words and do look like them phonetically.

\subsection{Primitive Predicates and Combining Forms (djifoa)}

The basic ``native" predicates of Loglan are of the five letter forms CCVCV and CVCCV.  The original stock of native predicates was generated by a rather {\em ad hoc\/} statistical comparison with words in major natural languages on which we have no intention of commenting, as we expect it never to be used again. 

Each of the native predicates has one or more combining forms (originally called ``affixes", a  deprecated usage;  now usually called {\em djifoa\/}, the Loglan word for these forms).

Each five-letter native predicate has a djifoa formed by replacing its final vowel with {\bf y}.  This does mean that five letter predicates which have the same final vowel must be semantically very closely related (words for animals and languages can be given fine shades of meaning by adjusting the final letter;  we do not intend to create further declensions of this kind, but we see nothing wrong with the ones we have).

In addition, many five-letter djifoa have one or more than one associated three letter djifoa, of one of the forms CVV, CVC, or CCV, which is formed by choosing three letters from 
the five letter djifoa in order of their occurrence.  The process of choosing these djifoa is not likely to be modified or extended at this point, though there is some tension about the ones with doubled vowels which force stress.

There is also a short list of three-letter djifoa built from CV cmapua, by appending {\bf r}, which we supply:
\begin{description}
\item[fer:] from {\bf fe}, five
\item[for:]  from {\bf fo}, four
\item[fur:] from {\bf fu}, 3d place passive
\item[jur:] from {\bf  ju}, 4th place passive
\item[ner:]  from {\bf  ne}, one
\item[nir:]  from {\bf ni}, zero
\item [nor:]  from {\bf  no}, logical negation
\item[nun:]  from {\bf nu}, 2nd place passive, before{\bf  r}
\item[nur:]  from {\bf nu}, 2nd place passive, not before {\bf r}
\item[por:]  from {\bf po}, state particle
\item[rar:]  from {\bf ra}, all
\item[rer:]  from {\bf re}, most of
\item[ror:]  from {\bf ro}, many of
\item[ser:] from {\bf se}, seven
\item[sor:]  from {\bf  so}, six
\item[sur:]  from {\bf  su}, at least one of, some
\item[ter:] from {\bf te}, three
\item[tor:]  from {\bf to}, two
\item[ver:]  from {\bf ve}, nine
\item[vor:] from {\bf vo}, eight
\end{description}

Further, every CV cmapua unit  has a corresponding CV{\bf h} djifoa.  CVV cmapua  may be extended with {\bf (h)y} (not with {\bf n} or {\bf r}) and used as djifoa:  thus
{\bf zaiytrena}, A-train.  This must be done with care as there may be djifoa derived from cmapua of the same shape.

Loglan complex predicates (native compound predicates) are built from sequences of these djifoa (in which the last item may be a full primitive or borrowed predicate).  There are also borrowing djifoa built from borrowed predicates, which we discuss in the next section.

It is necessary to supply additional phonetic glue so that sequences of these djifoa actually can produce predicate words.  Each three-letter djifoa has an alternative form
with suffixed {\bf y}.  CVC{\bf y}  djifoa can be broken into syllables either as CVC-{\bf y} or as {CV-C{\bf y}.  CVV djifoa in initial position will ``fall off":  so CVV{\bf r} is available as an alternative form (which will form a consonant pair with the following djifoa or predicate word), and CVV{\bf n} is available as an alternative form when it is followed by {\bf r}.  Other problems
are that CVC djifoa may not occur in final position, and CVV djifoa which have a doubled vowel can only occur in final or in penultimate position, because a predicate word can only contain one stress in a penultimate position.  We propose for CVV{\bf y} the alternative form CVV{\bf hy}:  a predicate starting this way will not be confused with any other kind of word, and this should be easier to pronounce distinctly.

A pronunciation difficulty with CVV{\bf r} extended djifoa (at least for English speakers) is fixed by a permission which will probably seldom be expressed in writing, but can be:
CVV{\bf r} can take the alternative form CVV-{\bf rr} when the VV is a mandatory monosyllable.  If this is stressed, the stress falls on the VV and the vocalic continuant is an additional unstressed syllable before the final unstressed regular syllable.

Broken djifoa forms are also available to the parser, obtained from legal djifoa by inserting junctures (C-CV or C-VV  or CV-C, for example).  These are used to enforce the condition that borrowed predicates cannot decompose into djifoa, and it should not be possible to convert a complex (especially one which is illegal for reasons peculiar to complexes) to a legal borrowed predicate by moving junctures around.

The general point about syllable breaks is that while djifoa are not syllables, the boundaries between djifoa will be syllable breaks in a legal complex.  Internal breaks are sometimes
optional:  a CVV djifoa with an optional monosyllable has two possible forms.  The CVCCV five letter predicates may admit two forms CVC-CV and CV-CCV if the medial CC is an initial pair.
The parser prefers the first form for technical reasons;  it can be coerced by writing an explicit syllable break, and the latter version is often easier to pronounce.

\subsection{Phonetic predicate words:  general principles, and recognizing the beginning of a predicate}

A phonetic predicate word ends with a regular vowel (so it is not a phonetic name), contains an adjacent pair of consonants (so it is not a structure word), and has
penultimate stress (with the exception that an additional unstressed syllable with {\bf y} or a vocalic continuant may intervene between the stressed and the final syllable), so that one can tell where it ends.

The part of the predicate before the consonant pair can be null (the predicate may start with an initial pair or triple of consonants).  This will be followed by a regular vowel,
and there are no CC(C)V(V) predicates, in which the initial consonant group is followed just by one or two vowels.

The initial segment before the consonant pair can be an optional consonant followed by one to three regular vowels\footnote{Previous versions of Loglan have allowed arbitrarily long sequences of vowels after the initial consonant in this case, but these have never been used and I like the bound on lookahead in this test obtained by forbidding more than three vowels.}   This is the only alternative which can occur in a borrowed predicate.  This can be ensured if the string starting with the consonant group doesn't satisfy the conditions given above to be the start of a predicate:  the predicate must then begin with the initial (C)V(V)(V) (it cannot begin with part of the vowel sequence because one must pause before a vowel-initial word).  It can also be ensured if the final syllable of the initial (C)V(V)(V) is stressed:  if it were the end of a cmapua, it could only be followed by a predicate,
and one cannot have a stressed cmapua followed by a predicate without pause.  There is a further technical issue, which is explained below in the discussion of borrowing djifoa:
for technical reasons, if the (C)V(V)(V) is not of one of the forms CV or CVV, the consonant group cannot be an initial pair followed by a regular vowel.

In a complex, there are other possibilities.  A complex might start with one or more CVV{\bf y} djifoa; no other sort of word can start this way.  It might start with
a CVCC{\bf y} djifoa;  no other sort of word can start this way.  It might start with a an extended CVC djifoa:  CVC{\bf y}. Again, no other Loglan word can start in this way.

We describe a  lookahead test:  a string which is already known not to be a possible phonetic name cannot be anything but a predicate (or ill-formed) if one of the following things are true (and one of these things will be true of any actual predicate):

\begin{enumerate}

\item  It begins with a permissible initial group of two or three consonants  and is followed by a regular vowel, but not by one or two vowels (with possible junctures) followed by a non-character, nor by a stressed vowel followed by a vowel not in a diphthong (short words of the shapes CC(C)V(V) are not predicates).  Junctures may appear after the vowels.  This clause of the test ignores any junctures which may appear in the initial group of consonants (to support its use in following clauses).

\item  It begins with CV(V) followed by a consonant group, with either the final syllable in the CV(V) [which might extend to a juncture in the consonant group] stressed or the part of the word beginning at the consonant group not meeting the test to start a predicate above.  Junctures may appear after the vowels, and there might be a juncture in the consonant group, which will be ignored in testing whether it begins a predicate.

\item  It begins with (C)V(V)(V) followed by a consonant group which is not an initial pair (even one broken by a juncture)  followed by a regular vowel, with either the final syllable in the (C)V(V)(V) [which might extend to a juncture in the consonant group]  stressed or the part of the word beginning at the consonant group not meeting the test to start a predicate above.  Junctures may be inserted after the vowels or there might be one in the consonant group, which will be ignored in testhing whether it begins a predicate.

\item It begins with a consonant and a regular vowel, followed by a regular vowel followed by {\bf y} [or {\bf hy}], or a consonant or pair of consonants followed by {\bf y} (with possible intervening junctures).  Both of these initial sequences
are possible beginnings for a predicate complex, and what follows the CV$^n$ could not start any legal word in that context.

\end{enumerate}

The point of including this list is making it clear that it is fairly easy to see or hear the beginning of a predicate word (though the detailed description of the cases is admittedly annoying!)

\subsection{Borrowed Predicates and Borrowing Djifoa}

After all that about djifoa, we discuss borrowings first!

A borrowing must resolve into syllables.  It must end in a regular vowel.  Any explicit stress must be on the second-to-last syllable, not counting syllables with vocalic continuants (one such unstressed syllable may intervene between the stressed syllable and the unstressed final syllable).     The end of a predicate is determined by either non-characters such as whitespace or punctuation, or by an explicit stress.  A borrowing cannot be followed without intervening whitespace or explicit pause by a vowel, nor by optional whitespace followed by a connective.  An explicit stress  may force monosyllabic pronunciation on a last syllable which could otherwise be pronounced as a disyllable.   There can be only one explicit stress in a borrowing.  The deduced stresses in doubled vowels do not play a role in parsing borrowings, as disyllabic doubled vowels are forbidden in borrowings.  It is permissible to write a borrowing with explicit junctures, but this cannot change the meaning of the word, and broken monosyllables are not permitted.  A borrowing must parse correctly in the absence of explicit junctures.

The beginning of a borrowed predicate must pass the test for beginnings of predicates given above.  Since borrowed predicates cannot contain {\bf y}, this enforces the condition that there be a pair of adjacent consonants in a borrowed predicate.  We reiterate that a borrowed predicate cannot have the shape CC(C)VV (which is of course not resolvable into djifoa, so could not be the shape of a complex predicate).

There are some restrictions on the phonetics of borrowed predicates.  Borrowed predicates may not contain {\bf y} or any doubled vowel other than monosyllabic {\bf ii} and {\bf uu}.\footnote{Forbidding doubled vowels in borrowings was an action of ours:  admitting them made reasoning about the penultimate stress in a borrowing more difficult.  Exactly one predicate {\bf alkooli} had to be changed to the better {\bf alkoholi}.}
Borrowed predicates may contain syllables with vocalic continuants.  These never follow a vowel and so far never precede a consonant, and such a syllable is always medial (not first or last).  There may not be two such syllables in succession, and such a syllable cannot be stressed. The point of such syllables is that they cannot occur in complexes (with one minor exception which cannot be confused with occurrences of syllabic continuants in borrowings:  a CVV{\bf r} djifoa may be expressed as CVV{\bf rr}, in which the syllabic continuant follows a vowel).  A borrowed predicate may not resolve into djifoa, including resolutions involving broken forms in which junctures are misplaced.

Borrowing djifoa are formed from borrowings by adding final {\bf y} and moving the stress to the final syllable of the borrowing (still penultimate in the borrowing djifoa).  It is permitted to stress the penultimate syllable in a borrowing djifoa and pause after the {\bf y}, if what follows the borrowing djifoa contains a penultimate stress.  Thus
one may pronounce {\bf bakteriyrodhopsini} as {\bf bakteri'y, rodhopsi'ni}, but one may not pronounce {\bf iglluymao} with a pause.  The stress shift is a strong signal that one
is not saying {\bf bakte'ri} but its djifoa.\footnote{Nothing in this situation is due to me!  The strange provision to pause after a borrowing djifoa is in Loglan 1.  The definition of borrowing djifoa we use was given in the 1990's.  I do not know if anyone noticed the stress shift caused by the change in the definition of borrowing djifoa, but it is all a logical consequence of the way things stood at Brown's death.}

The reason for the difference between the treatment of CV(V) and other (C)V(V)(V) prefixes in the predicate start rules  is caused by the danger that a borrowing djifoa for a (C)V$^n$CCV predicate might turn into a cmapua followed
by a stressed CCV{\bf y} djifoa when a borrowing djifoa was formed (this requires the CC to be an initial pair, of course).  This is not a danger when the consonant is present and $n = 1$ or 2 because a borrowing will not be of the shape
CV(V)CCV with the CC initial.  Thus a string (C)V$^n$CCV with the CC an initial pair is not allowed to start a predicate unless the initial consonant is present and $n$ is 1 or 2.

\subsection{Complex Predicates}

A complex predicate is formed from a sequence of djifoa in which the final element may be a full primitive or borrowed predicate (and if it is a djifoa may not be CVC, nor may it be extended with {\bf r}, {\bf n} or {\bf y}, nor may it contain {\bf y} at all).   A complex may not be formed entirely from CVV{\bf y} djifoa and a final CVV (this prevents forms without adjacent consonants;  adjacent consonants may be separated by {\bf y} as in {\bf mekykiu}).  A complex must pass the predicate start test.
CVV djifoa may need to be extended with {\bf r}, {\bf n}, or {\bf y} when they appear in initial position to keep from being read as cmapua.  CVC djifoa in initial position of a complex not of one of the six-letter forms CVCCVV or CVCCCV must
be extended with {\bf y} if the final consonant of the djifoa and the following consonant would form an initial pair (a juncture does not affect this):  this prevents the complex
from being read as a CV djifoa followed by a borrowing.\footnote{This rule superseded the historical {\bf slinkui} test, which prevented the CV from falling off the front of a CVCC... complex with the CC an initial pair by forbidding the formation of borrowed predicates obtained by extending a complex initially with a single consonant forming an admissable pair:  instead of forbidding {\bf paslinkui} in favor of {\bf pasylinkui}, {\bf paslinkui} was permitted and {\bf slinkui} was forbidden to be a borrowing.}  Thus {\bf ficynirli}, mermaid.  A CVC djifoa may need to be extended with {\bf y} to prevent formation of an illegal consonant group with the following consonant.   Thus {\bf mekykiu}, eye doctor.  Regular djifoa must be extended with {\bf y} when they appear before borrowing djifoa:  since borrowings cannot contain {\bf y}, this gives us precise information about their boundaries.

Complexes must have penultimate stress among those syllables not containing {\bf y}:  an unstressed syllable containing {\bf y} (or a syllable {\bf rr} serving as glue to a stressed CVV) may intervene between the stressed syllable and the unstressed final syllable.  The parser does know about the doubled vowel stress rule, and will not accept a complex with a CVV with a doubled vowel unless one of the vowels can carry the penultimate stress.  If a CVV with doubled vowel is followed by a CVV with optional monosyllable, the monosyllabic pronunciation is forced;  otherwise a CVV with optional monosyllable gives the speaker a choice about where to place the stress.  The parser determines where a predicate ends either by the occurrence of explicit stress or by the occurrence of a non-character from which it back-figures the location of the stress.  The only stresses in a predicate are its penultimate stress and optionally stresses in its borrowing djifoa (mandatory if the borrowing djifoa is followed by a pause).

A complex may not be followed without intervening space (a juncture doesn't help) by a vowel.  A complex may not be followed by whitespace followed by a connective.

An alternative formulation allows the formation of a complex from a sequence of cmapua units and predicate words in which the last item is a predicate word, with successive items 
separated by the ``word" {\bf zao}, optionally flanked on either side by whitespace or comma-marked pauses.  This form might be used to avoid borrowing djifoa.  I can also imagine its use to clarify the meaning of a complex by replacing its constituent djifoa with the corresponding predicates in full.\footnote{This is a proposal of John Cowan.}

Our general view is that the replacement of a djifoa in a complex by another djifoa for the same predicate or even by the full predicate linked with {\bf zao} gives another form for the same predicate word:  such forms should not appear as separate dictionary items.\footnote{For example, we think the word {\bf heirslicui}, molasses, presents difficulties for an English speaker and she might want to say {\bf hekryslicui} instead;  this is precisely the same word.}

\subsection{Phonetic Quotes and Parenthetical Expressions}

A phonetically valid quoted utterance begins with {\bf li} and ends with {\bf lu} with optional pauses after {\bf li} and before {\bf lu} and with the intervening text, which must be a phonetic utterance, optionally enclosed in double quotes (the quotes being between {\bf li} and {\bf lu}).  Replace {\bf li} and {\bf lu} with {\bf kie} and {\bf kiu} and optonal use of quotes with optional use of parentheses (opening with an open parenthesis, closing with a closing parenthesis, and you have the rule for spoken parenthetical expressions.\footnote{The use of punctuation here is a proposal of ours:  it is quite natural but did necessitate the phonetic parser knowing about these forms.}

We give examples:  {\bf li, la Djan, lu}; {\bf li ``la Djan", lu}; {\bf kie (ji cluva mi) kiu}.

\subsection{Phonetically valid utterances}

A phonetically valid unit phonetic utterance is a  phonetic name, phonetic structure word, marked alien text item, phonetic predicate word, quoted or parenthesized expression, hyphen, or ellipsis, possibly with initial whitespace.  A phonetic utterance is a sequence of unit phonetic utterances, explicit comma pauses, and items of terminal punctuation.  Any Loglan utterance must be a phonetic utterance:  of course it must also be grammatically correct, a matter for subsequent chapters.  \footnote{We note that CCV djifoa are alse unit phonetic utterances, strictly so that they may be quoted with {\bf liu}.  CVV djifoa are also cmapua, and CVC djifoa are also names, so they can be {\bf liu}-quoted without special ceremony.}

It is useful to be aware that the parser proceeds in effect in two passes:  it checks an entire utterance for phonetic validity, then checks whether it is grammatical in another pass.

\section{Historical and philosophical note}

Loglan phonetics is to our mind rather weird and wonderful.

James Jennings has commented that the choice to recognize word classes by patterns of consonants and vowels in the first instance was the ``original sin of the language".  Perhaps so, but once this sin was committed, it could not be undone without discarding everything and creating a different language.

1975 Loglan had a very simple procedure for resolution of words, but a method for construction of new predicates which was ultimately found unsatisfactory.  The great morphological revolution which consisted in introducing complex predicates as predicates built from djifoa and borrowings as all the phonetically acceptable predicates which were not complexes made the definition of the predicate much more complicated, and more of a challenge for the parser builder.

The problem of false name markers also led to a certain amount of excitement, once it was decided that {\bf la} was too common to ban from names.

We find Loglan phonetics charming as well as weird;  the language has a definite phonetic flavor and avoids monotonous regularity.  We would not say that it is always easy to pronounce, and its resemblance to a Romance language can be overstated:  it does allow quite a lot of consonant clustering.  A further charm is that the baroque rules, however arbitrary they may seem, actually work out as almost inevitable consequences of a fairly small number of design decisions.  We hope that we have given some hint in our discussion of what these design decisions were.

\chapter{The Grammar Proper}

In this section, we will present the grammar, pausing now and then to introduce word classes that are needed;  we will give short lists of commonly used words in the grammar text and full word lists in an appendix.

\section{Simple sentence shapes}

It is an interesting question where to start.  We will begin with the basic Loglan sentence, and work upward to more complicated utterances and other kinds of utterance fragments, and downward to sentence components.

The simplest kind of Loglan sentence is examplified by {\bf La Djan, kamla} (John comes) and {\bf La Djan, donsu le bakso, la Meris} (John gave Mary the box).  This is an S(VO) sentence:  each of these consists of a subject ({\bf la Djan} in both cases), a species of noun phrase, followed by a verb phrase, which is a verb ({\bf kamla}) in the first, {\bf donsu} in the other), followed optionally by a list of ``objects" (noun phrases, none in the first example, two, {\bf la Meris} (Mary) and {\bf le bakso} (the box) in the second.

We pause here to discuss grammatical terminology.  We have used the words ``noun" and ``verb"  though Loglan actually has no such word classes.  It does however have those functional roles, and we will use (hopefully with care) these words to help communicate what is going on in the grammar.  The words {\bf kamla}, {\bf donsu}, and {\bf bakso} all belong to the same word class, Loglan predicates, and can appear in any of the roles.  {\bf Ti bakso} (this is a box) is a sentence in which the very nounish (to the English mind) word {\bf bakso} appears...as the verb!  What we call a ``noun phrase" can also be called an ``argument" (terminology taken from logic and also already used in Loglan grammar).  Loglan grammarians have up until now used ``predicate" indifferently for predicate words and for what we call ``verbs" and ``verb phrases";  we will continue to use ``verb" and ``verb phrase" as grammatical terms.

We discuss the difference between this kind of basic sentence and atomic sentences of predicate logic.  An atomic sentence of predicate logic is of a form $Mx$ ($x$ is a man),
$Bxy$ ($x$ is bluer than $y$), $Gxyz$ ($x$ gives $y$ to $z$).  There is a predicate ($M$, $B$, $G$) and argument lists ($x$, $xy$, $xyz$).  A sentence like $Bxy$ might parse
$(B)(x)(y)$ or perhaps $(B)((x)(y))$;  the predicate and the individual arguments might be components at the same level or the predicate and the argument list might be components (the list further resolving into individual arguments).

The Loglan parse is different, in a way which brings Loglan closer to natural languages:  {\bf da mrenu}:  $xM$, {\bf da blanu de}:  $x((B)(y))$, {\bf da donsu de di}:  $(x)G((y)(z))$.  The weird thing here, from the standpoint of a logician, is the very special role of the subject:  the whole sentence breaks at the top into the noun phrase subject and the verb phrase containing the verb and  a list of the second and subsequent arguments.  The oddities of the way this breaks down become clearly important later when we discuss logically connected predicates.

\subsection{A brief review of components we use in example sentences}

To support examples, we should say something about the components of this sentence and what we are currently putting in for these components.

Any Loglan predicate word can play the role of the ``verb".  There are more complicated verbs than single predicate words, and we will see some possible additional complexities quite soon.  A name such as {\bf la Meris}, {\bf la Djan Braon} is eligible to be a noun phrase (either a subject or an object).  Pronouns such as {\bf ti}, {\bf ta} (this, that), or {\bf da, de, di, do du} (pronouns referring to recently mentioned noun phrases by a scheme we will discuss below), or letter names (referring to recently mentioned noun phrases with the given initial), or noun phrases built from predicates such as {\bf le mrenu}, (the man), are other possible arguments we may use before we have fully explained the range of possibilities for noun phrases.

\subsection{Changes of argument order and omission of arguments}

The sentence ``I am better than you" is expressed {\bf Mi gudbi tu}.  The sentence ``You are better than me" can of course be expressed {\bf Tu gudbi mi}, but it can also be expressed {\bf Mi nu gudbi tu}.  {\bf nu gudbi} is a verb, just as {\bf gudbi} is.  The effect of the particle {\bf nu} is to reverse the first and second arguments.  The particle {\bf fu}
interchanges the first and third arguments;  the particle {\bf ju} interchanges the first and fourth arguments.  Compounds are possible:  {\bf nufunu} interchanges the second and third arguments, for example:  {\bf La Meris, nufunu donsu la Djan, le bakso} (Mary gave John the box).  No one is going to carry out the transformation expressed by {\bf nufunu} in three separate steps in their head during a conversation:  it should be learned as a separate dictionary word.  But if you work it out step by step, you will find that that is what it does.  These contructions implement what would be ``passives" in other languages:  the Loglan term is ``conversion".

A variation implements ``reflexives".  {\bf nuo, fuo, juo} have the effect of eliminating the second, third, fourth argument, respectively, by supplying the subject as that argument.
{\bf La Meris, nuo donsu la Djan} (Mary gave herself to John) or {\bf la Meris, fuo donsu le bakso} (Mary gave herself the box) exemplify this transformation.  Compounds can be formed using the reflexives:  for example {\bf nufuonu} eliminates the third argument by identifying it with the second.

Any Loglan predicate word has a certain number of arguments, which have a certain order in the situation it represents, which can be seen in its dictionary entry.  Without special contrivances, you cannot supply the predicate with {\bf more} arguments.  But you can supply it with fewer arguments.  {\bf Mi gudbi tu} means ``I am better than you".
Just {\bf Mi gudbi} means ``I am good", with the underlying assertion being ``I am better than someone".  Another example {\bf La Meris, donsu le bakso}:  ``Mary gave the box away (gave it to someone)".

This can be combined with changes of argument order:  {\bf Mi nu gudbi}, ``I am bad" (I am worse than someone);  {\bf La Meris, nufunu donsu la Djan}:  ``Mary gave (something) to John.".  The argument omitted is always the last one, but if one changes the order of the arguments, one can put an argument one wishes to omit in the last position.

If the reader wonders why we introduce this transformation of verbs here, they should note two things:  as we will see later, this is one of the most tightly binding operations on verbs, and further, it acts exactly on the very simplest features of the structure of the simple Loglan sentence.

\subsection{Tenses and variations}

In this section we introduce tenses of the Loglan verb, which do not necessarily have anything to do with time.  Tense is achieved using a structure word (either {\bf ga} or a word of the PA class):  the simplest examples
are {\bf na}, {\bf pa}, {\bf fa}. the present, past, and future tenses.  A nontemporal examples is {\bf vi} (here).  There is also a null tense {\bf ga}, which is used in situations where it is grammatically useful to have a tense but we do not actually want to say anything about the time, place or conditions of the assertion.

{\bf la Meris, pa cluva la Djan} Mary loved John

{\bf Mi fa nufunu donsu tu}  I will give you something

{\bf La Ailin, vi danse}  Eileen dances here

{\bf Le mrenu ga sadji}  The man is wise (in general, no commitment to a particular time).  Here the {\bf ga} is grammatically a tense but it doesn't add anything to the semantics.  It is needed, because as we will see later, {\bf le mrenu sadji} is not a sentence, but a noun phrase, ``the wise man".

We aren't introducing tenses at length:  we actually need to introduce them in order to describe a further manipulation of basic sentences.  Notice that
{\bf nufunu donsu} is tensed, rather than {\bf fa donsu} being converted:  the tense is much more loosely attached than the conversion operator.  In fact, the tense
attaches to the verb phrase as a whole rather than to the verb.\footnote{It is actually possible to convert a sort-of-tensed verb but it is tricky:  {\bf Mi nufunu ge donsu je fa gue tu}, in which quite a lot is going on which we will not explain yet!}

\subsection{Variations in sentence order}

We can put the subject in a sentence after the verb in two ways.

  The first kind of sentence we can produce has a tensed verb phrase  with its objects (it might be tensed with {\bf ga} strictly for grammatical purposes) followed optionally by {\bf ga} then the subject:

{\bf Ga gudbi tu ga mi}  I am better than you

{\bf Ga gudbi tu}  There are better than you (here the subject is omitted!)

{\bf Ga donsu le bakso la Djan, ga la Meris}  Mary gave the box to John

{\bf Nia nu gudbi tu ga mi}  You are being better than me (combining conversion of the predicate with reordering of sentence components!)  The tense word {\bf nia} is the present progressive.

The second kind of sentence with subject delay consists of a tensed verb phrase with no objects followed by {\bf ga} then all the arguments in the sentence.

{\bf Ga donsu ga  la Meris, le bakso la Djan.}

This allows us to achieve VOS and VSO word orders.\footnote{It is a reform of ours to require that gasents (the Loglan jargon for subject-delayed sentences) must
have either exactly one argument delayed or all arguments delayed.  We want to avert listeners being forced to retroactively change their understanding of the meanings of arguments appearing earlier as in {\bf $^*$Ga donsu le bakso, ga mi tu}  in which one would reasonably start out thinking that {\bf le bakso} was the object being given (the second argument) but  the presence of two arguments after {\bf ga} (permitted in 1989 Loglan) forces the listener to revise this:  the actual meaning of the sentence would be ``I gave you to the box".}

We can put some objects before the verb, if we separate those objects from the subject with the particle {\bf gio}.

{\bf Mi gio le bakso ga donsu tu}  I give the box to you (the tense {\bf ga} is actually needed to keep from saying {\bf le bakso donsu}.the boxy giver), or

{\bf Mi gio le bakso tu ga donsu} 

This supports SOV(O) sentence order.

The particle {\bf gio} may optionally be used to set the subject apart from the objects in a VSO sentence:

{\bf Ga donsu ga  la Meris, gio  le bakso la Djan.}

There is another device for modifying sentence order by bringing objects to the front, but this device cannot be properly introduced until after we discuss logically connected sentences.

\subsection{Modifiers and tagged arguments}

Tense, location and modal operators (the same words which can decorate verb phrases as tenses) can form {\em sentence modifiers\/} which are rather like additional arguments
which can be supplied with any verb.  In English grammar, these would be ``prepositional phrases".

Relative modifiers and arguments (including the tagged arguments introduced below) are called terms.  The subject and the list of objects in a verb phrase are both
term lists (of slightly different kinds, as we will see).  The subject is an arbitrary list of terms containing at least one argument and no more than one untagged argument.  It is important to notice that if the list of terms before the verb in a sentence does not contain any arguments, the sentence will be either a gasent (if tensed) or an imperative (if not tensed).
The object list can contain no more than four untagged arguments (since there is no predicate taking more than five arguments).

Formally, a relative modifier is either a word of class PA (a tense, location or modal operator) followed by a noun phrase, followed optionally by the particle {\bf guua}\footnote{{\bf guua} and some other right closers are new;  we did a survey of grammatical constructions closable with {\bf gu} and provided special forms for most of them.} or the general right closer particle {\bf gu}, or simply a PA word followed optionally by {\bf gu}.  In the last kind of sentence modifier, you should suppose that the omitted argument
of the PA word is the present situation in which the speaker is delivering their speech (the referent of the pronoun {\bf tio}).

Lists of PA words will appear in the next section.

Examples of such modifiers:

{\bf Vi la Djan, mi bleka le nirda}  Near John I watched the bird

{\bf mi bleka le nirda, vi la Djan}

{\bf Mi godzi na}  I go now (here the ``now" is not a tense but a sentence modifier).

The relationship of a sentence modifier to the sentence is exactly the same no matter where it appears in the sentence.  It modifies the verb phrase, or equivalently, the entire situation represented by the sentence, not an argument it happens to be near.

{\bf Mi bleka le nirda vi la Djan}  means that I was near John when I was watching the bird.  

{\bf Mi bleka le nirda ji vi la Djan}  means that I was watching the bird which was near John (a different grammatical construction, the subordinate clause, which we have not seen yet).

A modifier or modifiers may appear before the {\bf ga} or tense in a subject-delayed sentence.

{\bf Na la Ven, pa kamla ga la Djan}  John will come at nine

Tagged arguments are arguments which are allowed to float free in a sentence in the same way that relative modifiers do.  This can be done with numerical place tags or case tags.

The numerical place tags {\bf zua}, {\bf zue}, {\bf zui}, {\bf zuo}, {\bf zuu} are signs of the first, second, third, fourth and fifth argument of a predicate.  This allows
arguments to be freely reordered and moreover allows medial arguments to be omitted.

{\bf Zui la Djan, donsu zua la Meris}  Mary gave (something) to John.

It also allows arguments to be placed with the subject, as long as at most one argument in the subject is untagged:

{\bf La Meris, zui la Djan, pa donsu le bakso}

It also allows more than one argument to be supplied for the same place.

{\bf Zua la Meris, zua la Djan, cluva la Ailin}  Mary and John love Eileen.

Untagged arguments are taken as usual to represent the places of the argument in order, skipping places corresponding to numerical place tags (or case tags) which have already appeared earlier in the sentence.  We do not require a listener or reader to displace a sequence of arguments when a {\bf zua} is encountered at the end of a sentence.  Numerical place tags have a special effect in term lists appearing before the particle {\bf gi}, which we will describe below.

The case tags are a bizarre idea which {\em we\/} would not have installed in this language.  This is not to say that similar ideas do not occur in natural languages.  With each place of a Loglan predicate, a {\em case\/} is associated in the dictionary, and that case tag may be used to reference that argument of that particular predicate in the same way the appropriate numerical case tag would reference it.  Another possible use of a case tag is to suggest an argument whose place the speaker has forgotten, or perhaps an argument of the predicate which does not appear in the dictionary!

There is a further issue that the dictionary includes words in which distinct arguments have the same case.  To support this, we have recently provided forms which reference the first, second, third, etc. argument of a given case.

The list of case tags will appear in the next section.

We give examples of use of these tags.

{\bf La Djan, dio la Meris, cluva}  John loves Mary.  This is another way to get SOV order, and notice that {\bf gio} is not needed.

{\bf Dio la Meris, (kao) la Djan, cluva}, with the same meaning.  The nominative case tag {\bf kao} is optional:  it can be used, which illustrates the fact that the subject needs to contain at least one argument
and at most one untagged argument (two tagged arguments are all right).

{\bf La Djan, pa donsu dio la Meris, le bakso}  John gave the box to Mary.  The {\bf dio} indicates which argument {\bf la Meris} is (she is not being given as a gift) and {\bf le bakso} falls tidily into the first unused argument place.

{\bf Dio la Meris, beu le bakso, donsu la Djan}  means the same thing as the previous sentence.

\subsection{Imperatives (and observatives)}

An untensed sentence consisting of a verb followed by an object list, with possibly some modifiers before the verb, is an imperative.

{\bf Donsu le bakso la Djan!}  Give the box to John.

{\bf Na la Ven, donsu le bakso la Djan!}  At nine, give the box to John.

If a tensed verb followed by an object list (possibly preceded by modifiers) is given, this is actually a subject-delayed sentence with the subject omitted.  We call this
an {\em observative\/}:  we note it as a special form mostly to indicate that such sentences are not imperatives.

{\bf Na crina}  It is raining (literally, someone is being rained on).  This is a shortening of {\bf Na crina ga ba}

{\bf Na donsu le bakso la Djan}  Someone is giving the box to John.

As an experiment, Loglan has borrowed a concept from Lojban and installed the imperative pronoun {\bf koo}.  This is used just like {\bf tu} (you) with the extra force that the usual
referent of {\bf tu} is commanded to make the statement true.

{\bf Koo donsu le bakso la Djan!}  Give the box to John!

but also

{\bf La Meris, cluva koo!}  Make Mary love you!

{\bf Mi jupni lepo koo gudbi!}  Make me think well of you! (lit.  Make me think you are good).

The imperatives with {\bf koo} are sentences of perfectly general structure and do not belong to the imperative grammatical class which is the subject of this section, usually, though consider

{\bf Cluva koo!}  Love yourself!

\section{Logically connected sentences}

\subsection{Forethought connected sentences}

\chapter{Lexicography Appendix:  full word lists}

\section{Case tags and indirect reference particles}

The case tags, including the positional ones are listed:  

\begin{description}

\item[beu:] (patients/parts), 

\item[cau:] (quantities/amounts/values), 

\item[dio:] (destinations/receivers), 

\item[foa:] (wholes/sets/collectives), 

\item[kao:] (actors/agents/doers), 

\item[jui:] (lessers),

\item[neu:] (conditions/circumstances/fields), 

\item[pou:] (products/purposes), 

\item[goa:] (greaters), 

\item[sau:] (sources/reasons/causes), 

\item[veu:] (effects/states/effects/deeds/means/routes), 

\item[zua:] (first argument), 

\item[zue:] (second argument), 

\item[zui:] (third argument), 

\item[zuo:] (fourth argument), 

\item[zuu:] (fifth argument), 

\item[lae:] (lae X = what is referred to by X), 

\item[lue:] (lue X = something which refers to X)

\end{description}  The operators of indirect reference {\bf lae} and {\bf lue} are a different sort of creature, which originally had the same grammar as case tags, but now have somewhat different behavior.   The latter two operators can be iterated (and so can case tags, probably indicating that more than one applies to the same argument).

For each semantic case tag there are forms like {\bf beuzi, beucine} to reference the first argument with that tag,  {\bf beuza, beucito} to reference the second argument with that tag,
and {\bf beuzu, beucite} to reference the third argument with that tag.  Forms like {\bf beucifo, beucife} are theoretically possible.

\chapter{The Formal Grammar in PEG Notation}

This chapter contains the actual Parsing Expression Grammar (PEG) notation in which the formal grammar is represented:  this is the source from which the computer parser is constructed.
It is also intended to be the basis of the presentation of the grammar.

I found a package which makes it so that this file can be included here in a way which does not run off the margins.  Lines are numbered, but this should not be taken too seriously.
Certainly any line number references in the text above should  be checked whenever the grammar file source is modified.  We do not have the text embedded in this file:  the current contents of the file are read in.

\tt

\lstinputlisting[numbers = left,breaklines=true]{draft-grammar-with-comments-alternative.peg}




\end{document}