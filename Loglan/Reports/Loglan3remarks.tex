\documentclass[12pt]{article}

\title{Commentary on Loglan 3}

\author{Randall Holmes}

\begin{document}

\maketitle

Stephen Rice's Loglan 3 is the best teaching material we have but it is somewhat out of date and Rice had at least one idea about the interpretation of the language with which I strongly disagree (as does the plain sense of language in Loglan 1) on logical grounds.  So I am providing commentary.

\section{Introduction}

\begin{description}

\item[p. 10:]  The eccentric letters {\bf q,x} are gone.  The letter {\bf h} has an alternative pronunciation as the final consonant in Scottish {\bf loch}, which is always used in syllable-final position (which is now possible).

\item[p. 11:] The irregular vowel {\bf y} can also be pronounced as the vowel in English {\em look\/} or as the Russian vowel bI.

Consonants {\bf m,n,l,r} used vocalically are now always doubled.

\item[p. 12:]  Glottal stop is now permitted between adjacent vowels forming two syllables (the pronunciation Rice describes is still legal).

When you want to force a syllable break (as in ``Lois") use a hyphen, not a close comma:  {\bf Lo-is}.  This can be used for any syllable break:  the new parser will read syllable breaks (and check for correctness).

There is a new series of lower case vowel letters {\bf zia, zie, zii, zio, ziu, ziy} which can be suffixed with {\bf -ma} to get upper case vowels.  The old ones are still supported.

Little words are now often called {\bf cmapua} in Loglan.

\item[p. 13:]  The statement that a compound little word must be penultimately stressed is {\em incorrect\/} (not out of date, it was always incorrect).  Stress on cmapua is completely free, with the remark that one must pause between a finally stressed cmapua and an immediately following predicate word (which he does allude to for one-syllable little words).  It is not an unreasonable style directive.

\item[p. 14:]  The charming {\bf guypli} would now be {\bf guhypli}.

\item[p. 15:]  The pause after a name {\em must\/} be written for the current parser.  This is a style point that could be changed.

Since glottal stop is now permitted between vowels forming a disyllable, the pause before a vowel initial word may be brief, but must be definite.

\item[p. 16:]  The pauses in a serial name are now of the same grade as the pause at the end of the name, and can, but need not be, represented by commas (whitespace is acceptable).

\end{description}

\section{Lesson 1}

\begin{description}

\item[no comments!]  Lesson 1 is perfect.

\end{description}

\section{Lesson 2}

\begin{description}

\item[p. 26:]  An imperative is created by omitting the first argument of a sentence and also omitting any tense on the verb.  We now view a tensed sentence without a subject as if its subject were {\bf ba} (the observative construction).  On p. 29, Rice says not to tense imperatives, for reasons he will announce later.

\end{description}

\section{Lesson 3}
\begin{description}
\item[p. 34:]  I do not believe in pronouncing punctuation.  {\bf kie X kiu} can be modified to {\bf kie (X) kiu} 
but not to {\bf (X)} for the current parser.

{\bf lie}  is simpler now.  Just {\bf lie house} not {\bf lie gleca house gleca}.

\item[p. 38:]  {\bf lie} now works differently.  It quotes a block of symbols following it (apart from comma or terminal punctuation at the end of the block);  whitespace when quoted is replaced by the little word {\bf y} set off from what surrounds it by whitespace.  {\bf lie house};  {\bf lie John y Brown}; {\bf lie War y and y Peace}.  The latter two
can be written {\bf lie ``John Brown"}; {\bf lie ``War and Peace"}, but this is pronounced the same, with {\bf y}.
One must pause before and after the alien text quoted.  One can also say {\bf lie War-and-Peace}, which includes no pauses.

\end{description}

\section{Lesson 4}

\begin{description}

\item[p. 49:]  I'm very dubious that there is a general principle that little words deducible from context can be omitted.  Test.
\item[p. 54:]  I don't think {\bf bie} is identifying.  I'm a mathematician and I think membership is a relation like any other.

{\bf hoi} may {\bf NOT} be omitted before regular names.  This leads to phonetic disaster.   This is a (necessary) later reform.



\end{description}

\section{Lesson 5}

\begin{description}

\item[p. 60:]  I have always thought {\bf mea} was useless (this was a disagreement I had with JCB) and now I can say so officially.  The change in meaning in {\bf me} when it is used as a modifier entirely covers the use of {\bf mea}.  {\bf Le meala Ford} is adequately captured by {\bf Le me la Ford, bekti}.  I think the parser still accepts {\bf mea} but this cannot be counted on to continue.

\end{description}

\section{Lesson 6}

\begin{description}

\item[p. 69:]  The parser now requires a pause before {\bf ha}.

\item[p. 72:]  There is an additional situation where a {\bf CVC} djifoa becomes {\bf CVCy}: it does so if it is followed by a consonant which makes a pair of consonants which could be initial in a complex.  {\bf tosmabru} is not permitted, correct to {\bf tosymabru}.  This is a later reform.  There could easily be predicates in the text which need to be corrected.

\end{description}

\section{Letter variables}

The scope and purpose of the {\bf gao} construction of special letters has changed.  {\bf gao azi} is currently not
well-formed.

\section{Lesson 7}

\begin{description}

\item[(volume 2) p. 5:]  It is better to use {\bf guu} to effect shared object arguments.

It is important to note a later reform:  words like {\bf ena}, {\bf epa} must be followed by a pause, or suffixed with {\bf fi}, which removes the need for the pause.  This applies to all logical and utterance connectives ending in PA or KOU words.

\item[p. 10:]  I repeat that my parser always requires pauses after regular Loglan names.

\item[p. 15:]  Insert commas after {\bf ena}, {\bf efa} and similar words.  Also, ``and later" is
{\bf efa}, not {\bf epa} (as I believe it is in Loglan).  This isn't an error:  it is an attempt at reform which was official and which I officially reversed.

\end{description}

\section{Lesson 8}

\begin{description}

\item[p. 17:]  TLI deprecates {\bf nigro}, black and hopes all users will instead use {\bf hekri}, black.  If it were derived from Spanish, that would be one thing.  But a large part of its score comes from English Negro, which is politically and semantically unacceptable.

\end{description}

\section{Lesson 9}
\begin{description}
\item[p. 27:]  Item 8 no longer works.  Pausing between {\bf le} and {\bf po} has no effect, 
and {\bf le, po ridcue ditca ga corta}  means that the event of teaching reading was short.
The best way to say what is wanted in item 8 is {\bf le poi ridcue ditca ga corta}:  {\bf poi} takes over all short scope uses of {\bf po}.  

\item[p. 31:]  It is a new rule not known to Rice that one must write an explicit comma pause after an argument {\bf liu} X.

\item[p. 31:]  The dialect Rice is teaching allows a lot more use of pauses to close grammatical constructions (what JCB called pause/{\bf gu} equivalence than ours does;  in fact, ours allows almost none.  Pauses in current Loglan generally are purely for phonetic reasons with very rare exceptions.

\item[p. 35, point 1:]  We repeat that this is now incorrect.  {\bf po} picks up an entire following sentence, pause or no pause (which may be missing its subject and so be a predicate).  Short scope is achieved by using {\bf poi} (and similar forms {\bf pui} {\bf zoi}), or by explicitly shortening the scope with {\bf guo}.

Further, if {\bf le} is followed by {\bf po} in a way which does not make an event abstraction
(as in {\bf le ge po ridcue guo ditca ga corta}, the reading teacher is short, the {\bf ge} is required.  This is quite unusual, and usually would involve replacing {\bf po} with {\bf poi} anyway.

\item[point 3:]  This is making a nice point but replace {\bf po} with {\bf poi} in all cases, and don't worry about pauses.



\end{description}

\begin{description}

\section{Lesson 10}

\item[p. 37:]  One needs to pause after the connectives such as {\bf ikou}, {\bf inuknou} as well as before
them.  The issue is the same as with {\bf ena} (and {\bf emou}) and kin.  One can also add the suffix {\bf -fi} and not need to pause.
I would suggest the {\bf fi} strategy for the logical connectives such as {\bf ena} and the pause strategy for the utterance connectives in this section.  All APA, AKOU, IPA, IKOU words follow this pattern (one must pause after them to shield the PA or KOU component from being misinterpreted, or guard it with {\bf -fi}).

\end{description}

\section{Lesson 11}

\begin{description}


\item[p. 54:]  A semantic point.  Sets do not carry logs.  Mass objects carry logs (things built with {\bf lo} or {\bf ze}.
All that a set does is have members.  I absolutely insist that it is {\bf Lo to mrenu} not {\bf Leu to mrenu} that carried the log.

NOTE ADDED:  there is another possibility.  Perhaps it is better for {\bf leu mrenu} to be an intended mass object composed of men, rather than a set.

\item[pp. 55:]   The various forms of we ({\bf mu}, {\bf muu} do not represent sets.  They represent mass objects.

The forms {\bf mo}, {\bf mio}, {\bf mio} are multiples (non-deignating arguments):  each of us do something.

The same remarks apply in the other persons.

We also note that we absolutely deny that {\bf mi} and {\bf tu} are singular:  they may certainly be plural.  No argument in the language is singular on its own.  What they are is less specific, and able to be singular.


\end{description}

\section{Lesson 12}

\begin{description}

\item[p. 66:]  Dimensioned numbers require {\bf mue} between the number and the unit.  {\bf lio tonimuekeigei}.

\item[p. 68:]  {\bf cao} emphasises the following word, but it will be a freemod attached to a previous construction in the parse.  It's just the way life is.

Another way of emphasising words is to use the explicit stress markers now available.  Stress on a syllable can be indicated by writing {\bf '} at the end of the syllable ({\em not\/} after the vowel) and emphatic stress (needed for a predicate whose main syllable is already stressed) using {\bf *} instead.

\end{description}

\section{Lesson 13}

\begin{description}

\item[p. 8:]   A set of men did not take the piano.  The {\bf -cu} suffix was defined off my watch, and I actually need to take time to figure out what if anything it actually means.  I think that {\bf Tocu mrenu pa tokna lemi pianfa} means that for some two men, the mass of those two men took the piano.

Rice is generally talking about sets in situations where what is being talked about is not sets.  Semantic cleanup is needed.  Grammar is probably fine.

\item[p. 9:]  I cannot agree with the reading of {\bf Tocu leu nema mrenu}.  Perhaps {\bf Tocu le nemacu mrenu}?

{\bf Lo pife le nema mrenu}, not {\bf Pife leu nema mrenu}.

{\bf sara} is a number, {\bf sarara} is a numerical predicate.  This was an ambiguity created by {\bf -ra} which I had to resolve.

\item[p. 10:]  I am fairly certain that if I accept {\bf cu} as sensible, it produces mass objects.



\end{description}

\section{Lesson 14}

\begin{description}

\item[p. 16:]  There is a general problem with Rice's treatment of negative arguments.  {\bf No la Meris} does not mean someone other than Mary.  When it is used, it says that the action of the sentence on Mary is false.

Sentences 1 and 2 mean the same thing (the meaning he gives for sentence 1.

Sentence 7 means that Mary did not go from Paris to Rome.  It does not assert any going to Rome at all.

Sentence 12 is a simple double negative and asserts that I am the father of someone through you.  There is no reference to anyone but me and you.  There is some subtletly about the quantifier (discussed in the section).

Sentence 13 is a double negative asserting that he went to Spain from France.

\item[p. 17:]  reiterating, point 2 is simply wrong.  A {\bf no} in front of an argument negates the entire sentence, and can be exported to the front of the sentence without changing the meaning.

\item[p. 18:]  Point 7 is wrong.



\end{description}

\section{Lesson 15}

\begin{description}

\item[p. 30:]  This affects a number of sentences appearing earlier.  My parser does not allow (and Im very doubtful that the grammar ever allowed) dropping {\bf goi} in front of keks.  Carefully put the {\bf goi}'s back in all the examples.  I believe this was an actual error.  (It might also mean I didnt digest some peculiar rule correctly in the old grammar, but in any case my grammar now requires the {\bf goi} in all cases.)

\end{description}

\section{Lesson 16}

\begin{description}

\item[p. 44:]  Point 8, along with any examples it refers to, has been changed.  If more than one argument appears before the predicate, the little word {\bf gio} must appear after the first one.

{\bf Mi gio la Djan, la Meris, farfu}

\item[p. 45:]  I have always found the rule for fronting arguments with {\bf gi} incredible.  I have changed it.
If the first argument in the fronted string is not case-tagged, it will be the first argument other than the subject not used once one has read the main part of the sentence.  If it is tagged, we know which argument it is and the arguments following it will be the ones following the tagged argument (skipping ones already used) in the natural order (unless they are themselves tagged, which can reset this).  In neither case does it cut to the last argument.  I'm not going to edit the example sentences.

\item[p. 48:]  Here Rice made a change in the language which we have all accepted.  The original sole purpose of
{\bf lao} was the construction of Linnaean names.  This was a very strange feature for a language to have.  The use of {\bf lao} that Rice proposes is a nice general purpose which includes the original one intended by JCB.

The replacement of whitespace with {\bf y} in multi block strings labelled with {\bf lao} happens just as with {\bf lie}.
This transformation was originally proposed by Bill Gober for Linnaean names.  We use it for {\bf lie} and {\bf sao} as well:  {\bf lao John y Brown}, {\bf sao ice y cream}

\item[p. 53:]  To reiterate, the block of arguments fronted with {\bf gi} is not determined by the last argument of the predicate.  It is either the first available block not including the subject of unused arguments of the right length, or a block starting in the position indicated by a case tag on the first argument (the position can be reset by another case tag, and {\bf zua} can be used:  so it can in this case be the first argument).

\end{description}

\end{document}