\documentclass[12pt]{article}

\title{Dictionary Notes}

\author{Randall Holmes}

\begin{document}

\maketitle

Here I am going to collect comments on the dictionary, a few of which may rise to the level of proposals for changes.  There are two main sources:  read the dictionary itself, and read discussions in the SL transcripts.

All the short paragraphs begin with ``entry" now, but when I start mining the SL transcripts we also might have ``new word" paragraphs.

entry \verb a  in the dictionary:  I don't like giving comma as a meaning of \verb a :  I would suggest including in the entry a statement with example that points out that
Loglan doesn't do this with connectives:  where English has {\tt A, B, or C} , we have {\tt A, a B, a C}.  The same remark would apply to other logical connectives.

It is useful to note (perhaps in a key) that references to Linnaean nomenclature are no longer essential to the language.  The Linnaean predicates are just borrowed predicates;
the Linnaean names are a class of foreign names.  I don't see any reason to remove the annotation, but it would be nice to have a remark somewhere that it is now only of historical interest.

entry \verb ai :  the form \verb -ai  as found in upper case consonant names is to my mind not even an affix;  I don't think it merits an entry, and if it does of course it would be a separate one.   What do others think?   I might be wrong here:  but certainly if it is an entry it would be a separate hyphenated one.

entry \verb aktci:  This is a challenging consonant combination.  It might be nasty if {\tt akci} were also a word.  I'm not suggesting a change;  I do enjoy noting now and then that Loglan is not actually always easy to pronounce.

entries (for example) \verb afi  \verb Ama   These entries should be revised.  The primary meaning of these words and their primary use is as pronouns.  \verb Ama  means capital A {\em only} in the special context {\tt lii Ama}, and this ought to be clear from the entry.  This applies to all entries for letterals.

entry \verb anhasiydzo  (duck-walk, to waddle)  I just like this word :-)

entry \verb arba  correct to Arabic language

entry {\tt Archaepteryx lithographica}:  since I now require the \verb y  in writing, perhaps it should be inserted in entries for full Linnaean names.

entry \verb ba :  This is a comment on the example:  {\tt ba crina} actually literally means ``someone is rained on".

I do not really  like the translation of \verb ba  as ``there is".  {\tt ba} is not the quantifier in the sentence {\tt toba mrenu}:  the entire sense of {\em there are two$\ldots$\/} is actually in the quantifier {\tt to}.

entry \verb baa  to be \verb -baa .

entry \verb bac  to be \verb -bac .

entry \verb bacfundi  :  what an odd entry!  I am not objecting, just amused.


entry \verb bad  to be \verb -bad .

entry \verb badlo :  just an idle thought.  Could this include a third argument indicating what the package is contained or wrapped in?  I see that \verb badmao  has this argument, but perhaps
\verb badlo  should too.

entry \verb bae  to be \verb -bae .

second entry \verb bai   to be \verb -bai .

entry \verb baj  to be \verb -baj .

entry \verb bajpro :  No criticism here, just a comment that the remark on the second argument is interesting.  Surely the second argument can be singular if one is interested (oddly) in a particular smaller branch coming from a larger one.  But usually one will be interested in the many smaller branches comeing from the larger one.


entry \verb bak  to be \verb -bak .

entry \verb bakduo :  is what one is bailing necessarily water?  A third argument could be provided in case one is bailing some other fluid ;-)

entry \verb bakso :  would a third argument indicating what the box is made of be amiss?  

entry \verb bakto :  same remark, a third argument indicating what the bucket is made of might not be amiss.

entry \verb bakvla:  no criticism:  this is a striking consonant cluster :-)

entry \verb bal  to be \verb -bal .

entry \verb balbiu :  this is just a speculative remark.   This word for poised/balanced behavior has two different pronunciations.  I wonder whether in a living Loglan these two forms would acquire different connotations.  One you say quickly, the other you linger over.

entry \verb balma :  I wonder whether this word should have arguments (center and radius suggest themselves).  They might not be used often.

entry \verb balpi :  I suggest attention to argument structure here.  A balance is around a point of balance as well as with respect to forces.

entry \verb bam  to be \verb -bam .

entry \verb bamcli :  same remark about lack of argument structure.

entry \verb bamfoa :  I notice that this has the arguments I missed in {\tt balma}.  But still, perhaps {\tt balma} ought to have them.

entry \verb bamfomcartu :  this is a wonderfully strange word (no criticism).

entry \verb bamhanco :  this predicate is certainly missing a second argument, the person or animal whose fist it is.  This is a necessary correction.  {\tt nu bamhanco} is then a person or animal making a fist :-)  Cyril suggests a third argument, the paw or hand used.

entry \verb bamkuspypae :  another wonderfully strange word.  I thought carelessly thet it refers to a toilet paper roll, but actually it refers to certain antiperspirant applicators or to ballpoint pens.  It is fun.

entry \verb ban  to be \verb -ban .

entry \verb banci :  I sense two possible meanings for the second argument.  I suggest X takes a bath in tub Y in fluid Z -- taking a bath in the claw tub is not the same thing as taking a bath in milk:  add a third argument to distinguish these two issues.

entry \verb bangoi:  No criticism, but this has interesting semantics.

entry \verb banhane  should a fruit have the tree it comes from as an argument?

entry \verb bao  should be \verb -bao .

entries \verb baorduo , \verb baormao  exhibit the pronounciation issue for English speakers which I support by allowing alternative forms \verb baorrduo , \verb baorrmao ; under some circumstances the {\tt r} hyphen can be vocalic and so an extra syllable.  There is no need to comment on this in the dictionary, of course.

entry \verb bap   should be \verb -bap .

entry  \verb bapduo  should probably have a third argument for the bottles.

entry \verb bar  should be \verb -bar .

entry \verb bardua  No intention to make a change, but wouldnt it be fun if we had a word for bangle which took the person it adorns as an argument...so it is not a bangle if it is not bangling :-)

entry \verb barteu  It is notable how the English definition supports a metaphorical use of ``reach" here.

entry \verb bas  should be \verb -bas .

entry \verb bat  should be \verb -bat .

entry \verb batfantoa   I question the argument order:  rebate for what is one of the main questions one would ask.

entry \verb batmi  is really interesting because it has arguments which clearly have exactly the same case.

entry  \verb batmra  has a fun consonant combination.


entry \verb batpi  could have a third argument indicating what the bottle is made of.

entry \verb be :  I have quibbles about the first and last lines of this definition.

entry \verb bed-  has the hyphen in my copy but not in Torrua's.   Cyril notes that in the case of djifoa these hyphens could be added automatically without the need to change dictionary keys.

entry \verb bedgoi   Should this have an argument for which bed you are going to?  I think perhaps yes.

entry \verb bedhedto  This should definitely have a second argument.  The phrasing of the entry even suggests it.  X is head of bed Y.

entry \verb bedkru  Should there be a third argument for the bed or beds which make it a bedroom?

entry \verb bedpli  The second argument which is the bed or thing used as a bed is there in my mind.


entry \verb beg-   needs the hyphen

entry \verb begdou   can the second argument be an action as in \verb begco ?

entry \verb begnordou  is the second argument the object or action denied to the would be recipient (which seems better) or the actual petition (an act or event) as the phrasing of the definition seems to suggest?  I suggest parallelism with \verb begdou .

entry \verb bei  As noted above, clarify all such entries to make it clear that their primary use is as pronouns (lower case standing for descriptive arguments rather than proper names).

entry \verb beinracveo  It might be too baroque but I'd like a third argument for what the luggage is made of.

entry \verb beinreu  Should this have an argument for the person or animal it is on?

entry \verb bek-  needs a hyphen

entry \verb bekciu   Just a snide remark:  these tomatoes have no peer suggests that they are best, but this word only provides that no tomatoes have the same qualities;
some might be better.  Is there an analogous word using $\geq$?  This is an interesting word.  I suppose the difference from \verb ciktu  is that the first two arguments are actually quantities.  But the last example under \verb ciktu  seems as if it should rather be an example under \verb bekciu .

entry \verb bekfao  Could this have an argument for a direction?

entry \verb beklia  Are second arguments individual objects in the formation, or are masses of objects in the formation also acceptable arguments?  I would think the former, but there are idioms which might threaten the latter usage.

entry \verb bekykrosio  I would think the items in the traffic and the field in which the traffic takes place should occupy separate argument places.  These are not really conflatable.

entry \verb bel-  needs hyphen.

entries \verb belcpu,  \verb belduo  Perhaps one of these should have a third argument for the sound (see \verb belsoa )

entry \verb ben-  needs hyphen

entry \verb bendu  An argument for the kind of music played?

entry \verb beo  Usual notes about rewriting entries about letterals to emphasize pronoun-hood.

entry \verb ber-  hyphenate

entry \verb bercykeu  an argument for the flock?

entry \verb bercymio  an argument for the victim (mutton from sheep S)

entry \verb berprati  the use of ``freight"  is not the usual one:  ``freight charge (or rate)"  would probably be better.

entry \verb bet-  hyphenate

entry \verb betytue  An elaborate word for a simple concept.  Just a comment.

entry \verb betytuebongu  an amazingly long word for kneecap!

entry \verb beu-  separate hyphenated entry for the djifoa

entry \verb bia-   separate entry for the djifoa

entries \verb biarskapi,   \verb biartorspa  examples of symmetric predicates with arguments with the same case

entries \verb biaski, \verb biaspa  why two words?  No action required, just an example of double coinage, which happens in various places, and is probably a natural phenomenon in Loglan.

entry \verb Biblos  I would rather say, the Christian scripture, and have a separate entry for {\tt la Tanah}, the Jewish scripture.

entry \verb bic-   hyphenate

entry \verb bid-  hyphenate

entry \verb bidvetfa  attention to argument structure?  This might reasonable have an additional argument, as for example for the field of practice in which the innovation is made?

entry \verb bidzi   I suppose that a bead which is self-sufficient, not on an ornament or garment, can be {\tt nuo bidzi}.

entry \verb bie-  separate hyphenated entry for the djifoa

entry \verb bierfodydjoso   This is a magnificent word.  One doesn't just hem a garment, one hems it in a particular place.  I think another argument place is needed.

















 








\end{document}