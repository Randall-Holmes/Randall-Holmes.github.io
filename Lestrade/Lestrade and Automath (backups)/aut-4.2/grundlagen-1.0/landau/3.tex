%\jutting{3}{1-69}

\line{}\vskip 7\baselineskip
\centerline{\sl Kapitel 3.}
\medskip

\centerline{\bf Schnitte.}
\bigskip

\centerline{{\S}~1.}
\medskip

\centerline{\bf Definition.}
\bigskip

{\bf Definition 28:} {\it Eine Menge von rationalen Zahlen hei{\ss}t Schnitt, wenn

1) sie eine rationale Zahl, aber nicht jede rationale Zahl enth\"alt;

2) jede rationale Zahl der Menge kleiner ist als jede nicht zur
Menge geh\"orige rationale Zahl;

3) in ihr keine gr\"o{\ss}te rationale Zahl vorkommt {\rm (d. h. Zahl, die
gr\"o{\ss}er als jede etwaige andere, von ihr verschiedene ist).}}

Man nennt auch die Menge Unterklasse, die Menge der nicht
in ihr enthaltenen rationalen Zahlen Oberklasse und redet ent%
sprechend von Unterzahlen und Oberzahlen.

Kleine griechische Buchstaben bedeuten durchweg, wenn nichts
anderes gesagt wird, Schnitte.
\medskip

%\jutting{3}{70-93}

{\bf Definition 29:} {\it $$\xi = \eta$$
{\rm (= sprich: gleich),} wenn jede Unterzahl bei $\xi$ Unterzahl bei $\eta$ und
jede Unterzahl bei $\eta$ Unterzahl bei $\xi$ ist.}

Mit anderen Worten: wenn die Mengen identisch sind.

{\it Anderenfalls
$$\xi \ne \eta$$
{\rm ($\ne$ sprich: ungleich).}}

Trivial sind die drei S\"atze:
\medskip

%\jutting{3}{94-94}

{\bf Satz 116:} {\it $$\xi = \xi.$$}%
\medskip

%\jutting{3}{95-96}

{\bf Satz 117:} {\it Aus
$$\xi = \eta$$
folgt
$$\eta = \xi.$$}%
\medskip

%\jutting{3}{97-98}

{\bf Satz 118:} {\it Aus
$$\xi = \eta,\quad \eta = \zeta$$
folgt
$$\xi = \zeta.$$}%
\medskip

%\jutting{3}{99-106}

{\bf Satz 119:} {\it Ist $X$ Oberzahl bei $\xi$ und
$$X_1 > X,$$
so ist $X_1$ Oberzahl bei $\xi$.}

{\bf Beweis:} Folgt aus 2) der Definition 28.
\medskip

%\jutting{3}{107-115}

{\bf Satz 120:} {\it Ist $X$ Unterzahl bei $\xi$ und
$$X_1 < X,$$
so ist $X_1$ Unterzahl bei $\xi$.}

{\bf Beweis:} Folgt aus 2) der Definition 28.

Nat\"urlich ist umgekehrt die Forderung des Satzes 120 mit
2) der Definition 28 identisch.  Um also von irgend einer Menge
rationaler Zahlen zu zeigen, da{\ss} sie ein Schnitt ist, gen\"ugt stets
der Nachweis von:

1) Sie ist nicht leer, und es gibt eine rationale Zahl, die nicht
darin liegt.

2) Mit jeder ihrer Zahlen geh\"ort jede kleinere dazu.

3) Zu jeder ihrer Zahlen gibt es in ihr eine gr\"o{\ss}ere.
\vfill\eject


%\jutting{3}{116-161}

\line{}\vskip 7\baselineskip
\centerline{{\S}~2.}
\medskip

\centerline{\bf Ordnung.}
\bigskip

{\bf Definition 30:} {\it Sind $\xi$ und $\eta$ Schnitte, so ist
$$\xi > \eta$$
{\rm ($>$ sprich: gr\"o{\ss}er als),} wenn es eine Unterzahl bei $\xi$ gibt, die Ober%
zahl bei $\eta$ ist.}
\medskip

%\jutting{3}{162-169}

{\bf Definition 31:} {\it Sind $\xi$ und $\eta$ Schnitte, so ist
$$\xi < \eta$$
{\rm ($<$ sprich: kleiner als),} wenn es eine Oberzahl bei $\xi$ gibt, die Unter%
zahl bei $\eta$ ist.}
\medskip

%\jutting{3}{170-176}

{\bf Satz 121:} {\it Aus
$$\xi > \eta$$
folgt
$$\eta < \xi.$$}%

{\bf Beweis:} Es gibt eben eine Oberzahl bei $\eta$, die Unterzahl bei
$\xi$ ist.
\medskip

%\jutting{3}{177-183}

{\bf Satz 122:} {\it Aus
$$\xi < \eta$$
folgt
$$\eta > \xi.$$}%

{\bf Beweis:} Es gibt eben eine Unterzahl bei $\eta$, die Oberzahl bei
$\xi$ ist.
\medskip

%\jutting{3}{184-215}

{\bf Satz 123:} {\it Sind $\xi$, $\eta$ beliebig, so liegt genau einer der F\"alle
$$\xi = \eta,\quad \xi > \eta,\quad \xi < \eta$$
vor.}

{\bf Beweis:} 1)
$$\xi = \eta,\quad \xi > \eta$$
sind unvertr\"aglich nach Definition 29 und Definition 30.

$$\xi = \eta,\quad \xi < \eta$$
sind unvertr\"aglich nach Definition 29 und Definition 31.

Aus
$$\xi > \eta,\quad \xi < \eta$$
w\"urde folgen, da{\ss} es eine Unterzahl $X$ bei $\xi$ gibt, die Oberzahl
bei $\eta$ ist, und eine Oberzahl $Y$ bei $\xi$, die Unterzahl bei $\eta$ ist.
Nach 2) der Definition 28 w\"are also zugleich
$$X < Y,\quad X > Y.$$

Folglich liegt h\"ochstens einer der drei F\"alle vor.

2) Ist
$$\xi \ne \eta,$$
so stimmen die Unterklassen nicht \"uberein.  Also ist entweder eine
gewisse Unterzahl bei $\xi$ Oberzahl bei $\eta$ und alsdann
$$\xi > \eta,$$
oder eine gewisse Unterzahl bei $\eta$ Oberzahl bei $\xi$ und alsdann
$$\xi < \eta.$$
\medskip

%\jutting{3}{216-216}

{\bf Definition 32:} {\it $$\xi \ge \eta$$
bedeutet
$$\xi > \eta\quad\hbox{\it oder}\quad \xi = \eta.$$
{\rm ($\ge$ sprich: gr\"o{\ss}er oder gleich.)}}
\medskip

%\jutting{3}{217-217}

{\bf Definition 32:} {\it $$\xi \le \eta$$
bedeutet
$$\xi < \eta\quad\hbox{\it oder}\quad \xi = \eta.$$
{\rm ($\le$ sprich: kleiner oder gleich.)}}
\medskip

%\jutting{3}{218-219}

{\bf Satz 124:} {\it Aus
$$\xi \ge \eta$$
folgt
$$\eta \le \xi.$$}%

{\bf Beweis:} Satz 121.
\medskip

%\jutting{3}{220-268}

{\bf Satz 125:} {\it Aus
$$\xi \le \eta$$
folgt
$$\eta \ge \xi.$$}%

{\bf Beweis:} Satz 122.
\medskip

%\jutting{3}{269-281}

{\bf Satz 126} (Transitivit\"at der Ordnung): {\it Aus
$$\xi < \eta,\quad \eta < \zeta$$
folgt
$$\xi < \zeta.$$}%

{\bf Beweis:} Es gibt eine Oberzahl $X$ bei $\xi$, die Unterzahl bei $\eta$
ist; und eine Oberzahl $Y$ bei $\eta$, die Unterzahl bei $\zeta$ ist.  Wegen
Schnitteigenschaft 2) von $\eta$ ist
$$X < Y,$$
also $Y$ Oberzahl bei $\xi$.  Daher ist
$$\xi < \zeta.$$
\medskip

%\jutting{3}{282-289}

{\bf Satz 127:} {\it Aus
$$\xi \le \eta,\quad \eta < \zeta\quad\hbox{\it oder}\quad \xi < \eta,\quad \eta \le \zeta$$
folgt
$$\xi < \zeta.$$}%

{\bf Beweis:} Mit dem Gleichheitszeichen in der Voraussetzung klar;
sonst durch Satz 126 erledigt.
\medskip

%\jutting{3}{290-303}

{\bf Satz 128:} {\it Aus
$$\xi \le \eta,\quad \eta \le \zeta$$
folgt
$$\xi \le \zeta.$$}%

{\bf Beweis:} Mit zwei Gleichheitszeichen in der Voraussetzung
klar; sonst durch Satz 127 erledigt.
\vfill\eject


%\jutting{3}{304-366}

\line{}\vskip 7\baselineskip
\centerline{{\S}~3.}
\medskip

\centerline{\bf Addition.}
\bigskip

{\bf Satz 129:} {\it {\rm I)} Es seien $\xi$ und $\eta$ Schnitte.  Dann ist die Menge
der rationalen Zahlen, die sich in der Form $X + Y$ darstellen lassen,
wo $X$ Unterzahl bei $\xi$, $Y$ Unterzahl bei $\eta$ ist, ein Schnitt.

{\rm II)} Keine Zahl dieser Menge l\"a{\ss}t sich als Summe einer Oberzahl
bei $\xi$ und einer Oberzahl bei $\eta$ darstellen.}

{\bf Beweis:} 1) Geht man von irgend einer Unterzahl $X$ bei $\xi$ und
irgend einer Unterzahl $Y$ bei $\eta$ aus, so geh\"ort $X+ Y$ zur Menge.

Geht man von irgend einer Oberzahl $X_1$ bei $\xi$ und irgend einer
Oberzahl $Y_1$ bei $\eta$ aus, so ist f\"ur alle Unterzahlen $X$ bzw. $Y$ bei
$\xi$ bzw. $\eta$
$$X < X_1,\quad Y < Y_1,$$
also
$$X + Y < X_1 + Y_1,$$
$$X_1 + Y_1 \ne X + Y;$$
$X_1 + Y_1$ geh\"ort also nicht zur Menge.  Und II) ist schon mitbewiesen.

2) Es ist zu zeigen, da{\ss} jede Zahl, die kleiner als eine Zahl
der Menge ist, auch zur Menge geh\"ort.  Es sei also $X$ Unterzahl
bei $\xi$, $Y$ Unterzahl bei $\eta$ und
$$Z < X + Y.$$
Dann ist
$$(X + Y){Z \over X + Y} < (X + Y) \cdot 1,$$
also nach Satz 106
$${Z \over X + Y} < 1,$$
also nach Satz 105
$$X{Z \over X + Y} < X \cdot 1 = X$$
und
$$Y{Z \over X + Y} < Y \cdot 1 = Y;$$
nach der zweiten Schnitteigenschaft bei $\xi$ bzw. $\eta$ ist also $X{Z \over X + Y}$
bzw. $Y{Z \over X + Y}$ Unterzahl bei $\xi$ bzw. $\eta$.

Die Summe dieser beiden rationalen Zahlen ist das gegebene
$Z$, wegen

$$X{Z \over X + Y} + Y{Z \over X + Y} = (X + Y){Z \over X + Y} = Z.$$

3) Ist eine Zahl der Menge gegeben, so hat sie die Form
$X + Y$, wo $X$ Unterzahl bei $\xi$, $Y$ Unterzahl bei $\eta$ ist.  Man w\"ahle
nach der dritten Schnitteigenschaft eine Unterzahl
$$X_1 > X$$
bei $\xi$; dann ist
$$X_1 + Y > X + Y,$$
also eine Zahl der Menge $> X + Y$ vorhanden.
\medskip

%\jutting{3}{367-384}

{\bf Definition 34:} {\it Der in Satz 129 konstruierte Schnitt hei{\ss}t $\xi + \eta$
{\rm ($+$ sprich: plus).}  Er hei{\ss}t auch die Summe von $\xi$ und $\eta$ oder der
durch Addition von $\eta$ und $\xi$ entstehende Schnitt.}
\medskip

%\jutting{3}{385-392}

{\bf Satz 130} (kommutatives Gesetz der Addition):
{\it $$\xi + \eta = \eta + \xi.$$}%

{\bf Beweis:} Jedes $X + Y$ ist auch $Y + X$ und umgekehrt.
\medskip

%\jutting{3}{393-409}

{\bf Satz 131} (assoziatives Gesetz der Addition):
{\it $$(\xi + \eta) + \zeta = \xi + (\eta + \zeta).$$}%

{\bf Beweis:} Jedes $(X + Y) + Z$ ist auch $X + (Y+ Z)$ und umgekehrt.
\medskip

%\jutting{3}{410-478}

{\bf Satz 132:} {\it Bei jedem Schnitt gibt es, wenn $A$ gegeben ist, eine
Unterzahl $X$ und eine Oberzahl $U$ mit
$$U - X = A.$$}%

{\bf Beweis:} $X_1$ sei irgend eine Unterzahl.  Wir betrachten alle
rationalen Zahlen
$$X_1 + nA,$$
wo $n$ ganz ist.  Sie sind nicht lauter Unterzahlen; denn ist Y
irgend eine Oberzahl, so ist
$$Y > X_1$$
also nach Satz 115 bei passendem $n$
$$nA > Y - X_1,$$
$$X_1 + nA > (Y - X_1) + X_1 = Y,$$
also $X_1 + nA$ Oberzahl.

In der Menge der $n$, f\"ur die $X_1 + nA$ Oberzahl ist, gibt es
nach Satz 27 eine kleinste ganze Zahl; sie hei{\ss}e $u$.

Ist
$$u = 1,$$
so setze man
$$X = X_1,\quad U = X_1 + A;$$
ist
$$u > 1,$$
so setze man
$$X = X_1 + (u - 1)A,\quad U = X_1 + uA = X + A.$$
Jedesmal ist $X$ Unterzahl, $U$ Oberzahl und
$$U - X = A.$$
\medskip

%\jutting{3}{479-488}

{\bf Satz 133:} {\it $$\xi + \eta > \xi.$$}%

{\bf Beweis:} $Y$ sei eine Unterzahl bei $\eta$.  Nach Satz 132 w\"ahle
man eine Unterzahl $X$ bei $\xi$ und eine Oberzahl $U$ bei $\xi$ mit
$$U - X = Y$$
dann ist
$$U = X + Y$$
Oberzahl bei $\xi$ und Unterzahl bei $\xi + \eta$.  Daher ist
$$\xi + \eta > \xi.$$
\medskip

%\jutting{3}{489-504}

{\bf Satz 134:} {\it Aus
$$\xi > \eta$$
folgt
$$\xi + \zeta > \eta + \zeta.$$}%

{\bf Beweis:} Es gibt eine Oberzahl $Y$ bei $\eta$, die Unterzahl bei $\xi$
ist.  Man w\"ahle eine gr\"o{\ss}ere Unterzahl
$$X > Y$$
bei $\xi$; $X$ ist also Oberzahl bei $\eta$.  Nach Satz 132 w\"ahle man bei
$\zeta$ eine Oberzahl $Z$ und eine Unterzahl $U$ mit
$$Z - U = X - Y.$$
Dann ist
$$Y + Z = Y + ((X - Y) + U) = (Y + (X - Y)) + U = X + U,$$
also Unterzahl bei $\xi + \zeta$ und (nach Satz 129, II)) Oberzahl bei $\eta + \zeta$.
Daher ist
$$\xi + \zeta > \eta + \zeta.$$
\medskip

%\jutting{3}{505-519}

{\bf Satz 135:} {\it Aus
$$\xi > \eta\quad\hbox{\it bzw.}\quad \xi = \eta\quad\hbox{\it bzw.}\quad \xi < \eta$$
folgt
$$\xi + \zeta > \eta + \zeta\quad\hbox{\it bzw.}\quad \xi + \zeta = \eta + \zeta\quad\hbox{\it bzw.}\quad \xi + \zeta < \eta + \zeta.$$}%

{\bf Beweis:} Der erste Teil ist Satz 134, der zweite klar, der
dritte eine Folge des ersten wegen
$$\eta > \xi,$$
$$\eta + \zeta > \xi + \zeta,$$
$$\xi + \zeta < \eta + \zeta.$$
\medskip

%\jutting{3}{520-535}

{\bf Satz 136:} {\it Aus
$$\xi + \zeta > \eta + \zeta\quad\hbox{\it bzw.}\quad \xi + \zeta = \eta + \zeta\quad\hbox{\it bzw.}\quad \xi + \zeta < \eta + \zeta$$
folgt
$$\xi > \eta\quad\hbox{\it bzw.}\quad \xi = \eta\quad\hbox{\it bzw.}\quad \xi < \eta.$$}%

{\bf Beweis:} Folgt aus Satz 135, da die drei F\"alle beide Male
sich ausschlie{\ss}en und alle M\"oglichkeiten ersch\"opfen.
\medskip

%\jutting{3}{536-543}

{\bf Satz 137:} {\it Aus
$$\xi > \eta,\quad \zeta > \upsilon$$
folgt
$$\xi + \zeta > \eta + \upsilon.$$}%

{\bf Beweis:} Nach Satz 134 ist
$$\xi + \zeta > \eta + \zeta$$
und
$$\eta + \zeta = \zeta + \eta > \upsilon + \eta = \eta + \upsilon,$$
also
$$\xi + \zeta > \eta + \upsilon.$$
\medskip

%\jutting{3}{544-551}

{\bf Satz 138:} {\it Aus
$$\xi \ge \eta,\quad \zeta > \upsilon\quad\hbox{\it oder}\quad \xi > \eta,\quad \zeta \ge \upsilon$$
folgt
$$\xi + \zeta > \eta + \upsilon.$$}%

{\bf Beweis:} Mit dem Gleichheitszeichen in der Voraussetzung
durch Satz 134, sonst durch Satz 137 erledigt.
\medskip

%\jutting{3}{552-564}

{\bf Satz 139:} {\it Aus
$$\xi \ge \eta,\quad \zeta \ge \upsilon$$
folgt
$$\xi + \zeta \ge \eta + \upsilon.$$}%

{\bf Beweis:} Mit zwei Gleichheitszeichen in der Voraussetzung
klar; sonst durch Satz 138 erledigt.
\medskip

%\jutting{3}{565-706}

{\bf Satz 140: Ist
$$\xi > \eta,$$
so hat
$$\eta + \upsilon = \xi$$
genau eine L\"osung $\upsilon$.}

{\bf Vorbemerkung:} F\"ur
$$\xi \le \eta$$
gibt es nach Satz 133 keine L\"osung.

{\bf Beweis:} I) Es gibt h\"ochstens eine L\"osung; denn f\"ur
$$\upsilon_1 \ne \upsilon_2$$
ist nach Satz 135
$$\eta + \upsilon_1 \ne \eta + \upsilon_2.$$

II) Ich zeige zun\"achst, da{\ss} die Menge der rationalen Zahlen
der Form $X - Y$ (also $X > Y$), wo $X$ Unterzahl bei $\xi$, $Y$ Oberzahl
bei $\eta$ ist, einen Schnitt bildet.

1) Wir wissen aus dem Anfang des Beweises des Satzes 134,
da{\ss} es ein solches $X - Y$ gibt.

Keine Oberzahl $X_1$ bei $\xi$ ist ein solches $X - Y$; denn f\"ur jede
Zahl dieser form ist
$$X - Y < (X - Y) + Y = X < X_1.$$

2) Ist ein $X - Y$ obiger Art gegeben und
$$U < X - Y,$$
so ist
$$U + Y < (X - Y) + Y = X,$$
also
$$U + Y = X_2$$
Unterzahl bei $\xi$,
$$U = X_2 - Y$$
zu unserer Menge geh\"orig.

3) Ist ein $X - Y$ obiger Art gegeben, so w\"ahle man bei $\xi$ eine
Unterzahl
$$X_3 > X.$$
Dann ist
$$(X_3 - Y) + Y > (X - Y) + Y,$$
$$X_3 - Y > X - Y,$$
also $X_3 - Y$ eine gr\"o{\ss}ere Zahl unserer Menge als die gegebene
$X - Y$.

Unsere Menge ist also ein Schnitt; er hei{\ss}e $\upsilon$.

Von ihm werden wir
$$\eta + \upsilon = \xi$$
beweisen. Hierzu gen\"ugt es, zweierlei zu zeigen:

A) Jede Unterzahl bei $\upsilon + \eta$ ist Unterzahl bei $\xi$.

B) Jede Unterzahl bei $\xi$ ist Unterzahl bei $\upsilon + \eta$.

Ad A) Jede Unterzahl bei $\upsilon + \eta$ hat die Form
$$(X - Y) + Y_1,$$
wo $X$ Unterzahl bei $\xi$, $Y$ Oberzahl bei $\eta$, $Y_1$ Unterzahl bei $\eta$ und
$$X > Y$$
ist.  Nun ist
$$Y > Y_1,$$
$$\bigl((X - Y) + Y_1\bigr) + (Y - Y_1) = (X - Y) + \bigl(Y_1 + (Y - Y_1)\bigr) = (X - Y) + Y = X,$$
$$(X - Y) + Y_1 < X,$$
also $(X - Y) + Y_1$ Unterzahl bei $\xi$.

\ifx\fr\undefined
  \font\teneufm=eufm10 \font\seveneufm=eufm7 \font\fiveeufm=eufm5
  \csname newfam\endcsname\eufmfam
  \textfont\eufmfam=\teneufm \scriptfont\eufmfam=\seveneufm \scriptscriptfont\eufmfam=\fiveeufm
  \def\fr{\fam\eufmfam}
\fi

Ad B) $\fr a$) Die gegebene Unterzahl bei $\xi$ sei zugleich Oberzahl
bei $\eta$ und hei{\ss}e alsdann $Y$.  Man w\"ahle eine Unterzahl $X$ bei $\xi$ mit
$$X > Y$$
und nach Satz 132 bei $\eta$ eine Unterzahl $Y_1$ und eine Oberzahl
$Y_2$ mit
$$Y_2 - Y_1 = X - Y.$$
Dann ist
$$Y > Y_1,$$
also
$$\displaylines{Y_2 + (Y - Y_1) = \bigl((X - Y) + Y_1\bigr) + (Y - Y_1) = (X - Y) + \bigl(Y_1 + (Y - Y_1)\bigr)\cr
= (X - Y) + Y = X,\cr}$$
$$Y - Y_1 = X - Y_2,$$
$$Y = (Y - Y_1) + Y_1 = (X - Y_2) + Y_1,$$
also $Y$ Unterzahl bei $\upsilon + \eta$.

$\fr b$) Ist die gegebene Unterzahl bei $\xi$ Unterzahl bei $\eta$, so ist
sie kleiner als jede in $\fr a$) als Unterzahl bei $\upsilon + \eta$ nachgewiesene
rationale Zahl, also selbst Unterzahl bei $\upsilon + \eta$.
\medskip

{\bf Definition 35:} {\it Das $\upsilon$ des Satzes 140 hei{\ss}t $\xi - \eta$ {\rm ($-$ sprich:
minus).}  $\xi - \eta$ hei{\ss}t auch die Differenz $\xi$ minus $\eta$ oder der durch
Subtraktion des $\eta$ von $\xi$ entstehende Schnitt.}
\vfill\eject


%\jutting{3}{707-790}

\line{}\vskip 7\baselineskip
\centerline{{\S}~4.}
\medskip

\centerline{\bf Multiplikation.}
\bigskip

{\bf Satz 141:} {\it {\rm I)} Es seien $\xi$ und $\eta$ Schnitte.  Dann ist die Menge
der rationalen Zahlen, die sich in der Form $XY$ schreiben lassen, wo
$X$ Unterzahl bei $\xi$, $Y$ Unterzahl bei $\eta$ ist, ein Schnitt.

{\rm II)} Keine Zahl dieser Menge l\"a{\ss}t sich als Produkt einer Oberzahl
bei $\xi$ und einer Oberzahl bei $\eta$ darstellen.}

{\bf Beweis:} 1) Geht man von irgend einer Unterzahl $X$ bei $\xi$ und
irgend einer Unterzahl $Y$ bei $\eta$ aus, so geh\"ort $XY$ zur Menge.

Geht man von irgend einer Oberzahl $X_1$ bei $\xi$ und irgend einer
Oberzahl $Y_1$ bei $\eta$ aus, so ist f\"ur alle Unterzahlen $X$ bzw. $Y$ bei
$\xi$ bzw. $\eta$
$$X < X_1,\quad Y < Y_1,$$
also
$$XY < X_1 Y_1,$$
$$X_1 Y_1 \ne XY;$$
$X_1 Y_1$ geh\"ort also nicht zur Menge.  Und II) ist schon mitbewiesen.

2) Es sei $X$ Unterzahl bei $\xi$, $Y$ Unterzahl bei $\eta$ und
$$Z < XY.$$
Dann ist
$$X\left({1 \over X}Z\right) = \left(X{1 \over X}\right)Z = 1 \cdot Z = Z,$$
$${Z \over X} = {1 \over X}Z < {1 \over X}(XY) = \left({1 \over X}X\right)Y = Y,$$
also $Z \over X$ Unterzahl bei $\eta$.  Die Gleichung
$$Z = X{Z \over X}$$
zeigt also, da{\ss} $Z$ zu unserer Menge geh\"ort.

3) Ist eine Zahl der Menge gegeben, so hat sie die Form $XY$,
wo $X$ Unterzahl bei $\xi$, $Y$ Unterzahl bei $\eta$ ist.  Man w\"ahle bei $\xi$
eine Unterzahl
$$X_1 > X;$$
dann ist
$$X_1 Y > XY,$$
also eine Zahl der Menge $> XY$ vorhanden.
\medskip

{\bf Definition 36:} {\it Der in Satz 141 konstruierte Schnitt hei{\ss}t $\xi \cdot \eta$
{\rm ($\cdot$ sprich: mal; aber man scbreibt den Punkt meist nicht).}  Er hei{\ss}t
auch das Produkt von $\xi$ mit $\eta$ oder der durch Multiplikation von $\xi$
mit $\eta$ entstehende Schnitt.}
\medskip

%\jutting{3}{791-798}

{\bf Satz 142} (kommutatives Gesetz der Multiplikation):
{\it $$\xi\eta = \eta\xi.$$}%

{\bf Beweis:} Jedes $XY$ ist auch $YX$ und umgekehrt.
\medskip

%\jutting{3}{799-815}

{\bf Satz 143} (assoziatives Gesetz der Multiplikation):
{\it $$(\xi\eta)\zeta = \xi(\eta\zeta).$$}%

{\bf Beweis:} Jedes $(XY)Z$ ist auch $X(YZ)$ und umgekehrt.
\medskip

%\jutting{3}{816-854}

{\bf Satz 144} (distributives Gesetz):
{\it $$\xi(\eta + \zeta) = \xi\eta + \xi\zeta.$$}%

{\bf Beweis:} I) Jede Unterzahl bei $\xi(\eta + \zeta)$ ist
$$X(Y + Z) = XY + XZ,$$
wo $X$, $Y$, $Z$ bzw. Unterzahlen bei $\xi$, $\eta$, $\zeta$ sind.  Die Zahl $XY + XZ$
ist Unterzahl bei $\xi\eta + \xi\zeta$.

II) Jede Unterzahl bei $\xi\eta + \xi\zeta$ hat die Form
$$XY + X_1 Z,$$
wo $X$, $Y$, $X_1$, $Z$ bzw. Unterzahlen bei $\xi$, $\eta$, $\xi$, $\zeta$ sind.  Im Falle
$X \ge X_1$ sei die Zahl $X$, im Falle $X \le X_1$ die Zahl $X_1$ mit $X_2$ be%
zeichnet.  Dann ist $X_2$ Unterzahl bei $\xi$, also $X_2 (Y + Z)$ Unterzahl
bei $\xi(\eta + \zeta)$.  Aus
$$XY \le X_2 Y,$$
$$X_1 Z \le X_2 Z$$
folgt
$$XY + X_1 Z \le X_2 Y + X_2 Z = X_2 (Y + Z);$$
also ist $XY + X_1 Z$ Unterzahl bei $\xi(\eta + \zeta)$.
\medskip

%\jutting{3}{855-874}

{\bf Satz 145:} {\it Aus
$$\xi > \eta\quad\hbox{bzw.}\quad \xi = \eta\quad\hbox{bzw.}\quad \xi < \eta$$
folgt
$$\xi\zeta > \eta\zeta\quad\hbox{bzw.}\quad \xi\zeta = \eta\zeta\quad\hbox{bzw.}\quad \xi\zeta < \eta\zeta.$$}%

{\bf Beweis:} 1) Aus
$$\xi > \eta$$
folgt nach Satz 140 bei passendem $\upsilon$
$$\xi = \eta + \upsilon,$$
also
$$\xi\zeta = (\eta + \upsilon)\zeta = \eta\zeta + \upsilon\zeta > \eta\zeta.$$

2) Aus
$$\xi = \eta$$
folgt nat\"urlich
$$\xi\zeta = \eta\zeta.$$

3) Aus
$$\xi < \eta$$
folgt
$$\eta > \xi,$$
also nach 1)
$$\eta\zeta > \xi\zeta,$$
$$\xi\zeta < \eta\zeta.$$
\medskip

%\jutting{3}{875-890}

{\bf Satz 146:} {\it Aus
$$\xi\zeta > \eta\zeta\quad\hbox{bzw.}\quad \xi\zeta = \eta\zeta\quad\hbox{bzw.}\quad \xi\zeta < \eta\zeta$$
folgt
$$\xi > \eta\quad\hbox{bzw.}\quad \xi = \eta\quad\hbox{bzw.}\quad \xi < \eta.$$}%

{\bf Beweis}: Folgt aus Satz 145, da die drei F\"alle beide Male
sich ausschlie{\ss}en und alle M\"oglichkeiten ersch\"opfen.
\medskip

%\jutting{3}{891-898}

{\bf Satz 147:} {\it Aus
$$\xi > \eta,\quad \zeta > \upsilon$$
folgt
$$\xi\zeta > \eta\upsilon.$$}%

{\bf Beweis:} Nach Satz 145 ist
$$\xi\zeta > \eta\zeta$$
und
$$\eta\zeta = \zeta\eta > \upsilon\eta = \eta\upsilon,$$
also
$$\xi\zeta > \eta\upsilon.$$
\medskip

%\jutting{3}{899-906}

{\bf Satz 148:} {\it Aus
$$\xi \ge \eta,\quad \zeta > \upsilon\quad\hbox{\it oder}\quad \xi > \eta,\quad \zeta \ge \upsilon$$
folgt
$$\xi\zeta > \eta\upsilon.$$}%

{\bf Beweis:} Mit dem Gleichheitszeichen in der Voraussetzung
durch Satz 145, sonst durch Satz 147 erledigt.
\medskip

%\jutting{3}{907-920}

{\bf Satz 149:} {\it Aus
$$\xi \ge \eta,\quad \zeta \ge \upsilon$$
folgt
$$\xi\zeta \ge \eta\upsilon.$$}%

{\bf Beweis:} Mit zwei Gleichheitszeichen in der Voraussetzung
klar; sonst durch Satz 148 erledigt.
\medskip

%\jutting{3}{921-953}

{\bf Satz 150:} {\it F\"ur jede rationale Zahl $R$ bildet die Menge der ratio-
nalen Zahlen $< R$ einen Schnitt.}

{\bf Beweis:} 1) Nach Satz 90 gibt es ein $X < R$.  $R$ selbst ist
nicht $< R$.

2) Ist
$$X < R,\quad X_1 \ge R,$$
so ist
$$X < X_1.$$

3) Ist
$$X < R,$$
so gibt es nach Satz 91 ein $X_1$ mit
$$X < X_1 < R.$$
\medskip

{\bf Definition 37:} {\it Der in Satz 150 konstruierte Schnitt hei{\ss}t $R^*$.}

(Gro{\ss}e lateinische Buchstaben mit Sternen bedeuten also Schnitte,
nicht rationale Zahlen.)
\medskip

%\jutting{3}{954-971}

{\bf Satz 151:} {\it $$\xi \cdot 1^* = \xi.$$}%

{\bf Beweis:} $\xi \cdot 1^*$ ist die Menge aller $XY$, wo $X$ Unterzahl bei
$\xi$ und
$$Y < 1$$
ist.

Jedes solche $XY$ ist $< X$, also Unterzahl bei $\xi$.

Umgekehrt sei eine Unterzahl $X$ bei $\xi$ gegeben.  Dann w\"ahle
man bei $\xi$ eine Unterzahl
$$X_1 > X$$
und setze
$$Y = {1 \over X_1}X.$$
Dann ist
$$Y < {1 \over X_1}X_1 = 1$$
also
$$X = X_1 Y$$
Unterzahl bei $\xi \cdot 1^*$.
\medskip

%\jutting{3}{972-1079}

{\bf Satz 152:} {\it Ist $\xi$ gegeben, so hat die Gleichung
$$\xi\upsilon = 1^*$$
eine l\"osung $\upsilon$.}

{\bf Beweis:} Wir betrachten die Menge aller Zahlen $1 \over X$, wo $X$
eine beliebige Oberzahl bei $\xi$ mit etwaiger Ausnahme der kleinsten
(wenn es n\"amlich eine gibt) ist.  Wir zeigen, da{\ss} diese Menge ein
Schnitt ist.

1) Es gibt eine Zahl der Menge; denn wenn $X$ eine Oberzahl
bei $\xi$ ist, ist $X + X$ auch eine, aber nicht die kleinste, also $1 \over X + X$
zur Menge geh\"orig.

Es gibt eine rationale Zahl, die nicht zur Menge geh\"ort; denn
ist $X_1$ irgend eine Unterzahl bei $\xi$, so ist f\"ur alle Oberzahlen $X$
bei $\xi$
$$X \ne X_1,$$
also, wegen
$$X{1 \over X} = 1 = X_1{1 \over X_1},$$
$${1 \over X} \ne {1 \over X_1};$$
$1 \over X_1$ ist daher nicht zu unserer Menge geh\"orig.

2) Ist eine Zahl $1 \over X$ unserer Menge gegeben, also $X$ Oberzahl
bei $\xi$, und
$$U < {1 \over X},$$
so ist
$$UX < {1 \over X}X = 1 = U{1 \over U},$$
also
$$X < {1 \over U},$$
also $1 \over U$ Oberzahl bei $\xi$ und nicht die kleinste; wegen
$$U{1 \over U} = 1,$$
$$U = {1 \over {1 \over U}}$$
ist also U zu unserer Menge geh\"orig.

3) Ist eine Zahl $1 \over X$ unserer Menge gegeben, also $X$ Oberzahl
bei $\xi$ und nicht die kleinste, so w\"ahle man bei $\xi$ eine Oberzahl
$$X_1 < X$$
und alsdann nach Satz 91 ein $X_2$, mit
$$X_1 < X_2 < X.$$
Dann ist $X_2$ Oberzahl bei $\xi$ und nicht die kleinste; aus
$$X_2{1 \over X} < X{1 \over X} = 1 = X_2{1 \over X_2}$$
folgt
$${1 \over X_2} > {1 \over X},$$
so da{\ss} wir eine Zahl unserer Menge gefunden haben, die gr\"o{\ss}er
ist als die gegebene.

Unsere Menge ist also ein Schnitt: er hei{\ss}e $\upsilon$.

Von ihm werden wir
$$\xi\upsilon = 1^*$$
beweisen.  Hierzu gen\"ugt es, zweierlei zu zeigen:

A) Jede Unterzahl bei $\xi\upsilon$ ist $< 1$.

B) Jede rationale Zahl $< 1$ ist Unterzahl bei $\xi\upsilon$.

Ad A) Jede Unterzahl bei $\xi\upsilon$ hat die Form
$$X{1 \over X_1},$$
wo $X$ Unterzahl bei $\xi$, $X_1$ Oberzahl bei $\xi$ ist.  Aus
$$X < X_1$$
folgt
$$X{1 \over X_1} < X_1{1 \over X_1} = 1.$$

Ad B) Es sei
$$U < 1.$$
Wir w\"ahlen irgend eine Unterzahl $X$ bei $\xi$ und dann nach Satz
132 eine Unterzahl $X_1$ bei $\xi$ und eine Oberzahl $X_2$ bei $\xi$ mit
$$X_2 - X_1 = (1 - U)X.$$
Dann ist
$$X_2 - X_1 < (1 - U)X_2,$$
$$(X_2 - X_1) + U X_2 < (1 - U)X_2 + U X_2 = X_2 = (X_2 - X_1) + X_1,$$
$$U X_2 < X_1,$$
$$X_2 = \left({1 \over U} U\right)X_2 = {1 \over U}(U X_2) < {1 \over U}X_1 = {X_1 \over U}.$$
$X_1 \over U$ ist also Oberzahl bei $\xi$ und nicht die kleinste.  Aus
$$U{X_1 \over U} = X_1$$
folgt
$$U = {X_1 \over {X_1 \over U}} = X_1{1 \over {X_1 \over U}};$$
hier ist $X_1$ Unterzahl bei $\xi$, $1 \over {X_1 \over U}$ Unterzahl bei $\upsilon$; also ist $U$ Unter%
zahl bei $\xi\upsilon$.
\medskip

%\jutting{3}{1080-1111}

{\bf Satz 153:} {\it Die Gleichung
$$\eta\upsilon = \xi,$$
wo $\xi$, $\eta$ gegeben sind, hat genau eine L\"osung $\upsilon$.}

{\bf Beweis:} I) Es gibt h\"ochstens eine L\"osung; denn f\"ur
$$\upsilon_1 \ne \upsilon_2$$
ist nach Satz 145
$$\eta\upsilon_1 \ne \eta\upsilon_2.$$

II) Ist $\tau$ die durch Satz 152 als vorhanden nachgewiesene L\"o-
sung von
$$\eta\tau = 1^*,$$
so gen\"ugt
$$\upsilon = \tau\xi$$
unserer Gleichung; denn nach Satz 151 ist
$$\eta\upsilon = \eta(\tau\xi) = (\eta\tau)\xi = 1^* \xi = \xi.$$
\medskip

{\bf Definition 38:} {\it Das $\upsilon$ des Satzes 153 hei{\ss}t $\xi \over \eta$ {\rm (sprich: $\xi$ durch
$\eta$).}  $\xi \over \eta$ hei{\ss}t auch der Quotient von $\xi$ durch $\eta$ oder der durch Division
von $\xi$ durch $\eta$ entstehende Schnitt.}
\vfill\eject


%\jutting{3}{1112-1115}

\line{}\vskip 7\baselineskip
\centerline{{\S}~5.}
\medskip

\centerline{\bf Rationale Schnitte und ganze Schnitte.}
\bigskip

{\bf Definition 39:} {\it Ein Schnitt der Form $X^*$ hei{\ss}t rationaler Schnitt.}
\medskip

%\jutting{3}{1116-1125}

{\bf Definition 40:} {\it Ein Schnitt der Form $x^*$ hei{\ss}t ganzer Schnitt.}

(Kleine lateinische Buchstaben mit Sternen bedeuten also
Schnitte, nicht ganze Zahlen.)
\medskip

%\jutting{3}{1126-1180}

{\bf Satz 154:} {\it Aus
$$X > Y \quad\hbox{\it bzw.}\quad X = Y \quad\hbox{\it bzw.}\quad X < Y$$
folgt
$$X^* > Y^* \quad\hbox{\it bzw.}\quad X^* = Y^* \quad\hbox{\it bzw.}\quad X^* < Y^*$$
und umgekehrt.}

{\bf Beweis:} I) 1) Aus
$$X > Y$$
folgt, da{\ss} $Y$ Unterzahl bei $X^*$ ist.  $Y$ ist Oberzahl bei $Y^*$.  Also
$$X^* > Y^*.$$

2) Aus
$$X = Y$$
folgt nat\"urlich
$$X^* = Y^*.$$

3) Aus
$$X < Y$$
folgt
$$Y > X,$$
also nach 1)
$$Y^* > X^*,$$
$$X^* < Y^*.$$

II) Die Umkehrung ist klar, da die drei F\"alle beide Male sich
ausschlie{\ss}en und alle M\"oglichkeiten ersch\"opfen.
\medskip

%\jutting{3}{1181-1239}

{\bf Satz 155:} {\it $${(X + Y)}^* = X^* + Y^*;$$
$${(X - Y)}^* = X^* - Y^*,\quad\hbox{\it falls}\quad X > Y;$$
$${(XY)}^* = X^* Y^*;$$
$${\left({X \over Y}\right)}^* = {X^* \over Y^*}.$$}%

{\bf Beweis:} I) $\fr a$) Jede Unterzahl bei $X^* + Y^*$ ist die Summe einer
rationalen Zahl $< X$ und einer rationalen Zahl $< Y$; sie ist also
$< X + Y$, also Unterzahl bei ${(X + Y)}^*$.

$\fr b$) Jede Unterzahl $U$ bei ${(X + Y)}^*$ ist $< X + Y$.  Aus
$${U \over X + Y} < 1,$$
$$U = X{U \over X + Y} + Y{U \over X + Y}$$
folgt, da{\ss} $U$ Summe einer rationalen Zahl $< X$ und einer ratio%
nalen Zahl $< Y$ ist, also Unterzahl bei $X^* + Y^*$ ist.

Daher ist
$${(X + Y)}^* = X^* + Y^*.$$

II) Aus
$$X > Y$$
folgt
$$X = (X - Y) + Y,$$
also nach I)
$$X^* = {(X - Y)}^* + Y^*,$$
$${(X - Y)}^*= X^* - Y^*.$$

III) $\fr a$) Jede Unterzahl bei $X^* Y^*$ ist Produkt einer rationalen
Zahl $< X$ und einer rationalen Zahl $< Y$; sie ist also $< XY$,
also Unterzahl bei ${(XY)}^*$.

$\fr b$) Jede Unterzahl $U$ bei ${(XY)}^*$ ist $< XY$. Es werde nach
Satz 91 eine rationale Zahl $U_1$ mit
$$U < U_1 < XY$$
gew\"ahlt.  Dann ist
$${U \over U_1} < 1$$
und
$${U_1 \over Y} < X.$$
Durch
$$U = {U_1 \over Y}\left(Y{U \over U_1}\right)$$
ist also $U$ als das Produkt einer Unterzahl bei $X^*$ und einer Unter%
zahl bei $Y^*$ dargestellt.  $U$ ist also Unterzahl bei $X^* Y^*$.

Daher ist
$${(XY)}^*= X^* Y^*.$$

IV) $$X = {X \over Y}Y,$$
also nach III)
$$X^* = {\left({X \over Y}\right)}^* Y^*,$$
$${\left({X \over Y}\right)}^* = {X^* \over Y^*}.$$
\medskip

%\jutting{3}{1240-1319}

{\bf Satz 156:} {\it Die ganzen Schnitte gen\"ugen den f\"unf Axiomen der
nat\"urlichen Zahlen, wenn $1^*$ an Stelle von $1$ genommen wird und
$$(x^*)' = (x')^*$$
gesetzt wird.}

{\bf Beweis:} ${\fr Z}^*$ sei die Menge der ganzen Schnitte.

1) $1^*$ geh\"ort zu ${\fr Z}^*$.

2) Zu $x^*$ ist $(x^*)'$ in ${\fr Z}^*$ vorhanden.

3) Stets ist
$$x' \ne 1,$$
also
$$(x')^* \ne 1^*,$$
$$(x^*)' \ne 1^*.$$

4) Aus
$$(x^*)' = (y^*)'$$
folgt
$$(x')^* = (y')^*,$$
$$x' = y',$$
$$x = y,$$
$$x^* = y^*.$$

5) Eine Menge ${\fr M}^*$ von ganzen Schnitten habe die Eigen%
schaften:

I) $1^*$ geh\"ort zu ${\fr M}^*$.

II) Falls $x^*$ zu ${\fr M}^*$ geh\"ort, so geh\"ort $(x^*)'$ zu ${\fr M}^*$.

Dann bezeichne $\fr M$ die Menge der $x$, f\"ur die $x^*$ zu ${\fr M}^*$ geh\"ort.
Alsdann ist $1$ zu $\fr M$ geh\"orig und mit jedem $x$ von $\fr M$ auch $x'$ zu
$\fr M$ geh\"orig.  Also geh\"ort jede ganze Zahl zu $\fr M$, also jeder ganze
Schnitt zu ${\fr M}^*$.
\bigskip

%\jutting{3}{1320-1364}

Da $=$, $>$, $<$, Summe, Differenz (wofern vorhanden), Produkt
und Quotient bei rationalen Schnitten nach Satz 154 und Satz 155
den alten Begriffen entsprechen, haben die rationaleh Schnitte alle
Eigenschaften, die wir in Kapitel 2 f\"ur rationale Zahlen bewiesen
haben, und insbesondere die ganzen Schnitte alle bewiesenen Eigen%
schaften der ganzen Zahlen.

Daher werfen wir die rationalen Zahlen weg, ersetzen sie
durch die entsprechenden rationalen Schnitte und haben fortan in
bezug auf das Bisherige nur noch von Schnitten zu reden.  (Die
rationalen Zahlen verbleiben aber in Mengen beim Begriff des
Schnittes.)
\medskip

{\bf Definition 41:} {\it {\rm (Das freigewordene Zeichen)} $X$ bezeichnet den
rationalen Schnitt $X^*$, auf den auch das Wort rationale Zahl \"uber%
geht; ebenso geht das Wort ganze Zahl auf die ganzen Schnitte \"uber.}

%\jutting{3}{1365-1366}

Also schreiben wir jetzt z. B. statt
$$\xi{1^* \over \xi} = 1^*$$
einfach
$$\xi{1 \over \xi} = 1.$$
\medskip

%\jutting{3}{1367-1406}

{\bf Satz 157:} {\it Die rationalen Zahlen sind die Schnitte, bei denen es
eine kleinste Oberzahl X gibt.  Und zwar ist alsdann X der Schnitt.}

{\bf Beweis:} 1) Beim Schnitt $X$ (dem alten $X^*$) ist $X$ (rationale
Zahl im alten Sinne) kleinste Oberzahl.

2) Gibt es bei einem Schnitt $\xi$ eine kleinste Oberzahl $X$, so
ist jede Unterzahl $< X$, jede Oberzahl $\ge X$, der Schnitt also $X$
(das alte $X^*$).
\medskip

%\jutting{3}{1407-1446}

{\bf Satz 158:} {\it Es sei $\xi$ ein Schnitt.  Dann ist $X$ Unterzahl genau
dann, wenn
$$X < \xi,$$
also Oberzahl genau dann, wenn
$$X \ge \xi.$$}%

{\bf Beweis:} 1) Ist $X$ Unterzahl bei $\xi$, so ist, da $X$ Oberzahl bei
$X$ (dem alten $X^*$) ist,
$$X < \xi.$$

2) Ist $X$ Oberzahl bei $\xi$ und zwar die kleinste, so ist nach
Satz 157
$$X = \xi.$$

3) Ist $X$ Oberzahl bei $\xi$ und zwar nicht die kleinste, so w\"ahle
man eine kleinere Oberzahl $X_1$.  Dann ist $X_1$ Unterzahl bei $X$,
also
$$X > \xi.$$
\medskip

%\jutting{3}{1447-1475}

{\bf Satz 159:} {\it Ist
$$\xi < \eta,$$
so gibt es ein $Z$ mit
$$\xi < Z < \eta.$$}%

{\bf Beweis:} Man w\"ahle eine Oberzahl $X$ bei $\xi$, die Unterzahl bei
$\eta$ ist, und dann eine gr\"o{\ss}ere Unterzahl $Z$ bei $\eta$.  Dann ist nach
Satz 158
$$\xi \le X < Z < \eta.$$
\medskip

%\jutting{3}{1476-1546}

{\bf Satz 160:} {\it Jedes
$$Z > \xi\eta$$
l\"a{\ss}t sich auf die Form bringen
$$Z = XY,\quad X \ge \xi,\quad Y \ge \eta.$$}%

{\bf Beweis:} Es bezeichne $\zeta$ den kleinsten der beiden Schnitte $1$ und
$Z - \xi\eta \over (\xi + \eta) + 1$.  Dann ist
$$\zeta \le 1,\quad \zeta \le {Z - \xi\eta \over (\xi + \eta) + 1}.$$

Man w\"ahle $Z_1$ und $Z_2$ nach Satz 159 mit
$$\xi < Z_1 < \xi + \zeta,\quad \eta < Z_2 < \eta + \zeta.$$
Dann ist
$$\displaylines{Z_1 Z_2 < (\xi + \zeta)(\eta + \zeta) = (\xi + \zeta)\eta + (\xi + \zeta)\zeta \le (\xi + \zeta)\eta + (\xi + 1)\zeta \cr
= (\xi\eta + \eta\zeta) + (\xi + 1)\zeta = \xi\eta + \bigl((\xi + \eta) + 1\bigr)\zeta \le \xi\eta + (Z - \xi\eta) = Z.\cr}$$

In
$$Z = {Z \over Z_2}Z_2$$
ist
$$X = {Z \over Z_2} = Z{1 \over Z_2} > (Z_1 Z_2){1 \over Z_2} = Z_1 > \xi,$$
$$Y = Z_2 > \eta,$$
also $Z$ in gew\"unschter Weise zerlegt.
\medskip

%\jutting{3}{1547-1720}

{\bf Satz 161:} {\it Bei jedem $\zeta$ hat
$$\xi\xi = \zeta$$
genau eine L\"osung.}

{\bf Beweis:} I) Es gibt h\"ochstens eine L\"osung; denn aus
$$\xi_1 > \xi_2$$
folgt
$$\xi_1 \xi_1 > \xi_2 \xi_2.$$

II) Wir betrachten die Menge der rationalen Zahlen $X$ mit
$$XX < \zeta.$$
Sie bildet einen Schnitt.  Denn:

1) Ist
$$X < 1 \quad\hbox{\rm und}\quad X < \zeta,$$
so ist
$$XX < X \cdot 1 = X < \zeta.$$

Ist
$$X \ge 1 \quad\hbox{\rm und}\quad X \ge \zeta,$$
so ist
$$XX \ge X \cdot 1 = X \ge \zeta.$$

2) Aus
$$XX < \zeta,\quad Y < X$$
folgt
$$YY < XX < \zeta.$$

3) Es sei
$$XX < \zeta.$$
Man w\"ahle $Z$ kleiner als der kleinste der beiden Schnitte $1$ und
$\zeta - XX \over X + (X + 1)$.  Dann ist
$$Z < 1,\quad Z < {\zeta - XX \over X + (X + 1)};$$
alsdann ist
$$X + Z > X$$
und
$$\displaylines{(X + Z)(X + Z) = (X + Z)X + (X + Z)Z < (XX + ZX) + (X + 1)Z\cr
= XX + \bigl(X + (X + 1)\bigr)Z < XX + (\zeta - XX) = \zeta.\cr}$$

Nennen wir den konstruierten Schnitt $\xi$, so behaupten wir
nunmehr
$$\xi\xi = \zeta.$$

W\"are
$$\xi\xi > \zeta,$$
so w\"ahlen wir $Z$ nach Satz 159 mit
$$\xi\xi > Z > \zeta.$$
Als Unterzahl bei $\xi\xi$ w\"are
$$Z = X_1 X_2,\quad X_1 < \xi,\quad X_2 < \xi;$$
wenn $X$ die gr\"o{\ss}te der Zahlen $X_1$ und $X_2$ bedeutet, w\"are
$$X < \xi,$$
$$Z \le XX < \zeta,$$
gegen das Obige.

W\"are
$$\xi\xi < \zeta,$$
so w\"ahlen wir $Z$ nach Satz 159 mit
$$\xi\xi < Z < \zeta.$$
$Z$ h\"atte nach Satz 160 die Form
$$Z = X_1 X_2,\quad X_1 \ge \xi,\quad X_2 \ge \xi;$$
wenn $X$ die kleinste der Zahlen $X_1$ und $X_2$ bedeutet, w\"are
$$X \ge \xi,$$
$$Z \ge XX \ge \zeta,$$
gegen das Obige.
\medskip

%\jutting{3}{1721-1724}

{\bf Definition 42:} {\it Jeder Schnitt, der keine rationale Zahl ist, hei{\ss}t
irrationale Zahl.}
\medskip

%\jutting{3}{1725-1847}

{\bf Satz 162:} {\it Es gibt eine irrationale Zahl.}

{\bf Beweis:} Es gen\"ugt zu zeigen, da{\ss} die nach Satz 161 vor%
handene L\"osung von
$$\xi\xi = 1'$$
irrational ist.

Sonst w\"are
$$\xi = {x \over y};$$
unter allen solchen Darstellungen w\"ahlen wir nach Satz 27 eine
solche, in der $y$ m\"oglichst klein ist.  Wegen
$$1' = \xi\xi = {x \over y} \cdot {x \over y} = {xx \over yy}$$
ist
$$yy < 1'(yy) = xx = (1'y)y < (1'y)(1'y),$$
$$y < x < 1'y.$$

Wir setzen
$$x - y = u.$$
Dann ist
$$y + u = x < 1'y = y + y,$$
$$u < y.$$

Nun ist
$$\displaylines{(v + w)(v + w) = (v + w)v + (v + w)w = (vv + wv) + (vw + ww)\cr
= \bigl(vv + 1'(vw)\bigr) + ww,\cr}$$
also,
$$y - u = t$$
gesetzt,
$$\displaylines{xx + tt = (y + u)(y + u) + tt = \bigl(yy + 1'(yu)\bigr) + (uu + tt)\cr
= \bigl(yy + (1'u)(u + t)\bigr) + (uu + tt)\cr
= \bigl(yy + 1'(uu)\bigr) + \Bigl(\bigl(1'(ut) + uu\bigr) + tt\Bigr)\cr
= \bigl(yy + 1'(uu)\bigr) + (u + t)(u + t)\cr
= \bigl(yy + 1'(uu)\bigr) + yy = 1'(yy) + 1'(uu) = xx + 1'(uu),\cr}$$
$$tt = 1'(uu),$$
$${t \over u} \cdot {t \over u} = 1',$$
gegen
$$u < y.$$
\vfill\eject


