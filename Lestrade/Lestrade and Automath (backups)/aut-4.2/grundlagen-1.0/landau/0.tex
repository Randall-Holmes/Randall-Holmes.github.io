%\jutting{0}{1-942}

\line{}\vskip 7\baselineskip
\centerline{\bf Vorwort f\"ur den Lernenden.}
\bigskip

1. Bitte lies nicht das nachstehende Vorwort f\"ur den Kenner!

2. Ich setze nur logisches Denken und die deutsche Sprache
als bekannt voraus; nichts aus der Schulmathematik oder gar der
h\"oheren Mathematik.

Um Einw\"anden vorzubeugen: {\bf Eine} Zahl, {\bf keine} Zahl, {\bf zwei}
F\"alle, {\bf alle} Dinge aus einer gegebenen Gesamtheit u.~a.~m.\ sind
klare Wortgebilde der deutschen Sprache.  Satz 1, Satz 2, \dots,
Satz 301 (desgleichen bei Axiomen, Definitionen, Kapiteln, Paragra%
phen) oder 1), 2) u.~dgl.\ bei Fallunterscheidungen sind Marken, die
die S\"atze, Axiome, \dots, F\"alle unterscheiden und bei Nachschlagungen
bequemer sind, als wenn ich etwa von Satz Hellblau, Satz Dunkel%
blau u.~dgl.\ redete.  Bis ``301'' w\"urde \"uberhaupt die Einf\"uhrung
der sogenannten positiven ganzen Zahlen keine Schwierigkeit machen;
die erste -- in Kap.~1 \"uberwundene -- Schwierigkeit liegt in der
positiven ganzen Zahlen
$$1, \ldots$$
mit der geheimnisvollen Punktreihe hinter dem Komma (in Kap.~1
nat\"urliche Zahlen genannt), in der Definition der mit ihnen anzu%
stellenden Rechenoperationen und den Beweisen der zugeh\"origen
S\"atze.

Ich entwickle nacheinander alles Entsprechende in Kap.~1 f\"ur
die nat\"urlichen Zahlen, in Kap.~2 f\"ur die positiven Br\"uche und
positiven rationalen Zahlen, in Kap.~3 f\"ur die positiven (rationalen
und irrationalen) Zahlen, in Kap.~4 f\"ur die reellen Zahlen (positive,
negative und Null), in Kap.~5 f\"ur die komplexen Zahlen: ich spreche
also nur von solchen Zahlen, mit denen Du Dich schon in der Schule
besch\"aftigt hast.

In diesem Sinne:

3. Bitte vergi{\ss} alles, was Du auf der Schule gelernt hast;
denn Du hast es nicht gelernt.

Bitte denke bei allem an die entsprechenden Stellen des Schul%
pensums; denn Du hast es doch nicht vergessen.

4. Das kleine Einmaleins, bereits der Satz
$$2 \cdot 2 = 4,$$
kommt nicht vor; ich empfehle Dir aber als \"Ubungsaufgabe zu
Kap.~1, {\S}~4,
$$\eqalign{2 &= 1 + 1,\cr
4 &= \bigl((1 + 1) + 1\bigr) + 1\cr}$$
zu definieren und jenen Satz zu beweisen.

5. Entschuldige, da{\ss} ich Dich duze; dies geschieht nicht nur,
weil man den Leser mit ``lies'' und ``siehe'' anzureden pflegt, son%
dern weil dies Buch zum Teil in usum delphinarum geschrieben
ist, indem meine T\"ochter bekanntlich (siehe {\sl E.~Landau,} {\it Vor%
lesungen \"uber Zahlentheorie,} Bd.~1, S.~V) schon mehrere Semester
studieren (Chemie), schon auf der Schule Differential- und Integral%
rechnung gelernt zu haben glauben und heute noch nicht wissen,
warum
$$x \cdot y = y \cdot x$$
ist.
\bigskip

{\sl Berlin,} den 28.~Dezember 1929.
\bigskip

\rightline{\bf Edmund Landau.}
\vfill\eject


\line{}\vskip 7\baselineskip
\centerline{\bf Vorwort f\"ur den Kenner.}
\bigskip

Dies B\"uchlein ist eine Konzession an die (leider in der Mehr%
zahl befindlichen) Kollegen, welche meinen Standpunkt in der fol%
genden Frage {\bf nicht} teilen.

W\"ahrend auf der Schule naturgem\"a{\ss} auf strengen und l\"ucken%
losen Aufbau der Elementarmathematik verzichtet werden mu{\ss},
soll der mathematische Hochschulunterricht den H\"orer nicht nur
mit dem Stoff und den Ergebnissen, sondern auch mit den Beweis%
methoden bekannt machen.  Auch wer Mathematik haupts\"achlich
f\"ur die Anwendungen auf Physik und andere Wissenschaften lernt,
also vielfach sich selbst weitere mathematische Hilfss\"atze zurecht%
legen mu{\ss}, kann auf dem betretenen Pfade nur dann sicher weiter%
schreiten, wenn er gehen gelernt hat, d.~h.\ zwischen falsch und
wahr, zwischen Vermutungen und Beweisen (oder, wie manche so
sch\"on sagen, zwischen unstrengen und strengen Beweisen) unter%
scheiden kann.

Darum finde ich es -- im Anschlu{\ss} an einige meiner Lehrer
und Kollegen, an einige Autoren, aus deren Schriften ich gesch\"opft
habe, und an die meisten meiner Sch\"uler -- richtig, da{\ss} der Stu%
dierende bereits im ersten Semester lernt, auf welchen als Axiomen
angenommenen Grundtatsachen sich l\"uckenlos die Analysis aufbaut
und wie dieser Aufbau begonnen werden kann.  Bei der Wahl
der Axiome kann man bekanntlich verschieden verfahren; ich er%
kl\"are es also nicht etwa f\"ur falsch, sondern f\"ur meinem pers\"onlichen
Standpunkt fast diametral entgegengesetzt, wenn man f\"ur reelle
Zahlen zahlreiche der \"ublichen Rechengesetze und den {\sl Dedekind\/}%
schen Hauptsatz 205 der folgenden Schrift als Axiome postuliert.
Gewi{\ss} beweise ich nicht die Widerspruchslosigkeit der f\"unf {\sl Peano\/}%
schen Axiome (weil man es n\"amlich nicht kann); aber jedes der%
selben ist offenkundig von den vorigen unabh\"angig.  Andrerseits
dr\"angt sich bei jener erweiterten Zahl von Axiomen dem Lernenden
sofort die Frage auf, ob sich nicht so manches darunter aus dem
Vorangehenden beweisen (ein Schlauer w\"urde hinzuf\"ugen: oder
widerlegen) l\"a{\ss}t; und da man seit vielen Jahrzehnten die Beweis%
barkeit aller dieser Dinge kennt, ist es dem Lernenden wirklich
zu g\"onnen, da{\ss} er die (durchweg ganz leichten) Beweise zu Beginn
seines Studiums lernt.

Ich will gar nicht erst ausf\"uhrlich dar\"uber reden, da{\ss} vielfach
nicht einmal der {\sl Dedekind\/}sche Hauptsatz (oder sein gleich%
wertiges Surrogat bei Begr\"undung der reellen Zahlen durch Funda%
mentalreihen) zugrunde gelegt wird; so da{\ss} dann Dinge wie der
Mittelwertsatz der Differentialrechnung, der hierauf fu{\ss}ende Satz,
da{\ss} eine Funktion mit in einem Intervall verschwindender Ab%
leitung dort konstant ist, oder z.~B.\ der Satz, da{\ss} eine best\"andig
fallende, beschr\"ankte Folge von Zahlen gegen einen Grenzwert
strebt, ohne jeden Beweis erscheinen oder, was noch schlimmer ist,
mit einem vermeintlichen Beweis, der keiner ist.  Die Anzahl der
Vertreter dieser extremen Spielart des anderen Standpunktes scheint
mir nicht nur monoton zu fallen; sondern der Grenzwert, dem diese
Anzahl nach dem oben genannten Satze zustrebt, ist vielleicht so%
gar Null.

Aber mit einer Begr\"undung der nat\"urlichen Zahlen wird nur
selten angefangen.  Auch ich gestehe, da{\ss} ich von jeher nicht
unterlie{\ss}, nach {\sl Dedekind\/} die Theorie der reellen Zahlen durch%
zunehmen, fr\"uher aber die Eigenschaften der ganzen und der ratio%
nalen Zahlen voraussetzte.  Die drei letzten Male zog ich allerdings
vor, mit den ganzen Zahlen zu beginnen.  Einmal und auch f\"ur
das kommende Sommersemester als Konzession gegen die Zuh\"orer,
die doch gleich differentiieren wollen oder gar die ganze Erl\"aute%
rung des Zahlbegriffs nicht im ersten Semester (oder wom\"oglich
\"uberhaupt nicht) lernen wollen, habe ich allerdings meine Vor%
lesung in zwei gleichzeitige geteilt, deren eine ``Grundlagen der
Analysis'' hie{\ss}.  In dieser gelange ich, von den {\sl Peano\/}schen
Axiomen der nat\"urlichen Zahlen ausgehend, bis zur Theorie der
reellen Zahlen und der komplexen Zahlen; \"ubrigens braucht der
H\"orer die komplexen Zahlen im ersten Semester noch nicht; aber
deren Einf\"uhrung ist ja ganz einfach und l\"a{\ss}t sich m\"uhelos gleich
anbringen.

Nun gibt es in der ganzen Literatur kein Lehrbuch, das sich
das bescheidene Ziel setzt, {\bf nur} das Rechnen mit Zahlen im obigen
Sinne zu begr\"unden.  Und auch die umfangreichen Werke, in denen
dies in den einleitenden Kapiteln unternommen wird, \"uberlassen
dabei (bewu{\ss}t oder unbewu{\ss}t) so manches dem Leser.

{\bf Diese} Schrift -- wenn sie von ihm f\"ur passend befunden
wird -- soll jedem Kollegen der anderen p\"adagogischen Richtung,
der also die Grundlagen nicht durchnimmt, wenigstens die M\"og%
lichkeit geben, auf eine Quelle zu verweisen, wo das Fehlende und
nur das Fehlende in l\"uckenlosem Zusammenhang dargestellt ist.
Die Lekt\"ure ist ganz leicht, wenn man -- was ja der Fall ist --
schon in der Schule die Ergebnisse erfahren hat und wenn man
\"uber die abstrakten vier oder f\"unf ersten Seiten hinweggekommen ist.

Ich trete mit Z\"ogern mit dieser Schrift an die \"Offentlichkeit,
weil ich damit \"uber ein Gebiet publiziere, in dem ich (au{\ss}er einer
m\"undlichen Mitteilung von Herrn {\sl Kalm\'ar\/}) nichts Neues zu sagen
habe; aber ein anderer hat sich meine, zum Teil langweilige M\"uhe
nicht gemacht.

Den definitiven Ansto{\ss} zu dieser ``Flucht in die \"Offentlichkeit''
hat aber ein konkreter Vorfall gegeben.

Die andere Richtung meint immer, w\"ahrend des sp\"ateren Ver%
laufes des Studiums w\"urde der Sch\"uler an Hand einer Vorlesung
oder der Literatur die Sache schon lernen.  Und keiner jener
meiner verehrten Freunde und Feinde w\"urde bezweifelt haben, da{\ss}
z.~B.\ in meinen Vorlesungen sich alles N\"otige findet.  Auch ich
glaubte das.  Und nun passierte mir folgendes schreckliche Aben%
teuer.  An Hand meines Kollegheftes las mein damaliger Assistent
und lieber Kollege Privatdozent Dr.~{\sl Grandjot\/} (jetzt Professor
an der Universit\"at Santiago) \"uber Grundlagen der Analysis und
gab mir mein Manuskript mit dem Bemerken zur\"uck, er h\"atte es
f\"ur notwendig befunden, zu den {\sl Peano\/}schen Axiomen im weiteren
Verlaufe andere hinzuzuf\"ugen, da der \"ubliche Weg, den ich ge%
gangen war, eine bestimmte L\"ucke aufweise.  Ehe ich auf die
Einzelheiten eingehe, will ich gleich vorgreifend erw\"ahnen:

1. {\sl Grandjot\/}s Einwand war berechtigt.

2. Axiome, die nicht zu Anfang des Ganzen aufgez\"ahlt werden
k\"onnen (weil sie an sp\"atere Begriffe ankn\"upfen), sind sehr be%
dauerlich.

3. {\sl Grandjot\/}s Axiome sind (wie wir schon von {\sl Dedekind\/}
h\"atten leruen k\"onnen) alle beweisbar, und es bleibt (s.\ die ganze
folgende Schrift) bei den {\sl Peano\/}schen Axiomen.

Es sind drei Stellen, an denen der Einwand Platz greift:

I. Bei der Definition von $x + y$ f\"ur nat\"urliche Zahlen.

II. Bei der Definition von $x \cdot y$ f\"ur nat\"urliche Zahlen.

III. Bei der Definition von $\sum_{n = 1}^m x_n$ und $\prod_{n = 1}^m x_n$, nachdem man
f\"ur irgend ein Zahlgebiet $x + y$ und $x \cdot y$ schon hat.

Da alle drei Male die Sache analog liegt, spreche ich hier
nur von $x + y$ f\"ur nat\"urliche Zahlen $x$, $y$.  Wenn ich etwa in einer
Vorlesung \"uber Zahlentheorie irgend einen Satz \"uber nat\"urliche
Zahlen so beweise, da{\ss} ich erst die Richtigkeit f\"ur $1$ und dann
aus der Richtigkeit f\"ur $x$ die f\"ur $x + 1$ beweise, so pflegt gelegent%
lich ein Zuh\"orer den Einwand zu erheben, ich h\"atte die Behaup%
tung ja gar nicht vorher f\"ur $x$ bewiesen.  Der Einwand ist unbe%
rechtigt, aber verzeiblich; der Student hatte eben nie vom Induk%
tionsaxiom geh\"ort.  {\sl Grandjot\/}s Einwand klingt \"ahnlich mit dem
Unterschiede, da{\ss} er berechtigt war, so da{\ss} ich ihn auch verzeihen
mu{\ss}te.  Auf Grund seiner f\"unf Axiome definiert {\sl Peano\/} $x + y$ bei
festem $x$ f\"ur alle $y$ folgenderma{\ss}en:
$$\eqalign{x + 1 &= x',\cr
x + y' &= (x + y)',\cr}$$
und er und Nachfolger meinen damit: $x + y$ ist allgemein definiert;
denn die Menge der $y$, f\"ur die es definiert ist, enth\"alt $1$ und mit
$y$ auch $y'$.

Aber man hat ja $x + y$ gar nicht definiert.

Es w\"are in Ordnung, wenn man (was beim {\sl Peano\/}schen Wege
nicht der Fall ist, da die Ordnung erst nach der Addition einge%
f\"uhrt wird) den Begriff ``Zahlen $\le y$'' h\"atte und von der Menge
der $y$ spr\"ache, zu denen es ein f\"ur $z \le y$ definiertes $f(z)$ mit den
Eigenschaften gibt:
$$\eqalign{f(1) &= x',\cr
f(z') &= \bigl(f(z)\bigr)'\hbox{ f\"ur $z < y$.}\cr}$$
So verl\"auft {\sl Dedekind\/}s Begr\"undung.  Mit freundlicher Hilfe des
Kollegen {\sl von Neumann\/} in Princeton hatte ich nach vorheriger
Einf\"uhrung der Ordnung (was dem Leser nicht bequem gewesen
w\"are) einen derartigen Weg f\"ur dies B\"uchlein ausgearbeitet.
In letzter Stunde erfuhr ich aber einen sehr viel einfacheren Be%
weis von Dr.~{\sl Kalm\'ar\/} in Szeged; jetzt sieht die Sache so einfach
und der Beweis den \"ubrigen Beweisen des ersten Kapitels so \"ahn%
lich aus, da{\ss} auch der Kenner diese Pointe nicht gemerkt h\"atte,
wenn ich nicht mein obiges Gest\"andnis von Schuld und S\"uhne so
ausf\"uhrlich zu Protokoll gegeben h\"atte.  Bei $x \cdot y$ geht es genau
ebenso; $\sum_{n = 1}^m x_n$ und $\prod_{n = 1}^m x_n$ ist allerdings nur auf dem {\sl Dedekind\/}%
schen Wege m\"oglich; aber von Kap.~1, {\S}~3 an hat man ja die Menge
der $z \le y$.

Um es dem Leser m\"oglichst leicht zu machen, habe ich manche
(nicht sehr umfangreiche) Wortmengen in mehreren oder allen Ka%
piteln wiederholt.  F\"ur den Kenner w\"are es nat\"urlich ausreichend,
z.~B.\ ein f\"ur allemal beim Beweise der S\"atze 16 und 17 zu sagen:
Diese Begr\"undung gilt f\"ur jede Klasse von Zahlen, bei denen die
Zeichen $<$ und $=$ definiert sind und bestimmte vorangegangene
Eigenschaften haben.  Derartige wiederholte Schlu{\ss}weisen betreffen
S\"atze, die in allen betreffenden Kapiteln vorkommen mu{\ss}ten, weil
die S\"atze im nachfolgenden angewendet wurden.  Aber $\sum_{n = 1}^m a_n$ und
$\prod_{n = 1}^m a_n$ braucht man nur im letzten Kapitel einzuf\"uhren, um es
damit auch f\"ur die niederen Zahlarten zu haben.  Daher warte ich
damit bis zu den komplexen Zahlen, desgl.\ mit den S\"atzen \"uber
Subtraktion und Division; erstere gelten selbstverst\"andlich z.~B.\ %
f\"ur nat\"urliche Zahlen nur, wenn jeder Minuendus gr\"o{\ss}er ist als
der Subtrahendus, letztere z.~B.\ bei nat\"urlichen Zahlen nur, wenn
alle Divisionen aufgehen.

Mein Buch ist unter Verzicht auf Nebenbemerkungen in dem
unbarmherzigen Telegrammstil (``{\bf Axiom}'', ``{\bf Definition}'', ``{\bf Satz}'',
``{\bf Beweis}'', nur gelegentlich ``{\bf Vorbemerkung}''; selten Worte, die
zu keiner dieser f\"unf Rubriken geh\"oren) geschrieben, der bei einer
so leichten Materie am Platz ist.

Ich hoffe, nach jahrzehntelanger Vorbereitung diese Schrift so
abgefa{\ss}t zu haben, da{\ss} ein normaler Student sie in zwei Tagen
lesen kann.  Und dann darf er sogar (da er die formalen Regeln
ja schon von der Schule her kennt) den ganzen Inhalt bis auf das
Induktionsaxiom und den {\sl Dedekind\/}schen Hauptsatz vergessen.

Wenn aber gar dem einen oder anderen Kollegen der anderen
Richtung die Sache so leicht erscheint, da{\ss} er sie in seinen An%
f\"angervorlesungen (auf dem folgenden oder irgend einem anderen
Wege) bringt, w\"urde ich ein Ziel erreicht haben, auf das ich in
gr\"o{\ss}erem Umfange nicht zu hoffen wage.
\bigskip

{\sl Berlin,} den 28.~Dezember 1929.
\bigskip

\rightline{\bf Edmund Landau.}
\vfill\eject


