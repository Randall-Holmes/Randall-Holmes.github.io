%\jutting{4}{1-94}

\line{}\vskip 7\baselineskip
\centerline{\sl Kapitel 4.}
\medskip

\centerline{\bf Reelle Zahlen.}
\bigskip

\centerline{{\S}~1.}
\medskip

\centerline{\bf Definition.}
\bigskip

{\bf Definition 43:} {\it Die Schnitte nennen wir jetzt positive Zahlen;
und entsprechend sagen wir jetzt positive rationale Zahl statt bisher
rationale Zahl und positive ganze Zahl statt bisher ganze Zahl.

Wir erschaffen eine neue, von den positiven Zahlen verschiedene
Zahl $0$ {\rm (sprich: Null).}

Wir erschaffen ferner Zahlen, die von den positiven und von $0$
verschieden sind, negative genannt, derart, da{\ss} wir jedem $\xi$ {\rm (d. h. jeder
positiven Zahl)} eine negative Zahl zuordnen, die wir $-\xi$ {\rm ($-$ sprich:
minus)} nennen.

Dabei gelten $-\xi$ und $-\eta$ als dieselbe Zahl (als gleich) genau
dann, wenn $\xi$ und $\eta$ dieselbe Zahl sind.

Die Gesamtheit der positiven Zahlen, der $0$ und der negativen
Zahlen nennen wir reelle Zahlen.}

Gro{\ss}e griechische Buchstaben bedeuten, wenn nichts anderes
gesagt wird, durchweg reelle Zahlen.  Gleich schreiben wir $=$,
ungleich (verschieden) $\ne$.

\ifx\Alpha\undefined
  \let\Alpha=A \let\Beta=B \let\Zeta=Z \let\Eta=H \let\Upsilon=Y
\fi

F\"ur jedes $\Xi$ und jedes $\Eta$ liegt somit genau einer der F\"alle
$$\Xi = \Eta,\quad \Xi \ne \Eta$$
vor.  Bei den reellen Zahlen vermischen sich die Begriffe der
Identit\"at und Gleichheit, so da{\ss} die drei S\"atze trivial sind:
\medskip

%\jutting{4}{95-95}

{\bf Satz 163:} {\it $$\Xi = \Xi.$$}%
\medskip

%\jutting{4}{96-97}

{\bf Satz 164:} {\it Aus
$$\Xi = \Eta$$
folgt
$$\Eta = \Xi.$$}%
\medskip

%\jutting{4}{98-99}

{\bf Satz 165:} {\it Aus
$$\Xi = \Eta,\quad \Eta = \Zeta$$
folgt
$$\Xi = \Zeta.$$}%
\vfill\eject


\line{}\vskip 7\baselineskip
\centerline{{\S}~2.}
\medskip

\centerline{\bf Ordnung.}
\bigskip

%\jutting{4}{100-112}

{\bf Definition 44:}
{\it $$|\Xi| = \cases{\xi,\quad\hbox{\it wenn}\quad \Xi = \xi,\cr
0,\quad\hbox{\it wenn}\quad \Xi = 0,\cr
\xi,\quad\hbox{\it wenn}\quad \Xi = -\xi.\cr}$$
Die Zahl $|\Xi|$ hei{\ss}t der absolute Betrag von $\Xi$.}
\medskip

%\jutting{4}{113-122}

{\bf Satz 166:} {\it $|\Xi|$ ist f\"ur positives und negatives $\Xi$ positiv.}

{\bf Beweis:} Definition 44.
\medskip

%\jutting{4}{123-199}

{\bf Definition 45:} {\it Sind $\Xi$ und $\Eta$ nicht beide positiv, so ist
$$\Xi > \Eta$$
genau dann, wenn\hfill\break
\centerline{entweder $\Xi$ negativ, $\Eta$ negativ und $|\Xi| < |\Eta|$,}
\centerline{oder $\Xi = 0$, $\Eta$ negativ,}
\centerline{oder $\Xi$ positiv, $\Eta$ negativ,}
\centerline{oder $\Xi$ positiv, $\Eta = 0.$}
{\rm ($>$ sprich: gr\"o{\ss}er als.)}}

Man beachte, da{\ss} wir f\"ur positives $\Xi$ nebst positivem $\Eta$ die
Begriffe $>$ und $<$ schon haben und letzteren sogar in dem einen
Falle der Definition 45 benutzten.
\medskip

%\jutting{4}{200-221}

{\bf Definition 46:} {\it $$\Xi < \Eta$$
genau dann, wenn
$$\Eta > \Xi.$$
{\rm ($<$ sprich: kleiner als.)}}

Man beachte, da{\ss} f\"ur positives $\Xi$ nebst positivem $\Eta$ Definition
46 im Einklang mit unseren alten Begriffen steht.
\medskip

%\jutting{4}{222-350}

{\bf Satz 167:} {\it Sind $\Xi$, $\Eta$ beliebig, so liegt genau einer der F\"alle
$$\Xi = \Eta,\quad \Xi > \Eta,\quad \Xi < \Eta$$
vor.}

{\bf Beweis:} 1) Sind $\Xi$ und $\Eta$ positiv, wissen wir dies aus Satz 123.

2) Ist $\Xi$ positiv, $\Eta = 0$ oder $\Eta$ negativ, so ist
$$\Xi \ne \Eta,$$
ferner nach Definition 45
$$\Xi > \Eta$$
und nach Definition 46
$$\Xi\ \hbox{\rm nicht} < \Eta.$$

3) Ist $\Xi = 0$, $\Eta$ positiv, so ist
$$\Xi \ne \Eta,$$
ferner nach Definition 45
$$\Xi\ \hbox{\rm nicht} > \Eta$$
und nach Definition 46
$$\Xi < \Eta.$$

4) Ist $\Xi = 0$, $\Eta = 0$, so ist
$$\Xi = \Eta,$$
$$\Xi\ \hbox{\rm nicht} > \Eta,$$
$$\Xi\ \hbox{\rm nicht} < \Eta.$$

5) Ist $\Xi = 0$, $\Eta$ negativ, so ist
$$\Xi \ne \Eta,$$
$$\Xi > \Eta,$$
$$\Xi\ \hbox{\rm nicht} < \Eta.$$

6) Ist $\Xi$ negativ, $\Eta$ positiv oder $\Eta = 0$, so ist
$$\Xi \ne \Eta,$$
$$\Xi\ \hbox{\rm nicht} > \Eta,$$
$$\Xi < \Eta.$$

7) Ist $\Xi$ negativ, $\Eta$ negativ, so ist

$$\vcenter{\halign{#\hfil\quad&#\hfil\quad&#\hfil\ &#\cr
$\Xi \ne \Eta$,&$\Xi > \Eta$,&$\Xi\ \hbox{\rm nicht} < \Eta$&f\"ur $|\Xi| < |\Eta|$,\cr
$\Xi = \Eta$,&$\Xi\ \hbox{\rm nicht} > \Eta$,&$\Xi\ \hbox{\rm nicht} < \Eta$&f\"ur $|\Xi| = |\Eta|$,\cr
$\Xi \ne \Eta$,&$\Xi\ \hbox{\rm nicht} > \Eta$,&$\Xi < \Eta$&f\"ur $|\Xi| > |\Eta|$.\cr}}$$
\medskip

%\jutting{4}{351-351}

{\bf Definition 47:} {\it $$\Xi \ge \Eta$$
bedeutet
$$\Xi > \Eta \quad\hbox{\it oder}\quad \Xi = \Eta.$$
{\rm ($\ge$ sprich: gr\"o{\ss}er oder gleich.)}}
\medskip

%\jutting{4}{352-352}

{\bf Definition 48:} {\it $$\Xi \le \Eta$$
bedeutet
$$\Xi < \Eta \quad\hbox{\it oder}\quad \Xi = \Eta.$$
{\rm ($\le$ sprich: kleiner oder gleich.)}}
\medskip

%\jutting{4}{353-403}

{\bf Satz 168:} {\it Aus
$$\Xi \ge \Eta$$
folgt
$$\Eta \le \Xi$$
und umgekehrt.}

{\bf Beweis:} Definition 46.
\medskip

%\jutting{4}{404-411}

{\bf Satz 169:} {\it Die positiven Zahlen sind die Zahlen $> 0$; die nega%
tiven Zahlen sind die Zahlen $< 0$.}

{\bf Beweis:} 1) Nach Definition 45 ist
$$\xi > 0.$$

2) Aus
$$\Xi > 0$$
folgt nach Definition 45
$$\Xi = \xi.$$

3) Nach Definition 46 ist
$$-\xi < 0.$$

4) Aus
$$\Xi < 0$$
folgt nach Definition 46
$$\Xi = -\xi.$$
\medskip

%\jutting{4}{412-420}

{\bf Satz 170:} {\it $$ |\Xi| \ge 0.$$}%

{\bf Beweis:} Definition 44, Satz 166 und Satz 169.
\medskip

%\jutting{4}{421-449}

{\bf Satz 171} (Transitivit\"at der Ordnung): {\it Aus
$$\Xi < \Eta,\quad \Eta < \Zeta$$
folgt
$$\Xi < \Zeta.$$}%

{\bf Beweis:} 1) Es sei
$$\Zeta > 0.$$

Falls
$$\Xi > 0,$$
ist
$$\Eta > 0,$$
und wir haben den alten Satz 126.

Falls
$$\Xi \le 0,$$
ist gewi{\ss}
$$\Xi < \Zeta.$$

2) Es sei
$$\Zeta = 0.$$
Dann ist
$$\Eta < 0,$$
also
$$\Xi < 0,$$
$$\Xi < \Zeta.$$

3) Es sei
$$\Zeta < 0.$$
Dann ist
$$\Eta < 0,$$
$$\Xi < 0.$$
Ferner ist
$$|\Xi| > |\Eta|,\quad |\Eta| > |\Zeta|,$$
also
$$|\Xi| > |\Zeta|,$$
$$\Xi < \Zeta.$$
\medskip

%\jutting{4}{450-457}

{\bf Satz 172:} {\it Aus
$$\Xi \le \Eta,\quad \Eta < \Zeta\quad\hbox{\it oder}\quad \Xi < \Eta,\quad \Eta \le \Zeta$$
folgt
$$\Xi < \Zeta.$$}%

{\bf Beweis:} Mit dem Gleichheitszeichen in der Voraussetzung klar:
sonst durch Satz 171 erledigt.
\medskip

%\jutting{4}{458-471}

{\bf Satz 173:} {\it Aus
$$\Xi \le \Eta,\quad \Eta \le \Zeta$$
folgt
$$\Xi \le \Zeta.$$}%

{\bf Beweis:} Mit zwei Gleichheitszeichen in der Voraussetzung
klar; sonst durch Satz 172 erledigt.
\medskip

%\jutting{4}{472-502}

{\bf Definition 49:} {\it Ist
$$\Xi \le 0,$$
so hei{\ss}t $\Xi$ rational, wenn
$$\Xi = 0$$
oder
$$\Xi < 0,\quad\hbox{\it $|\Xi|$ rational}.$$}%

Wir haben also jetzt positive rationale Zahlen, die rationale
Zahl $0$ und negative rationale Zahlen.
\medskip

%\jutting{4}{503-518}

{\bf Definition 50:} {\it Ist
$$\Xi < 0,$$
so hei{\ss}t $\Xi$ irrational, wenn es nicht rational ist.}

Wir haben also jetzt positive irrationale Zahlen und negative
irrationale Zahlen.  (Zahl{\bf en}? Ja; wir hatten ein irrationales $\xi$;
also ist die positive Zahl $\xi + X$ stets irrational, da aus
$$\xi + X = Y$$
folgen w\"urde
$$\xi = Y - X;$$
und $-(\xi + X)$ ist stets negativ irrational.)
\medskip

%\jutting{4}{519-549}

{\bf Definition 51:} {\it Ist
$$\Xi \le 0,$$
so hei{\ss}t $\Xi$ ganz, wenn
$$\Xi = 0$$
oder
$$\Xi < 0,\quad\hbox{\it $|\Xi|$ ganz}.$$}%

Wir haben also jetzt positive ganze Zahlen, die ganze Zahl 0
und negative ganze Zahlen.
\medskip

%\jutting{4}{550-561}

{\bf Satz 174:} {\it Jede ganze Zahl ist rational.}

{\bf Beweis:} F\"ur die positiven Zahlen wissen wir das; f\"ur 0 und
negative Zahlen folgt es aus Definition 49 und Definition 51.
\vfill\eject


%\jutting{4}{562-812}

\line{}\vskip 7\baselineskip
\centerline{{\S}~3.}
\medskip

\centerline{\bf Addition.}
\bigskip

{\bf Definition 52:}
{\it $$\Xi + \Eta = \cases{-(|\Xi| + |\Eta|),\quad\hbox{\it wenn}\quad \Xi < 0,\quad \Eta < 0;\cr
|\Xi| - |\Eta|,\quad\hbox{\it wenn}\quad \Xi > 0,\quad \Eta < 0,\quad |\Xi| > |\Eta|;\cr
0,\quad\hbox{\it wenn}\quad \Xi > 0,\quad \Eta < 0,\quad |\Xi| = |\Eta|;\cr
-(|\Eta| - |\Xi|),\quad\hbox{\it wenn}\quad \Xi > 0,\quad \Eta < 0,\quad |\Xi| < |\Eta|;\cr
\Eta + \Xi,\quad\hbox{\it wenn}\quad \Xi < 0,\quad \Eta > 0;\cr
\Eta,\quad\hbox{\it wenn}\quad \Xi = 0;\cr
\Xi,\quad\hbox{\it wenn}\quad \Eta = 0.\cr}$$
{\rm (+ sprich: plus.)}  $\Xi + \Eta$ hei{\ss}t die Summe von $\Xi$ und $\Eta$ oder die
durch Addition von $\Eta$ zu $\Xi$ entstehende Zahl.}

Man beachte bei dieser Definition:

1) F\"ur
$$\Xi > 0,\quad \Eta > 0$$
haben wir den Begriff $\Xi + \Eta$ schon aus Definition 34.

2) Er wurde auch in Definition 52 benutzt.

3) Der dritte Fall der Definition benutzt den Begriff der
Summe im zweiten Fall.

4) Der vierte und f\"unfte Fall \"uberdecken sich, wenn
$$\Xi = \Eta = 0;$$
aber dann ist die als $\Xi + \Eta$ definierte Zahl die gleiche (n\"amlich $0$).
\medskip

%\jutting{4}{813-839}

{\bf Satz 175} (kommutatives Gesetz der Addition):
{\it $$\Xi + \Eta = \Eta + \Xi.$$}%

{\bf Beweis:} F\"ur
$$\Xi = 0$$
sind beide Zahlen $\Eta$; f\"ur
$$\Eta = 0$$
sind beide $\Xi$.

F\"ur
$$\Xi > 0,\quad \Eta > 0$$
liegt der alte Satz 130 vor.

F\"ur
$$\Xi < 0,\quad \Eta < 0$$
ist nach Satz 130
$$\Xi + \Eta = -(|\Xi| + |\Eta|) = -(|\Eta| + |\Xi|) = \Eta + \Xi.$$

F\"ur
$$\Xi < 0,\quad \Eta > 0$$
war die Behauptung geradezu Definition.

F\"ur
$$\Xi > 0,\quad \Eta < 0$$
ist nach dem vorangehenden Fall
$$\Eta + \Xi = \Xi + \Eta,$$
also
$$\Xi + \Eta = \Eta + \Xi.$$
\medskip

%\jutting{4}{840-871}

{\bf Definition 53:} {\it $$-\Xi = \cases{0 \quad\hbox{\it f\"ur}\quad \Xi = 0,\cr
|\Xi| \quad\hbox{\it f\"ur}\quad \Xi < 0.\cr}$$
{\rm ($-$ sprich: minus.)}}

Man beachte, da{\ss} wir f\"ur $\Xi > 0$ den Begriff $-\Xi$ aus Defi%
nition 43 schon haben.
\medskip

%\jutting{4}{872-886}

{\bf Satz 176:} {\it Ist
$$\Xi > 0 \quad\hbox{\it bzw.}\quad \Xi = 0 \quad\hbox{\it bzw.}\quad \Xi < 0,$$
so ist
$$-\Xi < 0 \quad\hbox{\it bzw.}\quad -\Xi = 0 \quad\hbox{\it bzw.}\quad -\Xi > 0,$$
und umgekehrt.}

{\bf Beweis:} Definition 43 und Definition 53.
\medskip

%\jutting{4}{887-898}

{\bf Satz 177:} {\it $$-(-\Xi) = \Xi.$$}%

{\bf Beweis:} Definitionen 43, 44 und 53.
\medskip

%\jutting{4}{899-908}

{\bf Satz 178:} {\it $$\left|-\Xi\right| = |\Xi|.$$}%

{\bf Beweis:} Definitionen 43, 44 und 53.
\medskip

%\jutting{4}{909-918}

{\bf Satz 179:} {\it $$\Xi + (-\Xi) = 0.$$}%

{\bf Beweis:} Definition 52, Definition 53 und Satz 178.
\medskip

%\jutting{4}{919-955}

{\bf Satz 180:} {\it $$-(\Xi + \Eta) = -\Xi + (-\Eta).$$}%

{\bf Beweis:} Nach Satz 175 ist
$$-(\Xi + \Eta) = -(\Eta + \Xi)$$
und
$$-\Xi + (-\Eta) = -\Eta + (-\Xi);$$
daher darf ohne Beschr\"ankung der Allgemeinheit
$$\Xi \ge \Eta$$
vorausgesetzt werden; denn mindestens eine der Relationen
$$\Xi \ge \Eta,\quad \Eta \ge \Xi$$
besteht, und aus
$$-(\Eta + \Xi) = -\Eta + (-\Xi)$$
folgt eben
$$-(\Xi + \Eta) = -\Xi + (-\Eta).$$

Es sei also
$$\Xi \ge \Eta.$$

1) Ist
$$\Xi > 0,\quad \Eta > 0,$$
so ist
$$-\Xi + (-\Eta) = -(\Xi + \Eta).$$

2) Ist
$$\Xi > 0,\quad \Eta = 0,$$
so ist
$$-\Xi + (-\Eta) = -\Xi + 0 = -\Xi = -(\Xi + 0) = -(\Xi + \Eta).$$

3) Ist
$$\Xi > 0,\quad \Eta < 0,$$
so ist

entweder
$$\Xi > |\Eta|,$$
also
$$\Xi + \Eta = \Xi - |\Eta|,$$
$$-\Xi + (-\Eta) = -\Xi + |\Eta| = -(\Xi - |\Eta|) = -(\Xi + \Eta);$$

oder
$$\Xi = |\Eta|,$$
also
$$\Xi + \Eta = 0,$$
$$-\Xi + (-\Eta) = -\Xi + |\Eta| = 0 = -(\Xi + \Eta);$$

oder
$$\Xi < |\Eta|,$$
also
$$\Xi + \Eta = -(|\Eta| - \Xi),$$
$$-\Xi + (-\Eta) = -\Xi + |\Eta| = |\Eta| - \Xi = -(\Xi + \Eta);$$

4) Ist
$$\Xi = 0,$$
so ist
$$-\Xi + (-\Eta) = 0 + (-\Eta) = -\Eta = -(0 + \Eta) = -(\Xi + \Eta).$$

5) Ist
$$\Xi < 0,$$
so ist
$$\Eta < 0,$$
$$\Xi + \Eta = -(|\Xi| + |\Eta|),$$
$$-\Xi + (-\Eta) = |\Xi| + |\Eta| = -(\Xi + \Eta).$$
\medskip

%\jutting{4}{956-966}

{\bf Definition 54:} {\it $$\Xi - \Eta = \Xi + (-\Eta).$$
{\rm ($-$ sprich: minus.)}  $\Xi - \Eta$ hei{\ss}t die Differenz $\Xi$ minus $\Eta$ oder die
durch Subtraktion des $\Eta$ von $\Xi$ entstehende Zahl.}

Man beachte, da{\ss} Definition 54 (wie es sein mu{\ss}) f\"ur
$$\Xi > \Eta > 0$$
mit der alten Definition 35 \"ubereinstimmt; denn dann ist
$$\Xi > 0,\quad -\Eta < 0,\quad |\Xi| > \left|-\Eta\right|,\quad \Xi + (-\Eta) = |\Xi| - \left|-\Eta\right| = \Xi - \Eta.$$
\medskip

%\jutting{4}{967-968}

{\bf Satz 181:} {\it $$-(\Xi - \Eta) = \Eta - \Xi.$$}%

{\bf Beweis:} Nach Satz 180 und Satz 177 ist
$$\displaylines{-(\Xi + \Eta) = -\bigl(\Xi + (-\Eta)\bigr) = -\Xi + \bigl(-(-\Eta)\bigr) = -\Xi + \Eta = \Eta + (-\Xi)\cr
= \Eta - \Xi.\cr}$$
\medskip

%\jutting{4}{969-1069}

{\bf Satz 182:} {\it Aus
$$\Xi - \Eta > 0 \quad\hbox{\it bzw.}\quad \Xi - \Eta = 0 \quad\hbox{\it bzw.}\quad \Xi - \Eta < 0$$
folgt
$$\Xi > \Eta \quad\hbox{\it bzw.}\quad \Xi = \Eta \quad\hbox{\it bzw.}\quad \Xi < \Eta$$
und umgekehrt.}

{\bf Beweis:} Da $-\Eta$ auch eine beliebige reelle Zahl ist, darf
man $-\Eta$ statt $\Eta$ schreiben und hat demnach das Entsprechen der
F\"alle bei
$$\Xi + \Eta > 0 \quad\hbox{\rm bzw.}\quad \Xi + \Eta = 0 \quad\hbox{\rm bzw.}\quad \Xi + \Eta < 0$$
und
$$\Xi > -\Eta \quad\hbox{\rm bzw.}\quad \Xi = -\Eta \quad\hbox{\rm bzw.}\quad \Xi < -\Eta$$
zu zeigen.

In der Tat ist f\"ur $\Xi = 0$ oder $\Eta = 0$ die Behauptung klar;
im \"ubrigen gelten im Fall
$$\Xi > 0,\quad \Eta > 0$$
und in den drei ersten F\"allen der Definition 52, wenn der dritte
in die drei Unterf\"alle
$$|\Eta| > |\Xi|,\quad |\Eta| = |\Xi|,\quad |\Eta| < |\Xi|$$
zerlegt wird, beide Male resp. die Zeichen
$$> \quad < \quad > \quad = \quad < \quad > \quad = \quad <.$$
\medskip

%\jutting{4}{1070-1113}

{\bf Satz 183:} {\it Aus
$$\Xi > \Eta \quad\hbox{\it bzw.}\quad \Xi = \Eta \quad\hbox{\it bzw.}\quad \Xi < \Eta$$
folgt
$$-\Xi < -\Eta \quad\hbox{\it bzw.}\quad -\Xi = -\Eta \quad\hbox{\it bzw.}\quad -\Xi > -\Eta$$
und umgekehrt.}

{\bf Beweis:} Nach Satz 182 entspricht ersteres den F\"allen
$$\Xi - \Eta > 0 \quad\hbox{\rm bzw.}\quad \Xi - \Eta = 0 \quad\hbox{\rm bzw.}\quad \Xi - \Eta < 0,$$
letzteres den F\"allen
$$-\Eta - (-\Xi) > 0 \quad\hbox{\rm bzw.}\quad -\Eta - (-\Xi) = 0 \quad\hbox{\rm bzw.}\quad -\Eta - (-\Xi) < 0,$$
also liefert
$$-\Eta - (-\Xi) = -\Eta + \bigl(-(-\Xi)\bigr) = -\Eta + \Xi = \Xi + (-\Eta) = \Xi - \Eta$$
alles.
\medskip

%\jutting{4}{1114-1178}

{\bf Satz 184:} {\it Jede reelle Zahl l\"a{\ss}t sich als Differenz zweier positiver
Zahlen darstellen.}

{\bf Beweis:} 1) Ist
$$\Xi > 0,$$
so ist
$$\Xi = (\Xi + 1) - 1.$$

2) Ist
$$\Xi = 0,$$
so ist
$$\Xi = 1 - 1.$$

3) Ist
$$\Xi < 0,$$
so ist
$$-\Xi = |\Xi| = (|\Xi| + 1) - 1,$$
$$\Xi = -\bigl((|\Xi| + 1) - 1\bigr) = 1 - (|\Xi| + 1).$$
\medskip

%\jutting{4}{1179-1256}

{\bf Satz 185:} {\it Aus
$$\Xi = \xi_1 - \xi_2,\quad \Eta = \eta_1 - \eta_2$$
folgt
$$\Xi + \Eta = (\xi_1 + \eta_1) - (\xi_2 + \eta_2).$$}%

{\bf Beweis:} 1) Es sei
$$\Xi > 0,\quad \Eta > 0.$$
Dann ist, da
$$\displaylines{(\alpha + \beta) + (\gamma + \delta) = (\alpha + \beta) + (\delta + \gamma) = \bigl((\alpha + \beta) + \delta\bigr) + \gamma\cr
= \gamma + \bigl(\alpha + (\beta + \delta)\bigr) = (\gamma + \alpha) + (\beta + \delta)\cr}$$
ist,
$$(\Xi + \Eta) + (\xi_2 + \eta_2) = \xi_1 + \eta_1,$$
also die Behauptung wahr.

2) Es sei
$$\Xi < 0,\quad \Eta < 0.$$
Dann ist nach Satz 181
$$\xi_2 - \xi_1 = -\Xi > 0,\quad \eta_2 - \eta_1 = -\Eta > 0,$$
also nach 1)
$$-\Xi + (-\Eta) = (\xi_2 + \eta_2) - (\xi_1 + \eta_1),$$
$$\Xi + \Eta = -\bigl(-\Xi + (-\Eta)\bigr) = (\xi_1 + \eta_1) - (\xi_2 + \eta_2).$$

3) Es sei
$$\Xi > 0, \Eta < 0.$$
also
$$\xi_1 - \xi_2 > 0,\quad \eta_2 - \eta_1 > 0.$$

A) Ist
$$\Xi > |\Eta|,$$
so ist
$$\xi_1 - \xi_2 > \eta_2 - \eta_1,$$
also
$$\displaylines{\xi_1 + \eta_1 = \bigl((\xi_1 - \xi_2) + \xi_2\bigr) + \eta_1 = (\xi_1 - \xi_2) + (\xi_2 + \eta_1) = (\xi_2 + \eta_1) + (\xi_1 - \xi_2)\cr
= (\xi_2 + \eta_1) + \Bigl((\eta_2 - \eta_1) + \bigl((\xi_1 - \xi_2) - (\eta_2 - \eta_1)\bigr)\Bigr)\cr
= \bigl((\xi_2 + \eta_1) + (\eta_2 - \eta_1)\bigr) + \bigl((\xi_1 - \xi_2) - (\eta_2 - \eta_1)\bigr)\cr
= \Bigl(\xi_2 + \bigl(\eta_1 + (\eta_2 - \eta_1)\bigr)\Bigr) + \bigl((\xi_1 - \xi_2) - (\eta_2 - \eta_1)\bigr)\cr
= (\xi_2 + \eta_2) + \bigl((\xi_1 - \xi_2) - (\eta_2 - \eta_1)\bigr),\cr}$$
$$(\xi_1 + \eta_1) - (\xi_2 + \eta_2) = (\xi_1 - \xi_2) - (\eta_2 - \eta_1) = \Xi - |\Eta| = \Xi + \Eta.$$

B) Ist
$$\Xi < \Eta,$$
so ist nach A)
$$\displaylines{\Xi + \Eta = -\bigl(-\Eta + (-\Xi)\bigr) = -\bigl((\eta_2 - \eta_1) + (\xi_2 - \xi_1)\bigr)\cr
= -\bigl((\eta_2 + \xi_2) - (\eta_1 + \xi_1)\bigr) = (\eta_1 + \xi_1) - (\eta_2 + \xi_2)\cr
= (\xi_1 + \eta_1) - (\xi_2 + \eta_2).\cr}$$

C) Ist
$$\Xi = |\Eta|,$$
also
$$\xi_1 - \xi_2 = \eta_2 - \eta_1,$$
so ist
$$\xi_1 = \xi_2 + (\eta_2 - \eta_1),$$
$$\xi_1 + \eta_1 = \xi_2 + \eta_2,$$
$$\Xi + \Eta = 0 = (\xi_1 + \eta_1) - (\xi_2 + \eta_2).$$

4) Es Sei
$$\Xi < 0,\quad \Eta > 0.$$
Dann ist nach 3)
$$\Eta + \Xi = (\eta_1 + \xi_1) - (\eta_2 + \xi_2),$$
$$\Xi + \Eta = (\xi_1 + \eta_1) - (\xi_2 + \eta_2).$$

5) Es sei
$$\Xi = 0.$$
Dann ist
$$\xi_1 = \xi_2,$$
$$\Xi + \Eta = \Eta.$$

a) F\"ur
$$\eta_1 > \eta_2$$
ist
$$(\eta_1 - \eta_2) + (\xi_1 + \eta_2) = \bigl((\eta_1 - \eta_2) + \eta_2) + \xi_1 = \eta_1 + \xi_1 = \xi_1 + \eta_1,$$
$$\Eta = \eta_1 - \eta_2 = (\xi_1 + \eta_1) - (\xi_1 + \eta_2) = (\xi_1 + \eta_1) - (\xi_2 + \eta_2).$$

b) F\"ur
$$\eta_1 = \eta_2$$
ist
$$\Eta = 0 = (\xi_1 + \eta_1) - (\xi_2 + \eta_2).$$

c) F\"ur
$$\eta_1 < \eta_2$$
ist nach a)
$$-\Eta = \eta_2 - \eta_1 = (\xi_2 + \eta_2) - (\xi_1 + \eta_1),$$
$$\Eta = -(-\Eta) = (\xi_1 + \eta_1) - (\xi_2 + \eta_2).$$

6) Es sei
$$\Eta = 0.$$
Dann ist nach 5)
$$\Xi + \Eta = \Eta + \Xi = (\eta_1 + \xi_1) - (\eta_2 + \xi_2) = (\xi_1 + \eta_1) - (\xi_2 + \eta_2).$$
\medskip

%\jutting{4}{1257-1274}

{\bf Satz 186} (assoziatives Gesetz der Addition):
{\it $$(\Xi + \Eta) + \Zeta = \Xi + (\Eta + \Zeta).$$}%

{\bf Beweis:} Nach Satz 184 ist
$$\Xi = \xi_1 - \xi_2,\quad \Eta = \eta_1 - \eta_2,\quad \Zeta = \zeta_1 - \zeta_2.$$
Nach Satz 185 ist
$$\displaylines{(\Xi + \Eta) + \Zeta = \bigl((\xi_1 + \eta_1) - (\xi_2 + \eta_2)\bigr) + (\zeta_1 - \zeta_2)\cr
= \bigl((\xi_1 + \eta_1) + \zeta_1\bigr) - \bigl((\xi_2 + \eta_2) + \zeta_2\bigr) = \bigl(\xi_1 + (\eta_1 + \zeta_1)\bigr) - \bigl(\xi_2 + (\eta_2 + \zeta_2)\bigr)\cr
=(\xi_1 - \xi_2) + \bigl((\eta_1 + \zeta_1) - (\eta_2 + \zeta_2)\bigr) = \Xi + (\Eta + \Zeta).\cr}$$
\medskip

%\jutting{4}{1275-1291}

{\bf Satz 187:} {\it Bei gegebenen $\Xi$, $\Eta$ hat
$$\Eta + \Upsilon = \Xi$$
genau eine L\"osung, n\"amlich
$$\Upsilon = \Xi - \Eta.$$}%

{\bf Beweis:} 1) $$\Upsilon = \Xi - \Eta$$
ist eine L\"osung, da nach Satz 186
$$\displaylines{\Eta + (\Xi - \Eta) = (\Xi - \Eta) + \Eta = \bigl(\Xi + (-\Eta)\bigr) + \Eta = \Xi + (-\Eta + \Eta)\cr
= \Xi + 0 = \Xi.\cr}$$

2) Aus
$$\Eta + \Upsilon = \Xi$$
folgt
$$\displaylines{\Xi - \Eta = \Xi + (-\Eta) = -\Eta + \Xi = -\Eta + (\Eta + \Upsilon) = (-\Eta + \Eta) + \Upsilon\cr
= 0 + \Upsilon = \Upsilon.\cr}$$
\medskip

%\jutting{4}{1292-1344}

{\bf Satz 188:} {\it Es ist
$$\Xi + \Zeta > \Eta + \Zeta \quad\hbox{\it bzw.}\quad \Xi + \Zeta = \Eta + \Zeta \quad\hbox{\it bzw.}\quad \Xi + \Zeta < \Eta + \Zeta,$$
je nachdem
$$\Xi > \Eta \quad\hbox{\it bzw.}\quad \Xi = \Eta \quad\hbox{\it bzw.}\quad \Xi < \Eta.$$}%

{\bf Beweis:} Nach Satz 182 gilt ersteres, je nachdem
$$\displaylines{(\Xi + \Zeta) - (\Eta + \Zeta) > 0 \quad\hbox{\rm bzw.}\quad (\Xi + \Zeta) - (\Eta + \Zeta) = 0\cr
\hbox{\rm bzw.}\quad (\Xi + \Zeta) - (\Eta + \Zeta) < 0;\cr}$$
letzteres, je nachdem
$$\Xi - \Eta > 0 \quad\hbox{\rm bzw.}\quad \Xi - \Eta = 0 \quad\hbox{\rm bzw.}\quad \Xi - \Eta < 0.$$

Aus
$$\displaylines{(\Xi + \Zeta) - (\Eta + \Zeta) = (\Xi + \Zeta) + \bigl(-\Zeta + (-\Eta)\bigr) = \Bigl(\Xi + \bigl(\Zeta + (-\Zeta)\bigr)\Bigr) + (-\Eta)\cr
=\Xi + (-\Eta) = \Xi - \Eta\cr}$$
folgen also die Behauptungen.
\medskip

%\jutting{4}{1345-1352}

{\bf Satz 189:} {\it Aus
$$\Xi > \Eta,\quad \Zeta > \Upsilon$$
folgt
$$\Xi + \Zeta > \Eta + \Upsilon.$$}%

{\bf Beweis:} Nach Satz 188 ist
$$\Xi + \Zeta > \Eta + \Zeta$$
und
$$\Eta + \Zeta = \Zeta + \Eta > \Upsilon + \Eta = \Eta + \Upsilon,$$
also
$$\Xi + \Zeta > \Eta + \Upsilon.$$
\medskip

%\jutting{4}{1353-1360}

{\bf Satz 190:} {\it Aus
$$\Xi \ge \Eta,\quad \Zeta > \Upsilon \quad\hbox{\it oder}\quad \Xi > \Eta,\quad \Zeta \ge \Upsilon$$
folgt
$$\Xi + \Zeta > \Eta + \Upsilon.$$}%

{\bf Beweis:} Mit dem Gleichheitszeichen in der Voraussetzung
durch Satz 188, sonst durch Satz 189 erledigt.
\medskip

%\jutting{4}{1361-1373}

{\bf Satz 191:} {Aus
$$\Xi \ge \Eta,\quad \Zeta \ge \Upsilon$$
folgt
$$\Xi + \Zeta \ge \Eta + \Upsilon.$$}%

Beweis: Mit zwei Gleichheitszeichen in der Voraussetzung
klar; sonst durch Satz 190 erledigt.
\vfill\eject


%\jutting{4}{1374-1539}

\line{}\vskip 7\baselineskip
\centerline{{\S}~4.}
\medskip

\centerline{\bf Multiplikation.}
\bigskip

{\bf Definition 55:}
{\it $$\Xi \cdot \Eta = \cases{-(\left|\Xi\right|\left|\Eta\right|),\quad\hbox{\it wenn}\quad \Xi > 0,\quad \Eta < 0 \quad\hbox{\it oder}\quad \Xi < 0,\quad \Eta > 0;\cr
\left|\Xi\right|\left|\Eta\right|,\quad\hbox{\it wenn}\quad \Xi < 0,\quad \Eta < 0;\cr
0,\quad\hbox{\it wenn}\quad \Xi = 0 \quad\hbox{\it oder}\quad \Eta = 0.\cr}$$
{\rm ($\cdot$ sprich: mal; aber man schreibt den Punkt meist nicht.)}  $\Xi \cdot \Eta$
hei{\ss}t das Produkt von $\Xi$ mit $\Eta$ oder die durch Multiplikation von $\Xi$
mit $\Eta$ entstehende Zahl.}

Man beachte, da{\ss} $\Xi \cdot \Eta$ f\"ur $\Xi > 0$, $\Eta > 0$ uns schon aus Defi%
nition 36 bekannt ist, was ja auch in Definition 55 benutzt wurde.
\medskip

%\jutting{4}{1540-1561}

{\bf Satz 192:} {\it Es ist
$$\Xi\Eta = 0$$
dann und nur dann, wenn mindestens eine der beiden Zahlen $\Xi$, $\Eta$
Null ist.}

{\bf Beweis:} Definition 55.
\medskip

%\jutting{4}{1562-1585}

{\bf Satz 193:} {\it $$|\Xi\Eta| = \left|\Xi\right|\left|\Eta\right|.$$}%

{\bf Beweis:} Definition 55.
\medskip

%\jutting{4}{1586-1604}

{\bf Satz 194} (kommutatives Gesetz der Multiplikation):
{\it $$\Xi\Eta = \Eta\Xi.$$}%

{\bf Beweis:} Das ist f\"ur $\Xi > 0$, $\Eta > 0$ der Satz 142 und folgt
sonst aus Definition 55, da die rechte Seite dieser Definition (nach
Satz 142) und die Fallunterscheidung in $\Xi$, $\Eta$ symmetrisch sind.
\medskip

%\jutting{4}{1605-1619}

{\bf Satz 195:} {\it $$\Xi \cdot 1 = \Xi.$$}%

{\bf Beweis:} F\"ur $\Xi > 0$ folgt dies aus Satz 151; f\"ur $\Xi = 0$ aus
Definition 55; f\"ur $\Xi < 0$ ist nach Definition 55
$$\Xi \cdot 1 = -(|\Xi| \cdot 1) = -|\Xi| = \Xi.$$
\medskip

%\jutting{4}{1620-1702}

{\bf Satz 196:} {\it Ist
$$\Xi \ne 0,\quad \Eta \ne 0,$$
so ist
$$\Xi\Eta = \left|\Xi\right|\left|\Eta\right| \quad\hbox{\it bzw.}\quad \Xi\Eta = -(\left|\Xi\right|\left|\Eta\right|),$$
je nachdem keine oder zwei bzw. genau eine der Zahlen $\Xi$, $\Eta$ negativ sind.}

{\bf Beweis:} Definition 55.
\medskip

%\jutting{4}{1703-1749}

{\bf Satz 197:} {\it $$(-\Xi)\Eta = \Xi(-\Eta) = -(\Xi\Eta).$$}%

{\bf Beweis:} 1) Ist eine der Zahlen $\Xi$, $\Eta$ Null, so sind alle drei
Ausdr\"ucke $0$.

2) Ist
$$\Xi \ne 0,\quad \Eta \ne 0,$$
so haben nach Satz 193 alle drei Ausdr\"ucke denselben absoluten
Betrag $\left|\Xi\right|\left|\Eta\right|$, und nach Satz 196 sind alle drei $> 0$ bzw. $< 0$,
je nachdem genau eine bzw. keine oder zwei der Zahlen $\Xi$, $\Eta$
negativ sind.
\medskip

%\jutting{4}{1750-1751}

{\bf Satz 198:} {\it $$(-\Xi)(-\Eta) = \Xi\Eta.$$}%

{\bf Beweis:} Nach Satz 197 ist
$$(-\Xi)(-\Eta) = \Xi\bigl(-(-\Eta)\bigr) = \Xi\Eta.$$
\medskip

%\jutting{4}{1752-1814}

{\bf Satz 199} (assoziatives Gesetz der Multiplikation):
{\it $$(\Xi\Eta)\Zeta = \Xi(\Eta\Zeta).$$}%

{\bf Beweis:} 1) Ist eine der Zahlen $\Xi$, $\Eta$, $\Zeta$ Null, so sind beide
Seiten der Behauptung 0.

2) Ist
$$\Xi \ne 0,\quad \Eta \ne 0,\quad \Zeta \ne 0,$$
so haben nach Satz 193 beide Seiten denselben absoluten Betrag
$$(\left|\Xi\right|\left|\Eta\right|)\left|\Zeta\right| = \left|\Xi\right|(\left|\Eta\right|\left|\Zeta\right|),$$
und nach Satz 196 sind beide Seiten $> 0$ bzw. $< 0$, je nachdem
keine oder genau zwei bzw. genau eine oder drei der Zahlen $\Xi$,
$\Eta$, $\Zeta$ negativ sind.
\medskip

%\jutting{4}{1815-1844}

{\bf Satz 200:} {\it $$\xi(\eta - \zeta) = \xi\eta - \xi\zeta.$$}%

{\bf Beweis:} 1) F\"ur
$$\eta > \zeta$$
ist
$$(\eta - \zeta) + \zeta = \eta,$$
also nach Satz 144
$$\xi(\eta - \zeta) + \xi\zeta = \xi\eta,$$
$$\xi(\eta - \zeta) = \xi\eta - \xi\zeta.$$

2) F\"ur
$$\eta = \zeta$$
ist
$$\xi\eta = \xi\zeta,$$
$$\xi(\eta - \zeta) = \xi \cdot 0 = 0 = \xi\eta - \xi\zeta.$$

3) F\"ur
$$\eta < \zeta$$
ist nach 1)
$$\xi(\eta - \zeta) = \xi\bigl(-(\zeta - \eta)\bigr) = -\bigl(\xi(\zeta - \eta)\bigr) = -(\xi\zeta - \xi\eta) = \xi\eta - \xi\zeta.$$
\medskip

%\jutting{4}{1845-1868}

{\bf Satz 201} (distributives Gesetz):
{\it $$\Xi(\Eta + \Zeta) = \Xi\Eta + \Xi\Zeta.$$}%

{\bf Beweis:} 1) Es sei
$$\Xi > 0.$$
Nach Satz 184 ist
$$\Eta = \eta_1 - \eta_2,\quad \Zeta = \zeta_1 - \zeta_2,$$
nach Satz 185 somit
$$\Eta + \Zeta = (\eta_1 + \zeta_1) - (\eta_2 + \zeta_2),$$
also nach Satz 200 und Satz 144
$$\Xi(\Eta + \Zeta) = \Xi(\eta_1 + \zeta_1) - \Xi(\eta_2 + \zeta_2) = (\Xi\eta_1 + \Xi\zeta_1) - (\Xi\eta_2 + \Xi\zeta_2),$$
also nach Satz 185 und Satz 200
$$\displaylines{\Xi(\Eta + \Zeta) = (\Xi\eta_1 - \Xi\eta_2) + (\Xi\zeta_1 - \Xi\zeta_2) = \Xi(\eta_1 - \eta_2) + \Xi(\zeta_1 - \zeta_2)\cr
= \Xi\Eta + \Xi\Zeta.\cr}$$

2) Es sei
$$\Xi = 0.$$
Dann ist
$$\Xi(\Eta + \Zeta) = 0 = \Xi\Eta + \Xi\Zeta.$$

3) Es sei
$$\Xi < 0.$$
Dann ist nach 1)
$$(-\Xi)(\Eta + \Zeta) = (-\Xi)\Eta + (-\Xi)\Zeta.$$
also
$$-\bigl(\Xi(\Eta + \Zeta)\bigr) = (-\Xi)\Eta + (-\Xi)\Zeta,$$
$$\displaylines{\Xi(\Eta + \Zeta) = -\bigl((-\Xi)\Eta + (-\Xi)\Zeta\bigr) = -\bigl((-\Xi)\Eta\bigr) + \Bigl(-\bigl((-\Xi)\Zeta\bigr)\Bigr)\cr
= \Xi\Eta + \Xi\Zeta.\cr}$$
\medskip

%\jutting{4}{1869-1873}

{\bf Satz 202:} {\it $$\Xi(\Eta - \Zeta) = \Xi\Eta - \Xi\Zeta.$$}%

{\bf Beweis:} Nach Satz 201 ist
$$\displaylines{\Xi(\Eta - \Zeta) = \Xi\bigl(\Eta + (-\Zeta)\bigr) = \Xi\Eta + \Xi(-\Zeta) = \Xi\Eta + \bigl(-(\Xi\Zeta)\bigr)\cr
= \Xi\Eta - \Xi\Zeta.\cr}$$
\medskip

%\jutting{4}{1874-1907}

{\bf Satz 203:} {\it Es sei
$$\Xi > \Eta.$$
Aus
$$\Zeta > 0 \quad\hbox{\it bzw.}\quad \Zeta = 0 \quad\hbox{\it bzw.}\quad \Zeta < 0$$
folgt dann
$$\Xi\Zeta > \Eta\Zeta \quad\hbox{\it bzw.}\quad \Xi\Zeta = \Eta\Zeta \quad\hbox{\it bzw.}\quad \Xi\Zeta < \Eta\Zeta.$$}%

{\bf Beweis:} $$\Xi - \Eta > 0,$$
also
$$(\Xi - \Eta)\Zeta > 0 \quad\hbox{\rm bzw.}\quad (\Xi - \Eta)\Zeta = 0 \quad\hbox{\rm bzw.}\quad (\Xi - \Eta)\Zeta < 0,$$
je nachdem
$$\Zeta > 0 \quad\hbox{\rm bzw.}\quad \Zeta = 0 \quad\hbox{\rm bzw.}\quad \Zeta < 0.$$

Da nach Satz 202
$$(\Xi - \Eta)\Zeta = \Zeta(\Xi - \Eta) = \Zeta\Xi - \Zeta\Eta = \Xi\Zeta - \Eta\Zeta$$
ist, ist in diesen F\"allen nach Satz 182
$$\Xi\Zeta > \Eta\Zeta \quad\hbox{\rm bzw.}\quad \Xi\Zeta = \Eta\Zeta \quad\hbox{\rm bzw.}\quad \Xi\Zeta < \Eta\Zeta.$$
\medskip

%\jutting{4}{1908-1949}

{\bf Satz 204:} {\it Die Gleichung
$$\Eta\Upsilon = \Xi,$$
wo $\Xi$, $\Eta$ gegeben sind und
$$\Eta \ne 0$$
ist, hat genau eine L\"osung $\Upsilon$.}

{\bf Beweis:} I) Es gibt h\"ochstens eine L\"osung; denn aus
$$\Eta\Upsilon_1 = \Xi = \Eta\Upsilon_2$$
folgt
$$0 = \Eta\Upsilon_1 - \Eta\Upsilon_2 = \Eta(\Upsilon_1 - \Upsilon_2),$$
also nach Satz 192
$$0 = \Upsilon_1 - \Upsilon_2,$$
$$\Upsilon_1 = \Upsilon_2.$$

II) 1) Es sei
$$\Eta > 0.$$
Dann ist
$$\Upsilon = {1 \over \Eta}\Xi$$
eine L\"osung wegen
$$\Eta\Upsilon = \Eta\left({1 \over \Eta}\Xi\right) = \left(\Eta{1 \over \Eta}\right)\Xi = 1 \cdot \Xi = \Xi.$$

2) Es sei
$$\Eta < 0.$$
Dann ist
$$\Upsilon = -\left({1 \over |\Eta|}\Xi\right)$$
eine L\"osung.  Denn nach 1) ist
$$\Xi = \left|\Eta\right|\left({1 \over |\Eta|}\Xi\right) = \left|\Eta\right|(-\Upsilon) = (-|\Eta|)\Upsilon = \Eta\Upsilon.$$
\medskip

{\bf Definition 56:} {\it Das $\Upsilon$ des Satzes 204 hei{\ss}t $\Xi \over \Eta$ {\rm (sprich: $\Xi$ durch
\Eta).}  $\Xi \over \Eta$ hei{\ss}t auch der Quotient von $\Xi$ durch $\Eta$ oder die durch Division
von $\Xi$ durch $\Eta$ entstehende Zahl.}

Man beachte, da{\ss} (wie es sein mu{\ss}) dies f\"ur $\Xi > 0$, $\Eta > 0$ mit
der alten Definition 38 \"ubereinstimmt.
\vfill\eject


%\jutting{4}{1950-2295}

\line{}\vskip 7\baselineskip
\centerline{{\S}~5.}
\medskip

\centerline{\bf Dedekindscher Hauptsatz.}
\bigskip

{\bf Satz 205:} {\it Gegeben sei irgend eine Einteilung aller reellen Zahlen
in zwei Klassen mit folgenden Eigenschaften.

{\rm 1)} Es gibt eine Zahl der ersten Klasse und eine Zahl der zweiten
Klasse.

{\rm 2)} Jede Zahl der ersten Klasse ist kleiner als jede Zahl der
zweiten Klasse.

Dann gibt es genau eine reelle Zahl $\Xi$, so da{\ss} jedes $\Eta < \Xi$ zur
ersten, jedes $\Eta > \Xi$ zur zweiten Klasse geh\"ort.}

Mit anderen Worten: Jede Zahl der ersten Klasse ist $\le \Xi$,
jede Zahl der zweiten Klasse $\ge \Xi$.

{\bf Vorbemerkung:} Es ist umgekehrt klar, da{\ss} jede reelle Zahl
$\Xi$ genau zwei solche Einteilungen erzeugt; die eine mit $\Eta \le \Xi$
als erster, $\Eta > \Xi$ als zweiter Klasse; die andere mit $\Eta < \Xi$ als
erster, $\Eta \ge \Xi$ als zweiter Klasse.

{\bf Beweis:} A) Mehr als ein solches $\Xi$ kann es nicht geben; denn
w\"are
$$\Xi_1 < \Xi_2$$
und leisteten $\Xi_1$ und $\Xi_2$ das Gew\"unschte, so w\"urde $\Xi_1 + \Xi_2 \over 1 + 1$ wegen
$$(1 + 1)\Xi_1 = \Xi_1 + \Xi_1 < \Xi_1 + \Xi_2 < \Xi_2 + \Xi_2 = (1 + 1)\Xi_2,$$
$$\Xi_1 < {\Xi_1 + \Xi_2 \over 1 + 1} < \Xi_2$$
sowohl zur zweiten als auch zur ersten Klasse geh\"oren.

B) Zum Nachweis der Existenz eines $\Xi$ unterscheiden wir vier
F\"alle:

I) Es gebe eine positive Zahl in der ersten Klasse.

Wir betrachten den Schnitt, der folgenderma{\ss}en erzeugt wird:
Jede positive rationale Zahl kommt in die Unterklasse, wenn sie
in der ersten Klasse liegt, ohne die etwaige gr\"o{\ss}te rationale Zahl
der ersten Klasse zu sein; sonst (d. h. wenn sie die etwaige gr\"o{\ss}te
rationale Zahl der ersten Klasse ist oder in der zweiten Klasse
liegt) in die Oberklasse. Das ist wirklich ein Schnitt.  Denn:

1) Da die erste Klasse eine positive Zahl enth\"alt, enth\"alt sie
jede kleinere positive rationale Zahl (eine solche gibt es nach
Satz 158), also eine, zu der es in der ersten Klasse eine gr\"o{\ss}ere
gibt.  Die Unterklasse ist also nicht leer.

Da die zweite Klasse eine Zahl enth\"alt, enth\"alt sie jede
gr\"o{\ss}ere positive rationale Zahl (eine solche gibt es nach Satz 158).
Die Oberklasse ist also nicht leer.

2) Jede Zahl der Unterklasse ist kleiner als jede der Ober%
klasse; denn jede Zahl der ersten Klasse ist kleiner als jede der
zweiten Klasse, und die etwaige gr\"o{\ss}te positive rationale Zahl der
ersten Klasse ist gewi{\ss} gr\"o{\ss}er als jede Zahl der Unterklasse.

3) Die Unterklasse enth\"alt keine gr\"o{\ss}te positive rationale
Zahl.  Denn entweder die erste Klasse enth\"alt schon keine solche.
Oder sie enth\"alt eine solche; dann war diese in die Oberklasse
getan, und unter den positiven rationalen Zahlen, die kleiner als
eine gegebene sind, gibt es schon nach Satz 91 keine gr\"o{\ss}te.

Die durch unseren Schnitt definierte positive Zahl nennen wir
$\Xi$ und behaupten, da{\ss} sie die gestellten Forderungen erf\"ullt.

a) Es sei $\Eta$ mit
$$\Eta < \Xi$$
gegeben.  Wir w\"ahlen nach Satz 159 (mit $\xi = \Eta$, $\eta = \Xi$, wenn
$\Eta > 0$ ist; mit $\xi = {\Xi \over 1 + 1}$, $\eta = \Xi$, wenn $\Eta \le 0$ ist) ein $\Zeta$ mit
$$\Eta < \Zeta < \Xi.$$
Dann ist $\Zeta$ Unterzahl bei $\Xi$, also zur ersten Klasse geh\"orig; daher
geh\"ort $\Eta$ zur ersten Klasse.

b) Es sei $\Eta$ mit
$$\Eta > \Xi$$
gegeben.  Wir w\"ahlen nach Satz 159 ein $\Zeta$ mit
$$\Xi < \Zeta < \Eta.$$
Dann ist $\Zeta$ Oberzahl bei $\Xi$ und (nach Satz 159) nicht die kleinste,
also zur zweiten Klasse geh\"orig; daher geh\"ort $\Eta$ zur zweiten
Klasse.

II) Jede positive Zahl liege in der zweiten Klasse; $0$ liege in
der ersten Klasse.

Dann liegt jede negative Zahl in der ersten Klasse, und
$$\Xi = 0$$
leistet das Gew\"unschte.

III) 0 liege in der zweiten Klasse; jede negative Zahl liege
in der ersten Klasse.

Dann liegt jede positive Zahl in der zweiten Klasse, und
$$\Xi = 0$$
leistet das Gew\"unschte.

IV) Es gebe eine negative Zahl in der zweiten Klasse.
Dann betrachten wir folgende neue Einteilung:
\bigskip

$\Eta$ in der neuen ersten Klasse, wenn $-\Eta$ in der alten zweiten
Klasse lag;

$\Eta$ in der neuen zweiten Klasse, wenn $-\Eta$ in der alten ersten
Klasse lag.
\bigskip
\noindent
Diese Einteilung gen\"ugt offenbar den beiden Bedingungen des
Satzes 205.  Denn

1) in jeder Klasse liegt eine Zahl;

2) aus
$$\Eta_1 < \Eta_2$$
folgt nach Satz 183
$$-\Eta_2 < -\Eta_1.$$

\"Uberdies f\"allt die neue Einteilung unter Fall 1), da es eine
positive Zahl in der neuen ersten Klasse gibt.  Nach I) existiert
also eine Zahl $\Xi_1$, so da{\ss} jedes
$$\Eta < \Xi_1$$
in der neuen ersten Klasse, jedes
$$\Eta > \Xi_1$$
in der neuen zweiten Klasse liegt.  Wird
$$-\Xi_1 = \Xi$$
gesetzt, so folgt aus
$$\Eta < \Xi \quad\hbox{\rm bzw.}\quad \Eta > \Xi,$$
da{\ss}
$$-\Eta > \Xi_1 \quad\hbox{\rm bzw.}\quad -\Eta < \Xi_1,$$
ist. Also liegt $-\Eta$ in der neuen zweiten bzw. neuen ersten Klasse,
also $\Eta$ in der alten ersten bzw. alten zweiten Klasse.
\vfill\eject


