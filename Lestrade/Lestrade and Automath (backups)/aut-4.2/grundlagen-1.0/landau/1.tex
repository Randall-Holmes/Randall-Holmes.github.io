%\jutting{1}{1-13}

\line{}\vskip 7\baselineskip
\centerline{\sl Kapitel 1.}
\medskip

\centerline{\bf Nat\"urliche Zahlen.}
\bigskip

\centerline{{\S}~1.}
\medskip

\centerline{\bf Axiome.}
\bigskip

Wir nehmen als gegeben an:

Eine Menge, d.~h.\ Gesamtheit, von Dingen, nat\"urliche Zahlen
genannt, mit den nachher aufzuz\"ahlenden Eigenschaften, Axiome
genannt.

Vor der Formulierung der Axiome sei einiges in bezug auf
die benutzten Zeichen $=$ und $\ne$ vorangeschickt.

Kleine lateinische Buchstaben bedeuten in diesem Buch, wenn
nichts anderes gesagt wird, durchweg nat\"urliche Zahlen.

Ist $x$ gegeben und $y$ gegeben, so sind

entweder $x$ und $y$ dieselbe Zahl; das kann man auch
$$x = y$$
schreiben ($=$ sprich: gleich);

oder $x$ und $y$ nicht dieselbe Zahl; das kann man auch
$$x \ne y$$
schreiben ($\ne$ sprich: ungleich).

Hiernach gilt aus rein logischen Gr\"unden:

1)
$$x = x$$
f\"ur jedes x.

2) Aus
$$x = y$$
folgt
$$y = x.$$

3) Aus
$$x = y,\quad y = z$$
folgt
$$x = z.$$

Eine Schreibweise wie
$$a = b = c = d,$$
mit der zun\"achst nur
$$a = b,\quad b = c,\quad c = d$$
gemeint ist, enth\"alt also \"uberdies z.~B.\ %
$$a = c,\quad a = d,\quad b = d.$$
(Entsprechend in den sp\"ateren Kapiteln.)

Von der Menge der nat\"urlichen Zahlen nehmen wir nun an,
da{\ss} sie die Eigenschaften hat:
\medskip

{\bf Axiom 1:} {\it $1$ ist eine nat\"urliche Zahl.}

D.~h.\ unsere Menge ist nicht leer; sie enth\"alt ein Ding, das $1$
(sprich: Eins) hei{\ss}t.
\medskip

%\jutting{1}{14-16}

{\bf Axiom 2:} {\it Zu jedem $x$ gibt es genau eine nat\"urliche Zahl, die
der Nachfolger von $x$ hei{\ss}t und mit $x'$ bezeichnet uerden m\"oge.}

Bei komplizierten $x$ wird die Zahl, um deren Nachfolger es
sich handelt, eingeklammert, wenn sonst ein Mi{\ss}verst\"andnis zu be%
f\"urchten ist.  Entsprechendes gilt im ganzen Buch bei $x + y$, $xy$,
$x - y$, $- x$, $x^y$ u.~dgl.

Aus
$$x = y$$
folgt also
$$x' = y'.$$
\medskip

%\jutting{1}{17-17}

{\bf Axiom 3:} {\it Stets ist
$$x' \ne 1.$$}%

D.~h.\ es gibt keine Zahl mit dem Nachfolger $1$.
\medskip

%\jutting{1}{18-18}

{\bf Axiom 4:} {\it Aus
$$x' = y'$$
folgt
$$x = y.$$}%

D.~h.\ zu jeder Zahl gibt es keine oder genau eine, deren Nach%
folger jene Zahl ist.
\medskip

%\jutting{1}{19-32}

\ifx\fr\undefined
  \font\teneufm=eufm10 \font\seveneufm=eufm7 \font\fiveeufm=eufm5
  \csname newfam\endcsname\eufmfam
  \textfont\eufmfam=\teneufm \scriptfont\eufmfam=\seveneufm \scriptscriptfont\eufmfam=\fiveeufm
  \def\fr{\fam\eufmfam}
\fi

{\bf Axiom 5} (Induktionsaxiom): {\it Es sei $\fr M$ eine Menge nat\"ulicher
Zahlen mit den Eigenschaften:

{\rm I)} $1$ geh\"ort zu $\fr M$.

{\rm II)} Wenn $x$ zu $\fr M$ geh\"ort, so geh\"ort $x'$ zu $\fr M$.

Dann umfa{\ss}t $\fr M$ alle nat\"urlichen Zahlen.}
\vfill\eject


%\jutting{1}{33-38}

\line{}\vskip 7\baselineskip
\centerline{{\S}~2.}
\medskip

\centerline{\bf Addition.}
\bigskip

{\bf Satz 1:} {\it Aus
$$x \ne y$$
folgt
$$x' \ne y'.$$}%

{\bf Beweis:} Sonst w\"are
$$x' = y',$$
also nach Axiom 4
$$x = y.$$
\medskip

%\jutting{1}{39-45}

{\bf Satz 2:} {\it $$x' \ne x.$$}%

{\bf Beweis:} $\fr M$ sei die Menge der $x$, f\"ur die dies gilt.

I) Nach Axiom 1 und Axiom 3 ist
$$ 1' \ne 1;$$
also geh\"ort $1$ zu $\fr M$.

II) Ist $x$ zu $\fr M$ geh\"orig, so ist
$$x' \ne x,$$
also nach Satz 1
$$(x')' \ne x',$$
also $x'$ zu $\fr M$ geh\"orig.

Nach Axiom 5 umfa{\ss}t also $\fr M$ alle nat\"urlichen Zahlen; d.~h.\ %
f\"ur jedes $x$ ist
$$x' \ne x.$$
\medskip

%\jutting{1}{46-60}

{\bf Satz 3:} {\it Ist
$$x \ne 1,$$
so gibt es ein {\rm (also nach Axiom 4 genau ein)} $u$ mit
$$x = u'.$$}%

{\bf Beweis:} $\fr M$ sei die Menge, die aus der Zahl $1$ und denjenigen
$x$ besteht, zu denen es ein solches $u$ gibt.  (Von selbst ist jedes
derartige
$$x \ne 1$$
nach Axiom 3.)

I) $1$ geh\"ort zu $\fr M$.

II) Ist $x$ zu $\fr M$ geh\"orig, so ist, wenn unter $u$ die Zahl $x$ ver%
standen wird,
$$x' = u',$$
also $x'$ zu $\fr M$ geh\"orig.

Nach Axiom 5 umfa{\ss}t also $\fr M$ alle nat\"urlichen Zahlen; zu
jedem
$$x \ne 1$$
gibt es also ein $u$ mit
$$x = u'.$$
\medskip

%\jutting{1}{61-127}

{\bf Satz 4,} zugleich {\bf Definition 1:} {\it Auf genau eine Art l\"a{\ss}t sich
jedem Zahlenpaar $x$, $y$ eine nat\"urliche Zahl, $x + y$ genannt {\rm ($+$ sprich:
plus),} so zuordnen, da{\ss}

{\rm 1)} $$x + 1 = x'\quad\hbox{f\"ur jedes $x$,}$$

{\rm 2)} $$x + y' = (x + y)'\quad\hbox{f\"ur jedes $x$ und jedes $y$.}$$

$x + y$ hei{\ss}t die Summe von $x$ und $y$ oder die durch Addition von
$y$ zu $x$ entstehende Zahl.}

{\bf Beweis:} A) Zun\"achst zeigen wir, da{\ss} es bei jedem festen $x$
h\"ochstens eine M\"oglichkeit gibt, $x + y$ f\"ur alle $y$ so zu definieren, da{\ss}
$$x + 1 = x'$$
und
$$x + y' = (x + y)'\quad\hbox{f\"ur jedes $y$}.$$

Es seien $a_y$ und $b_y$ f\"ur alle $y$ definiert und so beschaffen, da{\ss}
$$a_1 = x',\quad b_1 = x',$$
$$a_{y'} = (a_y)',\quad b_{y'} = (b_y)'\quad\hbox{f\"ur jedes $y$}.$$
$\fr M$ sei die Menge der $y$ mit
$$a_y = b_y.$$

I) $$a_1 = x' = b_1;$$
$1$ geh\"ort also zu $\fr M$.

II) Ist $y$ zu $\fr M$ geh\"orig, so ist
$$a_y = b_y,$$
also nach Axiom 2
$$(a_y)' = (b_y)',$$
also
$$a_{y'} = (a_y)' = (b_y)' = b_{y'},$$
also $y'$ zu $\fr M$ geh\"orig.

Daher ist $\fr M$ die Menge aller nat\"urlichen Zahlen; d.~h.\ f\"ur
jedes $y$ ist
$$a_y = b_y.$$

B) Wir zeigen jetzt, da{\ss} es zu jedem $x$ eine M\"oglichkeit gibt,
$x + y$ f\"ur alle $y$ so zu definieren, da{\ss}
$$x + 1 = x'$$
und
$$x + y' = (x + y)'\quad\hbox{f\"ur jedes $y$}.$$

$\fr M$ sei die Menge der $x$, zu denen es eine (also nach A) genau
eine) solche M\"oglichkeit gibt.

I) F\"ur
$$x = 1$$
leistet
$$x + y = y'$$
das Gew\"unschte.  Denn
$$x + 1 = 1' = x',$$
$$x + y' = (y')'= (x + y)'.$$
Also geh\"ort $1$ zu $\fr M$.

II) Es sei $x$ zu $\fr M$ geh\"orig, also ein $x + y$ f\"ur alle $y$ vor%
handen.  Dann leistet
$$x' + y = (x + y)'$$
das Gew\"unschte bei $x'$.  Denn
$$x' + 1 = (x + 1)' = (x')'$$
und
$$x' + y' = (x + y')' = \bigl((x + y)'\bigr)' = (x' + y)'.$$
Also geh\"ort $x'$ zu $\fr M$.

Daher umfa{\ss}t $\fr M$ alle $x$.
\medskip

%\jutting{1}{128-137}

{\bf Satz 5} (assoziatives Gesetz der Addition):
{\it $$(x + y) + z = x + (y + z).$$}%

{\bf Beweis:} $x$ und $y$ seien fest, $\fr M$ die Menge der $z$, f\"ur die die
Behauptung gilt.

I) $$(x + y) + 1 = (x + y)' = x + y' = x + (y + 1)$$
also geh\"ort $1$ zu $\fr M$.

II) $z$ geh\"ore zu $\fr M$.  Dann ist
$$(x + y) + z = x + (y + z),$$
also
$$(x + y) + z' = \bigl((x + y) + z\bigr)' = \bigl(x + (y + z)\bigr)' = x + (y + z)' = x + (y + z'),$$
also $z'$ zu $\fr M$ geh\"orig.

Die Behauptung gilt also f\"ur alle $z$.
\medskip

%\jutting{1}{138-155}

{\bf Satz 6} (kommutatives Gesetz der Addition):
{\it $$x + y = y + x.$$}%

{\bf Beweis:} $y$ sei fest, $\fr M$ die Menge der $x$, f\"ur die die Be%
hauptung gilt.

I) Es ist
$$y + 1 = y'$$
und nach der Konstruktion beim Beweise des Satzes 4
$$1 + y = y',$$
also
$$1 + y = y + 1,$$
$1$ zu $\fr M$ geh\"orig.

II) Ist $x$ zu $\fr M$ geh\"orig, so ist
$$x + y = y + x,$$
also
$$(x + y)' = (y + x)' = y + x'.$$
Nach der Konstruktion beim Beweise des Satzes 4 ist
$$x' + y = (x + y)',$$
also
$$x' + y = y + x'$$
also $x'$ zu $\fr M$ geh\"orig.

Die Behauptung gilt also f\"ur alle $x$.
\medskip

%\jutting{1}{156-164}

{\bf Satz 7:} {\it $$y \ne x + y.$$}%

{\bf Beweis:} $x$ sei fest, $\fr M$ die Menge der $y$, f\"ur die die Be%
hauptung gilt.

I) $$1 \ne x',$$
$$1 \ne x + 1;$$
$1$ geh\"ort zu $\fr M$.

II) Ist $y$ zu $\fr M$ geh\"orig, so ist
$$y \ne x + y,$$
also
$$y' \ne (x + y)',$$
$$y' \ne x + y',$$
$y'$ zu $\fr M$ geh\"orig.

Die Behauptung gilt also f\"ur alle $y$.
\medskip

%\jutting{1}{165-182}

{\bf Satz 8:} {\it Aus
$$y \ne z$$
folgt
$$x + y \ne x + z.$$}%

{\bf Beweis:} Bei festen $y$, $z$ mit
$$y \ne z$$
sei $\fr M$ die Menge der $x$ mit
$$x + y \ne x + z.$$

I) $$y' \ne z',$$
$$1 + y \ne 1 + z;$$
$1$ geh\"ort also zu $\fr M$.

II) Ist $x$ zu $\fr M$ geh\"orig, so ist
$$x + y \ne x + z$$
also
$$(x + y)' \ne (x + z)',$$
$$x' + y \ne x' + z,$$
$x'$ zu $\fr M$ geh\"orig.

Also gilt die Behauptung stets.
\medskip

%\jutting{1}{183-237}

{\bf Satz 9:} {\it Sind $x$ und $y$ gegeben, so liegt genau einer der F\"alle vor:

1) $$x = y.$$

2) Es gibt ein {\rm (also nach Satz 8 genau ein)} $u$ mit
$$x = y + u.$$

3) Es gibt ein {\rm (also nach Satz 8 genau ein)} $v$ mit
$$y = x + v.$$}%

{\bf Beweis:} A) Nach Satz 7 sind 1), 2) unvertr\"aglich und 1), 3)
unvertr\"aglich.  Aus Satz 7 folgt auch die Unvertr\"aglichkeit von
2), 3); denn sonst w\"are
$$x = y + u = (x + v) + u = x + (v + u) = (v + u) + x.$$

Also liegt h\"ochstens einer der F\"alle 1), 2), 3) vor.

B) $x$ sei fest, $\fr M$ die Menge der $y$, f\"ur die einer (also nach A)
genau einer) der F\"alle 1), 2), 3) vorliegt.

I) F\"ur $y = 1$ ist nach Satz 3 entweder
$$x = 1 = y\quad\hbox{(Fall 1))}$$
oder
$$x = u' = 1 + u = y + u\quad\hbox{(Fall 2)).}$$
Daher geh\"ort $1$ zu $\fr M$.

II) Es geh\"ore $y$ zu $\fr M$.  Dann ist

entweder (Fall 1) bei $y$)
$$x = y,$$
also
$$y' = y + 1 = x + 1\quad\hbox{(Fall 3) f\"ur $y'$)};$$

oder (Fall 2) bei $y$)
$$x = y + u,$$
also, wenn
$$u = 1,$$
$$x = y + 1 = y'\quad\hbox{(Fall 1) f\"ur $y'$)};$$
wenn
$$u \ne 1,$$
nach Satz 3
$$u = w' = 1 + w,$$
$$x = y + (1 + w) = (y + 1) + w = y' + w\quad\hbox{(Fall 2) f\"ur $y'$)};$$
oder (Fall 3) bei $y$)
$$y = x + v,$$
also
$$y' = (x + v)' = x + v'\quad\hbox{(Fall 3) f\"ur $y'$)}.$$

Jedenfalls geh\"ort also $y'$ zu $\fr M$.

Daher liegt stets einer der F\"alle 1), 2), 3) vor.
\vfill\eject


%\jutting{1}{238-238}

\line{}\vskip 7\baselineskip
\centerline{{\S}~3.}
\medskip

\centerline{\bf Ordnung.}
\bigskip

{\bf Definition 2:} {\it Ist
$$x = y + u,$$
so ist
$$x > y.$$
{\rm ($>$ sprich: gr\"o{\ss}er als.)}}
\medskip

%\jutting{1}{239-239}

{\bf Definition 3:} {\it Ist
$$y = x + v,$$
so ist
$$x < y.$$
{\rm ($<$ sprich: kleiner als.)}}
\medskip

%\jutting{1}{240-242}

{\bf Satz 10:} {\it Sind $x$, $y$ beliebig, so liegt genau einer der F\"alle
$$x = y,\quad x > y,\quad x < y$$
vor.}

{\bf Beweis:} Satz 9, Definition 2 und Definition 3.
\medskip

%\jutting{1}{243-244}

{\bf Satz 11:} {\it Aus
$$x > y$$
folgt
$$y < x.$$}%

{\bf Beweis:} Beides besagt
$$x = y + u$$
bei passendem $u$.
\medskip

%\jutting{1}{245-246}

{\bf Satz 12:} {\it Aus
$$x < y$$
folgt
$$y > x.$$}%

{\bf Beweis:} Beides besagt
$$y = x + v$$
bei passendem $v$.
\medskip

%\jutting{1}{247-247}

{\bf Definition 4:} {\it $$x \ge y$$
bedeutet
$$x > y\quad\hbox{oder}\quad x = y.$$
{\rm ($\ge$ sprich: gr\"o{\ss}er oder gleich.)}}
\medskip

%\jutting{1}{248-248}

{\bf Definition 5:} {\it $$x \le y$$
bedeutet
$$x < y\quad\hbox{oder}\quad x = y.$$
{\rm ($\le$ sprich: kleiner oder gleich.)}}
\medskip

%\jutting{1}{249-250}

{\bf Satz 13:} {\it Aus
$$x \ge y$$
folgt
$$y \le x.$$}%

{\bf Beweis:} Satz 11.
\medskip

%\jutting{1}{251-299}

{\bf Satz 14:} {\it Aus
$$x \le y$$
folgt
$$y \ge x.$$}%

{\bf Beweis:} Satz 12.
\medskip

%\jutting{1}{300-313}

{\bf Satz 15} (Transitivit\"at der Ordnung): {\it Aus
$$x < y,\quad y < z$$
folgt
$$x < z.$$}%

{\bf Vorbemerkung:} Aus
$$x > y,\quad y > z$$
folgt also (wegen
$$z < y,\quad y < x,$$
$$z < x)$$
$$x > z;$$
aber solche trivialerweise durch R\"uckw\"artslesen entstehenden Wort%
laute schreibe ich in der Folge nicht erst auf.

{\bf Beweis:} Bei passenden $v$, $w$ ist
$$y = x + v,\quad z = y + w,$$
also
$$z = (x + v) + w = x + (v + w),$$
$$x < z.$$
\medskip

%\jutting{1}{314-321}

{\bf Satz 16:} {\it Aus
$$x \le y,\quad y < z\quad\hbox{oder}\quad x < y,\quad y \le z$$
folgt
$$x < z.$$}%

{\bf Beweis:} Mit dem Gleichheitszeichen in der Voraussetzung klar;
sonst durch Satz 15 erledigt.
\medskip

%\jutting{1}{322-342}

{\bf Satz 17:} {\it Aus
$$x \le y,\quad y \le z$$
folgt
$$x \le z.$$}%

{\bf Beweis:} Mit zwei Gleichheitszeichen in der Voraussetzung
klar; sonst durch Satz 16 erledigt.

Nach den S\"atzen 15 bis 17 ist eine Schreibweise wie
$$a < b \le c < d$$
gerechtfertigt; das hei{\ss}t zun\"achst
$$a < b,\quad b \le c,\quad c < d,$$
enth\"alt aber nach jenen S\"atzen auch z.~B.\ %
$$a < c,\quad a < d,\quad b < d.$$
(Entsprechend in den sp\"ateren Kapiteln.)
\medskip

%\jutting{1}{343-346}

{\bf Satz 18:} {\it $$x + y > x.$$}%

{\bf Beweis:} $$x + y = x + y.$$
\medskip

%\jutting{1}{347-385}

{\bf Satz 19:} {\it Aus
$$x > y\quad\hbox{\it bzw.}\quad x = y\quad\hbox{\it bzw.}\quad x < y$$
folgt
$$x + z > y + z\quad\hbox{\it bzw.}\quad x + z = y + z\quad\hbox{\it bzw.}\quad x + z < y + z.$$}%

{\bf Beweis:} 1) Aus
$$x > y$$
folgt
$$x = y + u,$$
$$x + z = (y + u) + z = (u + y) + z = u + (y + z) = (y + z) + u,$$
$$x + z > y + z.$$

2) Aus
$$x = y$$
folgt nat\"urlich
$$x + z = y + z.$$

3) Aus
$$x < y$$
folgt
$$y > x,$$
also nach 1)
$$y + z > x + z,$$
$$x + z < y + z.$$
\medskip

%\jutting{1}{386-406}

{\bf Satz 20:} {\it Aus
$$x + z > y + z\quad\hbox{\it bzw.}\quad x + z = y + z\quad\hbox{\it bzw.}\quad x + z < y + z$$
folgt
$$x > y\quad\hbox{\it bzw.}\quad x = y\quad\hbox{\it bzw.}\quad x < y.$$}%

{\bf Beweis:} Folgt aus Satz 19, da die drei F\"alle beide Male sich
ausschlie{\ss}en und alle M\"oglichkeiten ersch\"opfen.
\medskip

%\jutting{1}{407-420}

{\bf Satz 21:} {\it Aus
$$x > y,\quad z > u$$
folgt
$$x + z > y + u.$$}%

{\bf Beweis:} Nach Satz 19 ist
$$x + z > y + z$$
und
$$y + z = z + y > u + y = y + u,$$
also
$$x + z > y + u.$$
\medskip

%\jutting{1}{421-428}

{\bf Satz 22:} {\it Aus
$$x \ge y,\quad z > u\quad oder\quad x > y,\quad z \ge u$$
folgt
$$x + z > y + u.$$}%

{\bf Beweis:} Mit dem Gleichheitszeichen in der Voraussetzung
durch Satz 19, sonst durch Satz 21 erledigt.
\medskip

%\jutting{1}{429-448}

{\bf Satz 23:} {\it Aus
$$x \ge y,\quad z \ge u$$
folgt
$$x + z \ge y + u.$$}%

{\bf Beweis:} Mit zwei Gleichheitszeichen in der Voraussetzung
klar; sonst durch Satz 22 erledigt.
\medskip

%\jutting{1}{449-458}

{\bf Satz 24:} {\it $$x \ge 1.$$}%

{\bf Beweis:} Entweder ist
$$x = 1$$
oder
$$x = u' = u + 1 > 1.$$
\medskip

%\jutting{1}{459-469}

{\bf Satz 25:} {\it Aus
$$y > x$$
folgt
$$y \ge x + 1$$}%

{\bf Beweis:} $$y = x + u,$$
$$u \ge 1,$$
also
$$y \ge x + 1.$$
\medskip

%\jutting{1}{470-482}

{\bf Satz 26:} {\it Aus
$$y < x + 1$$
folgt
$$y \le x.$$}%

{\bf Beweis:} Sonst w\"are
$$y > x,$$
also nach Satz 25
$$y \ge x + 1.$$
\medskip

%\jutting{1}{483-557}

{\bf Satz 27:} {\it In jeder nicht leeren Menge nat\"urlicher Zahlen gibt es
eine kleinste} (d.~h.\ eine, die kleiner ist als jede etwaige andere).

{\bf Beweis:} $\fr N$ sei die gegebene Menge.  $\fr M$ sei die Menge der $x$,
die $\le$ jeder Zahl aus $\fr N$ sind.

1 geh\"ort zu $\fr M$ nach Satz 24.  Nicht jedes $x$ geh\"ort zu $\fr M$;
denn f\"ur jedes $y$ aus $\fr N$ geh\"ort $y + 1$ nicht zu $\fr M$, wegen
$$y + 1 > y.$$

Also gibt es in $\fr M$ ein $m$, so da{\ss} $m + 1$ nicht zu $\fr M$ geh\"ort;
denn sonst m\"u{\ss}te nach Axiom 5 jede nat\"urliche Zahl zu $\fr M$ ge%
h\"oren.

Von jenem $m$ behaupte ich, da{\ss} es $\le$ jedem $n$ aus $\fr N$ ist und
zu $\fr N$ geh\"ort.  Ersteres steht schon fest.  Letzteres folgt indirekt
so: W\"are $m$ nicht zu $\fr N$ geh\"orig, so w\"are f\"ur jedes $n$ aus $\fr N$
$$m < n,$$
also nach Satz 25
$$m + 1 \le n;$$
$m + 1$ w\"urde also zu $\fr M$ geh\"oren, gegen das Obige.
\vfill\eject


%\jutting{1}{558-625}

\line{}\vskip 7\baselineskip
\centerline{{\S}~4.}
\medskip

\centerline{\bf Multiplikation.}
\bigskip

{\bf Satz 28,} zugleich {\bf Definition 6:} {\it Auf genau eine Art l\"a{\ss}t sich
jedem Zahlenpaar $x$, $y$ eine nat\"urliche Zahl, $x \cdot y$ genannt {\rm ($\cdot$ sprich:
mal; aber man schreibt den Punkt meist nicht),} so zuordnen, da{\ss}

{\rm 1)} $$x \cdot 1 = x\quad\hbox{f\"ur jedes $x$},$$

{\rm 2)} $$x \cdot y' = x \cdot y + x\quad\hbox{f\"ur jedes $x$ und jedes $y$}.$$

$x \cdot y$ hei{\ss}t das Produkt von $x$ mit $y$ oder die durch Multiplikation
von $x$ mit $y$ entstehende Zahl.}

{\bf Beweis} (mutatis mutandis w\"ortlich mit dem des Satzes 4 \"uber%
einstimmend): A) Zun\"achst zeigen wir, da{\ss} es bei jedem festen $x$
h\"ochstens eine M\"oglichkeit gibt, $xy$ f\"ur alle $y$ so zu definieren, da{\ss}
$$x \cdot 1 = x$$
und
$$xy' = xy + x\quad\hbox{f\"ur jedes $y$}.$$

Es seien $a_y$ und $b_y$ f\"ur alle $y$ definiert und so beschaffen, da{\ss}
$$a_1 = x,\quad b_1 = x,$$
$$a_{y'} = a_y + x,\quad b_{y'} = b_y + x\quad\hbox{f\"ur jedes $y$}.$$
$\fr M$ sei die Menge der $y$ mit
$$a_y = b_y.$$

I) $$a_1 = x = b_1;$$
$1$ geh\"ort also zu $\fr M$.

II) Ist $y$ zu $\fr M$ geh\"orig, so ist
$$a_y = b_y,$$
also
$$a_{y'} = a_y + x = b_y + x = b_{y'},$$
also $y'$ zu $\fr M$ geh\"orig.

Daher ist $\fr M$ die Menge aller nat\"urlichen Zahlen; d.~h.\ f\"ur
jedes $y$ ist
$$a_y = b_y.$$

B) Wir zeigen jetzt, da{\ss} es zu jedem $x$ eine M\"oglichkeit gibt,
$xy$ f\"ur alle $y$ so zu definieren, da{\ss}
$$x \cdot 1 = x$$
und
$$xy' = xy + x\quad\hbox{f\"ur jedes $y$}.$$

$\fr M$ sei die Menge der $x$, zu denen es eine (also nach A) genau
eine) solche M\"oglichkeit gibt.

I) F\"ur
$$x = 1$$
leistet
$$xy = y$$
das Gew\"unschte.  Denn
$$x \cdot 1 = 1 = x,$$
$$xy' = y' = y + 1 = xy + x.$$
Also geh\"ort $1$ zu $\fr M$.

II) Es sei $x$ zu $\fr M$ geh\"orig, also ein $xy$ f\"ur alle $y$ vorhanden.
Dann leistet
$$x'y = xy + y$$
das Gew\"unschte bei $x'$.  Denn
$$x' \cdot 1 = x \cdot 1 + 1 = x + 1 = x'$$
und
$$\displaylines{x'y' = xy' + y' = (xy + x) + y' = xy + (x + y') = xy + (x + y)'\cr
= xy + (x' + y) = xy + (y + x') = (xy + y) + x' = x'y + x'.\cr}$$
Also geh\"ort $x'$ zu $\fr M$.

Daher umfa{\ss}t $\fr M$ alle $x$.
\medskip

%\jutting{1}{626-643}

{\bf Satz 29} (kommutatives Gesetz der Multiplikation):
{\it $$xy = yx.$$}%

{\bf Beweis:} $y$ sei fest, $\fr M$ die Menge der $x$, f\"ur die die Behaup%
tuug gilt.

I) Es ist
$$y \cdot 1 = y$$
und nach der Konstruktion beim Beweise des Satzes 28
$$1 \cdot y = y,$$
also
$$1 \cdot y = y \cdot 1,$$
$1$ zu $\fr M$ geh\"orig.

II) Ist $x$ zu $\fr M$ geh\"orig, so ist
$$xy = yx,$$
also
$$xy + y = yx + y = yx'.$$
Nach der Konstruktion beim Beweise des Satzes 28 ist
$$x'y = xy + y,$$
also
$$x'y = yx'$$
also $x'$ zu $\fr M$ geh\"orig.

Die Behauptung gilt also f\"ur alle $x$.
\medskip

%\jutting{1}{644-655}

{\bf Satz 30} (distributives Gesetz):
{\it $$x (y + z) = xy + xz.$$}%

{\bf Vorbemerkung:} Die aus Satz 30 und Satz 29 flie{\ss}ende Formel
$$(y + z)x = yx + zx$$
und \"ahnliche Analoga sp\"aterhin brauchen nicht besonders als S\"atze
formuliert oder auch nur aufgeschrieben zu werden.

{\bf Beweis:} Bei festen $x$, $y$ sei $\fr M$ die Menge der $z$, f\"ur die die
Behauptung gilt.

I) $$x(y + 1) = xy' = xy + x = xy + x \cdot 1;$$
$1$ geh\"ort zu $\fr M$.

II) Wenn $z$ zu $\fr M$ geh\"ort, ist
$$x(y + z) = xy + xz,$$
also
$$\displaylines{x(y + z') = x\bigl((y + z)'\bigr) = x(y + z) + x = (xy + xz) + x\cr
= xy + (xz + x) = xy + xz',\cr}$$
also $z'$ zu $\fr M$ geh\"orig.

Daher gilt die Behauptung stets.
\medskip

%\jutting{1}{656-664}

{\bf Satz 31} (assoziatives Gesetz der Multiplikation):
{\it $$(xy)z = x(yz).$$}%

{\bf Beweis:} $x$ und $y$ seien fest, $\fr M$ die Menge der $z$, f\"ur die die
Behauptung gilt.

I) $$(xy) \cdot 1 = xy = x(y \cdot 1);$$
also geh\"ort $1$ zu $\fr M$.

II) $z$ geh\"ore zu $\fr M$.  Dann ist
$$(xy)z = x(yz),$$
also unter Benutzung von Satz 30
$$(xy)z' = (xy)z + xy = x(yz) + xy = x(yz + y) = x(yz'),$$
also $z'$ zu $\fr M$ geh\"orig.

$\fr M$ umfa{\ss}t also alle nat\"urlichen Zahlen.
\medskip

%\jutting{1}{665-702}

{\bf Satz 32:} {\it Aus
$$x > y\quad\hbox{\it bzw.}\quad x = y\quad\hbox{\it bzw.}\quad x < y$$
folgt
$$xz > yz\quad\hbox{\it bzw.}\quad xz = yz\quad\hbox{\it bzw.}\quad xz < yz.$$}%

{\bf Beweis:} 1) Aus
$$x > y$$
folgt
$$x = y + u,$$
$$xz = (y + u)z = yz + uz > yz.$$

2) Aus
$$x = y$$
folgt nat\"urtlich
$$xz = yz.$$

3) Aus
$$x < y$$
folgt
$$y > x,$$
also nach 1)
$$yz > xz,$$
$$xz < yz.$$
\medskip

%\jutting{1}{703-715}

{\bf Satz 33:} {\it Aus
$$xz > yz\quad\hbox{\it bzw.}\quad xz = yz\quad\hbox{\it bzw.}\quad xz < yz$$
folgt
$$x > y\quad\hbox{\it bzw.}\quad x = y\quad\hbox{\it bzw.}\quad x < y.$$}%

{\bf Beweis:} Folgt aus Satz 32, da die drei F\"alle beide Male sich
ausschlie{\ss}en und alle M\"oglichkeiten ersch\"opfen.
\medskip

%\jutting{1}{716-729}

{\bf Satz 34:} {\it Aus
$$x > y,\quad z > u$$
folgt
$$xz > yu.$$}%

{\bf Beweis:} Nach Satz 32 ist
$$xz > yz$$
und
$$yz = zy > uy = yu,$$
also
$$xz > yu.$$
\medskip

%\jutting{1}{730-737}

{\bf Satz 35:} {\it Aus
$$x \ge y,\quad z > u\quad\hbox{oder}\quad x > y,\quad z \ge u$$
folgt
$$xz > yu.$$}%

{\bf Beweis:} Mit dem Gleichheitszeichen in der Voraussetzung
durch Satz 32, sonst durch Satz 34 erledigt.
\medskip

%\jutting{1}{738-901}

{\bf Satz 36:} {\it Aus
$$x \ge y,\quad z \ge u$$
folgt
$$xz \ge yu.$$}%

{\bf Beweis:} Mit zwei Gleichheitszeichen in der Voraussetzung
klar; sonst durch Satz 35 erledigt.
\vfill\eject


