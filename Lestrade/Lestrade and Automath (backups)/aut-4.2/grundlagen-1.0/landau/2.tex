%\jutting{2}{1-28}

\line{}\vskip 7\baselineskip
\centerline{\sl Kapitel 2.}
\medskip

\centerline{\bf Br\"uche.}
\bigskip

\centerline{{\S}~1.}
\medskip

\centerline{\bf Definition und \"Aquivalenz.}
\bigskip

{\bf Definition 7:} {\it Unter einem Bruch $x_1 \over x_2$ {\rm (sprich: $x_1$ \"uber $x_2$)} ver%
steht man das Paar der nat\"urlichen Zahlen $x_1, x_2$ {\rm (in dieser Reihen%
folge).}}
\medskip

%\jutting{2}{29-38}

{\bf Definition 8:} {\it $${x_1 \over x_2} \sim {y_1 \over y_2}$$
{\rm ($\sim$ sprich: \"aquivalent),} wenn
$$x_1 y_2 = y_1 x_2.$$}%
\medskip

%\jutting{2}{39-49}

{\bf Satz 37:} {\it $${x_1 \over x_2} \sim {x_1 \over x_2}.$$}%

{\bf Beweis:} $$x_1 x_2 = x_1 x_2.$$
\medskip

%\jutting{2}{50-61}

{\bf Satz 38:} {\it Aus
$${x_1 \over x_2} \sim {y_1 \over y_2}$$
folgt
$${y_1 \over y_2} \sim {x_1 \over x_2}.$$}%

{\bf Beweis:} $$x_1 y_2 = y_1 x_2,$$
also
$$y_1 x_2 = x_1 y_2.$$
\medskip

%\jutting{2}{62-84}

{\bf Satz 39:} {\it Aus
$${x_1 \over x_2} \sim {y_1 \over y_2},\quad {y_1 \over y_2} \sim {z_1 \over z_2}$$
folgt
$${x_1 \over x_2} \sim {z_1 \over z_2}.$$}%

{\bf Beweis:} $$x_1 y_2 = y_1 x_2,\quad y_1 z_2 = z_1 y_2,$$
also
$$(x_1 y_2)(y_1 z_2) = (y_1 x_2)(z_1 y_2).$$

Stets ist
$$\displaylines{(xy)(zu) = x\bigl(y(zu)\bigr) = x\bigl((yz)u\bigr) = x\bigl(u(yz)\bigr) = (xu)(yz)\cr
= (xu)(zy);\cr}$$
daher ist
$$(x_1 y_2)(y_1 z_2) = (x_1 z_2)(y_1 y_2)$$
und
$$(y_1 x_2)(z_1 y_2) = (y_1 y_2)(z_1 x_2) = (z_1 x_2)(y_1 y_2),$$
folglich nach dem Obigen
$$(x_1 z_2)(y_1 y_2) = (z_1 x_2)(y_1 y_2),$$
$$x_1 z_2 = z_1 x_2.$$
\bigskip

Auf Grund der S\"atze 37 bis 39 zerfallen alle Br\"uche in Klassen,
so da{\ss}
$${x_1 \over x_2} \sim {y_1 \over y_2}$$
dann und nur dann, wenn $x_1 \over x_2$ und $y_1 \over y_2$ derselben Klasse angeh\"oren.
\medskip

%\jutting{2}{85-90}

{\bf Satz 40:} {\it $${x_1 \over x_2} \sim {x_1 x \over x_2 x}.$$}%

{\bf Beweis:} $$x_1 (x_2 x) = x_1 (x x_2) = (x_1 x) x_2.$$
\vfill\eject


%\jutting{2}{91-91}

\line{}\vskip 7\baselineskip
\centerline{{\S}~2.}
\medskip

\centerline{\bf Ordnung.}
\bigskip

{\bf Definition 9:} {\it $${x_1 \over x_2} > {y_1 \over y_2}$$
{\rm ($>$ sprich: gr\"o{\ss}er als),} wenn
$$x_1 y_2 > y_1 x_2.$$}%
\medskip

%\jutting{2}{92-104}

{\bf Definition 10:} {\it $${x_1 \over x_2} < {y_1 \over y_2}$$
{\rm ($<$ sprich: kleiner als),} wenn
$$x_1 y_2 < y_1 x_2.$$}%
\medskip

%\jutting{2}{105-107}

{\bf Satz 41}: {\it Sind $x_1 \over x_2$, $y_1 \over y_2$ beliebig, so liegt genau einer der F\"alle
$${x_1 \over x_2} \sim {y_1 \over y_2},\quad {x_1 \over x_2} > {y_1 \over y_2},\quad {x_1 \over x_2} < {y_1 \over y_2}$$
vor.}

{\bf Beweis:} Es liegt f\"ur $x_1$, $x_2$, $y_1$, $y_2$, genau einer der F\"alle
$$x_1 y_2 = y_1 x_2,\quad x_1 y_2 > y_1 x_2,\quad x_1 y_2 < y_1 x_2$$
vor.
\medskip

%\jutting{2}{108-109}

{\bf Satz 42:} {\it Aus
$${x_1 \over x_2} > {y_1 \over y_2}$$
folgt
$${y_1 \over y_2} < {x_1 \over x_2}.$$}%

{\bf Beweis:} Aus
$$x_1 y_2 > y_1 x_2$$
folgt
$$y_1 x_2 < x_1 y_2.$$
\medskip

%\jutting{2}{110-111}

{\bf Satz 43:} {\it Aus
$${x_1 \over x_2} < {y_1 \over y_2}$$
folgt
$${y_1 \over y_2} > {x_1 \over x_2}.$$}%

{\bf Beweis:} Aus
$$x_1 y_2 < y_1 x_2$$
folgt
$$y_1 x_2 > x_1 y_2.$$
\medskip

%\jutting{2}{112-126}

{\bf Satz 44:} {\it Aus
$${x_1 \over x_2} > {y_1 \over y_2},\quad {x_1 \over x_2} \sim {z_1 \over z_2},\quad {y_1 \over y_2} \sim {u_1 \over u_2}$$
folgt
$${z_1 \over z_2} > {u_1 \over u_2}.$$}%

{\bf Vorbemerkung:} Ist also ein Bruch einer Klasse gr\"o{\ss}er als
ein Bruch einer anderen Klasse, so gilt dies f\"ur alle Repr\"asen%
tantenpaare der beiden Klassen.

{\bf Beweis:} $$y_1 u_2 = u_1 y_2,\quad z_1 x_2 = x_1 z_2,\quad x_1 y_2 > y_1 x_2,$$
also
$$(y_1 u_2)(z_1 x_2) = (u_1 y_2)(x_1 z_2),$$
also nach Satz 32
$$(y_1 x_2)(z_1 u_2) = (u_1 z_2)(x_1 y_2) > (u_1 z_2)(y_1 x_2),$$
also nach Satz 33
$$z_1 u_2 > u_1 z_2.$$
\medskip

%\jutting{2}{127-134}

{\bf Satz 45:} {\it Aus
$${x_1 \over x_2} < {y_1 \over y_2},\quad {x_1 \over x_2} \sim {z_1 \over z_2},\quad {y_1 \over y_2} \sim {u_1 \over u_2}$$
folgt
$${z_1 \over z_2} < {u_1 \over u_2}.$$}%

{\bf Vorbemerkung:} Ist also ein Bruch einer Klasse kleiner als
ein Bruch einer anderen Klasse, so gilt dies f\"ur alle Repr\"asen%
tantenpaare der beiden Klassen.

{\bf Beweis:} Nach Satz 43 ist
$${y_1 \over y_2} > {x_1 \over x_2};$$
wegen
$${y_1 \over y_2} \sim {u_1 \over u_2},\quad {x_1 \over x_2} \sim {z_1 \over z_2}$$
ist also nach Satz 44
$${u_1 \over u_2} > {z_1 \over z_2},$$
also nach Satz 42
$${z_1 \over z_2} < {u_1 \over u_2}.$$

\ifx\gesim\undefined
  \def\gesim{\mathrel{\mathpalette\oversim >}}
  \def\lesim{\mathrel{\mathpalette\oversim <}}
  \def\oversim#1#2{\lower4pt\vbox{\baselineskip0pt \lineskip.5pt
    \ialign{$\mathsurround=0pt #1\hfil##\hfil$\crcr#2\crcr\sim\crcr}}}
\fi
\medskip

%\jutting{2}{135-135}

{\bf Definition 11:} {\it $${x_1 \over x_2} \gesim {y_1 \over y_2}$$
bedeutet
$${x_1 \over x_2} > {y_1 \over y_2}\quad\hbox{\it oder}\quad{x_1 \over x_2} \sim {y_1 \over y_2}.$$
{\rm ($\gesim$ sprich: gr\"o{\ss}er oder \"aquivalent.)}}
\medskip

%\jutting{2}{136-159}

{\bf Definition 12:} {\it $${x_1 \over x_2} \lesim {y_1 \over y_2}$$
bedeutet
$${x_1 \over x_2} < {y_1 \over y_2}\quad\hbox{\it oder}\quad{x_1 \over x_2} \sim {y_1 \over y_2}.$$
{\rm ($\lesim$ sprich: kleiner oder \"aquivalent.)}}
\medskip

%\jutting{2}{160-173}

{\bf Satz 46:} {\it Aus
$${x_1 \over x_2} \gesim {y_1 \over y_2},\quad {x_1 \over x_2} \sim {z_1 \over z_2},\quad {y_1 \over y_2} \sim {u_1 \over u_2}$$
folgt
$${z_1 \over z_2} \gesim {u_1 \over u_2}.$$}%

{\bf Beweis:} Mit $>$ in der Voraussetzung ist dies durch Satz 44
klar; anderenfalls ist
$${z_1 \over z_2} \sim {x_1 \over x_2} \sim {y_1 \over y_2} \sim {u_1 \over u_2}.$$
\medskip

%\jutting{2}{174-187}

{\bf Satz 47:} {\it Aus
$${x_1 \over x_2} \lesim {y_1 \over y_2},\quad {x_1 \over x_2} \sim {z_1 \over z_2},\quad {y_1 \over y_2} \sim {u_1 \over u_2}$$
folgt
$${z_1 \over z_2} \lesim {u_1 \over u_2}.$$}%

{\bf Beweis:} Mit $<$ in der Voraussetzung ist dies durch Satz 45
klar; anderenfalls ist
$${z_1 \over z_2} \sim {x_1 \over x_2} \sim {y_1 \over y_2} \sim {u_1 \over u_2}.$$
\medskip

%\jutting{2}{188-189}

{\bf Satz 48:} {\it Aus
$${x_1 \over x_2} \gesim {y_1 \over y_2}$$
folgt
$${y_1 \over y_2} \lesim {x_1 \over x_2}.$$}%

{\bf Beweis:} Satz 38 und Satz 42.
\medskip

%\jutting{2}{190-191}

{\bf Satz 49:} {\it Aus
$${x_1 \over x_2} \lesim {y_1 \over y_2}$$
folgt
$${y_1 \over y_2} \gesim {x_1 \over x_2}.$$}%

{\bf Beweis:} Satz 38 und Satz 43.
\medskip

%\jutting{2}{192-200}

{\bf Satz 50} (Transitivit\"at der Ordnung): {\it Aus
$${x_1 \over x_2} < {y_1 \over y_2},\quad {y_1 \over y_2} < {z_1 \over z_2}$$
folgt
$${x_1 \over x_2} < {z_1 \over z_2}.$$}%

{\bf Beweis:} $$x_1 y_2 < y_1 x_2,\quad y_1 z_2 < z_1 y_2,$$
also
$$(x_1 y_2)(y_1 z_2) < (y_1 x_2)(z_1 y_2),$$
$$(x_1 z_2)(y_1 y_2) < (z_1 x_2)(y_1 y_2),$$
$$x_1 z_2 < z_1 x_2.$$
\medskip

%\jutting{2}{201-208}

{\bf Satz 51} {\it Aus
$${x_1 \over x_2} \lesim {y_1 \over y_2},\quad {y_1 \over y_2} < {z_1 \over z_2}\quad{\it oder}\quad{x_1 \over x_2} < {y_1 \over y_2},\quad {y_1 \over y_2} \lesim {z_1 \over z_2}$$
folgt
$${x_1 \over x_2} < {z_1 \over z_2}.$$}%

{\bf Beweis:} Mit dem \"Aquivalenzzeichen in der Voraussetzung
durch Satz 45, sonst durch Satz 50 erledigt.
\medskip

%\jutting{2}{209-229}

{\bf Satz 52:} {\it Aus
$${x_1 \over x_2} \lesim {y_1 \over y_2},\quad {y_1 \over y_2} \lesim {z_1 \over z_2}$$
folgt
$${x_1 \over x_2} \lesim {z_1 \over z_2}.$$}%

{\bf Beweis:} Mit zwei \"Aquivalenzzeichen in der Voraussetzung
durch Satz 39, sonst durch Satz 51 erledigt.
\medskip

%\jutting{2}{230-234}

{\bf Satz 53:} {\it Zu $x_1 \over x_2$ gibt es ein
$${z_1 \over z_2} > {x_1 \over x_2}.$$}%

{\bf Beweis:} $$(x_1 + x_1) x_2 = x_1 x_2 + x_1 x_2 > x_1 x_2,$$
$${x_1 + x_1 \over x_2} > {x_1 \over x_2}.$$
\medskip

%\jutting{2}{235-239}

{\bf Satz 54:} {\it Zu $x_1 \over x_2$ gibt es ein
$${z_1 \over z_2} < {x_1 \over x_2}.$$}%

{\bf Beweis:} $$x_1 x_2 < x_1 x_2 + x_1 x_2 = x_1 (x_2 + x_2),$$
$${x_1 \over x_2 + x_2} < {x_1 \over x_2}.$$
\medskip

%\jutting{2}{240-250}

{\bf Satz 55:} {\it Ist
$${x_1 \over x_2} < {y_1 \over y_2},$$
so gibt es ein $z_1 \over z_2$ mit
$${x_1 \over x_2} < {z_1 \over z_2} < {y_1 \over y_2}.$$}%

{\bf Beweis:} $$x_1 y_2 < y_1 x_2,$$
also
$$x_1 x_2 + x_1 y_2 < x_1 x_2 + y_1 x_2,\quad x_1 y_2 + y_1 y_2 < y_1 x_2 + y_1 y_2,$$
$$x_1 (x_2 + y_2) < (x_1 + y_1) x_2,\quad (x_1 + y_1) y_2 < y_1 (x_2 + y_2),$$
$${x_1 \over x_2} < {x_1 + y_1 \over x_2 + y_2} < {y_1 \over y_2}.$$
\vfill\eject


%\jutting{2}{251-272}

\line{}\vskip 7\baselineskip
\centerline{{\S}~3.}
\medskip

\centerline{\bf Addition.}
\bigskip

{\bf Definition 13:} {\it Unter ${x_1 \over x_2} + {y_1 \over y_2}$ {\rm ($+$ sprich: plus)} versteht man den
Bruch $x_1 y_2 + y_1 x_2 \over x_2 y_2$.

Er hei{\ss}t die Summe von $x_1 \over x_2$ und $y_1 \over y_2$ oder der durch Addition
von $y_1 \over y_2$ zu $x_1 \over x_2$ entstehende Bruch.}
\medskip

%\jutting{2}{273-287}

{\bf Satz 56:} {\it Aus
$${x_1 \over x_2} \sim {y_1 \over y_2},\quad {z_1 \over z_2} \sim {u_1 \over u_2}$$
folgt
$${x_1 \over x_2} + {z_1 \over z_2} \sim {y_1 \over y_2} + {u_1 \over u_2}.$$}%

{\bf Vorbemerkung:} Die Klasse der Summe h\"angt also nur von
den Klassen ab, zu denen die ``Summanden'' geh\"oren.

{\bf Beweis:} $$x_1 y_2 = y_1 x_2,\quad z_1 u_2 = u_1 z_2,$$
also
$$(x_1 y_2)(z_2 u_2) = (y_1 x_2)(z_2 u_2),\quad (z_1 u_2)(x_2 y_2) = (u_1 z_2)(x_2 y_2),$$
also
$$(x_1 z_2)(y_2 u_2) = (y_1 u_2)(x_2 z_2),\quad (z_1 x_2)(y_2 u_2) = (u_1 y_2)(x_2 z_2),$$
$$(x_1 z_2)(y_2 u_2) + (z_1 x_2)(y_2 u_2) = (y_1 u_2)(x_2 z_2) + (u_1 y_2)(x_2 z_2),$$
$$(x_1 z_2 + z_1 x_2)(y_2 u_2) = (y_1 u_2 + u_1 y_2)(x_2 z_2),$$
$${x_1 z_2 + z_1 x_2 \over x_2 z_2} \sim {y_1 u_2 + u_1 y_2 \over y_2 u_2}.$$
\medskip

%\jutting{2}{288-290}

{\bf Satz 57:} {\it $${x_1 \over x} + {x_2 \over x} \sim {x_1 + x_2 \over x}.$$}%

{\bf Beweis:} Nach Definition 13 und Satz 40 ist
$${x_1 \over x} + {x_2 \over x} \sim {x_1 x + x_2 x \over x x} \sim {(x_1 + x_2) x \over x x} \sim {x_1 + x_2 \over x}.$$
\medskip

%\jutting{2}{291-292}

{\bf Satz 58:} (kommutatives Gesetz der Addition):
{\it $${x_1 \over x_2} + {y_1 \over y_2} \sim {y_1 \over y_2} + {x_1 \over x_2}.$$}%

{\bf Beweis:} $${x_1 \over x_2} + {y_1 \over y_2} \sim {x_1 y_2 + y_1 x_2 \over x_2 y_2} \sim {y_1 x_2 + x_1 y_2 \over y_2 x_2} \sim {y_1 \over y_2} + {x_1 \over x_2}.$$
\medskip

%\jutting{2}{293-310}

{\bf Satz 59:} (assoziatives Gesetz der Addition):
{\it $$\left({x_1 \over x_2} + {y_1 \over y_2}\right) + {z_1 \over z_2} \sim {x_1 \over x_2} + \left({y_1 \over y_2} + {z_1 \over z_2}\right).$$}%

{\bf Beweis:} $$\displaylines{\left({x_1 \over x_2} + {y_1 \over y_2}\right) + {z_1 \over z_2} \sim {x_1 y_2 + y_1 x_2 \over x_2 y_2} + {z_1 \over z_2}\cr
\sim {(x_1 y_2 + y_1 x_2) z_2 + z_1 (x_2 y_2) \over (x_2 y_2) z_2} \sim {\bigl((x_1 y_2) z_2 + (y_1 x_2) z_2\bigr) + z_1 (y_2 x_2) \over x_2 (y_2 z_2)}\cr
\sim {\bigl(x_1 (y_2 z_2) + (x_2 y_1) z_2\bigr) + (z_1 y_2) x_2 \over x_2 (y_2 z_2)} \sim {\bigl(x_1 (y_2 z_2) + x_2 (y_1 z_2)\bigr) + (z_1 y_2) x_2 \over x_2 (y_2 z_2)}\cr
\sim {x_1 (y_2 z_2) + \bigl((y_1 z_2) x_2 + (z_1 y_2) x_2\bigr) \over x_2 (y_2 z_2)} \sim {x_1 (y_2 z_2) + (y_1 z_2 + z_1 y_2) x_2 \over x_2 (y_2 z_2)}\cr
\sim {x_1 \over x_2} + {y_1 z_2 + z_1 y_2 \over y_2 z_2} \sim {x_1 \over x_2} + \left({y_1 \over y_2} + {z_1 \over z_2}\right).\cr}$$
\medskip

%\jutting{2}{311-318}

{\bf Satz 60:} {\it $${x_1 \over x_2} + {y_1 \over y_2} > {x_1 \over x_2}.$$}%

{\bf Beweis:} $$x_1 y_2 + y_1 x_2 > x_1 y_2,$$
$$(x_1 y_2 + y_1 x_2) x_2 > (x_1 y_2) x_2 = x_1 (y_2 x_2) = x_1 (x_2 y_2),$$
$${x_1 \over x_2} + {y_1 \over y_2} \sim {x_1 y_2 + y_1 x_2 \over x_2 y_2} > {x_1 \over x_2}.$$
\medskip

%\jutting{2}{319-329}

{\bf Satz 61:} {\it Aus
$${x_1 \over x_2} > {y_1 \over y_2}$$
folgt
$${x_1 \over x_2} + {z_1 \over z_2} > {y_1 \over y_2} + {z_1 \over z_2}.$$}%

{\bf Beweis:} Aus
$$x_1 y_2 > y_1 x_2$$
folgt
$$(x_1 y_2) z_2 > (y_1 x_2) z_2.$$
Wegen
$$(xy)z = x(yz) = x(zy) = (xz)y$$
ist also
$$(x_1 z_2) y_2 > (y_1 z_2) x_2$$
und
$$(z_1 x_2) y_2 = (z_1 y_2) x_2,$$
also
$$(x_1 z_2 + z_1 x_2) y_2 > (y_1 z_2 + z_1 y_2) x_2,$$
$$(x_1 z_2 + z_1 x_2)(y_2 z_2) > (y_1 z_2 + z_1 y_2)(x_2 z_2),$$
$${x_1 \over x_2} + {z_1 \over z_2} \sim {x_1 z_2 + z_1 x_2 \over x_2 z_2} > {y_1 z_2 + z_1 y_2 \over y_2 z_2} \sim {y_1 \over y_2} + {z_1 \over z_2}.$$
\medskip

%\jutting{2}{330-344}

{\bf Satz 62:} {\it Aus
$${x_1 \over x_2} > {y_1 \over y_2}\quad\hbox{\it bzw.}\quad{x_1 \over x_2} \sim {y_1 \over y_2}\quad\hbox{\it bzw.}\quad{x_1 \over x_2} < {y_1 \over y_2}$$
folgt
$$\displaylines{{x_1 \over x_2} + {z_1 \over z_2} > {y_1 \over y_2} + {z_1 \over z_2}\quad\hbox{\it bzw.}\quad{x_1 \over x_2} + {z_1 \over z_2} \sim {y_1 \over y_2} + {z_1 \over z_2}\cr
\hbox{\it bzw.}\quad{x_1 \over x_2} + {z_1 \over z_2} < {y_1 \over y_2} + {z_1 \over z_2}.\cr}$$}%

{\bf Beweis:} Der erste Teil ist Satz 61, der zweite in Satz 56
enthalten, der dritte eine Folge des ersten wegen
$${y_1 \over y_2} > {x_1 \over x_2},$$
$${y_1 \over y_2} + {z_1 \over z_2} > {x_1 \over x_2} + {z_1 \over z_2},$$
$${x_1 \over x_2} + {z_1 \over z_2} < {y_1 \over y_2} + {z_1 \over z_2}.$$
\medskip

%\jutting{2}{345-360}

{\bf Satz 63:} {\it Aus
$$\displaylines{{x_1 \over x_2} + {z_1 \over z_2} > {y_1 \over y_2} + {z_1 \over z_2}\quad\hbox{\it bzw.}\quad{x_1 \over x_2} + {z_1 \over z_2} \sim {y_1 \over y_2} + {z_1 \over z_2}\cr
\hbox{\it bzw.}\quad{x_1 \over x_2} + {z_1 \over z_2} < {y_1 \over y_2} + {z_1 \over z_2}\cr}$$
folgt
$${x_1 \over x_2} > {y_1 \over y_2}\quad\hbox{\it bzw.}\quad{x_1 \over x_2} \sim {y_1 \over y_2}\quad\hbox{\it bzw.}\quad{x_1 \over x_2} < {y_1 \over y_2}.$$}%

{\bf Beweis:} Folgt aus Satz 62, da die drei F\"alle beide Male sich
ausschlie{\ss}en und alle M\"oglichkeiten ersch\"opfen.
\medskip

%\jutting{2}{361-368}

{\bf Satz 64:} {\it Aus
$${x_1 \over x_2} > {y_1 \over y_2},\quad {z_1 \over z_2} > {u_1 \over u_2}$$
folgt
$${x_1 \over x_2} + {z_1 \over z_2} > {y_1 \over y_2} + {u_1 \over u_2}.$$}%

{\bf Beweis:} Nach Satz 61 ist
$${x_1 \over x_2} + {z_1 \over z_2} > {y_1 \over y_2} + {z_1 \over z_2}$$
und
$${y_1 \over y_2} + {z_1 \over z_2} \sim {z_1 \over z_2} + {y_1 \over y_2} > {u_1 \over u_2} + {y_1 \over y_2} \sim {y_1 \over y_2} + {u_1 \over u_2},$$
also
$${x_1 \over x_2} + {z_1 \over z_2} > {y_1 \over y_2} + {u_1 \over u_2}.$$
\medskip

%\jutting{2}{369-376}

{\bf Satz 65:} {\it Aus
$${x_1 \over x_2} \gesim {y_1 \over y_2},\quad {z_1 \over z_2} > {u_1 \over u_2}\quad\hbox{\it oder}\quad{x_1 \over x_2} > {y_1 \over y_2},\quad {z_1 \over z_2} \gesim {u_1 \over u_2}$$
folgt
$${x_1 \over x_2} + {z_1 \over z_2} > {y_1 \over y_2} + {u_1 \over u_2}.$$}%

{\bf Beweis:} Mit dem \"Aquivalenzzeichen in der Voraussetzung
durch Satz 56 und Satz 61, sonst durch Satz 64 erledigt.
\medskip

%\jutting{2}{377-389}

{\bf Satz 66:} {\it Aus
$${x_1 \over x_2} \gesim {y_1 \over y_2},\quad {z_1 \over z_2} \gesim {u_1 \over u_2}$$
folgt
$${x_1 \over x_2} + {z_1 \over z_2} \gesim {y_1 \over y_2} + {u_1 \over u_2}.$$}%

{\bf Beweis:} Mit zwei \"Aquivalenzzeichen in der Voraussetzung
durch Satz 56, sonst durch Satz 65 erledigt.
\medskip

%\jutting{2}{390-413}

{\bf Satz 67:} {\it Ist
$${x_1 \over x_2} > {y_1 \over y_2},$$
so hat
$${y_1 \over y_2} + {u_1 \over u_2} \sim {x_1 \over x_2}$$
eine L\"osung $u_1 \over u_2$.  Sind $v_1 \over v_2$ und $w_1 \over w_2$ L\"osungen, so ist
$${v_1 \over v_2} \sim {w_1 \over w_2}.$$}%

{\bf Vorbemerkung:} F\"ur
$${x_1 \over x_2} \lesim {y_1 \over y_2}$$
gibt es nach Satz 60 keine L\"osung.

{\bf Beweis:} Die zweite Behauptung folgt unmittelbar aus Satz 63;
denn f\"ur
$${y_1 \over y_2} + {v_1 \over v_2} \sim {y_1 \over y_2} + {w_1 \over w_2}$$
ist nach jenem Satz
$${v_1 \over v_2} \sim {w_1 \over w_2}.$$

Die Existenz eines $u_1 \over u_2$ (erste Behauptung) ergibt sich folgender%
ma{\ss}en.  Es ist
$$x_1 y_2 > y_1 x_2.$$
Es werde $u$ aus
$$x_1 y_2 = y_1 x_2 + u$$
bestimmt und
$$u_1 = u,\quad u_2 = x_2 y_2$$
gesetzt.  Dann ist $u_1 \over u_2$ L\"osung wegen
$${y_1 \over y_2} + {u_1 \over u_2} \sim {y_1 \over y_2} + {u \over x_2 y_2} \sim {y_1 x_2 \over x_2 y_2} + {u \over x_2 y_2} \sim {y_1 x_2 + u \over x_2 y_2} \sim {x_1 y_2 \over x_2 y_2} \sim {x_1 \over x_2}.$$
\medskip

{\bf Definition 14:} {\it Das beim Beweise des Satzes 67 konstruierte spe%
zielle $u_1 \over u_2$ hei{\ss}t ${x_1 \over x_2} - {y_1 \over y_2}$ {\rm ($-$ sprich: minus)} oder die Differenz $x_1 \over x_2$ mi%
nus $y_1 \over y_2$ oder der durch Subtraktion des Bruches $y_1 \over y_2$ vom Bruche $x_1 \over x_2$
entstehende Bruch.}

Aus
$${x_1 \over x_2} \sim {y_1 \over y_2} + {v_1 \over v_2}$$
folgt also
$${v_1 \over v_2} \sim {x_1 \over x_2} - {y_1 \over y_2}.$$
\vfill\eject


%\jutting{2}{414-435}

\line{}\vskip 7\baselineskip
\centerline{{\S}~4.}
\medskip

\centerline{\bf Multiplikation.}
\bigskip

{\bf Definition 15:} {\it Unter ${x_1 \over x_2} \cdot {y_1 \over y_2}$ {\rm ($\cdot$ sprich: mal; aber man schreibt
den Punkt meist nicht)} versteht man den Bruch $x_1 y_1 \over x_2 y_2$.

Er hei{\ss}t das Produkt von $x_1 \over x_2$ mit $y_1 \over y_2$ oder der durch Multipli%
kation von $x_1 \over x_2$ mit $y_1 \over y_2$ entstehende Bruch.}
\medskip

%\jutting{2}{436-448}

{\bf Satz 68:} {\it Aus
$${x_1 \over x_2} \sim {y_1 \over y_2},\quad {z_1 \over z_2} \sim {u_1 \over u_2}$$
folgt
$${x_1 \over x_2}{z_1 \over z_2} \sim {y_1 \over y_2}{u_1 \over u_2}.$$}%

{\bf Vorbemerkung:} Die Klasse des Produktes h\"angt also nur
den Klassen ab, zu denen die ``Faktoren'' geh\"oren.

{\bf Beweis:} $$x_1 y_2 = y_1 x_2,\quad z_1 u_2 = u_1 z_2,$$
also
$$(x_1 y_2)(z_1 u_2) = (y_1 x_2)(u_1 z_2),$$
$$(x_1 z_1)(y_2 u_2) = (y_1 u_1)(x_2 z_2),$$
$${x_1 z_1 \over x_2 z_2} \sim {y_1 u_1 \over y_2 u_2}.$$
\medskip

%\jutting{2}{449-450}

{\bf Satz 69} (kommutatives Gesetz der Multiplikation):
{\it $${x_1 \over x_2}{y_1 \over y_2} \sim {y_1 \over y_2}{x_1 \over x_2}.$$}%

{\bf Beweis:} $${x_1 \over x_2}{y_1 \over y_2} \sim {x_1 y_1 \over x_2 y_2} \sim {y_1 x_1 \over y_2 x_2} \sim {y_1 \over y_2}{x_1 \over x_2}.$$
\medskip

%\jutting{2}{451-453}

{\bf Satz 70} (assoziatives Gesetz der Multiplikation):
{\it $$\left({x_1 \over x_2}{y_1 \over y_2}\right){z_1 \over z_2} \sim {x_1 \over x_2}\left({y_1 \over y_2}{z_1 \over z_2}\right).$$}%

{\bf Beweis:} $$\displaylines{\left({x_1 \over x_2}{y_1 \over y_2}\right){z_1 \over z_2} \sim {x_1 y_1 \over x_2 y_2}{z_1 \over z_2} \sim {(x_1 y_1) z_1 \over (x_2 y_2) z_2}\cr
\sim {x_1 (y_1 z_1) \over x_2 (y_2 z_2)} \sim {x_1 \over x_2}{y_1 z_1 \over y_2 z_2} \sim {x_1 \over x_2}\left({y_1 \over y_2}{z_1 \over z_2}\right).\cr}$$
\medskip

%\jutting{2}{454-463}

{\bf Satz 71} (distributives Gesetz):
{\it $${x_1 \over x_2}\left({y_1 \over y_2} + {z_1 \over z_2}\right) \sim {x_1 \over x_2}{y_1 \over y_2} + {x_1 \over x_2}{z_1 \over z_2}.$$}%

{\bf Beweis:} $$\displaylines{{x_1 \over x_2}\left({y_1 \over y_2} + {z_1 \over z_2}\right) \sim {x_1 \over x_2}{y_1 z_2 + z_1 y_2 \over y_2 z_2} \sim {x_1 (y_1 z_2 + z_1 y_2) \over x_2 (y_2 z_2)}\cr
\sim {x_1 (y_1 z_2) + x_1 (z_1 y_2) \over x_2 (y_2 z_2)} \sim {x_1 (y_1 z_2) \over x_2 (y_2 z_2)} + {x_1 (z_1 y_2) \over x_2 (y_2 z_2)} \sim {(x_1 y_1) z_2 \over (x_2 y_2) z_2} + {(x_1 z_1) y_2 \over (x_2 z_2) y_2}\cr
\sim {x_1 y_1 \over x_2 y_2} + {x_1 z_1 \over x_2 z_2} \sim {x_1 \over x_2}{y_1 \over y_2} + {x_1 \over x_2}{z_1 \over z_2}.\cr}$$
\medskip

%\jutting{2}{464-482}

{\bf Satz 72:} {\it Aus
$${x_1 \over x_2} > {y_1 \over y_2}\quad\hbox{\it bzw.}\quad{x_1 \over x_2} \sim {y_1 \over y_2}\quad\hbox{\it bzw.}\quad{x_1 \over x_2} < {y_1 \over y_2}$$
folgt
$${x_1 \over x_2}{z_1 \over z_2} > {y_1 \over y_2}{z_1 \over z_2}\quad\hbox{\it bzw.}\quad{x_1 \over x_2}{z_1 \over z_2} \sim {y_1 \over y_2}{z_1 \over z_2}\quad\hbox{\it bzw.}\quad{x_1 \over x_2}{z_1 \over z_2} < {y_1 \over y_2}{z_1 \over z_2}.$$}%

{\bf Beweis:} 1) Aus
$${x_1 \over x_2} > {y_1 \over y_2}$$
folgt
$$x_1 y_2 > y_1 x_2,$$
$$(x_1 y_2)(z_1 z_2) > (y_1 x_2)(z_1 z_2),$$
$$(x_1 z_1)(y_2 z_2) > (y_1 z_1)(x_2 z_2),$$
$${x_1 \over x_2}{z_1 \over z_2} \sim {x_1 z_1 \over x_2 z_2} > {y_1 z_1 \over y_2 z_2} \sim {y_1 \over y_2}{z_1 \over z_2}.$$

2) Aus
$${x_1 \over x_2} \sim {y_1 \over y_2}$$
folgt nach Satz 68
$${x_1 \over x_2}{z_1 \over z_2} \sim {y_1 \over y_2}{z_1 \over z_2}.$$

3) Aus
$${x_1 \over x_2} < {y_1 \over y_2}$$
folgt
$${y_1 \over y_2} > {x_1 \over x_2},$$
also nach 1)
$${y_1 \over y_2}{z_1 \over z_2} > {x_1 \over x_2}{z_1 \over z_2},$$
$${x_1 \over x_2}{z_1 \over z_2} < {y_1 \over y_2}{z_1 \over z_2}.$$
\medskip

%\jutting{2}{483-498}

{\bf Satz 73:} {\it Aus
$${x_1 \over x_2}{z_1 \over z_2} > {y_1 \over y_2}{z_1 \over z_2}\quad\hbox{\it bzw.}\quad{x_1 \over x_2}{z_1 \over z_2} \sim {y_1 \over y_2}{z_1 \over z_2}\quad\hbox{\it bzw.}\quad{x_1 \over x_2}{z_1 \over z_2} < {y_1 \over y_2}{z_1 \over z_2}$$
folgt
$${x_1 \over x_2} > {y_1 \over y_2}\quad\hbox{\it bzw.}\quad{x_1 \over x_2} \sim {y_1 \over y_2}\quad\hbox{\it bzw.}\quad{x_1 \over x_2} < {y_1 \over y_2}.$$}%

{\bf Beweis:} Folgt aus Satz 72, da die drei F\"alle beide Male sich
ausschlie{\ss}en und alle M\"oglichkeiten ersch\"opfen.
\medskip

%\jutting{2}{499-506}

{\bf Satz 74:} {\it Aus
$${x_1 \over x_2} > {y_1 \over y_2},\quad {z_1 \over z_2} > {u_1 \over u_2}$$
folgt
$${x_1 \over x_2}{z_1 \over z_2} > {y_1 \over y_2}{u_1 \over u_2}.$$}%

{\bf Beweis:} Nach Satz 72 ist
$${x_1 \over x_2}{z_1 \over z_2} > {y_1 \over y_2}{z_1 \over z_2}$$
und
$${y_1 \over y_2}{z_1 \over z_2} \sim {z_1 \over z_2}{y_1 \over y_2} > {u_1 \over u_2}{y_1 \over y_2} \sim {y_1 \over y_2}{u_1 \over u_2},$$
also
$${x_1 \over x_2}{z_1 \over z_2} > {y_1 \over y_2}{u_1 \over u_2}.$$
\medskip

%\jutting{2}{507-514}

{\bf Satz 75:} {\it Aus
$${x_1 \over x_2} \gesim {y_1 \over y_2},\quad {z_1 \over z_2} > {u_1 \over u_2}\quad\hbox{\it oder}\quad{x_1 \over x_2} > {y_1 \over y_2},\quad {z_1 \over z_2} \gesim {u_1 \over u_2}$$
folgt
$${x_1 \over x_2}{z_1 \over z_2} > {y_1 \over y_2}{u_1 \over u_2}.$$}%

{\bf Beweis:} Mit dem \"Aquivalenzzeichen in der Voraussetzung
durch Satz 68 und Satz 72, sonst durch Satz 74 erledigt.
\medskip

%\jutting{2}{515-527}

{\bf Satz 76:} {\it Aus
$${x_1 \over x_2} \gesim {y_1 \over y_2},\quad {z_1 \over z_2} \gesim {u_1 \over u_2}$$
folgt
$${x_1 \over x_2}{z_1 \over z_2} \gesim {y_1 \over y_2}{u_1 \over u_2}.$$}%

{\bf Beweis:} Mit zwei \"Aquivalenzzeichen in der Voraussetzung
durch Satz 68, sonst durch Satz 75 erledigt.
\medskip

%\jutting{2}{528-534}

{\bf Satz 77:} {\it Die \"Aquivalenz
$${y_1 \over y_2}{u_1 \over u_2} \sim {x_1 \over x_2}$$
wo $x_1 \over x_2$ und $y_1 \over y_2$ gegeben sind, hat eine L\"osung $u_1 \over u_2.$  Sind $v_1 \over v_2$ und $w_1 \over w_2$
L\"osungen, so ist
$${v_1 \over v_2} \sim {w_1 \over w_2}.$$}%

{\bf Beweis:} Die zweite Behauptung folgt unmittelbar aus Satz
73; denn f\"ur
$${y_1 \over y_2}{v_1 \over v_2} \sim {y_1 \over y_2}{w_1 \over w_2}$$
ist nach jenem Satz
$${v_1 \over v_2} \sim {w_1 \over w_2}.$$

Die Existenz eines $u_1 \over u_2$ (erste Behauptung) ergibt sich folgender%
ma{\ss}en.  F\"ur
$$u_1 = x_1 y_2,\quad u_2 = x_2 y_1$$
ist $u_1 \over u_2$ L\"osung wegen
$${y_1 \over y_2}{u_1 \over u_2} \sim {u_1 \over u_2}{y_1 \over y_2} \sim {x_1 y_2 \over x_2 y_1}{y_1 \over y_2} \sim {(x_1 y_2) y_1 \over (x_2 y_1) y_2} \sim {x_1 (y_1 y_2) \over x_2 (y_1 y_2)} \sim {x_1 \over x_2}.$$
\vfill\eject


%\jutting{2}{535-542}

\line{}\vskip 7\baselineskip
\centerline{{\S}~5.}
\medskip

\centerline{\bf Rationale Zahlen und ganze Zahlen.}
\bigskip

{\bf Definition 16:} {\it Unter einer rationalen Zahl versteht man die
Menge aller einem festen Bruch \"aquiva\-lenten Br\"uche {\rm (also eine Klasse
im Sinne des {\S}~1).}}

Gro{\ss}e lateinische Buchstaben bezeichnen durchweg, wofern
nichts anderes gesagt wird, rationale Zahlen.
\medskip

%\jutting{2}{543-587}

{\bf Definition 17:} {\it $$X = Y$$
{\rm ($=$ sprich: gleich),} wenn beide Mengen dieselben Br\"uche umfassen.
Anderenfalls
$$X \ne Y$$
{\rm ($\ne$ sprich: ungleich).}}

Trivial sind die drei S\"atze:
\medskip

%\jutting{2}{588-588}

{\bf Satz 78:} {\it $$X = X.$$}%
\medskip

%\jutting{2}{589-590}

{\bf Satz 79:} {\it Aus
$$X = Y$$
folgt
$$Y = X.$$}%
\medskip

%\jutting{2}{591-592}

{\bf Satz 80:} {\it Aus
$$X = Y,\quad Y = Z$$
folgt
$$X = Z.$$}%
\medskip

%\jutting{2}{593-619}

{\bf Definition 18:} {\it $$X > Y$$
{\rm ($>$ sprich: gr\"o{\ss}er als),} wenn f\"ur einen {\rm (also nach Satz 44 f\"ur je
einen)} Bruch $x_1 \over x_2$ bzw. $y_1 \over y_2$ aus der Menge X bzw. Y
$${x_1 \over x_2} > {y_1 \over y_2}$$
ist.}
\medskip

%\jutting{2}{620-646}

{\bf Definition 19:} {\it $$X < Y$$
{\rm ($<$ sprich: kleiner als),} wenn f\"ur einen {\rm (also nach Satz 45 f\"ur je
einen)} Bruch $x_1 \over x_2$ bzw. $y_1 \over y_2$ aus der Menge X bzw. Y
$${x_1 \over x_2} < {y_1 \over y_2}$$
ist.}
\medskip

%\jutting{2}{647-668}

{\bf Satz 81:} {\it Sind X, Y beliebig, so liegt genau einer der F\"alle
$$X = Y,\quad X > Y,\quad X < Y$$
vor.}

{\bf Beweis:} Satz 41.
\medskip

%\jutting{2}{669-674}

{\bf Satz 82:} {\it Aus
$$X > Y$$
folgt
$$Y < X.$$}%

{\bf Beweis:} Satz 42.
\medskip

%\jutting{2}{675-680}

{\bf Satz 83:} {\it Aus
$$X < Y$$
folgt
$$Y > X.$$}%

{\bf Beweis:} Satz 43.
\medskip

%\jutting{2}{681-695}

{\bf Definition 20:} {\it $$X \ge Y$$
bedeutet
$$X > Y\quad\hbox{\it oder}\quad X = Y.$$
{\rm ($\ge$ sprich: gr\"o{\ss}er oder gleich.)}}
\medskip

%\jutting{2}{696-726}

{\bf Definition 21:} {\it $$X \le Y$$
bedeutet
$$X < Y\quad\hbox{\it oder}\quad X = Y.$$
{\rm ($\le$ sprich: kleiner oder gleich.)}}
\medskip

%\jutting{2}{727-732}

{\bf Satz 84:} {\it Aus
$$X \ge Y$$
folgt
$$Y \le X.$$}%

{\bf Beweis:} Satz 48.
\medskip

%\jutting{2}{733-738}

{\bf Satz 85:} {\it Aus
$$X \le Y$$
folgt
$$Y \ge X.$$}%

{\bf Beweis:} Satz 49.
\medskip

%\jutting{2}{739-747}

{\bf Satz 86} (Transitivit\"at der Ordnung): {\it Aus
$$X < Y,\quad Y < Z$$
folgt
$$X < Z.$$}%

{\bf Beweis:} Satz 50.
\medskip

%\jutting{2}{748-763}

{\bf Satz 87:} {\it Aus
$$X \le Y,\quad Y < Z\quad\hbox{\it oder}\quad X < Y,\quad Y \le Z$$
folgt
$$X < Z.$$}%

{\bf Beweis:} Satz 51.
\medskip

%\jutting{2}{764-772}

{\bf Satz 88:} {\it Aus
$$X \le Y,\quad Y \le Z$$
folgt
$$X \le Z.$$}%

{\bf Beweis:} Satz 52.
\medskip

%\jutting{2}{773-778}

{\bf Satz 89:} {\it Zu $X$ gibt es ein
$$Z > X.$$}%

{\bf Beweis:} Satz 53.
\medskip

%\jutting{2}{779-784}

{\bf Satz 90:} {\it Zu $X$ gibt es ein
$$Z < X.$$}%

{\bf Beweis:} Satz 54.
\medskip

%\jutting{2}{785-794}

{\bf Satz 91:} {\it Ist
$$X < Y,$$
so gibt es ein $Z$ mit
$$X < Z < Y.$$}%

{\bf Beweis:} Satz 55.
\medskip

%\jutting{2}{795-810}

{\bf Definition 22:} {\it Unter $X + Y$ {\rm ($+$ sprich: plus)} versteht man die
Klasse, der eine {\rm (also nach Satz 56 jede)} Summe eines Bruches aus
$X$ und eines Bruches aus $Y$ angeh\"ort.

Diese rationale Zahl hei{\ss}t die Summe von $X$ und $Y$ oder die
durch Addition von $Y$ zu $X$ entstehende rationale Zahl.}
\medskip

%\jutting{2}{811-815}

{\bf Satz 92} (kommutatives Gesetz der Addition):
{\it $$X + Y = Y + X.$$}%

{\bf Beweis:} Satz 58.
\medskip

%\jutting{2}{816-824}

{\bf Satz 93} (assoziatives Gesetz der Addition):
{\it $$(X + Y) + Z = X + (Y + Z).$$}%

{\bf Beweis:} Satz 59.
\medskip

%\jutting{2}{825-829}

{\bf Satz 94:} {\it $$X + Y > X.$$}%

{\bf Beweis:} Satz 60.
\medskip

%\jutting{2}{830-835}

{\bf Satz 95:} {\it Aus
$$X > Y$$
folgt
$$X + Z > Y + Z.$$}%

{\bf Beweis:} Satz 61.
\medskip

%\jutting{2}{836-861}

{\bf Satz 96:} {\it Aus
$$X > Y\quad\hbox{\it bzw.}\quad X = Y\quad\hbox{\it bzw.}\quad X < Y$$
folgt
$$X + Z > Y + Z\quad\hbox{\it bzw.}\quad X + Z = Y + Z\quad\hbox{\it bzw.}\quad X + Z < Y + Z.$$}%

{\bf Beweis:} Satz 62.
\medskip

%\jutting{2}{862-882}

{\bf Satz 97:} {\it Aus
$$X + Z > Y + Z\quad\hbox{\it bzw.}\quad X + Z = Y + Z\quad\hbox{\it bzw.}\quad X + Z < Y + Z$$
folgt
$$X > Y\quad\hbox{\it bzw.}\quad X = Y\quad\hbox{\it bzw.}\quad X < Y.$$}%

{\bf Beweis:} Satz 63.
\medskip

%\jutting{2}{883-890}

{\bf Satz 98:} {\it Aus
$$X > Y,\quad Z > U$$
folgt
$$X + Z > Y + U.$$}%

{\bf Beweis:} Satz 64.
\medskip

%\jutting{2}{891-906}

{\bf Satz 99:} {\it Aus
$$X \ge Y,\quad Z > U\quad\hbox{\it oder}\quad X > Y,\quad Z \ge U$$
folgt
$$X + Z > Y + U.$$}%

{\bf Beweis:} Satz 65.
\medskip

%\jutting{2}{907-914}

{\bf Satz 100:} {\it Aus
$$X \ge Y,\quad Z \ge U$$
folgt
$$X + Z \ge Y + U.$$}%

{\bf Beweis:} Satz 66.
\medskip

%\jutting{2}{915-921}

{\bf Satz 101:} {\it Ist
$$X > Y,$$
so hat
$$Y + U = X$$
genau eine l\"osung $U$.}

{\bf Vorbemerkung:} F\"ur
$$X \le Y$$
gibt es nach Satz 94 keine L\"osung.

%\jutting{2}{922-939}

{\bf Beweis:} Satz 67.
\medskip

%\jutting{2}{940-946}

{\bf Definition 23:} {\it Dies $U$ hei{\ss}t $X - Y$ {\rm ($-$ sprich: minus)} oder die
Differenz $X$ minus $Y$ oder die durch Subtraktion der rationalen Zahl
$Y$ von der rationalen Zahl $X$ entstehende rationale Zahl.}
\medskip

%\jutting{2}{947-961}

{\bf Definition 24:} {\it Unter $X \cdot Y$ {\rm ($\cdot$ sprich: mal; aber man schreibt
den Punkt meist nicht)} versteht man die Klasse, der ein {\rm (also nach
Satz 68 jedes)} Produkt eines Bruches aus $X$ mit einem Bruche aus
$Y$ angeh\"ort.

Diese rationale Zahl hei{\ss}t das Produkt von $X$ mit $Y$ oder die
durch Multiplikation von $X$ mit $Y$ entstehende rationale Zahl.}
\medskip

%\jutting{2}{962-966}

{\bf Satz 102} (kommutatives Gesetz der Multiplikation):
{\it $$XY = YX.$$}%

{\bf Beweis:} Satz 69.
\medskip

%\jutting{2}{967-974}

{\bf Satz 103} (assoziatives Gesetz der Multiplikation):
{\it $$(XY)Z = X(YZ).$$}%

{\bf Beweis:} Satz 70.
\medskip

%\jutting{2}{975-984}

{\bf Satz 104} (distributives Gesetz):
{\it $$X(Y + Z) = XY + XZ.$$}%

{\bf Beweis:} Satz 71.
\medskip

%\jutting{2}{985-1009}

{\bf Satz 105:} {\it Aus
$$X > Y\quad\hbox{\it bzw.}\quad X = Y\quad\hbox{\it bzw.}\quad X < Y$$
folgt
$$XZ > YZ\quad\hbox{\it bzw.}\quad XZ = YZ\quad\hbox{\it bzw.}\quad XZ < YZ.$$}%

{\bf Beweis:} Satz 72.
\medskip

%\jutting{2}{1010-1031}

{\bf Satz 106:} {\it Aus
$$XZ > YZ\quad\hbox{\it bzw.}\quad XZ = YZ\quad\hbox{\it bzw.}\quad XZ < YZ$$
folgt
$$X > Y\quad\hbox{\it bzw.}\quad X = Y\quad\hbox{\it bzw.}\quad X < Y.$$}%

{\bf Beweis:} Satz 73.
\medskip

%\jutting{2}{1032-1038}

{\bf Satz 107:} {\it Aus
$$X > Y,\quad Z > U$$
folgt
$$XZ > YU.$$}%

{\bf Beweis:} Satz 74.
\medskip

%\jutting{2}{1039-1054}

{\bf Satz 108:} {\it Aus
$$X \ge Y,\quad Z > U\quad\hbox{\it oder}\quad X > Y,\quad Z \ge U$$
folgt
$$XZ > YU.$$}%

{\bf Beweis:} Satz 75.
\medskip

%\jutting{2}{1055-1062}

{\bf Satz 109:} {\it Aus
$$X \ge Y,\quad Z \ge U$$
folgt
$$XZ \ge YU.$$}%

{\bf Beweis:} Satz 76.
\medskip

%\jutting{2}{1063-1080}

{\bf Satz 110:} {\it Die Gleichung
$$YU = X,$$
wo $X$ und $Y$ gegeben sind, hat genau eine L\"osung $U$.}

{\bf Beweis:} Satz 77.
\medskip

%\jutting{2}{1081-1097}

{\bf Satz 111:} {\it Aus
$${x \over 1} > {y \over 1}\quad\hbox{\it bzw.}\quad{x \over 1} \sim {y \over 1}\quad\hbox{\it bzw.}\quad{x \over 1} < {y \over 1}$$
folgt
$$x > y\quad\hbox{\it bzw.}\quad x = y\quad\hbox{\it bzw.}\quad x < y$$
und umgekehrt.}

{\bf Beweis:} $$x \cdot 1 > y \cdot 1\quad\hbox{\rm bzw.}\quad x \cdot 1 = y \cdot 1\quad\hbox{\rm bzw.}\quad x \cdot 1 < y \cdot 1$$
bedeutet dasselbe wie
$$x > y\quad\hbox{\rm bzw.}\quad x = y\quad\hbox{\rm bzw.}\quad x < y.$$
\medskip

%\jutting{2}{1098-1124}

{\bf Definition 25:} {\it Eine rationale Zahl hei{\ss}t ganz, wenn unter den
Br\"uchen, deren Gesamtheit sie ist, ein Bruch $x \over 1$ vorkommt.}

Dies $x$ ist nach Satz 111 eindeutig bestimmt, und umgekehrt
entspricht jedem $x$ genau eine ganze Zahl.
\medskip

%\jutting{2}{1125-1193}

{\bf Satz 112:} {\it $${x \over 1} + {y \over 1} \sim {x + y \over 1},$$
$${x \over 1}{y \over 1} \sim {xy \over 1}.$$}%

{\bf Vorbemerkung:} Summe und Produkt zweier ganzer Zahlen
sind also ganze Zahlen.

{\bf Beweis:} 1) Nach Satz 57 ist
$${x \over 1} + {y \over 1} \sim {x + y \over 1}.$$

2) Nach Definition 15 ist
$${x \over 1}{y \over 1} \sim {xy \over 1 \cdot 1} \sim {xy \over 1}.$$
\medskip

%\jutting{2}{1194-1223}

{\bf Satz 113:} {\it Die ganzen Zahlen gen\"ugen den f\"unf Axiomen der
nat\"urlichen Zahlen, wenn die Klasse von $1 \over 1$ an Stelle von $1$ genommen
wird und als Nachfolger der Klasse von $x \over 1$ die Klasse von $x' \over 1$ an%
gesehen wird.}

\ifx\fr\undefined
  \font\teneufm=eufm10 \font\seveneufm=eufm7 \font\fiveeufm=eufm5
  \csname newfam\endcsname\eufmfam
  \textfont\eufmfam=\teneufm \scriptfont\eufmfam=\seveneufm \scriptscriptfont\eufmfam=\fiveeufm
  \def\fr{\fam\eufmfam}
\fi

{\bf Beweis:} $\overline{\fr Z}$ sei die Menge der ganzen Zahlen.

1) Die Klasse von $1 \over 1$ geh\"ort zu $\overline{\fr Z}$.

2) Zu jeder ganzen Zahl haben wir einen Nachfolger eindeutig
erkl\"art.

3) Er ist stets von der Klasse von $1 \over 1$ verschieden, da stets
$$x' \ne 1.$$

4) Stimmen die Klassen von $x' \over 1$ und $y' \over 1$ \"uberein, so ist
$${x' \over 1} \sim {y' \over 1},$$
$$x' = y',$$
$$x = y,$$
$${x \over 1} \sim {y \over 1},$$
und die Klassen von $x \over 1$ und $y \over 1$ stimmen \"uberein.

5) Eine Menge $\overline{\fr M}$ von ganzen Zahlen habe die Eigenschaften:

I) Die Klasse von $1 \over 1$ geh\"ort zu $\overline{\fr M}$.

II) Falls die Klasse von $x \over 1$ zu $\overline{\fr M}$ geh\"ort, so geh\"ort die Klasse
von $x'\over 1$ zu $\overline{\fr M}$.

Dann bezeichne $\fr M$ die Menge der $x$, f\"ur die die Klasse von
$x \over 1$ zu $\overline{\fr M}$ geh\"ort.  Alsdann ist $1$ zu $\fr M$ und mit jedem $x$ von $\fr M$ auch
$x'$ zu $\fr M$ geh\"orig.  Also geh\"ort jede nat\"urliche Zahl zu $\fr M$, also
jede ganze Zahl zu $\overline{\fr M}$.
\bigskip

%\jutting{2}{1224-1412}

Da $=$, $>$, $<$, Summe und Produkt (nach Satz 111 und 112)
den alten Begriflen entsprechen, haben die ganzen Zahlen alle
Eigenschaften, die wir in Kapitel 1 f\"ur die nat\"urlichen Zahlen be%
wiesen haben.

Daher werfen wir die nat\"urlichen Zahlen weg, ersetzen sie
durch die entsprechenden ganzen Zahlen und haben fortan (da auch
die Br\"uche \"uberfi\"ussig werden) in bezug auf das Bisherige nur von
rationalen Zahlen zu reden.  (Die nat\"urlichen Zahlen verbleiben
paarweise \"uber und unter dem Strich im Begriff des Bruches, und
die Br\"uche bleiben als Individuen der Menge, die rationale Zahl
hei{\ss}t.)
\medskip

%\jutting{2}{1413-1422}

{\bf Definition 26:} {\it {\rm (Das frei gewordene Zeichen)} $x$ bezeichnet die ganze
Zahl, die durch die Klasse von $x \over 1$ gegeben ist.}

In unserer neuen Sprache ist also z.~B.
$$X \cdot 1 = X;$$
denn
$${x_1 \over x_2} \cdot {1 \over 1} \sim {x_1 \cdot 1 \over x_2 \cdot 1} \sim {x_1 \over x_2}.$$
\medskip

%\jutting{2}{1423-1429}

{\bf Satz 114:} {\it Ist $Z$ die zum Bruch $x \over y$ geh\"orige rationale Zahl, so ist
$$yZ = x.$$}%

{\bf Beweis:} $${y \over 1}{x \over y} \sim {yx \over 1 \cdot y} \sim {xy \over 1 \cdot y} \sim {x \over 1}.$$
\medskip

%\jutting{2}{1430-1439}

{\bf Definition 27:} {\it Das $U$ des Satzes 110 hei{\ss}t Quotient von $X$
durch $Y$ oder die durch Division von $X$ durch $Y$ entstehende rationale
Zahl.  Es werde mit $X \over Y$ bezeichnet {\rm (sprich: $X$ durch $Y$).}}

Sind $X$ und $Y$ ganze Zahlen, also $X = x$, $Y = y$, so be%
deutet die durch die Definitionen 26 und 27 erkl\"arte rationale Zahl
$x \over y$ nach Satz 114 die Klasse, der der Bruch $x \over y$ im alten Sinne an%
geh\"ort.  Eine Verwechselung beider Zeichen $x \over y$ ist nicht zu be%
f\"urchten, da Br\"uche in Zukunft nicht mehr gesondert vorkommen
werden; es bezeichnet fortan $x \over y$ stets eine rationale Zahl. Um%
gekehrt l\"a{\ss}t sich jede rationale Zahl in der Form $x \over y$ darstellen,
auf Grund von Satz 114 und Definition 27.
\medskip

%\jutting{2}{1440-1465}

{\bf Satz 115:} {\it Sind $X$ und $Y$ gegeben, so gibt es ein $z$ mit
$$zX > Y.$$}%

{\bf Beweis:} $Y \over X$ ist eine rationale Zahl; nach Satz 89 gibt es (in
unserer neuen Sprache) ganze Zahlen $z$, $v$ mit
$${z \over v} > {Y \over X}.$$
Nach Satz 111 ist
$$v \ge 1,$$
also nach Satz 105
$$zX = Xz = X\left({z \over v}v\right) = \left(X{z \over v}\right)v \ge \left(X{z \over v}\right) \cdot 1 = X{z \over v} > X{Y \over X} = Y.$$
\vfill\eject


