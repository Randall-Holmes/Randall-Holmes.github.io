%\jutting{5}{1-175}

\line{}\vskip 7\baselineskip
\centerline{\sl Kapitel 5.}
\medskip

\centerline{\bf Komplexe Zahlen.}
\bigskip

\centerline{{\S}~1.}
\medskip

\centerline{\bf Definition.}
\bigskip

\ifx\Alpha\undefined
  \let\Alpha=A \let\Beta=B \let\Zeta=Z \let\Eta=H \let\Upsilon=Y
\fi

{\bf Definition 57:} {\it Eine komplexe Zahl ist ein Paar reeller Zahlen
$\Xi_1$, $\Xi_2$ {\rm (in bestimmter Reihenfolge).}  Wir bezeichnen die komplexe
Zahl mit $[\Xi_1,\> \Xi_2]$. Dabei gelten $[\Xi_1,\> \Xi_2]$ und $[\Eta_1,\> \Eta_2]$ als dieselbe
Zahl (als gleich; {\rm schreibe: $=$}) genau dann, wenn
$$\Xi_1 = \Eta_1,\quad \Xi_2 = \Eta_2$$
ist; sonst als ungleich {\rm (verschieden; schreibe: $\ne$).}}

Kleine deutsche Buchstaben bedeuten durchweg komplexe
Zahlen.

\ifx\fr\undefined
  \font\teneufm=eufm10 \font\seveneufm=eufm7 \font\fiveeufm=eufm5
  \csname newfam\endcsname\eufmfam
  \textfont\eufmfam=\teneufm \scriptfont\eufmfam=\seveneufm \scriptscriptfont\eufmfam=\fiveeufm
  \def\fr{\fam\eufmfam}
\fi

F\"ur jedes $\fr x$ und jedes $\fr y$ liegt somit genau einer der F\"alle
$$\fr x = y,\quad x \ne y$$
vor.  Bei den komplexen Zahlen vermischen sich die Begriffe der
Identit\"at und Gleichheit, so da{\ss} die drei S\"atze trivial sind:
\medskip

%\jutting{5}{176-176}

{\bf Satz 206:} {\it $$\fr x = x.$$}%
\medskip

%\jutting{5}{177-178}

{\bf Satz 207:} {\it Aus
$$\fr x = y$$
folgt
$$\fr y = x.$$}%
\medskip

%\jutting{5}{179-180}

{\bf Satz 208:} {\it Aus
$$\fr x = y,\quad y = z$$
folgt
$$\fr x = z.$$}%
\medskip

%\jutting{5}{181-181}

{\bf Definition 58:} {\it $${\fr n} = [0,\> 0].$$}%
\medskip

%\jutting{5}{182-182}

{\bf Definition 59:} {\it $${\fr e} = [1,\> 0].$$}%

Die Buchstaben $\fr n$ und $\fr e$ bleiben also f\"ur bestimmte komplexe
Zahlen reserviert.
\vfill\eject


%\jutting{5}{183-195}

\line{}\vskip 7\baselineskip
\centerline{{\S}~2.}
\medskip

\centerline{\bf Addition.}
\bigskip

{\bf Definition 60:} {\it Ist
$${\fr x} = [\Xi_1,\> \Xi_2],\quad {\fr y} = [\Eta_1,\> \Eta_2],$$
so ist
$${\fr x + y} = [\Xi_1 + \Eta_1,\> \Xi_2 + \Eta_2].$$
{\rm ($+$ sprich: plus.)}  $\fr x + y$ hei{\ss}t die Summe von $\fr x$ und $\fr y$ oder die durch
Addition von $\fr y$ zu $\fr x$ entstehende {\rm (komplexe)} Zahl.}
\medskip

%\jutting{5}{196-197}

{\bf Satz 209} (kommutatives Gesetz der Addition):
{\it $$\fr x + y = y + x.$$}%

{\bf Beweis:} $$[\Xi_1 + \Eta_1,\> \Xi_2 + \Eta_2] = [\Eta_1 + \Xi_1,\> \Eta_2 + \Xi_2].$$
\medskip

%\jutting{5}{198-201}

{\bf Satz 210:} {\it $$\fr x + n = x.$$}%

{\bf Beweis:} $$[\Xi_1,\> \Xi_2] + [0,\> 0] = [\Xi_1 + 0,\> \Xi_2 + 0] = [\Xi_1,\> \Xi_2].$$
\medskip

%\jutting{5}{202-204}

{\bf Satz 211} (assoziatives Gesetz der Addition):
{\it $$\fr (x + y) + z = x + (y + z).$$}%

{\bf Beweis:} Ist
$${\fr x} = [\Xi_1,\> \Xi_2],\quad {\fr y} = [\Eta_1,\> \Eta_2],\quad {\fr z} = [\Zeta_1,\> \Zeta_2],$$
so ist nach Satz 186
$$\displaylines{{\fr (x + y) + z} = [\Xi_1 + \Eta_1,\> \Xi_2 + \Eta_2] + [\Zeta_1,\> \Zeta_2] = [(\Xi_1 + \Eta_1) + \Zeta_1,\> (\Xi_2 + \Eta_2) + \Zeta_2]\cr
= [\Xi_1 + (\Eta_1 + \Zeta_1),\> \Xi_2 + (\Eta_2 + \Zeta_2)] = [\Xi_1,\> \Xi_2] + [\Eta_1 + \Zeta_1,\> \Eta_2 + \Zeta_2] = {\fr x + (y + z)}.\cr}$$
\medskip

%\jutting{5}{205-216}

{\bf Satz 212:} {\it Bei gegebenen $\fr x$, $\fr y$ hat
$$\fr y + u = x.$$
genau eine L\"osung $\fr u$, n\"amlich,
$${\fr x} = [\Xi_1,\> \Xi_2],\quad {\fr y} = [\Eta_1,\> \Eta_2]$$
gesetzt,
$${\fr u} = [\Xi_1 - \Eta_1,\> \Xi_2 - \Eta_2].$$}%

{\bf Beweis:} F\"ur jedes
$${\fr u} = [\Upsilon_1,\> \Upsilon_2]$$
ist
$${\fr y + u} = [\Eta_1 + \Upsilon_1,\> \Eta_2 + \Upsilon_2],$$
und es wird genau
$$\Eta_1 + \Upsilon_1 = \Xi_1,\quad \Eta_2 + \Upsilon_2 = \Xi_2$$
verlangt, so da{\ss} Satz 187 alles beweist.
\medskip

%\jutting{5}{217-236}

{\bf Definition 61:} {\it Das $\fr u$ des Satzes 212 hei{\ss}t ${\fr x - y}$ {\rm ($-$ sprich:
minus).}  ${\fr x - y}$ hei{\ss}t auch die Differenz $\fr x$ minus $\fr y$ oder die durch Sub%
traktion des $\fr y$ von $\fr x$ entstehende Zahl.}
\medskip

%\jutting{5}{237-250}

{\bf Satz 213:} {\it Es ist
$$\fr x - y = n$$
dann und nur dann, wenn
$$\fr x = y.$$}%

{\bf Beweis:} Es ist
$$\Xi_1 - \Eta_1 = \Xi_2 - \Eta_2 = 0$$
genau dann, wenn
$$\Xi_1 = \Eta_1,\quad \Xi_2 = \Eta_2.$$
\medskip

%\jutting{5}{251-251}

{\bf Definition 62:} {\it $$\fr -x = n - x.$$
{\rm ($-$ links sprich: minus.)}}
\medskip

%\jutting{5}{252-257}

{\bf Satz 214:} {\it F\"ur
$${\fr x} = [\Xi_1,\> \Xi_2]$$
ist
$${\fr -x} = [-\Xi_1,\> -\Xi_2].$$}%

{\bf Beweis:} $$\displaylines{-[\Xi_1,\> \Xi_2] = [0,\> 0] - [\Xi_1,\> \Xi_2] = [0 - \Xi_1,\> 0 - \Xi_2]\cr
= [-\Xi_1,\> -\Xi_2]\cr}$$
\medskip

%\jutting{5}{258-265}

{\bf Satz 215:} {\it $$\fr -(-x) = x.$$}%

{\bf Beweis:} Nach Satz 177 ist
$$-(-\Xi_1) = \Xi_1,\quad -(-\Xi_2) = \Xi_2.$$
\medskip

%\jutting{5}{266-270}

{\bf Satz 216:} {\it $$\fr x + (-x) = n.$$}%

{\bf Beweis:} Nach Satz 179 ist
$$\Xi_1 + (-\Xi_1) = 0,\quad \Xi_2 + (-\Xi_2) = 0.$$
\medskip

%\jutting{5}{271-272}

{\bf Satz 217:} {\it $$\fr -(x + y) = -x + (-y).$$}%

{\bf Beweis:} Nach Satz 180 ist,
$${\fr x} = [\Xi_1,\> \Xi_2],\quad {\fr y} = [\Eta_1,\> \Eta_2]$$
gesetzt,
$$\displaylines{{\fr -(x + y)} = [-(\Xi_1 + \Eta_1),\> -(\Xi_2 + \Eta_2)] = [-\Xi_1 + (-\Eta_1),\> -\Xi_2 + (-\Eta_2)]\cr
= [-\Xi_1,\> -\Xi_2] + [-\Eta_1,\> -\Eta_2] = {\fr -x + (-y)}.\cr}$$
\medskip

%\jutting{5}{273-274}

{\bf Satz 218:} {\it $$\fr x - y = x + (-y).$$}%

{\bf Beweis:} $$[\Xi_1 - \Eta_1,\> \Xi_2 - \Eta_2] = [\Xi_1,\> \Xi_2] + [-\Eta_1,\> -\Eta_2].$$
\medskip

%\jutting{5}{275-279}

{\bf Satz 219:} {\it $$\fr -(x - y) = y - x.$$}%

{\bf Beweis:} $$\displaylines{\fr -(x - y) = -\bigl(x + (-y)\bigr) = -x + \bigl(-(-y)\bigr) = -x + y\cr
= \fr y + (-x) = y - x.\cr}$$
\vfill\eject


%\jutting{5}{280-306}

\line{}\vskip 7\baselineskip
\centerline{{\S}~3.}
\medskip

\centerline{\bf Multiplikation.}
\bigskip

{\bf Definition 63:} {\it Ist
$${\fr x} = [\Xi_1,\> \Xi_2],\quad {\fr y} = [\Eta_1,\> \Eta_2],$$
so ist
$${\fr x \cdot y} = [\Xi_1\Eta_1 - \Xi_2\Eta_2,\> \Xi_1\Eta_2 + \Xi_2\Eta_1].$$
{\rm ($\cdot$ sprich: mal: aber man schreibt den Punkt meist nicht.)}  $\fr x \cdot y$
hei{\ss}t das Produkt von $\fr x$ mit $\fr y$ oder die durch Multiplikation von $\fr x$
mit $\fr y$ entstehende Zahl.}

%\jutting{5}{307-312}

{\bf Satz 220} (kommutatives Gesetz der Multiplikation):
{\it $$\fr xy = yx.$$}%

{\bf Beweis:} $$\displaylines{[\Xi_1,\> \Xi_2][\Eta_1,\> \Eta_2] = [\Xi_1\Eta_1 - \Xi_2\Eta_2,\> \Xi_1\Eta_2 + \Xi_2\Eta_1]\cr
= [\Eta_1\Xi_1 - \Eta_2\Xi_2,\> \Eta_1\Xi_2 + \Eta_2\Xi_1] = [\Eta_1,\> \Eta_2][\Xi_1,\> \Xi_2].\cr}$$
\medskip

%\jutting{5}{313-378}

{\bf Satz 221}:{\it  Es ist
$$\fr xy = n$$
dann und nur dann, wenn mindestens eine der beiden Zahlen $\fr x$, $\fr y$
gleich $\fr n$ ist.}

{\bf Beweis:} Es sei
$${\fr x} = [\Xi_1,\> \Xi_2],\quad {\fr y} = [\Eta_1,\> \Eta_2].$$

I) Aus
$$\fr x = n$$
folgt
$$\Xi_1 = \Xi_2 = 0,$$
$${\fr xy} = [0 \cdot \Eta_1 - 0 \cdot \Eta_2,\> 0 \cdot \Eta_2 + 0 \cdot \Eta_1] = [0,\> 0] = {\fr n}.$$

2) Aus
$$\fr y = n$$
folgt nach Satz 220 und 1)
$$\fr xy = yx = nx = n.$$

3) Aus
$$\fr xy = n$$
soll gefolgert werden, da{\ss}
$$\fr x = n \quad\hbox{\rm oder}\quad y = n$$
ist.  Wir d\"urfen daher voraussetzen
$$\fr y \ne n,$$
d. h.
$$\Eta_1\Eta_1 + \Eta_2\Eta_2 > 0,$$
und haben
$$\fr x = n,$$
d. h.
$$\Xi_1 = \Xi_2 = 0,$$
zu beweisen.

Nach Voraussetzung ist
$$\Xi_1\Eta_1 - \Xi_2\Eta_2 = 0 = \Xi_1\Eta_2 + \Xi_2\Eta_1,$$
also
$$\displaylines{0 = (\Xi_1\Eta_1 - \Xi_2\Eta_2)\Eta_1 + (\Xi_1\Eta_2 + \Xi_2\Eta_1)\Eta_2\cr
= \bigl((\Xi_1\Eta_1)\Eta_1 - (\Xi_2\Eta_2)\Eta_1\bigr) + \bigl((\Xi_1\Eta_2)\Eta_2 + (\Xi_2\Eta_1)\Eta_2\bigr)\cr
= \bigl(\Xi_1(\Eta_1\Eta_1) - \Xi_2(\Eta_2\Eta_1)\bigr) + \bigl(\Xi_1(\Eta_2\Eta_2) + \Xi_2(\Eta_1\Eta_2)\bigr)\cr
= \Bigl(\bigl(\Xi_1(\Eta_1\Eta_1) - \Xi_2(\Eta_1\Eta_2)\bigr) + \Xi_2(\Eta_1\Eta_2)\Bigr) + \Xi_1(\Eta_2\Eta_2)\cr
= \Xi_1(\Eta_1\Eta_1) + \Xi_1(\Eta_2\Eta_2) = \Xi_1(\Eta_1\Eta_1 + \Eta_2\Eta_2),\cr}$$
also
$$\Xi_1 = 0,$$
$$\Xi_2\Eta_2 = 0 = \Xi_2\Eta_1.$$
Da $\Eta_1$ und $\Eta_2$ nicht beide $0$ sind, ist also
$$\Xi_2 = 0.$$
\medskip

%\jutting{5}{379-386}

{\bf Satz 222:} {\it $$\fr xe = x.$$}%

{\bf Beweis:} $$[\Xi_1,\> \Xi_2][1,\> 0] = [\Xi_1 \cdot 1 - \Xi_2 \cdot 0,\> \Xi_1 \cdot 0 + \Xi_2 \cdot 1] = [\Xi_1,\> \Xi_2].$$
\medskip

%\jutting{5}{387-394}

{\bf Satz 223:} {\it $$\fr x(-e) = -x.$$}%

{\bf Beweis:} $$\displaylines{[\Xi_1,\> \Xi_2][-1,\> 0] = [\Xi_1 \cdot (-1) - \Xi_2 \cdot 0,\> \Xi_1 \cdot 0 + \Xi_2 \cdot (-1)]\cr
= [-\Xi_1,\> -\Xi_2].\cr}$$
\medskip

%\jutting{5}{395-408}

{\bf Satz 224:} {\it $$\fr (-x)y = x(-y) = -(xy).$$}%

{\bf Beweis:} 1)
$$\displaylines{[-\Xi_1,\> -\Xi_2][\Eta_1,\> \Eta_2] = [(-\Xi_1)\Eta_1 - (-\Xi_2)\Eta_2,\> (-\Xi_1)\Eta_2 + (-\Xi_2)\Eta_1]\cr
= [-(\Xi_1\Eta_1) + \Xi_2\Eta_2,\> -(\Xi_1\Eta_2) - \Xi_2\Eta_1]\cr
= [-(\Xi_1\Eta_1 - \Xi_2\Eta_2),\> -(\Xi_1\Eta_2 + \Xi_2\Eta_1)]\cr
= -([\Xi_1,\> \Xi_2][\Eta_1,\> \Eta_2]),\cr}$$
$$\fr (-x)y = -(xy).$$

2) Nach 1) ist
$$\fr x(-y) = (-y)x = -(yx) = -(xy).$$
\medskip

%\jutting{5}{409-410}

{\bf Satz 225:} {\it $$\fr (-x)(-y) = xy.$$}%

{\bf Beweis:} Nach Satz 224 ist
$$\fr (-x)(-y) = x\bigl(-(-y)\bigr) = xy.$$
\medskip

%\jutting{5}{411-444}

{\bf Satz 226} (assoziatives Gesetz der Multiplikation):
{\it $$\fr (xy)z = x(yz).$$}%

{\bf Beweis:} In diesem Beweise werde der \"Ubersichtlichkeit wegen
ausnahmsweise zur Abk\"urzung
$$(\Xi + \Eta) + \Zeta = \Xi + \Eta + \Zeta,$$
$$(\Xi\Eta)\Zeta = \Xi\Eta\Zeta$$
gesetzt, so da{\ss} auch
$$\Xi + (\Eta + \Zeta) = \Xi + \Eta + \Zeta,$$
$$\Xi(\Eta\Zeta) = \Xi\Eta\Zeta$$
ist.

Es werde
$${\fr x} = [\Xi_1,\> \Xi_2],\quad {\fr y} = [\Eta_1,\> \Eta_2],\quad {\fr z} = [\Zeta_1,\> \Zeta_2]$$
gesetzt. Dann ist
$$\displaylines{{\fr (xy)z} = [\Xi_1\Eta_1 - \Xi_2\Eta_2,\> \Xi_1\Eta_2 + \Xi_2\Eta_1][\Zeta_1,\> \Zeta_2]\cr
= [(\Xi_1\Eta_1 - \Xi_2\Eta_2)\Zeta_1 - (\Xi_1\Eta_2 + \Xi_2\Eta_1)\Zeta_2,\cr
(\Xi_1\Eta_1 - \Xi_2\Eta_2)\Zeta_2 + (\Xi_1\Eta_2 + \Xi_2\Eta_1)\Zeta_1]\cr
= [(\Xi_1\Eta_1\Zeta_1 - \Xi_2\Eta_2\Zeta_1) - (\Xi_1\Eta_2\Zeta_2 + \Xi_2\Eta_1\Zeta_2),\cr
(\Xi_1\Eta_1\Zeta_2 - \Xi_2\Eta_2\Zeta_2) + (\Xi_1\Eta_2\Zeta_1 + \Xi_2\Eta_1\Zeta_1)]\cr
= [\Bigl(\Xi_1\Eta_1\Zeta_1 + \bigl(-(\Xi_2\Eta_2\Zeta_1)\bigr)\Bigr) + \bigl(-(\Xi_1\Eta_2\Zeta_2 + \Xi_2\Eta_1\Zeta_2)\bigr),\cr
(\Xi_1\Eta_2\Zeta_1 + \Xi_2\Eta_1\Zeta_1) + \Bigl(\Xi_1\Eta_1\Zeta_2 + \bigl(-(\Xi_2\Eta_2\Zeta_2)\bigr)\Bigr)]\cr
= [\Xi_1\Eta_1\Zeta_1 - (\Xi_2\Eta_2\Zeta_1 + \Xi_1\Eta_2\Zeta_2 + \Xi_2\Eta_1\Zeta_2),\cr
(\Xi_1\Eta_2\Zeta_1 + \Xi_2\Eta_1\Zeta_1 + \Xi_1\Eta_1\Zeta_2) - \Xi_2\Eta_2\Zeta_2].\cr}$$

Wegen
$$\fr x(yz) = (yz)x$$
entsteht durch Buchstabenvertauschung ($\Eta$ statt $\Xi$, $\Zeta$ statt $\Eta$, $\Xi$
statt $\Zeta$)
$$\displaylines{{\fr x(yz)} = [\Eta_1\Zeta_1\Xi_1 - (\Eta_2\Zeta_2\Xi_1 + \Eta_1\Zeta_2\Xi_2 + \Eta_2\Zeta_1\Xi_2),\cr
(\Eta_1\Zeta_2\Xi_1 + \Eta_2\Zeta_1\Xi_1 + \Eta_1\Zeta_1\Xi_2) - \Eta_2\Zeta_2\Xi_2].\cr}$$

Wegen
$$\Xi\Eta\Zeta = \Xi(\Eta\Zeta) = (\Eta\Zeta)\Xi = \Eta\Zeta\Xi,$$
$$\Alpha + \Beta + \Gamma = \Alpha + (\Beta + \Gamma) = (\Beta + \Gamma) + \Alpha = \Beta + \Gamma + \Alpha$$
erkennt man an den ausgerechneten Ausdr\"ucken
$$\fr (xy)z = x(yz).$$
\medskip

%\jutting{5}{445-457}

{\bf Satz 227} (distributives Gesetz):
$$\fr x(y + z) = xy + xz.$$

{\bf Beweis:}
$$\displaylines{[\Xi_1,\> \Xi_2]([\Eta_1,\> \Eta_2] + [\Zeta_1,\> \Zeta_2]) = [\Xi_1,\> \Xi_2][\Eta_1 + \Zeta_1,\> \Eta_2 + \Zeta_2]\cr
= [\Xi_1(\Eta_1 + \Zeta_1) - \Xi_2(\Eta_2 + \Zeta_2),\> \Xi_1(\Eta_2 + \Zeta_2) + \Xi_2(\Eta_1 + \Zeta_1)]\cr
= [(\Xi_1\Eta_1 + \Xi_1\Zeta_1) + \Bigl(-(\Xi_2\Eta_2) + \bigl(-(\Xi_2\Zeta_2)\bigr)\Bigr),\> (\Xi_1\Eta_2 + \Xi_1\Zeta_2) + (\Xi_2\Eta_1 + \Xi_2\Zeta_1)]\cr
= [(\Xi_1\Eta_1 - \Xi_2\Eta_2) + (\Xi_1\Zeta_1 - \Xi_2\Zeta_2),\> (\Xi_1\Eta_2 + \Xi_2\Eta_1) + (\Xi_1\Zeta_2 + \Xi_2\Zeta_1)]\cr
= [\Xi_1\Eta_1 - \Xi_2\Eta_2,\> \Xi_1\Eta_2 + \Xi_2\Eta_1] + [\Xi_1\Zeta_1 - \Xi_2\Zeta_2,\> \Xi_1\Zeta_2 + \Xi_2\Zeta_1]\cr
= [\Xi_1,\> \Xi_2][\Eta_1,\> \Eta_2] + [\Xi_1,\> \Xi_2][\Zeta_1,\> \Zeta_2].\cr}$$
\medskip

%\jutting{5}{458-462}

{\bf Satz 228:} {\it $$\fr x(y - z) = xy - xz.$$}%

{\bf Beweis:} $$\displaylines{\fr x(y - z) = x\bigl(y + (-z)\bigr) = xy + x(-z) = xy + \bigl(-(xz)\bigr)\cr
= \fr xy - xz.\cr}$$
\medskip

%\jutting{5}{463-502}

{\bf Satz 229:} {\it Die Gleichung
$$\fr yu = x,$$
wo $\fr x$, $\fr y$ gegeben sind und
$$\fr y \ne n$$
ist, hat genau eine L\"osung $\fr u$.}

{\bf Beweis:} 1) Es gibt h\"ochstens eine L\"osung; denn aus
$${\fr y}{\fr u}_1 = {\fr x} = {\fr y}{\fr u}_2$$
folgt
$${\fr n} = {\fr y}{\fr u}_1 - {\fr y}{\fr u}_2 = {\fr y}({\fr u}_1 - {\fr u}_2),$$
also nach Satz 221
$${\fr n} = {\fr u}_1 - {\fr u}_2,$$
$${\fr u}_1 = {\fr u}_2.$$

2) Ist
$${\fr y} = [\Eta_1,\> \Eta_2],$$
so ist
$$\Eta = \Eta_1\Eta_1 + \Eta_2\Eta_2 > 0,$$
und
$${\fr u} = \left[{\Eta_1 \over \Eta},\> -{\Eta_2 \over \Eta}\right]{\fr x}$$
ist eine L\"osung wegen
$$\displaylines{{\fr yu} = \left([\Eta_1,\> \Eta_2]\left[{\Eta_1 \over \Eta},\> -{\Eta_2 \over \Eta}\right]\right){\fr x}\cr
= \left[\Eta_1{\Eta_1 \over \Eta} + \Eta_2{\Eta_2 \over \Eta},\> -\left(\Eta_1{\Eta_2 \over \Eta}\right) + \Eta_2{\Eta_1 \over \Eta}\right]{\fr x}\cr
= \left[{\Eta_1\Eta_1 + \Eta_2\Eta_2 \over \Eta},\> {-(\Eta_1\Eta_2) + \Eta_1\Eta_2 \over \Eta}\right]{\fr x} = [1,\> 0]{\fr x} = {\fr ex} = {\fr x}.\cr}$$
\medskip

{\bf Definition 64:} {\it Das $\fr u$ des Satzes 229 ha{\ss}t $\fr x \over y$ {\rm (sprich: $\fr x$ durch $\fr y$).}
$\fr x \over y$ hei{\ss}t auch der Quotient von $\fr x$ durch $\fr y$ oder die durch Division
von $\fr x$ durch $\fr y$ entstehende Zahl.}
\vfill\eject


%\jutting{5}{503-503}

\line{}\vskip 7\baselineskip
\centerline{{\S}~4.}
\medskip

\centerline{\bf Subtraktion.}
\bigskip

{\bf Satz 230:} {\it $$\fr (x - y) + y = x.$$}%

{\bf Beweis:} $$\fr (x - y) + y = y + (x - y) = x.$$
\medskip

%\jutting{5}{504-504}

{\bf Satz 231:} {\it $$\fr (x + y) - y = x.$$}%

{\bf Beweis:} $$\fr y + x = x + y.$$
\medskip

%\jutting{5}{505-505}

{\bf Satz 232:} {\it $$\fr x - (x - y) = y.$$}%

{\bf Beweis:} $$\fr (x - y) + y = x.$$
\medskip

%\jutting{5}{506-510}

{\bf Satz 233:} {\it $$\fr (x - y) - z = x - (y + z).$$}%

{\bf Beweis:} $$\displaylines{\fr (y + z) + \bigl((x - y) - z\bigr) = \bigl((x - y) - z\bigr) + (z + y)\cr
= \fr \Bigl(\bigl((x - y) - z\bigr) + z\Bigr) + y = (x - y) + y = x.\cr}$$
\medskip

%\jutting{5}{511-514}

{\bf Satz 234:} {\it $$\fr (x + y) - z = x + (y - z).$$}%

{\bf Beweis:} $$\fr \bigl(x + (y - z)\bigr) + z = x + \bigl((y - z) + z\bigr) = x + y.$$
\medskip

%\jutting{5}{515-518}

{\bf Satz 235:} {\it $$\fr (x - y) + z = x - (y - z).$$}%

{\bf Beweis:} $$\displaylines{\fr \bigl((x - y) + z\bigr) + (y - z) = (x - y) + \bigl(z + (y - z)\bigr)\cr
= \fr (x - y) + y = x.\cr}$$
\medskip

%\jutting{5}{519-520}

{\bf Satz 236:} {\it $$\fr (x + z) - (y + z) = x - y.$$}%

{\bf Beweis:} $$\fr (x - y) + (y + z) = \bigl((x - y) + y\bigr) + z = x + z.$$
\medskip

%\jutting{5}{521-527}

{\bf Satz 237:} {\it $$\fr (x - y) + (z - u) = (x + z) - (y + u).$$}%

{\bf Beweis:} $$\displaylines{\fr \bigl((x - y) + (z - u)\bigr) + (y + u) = (x - y) + \bigl((z - u) + (u + y)\bigr)\cr
= \fr (x - y) + \Bigl(\bigl((z - u) + u\bigr) + y\Bigr) = (x - y) + (z + y) = (x - y) + (y + z)\cr
= \fr \bigl((x - y) + y\bigr) + z = x + z.\cr}$$
\medskip

%\jutting{5}{528-532}

{\bf Satz 238:} {\it $$\fr (x - y) - (z - u) = (x + u) - (y + z).$$}%

{\bf Beweis:} Nach Satz 237 und Satz 236 ist
$$\displaylines{\fr \bigl((x + u) - (y + z)\bigr) + (z - u) = \bigl((x + u) + z\bigr) - \bigl((y + z) + u\bigr)\cr
= \fr \bigl(x + (u + z)\fr) - \bigl(y + (z + u)\bigr) = x - y.\cr}$$
\medskip

%\jutting{5}{533-542}

{\bf Satz 239:} {\it Es ist
$$\fr x - y = z - u$$
dann und nur dann, wenn
$$\fr x + u = y + z.$$}%

{\bf Beweis:} Satz 213 und Satz 238.
\vfill\eject


%\jutting{5}{543-544}

\line{}\vskip 7\baselineskip
\centerline{{\S}~5.}
\medskip

\centerline{\bf Division.}
\bigskip

{\bf Satz 240:} {\it Ist
$$\fr y \ne n,$$
so ist
$$\fr {x \over y}y = x.$$}%

{\bf Beweis:} $$\fr {x \over y}y = y{x \over y} = x.$$
\medskip

%\jutting{5}{545-545}

{\bf Satz 241:} {\it Ist
$$\fr y \ne n,$$
so ist
$$\fr {xy \over y} = x.$$}%

{\bf Beweis:} $$\fr yx = xy.$$
\medskip

%\jutting{5}{546-548}

{\bf Satz 242:} {\it Ist
$$\fr x \ne n,\quad y \ne n,$$
so ist
$$\fr {x \over {x \over y}} = y.$$}%

{\bf Beweis:} $$\fr {x \over y}y = x.$$
\medskip

%\jutting{5}{549-554}

{\bf Satz 243:} {\it Ist
$$\fr x \ne n,\quad y \ne n,$$
so ist
$$\fr {{x \over y} \over z} = {x \over yz}.$$}%

{\bf Beweis:} $$\fr (yz){{x \over y} \over z} = {{x \over y} \over z}(zy) = \left({{x \over y} \over z}z\right)y = {x \over y}y = x.$$
\medskip

%\jutting{5}{555-559}

{\bf Satz 244:} {\it Ist
$$\fr z \ne n,$$
so ist
$$\fr {xy \over z} = x{y \over z}.$$}%

{\bf Beweis:} $$\fr \left(x{y \over z}\right)z = x\left({y \over z}z\right) = xy.$$
\medskip

%\jutting{5}{560-564}

{\bf Satz 245:} {\it Ist
$$\fr y \ne n,\quad z \ne n,$$
so ist
$$\fr {x \over y}z = {x \over {y \over z}}.$$}%

{\bf Beweis:} $$\fr \left({x \over y}z\right){y \over z} = {x \over y}\left(z{y \over z}\right) = {x \over y}y = x.$$
\medskip

%\jutting{5}{565-566}

{\bf Satz 246:} {\it Ist
$$\fr y \ne n,\quad z \ne n,$$
so ist
$$\fr {xz \over yz} = {x \over y}.$$}%

{\bf Beweis:} $$\fr {x \over y}(yz) = \left({x \over y}y\right)z = xz.$$
\medskip

%\jutting{5}{567-573}

{\bf Satz 247:} {\it Ist
$$\fr y \ne n,\quad u \ne n,$$
so ist
$$\fr {x \over y} \cdot {z \over u} = {xz \over yu}.$$}%

{\bf Beweis:} $$\displaylines{\fr \left({x \over y} \cdot {z \over u}\right)(yu) = {x \over y}\left({z \over u}(uy)\right) = {x \over y}\biggl(\left({z \over u}u\right)y\biggr)\cr
= \fr {x \over y}(zy) = {x \over y}(yz) = \left({x \over y} y\right)z = xz.\cr}$$
\medskip

%\jutting{5}{574-579}

{\bf Satz 248:} {\it Ist
$$\fr y \ne n,\quad z \ne n,\quad u \ne n,$$
so ist
$$\fr {{x \over y} \over {z \over u}} = {xu \over yz}.$$}%

{\bf Beweis:} Nach Satz 247 und Satz 246 ist
$$\fr {xu \over yz} \cdot {z \over u} = {(xu)z \over (yz)u} = {x(uz) \over y(zu)} = {x \over y}.$$
\medskip

%\jutting{5}{580-581}

{\bf Satz 249:} {\it Ist
$$\fr x \ne n,$$
so ist
$$\fr {n \over x} = n.$$}%

{\bf Beweis:} $$\fr xn = n.$$
\medskip

%\jutting{5}{582-582}

{\bf Satz 250:} {\it Ist
$$\fr x \ne n,$$
so ist
$$\fr {x \over x} = e.$$}%

{\bf Beweis:} $$\fr xe = x.$$
\medskip

%\jutting{5}{583-586}

{\bf Satz 251:} {\it Ist
$$\fr y \ne n,$$
so ist
$$\fr {x \over y} = e$$
dann und nur dann, wenn
$$\fr x = y.$$}%

{\bf Beweis:} 1) Ist
$$\fr x = y,$$
so ist nach Satz 250
$$\fr {x \over y} = {y \over y} = e.$$

2) Ist
$$\fr {x \over y} = e,$$
so ist nach Satz 222
$$\fr x = ye = y.$$
\medskip

%\jutting{5}{587-608}

{\bf Satz 252:} {\it Ist
$$\fr y \ne n,\quad u \ne n,$$
so ist
$$\fr {x \over y} = {z \over u}$$
dann und nur dann, wenn
$$\fr xu = yz.$$}%

{\bf Beweis} F\"ur
$$\fr z = n$$
ist die Behauptung klar.

Sonst ist nach Satz 248
$$\fr {{x \over y} \over {z \over u}} = {xu \over yz},$$
so da{\ss} Satz 251 die Behauptung liefert.
\medskip

%\jutting{5}{609-613}

{\bf Satz 253:} {\it Ist
$$\fr y \ne n,$$
so ist
$$\fr {x \over y} + {z \over y} = {x + z \over y}.$$}%

{\bf Beweis:} $$\fr y\left({x \over y} + {z \over y}\right) = y{x \over y} + y{z \over y} = x + z.$$
\medskip

%\jutting{5}{614-615}

{\bf Satz 254:} {\it Ist
$$\fr y \ne n,\quad u \ne n,$$
so ist
$$\fr {x \over y} + {z \over u} = {xu + yz \over yu}.$$}%

{\bf Beweis:} Nach Satz 246 und Satz 253 ist
$$\fr {x \over y} + {z \over u} = {xu \over yu} + {yz \over yu} = {xu + yz \over yu}.$$
\medskip

%\jutting{5}{616-620}

{\bf Satz 255:} {\it Ist
$$\fr y \ne n,$$
so ist
$$\fr {x \over y} - {z \over y} = {x - z \over y}.$$}%

{\bf Beweis:} $$\fr y\left({x \over y} - {z \over y}\right) = y{x \over y} - y{z \over y} = x - z.$$
\medskip

%\jutting{5}{621-622}

{\bf Satz 256:} {\it Ist
$$\fr y \ne n,\quad u \ne n,$$
so ist
$$\fr {x \over y} - {z \over u} = {xu - yz \over yu}.$$}%

{\bf Beweis:} Nach Satz 246 und Satz 255 ist
$$\fr {x \over y} - {z \over u} = {xu \over yu} - {yz \over yu} = {xu - yz \over yu}.$$
\vfill\eject


%\jutting{5}{623-627}

\line{}\vskip 7\baselineskip
\centerline{{\S}~6.}
\medskip

\centerline{\bf Konjugierte Zahlen.}
\bigskip

{\bf Definition 65:} {\it Zu
$${\fr x} = [\Xi_1,\> \Xi_2]$$
hei{\ss}t
$$\overline{\fr x} = [\Xi_1,\> -\Xi_2]$$
konjugiert komplex.}
\medskip

%\jutting{5}{628-628}

{\bf Satz 257:} {\it $$\fr \overline{\overline{x}} = x.$$}%

{\bf Beweis:} $$[\Xi_1,\> -(-\Xi_2)] = [\Xi_1,\> \Xi_2].$$
\medskip

%\jutting{5}{629-641}

{\bf Satz 258:} {\it Es ist
$$\fr \overline{x} = n$$
dann und nur dann, wenn
$$\fr x = n.$$}%

{\bf Beweis:} $$\Xi_1 = 0,\quad -\Xi_2 = 0$$
ist dasselbe wie
$$\Xi_1 = 0,\quad \Xi_2 = 0$$
\medskip

%\jutting{5}{642-652}

{\bf Satz 259:} {\it Es ist
$$\fr \overline{x} = x$$
dann und nur dann, wenn $\fr x$ die Form
$${\fr x} = [\Xi,\> 0]$$
hat.}

{\bf Beweis:} Es ist
$$\Xi_1 = \Xi_1,\quad -\Xi_2 = \Xi_2$$
dann und nur dann, wenn
$$\Xi_2 = 0.$$
\medskip

%\jutting{5}{653-654}

{\bf Satz 260:} {\it $$\fr \overline{x + y} = \overline{x} + \overline{y}.$$}%

{\bf Beweis:} F\"ur
$${\fr x} = [\Xi_1,\> \Xi_2],\quad {\fr y} = [\Eta_1,\> \Eta_2]$$
ist
$$\displaylines{{\fr \overline{x + y}} = [\Xi_1 + \Eta_1,\> -(\Xi_2 + \Eta_2)] = [\Xi_1 + \Eta_1,\> -\Xi_2 + (-\Eta_2)]\cr
= [\Xi_1,\> -\Xi_2] + [\Eta_1,\> -\Eta_2] = {\fr \overline{x} + \overline{y}}.\cr}$$
\medskip

%\jutting{5}{655-659}

{\bf Satz 261:} {\it $$\fr \overline{xy} = \overline{x}\,\overline{y}.$$}%

{\bf Beweis:} F\"ur
$${\fr x} = [\Xi_1,\> \Xi_2],\quad {\fr y} = [\Eta_1,\> \Eta_2]$$
ist
$$\displaylines{{\fr \overline{xy}} = [\Xi_1\Eta_1 - \Xi_2\Eta_2,\> -(\Xi_1\Eta_2 + \Xi_2\Eta_1)]\cr
= [\Xi_1\Eta_1 - (-\Xi_2)(-\Eta_2),\> \Xi_1(-\Eta_2) + (-\Xi_2)\Eta_1)]\cr
= [\Xi_1,\> -\Xi_2][\Eta_1,\> -\Eta_2] = \overline{x}\,\overline{y}.\cr}$$
\medskip

%\jutting{5}{660-665}

{\bf Satz 262:} {\it $$\fr \overline{x - y} = \overline{x} - \overline{y}.$$}%

{\bf Beweis:} Wegen
$$\fr x = (x - y) + y$$
ist nach Satz 260
$$\fr \overline{x} = \overline{x - y} + \overline{y},$$
$$\fr \overline{x - y} = \overline{x} - \overline{y}.$$
\medskip

%\jutting{5}{666-675}

{\bf Satz 263:} {\it F\"ur
$$\fr y \ne n$$
ist
$$\fr \overline{\left({x \over y}\right)} = {\overline{x} \over \overline{y}}.$$}%

{\bf Beweis:} Wegen
$$\fr x = {x \over y}y$$
ist nach Satz 261
$$\fr \overline{x} = \overline{\left({x \over y}\right)}\overline{y};$$
nach Satz 258 ist
$$\fr \overline{y} \ne n,$$
also
$$\fr \overline{\left({x \over y}\right)} = {\overline{x} \over \overline{y}}.$$
\vfill\eject


\line{}\vskip 7\baselineskip
\centerline{{\S}~7.}
\medskip

\centerline{\bf Absoluter Betrag.}
\bigskip

%\jutting{5}{676-678}

{\bf Definition 66:} {\it Es bedeute $\sqrt{\zeta}$ die nach Satz 161 eindeutig
vorhandene {\rm (positive)} L\"osung $\xi$ von
$$\xi\xi = \zeta.$$}%
\medskip

{\bf Definition 67:} {\it $$\sqrt{0} = 0.$$}%
\medskip

{\bf Definition 68:} {\it $$|[\Xi_1,\> \Xi_2]| = \sqrt{\Xi_1\Xi_1 + \Xi_2\Xi_2}.$$
{\rm ($|\ |$ sprich: absoluter Betrag.)}}
\medskip

%\jutting{5}{679-684}

{\bf Satz 264:} {\it $$|{\fr x}| \cases{> 0\quad\hbox{f\"ur}\quad {\fr x \ne n},\cr
= 0\quad\hbox{f\"ur}\quad {\fr x = n}.\cr}$$}%

{\bf Beweis:} Definitionen 68, 66 und 67.
\medskip

%\jutting{5}{685-704}

{\bf Satz 265:} {\it $$|[\Xi_1,\> \Xi_2]| \ge |\Xi_1|,$$
$$|[\Xi_1,\> \Xi_2]| \ge |\Xi_2|.$$}%

{\bf Beweis:} $$|[\Xi_1,\> \Xi_2]|\>|[\Xi_1,\> \Xi_2]|$$
$$= \Xi_1\Xi_1 + \Xi_2\Xi_2 \cases{\ge \Xi_1\Xi_1 = |\Xi_1||\Xi_1|,\cr
\ge \Xi_2\Xi_2 = |\Xi_2||\Xi_2|.\cr}$$

Aus
$$\Xi\Xi \ge \Eta\Eta,\quad \Xi \ge 0,\quad \Eta \ge 0$$
folgt
$$\Xi \ge \Eta,$$
da sonst
$$0 < \Xi < \Eta,$$
$$\Xi\Xi < \Eta\Eta$$
w\"are.  Damit ist Satz 265 bewiesen.
\medskip

%\jutting{5}{705-716}

{\bf Satz 266:} {\it Aus
$$[\Xi,\> 0][\Xi,\> 0] = [\Eta,\> 0][\Eta,\> 0],\quad \Xi \ge 0,\quad \Eta \ge 0$$
folgt
$$\Xi = \Eta.$$}%

{\bf Beweis:} Wegen
$$[\Zeta,\> 0][\Zeta,\> 0] = [\Zeta\Zeta - 0 \cdot 0,\> \Zeta \cdot 0 + 0 \cdot \Zeta] = [\Zeta\Zeta,\> 0]$$
ist nach voraussetzung
$$[\Xi\Xi,\> 0] = [\Eta\Eta,\> 0],$$
$$\Xi\Xi = \Eta\Eta.$$

Ist
$$\Xi > 0,$$
so folgt
$$\Eta\Eta = \Xi\Xi > 0,$$
$$\Eta > 0,$$
also nach Satz 161
$$\Xi = \Eta.$$

Ist
$$\Xi = 0,$$
so folgt
$$\Eta\Eta = \Xi\Xi = 0,$$
$$\Eta = 0 = \Xi.$$
\medskip

%\jutting{5}{717-724}

{\bf Satz 267:} {\it $$[{\fr |x|},\> 0][{\fr |x|},\> 0] = {\fr x\overline{x}}.$$}%

{\bf Beweis:} Wird
$${\fr x} = [\Xi_1,\> \Xi_2]$$
gesetzt, so ist
$$\displaylines{[{\fr |x|},\> 0][{\fr |x|},\> 0] = [{\fr |x||x|},\> 0] = [\Xi_1\Xi_1 + \Xi_2\Xi_2,\> 0]\cr
= [\Xi_1\Xi_1 - \Xi_2(-\Xi_2),\> \Xi_1(-\Xi_2)+\Xi_2\Xi_1] = [\Xi_1,\> \Xi_2][\Xi_1,\> -\Xi_2] = {\fr  x\overline{x}}.\cr}$$
\medskip

%\jutting{5}{725-740}

{\bf Satz 268:} {\it $$\fr |xy| = |x||y|.$$}%

{\bf Beweis:} Nach Satz 267 und Satz 261 ist
$$\displaylines{[{\fr |xy|},\> 0][{\fr |xy|},\> 0] = {\fr (xy)\overline{xy} = (xy)(\overline{x}\,\overline{y}) = (x\overline{x})(y\overline{y})}\cr
= ([{\fr |x|},\> 0][{\fr |x|},\> 0])([{\fr |y|},\> 0][{\fr |y|},\> 0])\cr
= ([{\fr |x|},\> 0][{\fr |y|},\> 0])([{\fr |x|},\> 0][{\fr |y|},\> 0])\cr
= [{\fr |x||y|} - 0 \cdot 0,\> {\fr |x|} \cdot 0 + 0 \cdot {\fr |y|}][{\fr |x||y|} - 0 \cdot 0,\> {\fr |x|} \cdot 0 + 0 \cdot {\fr |y|}]\cr
= [{\fr |x||y|},\> 0][{\fr |x||y|},\> 0],\cr}$$
nach Satz 266 also
$$\fr |xy| = |x||y|.$$
\medskip

%\jutting{5}{741-747}

{\bf Satz 269:} {\it Ist
$$\fr y \ne n,$$
so ist
$$\fr \left|{x \over y}\right| = {|x| \over |y|}.$$}%

{\bf Beweis:} $${\fr |y|} > 0,$$
$$\fr {x \over y}y = x,$$
also nach Satz 268
$$\fr \left|{x \over y}\right||y| = |x|,$$
$$\fr \left|{x \over y}\right| = {|x| \over |y|}.$$
\medskip

%\jutting{5}{748-755}

{\bf Satz 270:} {\it Aus
$$\fr x + y = e$$
folgt
$${\fr |x| + |y|} \ge 1.$$}%

{\bf Beweis:} Ist
$${\fr x} = [\Xi_1,\> \Xi_2],\quad {\fr y} = [\Eta_1,\> \Eta_2],$$
so ist nach Satz 265
$${\fr |x|} \ge |\Xi_1| \ge \Xi_1,$$
$${\fr |y|} \ge |\Eta_1| \ge \Eta_1,$$
also
$${\fr |x| + |y|} \ge \Xi_1 + \Eta_1 = 1.$$
\medskip

%\jutting{5}{756-776}

{\bf Satz 271:} {\it $$\fr |x + y| \le |x| + |y|.$$}%

{\bf Beweis:} 1) Ist
$$\fr x + y = n,$$
so ist die linke Seite der Behauptung $0$, also $\le$ der rechten.

2) Ist
$$\fr x + y \ne n,$$
so ist, wegen
$$\fr {x \over x + y} + {y \over x + y} = {x + y \over x + y} = e,$$
nach Satz 270
$${\fr \left|{x \over x + y}\right| + \left|{y \over x + y}\right|} \ge 1,$$
also nach Satz 269
$${\fr {|x| \over |x + y|} + {|y| \over |x + y|}} \ge 1,$$
$$\fr |x| + |y| = |x + y|\left({|x| \over |x + y|} + {|y| \over |x + y|}\right) \ge |x + y|.$$
\medskip

%\jutting{5}{777-785}

{\bf Satz 272:} {\it $$\fr |-x| = |x|.$$}%

{\bf Beweis:} $$(-\Xi_1)(-\Xi_1) + (-\Xi_2)(-\Xi_2) = \Xi_1\Xi_1 + \Xi_2\Xi_2.$$
\medskip

%\jutting{5}{786-801}

{\bf Satz 273:} {\it $$\fr |x - y| \ge \bigl||x| - |y|\bigr|.$$}%

{\bf Beweis:} $$\fr x = y + (x - y),$$
also nach Satz 271
$$\fr |x| \le |y| + |x - y|,$$
$$\fr |x - y| \ge |x| - |y|.$$
Hieraus folgt, wenn $\fr x$ und $\fr y$ vertauscht werden,
$$\fr |y - x| \ge |y| - |x|,$$
also nach Satz 272
$$\fr |x - y| = |-(y - x)| = |y - x| \ge |y| - |x| = -(|x| - |y|).$$

Aus
$$\Xi \ge \Eta,\quad \Xi \ge -\Eta$$
folgt aber, da $|\Eta|$ entweder $\Eta$ oder $-\Eta$ ist,
$$\Xi \ge |\Eta|.$$
Daher ist
$$\fr |x - y| \ge \bigl||x| - |y|\bigr|.$$
\vfill\eject


%\jutting{5}{802-911}

\line{}\vskip 7\baselineskip
\centerline{{\S}~8.}
\medskip

\centerline{\bf Summen und Produkte.}
\bigskip

{\bf Satz 274:} {\it Ist
$$x < y,$$
so konnen die $m \le x$ nicht auf die $n \le y$ ein-eindeutig bezogen werden.}

Unter Beziehen verstehe ich in diesem Paragraphen immer
ein-eindeutiges Beziehen.

{\bf Beweis:} Es sei $\fr M$ die Menge der $x$, f\"ur die die Behauptung
bei allen $y > x$ wahr ist.

I) Ist
$$1 < y,$$
so kann $m = 1$ nicht auf die $n \le y$ bezogen werden; denn ent%
spricht dem $m = 1$ das $n = 1$, so bleibt kein $m$ f\"ur $n = y$ \"ubrig;
ist $m = 1$ auf ein $n > 1$ bezogen, so bleibt kein $m$ f\"ur $n = 1$
\"ubrig.

1 geh\"ort also zu $\fr M$.

II) Es geh\"ore $x$ zu $\fr M$, und es sei
$$x + 1 < y.$$
Wenn eine Beziehung der $m \le x + 1$ auf die $n \le y$ vorliegt, so
unterscheiden wir zwei F\"alle.

$\alpha$) Dem $m = x + 1$ entspricht $n = y$.  Dann sind die $m \le x$
auf die $n \le y - 1$ bezogen; das geht nicht wegen
$$x < y - 1.$$

$\beta$) Dem $m = x + 1$ entspricht ein $n = n_0 < y$. Dann sei $m = m_0$
die dem $n = y$ entsprechende Zahl, also $m_0 < x + 1$. Man be%
trachte nun folgende abge\"anderte Beziehung der $m \le x + 1$ auf
die $n \le y$.
$$\cases{\hbox{\rm Ist $m \ne m_0$, $m \ne x + 1$, so gelte das Alte.}\cr
\hbox{\rm $m = m_0$ entspreche $n = n_0$.}\cr
\hbox{\rm $m = x + 1$ entspreche $n = y$.}\cr}$$

Dann haben wir eine Beziehung von der soeben in $\alpha$) als unm\"og%
lich nachgewiesenen Art.

Also geh\"ort $x + 1$ zu $\fr M$, und die Behauptung ist bewiesen.
\bigskip

Da die Beweise der folgenden S\"atze 275 bis 278 und 280 bis
286 nebst zugeh\"origen Definitionen f\"ur Summen und Produkte
w\"ortlich dieselben w\"aren, machen wir das, um lange Wieder%
holungen zu vermeiden, nur einmal und w\"ahlen ein neutrales
Zeichen $\oplus$, welches durchweg $+$ oder durchweg $\cdot$ bedeuten soll.
Das einstweilen neutrale Zeichen $\coprod$ wird sp\"ater entsprechend in
zwei Zeichen ($\sum$ bei $+$, $\prod$ bei $\cdot$) gespalten werden.

Unter definiert verstehe ich in dieser ganzen Entwicklung:
als komplexe Zahl definiert.
\medskip

%\jutting{5}{912-1105}

{\bf Satz 275:} {\it Es sei $x$ fest, ${\fr f}(n)$ f\"ur $n \le x$ definiert.  Dann gibt
es genau ein f\"ur $n \le x$ definiertes
$${\fr g}_x(n)$$
{\rm (ausf\"uhrlicher geschrieben
$${\fr g}_{x,{\fr f}}(n)$$
abgek\"urzt geschrieben
$${\fr g}(n))$$}
mit folgenden Eigenschaften:
$${\fr g}_x(1) = {\fr f}(1),$$
$${\fr g}_x(n + 1) = {\fr g}_x(n) \oplus {\fr f}(n + 1)\quad\hbox{\rm f\"ur}\quad n < x.$$}%

{\bf Beweis:} 1) Zun\"achst zeigen wir, da{\ss} es h\"ochstens ein solches
${\fr g}_x(n)$ gibt.

Es m\"ogen ${\fr g}(n)$ und ${\fr h}(n)$ die geforderten Eigenschaften haben.
$\fr M$ sei die aus den $n \le x$ mit
$${\fr g}(n) = {\fr h}(n)$$
und den $n > x$ bestehende Menge.

I) $${\fr g}(1) = {\fr f}(1) = {\fr h}(1);$$
$1$ geh\"ort also zu $\fr M$.

II) $n$ geh\"ore zu $\fr M$.  Dann ist

entweder
$$n < x,\quad {\fr g}(n) = {\fr h}(n),$$
also
$${\fr g}(n + 1) = {\fr g}(n) \oplus {\fr f}(n + 1) = {\fr h}(n) \oplus {\fr f}(n + 1) = {\fr h}(n + 1),$$
also $n + 1$ zu $\fr M$ geh\"orig;

oder
$$n \ge x,$$
also
$$n + 1 > x$$
und $n + 1$ auch zu $\fr M$ geh\"orig.

Daher ist $\fr M$ die Menge aller positiven ganzen Zahlen: f\"ur
jedes $n \le x$ ist also
$${\fr g}(n) = {\fr h}(n),$$
w. z. b. w.

2) Wir zeigen jetzt, da{\ss} es zu jedem $x$, wenn ${\fr f}(n)$ f\"ur $n \le x$
definiert ist, ein passendes ${\fr g}_x(n)$ gibt.

$\fr M$ sei die Menge der $x$, f\"ur die dies wahr ist, f\"ur die es also,
wenn ${\fr f}(n)$ f\"ur $n \le x$ definiert ist, nach 1) genau ein passendes
${\fr g}_x(n)$ gibt.

I) F\"ur $x = 1$ leistet, wenn ${\fr f}(1)$ definiert ist.
$${\fr g}_x(1) = {\fr f}(1)$$
das Gew\"unschte (da die zweite Forderung wegen der Unm\"oglich%
keit von $n < 1$ nicht erhoben wird).  $1$ geh\"ort also zu $\fr M$.

II) Es sei $x$ zu $\fr M$ geh\"orig.  Wenn ${\fr f}(n)$ f\"ur $n \le x + 1$ definiert
ist, ist es f\"ur $n \le x$ definiert, also hier genau ein zugeh\"origes ${\fr g}_x(n)$
vorhanden.  Nun leistet
$${\fr g}_{x + 1}(n) = \cases{{\fr g}_x(n) \quad\hbox{f\"ur}\quad n \le x,\cr
{\fr g}_x(n) \oplus {\fr f}(n + 1) \quad\hbox{f\"ur}\quad n = x + 1\cr}$$
das Gew\"unschte bei $x + 1$.  Denn erstens ist
$${\fr g}_{x + 1}(1) = {\fr g}_x(1) = {\fr f}(1).$$
Zweitens gilt f\"ur
$$n < x$$
(wegen $n + 1 \le x$)
$${\fr g}_{x + 1}(n + 1) = {\fr g}_x(n + 1) = {\fr g}_x(n) \oplus {\fr f}(n + 1) = {\fr g}_{x + 1}(n) \oplus {\fr f}(n + 1),$$
w\"ahrend f\"ur
$$n = x$$
$${\fr g}_{x + 1}(n + 1) = {\fr g}_x(n) \oplus {\fr f}(n + 1) = {\fr g}_{x + 1}(n) \oplus {\fr f}(n + 1)$$
ist; aus
$$n < x + 1$$
folgt also jedenfalls
$${\fr g}_{x + 1}(n + 1) = {\fr g}_{x + 1}(n) \oplus {\fr f}(n + 1).$$

Daher geh\"ort $x + 1$ zu $\fr M$, und $\fr M$ umfa{\ss}t alle positiven ganzen
Zahlen.
\medskip

%\jutting{5}{1106-1140}

{\bf Satz 276:} {\it Wenn ${\fr f}(n)$ f\"ur $n \le x + 1$ definiert ist, gilt f\"ur die
zugeh\"origen ${\fr g}_x(n)$ und ${\fr g}_{x + 1}(n)$
$${\fr g}_{x + 1}(x + 1) = {\fr g}_x(x) \oplus {\fr f}(x + 1).$$}%

{\bf Beweis:} Das kam bei der Konstruktion in 2), II) des vorigen
Beweises vor.
\medskip

%\jutting{5}{1141-1145}

{\bf Definition 69:} {\it Ist ${\fr f}(n)$ f\"ur $n \le x$ definiert, so ist
$$\coprod_{n = 1}^x {\fr f}(n) = {\fr g}_x(x) \quad (= {\fr g}_{x,{\fr f}}(x)).$$

Wenn $\oplus$ die Bedeutung $+$ hat, schreibt man
$$\sum_{n = 1}^x {\fr f}(n);$$
wenn $\oplus$ die Bedeutung $\cdot$ hat, schreibt man
$$\prod_{n = 1}^x {\fr f}(n);$$
{\rm ($\sum$ sprich: Summe; $\prod$ sprich: Produkt.)}}

Statt $n$ kann in diesen Zeichen auch jeder andere Buchstabe
stehen, der positive ganze Zahlen bezeichnet.
\medskip

%\jutting{5}{1146-1150}

{\bf Satz 277:} {\it Ist ${\fr f}(1)$ definiert, so ist
$$\coprod_{n = 1}^1 {\fr f}(n) = {\fr f}(1).$$}%

{\bf Beweis:} $${\fr g}_1(1) = {\fr f}(1).$$
\medskip

%\jutting{5}{1151-1152}

{\bf Satz 278:} {\it Ist ${\fr f}(n)$ f\"ur $n \le x + 1$ definiert, so ist
$$\coprod_{n = 1}^{x + 1}{\fr f}(n) = \coprod_{n = 1}^{x}{\fr f}(n) \oplus {\fr f}(x + 1).$$}%

{\bf Beweis:} Satz 276.
\medskip

%\jutting{5}{1153-1180}

{\bf Satz 279:} {\it $$\sum_{n = 1}^x {\fr x} = {\fr x}[x,\> 0].$$}%

{\bf Beweis:} $\fr x$ sei fest, $\fr M$ die Menge der $x$, f\"ur die dies gilt.

I) Nach Satz 277 ist
$$\sum_{n = 1}^1{\fr x} = {\fr x} = {\fr xe} = {\fr x}[1,\> 0].$$
$1$ geh\"ort also zu $\fr M$.

II) Wenn $x$ zu $\fr M$ geh\"ort, so folgt aus Satz 278
$$\displaylines{\sum_{n = 1}^{x + 1}{\fr x} = \sum_{n = 1}^x{\fr x} + {\fr x} = {\fr x}[x,\> 0] + {\fr x}[1,\> 0] = {\fr x}([x,\> 0] + [1,\> 0])\cr
= {\fr x}[x + 1,\> 0].\cr}$$
$x + 1$ geh\"ort also zu $\fr M$.

Daher gilt die Behauptung f\"ur alle $x$.
\medskip

%\jutting{5}{1181-1193}

{\bf Satz 280:} {\it Sind ${\fr f}(1)$ und ${\fr f}(1 + 1)$ definiert, so ist
$$\coprod_{n = 1}^{1 + 1}{\fr f}(n) = {\fr f}(1) \oplus {\fr f}(1 + 1).$$}%

{\bf Beweis:} Nach Satz 278 und Satz 277 ist
$$\coprod_{n = 1}^{1 + 1}{\fr f}(n) = \coprod_{n = 1}^1{\fr f}(n) \oplus {\fr f}(1 + 1) = {\fr f}(1) \oplus {\fr f}(1 + 1).$$
\medskip

%\jutting{5}{1194-1282}

{\bf Satz 281:} {\it Ist ${\fr f}(n)$ f\"ur $n \le x + y$ definiert, so ist
$$\coprod_{n = 1}^{x + y}{\fr f}(n) = \coprod_{n = 1}^x{\fr f}(n) \oplus \coprod_{n = 1}^y{\fr f}(x + n).$$}%

{\bf Beweis:} Bei festem $x$ sei $\fr M$ die Menge der $y$, f\"ur die dies gilt.

I) Ist ${\fr f}(n)$ f\"ur $n \le x + 1$ definiert, so ist nach Satz 278 und
Satz 277
$$\coprod_{n = 1}^{x + 1}{\fr f}(n) = \coprod_{n = 1}^x{\fr f}(n) \oplus {\fr f}(x + 1) = \coprod_{n = 1}^x{\fr f}(n) \oplus \coprod_{n = 1}^1{\fr f}(x + n).$$
$1$ geh\"ort also zu $\fr M$.

II) $y$ geh\"ore zu $\fr M$.  Wenn ${\fr f}(n)$ f\"ur $n \le x + (y + 1)$ definiert
ist, so ist nach Satz 278 (auf $x + y$ statt $x$ angewendet)
$$\displaylines{\coprod_{n = 1}^{x + (y + 1)}{\fr f}(n) = \coprod_{n = 1}^{(x + y) + 1}{\fr f}(n) = \coprod_{n = 1}^{x + y}{\fr f}(n) \oplus {\fr f}\bigl((x + y) + 1\bigr)\cr
= \left(\coprod_{n = 1}^x{\fr f}(n) \oplus \coprod_{n = 1}^y{\fr f}(x + n)\right) \oplus {\fr f}\bigl(x + (y + 1)\bigr)\cr
= \coprod_{n = 1}^x{\fr f}(n) \oplus \left(\coprod_{n = 1}^y{\fr f}(x + n) \oplus {\fr f}\bigl(x + (y + 1)\bigr)\right),\cr}$$
also nach Satz 278 (auf $y$ statt $x$, ${\fr f}(x + n)$ statt ${\fr f}(n)$ angewendet)
$$= \coprod_{n = 1}^x{\fr f}(n) \oplus \coprod_{n = 1}^{y + 1}{\fr f}(x + n).$$
$y + 1$ geh\"ort also zu $\fr M$, und der Satz ist bewiesen.
\medskip

%\jutting{5}{1283-1330}

{\bf Satz 282:} {\it Sind ${\fr f}(n)$ und ${\fr g}(n)$ f\"ur $n \le x$ definiert, so ist
$$\coprod_{n = 1}^x\bigl({\fr f}(n) \oplus {\fr g}(n)\bigr) = \coprod_{n = 1}^x{\fr f}(n) \oplus \coprod_{n = 1}^x{\fr g}(n).$$}%

{\bf Beweis:} $\fr M$ sei die Menge der $x$, f\"ur die dies gilt.

I) Sind ${\fr f}(1)$ und ${\fr g}(1)$ definiert, so ist
$$\coprod_{n = 1}^1\bigl({\fr f}(n) \oplus {\fr g}(n)\bigr) = {\fr f}(1) \oplus {\fr g}(1) = \coprod_{n = 1}^1{\fr f}(n) \oplus \coprod_{n = 1}^1{\fr g}(n).$$
$1$ geh\"ort also zu $\fr M$.

II) $x$ geh\"ore zu $\fr M$. Sind ${\fr f}(n)$ und ${\fr g}(n)$ f\"ur $n \le x + 1$ definiert,
so ist, mit R\"ucksicht auf
$$\displaylines{\fr (x \oplus y) \oplus (z \oplus u) = \bigl((x \oplus y) \oplus z\bigr) \oplus u = \bigl(z \oplus (x \oplus y)\bigr) \oplus u\cr
\fr = \bigl((z \oplus x) \oplus y\bigr) \oplus u = (z \oplus x) \oplus (y \oplus u) = (x \oplus z) \oplus (y \oplus u),\cr}$$
$$\displaylines{\coprod_{n = 1}^{x + 1}\bigl({\fr f}(n) \oplus {\fr g}(n)\bigr) = \coprod_{n = 1}^x\bigl({\fr f}(n) \oplus {\fr g}(n)\bigr) \oplus \bigl({\fr f}(x + 1) \oplus {\fr g}(x + 1)\bigr)\cr
= \left(\coprod_{n = 1}^x{\fr f}(n) \oplus \coprod_{n = 1}^x{\fr g}(n)\right) \oplus \bigl({\fr f}(x + 1) \oplus {\fr g}(x + 1)\bigr)\cr
= \left(\coprod_{n = 1}^x{\fr f}(n) \oplus {\fr f}(x + 1)\right) \oplus \left(\coprod_{n = 1}^x{\fr g}(n) \oplus {\fr g}(x + 1)\right)\cr
= \coprod_{n = 1}^{x + 1}{\fr f}(n) \oplus \coprod_{n = 1}^{x + 1}{\fr g}(n).\cr}$$
Also geh\"ort $x + 1$ zu $\fr M$, und die Behauptung gilt stets.
\medskip

%\jutting{5}{1331-1670}

{\bf Satz 283:} {\it $s(n)$ beziehe die $n \le x$ auf die $m \le x$.  ${\fr f}(n)$ sei f\"ur
$n \le x$ definiert.  Dann ist
$$\coprod_{n = 1}^x{\fr f}\bigl(s(n)\bigr) = \coprod_{n = 1}^x{\fr f}(n).$$}%

{\bf Beweis:} Zur Abk\"urzung werde
$${\fr f}\bigl(s(n)\bigr) = {\fr g}(n)$$
gesetzt.

$\fr M$ sei die Menge der $x$, f\"ur die die Behauptung
$$\coprod_{n = 1}^x{\fr g}(n) = \coprod_{n = 1}^x{\fr f}(n)$$
(bei allen zul\"assigen $s$ und $\fr f$) wahr ist.

I) F\"ur
$$x = 1$$
ist
$$s(1) = 1,$$
also, wenn ${\fr f}(1)$ definiert ist,
$$\coprod_{n = 1}^x{\fr g}(n) = {\fr g}(1) = {\fr f}(1) = \coprod_{n = 1}^x{\fr f}(n).$$
$1$ geh\"ort also zu $\fr M$.

II) $x$ geh\"ore zu $\fr M$.  Es beziehe $s(n)$ die $n \le x + 1$ auf die
$m \le x + 1$, und ${\fr f}(n)$ sei f\"ur $n \le x + 1$ definiert.

1) Falls
$$s(x + 1) = x + 1,$$
bezieht $s(n)$ die $n \le x$ auf die $m \le x$.  Alsdann ist
$$\coprod_{n = 1}^x{\fr g}(n) = {\fr g}(1) = {\fr f}(1) = \coprod_{n = 1}^x{\fr f}(n),$$
$${\fr g}(x + 1) = {\fr f}(x + 1),$$
also
$$\coprod_{n = 1}^{x + 1}{\fr g}(n) = \coprod_{n = 1}^x{\fr g}(n) \oplus {\fr g}(x + 1) = \coprod_{n = 1}^x{\fr f}(n) \oplus {\fr f}(x + 1) = \coprod_{n = 1}^{x + 1}{\fr f}(n).$$

2) Falls
$$s(x + 1) < x + 1,\quad s(1) = 1,$$
bezieht $s(n)$ die $n$ mit $1 + 1 \le n \le x + 1$ auf die $m$ mit $1 + 1 \le m
\le x + 1$; also bezieht $s(1 + n) - 1$ die $n \le x$ auf die $m \le x$.
Daher ist
$$\displaylines{\coprod_{n = 1}^x{\fr g}(1 + n) = \coprod_{n = 1}^x{\fr f}\bigl(s(1 + n)\bigr) = \coprod_{n = 1}^x{\fr f}\Bigl(1 + \bigl(s(1 + n) - 1\bigr)\Bigr)\cr
= \coprod_{n = 1}^x{\fr f}(1 + n),\cr}$$
also nach Satz 281
$$\coprod_{n = 1}^{x + 1}{\fr g}(n) = {\fr g}(1) \oplus \coprod_{n = 1}^x{\fr g}(1 + n) = {\fr f}(1) \oplus \coprod_{n = 1}^x{\fr f}(1 + n) = \coprod_{n = 1}^{x + 1}{\fr f}(n).$$

3) Falls
$$s(x + 1) < x + 1,\quad s(1) > 1,$$
werde
$$s(1) = a$$
gesetzt und $b$ aus
$$1 \le b \le x + 1,\quad s(b) = 1$$
bestimmt.  Dann ist
$$a > 1,\quad b > 1.$$

$\alpha$) Es sei
$$a < x + 1.$$
Dann bezieht sowohl
$$s_1(n) = \cases{1&f\"ur $n = 1$,\cr
a&f\"ur $n = b$,\cr
s(n)&f\"ur $1 < n \le x +1$, $n \ne b$}$$
als auch
$$s_2(n) = \cases{a&f\"ur $n = 1$,\cr
1&f\"ur $n = a$,\cr
n&f\"ur $1 < n \le x +1$, $n \ne a$}$$
die $n \le x + 1$ auf die $m \le x + 1$.

Nun ist
$$s(n) = s_2\bigl(s_1(n)\bigr) \quad\hbox{\rm f\"ur}\quad n \le x + 1.$$
Denn durch $s_2\bigl(s_1(n)\bigr)$ geht \"uber
$$\hbox{\rm $1$ via $1$ in $a = s(1),$}$$
$$\hbox{\rm $b$ via $a$ in $1 = s(b),$}$$
$$\hbox{\rm jedes andere $n \le x + 1$ via $s(n)$ in $s(n)$.}$$

$s_1(n)$ l\"a{\ss}t $1$, und $s_2(n)$ l\"a{\ss}t $x + 1$ unver\"andert.  Nach 2) und
1) ist also
$$\coprod_{n = 1}^{x + 1}{\fr g}(n) = \coprod_{n = 1}^{x + 1}{\fr f}\bigl(s(n)\bigr) = \coprod_{n = 1}^{x + 1}{\fr f}\Bigl(s_2\bigl(s_1(n)\bigr)\Bigr) = \coprod_{n = 1}^{x + 1}{\fr f}\bigl(s_2(n)\bigr) = \coprod_{n = 1}^{x + 1}{\fr f}(n).$$

$\beta$) Es sei
$$a = x + 1,\quad b < x + 1.$$
Dann bezieht
$$s_3(n) = \cases{b&f\"ur $n = 1$,\cr
1&f\"ur $n = b$,\cr
n&f\"ur $1 < n \le x +1$, $n \ne b$}$$
die $n \le x + 1$ auf die $m \le x + 1$.  Ferner ist
$$s(n) = s_1\bigl(s_3(n)\bigr) \quad\hbox{\rm f\"ur}\quad n \le x + 1.$$
Denn durch $s_1\bigl(s_3(n)\bigr)$ geht \"uber
$$\hbox{\rm $1$ via $b$ in $a = s(1),$}$$
$$\hbox{\rm $b$ via $1$ in $1 = s(b),$}$$
$$\hbox{\rm jedes andere $n \le x + 1$ via $n$ in $s(n)$.}$$

$s_3(n)$ l\"a{\ss}t $x + 1$ unver\"andert.  Nach 1) und 2) ist also
$$\coprod_{n = 1}^{x + 1}{\fr g}(n) = \coprod_{n = 1}^{x + 1}{\fr f}\bigl(s(n)\bigr) = \coprod_{n = 1}^{x + 1}{\fr f}\Bigl(s_1\bigl(s_3(n)\bigr)\Bigr) = \coprod_{n = 1}^{x + 1}{\fr f}\bigl(s_1(n)\bigr) = \coprod_{n = 1}^{x + 1}{\fr f}(n).$$

$\gamma$) Es sei
$$a = b = x + 1.$$

Ist $x = 1$, so ist
$$\coprod_{n = 1}^{x + 1}{\fr g}(n) = \coprod_{n = 1}^{x + 1}{\fr f}(n)$$
trivial.

Ist $x > 1$, so bezieht
$$s_4(n) = \cases{1&f\"ur $n = 1$,\cr
x + 1&f\"ur $n = x + 1$,\cr
s(n)&f\"ur $1 < n < x +1$}$$
die $n \le x + 1$ auf die $m \le x + 1$.  Folglich ist nach 1)
$$\displaylines{\coprod_{n = 1}^{x + 1}{\fr g}(n) = \coprod_{n = 1}^x{\fr g}(n) \oplus {\fr g}(x + 1) = \left({\fr g}(1) \oplus \coprod_{n = 1}^{x - 1}{\fr g}(n + 1)\right) \oplus {\fr g}(x + 1)\cr
= {\fr g}(1) \oplus \left(\coprod_{n = 1}^{x - 1}{\fr g}(n + 1) \oplus {\fr g}(x + 1)\right)\cr
= \left({\fr g}(x + 1) \oplus \coprod_{n = 1}^{x - 1}{\fr g}(n + 1)\right) \oplus {\fr g}(1)\cr
= \left({\fr f}\bigl(s(x + 1)\bigr) \oplus \coprod_{n = 1}^{x - 1}{\fr f}\bigl(s(n + 1)\bigr)\right) \oplus {\fr f}\bigl(s(1)\bigr)\cr
= \left({\fr f}(1) \oplus \coprod_{n = 1}^{x - 1}{\fr f}\bigl(s_4(n + 1)\bigr)\right) \oplus {\fr f}(x + 1)\cr
= \left({\fr f}\bigl(s_4(1)\bigr) \oplus \coprod_{n = 1}^{x - 1}{\fr f}\bigl(s_4(n + 1)\bigr)\right) \oplus {\fr f}\bigl(s_4(x + 1)\bigr)\cr
= \coprod_{n = 1}^x{\fr f}\bigl(s_4(n)\bigr) \oplus {\fr f}\bigl(s_4(x + 1)\bigr) = \coprod_{n = 1}^{x + 1}{\fr f}\bigl(s_4(n)\bigr) = \coprod_{n = 1}^{x + 1}{\fr f}(n).\cr}$$

Daher geh\"ort $x + 1$ zu $\fr M$, und der Satz ist bewiesen.
\bigskip

In Definition 70 und Satz 284 bis Satz 286 bezeichnen aus%
nahmsweise lateinische Buchstaben ganze (nicht notwendig positive)
Zahlen.
\medskip

%\jutting{5}{1671-1687}

{\bf Definition 70:} {\it Es sei
$$y \le x,$$
${\fr f}(n)$ f\"ur
$$y \le n \le x$$
definiert. Dann ist
$$\coprod_{n = y}^x{\fr f}(n) = \coprod_{n = 1}^{(x + 1) - y}{\fr f}\bigl((n + y) - 1\bigr).$$}%

Statt $n$ kann auch irgend ein anderer Buchstabe stehen, der
ganze Zahlen bezeichnet.

Man beachte
$$\hbox{\rm $x + 1 > y$; $y \le (n + y) - 1 \le x$ f\"ur $1 \le n \le (x + 1) - y$;}$$
ferner, da{\ss} f\"ur $y = 1$ die Definition 70 (wie es sein mu{\ss}) im Ein%
klang mit Definition 69 steht.
\medskip

%\jutting{5}{1688-1766}

{\bf Satz 284:} {\it Es sei
$$y \le u < x;$$
${\fr f}(n)$ sei f\"ur
$$y \le n \le x$$
definiert. Dann ist
$$\coprod_{n = y}^x{\fr f}(n) = \coprod_{n = y}^u{\fr f}(n) \oplus \coprod_{n = u + 1}^x{\fr f}(n).$$}%

{\bf Beweis:} Nach Definition 70 und Satz 281 ist
$$\displaylines{\coprod_{n = y}^x{\fr f}(n) = \coprod_{n = 1}^{(x + 1) - y}{\fr f}\bigl((n + y) - 1\bigr)\cr
= \coprod_{n = 1}^{(u + 1) - y}{\fr f}\bigl((n + y) - 1\bigr) \oplus \coprod_{n = 1}^{x - u}{\fr f}\Biggl(\biggl(\Bigl(\bigl((u + 1) - y\bigr) + n\Bigr) + y\biggr) - 1\Biggr);\cr}$$
denn
$$\displaylines{\bigl((u + 1) - y\bigr) + (x - u) = \bigl(x + (-u)\bigr) + \bigl((u + 1) + (-y)\bigr)\cr
= \Bigl(x + \bigl((-u) + (u + 1)\bigr)\Bigr) + (-y) = (x + 1) - y.\cr}$$
Nun ist
$$\displaylines{\Bigl(\bigl((u + 1) - y\bigr) + n\Bigr) + y = \bigl((u + 1) - y\bigr) + (y + n)\cr
= \Bigl(\bigl((u + 1) - y\bigr) + y\Bigr) + n = n + (u + 1),\cr}$$
also nach Definition 70
$$\displaylines{\coprod_{n = y}^x{\fr f}(n) = \coprod_{n = y}^u{\fr f}(n) \oplus \coprod_{n = 1}^{(x + 1) - (u + 1)}{\fr f}\Bigl(\bigl(n + (u + 1)\bigr) - 1\Bigr)\cr
= \coprod_{n = y}^u{\fr f}(n) \oplus \coprod_{n = u + 1}^x{\fr f}(n).\cr}$$
\medskip

%\jutting{5}{1767-1812}

{\bf Satz 285:} {\it Es sei
$$y \le x,$$
${\fr f}(n)$ f\"ur
$$y \le n \le x$$
definiert.  Dann ist
$$\coprod_{n = y}^x{\fr f}(n) = \coprod_{n = y + v}^{x + v}{\fr f}(n - v).$$}%

{\bf Beweis:} Nach Definition 70 ist die linke Seite der Behauptung
$$= \coprod_{n = 1}^{(x + 1) - y}{\fr f}\bigl((n + y) - 1\bigr),$$
die rechte (man beachte $y \le n - v \le x$ f\"ur $y + v \le n \le x + v$)
$$= \coprod_{n = 1}^{((x + v) + 1) - (y + v)}{\fr f}\biggl(\Bigl(\bigl(n + (y + v)\bigr) - 1\Bigr) - v\biggr);$$
hierin ist
$$\displaylines{\bigl((x + v) + 1\bigr) - (y + v) = \bigl(1 + (x + v)\bigr) + \bigl((-v) + (-y)\bigr)\cr
= \Bigl(1 + \bigl((x + v) + (-v)\bigr)\Bigr) + (-y) = (1 + x) - y = (x + 1) - y\cr}$$
und
$$\displaylines{\Bigl(\bigl(n + (y + v)\bigr) - 1\Bigr) - v = \bigl(n + (y + v)\bigr) - (1 + v) = \bigl((n + y) + v\bigr) + \bigl(-v + (-1)\bigr)\cr
= \Bigl(\bigl((n + y) + v\bigr) + (-v)\Bigr) + (-1) = \Bigl((n + y) + \bigl(v + (-v)\bigr)\Bigr) - 1 = (n + y) - 1.\cr}$$
\medskip

%\jutting{5}{1813-1840}

{\bf Satz 286:} {\it Es sei
$$y \le x,$$
${\fr f}(n)$ f\"ur
$$y \le n \le x$$
definiert.  $s(n)$ beziehe die $n$ mit $y \le n \le x$ auf die $m$ mit $y \le m \le x$.
Dann ist
$$\coprod_{n = y}^x{\fr f}\bigl(s(n)\bigr) = \coprod_{n = y}^x{\fr f}(n).$$}%

{\bf Beweis:} $$s_1(n) = s\bigl((n + y) - 1\bigr) - (y - 1)$$
bezieht die positiven $n \le (x + 1) - y$ auf die positiven $m \le (x + 1) - y$.
Daher ist nach Satz 283
$$\displaylines{\coprod_{n = y}^x{\fr f}\bigl(s(n)\bigr) = \coprod_{n = 1}^{(x + 1) - y}{\fr f}\Bigr(s\bigl((n + y) - 1\bigr)\Bigr) = \coprod_{n = 1}^{(x + 1) - y}{\fr f}\bigl(s_1(n) + (y - 1)\bigr)\cr
= \coprod_{n = 1}^{(x + 1) - y}{\fr f}\bigl(n + (y - 1)\bigr) = \coprod_{n = 1}^{(x + 1) - y}{\fr f}\bigl((n + y) - 1\bigr) = \coprod_{n = y}^x{\fr f}(n).\cr}$$
\bigskip

\"Ublich ist statt
$$\sum_{n = y}^x{\fr f}(n)$$
auch die saloppe Schreibweise
$${\fr f}(y) + {\fr f}(y + 1) + \cdots + {\fr f}(x)$$
(und entsprechend beim Produkt); aber v\"ollig einwandfrei ist z. B.
$${\fr f}(1) + {\fr f}(1 + 1) + {\fr f}\bigl((1 + 1) + 1) + {\fr f}\Bigl(\bigl((1 + 1) + 1\bigr) + 1\Bigr),$$
mit anderen Worten
$$\fr a + b + c + d$$
(was also nach Definition auf die alte Addition zur\"uckf\"uhrt und
$$\fr \bigl((a + b) + c\bigr) + d$$
bedeutet). oder z. B.
$$\fr a b c d f g h i k l m o p q r s t u v w x y z.$$

Man kann auch ruhig z. B.
$$\fr a - b + c$$
im Sinne von
$$\fr a + (-b) + c$$
schreiben, da jedenfalls
$${\fr f}(1) + {\fr f}(1 + 1) + {\fr f}\bigl((1 + 1) + 1\bigr)$$
mit
$${\fr f}(1) = {\fr a},\quad {\fr f}(1 + 1) = {\fr -b},\quad {\fr f}\bigl((1 + 1) + 1\bigr) = {\fr c}$$
gemeint ist.
\bigskip

Nunmehr bedeuten kleine lateinische Buchstaben wiederum
positive ganze Zahlen.
\medskip

%\jutting{5}{1841-1882}

{\bf Satz 287:} {\it Ist ${\fr f}(n)$ f\"ur $n \le x$ definiert, so gibt es ein $\Xi$, so da{\ss}
$$\left|\sum_{n = 1}^x{\fr f}(n)\right| \le \Xi,$$
$$\sum_{n = 1}^x[|{\fr f}(n)|,\> 0] = [\Xi,\> 0].$$}%

{\bf Beweis:} $\fr M$ sei die Menge der $x$, f\"ur die es (bei beliebigem
${\fr f}(n)$) ein solches $\Xi$ gibt.

I) Ist ${\fr f}(1)$ definiert, so ist
$$\left|\sum_{n = 1}^1{\fr f}(n)\right| = |{\fr f}(1)|,$$
$$\sum_{n = 1}^1[|{\fr f}(n)|,\> 0] = [|{\fr f}(1)|,\> 0];$$
also leistet
$$\Xi = |{\fr f}(1)|$$
bei $x = 1$ das Gew\"unschte.  $1$ geh\"ort also zu $\fr M$.

II) $x$ geh\"ore zu $\fr M$.  Ist ${\fr f}(n)$ f\"ur $n \le x + 1$ definiert, so gibt
es ein $\Xi_1$ mit
$$\left|\sum_{n = 1}^x{\fr f}(n)\right| \le \Xi_1,$$
$$\sum_{n = 1}^x[|{\fr f}(n)|,\> 0] = [\Xi_1,\> 0].$$
Nach Satz 278 und Satz 271 ist
$$\displaylines{\left|\sum_{n = 1}^{x + 1}{\fr f}(n)\right| = \left|\sum_{n = 1}^x{\fr f}(n) + {\fr f}(x + 1)\right| \le \left|\sum_{n = 1}^x{\fr f}(n)\right| + |{\fr f}(x + 1)|\cr
\le \Xi_1 + |{\fr f}(x + 1)|,\cr}$$
also, wenn
$$\Xi = \Xi_1 + |{\fr f}(x + 1)|$$
gesetzt wird,
$$\left|\sum_{n = 1}^{x + 1}{\fr f}(n)\right| \le \Xi.$$
Andererseits ist nach Satz 278
$$\displaylines{\sum_{n = 1}^{x + 1}[|{\fr f}(n)|,\> 0] = \sum_{n = 1}^x[|{\fr f}(n)|,\> 0] + [|{\fr f}(x + 1)|,\> 0]\cr
= [\Xi_1,\> 0] + [|{\fr f}(x + 1)|,\> 0] = [\Xi_1 + |{\fr f}(x + 1)|,\> 0 + 0] = [\Xi,\> 0].\cr}$$
$\Xi$ leistet also das gew\"unschte bei $x + 1$; also geh\"ort $x + 1$ zu $\fr M$,
und der Satz ist bewiesen.
\medskip

%\jutting{5}{1883-1915}

{\bf Satz 288:} {\it Ist ${\fr f}(n)$ f\"ur $n \le x$ definiert, so ist
$$\left[\left|\prod_{n = 1}^x{\fr f}(n)\right|,\> 0\right] = \prod_{n = 1}^x[|{\fr f}(n)|,\> 0].$$}%

{\bf Beweis:} $\fr M$ sei die Menge der $x$, f\"ur die dies gilt.

I) Ist ${\fr f}(1)$ definiert, so ist
$$\left[\left|\prod_{n = 1}^1{\fr f}(n)\right|,\> 0\right] = [|{\fr f}(1)|,\> 0] = \prod_{n = 1}^1[|{\fr f}(n)|,\> 0].$$
Also geh\"ort $1$ zu $\fr M$.

II) $x$ geh\"ore zu $\fr M$.  Ist ${\fr f}(n)$ f\"ur $n \le x + 1$ definiert, so ist
nach Satz 278 und Satz 268
$$\displaylines{\prod_{n = 1}^{x + 1}[|{\fr f}(n)|,\> 0] = \prod_{n = 1}^x[|{\fr f}(n)|,\> 0] \cdot [|{\fr f}(x + 1)|,\> 0]\cr
= \left[\left|\prod_{n = 1}^x{\fr f}(n)\right|,\> 0\right] \cdot [|{\fr f}(x + 1)|,\> 0]\cr
= \left[\left|\prod_{n = 1}^x{\fr f}(n)\right| \cdot |{\fr f}(x + 1)| - 0 \cdot 0,\> \left|\prod_{n = 1}^x{\fr f}(n)\right| \cdot 0 + 0 \cdot |{\fr f}(x + 1)|\right]\cr
= \left[\left|\prod_{n = 1}^x{\fr f}(n)\right| \cdot |{\fr f}(x + 1)|,\> 0\right] = \left[\left|\prod_{n = 1}^x{\fr f}(n) \cdot {\fr f}(x + 1)\right|,\> 0\right]\cr
= \left[\left|\prod_{n = 1}^{x + 1}{\fr f}(n)\right|,\> 0\right]\cr}$$
also $x + 1$ zu $\fr M$ geh\"orig, und der Satz ist bewiesen.
\medskip

%\jutting{5}{1916-1981}

{\bf Satz 289:} {\it Ist ${\fr f}(n)$ f\"ur $n \le x$ definiert, so ist
$$\prod_{n = 1}^x{\fr f}(n) = {\fr n}$$
dann und nur dann, wenn ein $n \le x$ mit
$${\fr f}(n) = {\fr n}$$
vorhanden ist.}

{\bf Beweis:} {\fr M} sei die Menge der $x$, f\"ur die dies gilt.

I) $$\prod_{n = 1}^1{\fr f}(n) = {\fr n}$$
ist mit
$${\fr f}(1) = {\fr n}$$
identisch.  Also geh\"ort $1$ zu $\fr M$.

II) $x$ geh\"ore zu $\fr M$.
$$\prod_{n = 1}^{x + 1}{\fr f}(n) = {\fr n}$$
bedeutet
$$\prod_{n = 1}^x{\fr f}(n) \cdot {\fr f}(x + 1) = {\fr n};$$
nach Satz 221 ist hierf\"ur notwendig und hinreichend
$$\prod_{n = 1}^x{\fr f}(n) = {\fr n} \quad\hbox{\rm oder}\quad {\fr f}(x + 1) = {\fr n},$$
also (da $x$ zu $\fr M$ geh\"ort) notwendig und hinreichend
$$\hbox{\rm ${\fr f}(n) = {\fr n}$ f\"ur ein $n \le x$ oder f\"ur $n = x + 1$.}$$
$x + 1$ geh\"ort also zu $\fr M$, und der Satz ist bewiesen.
\vfill\eject


%\jutting{5}{1982-2073}

\line{}\vskip 7\baselineskip
\centerline{{\S}~9.}
\medskip

\centerline{\bf Potenzen.}
\bigskip

In diesem Paragraphen m\"ogen kleine lateinische Buchstaben
ganze Zahlen bezeichnen.

{\bf Definition 71:}
{\it $${\fr x}^x = \cases{\prod_{n = 1}^x{\fr x}&f\"ur $x > 0$,\cr
{\fr e}&f\"ur $\fr x \ne n$, $x = 0$,\cr
{\fr e} \over {\fr x}^{|x|}&f\"ur $\fr x \ne n$, $x < 0$.\cr}$$
{\rm (Sprich: $\fr x$ hoch $x$.)}}  Nicht definiert ist also ${\fr x}^x$ lediglich f\"ur
$${\fr x = n,}\quad x \le 0.$$
Man beachte, da{\ss} f\"ur
$${\fr x \ne n,}\quad x < 0.$$
nach der ersten Zeile der Definition 71 und Satz 289
$${\fr x}^{|x|} \ne {\fr n}$$
ist, so da{\ss} dann ${\fr e} \over {\fr x}^{|x|}$, einen Sinn hat.
\medskip

%\jutting{5}{2074-2092}

{\bf Satz 290:} {\it F\"ur
$$\fr x \ne n$$
ist
$${\fr x}^x \ne {\fr n}.$$}%

{\bf Beweis:} F\"ur $x > 0$ folgt dies aus Satz 289, f\"ur $x = 0$ aus
der Definition und f\"ur $x < 0$ aus
$${\fr x}^x {\fr x}^{|x|} \ne {\fr n}.$$
\medskip

%\jutting{5}{2093-2103}

{\bf Satz 291:} {\it $${\fr x}^1 = {\fr x}.$$}%

{\bf Beweis:} $${\fr x}^1 = \prod_{n = 1}^1{\fr x} = {\fr x}.$$
\medskip

%\jutting{5}{2104-2183}

{\bf Satz 292:} {\it Es sei
$$x > 0$$
oder
$$\fr x \ne n,\quad y \ne n.$$
Dann ist
$$({\fr xy})^x = {\fr x}^x {\fr y}^x.$$}%

{\bf Vorbemerkung:} Beide Seiten haben jedenfalls einen Sinn;
denn f\"ur $x \le 0$ ist
$$\fr xy \ne n.$$

{\bf Beweis:} 1) Bei festen $\fr x$, $\fr y$ sei $\fr M$ die Menge der $x > 0$ mit
$$({\fr xy})^x = {\fr x}^x {\fr y}^x.$$

I) Nach Satz 291 ist
$$({\fr xy})^1 = {\fr xy} = {\fr x}^1 {\fr y}^1,$$
also $1$ zu $\fr M$ geh\"orig.

II) Ist $x$ zu $\fr M$ geh\"orig, so ist
$$\displaylines{({\fr xy})^{x + 1} = \prod_{n = 1}^{x + 1}({\fr xy}) = \prod_{n = 1}^x({\fr xy}) \cdot ({\fr xy}) = ({\fr x}^x {\fr y}^x)({\fr xy}) = ({\fr x}^x {\fr x})({\fr y}^x {\fr y})\cr
= \left(\prod_{n = 1}^x{\fr x} \cdot {\fr x}\right)\left(\prod_{n = 1}^x{\fr y} \cdot {\fr y}\right) = \prod_{n = 1}^{x + 1}{\fr x} \cdot \prod_{n = 1}^{x + 1}{\fr y} = {\fr x}^{x + 1} {\fr y}^{x + 1},\cr}$$
also $x + 1$ zu $\fr M$ geh\"orig.

F\"ur $x > 0$ ist also stets
$$({\fr xy})^x = {\fr x}^x {\fr y}^x.$$

2) Es sei
$$x = 0,\quad {\fr x \ne n,\quad y \ne n.}$$
Dann ist
$$({\fr xy})^x = {\fr e} = {\fr ee} = {\fr x}^x {\fr y}^x.$$

3) Es sei
$$x < 0,\quad {\fr x \ne n,\quad y \ne n.}$$
Nach 1) ist
$$({\fr xy})^{|x|} = {\fr x}^{|x|} {\fr y}^{|x|},$$
$${{\fr e} \over ({\fr xy})^{|x|}} = {{\fr e} \over {\fr x}^{|x|} {\fr y}^{|x|}} = {{\fr e} \over {\fr x}^{|x|}} \cdot {{\fr e} \over {\fr y}^{|x|}},$$
$$({\fr xy})^x = {\fr x}^x {\fr y}^x.$$
\medskip

%\jutting{5}{2184-2198}

{\bf Satz 293:} {\it $${\fr e}^x = {\fr e}.$$}%

{\bf Beweis:} Nach Satz 292 ist
$${\fr e}^x {\fr e} = {\fr e} = ({\fr ee})^x = {\fr e}^x {\fr e}^x,$$
$${\fr n} = {\fr e}^x {\fr e}^x - {\fr e}^x {\fr e} = {\fr e}^x ({\fr e}^x - {\fr e}),$$
also (nach Satz 290 und Satz 221)
$${\fr e}^x - {\fr e} = {\fr n},$$
$${\fr e}^x = {\fr e}.$$
\medskip

%\jutting{5}{2199-2352}

{\bf Satz 294:} {\it Es sei
$$x > 0,\quad y > 0$$
oder
$$\fr x \ne n.$$
Dann ist
$${\fr x}^x {\fr x}^y = {\fr x}^{x + y}.$$}%

{\bf Beweis:} 1) Es sei
$$x > 0,\quad y > 0.$$
Dann ist nach Satz 281
$${\fr x}^x {\fr x}^y = \prod_{n = 1}^x{\fr x} \cdot \prod_{n = 1}^y{\fr x} = \prod_{n = 1}^{x + y}{\fr x} = {\fr x}^{x + y}.$$

2) Es sei
$$\fr x \ne n$$
und nicht zugleich
$$x > 0,\quad y > 0.$$

$\alpha$) Es sei
$$x < 0,\quad y < 0.$$
Dann ist nach 1)
$${\fr x}^{|x|} {\fr x}^{|y|} = {\fr x}^{|x| + |y|} = {\fr x}^{|x + y|},$$
$${\fr x}^x {\fr x}^y = {{\fr e} \over {\fr x}^{|x|}} \cdot {{\fr e} \over {\fr x}^{|y|}} = {{\fr e} \over {\fr x}^{|x|} {\fr x}^{|y|}} = {{\fr e} \over {\fr x}^{|x + y|}} = {\fr x}^{x + y}.$$

$\beta$) Es sei
$$x > 0,\quad y < 0.$$
Dann ist
$${\fr x}^x {\fr x}^y = {\fr x}^x {{\fr e} \over {\fr x}^{|y|}} = {{\fr x}^x \over {\fr x}^{|y|}}.$$

A) F\"ur
$$x > |y|$$
ist nach 1)
$${{\fr x}^x \over {\fr x}^{|y|}} = {{\fr x}^{|y|} {\fr x}^{x - |y|} \over {\fr x}^{|y|}} = {\fr x}^{x - |y|} = {\fr x}^{x + y}.$$

B) F\"ur
$$x = |y|$$
ist
$${{\fr x}^x \over {\fr x}^{|y|}} = {\fr e} = {\fr x}^0 = {\fr x}^{x + y}.$$

C) F\"ur
$$x < |y|$$
ist nach 1)
$${{\fr x}^x \over {\fr x}^{|y|}} = {{\fr x}^x {\fr e} \over {\fr x}^x {\fr x}^{|y| - x}} =  {{\fr e} \over {\fr x}^{|y| - x}} = {\fr x}^{x - |y|} = {\fr x}^{x + y}.$$

$\gamma$) Es sei
$$x < 0,\quad y > 0.$$
Dann ist nach $\beta$)
$${\fr x}^x {\fr x}^y = {\fr x}^y {\fr x}^x = {\fr x}^{y + x} = {\fr x}^{x + y}.$$

$\delta$) Es sei
$$x = 0.$$
Dann ist
$${\fr x}^x {\fr x}^y = {\fr e} {\fr x}^y = {\fr x}^y = {\fr x}^{0 + y} = {\fr x}^{x + y}.$$

$\epsilon$) Es sei
$$x \ne 0,\quad y = 0.$$
Dann ist nach $\delta$)
$${\fr x}^x {\fr x}^y = {\fr x}^y {\fr x}^x = {\fr x}^{y + x} = {\fr x}^{x + y}.$$
\medskip

%\jutting{5}{2353-2370}

{\bf Satz 295:} {\it F\"ur
$$\fr x \ne n$$
ist
$${{\fr x}^x \over {\fr x}^y} = {\fr x}^{x - y}.$$}%

{\bf Beweis:} Nach Satz 294 ist
$${\fr x}^{x - y} {\fr x}^y = {\fr x}^{(x - y) + y} = {\fr x}^x;$$
nach Satz 290 ist
$${\fr x}^y \ne {\fr n},$$
also
$${{\fr x}^x \over {\fr x}^y} = {\fr x}^{x - y}.$$
\medskip

%\jutting{5}{2371-2389}

{\bf Satz 296:} {\it F\"ur
$$\fr x \ne n$$
ist
$${{\fr e} \over {\fr x}^x} = {\fr x}^{-x}.$$}%

{\bf Beweis:} Nach Satz 295 ist
$${{\fr e} \over {\fr x}^x} = {{\fr x}^0 \over {\fr x}^x} = {\fr x}^{0 - x} = {\fr x}^{-x}.$$
\medskip

%\jutting{5}{2390-2506}

{\bf Satz 297:} {\it Es sei
$$x > 0,\quad y > 0$$
oder
$$\fr x \ne n.$$
Dann ist
$$\left({\fr x}^x\right)^y = {\fr x}^{xy}.$$}%

{\bf Beweis:} 1) Es sei
$${\fr x = n,}\quad x > 0,\quad y > 0.$$
Dann ist nach Satz 289
$$\left({\fr x}^x\right)^y = \left({\fr n}^x\right)^y = {\fr n}^y = {\fr n} = {\fr n}^{xy} = {\fr x}^{xy}.$$

2) Es sei
$$\fr x \ne n.$$

a) Bei festen $\fr x$, $x$ sei $\fr M$ die Menge der $y > 0$ mit
$$\left({\fr x}^x\right)^y = {\fr x}^{xy}.$$

I) $$\left({\fr x}^x\right)^1 = {\fr x}^x = {\fr x}^{x \cdot 1};$$
$1$ geh\"ort also zu $\fr M$.

II) $y$ geh\"ore zu $\fr M$.  Dann ist nach Satz 294
$$\left({\fr x}^x\right)^{y + 1} = \left({\fr x}^x\right)^y \left({\fr x}^x\right)^1 = {\fr x}^{xy} {\fr x}^x = {\fr x}^{xy + x} = {\fr x}^{x(y + 1)},$$
also $y + 1$ zu $\fr M$ geh\"orig.

F\"ur $y > 0$ ist also die Behauptung wahr.

b) Es sei
$$y = 0.$$
Dann ist
$$\left({\fr x}^x\right)^y = {\fr e} = {\fr x}^{xy}.$$

c) Es sei
$$y < 0.$$
Dann ist nach a)
$$\left({\fr x}^x\right)^{|y|} = {\fr x}^{x|y|},$$
also nach Satz 296 und a)
$$\left({\fr x}^x\right)^y = {{\fr e} \over \left({\fr x}^x\right)^{-y}} = {{\fr e} \over \left({\fr x}^x\right)^{|y|}} = {{\fr e} \over {\fr x}^{x|y|}} = {\fr x}^{-(x|y|)} = {\fr x}^{xy}.$$
\vfill\eject


%\jutting{5}{2507-2555}

\line{}\vskip 7\baselineskip
\centerline{{\S}~10.}
\medskip

\centerline{\bf Einordnung der reellen Zahlen.}
\bigskip

{\bf Satz 298:} {\it $$[\Xi + \Eta,\> 0] = [\Xi,\> 0] + [\Eta,\> 0];$$
$$[\Xi - \Eta,\> 0] = [\Xi,\> 0] - [\Eta,\> 0];$$
$$[\Xi\Eta,\> 0] = [\Xi,\> 0][\Eta,\> 0];$$
$$\left[{\Xi \over \Eta},\> 0\right] = {[\Xi,\> 0] \over [\Eta,\> 0]}, \quad\hbox{\it falls}\quad \Eta \ne 0;$$
$$[-\Xi,\> 0] = -[\Xi,\> 0];$$
$$|[\Xi,\> 0]| = |\Xi|.$$}%

{\bf Beweis:} 1) $$[\Xi,\> 0] + [\Eta,\> 0] = [\Xi + \Eta,\> 0 + 0] = [\Xi + \Eta,\> 0].$$

2) $$[\Xi,\> 0] - [\Eta,\> 0] = [\Xi - \Eta,\> 0 - 0] = [\Xi - \Eta,\> 0].$$

3) $$[\Xi,\> 0][\Eta,\> 0] = [\Xi\Eta - 0 \cdot 0,\> \Xi \cdot 0 + 0 \cdot \Eta] = [\Xi\Eta,\> 0].$$

4) Nach 3) ist, falls $\Eta \ne 0$,
$$[\Eta,\> 0]\left[{\Xi \over \Eta},\> 0\right] = \left[\Eta{\Xi \over \Eta},\> 0\right] = [\Xi,\> 0],$$
$${[\Xi,\> 0] \over [\Eta,\> 0]} = \left[{\Xi \over \Eta},\> 0\right].$$

5) $$-[\Xi,\> 0] = [-\Xi,\> -0] = [-\Xi,\> 0].$$

6) $$|\Xi| = \sqrt{|\Xi||\Xi|} = \sqrt{\Xi\Xi} = \sqrt{\Xi\Xi + 0 \cdot 0} = |[\Xi,\> 0]|.$$
\medskip

%\jutting{5}{2556-2665}

{\bf Satz 299:} {\it Die komplexen Zahlen der Form $[x,\> 0]$ gen\"ugen den
f\"unf Axiomen der nat\"urlichen Zahlen, wenn $[1,\> 0]$ an Stelle von $1$
genommen wird und
$$[x,\> 0]' = [x',\> 0]$$
gesetzt wird.}

{\bf Beweis:} $\fr [Z]$ sei die Menge der $[x,\> 0]$.

1) $[1,\> 0]$ geh\"ort zu $\fr [Z]$.

2) Mit $[x,\> 0]$ ist $[x,\> 0]'$ in $\fr [Z]$ vorhanden.

3) Stets ist
$$x' \ne 1,$$
also
$$[x',\> 0] \ne [1,\> 0],$$
$$[x,\> 0]' \ne [1,\> 0].$$

4) Aus
$$[x,\> 0]' = [y,\> 0]'$$
folgt
$$[x',\> 0] = [y',\> 0],$$
$$x' = y',$$
$$x = y,$$
$$[x,\> 0] = [y,\> 0].$$

5) Eine Menge $\fr [M]$ von Zahlen aus $\fr [Z]$ habe die Eigenschaften:

I) $[1,\> 0]$ geh\"ort zu $\fr [M]$.

II) Falls $[x,\> 0]$ zu $\fr [M]$ geh\"ort, so geh\"ort $[x,\> 0]'$ zu $\fr [M]$.

Dann bezeichne $\fr M$ die Menge der $x$, f\"ur die $[x,\> 0]$ zu $\fr [M]$ ge%
h\"ort.  Alsdann ist $1$ zu $\fr M$ geh\"orig und mit jedem $x$ von $\fr M$ auch
$x'$ zu $\fr M$ geh\"orig.  Also geh\"ort jede positive ganze Zahl $x$ zu
$\fr M$, also jedes $[x,\> 0]$ zu $\fr [M]$.
\bigskip

Da Summe, Differenz, Produkt und (wofern vorhanden) Quo%
tient zweier $[\Xi,\> 0]$ nach Satz 298 den alten Begriffen entsprechen,
desgleichen die Zeichen $-[\Xi,\> 0]$ und $|[\Xi,\> 0]|$; da man
$$[\Xi,\> 0] > [\Eta,\> 0] \quad\hbox{\rm f\"ur}\quad \Xi > \Eta,$$
$$[\Xi,\> 0] < [\Eta,\> 0] \quad\hbox{\rm f\"ur}\quad \Xi < \Eta$$
definieren kann, so haben also die komplexen Zahlen $[\Xi,\> 0]$ alle
Eigenschaften, die wir in Kapitel 4 f\"ur reelle Zahlen bewiesen
haben, und insbesondere die Zahlen $[x,\> 0]$ alle bewiesenen Eigen%
schaften der positiven ganzen Zahlen.

Daher werfen wir die reellen Zahlen weg, ersetzen sie durch
die entsprechenden komplexen Zahlen $[\Xi\> 0]$ und brauchen nur von
komplexen Zahlen zu reden.  (Die reellen Zahlen verbleiben aber
paarweise im Begriff der komplexen Zahl.)
\medskip

{\bf Definition 72:} {\it {\rm (Das freigewordene Zeichen)} $\Xi$ bezeichnet die
komplexe Zahl $[\Xi,\> 0]$, auf die auch das Wort reelle Zahl \"ubergeht.
Ebenso hei{\ss}t jetzt $[\Xi,\> 0]$ bei ganzem $\Xi$ ganze Zahl, bei rationalem $\Xi$
rationale Zahl. bei irrationalem $\Xi$ irrationale Zahl, bei positivem $\Xi$
positive Zahl, bei negativem $\Xi$ negative Zahl.}

Also schreiben wir z. B. $0$ statt $\fr n$, $1$ statt $\fr e$.

Nunmehr k\"onnen wir die komplexen Zahlen mit kleinen oder
gro{\ss}en Buchstaben beliebiger Alphabete (auch promiscue) bezeichnen.
F\"ur die folgende spezielle Zahl ist aber ein kleiner lateinischer
Buchstabe \"ublich auf Grund der
\medskip

%\jutting{5}{2666-2666}

{\bf Definition 73:} {\it $$i = [0,\> 1].$$}%
\medskip

%\jutting{5}{2667-2674}

{\bf Satz 300:} {\it $$ii = -1.$$}%

{\bf Beweis:} $$\displaylines{ii = [0,\> 1][0,\> 1] = [0 \cdot 0 - 1 \cdot 1,\> 0 \cdot 1 + 1 \cdot 0]\cr
= [-1,\> 0] = -1.\cr}$$
\medskip

%\jutting{5}{2675-2706}

{\bf Satz 301:} {\it F\"ur reelle $u_1$, $u_2$ ist
$$u_1 + u_2 i = [u_1,\> u_2].$$
Zu jeder komplexen Zahl $x$ gibt es also genau ein Paar reeller Zahlen
$u_1$, $u_2$ mit
$$x = u_1 + u_2 i.$$}%

{\bf Beweis:} F\"ur reelle $u_1$, $u_2$ ist
$$\displaylines{u_1 + u_2 i = [u_1,\> 0] + [u_2,\> 0][0,\> 1] = [u_1,\> 0] + [u_2 \cdot 0 - 0 \cdot 1,\> u_2 \cdot 1 + 0 \cdot 0]\cr
= [u_1,\> 0] + [0,\> u_2] = [u_1,\> u_2].\cr}$$
\bigskip

Durch Satz 301 ist das Zeichen $[\ ]$ unn\"otig geworden; die
komplexen Zahlen sind eben die Zahlen $u_1 + u_2 i$, wo $u_1$ und $u_2$
reell sind; gleichen bzw. verschiedenen Paaren $u_1$, $u_2$ entsprechen
gleiche bzw. verschiedene Zahlen, und Summe, Differenz, Produkt
zweier komplexer Zahlen $u_1 + u_2 i$, $v_1 + v_2 i$ (wo $u_1$, $u_2$, $v_1$, $v_2$ reell
sind) bildet man nach den Formeln
$$(u_1 + u_2 i) + (v_1 + v_2 i) = (u_1 + v_1) + (u_2 + v_2) i,$$
$$(u_1 + u_2 i) - (v_1 + v_2 i) = (u_1 - v_1) + (u_2 - v_2) i,$$
$$(u_1 + u_2 i)(v_1 + v_2 i) = (u_1 v_1 - u_2 v_2) + (u_1 v_2 + u_2 v_1) i.$$
Man braucht sich nicht einmal diese Formeln zu merken, sondern
nur, da{\ss} die Gesetze der reellen Zahlen erhalten bleiben und
Satz 300 gilt; danach rechnet man einfach so:
$$(u_1 + u_2 i) + (v_1 + v_2 i) = (u_1 + v_1) + (u_2 i + v_2 i) = (u_1 + v_1) + (u_2 + v_2) i,$$
$$(u_1 + u_2 i) - (v_1 + v_2 i) = (u_1 - v_1) + (u_2 i - v_2 i) = (u_1 - v_1) + (u_2 - v_2) i,$$
$$\displaylines{(u_1 + u_2 i)(v_1 + v_2 i) = (u_1 + u_2 i) v_1 + (u_1 + u_2 i) v_2 i =\cr
= u_1 v_1 + u_2 i v_1 + u_1 v_2 i + u_2 i v_2 i\cr
= u_1 v_1 + u_2 v_1 i + u_1 v_2 i + u_2 v_2 i i\cr
= u_1 v_1 + u_2 v_1 i + u_1 v_2 i + u_2 v_2 (-1)\cr
= (u_1 v_1 - u_2 v_2) + (u_1 v_2 + u_2 v_1)i.\cr}$$

Was die Division betrift, so ergibt die Rechnung, wenn $v_1$
und $v_2$ nicht beide $0$ sind,
$$\displaylines{{u_1 + u_2 i \over v_1 + v_2 i} = {(u_1 + u_2 i)(v_1 - v_2 i) \over (v_1 + v_2 i)(v_1 - v_2 i)} = {(u_1 v_1 + u_2 v_2) + \bigl(-(u_1 v_2) + u_2 v_1\bigr) i \over (v_1 v_1 + v_2 v_2) + \bigl(-(v_1 v_2) + v_2 v_1\bigr) i}\cr
= {(u_1 v_1 + u_2 v_2) + \bigl(-(u_1 v_2) + u_2 v_1\bigr) i \over v_1 v_1 + v_2 v_2} = {u_1 v_1 + u_2 v_2 \over v_1 v_1 + v_2 v_2} + {-(u_1 v_2) + u_2 v_1 \over v_1 v_1 + v_2 v_2} i\cr}$$
als kanonische Darstellung im Sinne des Satzes 301.
\vfill\eject


