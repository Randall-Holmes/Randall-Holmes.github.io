






\documentclass[12pt]{article}

\usepackage{amssymb}

\title{Implementation of Zermelo's work of 1908 in Lestrade:  Part I, Logic}

\author{M. Randall Holmes}

\begin{document}

\maketitle

\section{Introduction}
 
This document was originally titled as an essay on the proposition that mathematics is what can be done in Automath (as opposed to what can be done in ZFC, for example).  Such an essay is still in in my mind, but this particular document has transformed itself into the large project of implementing Zermelo's two important set theory papers of 1908 in Lestrade, with the further purpose of exploring the actual capabilities of Zermelo's system of 1908 as a mathematical foundation, which we think are perhaps underrated.

This is a new version of this document in modules, designed to make it possible to work more efficiently without repeated execution of slow log files when they do not need to be revisited.

\section{Introducing Lestrade}

Text in Lestrade is a series of declarations and definitions of entities of various sorts, which fall into two large categories, {\em objects\/} and {\em constructions\/}.

There are five kinds of sorts of object:  there is a sort {\tt prop} of propositions, there is a sort {\tt type} of type labels and there is a sort {\tt obj} of ``untyped mathematical objects" (which we will use as the sort of the sets in our implementation of Zermelo set theory and ZFC).  Then there are two systems of parameterized sorts which can be expanded as the Lestrade text develops:  for each proposition $p$ there is a sort {\tt that} $p$, inhabited by evidence for $p$ (a proof of $p$ is evidence, but we prefer the less definite term for the sort);  for each type label $\tau$ there is a sort {\tt in} $\tau$ (which we may call the type $\tau$).  The types are intended to be inhabited by typed  mathematical objects.

Constructions are ``functions" in the most general sense.  A construction sort is of the form $[x_1:\tau_1,x_2:\tau_2,\ldots,x_n:\tau_n \Rightarrow \tau]$, where the $\tau_i$ of each input parameter $x_i$  is an object or construction sort, $\tau$ is an object sort, and the output sort $\tau$ may depend on the values of any or all of the $x_i$'s, while each $\tau_j$ may depend on any or all $x_i$'s for $i<j$.  In other words, constructions are dependently typed.  We will see this at work in examples.  Constructions are not exactly first-class objects in a Lestrade theory:  we will see in our Zermelo development that (for example) a constructon taking sort {\tt obj} to sort {\tt prop} is a proper class rather than a set.   A rule of inference in logic as implemented in Lestrade is a construction, while an implication is an actual proposition, thus an object.  In general, we can think of constructions as proper class functions;  we reserve the term ``function" for objects implementing constructions in some locally appropriate sense.

We have just described basically the entire framework.  All further concepts are introduced by specific declarations and definitions in the Lestrade environment.

This is less powerful than Automath, it should be observed.  No construction produces a construction as output (though a construction can take a construction as input and produce an object as output and this feature can be used to encapsulate constructions in actual objects of appropriate types, which we would be more likely to call ``functions".  In Automath, functions are first class objects and function types are first class types, and we will see that this gives Automath strength which would make it difficult to replicate our implementation of (first-order) Zermelo set theory in Automath.

Nonetheless, philosophically it {\em is\/} Automath:  it is a system of dependent type theory intended to implement mathematical objects and proofs about mathematical objects in parallel ways, the parallelism mediated by the Curry-Howard isomorphism.   It might be characterized as more cautious Automath.



\section{Developing logic}

We will develop classical logic, but in a way which can readily be cut down to constructive logic.

\begin{verbatim}

begin Lestrade execution

   >>> declare p prop


   p : prop


   {move 1}

   >>> declare q prop


   q : prop


   {move 1}

   >>> postulate & p q prop


   &: [(p_1 : prop), (q_1 : prop) => 
       (--- : prop)]


   {move 0}

   >>> postulate -> p q prop


   ->: [(p_1 : prop), (q_1 : prop) => 
       (--- : prop)]


   {move 0}

   >>> postulate V p q prop


   V : [(p_1 : prop), (q_1 : prop) => 
       (--- : prop)]


   {move 0}

   >>> postulate ?? prop


   ??: prop


   {move 0}
end Lestrade execution
\end{verbatim}

Above are the declarations for basic operations of propositional logic.  We introduce propositional variables usable as parameters in our declarations using the {\tt declare\/} command, then
introduce primitives of conjunction, implicatation, disjunction, and the False using the {\tt postulate\/} command.  That {\bf ??} is a constant rather than a variable parameter
can be seen in its designation at {\tt move 0\/} rather than {\tt move 1\/}, the designation attached to the parameters {\tt p} and {\tt q\/}.

\begin{verbatim}

begin Lestrade execution

   >>> declare pp that p


   pp : that p


   {move 1}

   >>> declare qq that q


   qq : that q


   {move 1}

   >>> declare rr that p & q


   rr : that p & q


   {move 1}

   >>> postulate Conj pp qq that p & q


   Conj : [(.p_1 : prop), (.q_1 : prop), (pp_1 
       : that .p_1), (qq_1 : that .q_1) => 
       (--- : that .p_1 & .q_1)]


   {move 0}

   >>> postulate Simp1 rr that p


   Simp1 : [(.p_1 : prop), (.q_1 : prop), (rr_1 
       : that .p_1 & .q_1) => (--- : that 
       .p_1)]


   {move 0}

   >>> postulate Simp2 rr that q


   Simp2 : [(.p_1 : prop), (.q_1 : prop), (rr_1 
       : that .p_1 & .q_1) => (--- : that 
       .q_1)]


   {move 0}
end Lestrade execution
\end{verbatim}

Above we declare the rules of deduction for conjunction as constructions.   Parameters {\tt pp}, {\tt qq}, and {\tt rr} representing evidence for $p$, $q$, and $p \wedge q$ respectively, are introduced.  We then declare primitive functions which take proofs of $p$ and $q$ to a proof of $p \wedge q$, and which extract proofs of $p$ and $q$ from a proof of $p \wedge q$.

Notice that the dependently typed constructions {\tt Conj}, {\tt Simp1}, and {\tt Simp2} also have $p$ and $q$ as implicit arguments (the values of the implicit arguments can be deduced from the types of the explicit arguments;  this feature took work to install!)

\begin{verbatim}

begin Lestrade execution

   >>> clearcurrent

{move 1}

   >>> declare p prop


   p : prop


   {move 1}

   >>> declare q prop


   q : prop


   {move 1}

   >>> declare rr that p & q


   rr : that p & q


   {move 1}

   >>> define Conjsymm rr : Conj (Simp2 \
       rr, Simp1 rr)


   Conjsymm : [(.p_1 : prop), (.q_1 
       : prop), (rr_1 : that .p_1 & .q_1) => 
       ({def} Simp2 (rr_1) Conj Simp1 (rr_1) : that 
       .q_1 & .p_1)]


   Conjsymm : [(.p_1 : prop), (.q_1 
       : prop), (rr_1 : that .p_1 & .q_1) => 
       (--- : that .q_1 & .p_1)]


   {move 0}

   >>> open


      {move 2}

      >>> declare r prop


      r : prop


      {move 2}

      >>> declare s prop


      s : prop


      {move 2}

      >>> declare t prop


      t : prop


      {move 2}

      >>> declare ss that r & s & t


      ss : that r & s & t


      {move 2}

      >>> define conjtest ss : Conjsymm ss


      conjtest : [(.r_1 : prop), (.s_1 
          : prop), (.t_1 : prop), (ss_1 
          : that .r_1 & .s_1 & .t_1) => (--- 
          : that (.s_1 & .t_1) & .r_1)]


      {move 1}

      >>> close


   {move 1}
end Lestrade execution
\end{verbatim}

In this block of text we verify a rule of inference (from $p \wedge q$, infer $q \wedge p$) and present an example of its use.

The {\tt clearcurrent} command clears all the variables currently declared, so $p$, $q$ and {\tt rr} have to be declared again.

The proof of the rule of inference is simply presented as a calculation:  we will see examples below where proof development looks more the way we expect a proof to look.

The example is encapsulated in a local environment (indicated by the {\tt open} and {\tt close} commands).  Notice that the parameters introduced locally to that environment
are labelled {\tt move 2}, and the proof {\tt conjtest} constructed in the local environment is labelled {\tt move 1}:  it will be cleared at the next execution of the {\tt clearcurrent}
command, while {\tt Conjsymm}, developed at move 0, is a permanently recorded theorem.

The reader who carefully examines the terms may notice that the type information reported by Lestrade for declarations includes the body of definitions only in the case of
objects and constructions defined at move 0.  It is possible to set Lestrade to show all definition bodies, but this makes the Lestrade output larger.

\begin{verbatim}

begin Lestrade execution

   >>> clearcurrent

{move 1}

   >>> declare p prop


   p : prop


   {move 1}

   >>> declare q prop


   q : prop


   {move 1}

   >>> declare pp that p


   pp : that p


   {move 1}

   >>> declare rr that p -> q


   rr : that p -> q


   {move 1}

   >>> postulate Mp pp rr that q


   Mp : [(.p_1 : prop), (.q_1 : prop), (pp_1 
       : that .p_1), (rr_1 : that .p_1 
       -> .q_1) => (--- : that .q_1)]


   {move 0}

   >>> open


      {move 2}

      >>> declare pp1 that p


      pp1 : that p


      {move 2}

      >>> postulate ded pp1 that q


      ded : [(pp1_1 : that p) => (--- 
          : that q)]


      {move 1}

      >>> close


   {move 1}

   >>> postulate Ded ded that p -> q


   Ded : [(.p_1 : prop), (.q_1 : prop), (ded_1 
       : [(pp1_2 : that .p_1) => (--- 
          : that .q_1)]) => (--- : that 
       .p_1 -> .q_1)]


   {move 0}
end Lestrade execution
\end{verbatim}

In this block we present the primitive rules for implication.  The rule of modus ponens requires no special comment:  it is developed in much the same way as
the rules for conjunction above.

The rule {\tt Ded} which implements the deduction theorem is a more subtle object, and the way it is developed involves subtle use of the Lestrade environment system.

What {\tt Ded} does is readily described:  it takes implicit arguments $p$ and $q$ (deducible from the type of the explicit argument) and a construction of evidence for $q$ from evidence for $p$, and returns a proof of $p \rightarrow q$.

It makes an important point about the difference between Lestrade and Automath (and other Automath relatives) that a construction of type \newline $[p:{\tt prop} \Rightarrow {\tt that} \,q]$ is not itself a proof of (or even evidence for) $p \rightarrow q$:  such a construction is actually in itself a rule of inference rather than an implication.  That we can infer $q$ from $p$ (a rule of inference, and a construction) is cast to a proof of $p\rightarrow q$ (an entity) by the constructor {\tt Ded}.

To explain how it is done, we need to outline the story of Lestrade environments.  At any given point, we are committed to objects in a move $i$ (and all moves $j$ for $j<i$) and may declare variable parameters in
move $i+1$.  The system variable $i$ is usually 0:  the {\tt open} command has the effect of creating a new empty move $i+2$ and then incrementing $i$;  the {\tt close} command has
the effect of erasing all declarations and definitions in move $i+1$\footnote{We will see later that it is possible to save declarations in move $i+1$ so that one can return to (that version of) move $i+1$ if desired.} and decrementing $i$.  $i$ cannot be decremented below 0.  The {\tt clearcurrent} command has the effect
of clearing move $i+1$ without decrementing $i$:  this can be used to clear move 1, which would otherwise get cluttered.  Anything declared in move 0 is permanent.

The {\tt declare} command is used to introduce new objects in move $i+1$, which serve as parameters in declarations and definitions.

The {\tt postulate} command, the mechanism for introduction of primitive notions and axioms,  declares an identifier as a constant in move $i$ of a given type, either an object if it is presented with no arguments, or a construction if it is presented
with a list of arguments from move $i+1$ appearing in the order in which they are declared (the order condition is  a cheap way to ensure that type dependencies are coherent).  As we have seen above, some arguments may be left implicit, if they can be deduced from the types of the explicitly given arguments.

The {\tt define} command defines an identifier as equivalent to an object term in move $i$ (if it has no arguments) or as a construction in move $i$ if it is presented with
move $i+1$ arguments in the correct order.   Of course this only works if the term types correctly.  This command is the vehicle for introduction of defined notions and theorems.

Declaration information recorded in type $i$ will expand any definitions in type $i+1$ on which the declaration or definition depended, because such information may disappear
when move $i+1$ is cleared.

Notice that constructions are declared in prefix form, but once declared can be used in infix form.  Lestrade treats most arity 2 operators as infix in output;  it understands mixfix notation $x_1 f x_2 x_n\ldots$ for a construction $f$ with arity greater than 2, but will never present output in this way.

This is only a brief account, enough to support the origin story of {\tt Ded}.  We execute {\tt open}, so we are in move 1 instead of move 0.  We introduce a move 2
parameter {\tt pp1} representing evidence for $p$.  We declare a construction {\tt ded} which takes {\tt pp1} as input and outputs evidence for $q$:  this is a move 1 object.
We then execute {\tt close}, so that {\tt ded} appears at move 1 while we are in move 0.  Since {\tt ded} is not a defined object, it is viewed as a variable parameter of the appropriate type,
which allows us to use the {\tt postulate} command to declare {\tt Ded} with the expected behavior.  It is a subtle point that primitive notions and axioms declared in move $i+1$ using
{\tt postulate} become variable parameters when we transition to move $i$.

\begin{verbatim}

begin Lestrade execution

   >>> open


      {move 2}

      >>> declare pp2 that p


      pp2 : that p


      {move 2}

      >>> declare ded2 [pp2 => that q]


      ded2 : [(pp2_1 : that p) => (--- 
          : that q)]


      {move 2}

      >>> postulate Ded2 ded2 that p -> q


      Ded2 : [(ded2_1 : [(pp2_2 : that 
             p) => (--- : that q)]) => 
          (--- : that p -> q)]


      {move 1}

      >>> close


   {move 1}
end Lestrade execution
\end{verbatim}

Here we illustrate how the deduction theorem construction could be declared without a local environment (though we package the whole experiment in a local environment so that the redundant  construction we declare can be discarded).  The key move here is that {\tt ded2} is declared directly with a construction type.   Notice that {\tt Ded2} has the same dependent type as {\tt Ded}.  If the {\tt open} and {\tt close} were removed, this block would declare the deduction theorem as a primitive construction in move 0, functionally equivalent to {\tt Ded} already declared above.

\begin{verbatim}

begin Lestrade execution

   >>> clearcurrent

{move 1}

   >>> declare p prop


   p : prop


   {move 1}

   >>> declare q prop


   q : prop


   {move 1}

   >>> open


      {move 2}

      >>> declare rr that p & q


      rr : that p & q


      {move 2}

      >>> define line1 rr : Conjsymm rr


      line1 : [(rr_1 : that p & q) => 
          (--- : that q & p)]


      {move 1}

      >>> close


   {move 1}

   >>> define Consymmimp p q : Ded line1


   Consymmimp : [(p_1 : prop), (q_1 
       : prop) => 
       ({def} Ded ([(rr_2 : that p_1 & q_1) => 
          ({def} Conjsymm (rr_2) : that 
          q_1 & p_1)]) : that (p_1 & q_1) -> 
       q_1 & p_1)]


   Consymmimp : [(p_1 : prop), (q_1 
       : prop) => (--- : that (p_1 & q_1) -> 
       q_1 & p_1)]


   {move 0}
end Lestrade execution
\end{verbatim}

Here is an example of use of {\tt Ded}.  Do notice that {\tt line1} and {\tt Conjsymm} have different types:  the implicit type mechanism obscures the true type of {\tt Conjsymm}, which is not suitable for use as an input to {\tt Ded}.


\begin{verbatim}

begin Lestrade execution

   >>> clearcurrent

{move 1}

   >>> declare p prop


   p : prop


   {move 1}

   >>> define ~ p : p -> ??


   ~: [(p_1 : prop) => 
       ({def} p_1 -> ?? : prop)]


   ~: [(p_1 : prop) => (--- : prop)]


   {move 0}

   >>> declare absurd that ??


   absurd : that ??


   {move 1}

   >>> postulate Giveup p absurd that p


   Giveup : [(p_1 : prop), (absurd_1 
       : that ??) => (--- : that p_1)]


   {move 0}

   >>> declare maybe that ~ ~ p


   maybe : that ~ (~ (p))


   {move 1}

   >>> postulate Dneg maybe that p


   Dneg : [(.p_1 : prop), (maybe_1 : that 
       ~ (~ (.p_1))) => (--- : that 
       .p_1)]


   {move 0}

   >>> declare q prop


   q : prop


   {move 1}

   >>> define == p q : (p -> q) & q -> \
       p


   ==: [(p_1 : prop), (q_1 : prop) => 
       ({def} (p_1 -> q_1) & q_1 -> p_1 
       : prop)]


   ==: [(p_1 : prop), (q_1 : prop) => 
       (--- : prop)]


   {move 0}
end Lestrade execution
\end{verbatim}

Here we define negation $\neg p$ as $p \rightarrow \, \perp$ ($\perp$ being the absurd), introduce the rule that anything follows from the absurd, and introduce the primitive rule of  double negation which makes the logic classical.
All of our other primitive constructions of logic are constructive.

We incidentally define the biconditional.


\begin{verbatim}

begin Lestrade execution

   >>> clearcurrent

{move 1}

   >>> declare p prop


   p : prop


   {move 1}

   >>> declare q prop


   q : prop


   {move 1}

   >>> declare r prop


   r : prop


   {move 1}

   >>> declare pp that p


   pp : that p


   {move 1}

   >>> declare qq that q


   qq : that q


   {move 1}

   >>> declare rr that p V q


   rr : that p V q


   {move 1}

   >>> postulate Add1 q pp that p V q


   Add1 : [(.p_1 : prop), (q_1 : prop), (pp_1 
       : that .p_1) => (--- : that .p_1 
       V q_1)]


   {move 0}

   >>> postulate Add2 p qq that p V q


   Add2 : [(p_1 : prop), (.q_1 : prop), (qq_1 
       : that .q_1) => (--- : that p_1 V .q_1)]


   {move 0}

   >>> declare case1 [pp => that r]


   case1 : [(pp_1 : that p) => (--- : that 
       r)]


   {move 1}

   >>> declare case2 [qq => that r]


   case2 : [(qq_1 : that q) => (--- : that 
       r)]


   {move 1}

   >>> postulate Cases rr case1, case2 that \
       r


   Cases : [(.p_1 : prop), (.q_1 : prop), (.r_1 
       : prop), (rr_1 : that .p_1 V .q_1), (case1_1 
       : [(pp_2 : that .p_1) => (--- : that 
          .r_1)]), (case2_1 : [(qq_2 
          : that .q_1) => (--- : that .r_1)]) => 
       (--- : that .r_1)]


   {move 0}
end Lestrade execution
\end{verbatim}

Here we develop the primitive rules of disjunction, the two rules of addition and the rule of proof by cases.

Here we use the approach of declaring the two constructions which implement the cases directly as construction variables, rather than the indirect though philosophically interesting approach using local environments exhibited in the development of the deduction theorem.

\section{Propositional logic lemmas}

This space is reserved for proofs of lemmas in propositional logic as we require them for proofs below.

\begin{verbatim}

begin Lestrade execution

   >>> clearcurrent

{move 1}

   >>> declare p prop


   p : prop


   {move 1}

   >>> declare pp that p


   pp : that p


   {move 1}

   >>> define Fixform p pp : pp


   Fixform : [(p_1 : prop), (pp_1 : that 
       p_1) => 
       ({def} pp_1 : that p_1)]


   Fixform : [(p_1 : prop), (pp_1 : that 
       p_1) => (--- : that p_1)]


   {move 0}
end Lestrade execution
\end{verbatim}

{\tt Fixform} is a device for controlling the way that Lestrade expresses defined concepts in reported type information; we will see its use on many occasions below.


\begin{verbatim}

begin Lestrade execution

   >>> clearcurrent

{move 1}

   >>> declare p prop


   p : prop


   {move 1}

   >>> declare q prop


   q : prop


   {move 1}

   >>> declare pp that p


   pp : that p


   {move 1}

   >>> declare qq that q


   qq : that q


   {move 1}

   >>> declare iffev that p == q


   iffev : that p == q


   {move 1}

   >>> define Iff1 pp iffev : Mp pp Simp1 \
       iffev


   Iff1 : [(.p_1 : prop), (.q_1 : prop), (pp_1 
       : that .p_1), (iffev_1 : that .p_1 
       == .q_1) => 
       ({def} pp_1 Mp Simp1 (iffev_1) : that 
       .q_1)]


   Iff1 : [(.p_1 : prop), (.q_1 : prop), (pp_1 
       : that .p_1), (iffev_1 : that .p_1 
       == .q_1) => (--- : that .q_1)]


   {move 0}

   >>> define Iff2 qq iffev : Mp qq Simp2 \
       iffev


   Iff2 : [(.p_1 : prop), (.q_1 : prop), (qq_1 
       : that .q_1), (iffev_1 : that .p_1 
       == .q_1) => 
       ({def} qq_1 Mp Simp2 (iffev_1) : that 
       .p_1)]


   Iff2 : [(.p_1 : prop), (.q_1 : prop), (qq_1 
       : that .q_1), (iffev_1 : that .p_1 
       == .q_1) => (--- : that .p_1)]


   {move 0}
end Lestrade execution
\end{verbatim}

Rules of modus ponens for the biconditional

\begin{verbatim}

begin Lestrade execution

   >>> declare dir1 [pp => that q]


   dir1 : [(pp_1 : that p) => (--- : that 
       q)]


   {move 1}

   >>> declare dir2 [qq => that p]


   dir2 : [(qq_1 : that q) => (--- : that 
       p)]


   {move 1}

   >>> define Dediff dir1, dir2 : Fixform \
       (p == q, Conj (Ded dir1, Ded dir2))


   Dediff : [(.p_1 : prop), (.q_1 : prop), (dir1_1 
       : [(pp_2 : that .p_1) => (--- : that 
          .q_1)]), (dir2_1 : [(qq_2 
          : that .q_1) => (--- : that .p_1)]) => 
       ({def} (.p_1 == .q_1) Fixform Ded 
       (dir1_1) Conj Ded (dir2_1) : that 
       .p_1 == .q_1)]


   Dediff : [(.p_1 : prop), (.q_1 : prop), (dir1_1 
       : [(pp_2 : that .p_1) => (--- : that 
          .q_1)]), (dir2_1 : [(qq_2 
          : that .q_1) => (--- : that .p_1)]) => 
       (--- : that .p_1 == .q_1)]


   {move 0}

   >>> define Dediffdud dir1, dir2 : Conj \
       (Ded dir1, Ded dir2)


   Dediffdud : [(.p_1 : prop), (.q_1 
       : prop), (dir1_1 : [(pp_2 : that 
          .p_1) => (--- : that .q_1)]), (dir2_1 
       : [(qq_2 : that .q_1) => (--- : that 
          .p_1)]) => 
       ({def} Ded (dir1_1) Conj Ded (dir2_1) : that 
       (.p_1 -> .q_1) & .q_1 -> .p_1)]


   Dediffdud : [(.p_1 : prop), (.q_1 
       : prop), (dir1_1 : [(pp_2 : that 
          .p_1) => (--- : that .q_1)]), (dir2_1 
       : [(qq_2 : that .q_1) => (--- : that 
          .p_1)]) => (--- : that (.p_1 
       -> .q_1) & .q_1 -> .p_1)]


   {move 0}

   >>> define Iffrefl p : Dediff [pp => \
          pp] [pp => pp]


   Iffrefl : [(p_1 : prop) => 
       ({def} Dediff ([(pp_2 : that p_1) => 
          ({def} pp_2 : that p_1)], [(pp_2 
          : that p_1) => 
          ({def} pp_2 : that p_1)]) : that 
       p_1 == p_1)]


   Iffrefl : [(p_1 : prop) => (--- : that 
       p_1 == p_1)]


   {move 0}
end Lestrade execution
\end{verbatim}

The deduction theorem for biconditionals.  Notice the use of {\tt Fixform} to force the reported type to be {\tt that p == q} rather than {\tt that (p -> q) \& (q -> p)}.  The version {\tt Dediffdud} is provided to make this point. 
 Of course {\tt Fixform} only works here because the desired form is definitionally equivalent to the default form;  Lestrade does check this (not by any special provision in the software, but automatically in the course of type checking the term).

\begin{verbatim}

begin Lestrade execution

   >>> clearcurrent

{move 1}

   >>> declare p prop


   p : prop


   {move 1}

   >>> declare pp that p


   pp : that p


   {move 1}

   >>> declare porpev that p V p


   porpev : that p V p


   {move 1}
end Lestrade execution
\end{verbatim}

The idempotent property of disjunction.  This (below) demonstrates that from $p \vee p$ we can deduce $p$.

\begin{verbatim}

begin Lestrade execution

   >>> define Oridem porpev : Cases porpev \
       [pp => pp], [pp => pp]


   Oridem : [(.p_1 : prop), (porpev_1 
       : that .p_1 V .p_1) => 
       ({def} Cases (porpev_1, [(pp_2 
          : that .p_1) => 
          ({def} pp_2 : that .p_1)], [(pp_2 
          : that .p_1) => 
          ({def} pp_2 : that .p_1)]) : that 
       .p_1)]


   Oridem : [(.p_1 : prop), (porpev_1 
       : that .p_1 V .p_1) => (--- : that 
       .p_1)]


   {move 0}

   >>> clearcurrent

{move 1}

   >>> declare p prop


   p : prop


   {move 1}

   >>> declare pp that p


   pp : that p


   {move 1}

   >>> declare adabsurdum [pp => that ??]


   adabsurdum : [(pp_1 : that p) => (--- 
       : that ??)]


   {move 1}

   >>> define Negintro adabsurdum : Fixform \
       (~ p, Ded adabsurdum)


   Negintro : [(.p_1 : prop), (adabsurdum_1 
       : [(pp_2 : that .p_1) => (--- : that 
          ??)]) => 
       ({def} ~ (.p_1) Fixform Ded (adabsurdum_1) : that 
       ~ (.p_1))]


   Negintro : [(.p_1 : prop), (adabsurdum_1 
       : [(pp_2 : that .p_1) => (--- : that 
          ??)]) => (--- : that ~ (.p_1))]


   {move 0}
end Lestrade execution
\end{verbatim}


Here is an another example of using {\tt Fixform}:  {\tt Negintro} is actually a special case of the deduction theorem;  what {\tt Fixform} does is force the surface expression of the implication
$p \rightarrow \, \perp$ proved to $\neg p$.

We give the classical rules of inference from the negation of an implication.

\begin{verbatim}

begin Lestrade execution

   >>> clearcurrent

{move 1}

   >>> declare p prop


   p : prop


   {move 1}

   >>> declare q prop


   q : prop


   {move 1}

   >>> declare notimp that ~ (p -> q)


   notimp : that ~ (p -> q)


   {move 1}

   >>> open


      {move 2}

      >>> declare qev that q


      qev : that q


      {move 2}

      >>> open


         {move 3}

         >>> declare pev that p


         pev : that p


         {move 3}

         >>> define idqev pev : qev


         idqev : [(pev_1 : that p) => 
             (--- : that q)]


         {move 2}

         >>> close


      {move 2}

      >>> define line1 qev : Mp (Ded idqev, notimp)


      line1 : [(qev_1 : that q) => (--- 
          : that ??)]


      {move 1}

      >>> close


   {move 1}

   >>> define Notimp1 notimp : Negintro line1


   Notimp1 : [(.p_1 : prop), (.q_1 : prop), (notimp_1 
       : that ~ (.p_1 -> .q_1)) => 
       ({def} Negintro ([(qev_2 : that 
          .q_1) => 
          ({def} Ded ([(pev_4 : that .p_1) => 
             ({def} qev_2 : that .q_1)]) Mp 
          notimp_1 : that ??)]) : that 
       ~ (.q_1))]


   Notimp1 : [(.p_1 : prop), (.q_1 : prop), (notimp_1 
       : that ~ (.p_1 -> .q_1)) => (--- 
       : that ~ (.q_1))]


   {move 0}

   >>> open


      {move 2}

      >>> declare negpev that ~ p


      negpev : that ~ (p)


      {move 2}

      >>> open


         {move 3}

         >>> declare pev that p


         pev : that p


         {move 3}

         >>> define line2 pev : Giveup q, Mp \
             pev negpev


         line2 : [(pev_1 : that p) => 
             (--- : that q)]


         {move 2}

         >>> close


      {move 2}

      >>> define line3 negpev : Mp (Ded \
          line2, notimp)


      line3 : [(negpev_1 : that ~ (p)) => 
          (--- : that ??)]


      {move 1}

      >>> close


   {move 1}

   >>> define Notimp2 notimp : Dneg (Negintro \
       line3)


   Notimp2 : [(.p_1 : prop), (.q_1 : prop), (notimp_1 
       : that ~ (.p_1 -> .q_1)) => 
       ({def} Dneg (Negintro ([(negpev_3 
          : that ~ (.p_1)) => 
          ({def} Ded ([(pev_5 : that .p_1) => 
             ({def} .q_1 Giveup pev_5 Mp 
             negpev_3 : that .q_1)]) Mp 
          notimp_1 : that ??)])) : that 
       .p_1)]


   Notimp2 : [(.p_1 : prop), (.q_1 : prop), (notimp_1 
       : that ~ (.p_1 -> .q_1)) => (--- 
       : that .p_1)]


   {move 0}

   >>> declare notconjev that ~ (p & q)


   notconjev : that ~ (p & q)


   {move 1}

   >>> declare qev that q


   qev : that q


   {move 1}

   >>> open


      {move 2}

      >>> declare pev that p


      pev : that p


      {move 2}

      >>> define line4 pev : Mp (Conj pev \
          qev, notconjev)


      line4 : [(pev_1 : that p) => (--- 
          : that ??)]


      {move 1}

      >>> close


   {move 1}

   >>> define Notconj notconjev qev : Negintro \
       line4


   Notconj : [(.p_1 : prop), (.q_1 : prop), (notconjev_1 
       : that ~ (.p_1 & .q_1)), (qev_1 
       : that .q_1) => 
       ({def} Negintro ([(pev_2 : that 
          .p_1) => 
          ({def} pev_2 Conj qev_1 Mp notconjev_1 
          : that ??)]) : that ~ (.p_1))]


   Notconj : [(.p_1 : prop), (.q_1 : prop), (notconjev_1 
       : that ~ (.p_1 & .q_1)), (qev_1 
       : that .q_1) => (--- : that ~ (.p_1))]


   {move 0}
end Lestrade execution
\end{verbatim}

Above we present the rule of inference which gives us $\neg p$ if we have $\neg(p \wedge q)$ and $q$.

Now we develop the law of excluded middle.

\begin{verbatim}

begin Lestrade execution

   >>> clearcurrent

{move 1}

   >>> declare p prop


   p : prop


   {move 1}

   >>> open


      {move 2}

      >>> declare nona that ~ (p V ~ p)


      nona : that ~ (p V ~ (p))


      {move 2}

      >>> open


         {move 3}

         >>> declare thatp that p


         thatp : that p


         {move 3}

         >>> define line1 thatp : Add1 (~ p, thatp)


         line1 : [(thatp_1 : that p) => 
             (--- : that p V ~ (p))]


         {move 2}

         >>> define line2 thatp : Mp (line1 \
             thatp, nona)


         line2 : [(thatp_1 : that p) => 
             (--- : that ??)]


         {move 2}

         >>> close


      {move 2}

      >>> define line3 nona : Negintro line2


      line3 : [(nona_1 : that ~ (p V ~ (p))), 
          ({let} .line1_1 : [(thatp_2 : that 
             p) => 
             ({def} ~ (p) Add1 thatp_2 
             : that p V ~ (p))]) => (--- 
          : that ~ (p))]


      {move 1}

      >>> define line4 nona : Add2 (p, line3 \
          nona)


      line4 : [(nona_1 : that ~ (p V ~ (p))) => 
          (--- : that p V ~ (p))]


      {move 1}

      >>> define line5 nona : Mp (line4 \
          nona, nona)


      line5 : [(nona_1 : that ~ (p V ~ (p))) => 
          (--- : that ??)]


      {move 1}

      >>> close


   {move 1}

   >>> define Excmid p : Dneg (Negintro \
       line5)


   Excmid : [(p_1 : prop) => 
       ({def} Dneg (Negintro ([(nona_3 
          : that ~ (p_1 V ~ (p_1))) => 
          ({def} p_1 Add2 [
             ({let} .line1_5 : [(thatp_6 
                : that p_1) => 
                ({def} ~ (p_1) Add1 thatp_6 
                : that p_1 V ~ (p_1))]) => 
             ({def} Negintro ([(thatp_6 
                : that p_1) => 
                ({def} ~ (p_1) Add1 thatp_6 
                Mp nona_3 : that ??)]) : that 
             ~ (p_1))] Mp nona_3 : that 
          ??)])) : that p_1 V ~ (p_1))]


   Excmid : [(p_1 : prop) => (--- : that 
       p_1 V ~ (p_1))]


   {move 0}
end Lestrade execution
\end{verbatim}

We prove the rules of disjunctive syllogism.

\begin{verbatim}

begin Lestrade execution

   >>> clearcurrent

{move 1}

   >>> declare p prop


   p : prop


   {move 1}

   >>> declare q prop


   q : prop


   {move 1}

   >>> declare orev that p V q


   orev : that p V q


   {move 1}

   >>> declare negpev that ~ p


   negpev : that ~ (p)


   {move 1}

   >>> declare negqev that ~ q


   negqev : that ~ (q)


   {move 1}

   >>> open


      {move 2}

      >>> declare negpev2 that ~ p


      negpev2 : that ~ (p)


      {move 2}

      >>> declare negqev2 that ~ q


      negqev2 : that ~ (q)


      {move 2}

      >>> open


         {move 3}

         >>> declare thatp that p


         thatp : that p


         {move 3}

         >>> declare thatq that q


         thatq : that q


         {move 3}

         >>> define line1 thatp : Mp thatp \
             negpev2


         line1 : [(thatp_1 : that p) => 
             (--- : that ??)]


         {move 2}

         >>> define line2 thatp : Mp thatp \
             negpev


         line2 : [(thatp_1 : that p) => 
             (--- : that ??)]


         {move 2}

         >>> define line3 thatq : Mp thatq \
             negqev2


         line3 : [(thatq_1 : that q) => 
             (--- : that ??)]


         {move 2}

         >>> define line4 thatq : Mp thatq \
             negqev


         line4 : [(thatq_1 : that q) => 
             (--- : that ??)]


         {move 2}

         >>> close


      {move 2}

      >>> define line5 negpev2 : Cases orev, line1, line4


      line5 : [(negpev2_1 : that ~ (p)), 
          ({let} .line4_1 : [(thatq_2 : that 
             q) => 
             ({def} thatq_2 Mp negqev : that 
             ??)]) => (--- : that ??)]


      {move 1}

      >>> define line6 negqev2 : Cases orev, line2, line3


      line6 : [(negqev2_1 : that ~ (q)), 
          ({let} .line2_1 : [(thatp_2 : that 
             p) => 
             ({def} thatp_2 Mp negpev : that 
             ??)]) => (--- : that ??)]


      {move 1}

      >>> close


   {move 1}

   >>> define Ds1 orev negqev : Dneg (Negintro \
       line5)


   Ds1 : [(.p_1 : prop), (.q_1 : prop), (orev_1 
       : that .p_1 V .q_1), (negqev_1 : that 
       ~ (.q_1)) => 
       ({def} Dneg (Negintro ([(negpev2_3 
          : that ~ (.p_1)), 
          ({let} .line4_3 : [(thatq_4 : that 
             .q_1) => 
             ({def} thatq_4 Mp negqev_1 : that 
             ??)]) => 
          ({def} Cases (orev_1, [(thatp_4 
             : that .p_1) => 
             ({def} thatp_4 Mp negpev2_3 
             : that ??)], .line4_3) : that 
          ??)])) : that .p_1)]


   Ds1 : [(.p_1 : prop), (.q_1 : prop), (orev_1 
       : that .p_1 V .q_1), (negqev_1 : that 
       ~ (.q_1)) => (--- : that .p_1)]


   {move 0}

   >>> define Ds2 orev negpev : Dneg (Negintro \
       line6)


   Ds2 : [(.p_1 : prop), (.q_1 : prop), (orev_1 
       : that .p_1 V .q_1), (negpev_1 : that 
       ~ (.p_1)) => 
       ({def} Dneg (Negintro ([(negqev2_3 
          : that ~ (.q_1)), 
          ({let} .line2_3 : [(thatp_4 : that 
             .p_1) => 
             ({def} thatp_4 Mp negpev_1 : that 
             ??)]) => 
          ({def} Cases (orev_1, .line2_3, [(thatq_4 
             : that .q_1) => 
             ({def} thatq_4 Mp negqev2_3 
             : that ??)]) : that ??)])) : that 
       .q_1)]


   Ds2 : [(.p_1 : prop), (.q_1 : prop), (orev_1 
       : that .p_1 V .q_1), (negpev_1 : that 
       ~ (.p_1)) => (--- : that .q_1)]


   {move 0}
end Lestrade execution
\end{verbatim}

We enjoyed the sharing of environments in this proof.  The reader will notice increasing size of the definition bodies of objections and constructions defined at move 0:  recall that notions defined at moves of positive index are eliminated from these definitions by suitable expansions, which as we will see later can make definition bodies infeasibly large if we do not exercise care.


\section{Quantification}

In this section, we introduce the universal and existential quantifiers over untyped objects (sort {\tt obj}) which will be the sets and atoms of our theory.

\begin{verbatim}

begin Lestrade execution

   >>> clearcurrent

{move 1}

   >>> declare x obj


   x : obj


   {move 1}

   >>> declare pred [x => prop]


   pred : [(x_1 : obj) => (--- : prop)]


   {move 1}

   >>> postulate Forall pred prop


   Forall : [(pred_1 : [(x_2 : obj) => 
          (--- : prop)]) => (--- : prop)]


   {move 0}

   >>> postulate Exists pred prop


   Exists : [(pred_1 : [(x_2 : obj) => 
          (--- : prop)]) => (--- : prop)]


   {move 0}
end Lestrade execution
\end{verbatim}

Here are the basic declarations for the universal and existential quantifiers (which are treated as separate primitives because they are separate
primitives in constructive logic).  

\begin{verbatim}

begin Lestrade execution

   >>> declare univev that Forall pred


   univev : that Forall (pred)


   {move 1}

   >>> postulate Ui x univev that pred x


   Ui : [(x_1 : obj), (.pred_1 : [(x_2 
          : obj) => (--- : prop)]), (univev_1 
       : that Forall (.pred_1)) => (--- 
       : that .pred_1 (x_1))]


   {move 0}

   >>> declare univev2 [x => that pred x]


   univev2 : [(x_1 : obj) => (--- : that 
       pred (x_1))]


   {move 1}

   >>> postulate Ug univev2 that Forall pred


   Ug : [(.pred_1 : [(x_2 : obj) => 
          (--- : prop)]), (univev2_1 
       : [(x_2 : obj) => (--- : that .pred_1 
          (x_2))]) => (--- : that Forall 
       (.pred_1))]


   {move 0}

   >>> declare y obj


   y : obj


   {move 1}

   >>> postulate = x y prop


   =: [(x_1 : obj), (y_1 : obj) => 
       (--- : prop)]


   {move 0}

   >>> postulate Refleq x that x = x


   Refleq : [(x_1 : obj) => (--- : that 
       x_1 = x_1)]


   {move 0}

   >>> define Ugtest : Ug Refleq


   Ugtest : [
       ({def} Ug (Refleq) : that Forall 
       ([(x''_2 : obj) => 
          ({def} x''_2 = x''_2 : prop)]))]


   Ugtest : that Forall ([(x''_2 : obj) => 
       ({def} x''_2 = x''_2 : prop)])


   {move 0}
end Lestrade execution
\end{verbatim}

Here are initial declarations for equality and an example of universal generalization.  The equality relation is declared, the rule of reflexivity presented, 
and the theorem $(\forall x:x=x)$ proved.

\begin{verbatim}

begin Lestrade execution

   >>> declare existsev that pred x


   existsev : that pred (x)


   {move 1}

   >>> postulate Ei x pred, existsev that \
       Exists pred


   Ei : [(x_1 : obj), (pred_1 : [(x_2 
          : obj) => (--- : prop)]), (existsev_1 
       : that pred_1 (x_1)) => (--- : that 
       Exists (pred_1))]


   {move 0}

   >>> define Ei1 x existsev : Ei x pred, existsev


   Ei1 : [(x_1 : obj), (.pred_1 : [(x_2 
          : obj) => (--- : prop)]), (existsev_1 
       : that .pred_1 (x_1)) => 
       ({def} Ei (x_1, .pred_1, existsev_1) : that 
       Exists (.pred_1))]


   Ei1 : [(x_1 : obj), (.pred_1 : [(x_2 
          : obj) => (--- : prop)]), (existsev_1 
       : that .pred_1 (x_1)) => (--- : that 
       Exists (.pred_1))]


   {move 0}

   >>> define Ei2 pred, existsev : Ei x pred, existsev


   Ei2 : [(.x_1 : obj), (pred_1 : [(x_2 
          : obj) => (--- : prop)]), (existsev_1 
       : that pred_1 (.x_1)) => 
       ({def} Ei (.x_1, pred_1, existsev_1) : that 
       Exists (pred_1))]


   Ei2 : [(.x_1 : obj), (pred_1 : [(x_2 
          : obj) => (--- : prop)]), (existsev_1 
       : that pred_1 (.x_1)) => (--- : that 
       Exists (pred_1))]


   {move 0}

   >>> declare existsev2 that Exists pred


   existsev2 : that Exists (pred)


   {move 1}

   >>> declare r prop


   r : prop


   {move 1}

   >>> declare witnessev [x, existsev => \
          that r]


   witnessev : [(x_1 : obj), (existsev_1 
       : that pred (x_1)) => (--- : that 
       r)]


   {move 1}

   >>> postulate Eg existsev2 witnessev that \
       r


   Eg : [(.pred_1 : [(x_2 : obj) => 
          (--- : prop)]), (existsev2_1 
       : that Exists (.pred_1)), (.r_1 
       : prop), (witnessev_1 : [(x_2 
          : obj), (existsev_2 : that .pred_1 
          (x_2)) => (--- : that .r_1)]) => 
       (--- : that .r_1)]


   {move 0}

   >>> declare test obj


   test : obj


   {move 1}

   >>> define Witnesses existsev2 test : pred \
       test


   Witnesses : [(.pred_1 : [(x_2 : obj) => 
          (--- : prop)]), (existsev2_1 
       : that Exists (.pred_1)), (test_1 
       : obj) => 
       ({def} .pred_1 (test_1) : prop)]


   Witnesses : [(.pred_1 : [(x_2 : obj) => 
          (--- : prop)]), (existsev2_1 
       : that Exists (.pred_1)), (test_1 
       : obj) => (--- : prop)]


   {move 0}
end Lestrade execution
\end{verbatim}

These are the declarations for the rules for the existential quantifier.  The rule of existential instantiation comes in various flavors, because the ability of
Lestrade to deduce implicit arguments in higher-order situations is nonzero but limited.  {\tt Witnesses} is a gadget for extracting an instance of an existential statement whose
surface form might not appear to be existential (so the user doesn't have to look up a definition).

\begin{verbatim}

begin Lestrade execution

   >>> clearcurrent

{move 1}

   >>> declare x obj


   x : obj


   {move 1}

   >>> declare y obj


   y : obj


   {move 1}

   >>> declare eqev that x = y


   eqev : that x = y


   {move 1}

   >>> declare pred [x => prop]


   pred : [(x_1 : obj) => (--- : prop)]


   {move 1}

   >>> declare predev that pred x


   predev : that pred (x)


   {move 1}

   >>> postulate Subs eqev pred, predev \
       that pred y


   Subs : [(.x_1 : obj), (.y_1 : obj), (eqev_1 
       : that .x_1 = .y_1), (pred_1 : [(x_2 
          : obj) => (--- : prop)]), (predev_1 
       : that pred_1 (.x_1)) => (--- : that 
       pred_1 (.y_1))]


   {move 0}

   >>> define Subs1 eqev, predev : Subs \
       eqev pred, predev


   Subs1 : [(.x_1 : obj), (.y_1 : obj), (eqev_1 
       : that .x_1 = .y_1), (.pred_1 : [(x_2 
          : obj) => (--- : prop)]), (predev_1 
       : that .pred_1 (.x_1)) => 
       ({def} Subs (eqev_1, .pred_1, predev_1) : that 
       .pred_1 (.y_1))]


   Subs1 : [(.x_1 : obj), (.y_1 : obj), (eqev_1 
       : that .x_1 = .y_1), (.pred_1 : [(x_2 
          : obj) => (--- : prop)]), (predev_1 
       : that .pred_1 (.x_1)) => (--- 
       : that .pred_1 (.y_1))]


   {move 0}
end Lestrade execution
\end{verbatim}

The declaration of the rule of substitution (in two flavors) completes the logic of equality.  The two flavors are present because the implicit argument feature is not perfectly dependable where constructions are involved in matching, so one will sometimes want {\tt Subs} with the explicitly given predicate.

\section{Lemmas of first order logic with equality}

This is a space in which lemmas of first order logic with equality will be developed as needed below.  Substitution comes in a flavor with the predicate left implicit, but this cannot always be done.

\begin{verbatim}

begin Lestrade execution

   >>> clearcurrent

{move 1}

   >>> declare x obj


   x : obj


   {move 1}

   >>> declare y obj


   y : obj


   {move 1}

   >>> declare z obj


   z : obj


   {move 1}

   >>> declare eqev1 that x = y


   eqev1 : that x = y


   {move 1}

   >>> declare eqev2 that y = z


   eqev2 : that y = z


   {move 1}

   >>> define Eqsymm eqev1 : Subs eqev1 [z => \
          z = x] Refleq x


   Eqsymm : [(.x_1 : obj), (.y_1 : obj), (eqev1_1 
       : that .x_1 = .y_1) => 
       ({def} Subs (eqev1_1, [(z_2 : obj) => 
          ({def} z_2 = .x_1 : prop)], Refleq 
       (.x_1)) : that .y_1 = .x_1)]


   Eqsymm : [(.x_1 : obj), (.y_1 : obj), (eqev1_1 
       : that .x_1 = .y_1) => (--- : that 
       .y_1 = .x_1)]


   {move 0}

   >>> declare w obj


   w : obj


   {move 1}

   >>> define Eqtrans eqev1 eqev2 : Subs1 \
       eqev2 eqev1


   Eqtrans : [(.x_1 : obj), (.y_1 : obj), (.z_1 
       : obj), (eqev1_1 : that .x_1 = .y_1), (eqev2_1 
       : that .y_1 = .z_1) => 
       ({def} eqev2_1 Subs1 eqev1_1 : that 
       .x_1 = .z_1)]


   Eqtrans : [(.x_1 : obj), (.y_1 : obj), (.z_1 
       : obj), (eqev1_1 : that .x_1 = .y_1), (eqev2_1 
       : that .y_1 = .z_1) => (--- : that 
       .x_1 = .z_1)]


   {move 0}

   >>> declare a17 obj


   a17 : obj


   {move 1}

   >>> declare b17 obj


   b17 : obj


   {move 1}

   >>> declare diffev that ~ (a17 = b17)


   diffev : that ~ (a17 = b17)


   {move 1}

   >>> declare funsies [a17 => obj]


   funsies : [(a17_1 : obj) => (--- : obj)]


   {move 1}

   >>> declare eqev17 that a17 = b17


   eqev17 : that a17 = b17


   {move 1}

   >>> define bothsides funsies, eqev17 \
       : Subs eqev17 [b17 => funsies a17 = funsies \
          b17] Refleq funsies a17


   bothsides : [(.a17_1 : obj), (.b17_1 
       : obj), (funsies_1 : [(a17_2 : obj) => 
          (--- : obj)]), (eqev17_1 : that 
       .a17_1 = .b17_1) => 
       ({def} Subs (eqev17_1, [(b17_2 
          : obj) => 
          ({def} funsies_1 (.a17_1) = funsies_1 
          (b17_2) : prop)], Refleq (funsies_1 
       (.a17_1))) : that funsies_1 (.a17_1) = funsies_1 
       (.b17_1))]


   bothsides : [(.a17_1 : obj), (.b17_1 
       : obj), (funsies_1 : [(a17_2 : obj) => 
          (--- : obj)]), (eqev17_1 : that 
       .a17_1 = .b17_1) => (--- : that funsies_1 
       (.a17_1) = funsies_1 (.b17_1))]


   {move 0}

   >>> open


      {move 2}

      >>> declare sillyhyp that b17 = a17


      sillyhyp : that b17 = a17


      {move 2}

      >>> define linec1 sillyhyp : Mp (Eqsymm \
          sillyhyp, diffev)


      linec1 : [(sillyhyp_1 : that b17 
          = a17) => (--- : that ??)]


      {move 1}

      >>> close


   {move 1}

   >>> define Negeqsymm diffev : Negintro \
       linec1


   Negeqsymm : [(.a17_1 : obj), (.b17_1 
       : obj), (diffev_1 : that ~ (.a17_1 
       = .b17_1)) => 
       ({def} Negintro ([(sillyhyp_2 : that 
          .b17_1 = .a17_1) => 
          ({def} Eqsymm (sillyhyp_2) Mp 
          diffev_1 : that ??)]) : that 
       ~ (.b17_1 = .a17_1))]


   Negeqsymm : [(.a17_1 : obj), (.b17_1 
       : obj), (diffev_1 : that ~ (.a17_1 
       = .b17_1)) => (--- : that ~ (.b17_1 
       = .a17_1))]


   {move 0}
end Lestrade execution
\end{verbatim}

Here are the expected symmetry and transitivity properties of equality, plus the ability to apply the same construction to both sides of an equation and symmetry for inequality.  The exact lemmas present here and everywhere in this section are driven by what we turned out to need in the development of Zermelo's axiomatics and proof of the well-ordering theorem.





We prove the classical theorem that $\neg (\forall x:\phi) \rightarrow (\exists x:\neg\phi)$ (actually in the form of a rule of inference).

\begin{verbatim}

begin Lestrade execution

   >>> clearcurrent

{move 1}

   >>> declare x obj


   x : obj


   {move 1}

   >>> declare pred [x => prop]


   pred : [(x_1 : obj) => (--- : prop)]


   {move 1}

   >>> declare negunivev that ~ (Forall \
       pred)


   negunivev : that ~ (Forall (pred))


   {move 1}

   >>> open


      {move 2}

      >>> declare z obj


      z : obj


      {move 2}

      >>> declare negexistev that ~ (Exists \
          [z => ~ (pred z)])


      negexistev : that ~ (Exists ([(z_3 
          : obj) => 
          ({def} ~ (pred (z_3)) : prop)]))


      {move 2}

      >>> open


         {move 3}

         >>> declare y obj


         y : obj


         {move 3}

         >>> open


            {move 4}

            >>> declare absurdhyp that ~ (pred \
                y)


            absurdhyp : that ~ (pred (y))


            {move 4}

            >>> define line1 absurdhyp : Ei1 \
                y absurdhyp


            line1 : [(absurdhyp_1 : that 
                ~ (pred (y))) => (--- 
                : that Exists ([(x'_2 : obj) => 
                   ({def} ~ (pred (x'_2)) : prop)]))]


            {move 3}

            >>> define line2 absurdhyp : Mp \
                line1 absurdhyp negexistev


            line2 : [(absurdhyp_1 : that 
                ~ (pred (y))) => (--- 
                : that ??)]


            {move 3}

            >>> close


         {move 3}

         >>> define line3 y : Dneg Negintro \
             line2


         line3 : [(y_1 : obj) => (--- 
             : that pred (y_1))]


         {move 2}

         >>> close


      {move 2}

      >>> define line4 negexistev : Mp (Ug \
          line3, negunivev)


      line4 : [(negexistev_1 : that ~ (Exists 
          ([(z_4 : obj) => 
             ({def} ~ (pred (z_4)) : prop)]))) => 
          (--- : that ??)]


      {move 1}

      >>> close


   {move 1}

   >>> define Counterexample negunivev : Dneg \
       (Negintro line4)


   Counterexample : [(.pred_1 : [(x_2 
          : obj) => (--- : prop)]), (negunivev_1 
       : that ~ (Forall (.pred_1))) => 
       ({def} Dneg (Negintro ([(negexistev_3 
          : that ~ (Exists ([(z_6 : obj) => 
             ({def} ~ (.pred_1 (z_6)) : prop)]))) => 
          ({def} Ug ([(y_5 : obj) => 
             ({def} Dneg (Negintro ([(absurdhyp_7 
                : that ~ (.pred_1 (y_5))) => 
                ({def} y_5 Ei1 absurdhyp_7 
                Mp negexistev_3 : that ??)])) : that 
             .pred_1 (y_5))]) Mp negunivev_1 
          : that ??)])) : that Exists 
       ([(z_2 : obj) => 
          ({def} ~ (.pred_1 (z_2)) : prop)]))]


   Counterexample : [(.pred_1 : [(x_2 
          : obj) => (--- : prop)]), (negunivev_1 
       : that ~ (Forall (.pred_1))) => 
       (--- : that Exists ([(z_2 : obj) => 
          ({def} ~ (.pred_1 (z_2)) : prop)]))]


   {move 0}

   >>> define Counterexample0 pred, negunivev \
       : Counterexample negunivev


   Counterexample0 : [(pred_1 : [(x_2 
          : obj) => (--- : prop)]), (negunivev_1 
       : that ~ (Forall (pred_1))) => 
       ({def} Counterexample (negunivev_1) : that 
       Exists ([(z_2 : obj) => 
          ({def} ~ (pred_1 (z_2)) : prop)]))]


   Counterexample0 : [(pred_1 : [(x_2 
          : obj) => (--- : prop)]), (negunivev_1 
       : that ~ (Forall (pred_1))) => 
       (--- : that Exists ([(z_2 : obj) => 
          ({def} ~ (pred_1 (z_2)) : prop)]))]


   {move 0}
end Lestrade execution
\end{verbatim}





\section{Definite description}

In this section we define the uniqueness quantifier and definite description.

\begin{verbatim}

begin Lestrade execution

   >>> clearcurrent

{move 1}

   >>> declare x obj


   x : obj


   {move 1}

   >>> declare y obj


   y : obj


   {move 1}

   >>> declare pred [x => prop]


   pred : [(x_1 : obj) => (--- : prop)]


   {move 1}

   >>> define One pred : Exists [x => Forall \
          [y => (pred y) == y = x]]


   One : [(pred_1 : [(x_2 : obj) => 
          (--- : prop)]) => 
       ({def} Exists ([(x_2 : obj) => 
          ({def} Forall ([(y_3 : obj) => 
             ({def} pred_1 (y_3) == y_3 
             = x_2 : prop)]) : prop)]) : prop)]


   One : [(pred_1 : [(x_2 : obj) => 
          (--- : prop)]) => (--- : prop)]


   {move 0}

   >>> declare oneev that One pred


   oneev : that One (pred)


   {move 1}

   >>> postulate The oneev : obj


   The : [(.pred_1 : [(x_2 : obj) => 
          (--- : prop)]), (oneev_1 : that 
       One (.pred_1)) => (--- : obj)]


   {move 0}

   >>> postulate Theax oneev that pred (The \
       oneev)


   Theax : [(.pred_1 : [(x_2 : obj) => 
          (--- : prop)]), (oneev_1 : that 
       One (.pred_1)) => (--- : that .pred_1 
       (The (oneev_1)))]


   {move 0}

   >>> declare pred2 [x => prop]


   pred2 : [(x_1 : obj) => (--- : prop)]


   {move 1}

   >>> declare uniqueev that One pred


   uniqueev : that One (pred)


   {move 1}

   >>> declare impliesev that Forall [x => \
          (pred x) == pred2 x]


   impliesev : that Forall ([(x_2 : obj) => 
       ({def} pred (x_2) == pred2 (x_2) : prop)])


   {move 1}

   >>> goal that One pred2


   that One (pred2)


   {move 1}

   >>> open


      {move 2}

      >>> declare x1 obj


      x1 : obj


      {move 2}

      >>> declare y1 obj


      y1 : obj


      {move 2}

      >>> declare thisisit that Forall [y1 \
             => (pred y1) == y1 = x1]


      thisisit : that Forall ([(y1_2 : obj) => 
          ({def} pred (y1_2) == y1_2 = x1 
          : prop)])


      {move 2}

      >>> goal that Forall [y1 => (pred2 \
             y1) == y1 = x1]


      that Forall ([(y1 : obj) => 
          ({def} pred2 (y1) == y1 = x1 
          : prop)])


      {move 2}

      >>> open


         {move 3}

         >>> declare y2 obj


         y2 : obj


         {move 3}

         >>> open


            {move 4}

            >>> declare dir1 that pred2 y2


            dir1 : that pred2 (y2)


            {move 4}

            >>> define line1 dir1 : Iff2 \
                dir1, Ui y2 impliesev


            line1 : [(dir1_1 : that pred2 
                (y2)) => (--- : that pred 
                (y2))]


            {move 3}

            >>> define line2 dir1 : Iff1 \
                line1 dir1, Ui y2 thisisit


            line2 : [(dir1_1 : that pred2 
                (y2)) => (--- : that y2 
                = x1)]


            {move 3}

            >>> declare dir2 that y2 = x1


            dir2 : that y2 = x1


            {move 4}

            >>> define line3 dir2 : Iff2 \
                dir2, Ui y2 thisisit


            line3 : [(dir2_1 : that y2 
                = x1) => (--- : that pred 
                (y2))]


            {move 3}

            >>> define line4 dir2 : Iff1 \
                line3 dir2, Ui y2 impliesev


            line4 : [(dir2_1 : that y2 
                = x1) => (--- : that pred2 
                (y2))]


            {move 3}

            >>> close


         {move 3}

         >>> define line5 y2 : Dediff line2, line4


         line5 : [(y2_1 : obj) => (--- 
             : that pred2 (y2_1) == y2_1 
             = x1)]


         {move 2}

         >>> close


      {move 2}

      >>> define line6 thisisit : Ug line5


      line6 : [(.x1_1 : obj), (thisisit_1 
          : that Forall ([(y1_3 : obj) => 
             ({def} pred (y1_3) == y1_3 
             = .x1_1 : prop)])) => (--- 
          : that Forall ([(x''_2 : obj) => 
             ({def} pred2 (x''_2) == x''_2 
             = .x1_1 : prop)]))]


      {move 1}

      >>> define line7 thisisit : Fixform \
          (One pred2, Ei1 x1 line6 thisisit)


      line7 : [(.x1_1 : obj), (thisisit_1 
          : that Forall ([(y1_3 : obj) => 
             ({def} pred (y1_3) == y1_3 
             = .x1_1 : prop)])) => (--- 
          : that One (pred2))]


      {move 1}

      >>> close


   {move 1}

   >>> define Onequiv uniqueev impliesev \
       : Eg uniqueev line7


   Onequiv : [(.pred_1 : [(x_2 : obj) => 
          (--- : prop)]), (.pred2_1 
       : [(x_2 : obj) => (--- : prop)]), (uniqueev_1 
       : that One (.pred_1)), (impliesev_1 
       : that Forall ([(x_3 : obj) => 
          ({def} .pred_1 (x_3) == .pred2_1 
          (x_3) : prop)])) => 
       ({def} uniqueev_1 Eg [(.x1_2 : obj), (thisisit_2 
          : that Forall ([(y1_4 : obj) => 
             ({def} .pred_1 (y1_4) == y1_4 
             = .x1_2 : prop)])) => 
          ({def} One (.pred2_1) Fixform 
          .x1_2 Ei1 Ug ([(y2_5 : obj) => 
             ({def} Dediff ([(dir1_6 : that 
                .pred2_1 (y2_5)) => 
                ({def} dir1_6 Iff2 y2_5 Ui 
                impliesev_1 Iff1 y2_5 Ui thisisit_2 
                : that y2_5 = .x1_2)], [(dir2_6 
                : that y2_5 = .x1_2) => 
                ({def} dir2_6 Iff2 y2_5 Ui 
                thisisit_2 Iff1 y2_5 Ui impliesev_1 
                : that .pred2_1 (y2_5))]) : that 
             .pred2_1 (y2_5) == y2_5 = .x1_2)]) : that 
          One (.pred2_1))] : that One 
       (.pred2_1))]


   Onequiv : [(.pred_1 : [(x_2 : obj) => 
          (--- : prop)]), (.pred2_1 
       : [(x_2 : obj) => (--- : prop)]), (uniqueev_1 
       : that One (.pred_1)), (impliesev_1 
       : that Forall ([(x_3 : obj) => 
          ({def} .pred_1 (x_3) == .pred2_1 
          (x_3) : prop)])) => (--- 
       : that One (.pred2_1))]


   {move 0}
end Lestrade execution
\end{verbatim}

Here we develop the universal quantifier and definite description operator.  Note that the definite description operator and its axiom are for us primitives.


\end{document}

(* quit *)
