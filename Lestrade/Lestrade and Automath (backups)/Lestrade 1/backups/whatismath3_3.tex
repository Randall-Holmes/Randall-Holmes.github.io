\documentclass[12pt]{article}

\usepackage{amssymb}

\title{Implementation of Zermelo's work of 1908 in Lestrade:  Part II, Axiomatics of Zermelo set theory}

\author{M. Randall Holmes}

\begin{document}

\maketitle

\section{Introduction}
 
This document was originally titled as an essay on the proposition that mathematics is what can be done in Automath (as opposed to what can be done in ZFC, for example).  Such an essay is still in in my mind, but this particular document has transformed itself into the large project of implementing Zermelo's two important set theory papers of 1908 in Lestrade, with the further purpose of exploring the actual capabilities of Zermelo's system of 1908 as a mathematical foundation, which we think are perhaps underrated.

This is a new version of this document in modules, designed to make it possible to work more efficiently without repeated execution of slow log files when they do not need to be revisited.

\section{Basic concepts of set theory:  the axioms of extensionality and pairing}

In this section, we start to declare the basic notions and axioms of 1908 Zermelo set theory.  The membership relation is declared.  The axioms declared here are existence of the empty set, weak extensionality (atoms are allowed, following Zermelo's clear intentions in the 1908 paper), and pairing.

I have reedited this file to be a fairly direct implementation of Zermelo's axiomatics paper, currently just the first part discussing the axioms, but intended to include the development of theory of equivalence.  The way it was initially written was a correct implementation of the axioms, but concepts were not presented in the same order.  We will leave in the anachronistic demonstration of the basic property of the Kuratowski pair, which belongs at the same level of exposition.  I will add comments in this pass corresponding to paragraph  numbers in the Zermelo paper.

\begin{verbatim}Lestrade execution:


load whatismath1

>> Inspector Lestrade says:  Line is not a Lestrade command
>> Hit return to continue

quit
