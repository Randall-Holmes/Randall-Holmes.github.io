\documentclass[12pt]{article}

\usepackage{amssymb}

\title{Implementation of Zermelo's work of 1908 in Lestrade:  Part II, Axiomatics of Zermelo set theory}

\author{M. Randall Holmes}

\begin{document}

\maketitle

\section{Introduction}
 
This document was originally titled as an essay on the proposition that mathematics is what can be done in Automath (as opposed to what can be done in ZFC, for example).  Such an essay is still in in my mind, but this particular document has transformed itself into the large project of implementing Zermelo's two important set theory papers of 1908 in Lestrade, with the further purpose of exploring the actual capabilities of Zermelo's system of 1908 as a mathematical foundation, which we think are perhaps underrated.

This is a new version of this document in modules, designed to make it possible to work more efficiently without repeated execution of slow log files when they do not need to be revisited.

\section{Basic concepts of set theory:  the axioms of extensionality and pairing}

In this section, we start to declare the basic notions and axioms of 1908 Zermelo set theory.  The membership relation is declared.  The axioms declared here are existence of the empty set, weak extensionality (atoms are allowed, following Zermelo's clear intentions in the 1908 paper), and pairing.

I have reedited this file to be a fairly direct implementation of Zermelo's axiomatics paper, currently just the first part discussing the axioms, but intended to include the development of theory of equivalence.  The way it was initially written was a correct implementation of the axioms, but concepts were not presented in the same order.  We will leave in the anachronistic demonstration of the basic property of the Kuratowski pair, which belongs at the same level of exposition.  I will add comments in this pass corresponding to paragraph  numbers in the Zermelo paper.

\begin{verbatim}Lestrade execution:


load whatismath1
% Everything in Zermelo paragraph 1 is handled by Lestrade itself or in the logical preamble, except a definition of inequality, which I here provide but may or may not use.

clearcurrent


declare x obj

>> x: obj {move 1}



declare y obj

>> y: obj {move 1}



define =/= x y: ~(x=y)

>> =/=: [(x_1:obj),(y_1:obj) => (~((x_1 = y_1)):
>>      prop)]
>>   {move 0}


%%  Zermelo point 2:  declare membership relation, the empty set, and define sethood.  
%% the empty set is mentioned here by Zeremelo, too, before its official introduction
% in the axiom of elementary sets.


postulate E x y prop

>> E: [(x_1:obj),(y_1:obj) => (---:prop)]
>>   {move 0}



postulate 0 obj

>> 0: obj {move 0}



postulate Empty x that ~ (x E 0)

>> Empty: [(x_1:obj) => (---:that ~((x_1 E 0)))]
>>   {move 0}



define Isset x : (x=0) V Exists [y=>y E x] \
   



>> Isset: [(x_1:obj) => (((x_1 = 0) V Exists([(y_2:
>>         obj) => ((y_2 E x_1):prop)]))
>>      :prop)]
>>   {move 0}



declare u1 obj

>> u1: obj {move 1}



declare v1 obj

>> v1: obj {move 1}



declare nonemptyev that u1 E v1

>> nonemptyev: that (u1 E v1) {move 1}



define Inhabited nonemptyev: Fixform(Isset \
   v1,Add2(v1=0,Ei1 u1 nonemptyev))

>> Inhabited: [(.u1_1:obj),(.v1_1:obj),(nonemptyev_1:
>>      that (.u1_1 E .v1_1)) => ((Isset(.v1_1)
>>      Fixform ((.v1_1 = 0) Add2 (.u1_1 Ei1 nonemptyev_1))):
>>      that Isset(.v1_1))]
>>   {move 0}


% Zermelo point 3


declare z obj

>> z: obj {move 1}



define <<= x y : Forall [z=>(zE x) -> zE \
      y] \
   & (Isset x) & Isset y

>> <<=: [(x_1:obj),(y_1:obj) => ((Forall([(z_2:
>>         obj) => (((z_2 E x_1) -> (z_2 E y_1)):
>>         prop)])
>>      & (Isset(x_1) & Isset(y_1))):prop)]
>>   {move 0}



define disjoint x y : ~ Exists [z => (z E \
      x) & z E y] \
   & Isset x & Isset y

>> disjoint: [(x_1:obj),(y_1:obj) => ((~(Exists([(z_2:
>>         obj) => (((z_2 E x_1) & (z_2 E y_1)):
>>         prop)]))
>>      & (Isset(x_1) & Isset(y_1))):prop)]
>>   {move 0}


\end{verbatim}

We define the subset relation.  Note that we stipulate that it only holds between sets, which means that the atoms do not sneak into the power sets,
and the power set of an atom is the empty set.

The form of our definition of set agrees with what Zermelo says in the axiomatics paper:  it is a relation only between sets, not between the atoms which might exist. 

We further define the disjointness relation between sets.

\begin{verbatim}Lestrade execution:

clearcurrent


declare x obj

>> x: obj {move 1}



declare y obj

>> y: obj {move 1}



declare z obj

>> z: obj {move 1}



declare subsev1 that x <<= y

>> subsev1: that (x <<= y) {move 1}



declare subsev2 that y <<= z

>> subsev2: that (y <<= z) {move 1}



open

   declare u obj

>>    u: obj {move 2}



   open

      declare uinev that u E x

>>       uinev: that (u E x) {move 3}



      define line1 uinev: Mp uinev, Ui u \
         Simp1 subsev1

>>       line1: [(uinev_1:that (u E x)) => (---:
>>            that (u E y))]
>>         {move 2}



      define line2 uinev: Mp(line1 uinev, \
         Ui u Simp1 subsev2)

>>       line2: [(uinev_1:that (u E x)) => (---:
>>            that (u E z))]
>>         {move 2}



      close

   define linea3 u: Ded line2

>>    linea3: [(u_1:obj) => (---:that ((u_1
>>         E x) -> (u_1 E z)))]
>>      {move 1}



   close

define Transsub subsev1 subsev2:Fixform(x \
   <<= z,(Ug linea3) Conj (Simp1 Simp2 subsev1) \
   Conj (Simp2 Simp2 subsev2))

>> Transsub: [(.x_1:obj),(.y_1:obj),(subsev1_1:
>>      that (.x_1 <<= .y_1)),(.z_1:obj),(subsev2_1:
>>      that (.y_1 <<= .z_1)) => (((.x_1 <<= .z_1)
>>      Fixform (Ug([(u_4:obj) => (Ded([(uinev_5:
>>            that (u_4 E .x_1)) => (((uinev_5
>>            Mp (u_4 Ui Simp1(subsev1_1))) Mp
>>            (u_4 Ui Simp1(subsev2_1))):that
>>            (u_4 E .z_1))])
>>         :that ((u_4 E .x_1) -> (u_4 E .z_1)))])
>>      Conj (Simp1(Simp2(subsev1_1)) Conj Simp2(Simp2(subsev2_1))))):
>>      that (.x_1 <<= .z_1))]
>>   {move 0}


\end{verbatim}

We prove the transitive property of the subset relation.

\begin{verbatim}Lestrade execution:


declare issetx that Isset x

>> issetx: that Isset(x) {move 1}



open

   declare u obj

>>    u: obj {move 2}



   open

      declare uinev that u E x

>>       uinev: that (u E x) {move 3}



      define line1 uinev: uinev

>>       line1: [(uinev_1:that (u E x)) => (---:
>>            that (u E x))]
>>         {move 2}



      close

   define linea2 u: Ded line1

>>    linea2: [(u_1:obj) => (---:that ((u_1
>>         E x) -> (u_1 E x)))]
>>      {move 1}



   close

define Reflsubset issetx: Fixform(x<<= x, \
   (Ug linea2) Conj issetx Conj issetx)

>> Reflsubset: [(.x_1:obj),(issetx_1:that Isset(.x_1))
>>      => (((.x_1 <<= .x_1) Fixform (Ug([(u_4:
>>         obj) => (Ded([(uinev_5:that (u_4 E
>>            .x_1)) => (uinev_5:that (u_4 E .x_1))])
>>         :that ((u_4 E .x_1) -> (u_4 E .x_1)))])
>>      Conj (issetx_1 Conj issetx_1))):that (.x_1
>>      <<= .x_1))]
>>   {move 0}


\end{verbatim}

We prove the reflexive property of the subset relation (as a relation on sets).

\begin{verbatim}Lestrade execution:


declare inev that x E y

>> inev: that (x E y) {move 1}



declare subev that y <<= z

>> subev: that (y <<= z) {move 1}



define Mpsubs inev subev: Mp(inev,Ui x Simp1 \
   subev)

>> Mpsubs: [(.x_1:obj),(.y_1:obj),(inev_1:that
>>      (.x_1 E .y_1)),(.z_1:obj),(subev_1:that
>>      (.y_1 <<= .z_1)) => ((inev_1 Mp (.x_1
>>      Ui Simp1(subev_1))):that (.x_1 E .z_1))]
>>   {move 0}


\end{verbatim}

This is the frequently useful rule of inference taking $x \in y$ and $y \subseteq z$ to $x \in z$.



\begin{verbatim}Lestrade execution:


open

   declare X obj

>>    X: obj {move 2}



   open

      declare Xsetev that Isset X

>>       Xsetev: that Isset(X) {move 3}



      open

         declare u obj

>>          u: obj {move 4}



         open

            declare uinxev that u E X

>>             uinxev: that (u E X) {move 5}



            define line1 uinxev : uinxev


>>             line1: [(uinxev_1:that (u E X))
>>                  => (---:that (u E X))]
>>               {move 4}



            close

         define line2 u : Ded line1

>>          line2: [(u_1:obj) => (---:that ((u_1
>>               E X) -> (u_1 E X)))]
>>            {move 3}



         close

      define line3 : Ug line2

>>       line3: [(---:that Forall([(u_3:obj)
>>               => (((u_3 E X) -> (u_3 E X)):
>>               prop)]))
>>            ]
>>         {move 2}



      define line4 Xsetev : Fixform(X <<= \
         X,line3 Conj Xsetev Conj Xsetev)

>>       line4: [(Xsetev_1:that Isset(X)) =>
>>            (---:that (X <<= X))]
>>         {move 2}



      close

   define line5 X: Ded line4

>>    line5: [(X_1:obj) => (---:that (Isset(X_1)
>>         -> (X_1 <<= X_1)))]
>>      {move 1}



   close

define Subsetrefl : Ug line5

>> Subsetrefl: [(Ug([(X_1:obj) => (Ded([(Xsetev_2:
>>            that Isset(X_1)) => (((X_1 <<= X_1)
>>            Fixform (Ug([(u_5:obj) => (Ded([(uinxev_6:
>>                  that (u_5 E X_1)) => (uinxev_6:
>>                  that (u_5 E X_1))])
>>               :that ((u_5 E X_1) -> (u_5 E
>>               X_1)))])
>>            Conj (Xsetev_2 Conj Xsetev_2))):
>>            that (X_1 <<= X_1))])
>>         :that (Isset(X_1) -> (X_1 <<= X_1)))])
>>      :that Forall([(X_7:obj) => ((Isset(X_7)
>>         -> (X_7 <<= X_7)):prop)]))
>>      ]
>>   {move 0}


\end{verbatim}

I do not know why I proved reflexivity of the subset relation again, but I am going to leave it alone for now.

\begin{verbatim}Lestrade execution:


define Zeroisset : Fixform(Isset 0,Add1(Exists[x=>x \
      E 0] \
   ,Refleq 0))

>> Zeroisset: [((Isset(0) Fixform (Exists([(x_1:
>>         obj) => ((x_1 E 0):prop)])
>>      Add1 Refleq(0))):that Isset(0))]
>>   {move 0}


\end{verbatim}

The empty set is a set.



\begin{verbatim}Lestrade execution:


declare firstev that Isset x

>> firstev: that Isset(x) {move 1}



declare secondev that Isset y

>> secondev: that Isset(y) {move 1}



declare thirdev that ~(x <<= y)

>> thirdev: that ~((x <<= y)) {move 1}



open

   define linec1 : Counterexample(Notconj(thirdev, \
      Conj firstev secondev))

>>    linec1: [(---:that Exists([(z_3:obj) =>
>>            (~(((z_3 E x) -> (z_3 E y))):prop)]))
>>         ]
>>      {move 1}



   open

      declare z1 obj

>>       z1: obj {move 3}



      declare u1 obj

>>       u1: obj {move 3}



      declare evu1 that ~((u1 E x) -> u1 \
         E y)

>>       evu1: that ~(((u1 E x) -> (u1 E y)))
>>         {move 3}



      define linec2 u1 evu1 : Ei1 u1, Conj(Notimp2 \
         evu1,Notimp1 evu1)

>>       linec2: [(u1_1:obj),(evu1_1:that ~(((u1_1
>>            E x) -> (u1_1 E y)))) => (---:that
>>            Exists([(x_3:obj) => (((x_3 E x)
>>               & ~((x_3 E y))):prop)]))
>>            ]
>>         {move 2}



      close

   define Subsetcounter1 : Eg linec1,linec2


>>    Subsetcounter1: [(---:that Exists([(x_4:
>>            obj) => (((x_4 E x) & ~((x_4 E y))):
>>            prop)]))
>>         ]
>>      {move 1}



   close

define Subsetcounter firstev secondev thirdev: \
   Subsetcounter1

>> Subsetcounter: [(.x_1:obj),(firstev_1:that
>>      Isset(.x_1)),(.y_1:obj),(secondev_1:that
>>      Isset(.y_1)),(thirdev_1:that ~((.x_1 <<=
>>      .y_1))) => ((Counterexample((thirdev_1
>>      Notconj (firstev_1 Conj secondev_1)))
>>      Eg [(u1_6:obj),(evu1_6:that ~(((u1_6 E
>>         .x_1) -> (u1_6 E .y_1)))) => ((u1_6
>>         Ei1 (Notimp2(evu1_6) Conj Notimp1(evu1_6))):
>>         that Exists([(x_8:obj) => (((x_8 E
>>            .x_1) & ~((x_8 E .y_1))):prop)]))
>>         ])
>>      :that Exists([(x_9:obj) => (((x_9 E .x_1)
>>         & ~((x_9 E .y_1))):prop)]))
>>      ]
>>   {move 0}


\end{verbatim}

I don't think I used this result, but it is nice to have it in the library (existence of witnesses to failures of inclusion).

\begin{verbatim}Lestrade execution:

%% nothing corresponds to the discussion 
%% of definiteness in Zermelo point 4 
% (or this is handled by the basics of Lestrade).

%% Zermelo point 4 also includes
%% the axiom of extensionality
%% and the axiom of pairing
% (elementary sets).


declare setev1 that Isset x

>> setev1: that Isset(x) {move 1}



declare setev2 that Isset y

>> setev2: that Isset(y) {move 1}



declare extev [z=>that (z E x) == (z E y)] \
   



>> extev: [(z_1:obj) => (---:that ((z_1 E x)
>>      == (z_1 E y)))]
>>   {move 1}



postulate Ext setev1 setev2 extev that x=y


>> Ext: [(.x_1:obj),(setev1_1:that Isset(.x_1)),
>>      (.y_1:obj),(setev2_1:that Isset(.y_1)),
>>      (extev_1:[(z_2:obj) => (---:that ((z_2
>>         E .x_1) == (z_2 E .y_1)))])
>>      => (---:that (.x_1 = .y_1))]
>>   {move 0}


clearcurrent


declare x obj

>> x: obj {move 1}



declare y obj

>> y: obj {move 1}



declare z obj

>> z: obj {move 1}



declare setev that z E x

>> setev: that (z E x) {move 1}



declare setev2 that z E y

>> setev2: that (z E y) {move 1}



declare extev1 [setev => that z E y] \
   



>> extev1: [(setev_1:that (z E x)) => (---:that
>>      (z E y))]
>>   {move 1}



declare extev2 [setev2 => that z E y] \
   



>> extev2: [(setev2_1:that (z E y)) => (---:
>>      that (z E y))]
>>   {move 1}



postulate Ext1 setev extev1, extev2 that \
   x=y

>> Ext1: [(.z_1:obj),(.x_1:obj),(setev_1:that
>>      (.z_1 E .x_1)),(.y_1:obj),(extev1_1:[(setev_2:
>>         that (.z_1 E .x_1)) => (---:that (.z_1
>>         E .y_1))]),
>>      (extev2_1:[(setev2_3:that (.z_1 E .y_1))
>>         => (---:that (.z_1 E .y_1))])
>>      => (---:that (.x_1 = .y_1))]
>>   {move 0}


\end{verbatim}

Above we have declared the membership relation $x \in y$, the empty set 0 and the axiom that it has no members, defined sets as elements and 0, and stated the weak axiom of extensionality:  sets which have the same extension are equal.

The definition of ``set" (and the possibility of objects which are not sets) is clearly stated in Zermelo's axiomatics paper.

The alternative formulation {\tt Ext1} is better in not involving logic primitives, which would add a little more burden to needed definitions.  I should define one of these in terms of the other.

The  rule of inference {\tt Inhabited} from $x \in y$ to sethood of $y$ is often useful.

\begin{verbatim}Lestrade execution:


declare sev1 that x <<= y

>> sev1: that (x <<= y) {move 1}



declare sev2 that y <<= x

>> sev2: that (y <<= x) {move 1}



open

   declare u obj

>>    u: obj {move 2}



   open

      declare ineva1 that u E x

>>       ineva1: that (u E x) {move 3}



      declare ineva2 that u E y

>>       ineva2: that (u E y) {move 3}



      define dir1 ineva1 : Mpsubs ineva1 \
         sev1

>>       dir1: [(ineva1_1:that (u E x)) => (---:
>>            that (u E y))]
>>         {move 2}



      define dir2 ineva2 : Mpsubs ineva2 \
         sev2

>>       dir2: [(ineva2_1:that (u E y)) => (---:
>>            that (u E x))]
>>         {move 2}



      close

   define bothways u: Dediff dir1, dir2

>>    bothways: [(u_1:obj) => (---:that ((u_1
>>         E x) == (u_1 E y)))]
>>      {move 1}



   close

define Antisymsub sev1 sev2 : Ext(Simp1(Simp2 \
   sev1),Simp2(Simp2 sev1),bothways)

>> Antisymsub: [(.x_1:obj),(.y_1:obj),(sev1_1:
>>      that (.x_1 <<= .y_1)),(sev2_1:that (.y_1
>>      <<= .x_1)) => (Ext(Simp1(Simp2(sev1_1)),
>>      Simp2(Simp2(sev1_1)),[(u_4:obj) => (Dediff([(ineva1_5:
>>            that (u_4 E .x_1)) => ((ineva1_5
>>            Mpsubs sev1_1):that (u_4 E .y_1))]
>>         ,[(ineva2_6:that (u_4 E .y_1)) => ((ineva2_6
>>            Mpsubs sev2_1):that (u_4 E .x_1))])
>>         :that ((u_4 E .x_1) == (u_4 E .y_1)))])
>>      :that (.x_1 = .y_1))]
>>   {move 0}


\end{verbatim}

We prove that the subset relation is antisymmetric (which is an alternative way in which Zermelo states extensionality).

\begin{verbatim}Lestrade execution:

clearcurrent


declare x obj

>> x: obj {move 1}



declare y obj

>> y: obj {move 1}



declare z obj

>> z: obj {move 1}



postulate ; x y obj

>> ;: [(x_1:obj),(y_1:obj) => (---:obj)]
>>   {move 0}



postulate Pair x y that Forall [z=>(z E x;y) \
      == (z = x) V z = y] \
   



>> Pair: [(x_1:obj),(y_1:obj) => (---:that Forall([(z_2:
>>         obj) => (((z_2 E (x_1 ; y_1)) == ((z_2
>>         = x_1) V (z_2 = y_1))):prop)]))
>>      ]
>>   {move 0}



define Usc x : x;x

>> Usc: [(x_1:obj) => ((x_1 ; x_1):obj)]
>>   {move 0}



define $ x y : (x;x);(x;y)

>> $: [(x_1:obj),(y_1:obj) => (((x_1 ; x_1)
>>      ; (x_1 ; y_1)):obj)]
>>   {move 0}


\end{verbatim}

Above we present the operation of unordered pairing and the axiom of pairing which determines the extension of the pair.  We write {\tt x ; y} for $\{x,y\}$.  We define
 the singleton operation, borrowing Rosser's notation ${\tt USC}(x)$ for $\{x\}$.

We define the Kuratowski ordered pair, using the notation $x \$ y$ for $(x,y)$.  This is of course a notion unknown to Zermelo, but it is a formal feature of his system even if he did not know about it.

Our treatment differs from Zermelo's in treating the singleton as a special case of the unordered pair.  He treats the two as separate constructions.

\section{Developments from pairing, including the properties of the ordered pair}

Herein we do some development work with unordered pairs, singletons, and Kuratowski ordered pairs.  The results on Kuratowski ordered pairs are anachronistic, having nothing to do with Zermelo's development, and we do not make use  of these in implementing Zermelo's proofs;  lemmas provided about singletons and ordered pairs are used extensively, though it should be noted that strictly speaking Zermelo's well-ordering theorem proof does not actually depend on the axiom of pairing (pairs of objects taken from a set given in advance are provided by separation, and this is all that is actually needed in Zermelo's proof;  we might at some point revise the development here to highlight this fact).

\begin{verbatim}Lestrade execution:

clearcurrent

% Zermelo point 5 is addressed here in some detail.


declare x obj

>> x: obj {move 1}



declare y obj

>> y: obj {move 1}



declare inev that y E x;x

>> inev: that (y E (x ; x)) {move 1}



open

   define line1 : Ui (y, Pair x x)

>>    line1: [(---:that ((y E (x ; x)) == ((y
>>         = x) V (y = x))))]
>>      {move 1}



   define line2 : Iff1 inev line1

>>    line2: [(---:that ((y = x) V (y = x)))]
>>      {move 1}



   define line3 : Oridem line2

>>    line3: [(---:that (y = x))]
>>      {move 1}



   close

define Inusc1 inev : line3

>> Inusc1: [(.y_1:obj),(.x_1:obj),(inev_1:that
>>      (.y_1 E (.x_1 ; .x_1))) => (Oridem((inev_1
>>      Iff1 (.y_1 Ui (.x_1 Pair .x_1)))):that
>>      (.y_1 = .x_1))]
>>   {move 0}


clearcurrent


declare x obj

>> x: obj {move 1}



open

   define line1 : Add1 (x=x,Refleq x)

>>    line1: [(---:that ((x = x) V (x = x)))]
>>      {move 1}



   define line2 : Iff2 (line1,Ui(x,Pair x \
      x))

>>    line2: [(---:that (x E (x ; x)))]
>>      {move 1}



   close

define Inusc2 x : line2

>> Inusc2: [(x_1:obj) => ((((x_1 = x_1) Add1
>>      Refleq(x_1)) Iff2 (x_1 Ui (x_1 Pair x_1))):
>>      that (x_1 E (x_1 ; x_1)))]
>>   {move 0}


clearcurrent


declare x obj

>> x: obj {move 1}



declare y obj

>> y: obj {move 1}



open

   define scratch1 : Ui x (Pair x y)

>>    scratch1: [(---:that ((x E (x ; y)) ==
>>         ((x = x) V (x = y))))]
>>      {move 1}



   define scratch2 : Add1 (x=y,Refleq x)


>>    scratch2: [(---:that ((x = x) V (x = y)))]
>>      {move 1}



   define scratch3 :Iff2(scratch2 ,scratch1)


>>    scratch3: [(---:that (x E (x ; y)))]
>>      {move 1}



   close

define Inpair1 x y:scratch3

>> Inpair1: [(x_1:obj),(y_1:obj) => ((((x_1
>>      = y_1) Add1 Refleq(x_1)) Iff2 (x_1 Ui
>>      (x_1 Pair y_1))):that (x_1 E (x_1 ; y_1)))]
>>   {move 0}


clearcurrent


declare x obj

>> x: obj {move 1}



declare y obj

>> y: obj {move 1}



open

   define scratch1: Ui y (Pair x y)

>>    scratch1: [(---:that ((y E (x ; y)) ==
>>         ((y = x) V (y = y))))]
>>      {move 1}



   define scratch2: Add2 (y=x,Refleq y)

>>    scratch2: [(---:that ((y = x) V (y = y)))]
>>      {move 1}



   define scratch3 : Iff2 scratch2 scratch1


>>    scratch3: [(---:that (y E (x ; y)))]
>>      {move 1}



   close

define Inpair2 x y:scratch3

>> Inpair2: [(x_1:obj),(y_1:obj) => ((((y_1
>>      = x_1) Add2 Refleq(y_1)) Iff2 (y_1 Ui
>>      (x_1 Pair y_1))):that (y_1 E (x_1 ; y_1)))]
>>   {move 0}


clearcurrent


declare x obj

>> x: obj {move 1}



open

   declare y obj

>>    y: obj {move 2}



   open

      declare inev1 that y E Usc x

>>       inev1: that (y E Usc(x)) {move 3}



      declare inev2 that y=x

>>       inev2: that (y = x) {move 3}



      define dir1 inev1 : Inusc1 inev1

>>       dir1: [(inev1_1:that (y E Usc(x)))
>>            => (---:that (y = x))]
>>         {move 2}



      define line3 inev2: Eqsymm inev2

>>       line3: [(inev2_1:that (y = x)) => (---:
>>            that (x = y))]
>>         {move 2}



      define line4 : Fixform(x E Usc x, Inusc2 \
         x)

>>       line4: [(---:that (x E Usc(x)))]
>>         {move 2}



      declare z1 obj

>>       z1: obj {move 3}



      define dir2 inev2 : Subs (Eqsymm inev2, \
         [z1 =>z1 E Usc x] \
         , line4)

>>       dir2: [(inev2_1:that (y = x)) => (---:
>>            that (y E Usc(x)))]
>>         {move 2}



      define inuscone : Fixform((y E Usc \
         x)==y=x, Dediff dir1, dir2)

>>       inuscone: [(---:that ((y E Usc(x))
>>            == (y = x)))]
>>         {move 2}



      close

   define inuscone2 y:inuscone

>>    inuscone2: [(y_1:obj) => (---:that ((y_1
>>         E Usc(x)) == (y_1 = x)))]
>>      {move 1}



   define one1 : Ug inuscone2

>>    one1: [(---:that Forall([(y_2:obj) =>
>>            (((y_2 E Usc(x)) == (y_2 = x)):prop)]))
>>         ]
>>      {move 1}



   declare w obj

>>    w: obj {move 2}



   declare y2 obj

>>    y2: obj {move 2}



   define one2 : Fixform(One [w=>w E Usc \
         x] \
      ,Ei(x, [w=>Forall[y2 => (y2 E Usc x)==y2=w] \
         ] \
      ,one1))

>>    one2: [(---:that One([(w_4:obj) => ((w_4
>>            E Usc(x)):prop)]))
>>         ]
>>      {move 1}



   define one3 : Theax one2

>>    one3: [(---:that (The(one2) E Usc(x)))]
>>      {move 1}



   define one4 : Inusc1 one3

>>    one4: [(---:that (The(one2) = x))]
>>      {move 1}



   close

define Theeltthm x:one2

>> Theeltthm: [(x_1:obj) => ((One([(w_2:obj)
>>         => ((w_2 E Usc(x_1)):prop)])
>>      Fixform Ei(x_1,[(w_3:obj) => (Forall([(y2_4:
>>            obj) => (((y2_4 E Usc(x_1)) == (y2_4
>>            = w_3)):prop)])
>>         :prop)]
>>      ,Ug([(y_6:obj) => ((((y_6 E Usc(x_1))
>>         == (y_6 = x_1)) Fixform Dediff([(inev1_7:
>>            that (y_6 E Usc(x_1))) => (Inusc1(inev1_7):
>>            that (y_6 = x_1))]
>>         ,[(inev2_8:that (y_6 = x_1)) => (Subs(Eqsymm(inev2_8),
>>            [(z1_9:obj) => ((z1_9 E Usc(x_1)):
>>               prop)]
>>            ,((x_1 E Usc(x_1)) Fixform Inusc2(x_1))):
>>            that (y_6 E Usc(x_1)))]))
>>         :that ((y_6 E Usc(x_1)) == (y_6 = x_1)))]))
>>      ):that One([(w_10:obj) => ((w_10 E Usc(x_1)):
>>         prop)]))
>>      ]
>>   {move 0}



define Theelt x: Fixform(The(Theeltthm x)=x, \
   one4)

>> Theelt: [(x_1:obj) => (((The(Theeltthm(x_1))
>>      = x_1) Fixform Inusc1(Theax((One([(w_13:
>>         obj) => ((w_13 E Usc(x_1)):prop)])
>>      Fixform Ei(x_1,[(w_14:obj) => (Forall([(y2_15:
>>            obj) => (((y2_15 E Usc(x_1)) ==
>>            (y2_15 = w_14)):prop)])
>>         :prop)]
>>      ,Ug([(y_17:obj) => ((((y_17 E Usc(x_1))
>>         == (y_17 = x_1)) Fixform Dediff([(inev1_18:
>>            that (y_17 E Usc(x_1))) => (Inusc1(inev1_18):
>>            that (y_17 = x_1))]
>>         ,[(inev2_19:that (y_17 = x_1)) => (Subs(Eqsymm(inev2_19),
>>            [(z1_20:obj) => ((z1_20 E Usc(x_1)):
>>               prop)]
>>            ,((x_1 E Usc(x_1)) Fixform Inusc2(x_1))):
>>            that (y_17 E Usc(x_1)))]))
>>         :that ((y_17 E Usc(x_1)) == (y_17 =
>>         x_1)))]))
>>      )))):that (The(Theeltthm(x_1)) = x_1))]
>>   {move 0}


\end{verbatim}

We prove that $y \in \{x\}$ iff $y=x$, and that $(\theta y: y \in \{x\})=x$.  This involves careful manipulations of environments and forms of statements to avoid blowup.

We should also prove that if there is only one element in a set, it is the singleton of its element.

In the following block, we develop the operation which sends $x$ and $\{x,y\}$ to $y$.  It is not immediately clear (except to common sense) that there
{\em is\/} such an operation.  This might be useful for Zermelo's implementation of equivalence, later in this file.  I'm of two minds as to whether it will actually be useful, but it was an interesting exercise building the proofs and definitions.

\begin{verbatim}Lestrade execution:

clearcurrent


declare x obj

>> x: obj {move 1}



declare y obj

>> y: obj {move 1}



declare z obj

>> z: obj {move 1}



goal that One[z => (z E x;y) & (z=x) == y=x] \
   



>> Goal: that One([(z_39:obj) => (((z_39 E (x
>>      ; y)) & ((z_39 = x) == (y = x))):prop)])
>>   

goal that Forall[z => ((z E x;y) & (z=x) \
      == y=x) == z=y] \
   



>> Goal: that Forall([(z_68:obj) => ((((z_68
>>      E (x ; y)) & ((z_68 = x) == (y = x)))
>>      == (z_68 = y)):prop)])
>>   

open

   declare z1 obj

>>    z1: obj {move 2}



   open

      declare z2 obj

>>       z2: obj {move 3}



      declare dir1 that (z1 E x;y) & (z1=x) \
         == y=x

>>       dir1: that ((z1 E (x ; y)) & ((z1 =
>>         x) == (y = x))) {move 3}



      declare dir2 that z1 = y

>>       dir2: that (z1 = y) {move 3}



      define line1 dir1: Iff1 Simp1 dir1, \
         Ui z1, Pair x y

>>       line1: [(dir1_1:that ((z1 E (x ; y))
>>            & ((z1 = x) == (y = x)))) => (---:
>>            that ((z1 = x) V (z1 = y)))]
>>         {move 2}



      open

         declare case1 that z1=x

>>          case1: that (z1 = x) {move 4}



         define line2 case1: Iff1 case1 Simp2 \
            dir1

>>          line2: [(case1_1:that (z1 = x))
>>               => (---:that (y = x))]
>>            {move 3}



         define line3 case1: Subs1 Eqsymm \
            line2 case1 case1

>>          line3: [(case1_1:that (z1 = x))
>>               => (---:that (z1 = y))]
>>            {move 3}



         declare case2 that z1=y

>>          case2: that (z1 = y) {move 4}



         define line4 case2: case2

>>          line4: [(case2_1:that (z1 = y))
>>               => (---:that (z1 = y))]
>>            {move 3}



         close

      define line5 dir1: Cases line1 dir1 \
         line3, line4

>>       line5: [(dir1_1:that ((z1 E (x ; y))
>>            & ((z1 = x) == (y = x)))) => (---:
>>            that (z1 = y))]
>>         {move 2}



      define line6: Conj Inpair2 x y,Iffrefl \
         (y=x)

>>       line6: [(---:that ((y E (x ; y)) &
>>            ((y = x) == (y = x))))]
>>         {move 2}



      define line7 dir2: Subs Eqsymm dir2 \
         [z2 => (z2 E x;y) & (z2=x) == y=x] \
         line6

>>       line7: [(dir2_1:that (z1 = y)) => (---:
>>            that ((z1 E (x ; y)) & ((z1 = x)
>>            == (y = x))))]
>>         {move 2}



      close

   define line8 z1: Dediff line5, line7

>>    line8: [(z1_1:obj) => (---:that (((z1_1
>>         E (x ; y)) & ((z1_1 = x) == (y = x)))
>>         == (z1_1 = y)))]
>>      {move 1}



   close

define Theother1 x y: Ug line8

>> Theother1: [(x_1:obj),(y_1:obj) => (Ug([(z1_3:
>>         obj) => (Dediff([(dir1_4:that ((z1_3
>>            E (x_1 ; y_1)) & ((z1_3 = x_1) ==
>>            (y_1 = x_1)))) => (Cases((Simp1(dir1_4)
>>            Iff1 (z1_3 Ui (x_1 Pair y_1))),[(case1_6:
>>               that (z1_3 = x_1)) => ((Eqsymm((case1_6
>>               Iff1 Simp2(dir1_4))) Subs1 case1_6):
>>               that (z1_3 = y_1))]
>>            ,[(case2_8:that (z1_3 = y_1)) =>
>>               (case2_8:that (z1_3 = y_1))])
>>            :that (z1_3 = y_1))]
>>         ,[(dir2_9:that (z1_3 = y_1)) => (Subs(Eqsymm(dir2_9),
>>            [(z2_10:obj) => (((z2_10 E (x_1
>>               ; y_1)) & ((z2_10 = x_1) == (y_1
>>               = x_1))):prop)]
>>            ,((x_1 Inpair2 y_1) Conj Iffrefl((y_1
>>            = x_1)))):that ((z1_3 E (x_1 ; y_1))
>>            & ((z1_3 = x_1) == (y_1 = x_1))))])
>>         :that (((z1_3 E (x_1 ; y_1)) & ((z1_3
>>         = x_1) == (y_1 = x_1))) == (z1_3 =
>>         y_1)))])
>>      :that Forall([(z1_11:obj) => ((((z1_11
>>         E (x_1 ; y_1)) & ((z1_11 = x_1) ==
>>         (y_1 = x_1))) == (z1_11 = y_1)):prop)]))
>>      ]
>>   {move 0}



declare w obj

>> w: obj {move 1}



define Theother2 x y:Fixform One[z => ((z \
      E x;y) & (z=x) == y=x)] \
   , Ei y, [w => Forall[z => ((z E x;y) & (z=x) \
         == y=x)==z=w] \
      ] \
   , Theother1 x y

>> Theother2: [(x_1:obj),(y_1:obj) => ((One([(z_2:
>>         obj) => (((z_2 E (x_1 ; y_1)) & ((z_2
>>         = x_1) == (y_1 = x_1))):prop)])
>>      Fixform Ei(y_1,[(w_3:obj) => (Forall([(z_4:
>>            obj) => ((((z_4 E (x_1 ; y_1)) &
>>            ((z_4 = x_1) == (y_1 = x_1))) ==
>>            (z_4 = w_3)):prop)])
>>         :prop)]
>>      ,(x_1 Theother1 y_1))):that One([(z_5:
>>         obj) => (((z_5 E (x_1 ; y_1)) & ((z_5
>>         = x_1) == (y_1 = x_1))):prop)]))
>>      ]
>>   {move 0}



declare ispairev that z = x;y

>> ispairev: that (z = (x ; y)) {move 1}



declare z1 obj

>> z1: obj {move 1}



define Theother x ispairev: The (Theother2 \
   x y)

>> Theother: [(x_1:obj),(.z_1:obj),(.y_1:obj),
>>      (ispairev_1:that (.z_1 = (x_1 ; .y_1)))
>>      => (The((x_1 Theother2 .y_1)):obj)]
>>   {move 0}



open

   define it : Theother x ispairev

>>    it: [(---:obj)]
>>      {move 1}



   define line9: Fixform ((it E x;y) & (it \
      = x) == y=x, Theax(Theother2 x y))

>>    line9: [(---:that ((it E (x ; y)) & ((it
>>         = x) == (y = x))))]
>>      {move 1}



   define line10: Iff1 Simp1 line9, Ui it, \
      Pair x y

>>    line10: [(---:that ((it = x) V (it = y)))]
>>      {move 1}



   open

      declare case1 that it = x

>>       case1: that (it = x) {move 3}



      define line11 case1: Iff1 case1 Simp2 \
         line9

>>       line11: [(case1_1:that (it = x)) =>
>>            (---:that (y = x))]
>>         {move 2}



      define line12 case1: Subs1 Eqsymm line11 \
         case1 case1

>>       line12: [(case1_1:that (it = x)) =>
>>            (---:that (it = y))]
>>         {move 2}



      declare case2 that it = y

>>       case2: that (it = y) {move 3}



      define line13 case2:case2

>>       line13: [(case2_1:that (it = y)) =>
>>            (---:that (it = y))]
>>         {move 2}



      close

   define line14 : Cases line10 line12, line13


>>    line14: [(---:that (it = y))]
>>      {move 1}



   close

define Theother3 x ispairev: line14

>> Theother3: [(x_1:obj),(.z_1:obj),(.y_1:obj),
>>      (ispairev_1:that (.z_1 = (x_1 ; .y_1)))
>>      => (Cases((Simp1(((((x_1 Theother ispairev_1)
>>      E (x_1 ; .y_1)) & (((x_1 Theother ispairev_1)
>>      = x_1) == (.y_1 = x_1))) Fixform Theax((x_1
>>      Theother2 .y_1)))) Iff1 ((x_1 Theother
>>      ispairev_1) Ui (x_1 Pair .y_1))),[(case1_4:
>>         that ((x_1 Theother ispairev_1) = x_1))
>>         => ((Eqsymm((case1_4 Iff1 Simp2(((((x_1
>>         Theother ispairev_1) E (x_1 ; .y_1))
>>         & (((x_1 Theother ispairev_1) = x_1)
>>         == (.y_1 = x_1))) Fixform Theax((x_1
>>         Theother2 .y_1)))))) Subs1 case1_4):
>>         that ((x_1 Theother ispairev_1) = .y_1))]
>>      ,[(case2_7:that ((x_1 Theother ispairev_1)
>>         = .y_1)) => (case2_7:that ((x_1 Theother
>>         ispairev_1) = .y_1))])
>>      :that ((x_1 Theother ispairev_1) = .y_1))]
>>   {move 0}



define Theother4 x y: Theother3 x Refleq \
   (x;y)

>> Theother4: [(x_1:obj),(y_1:obj) => ((x_1
>>      Theother3 Refleq((x_1 ; y_1))):that ((x_1
>>      Theother Refleq((x_1 ; y_1))) = y_1))]
>>   {move 0}


\end{verbatim}

Our aim in the next blocks of code is to characterize projections of the pair.  $x$ is the unique object which belongs to all elements of $x;y$.  $y$ is the unique object which belongs to exactly one element of $x;y$.  These theorems allow us to prove that an ordered pair is determined by its projections.

\begin{verbatim}Lestrade execution:

clearcurrent


declare x obj

>> x: obj {move 1}



declare y obj

>> y: obj {move 1}



open

   declare z obj

>>    z: obj {move 2}



   open

      declare inev that z E x$y

>>       inev: that (z E (x $ y)) {move 3}



      open

         define line1 : Ui z (Pair x;x x;y)


>>          line1: [(---:that ((z E ((x ; x)
>>               ; (x ; y))) == ((z = (x ; x))
>>               V (z = (x ; y)))))]
>>            {move 3}



         define line2 : Iff1 inev line1

>>          line2: [(---:that ((z = (x ; x))
>>               V (z = (x ; y))))]
>>            {move 3}



         declare eqev1 that z=x;x

>>          eqev1: that (z = (x ; x)) {move
>>            4}



         declare w obj

>>          w: obj {move 4}



         define dir1 eqev1: Subs1 (Eqsymm \
            eqev1, Inusc2 x)

>>          dir1: [(eqev1_1:that (z = (x ; x)))
>>               => (---:that (x E z))]
>>            {move 3}



         declare eqev2 that z=x;y

>>          eqev2: that (z = (x ; y)) {move
>>            4}



         define dir2 eqev2: Subs1(Eqsymm \
            eqev2, Inpair1 x y)

>>          dir2: [(eqev2_1:that (z = (x ; y)))
>>               => (---:that (x E z))]
>>            {move 3}



         define line3 : Cases line2 dir1, \
            dir2

>>          line3: [(---:that (x E z))]
>>            {move 3}



         close

      define scratch inev: line3

>>       scratch: [(inev_1:that (z E (x $ y)))
>>            => (---:that (x E z))]
>>         {move 2}



      define scratch2 : Ded scratch

>>       scratch2: [(---:that ((z E (x $ y))
>>            -> (x E z)))]
>>         {move 2}



      close

   define scratch3 z: scratch2

>>    scratch3: [(z_1:obj) => (---:that ((z_1
>>         E (x $ y)) -> (x E z_1)))]
>>      {move 1}



   close

define Firstprojthm1 x y:Ug scratch3

>> Firstprojthm1: [(x_1:obj),(y_1:obj) => (Ug([(z_3:
>>         obj) => (Ded([(inev_4:that (z_3 E (x_1
>>            $ y_1))) => (Cases((inev_4 Iff1
>>            (z_3 Ui ((x_1 ; x_1) Pair (x_1 ;
>>            y_1)))),[(eqev1_6:that (z_3 = (x_1
>>               ; x_1))) => ((Eqsymm(eqev1_6)
>>               Subs1 Inusc2(x_1)):that (x_1
>>               E z_3))]
>>            ,[(eqev2_8:that (z_3 = (x_1 ; y_1)))
>>               => ((Eqsymm(eqev2_8) Subs1 (x_1
>>               Inpair1 y_1)):that (x_1 E z_3))])
>>            :that (x_1 E z_3))])
>>         :that ((z_3 E (x_1 $ y_1)) -> (x_1
>>         E z_3)))])
>>      :that Forall([(z_10:obj) => (((z_10 E
>>         (x_1 $ y_1)) -> (x_1 E z_10)):prop)]))
>>      ]
>>   {move 0}


clearcurrent


declare x obj

>> x: obj {move 1}



declare y obj

>> y: obj {move 1}



open

   declare w obj

>>    w: obj {move 2}



   open

      declare z obj

>>       z: obj {move 3}



      declare firstev that Forall [z=>(z \
            E x$y)->w E z] \
         



>>       firstev: that Forall([(z_1:obj) =>
>>            (((z_1 E (x $ y)) -> (w E z_1)):
>>            prop)])
>>         {move 3}



      define line1 firstev : Ui(Usc x,firstev)


>>       line1: [(firstev_1:that Forall([(z_2:
>>               obj) => (((z_2 E (x $ y)) ->
>>               (w E z_2)):prop)]))
>>            => (---:that ((Usc(x) E (x $ y))
>>            -> (w E Usc(x))))]
>>         {move 2}



      define line2 firstev: Fixform((Usc \
         x) E x $ y,Inpair1(x;x,x;y))

>>       line2: [(firstev_1:that Forall([(z_2:
>>               obj) => (((z_2 E (x $ y)) ->
>>               (w E z_2)):prop)]))
>>            => (---:that (Usc(x) E (x $ y)))]
>>         {move 2}



      define line3 firstev: Mp (line2 firstev, \
         line1 firstev)

>>       line3: [(firstev_1:that Forall([(z_2:
>>               obj) => (((z_2 E (x $ y)) ->
>>               (w E z_2)):prop)]))
>>            => (---:that (w E Usc(x)))]
>>         {move 2}



      define line4 firstev : Inusc1 line3 \
         firstev

>>       line4: [(firstev_1:that Forall([(z_2:
>>               obj) => (((z_2 E (x $ y)) ->
>>               (w E z_2)):prop)]))
>>            => (---:that (w = x))]
>>         {move 2}



      close

   define line5 w: Ded line4

>>    line5: [(w_1:obj) => (---:that (Forall([(z_6:
>>            obj) => (((z_6 E (x $ y)) -> (w_1
>>            E z_6)):prop)])
>>         -> (w_1 = x)))]
>>      {move 1}



   close

define Firstprojthm2 x y: Ug line5

>> Firstprojthm2: [(x_1:obj),(y_1:obj) => (Ug([(w_4:
>>         obj) => (Ded([(firstev_6:that Forall([(z_7:
>>               obj) => (((z_7 E (x_1 $ y_1))
>>               -> (w_4 E z_7)):prop)]))
>>            => (Inusc1((((Usc(x_1) E (x_1 $
>>            y_1)) Fixform ((x_1 ; x_1) Inpair1
>>            (x_1 ; y_1))) Mp (Usc(x_1) Ui firstev_6))):
>>            that (w_4 = x_1))])
>>         :that (Forall([(z_9:obj) => (((z_9
>>            E (x_1 $ y_1)) -> (w_4 E z_9)):prop)])
>>         -> (w_4 = x_1)))])
>>      :that Forall([(w_10:obj) => ((Forall([(z_11:
>>            obj) => (((z_11 E (x_1 $ y_1)) ->
>>            (w_10 E z_11)):prop)])
>>         -> (w_10 = x_1)):prop)]))
>>      ]
>>   {move 0}


\end{verbatim}

At this point we have proved that $x$ belongs to all (both) elements of $(x,y)$, and that any $w$ which belongs to both elements of $(x,y)$ is actually equal to $x$.

The corresponding result for $y$ will be a bit harder.  We first want to prove $(\exists! z: z \in (x,y) \wedge y \in z)$.
Then we want to prove for any $w$ that if $(\exists! z: z \in (x,y) \wedge w \in z)$, then $w=y$.

Expanding things a bit, for the first part we want to prove $(\exists z:(\forall w: w \in (x,y) \wedge y \in w) \leftrightarrow w=z)$.

To be exact, this $w$ is $\{x,y\}$, so we want to prove $(\forall w: (w \in (x,y) \wedge y \in w) \leftrightarrow w = \{x,y\})$.

\begin{verbatim}Lestrade execution:

clearcurrent


declare x obj

>> x: obj {move 1}



declare y obj

>> y: obj {move 1}



open

   declare w obj

>>    w: obj {move 2}



   open

      declare yinitinpairev that (w E x$y) \
         & y E w

>>       yinitinpairev: that ((w E (x $ y))
>>         & (y E w)) {move 3}



      open

         define line1 : Simp1 yinitinpairev


>>          line1: [(---:that (w E (x $ y)))]
>>            {move 3}



         define line2 : Ui(w,Pair(x;x,x;y))


>>          line2: [(---:that ((w E ((x ; x)
>>               ; (x ; y))) == ((w = (x ; x))
>>               V (w = (x ; y)))))]
>>            {move 3}



         open

            declare casehyp1 that w=x;x

>>             casehyp1: that (w = (x ; x))
>>               {move 5}



            define line3 casehyp1 : Subs1 \
               (casehyp1, Simp2 \
               yinitinpairev)

>>             line3: [(casehyp1_1:that (w =
>>                  (x ; x))) => (---:that (y
>>                  E (x ; x)))]
>>               {move 4}



            define line4 casehyp1: Inusc1 \
               line3 casehyp1

>>             line4: [(casehyp1_1:that (w =
>>                  (x ; x))) => (---:that (y
>>                  = x))]
>>               {move 4}



            declare q obj

>>             q: obj {move 5}



            define dir1 casehyp1 : Subs(Eqsymm \
               line4 casehyp1, [q=>w=x;q] \
               ,casehyp1)

>>             dir1: [(casehyp1_1:that (w =
>>                  (x ; x))) => (---:that (w
>>                  = (x ; y)))]
>>               {move 4}



            declare casehyp2 that w=x;y

>>             casehyp2: that (w = (x ; y))
>>               {move 5}



            define dir2 casehyp2:casehyp2


>>             dir2: [(casehyp2_1:that (w =
>>                  (x ; y))) => (---:that (w
>>                  = (x ; y)))]
>>               {move 4}



            close

         define line5 : Iff1 line1 line2


>>          line5: [(---:that ((w = (x ; x))
>>               V (w = (x ; y))))]
>>            {move 3}



         define line6 : Cases line5 dir1, \
            dir2

>>          line6: [(---:that (w = (x ; y)))]
>>            {move 3}



         close

      define Line6 yinitinpairev: line6

>>       Line6: [(yinitinpairev_1:that ((w E
>>            (x $ y)) & (y E w))) => (---:that
>>            (w = (x ; y)))]
>>         {move 2}



      declare isunorderedxy that w=x;y

>>       isunorderedxy: that (w = (x ; y)) {move
>>         3}



      declare q obj

>>       q: obj {move 3}



      define Line7 isunorderedxy: Subs(Eqsymm \
         isunorderedxy,[q=>(q E \
            x$y)&y E q] \
         , Conj(Inpair2(x;x,x;y),Inpair2 x \
         y))

>>       Line7: [(isunorderedxy_1:that (w =
>>            (x ; y))) => (---:that ((w E (x
>>            $ y)) & (y E w)))]
>>         {move 2}



      close

   define line8 w: Dediff Line6, Line7

>>    line8: [(w_1:obj) => (---:that (((w_1
>>         E (x $ y)) & (y E w_1)) == (w_1 = (x
>>         ; y))))]
>>      {move 1}



   define line9: Ug line8

>>    line9: [(---:that Forall([(w_2:obj) =>
>>            ((((w_2 E (x $ y)) & (y E w_2))
>>            == (w_2 = (x ; y))):prop)]))
>>         ]
>>      {move 1}



   declare q obj

>>    q: obj {move 2}



   define line10 : Fixform(One [q=>(q E x$y)&y \
         E q] \
      ,Ei1 (x;y,line9))

>>    line10: [(---:that One([(q_4:obj) => (((q_4
>>            E (x $ y)) & (y E q_4)):prop)]))
>>         ]
>>      {move 1}



   close

define Secondprojthm1 x y:line10

>> Secondprojthm1: [(x_1:obj),(y_1:obj) => ((One([(q_2:
>>         obj) => (((q_2 E (x_1 $ y_1)) & (y_1
>>         E q_2)):prop)])
>>      Fixform ((x_1 ; y_1) Ei1 Ug([(w_6:obj)
>>         => (Dediff([(yinitinpairev_7:that ((w_6
>>            E (x_1 $ y_1)) & (y_1 E w_6))) =>
>>            (Cases((Simp1(yinitinpairev_7) Iff1
>>            (w_6 Ui ((x_1 ; x_1) Pair (x_1 ;
>>            y_1)))),[(casehyp1_9:that (w_6 =
>>               (x_1 ; x_1))) => (Subs(Eqsymm(Inusc1((casehyp1_9
>>               Subs1 Simp2(yinitinpairev_7)))),
>>               [(q_11:obj) => ((w_6 = (x_1 ;
>>                  q_11)):prop)]
>>               ,casehyp1_9):that (w_6 = (x_1
>>               ; y_1)))]
>>            ,[(casehyp2_12:that (w_6 = (x_1
>>               ; y_1))) => (casehyp2_12:that
>>               (w_6 = (x_1 ; y_1)))])
>>            :that (w_6 = (x_1 ; y_1)))]
>>         ,[(isunorderedxy_13:that (w_6 = (x_1
>>            ; y_1))) => (Subs(Eqsymm(isunorderedxy_13),
>>            [(q_14:obj) => (((q_14 E (x_1 $
>>               y_1)) & (y_1 E q_14)):prop)]
>>            ,(((x_1 ; x_1) Inpair2 (x_1 ; y_1))
>>            Conj (x_1 Inpair2 y_1))):that ((w_6
>>            E (x_1 $ y_1)) & (y_1 E w_6)))])
>>         :that (((w_6 E (x_1 $ y_1)) & (y_1
>>         E w_6)) == (w_6 = (x_1 ; y_1))))]))
>>      ):that One([(q_15:obj) => (((q_15 E (x_1
>>         $ y_1)) & (y_1 E q_15)):prop)]))
>>      ]
>>   {move 0}


\end{verbatim}

We report that our text plan given just before the block of Lestrade code worked exactly to plan the proof.  We still have the second part, to show that for any $w$ that if $(\exists! z: z \in (x,y) \wedge w \in z)$, then $w=y$.

We used environment nesting carefully to avoid declaring anything in move 0 in this block other than {\tt Secondprojthm1}.

\begin{verbatim}Lestrade execution:

clearcurrent


declare x obj

>> x: obj {move 1}



declare y obj

>> y: obj {move 1}



declare w obj

>> w: obj {move 1}



declare z obj

>> z: obj {move 1}



declare secondprojev that One[z => (z E x$y) \
      & w E z] \
   



>> secondprojev: that One([(z_1:obj) => (((z_1
>>      E (x $ y)) & (w E z_1)):prop)])
>>   {move 1}



open

   declare u obj

>>    u: obj {move 2}



   declare wev that Witnesses secondprojev \
      u

>>    wev: that (secondprojev Witnesses u) {move
>>      2}



   open

      define fact1 : Ui (u, wev)

>>       fact1: [(---:that (((u E (x $ y)) &
>>            (w E u)) == (u = u)))]
>>         {move 2}



      define fact2: Iff2 (Refleq u, fact1)


>>       fact2: [(---:that ((u E (x $ y)) &
>>            (w E u)))]
>>         {move 2}



      define fact3: Simp1 fact2

>>       fact3: [(---:that (u E (x $ y)))]
>>         {move 2}



      define fact4: Simp2 fact2

>>       fact4: [(---:that (w E u))]
>>         {move 2}



      define fact5: Ui u (x;x) Pair (x;y)


>>       fact5: [(---:that ((u E ((x ; x) ;
>>            (x ; y))) == ((u = (x ; x)) V (u
>>            = (x ; y)))))]
>>         {move 2}



      define fact6: Iff1 fact3 fact5

>>       fact6: [(---:that ((u = (x ; x)) V
>>            (u = (x ; y))))]
>>         {move 2}



      open

         declare casehyp1 that u=x;x

>>          casehyp1: that (u = (x ; x)) {move
>>            4}



         declare casehyp2 that u=x;y

>>          casehyp2: that (u = (x ; y)) {move
>>            4}



         define line1 casehyp1 : Inusc1(Subs1 \
            casehyp1 fact4)

>>          line1: [(casehyp1_1:that (u = (x
>>               ; x))) => (---:that (w = x))]
>>            {move 3}



         define fact7 : Ui (x;y,wev)

>>          fact7: [(---:that ((((x ; y) E (x
>>               $ y)) & (w E (x ; y))) == ((x
>>               ; y) = u)))]
>>            {move 3}



         define line2 casehyp1 : Subs1 (line1 \
            casehyp1, fact7)

>>          line2: [(casehyp1_1:that (u = (x
>>               ; x))) => (---:that ((((x ; y)
>>               E (x $ y)) & (x E (x ; y))) ==
>>               ((x ; y) = u)))]
>>            {move 3}



         define line3 casehyp1 : Subs1(casehyp1, \
            line2 casehyp1)

>>          line3: [(casehyp1_1:that (u = (x
>>               ; x))) => (---:that ((((x ; y)
>>               E (x $ y)) & (x E (x ; y))) ==
>>               ((x ; y) = (x ; x))))]
>>            {move 3}



         define line4 casehyp1: Iff1(Conj(Inpair2(x;x, \
            x;y), Inpair1(x,y)),line3 casehyp1)


>>          line4: [(casehyp1_1:that (u = (x
>>               ; x))) => (---:that ((x ; y)
>>               = (x ; x)))]
>>            {move 3}



         define line5 casehyp1: Inusc1(Subs1(line4 \
            casehyp1, Inpair2 x y))

>>          line5: [(casehyp1_1:that (u = (x
>>               ; x))) => (---:that (y = x))]
>>            {move 3}



         define line6 casehyp1: Subs1(Eqsymm \
            line5 casehyp1,line1 casehyp1)


>>          line6: [(casehyp1_1:that (u = (x
>>               ; x))) => (---:that (w = y))]
>>            {move 3}



         define line7 casehyp2 : (Subs1 casehyp2 \
            fact4)

>>          line7: [(casehyp2_1:that (u = (x
>>               ; y))) => (---:that (w E (x ;
>>               y)))]
>>            {move 3}



         define line8 casehyp2: Iff1(line7 \
            casehyp2, Ui w x Pair \
            y)

>>          line8: [(casehyp2_1:that (u = (x
>>               ; y))) => (---:that ((w = x)
>>               V (w = y)))]
>>            {move 3}



         open

            declare case1 that w=x

>>             case1: that (w = x) {move 5}



            declare case2 that w=y

>>             case2: that (w = y) {move 5}



            define dir2 case2: case2

>>             dir2: [(case2_1:that (w = y))
>>                  => (---:that (w = y))]
>>               {move 4}



            define fact8: Ui(x;x,wev)

>>             fact8: [(---:that ((((x ; x)
>>                  E (x $ y)) & (w E (x ; x)))
>>                  == ((x ; x) = u)))]
>>               {move 4}



            define line9 case1: Subs1(casehyp2, \
               Subs1(case1,fact8))

>>             line9: [(case1_1:that (w = x))
>>                  => (---:that ((((x ; x) E
>>                  (x $ y)) & (x E (x ; x)))
>>                  == ((x ; x) = (x ; y))))]
>>               {move 4}



            define line10 case1: Iff1(Conj(Inpair1(x;x, \
               x;y), Inusc2 x),line9 case1)


>>             line10: [(case1_1:that (w = x))
>>                  => (---:that ((x ; x) = (x
>>                  ; y)))]
>>               {move 4}



            define line11 case1: Inusc1(Subs1(Eqsymm(line10 \
               case1), Inpair2 x y))

>>             line11: [(case1_1:that (w = x))
>>                  => (---:that (y = x))]
>>               {move 4}



            define dir1 case1: Subs1(Eqsymm \
               line11 case1,case1)

>>             dir1: [(case1_1:that (w = x))
>>                  => (---:that (w = y))]
>>               {move 4}



            close

         define line13 casehyp2:Cases(line8 \
            casehyp2, dir1,dir2)

>>          line13: [(casehyp2_1:that (u = (x
>>               ; y))) => (---:that (w = y))]
>>            {move 3}



         close

      define line14: Cases(fact6,line6,line13)


>>       line14: [(---:that (w = y))]
>>         {move 2}



      close

   define line15 u wev: line14

>>    line15: [(u_1:obj),(wev_1:that (secondprojev
>>         Witnesses u_1)) => (---:that (w = y))]
>>      {move 1}



   define line16: Eg secondprojev line15


>>    line16: [(---:that (w = y))]
>>      {move 1}



   close

define Secondprojthm2 x y w secondprojev: \
   line16

>> Secondprojthm2: [(x_1:obj),(y_1:obj),(w_1:
>>      obj),(secondprojev_1:that One([(z_2:obj)
>>         => (((z_2 E (x_1 $ y_1)) & (w_1 E z_2)):
>>         prop)]))
>>      => ((secondprojev_1 Eg [(u_5:obj),(wev_5:
>>         that (secondprojev_1 Witnesses u_5))
>>         => (Cases((Simp1((Refleq(u_5) Iff2
>>         (u_5 Ui wev_5))) Iff1 (u_5 Ui ((x_1
>>         ; x_1) Pair (x_1 ; y_1)))),[(casehyp1_10:
>>            that (u_5 = (x_1 ; x_1))) => ((Eqsymm(Inusc1((((((x_1
>>            ; x_1) Inpair2 (x_1 ; y_1)) Conj
>>            (x_1 Inpair1 y_1)) Iff1 (casehyp1_10
>>            Subs1 (Inusc1((casehyp1_10 Subs1
>>            Simp2((Refleq(u_5) Iff2 (u_5 Ui
>>            wev_5))))) Subs1 ((x_1 ; y_1) Ui
>>            wev_5)))) Subs1 (x_1 Inpair2 y_1))))
>>            Subs1 Inusc1((casehyp1_10 Subs1
>>            Simp2((Refleq(u_5) Iff2 (u_5 Ui
>>            wev_5)))))):that (w_1 = y_1))]
>>         ,[(casehyp2_20:that (u_5 = (x_1 ; y_1)))
>>            => (Cases(((casehyp2_20 Subs1 Simp2((Refleq(u_5)
>>            Iff2 (u_5 Ui wev_5)))) Iff1 (w_1
>>            Ui (x_1 Pair y_1))),[(case1_24:that
>>               (w_1 = x_1)) => ((Eqsymm(Inusc1((Eqsymm(((((x_1
>>               ; x_1) Inpair1 (x_1 ; y_1)) Conj
>>               Inusc2(x_1)) Iff1 (casehyp2_20
>>               Subs1 (case1_24 Subs1 ((x_1 ;
>>               x_1) Ui wev_5))))) Subs1 (x_1
>>               Inpair2 y_1)))) Subs1 case1_24):
>>               that (w_1 = y_1))]
>>            ,[(case2_30:that (w_1 = y_1)) =>
>>               (case2_30:that (w_1 = y_1))])
>>            :that (w_1 = y_1))])
>>         :that (w_1 = y_1))])
>>      :that (w_1 = y_1))]
>>   {move 0}


\end{verbatim}

This completes the proof of the characterizations of first and second projections.  Now we prove that pairs are characterized exactly by their projections.  It is worth noting that
the size of the Lestrade proof is more accurately determined if one ignores Lestrade's responses in the dialogue and considers only the input lines.  Another alternative would be to consider the size of the Lestrade terms saved at move 0.  We are currently generating this text with a setting in the prover which suppresses display of proof terms
(and more generally of the definitions of defined terms) except at move 0.  At move 0, displayed proof terms/definitions can be quite large because all definitions at higher indexed moves are expanded out.

\begin{verbatim}Lestrade execution:

clearcurrent


declare x obj

>> x: obj {move 1}



declare y obj

>> y: obj {move 1}



declare z obj

>> z: obj {move 1}



declare w obj

>> w: obj {move 1}



declare paireqev that (x$y) = z$w

>> paireqev: that ((x $ y) = (z $ w)) {move
>>   1}



open

   define line1: Firstprojthm1 x y

>>    line1: [(---:that Forall([(z_1:obj) =>
>>            (((z_1 E (x $ y)) -> (x E z_1)):
>>            prop)]))
>>         ]
>>      {move 1}



   define line2: Subs1 paireqev line1

>>    line2: [(---:that Forall([(z_3:obj) =>
>>            (((z_3 E (z $ w)) -> (x E z_3)):
>>            prop)]))
>>         ]
>>      {move 1}



   define line3: Firstprojthm2 z w

>>    line3: [(---:that Forall([(w_1:obj) =>
>>            ((Forall([(z_2:obj) => (((z_2 E
>>               (z $ w)) -> (w_1 E z_2)):prop)])
>>            -> (w_1 = z)):prop)]))
>>         ]
>>      {move 1}



   define line4 : Ui x line3

>>    line4: [(---:that (Forall([(z_3:obj) =>
>>            (((z_3 E (z $ w)) -> (x E z_3)):
>>            prop)])
>>         -> (x = z)))]
>>      {move 1}



   define line5: Mp line2 line4

>>    line5: [(---:that (x = z))]
>>      {move 1}



   define line6 : Secondprojthm1 x y

>>    line6: [(---:that One([(q_1:obj) => (((q_1
>>            E (x $ y)) & (y E q_1)):prop)]))
>>         ]
>>      {move 1}



   define line7: Subs1 paireqev line6

>>    line7: [(---:that One([(q_3:obj) => (((q_3
>>            E (z $ w)) & (y E q_3)):prop)]))
>>         ]
>>      {move 1}



   define line8: Secondprojthm2 z w y line7


>>    line8: [(---:that (y = w))]
>>      {move 1}



   close

define Pairseq paireqev : Conj(line5,line8)


>> Pairseq: [(.x_1:obj),(.y_1:obj),(.z_1:obj),
>>      (.w_1:obj),(paireqev_1:that ((.x_1 $ .y_1)
>>      = (.z_1 $ .w_1))) => ((((paireqev_1 Subs1
>>      (.x_1 Firstprojthm1 .y_1)) Mp (.x_1 Ui
>>      (.z_1 Firstprojthm2 .w_1))) Conj Secondprojthm2(.z_1,
>>      .w_1,.y_1,(paireqev_1 Subs1 (.x_1 Secondprojthm1
>>      .y_1)))):that ((.x_1 = .z_1) & (.y_1 =
>>      .w_1)))]
>>   {move 0}


\end{verbatim}

The details of the implementation of the ordered pair take up quite a lot of space but it is an important feature of the system.

It is very interesting to observe that a definition of the pair local to the collection of relations from a given set to a given other set appears to be implicit in Zermelo's definition of correspondences;  I'll be explicit about this in constructions to appear below in this document, when I add them.

\begin{verbatim}Lestrade execution:

% these also fall under point 5.


declare s obj

>> s: obj {move 1}



declare t obj

>> t: obj {move 1}



declare u obj

>> u: obj {move 1}



open

   declare dir1 that (s;t) <<= u

>>    dir1: that ((s ; t) <<= u) {move 2}



   define linea1 dir1: Conj Mp Inpair1 s \
      t, Ui s Simp1 dir1, Mp Inpair2 s t, Ui \
      t Simp1 dir1

>>    linea1: [(dir1_1:that ((s ; t) <<= u))
>>         => (---:that ((s E u) & (t E u)))]
>>      {move 1}



   declare dir2 that (s E u) & t E u

>>    dir2: that ((s E u) & (t E u)) {move 2}



   open

      declare x1 obj

>>       x1: obj {move 3}



      open

         declare xev1 that x1 E s;t

>>          xev1: that (x1 E (s ; t)) {move
>>            4}



         define linebb2 xev1: Iff1 xev1, \
            Ui x1, Pair s t

>>          linebb2: [(xev1_1:that (x1 E (s
>>               ; t))) => (---:that ((x1 = s)
>>               V (x1 = t)))]
>>            {move 3}



         open

            declare case1 that x1 = s

>>             case1: that (x1 = s) {move 5}



            define linebb3 case1: Subs1 (Eqsymm \
               case1,Simp1 dir2)

>>             linebb3: [(case1_1:that (x1 =
>>                  s)) => (---:that (x1 E u))]
>>               {move 4}



            declare case2 that x1 = t

>>             case2: that (x1 = t) {move 5}



            define linea4 case2: Subs1 (Eqsymm \
               case2,Simp2 dir2)

>>             linea4: [(case2_1:that (x1 =
>>                  t)) => (---:that (x1 E u))]
>>               {move 4}



            close

         define linea5 xev1: Cases linebb2 \
            xev1, linebb3, linea4

>>          linea5: [(xev1_1:that (x1 E (s ;
>>               t))) => (---:that (x1 E u))]
>>            {move 3}



         close

      define linea6 x1: Ded linea5

>>       linea6: [(x1_1:obj) => (---:that ((x1_1
>>            E (s ; t)) -> (x1_1 E u)))]
>>         {move 2}



      close

   define linebb7 dir2: Fixform((s;t)<<= \
      u,Conj(Ug linea6,Conj(Inhabited Inpair1 \
      s t , Inhabited Simp1 dir2)))

>>    linebb7: [(dir2_1:that ((s E u) & (t E
>>         u))) => (---:that ((s ; t) <<= u))]
>>      {move 1}



   close

define Pairsubs s t u: Dediff linea1,linebb7


>> Pairsubs: [(s_1:obj),(t_1:obj),(u_1:obj)
>>      => (Dediff([(dir1_2:that ((s_1 ; t_1)
>>         <<= u_1)) => ((((s_1 Inpair1 t_1) Mp
>>         (s_1 Ui Simp1(dir1_2))) Conj ((s_1
>>         Inpair2 t_1) Mp (t_1 Ui Simp1(dir1_2)))):
>>         that ((s_1 E u_1) & (t_1 E u_1)))]
>>      ,[(dir2_7:that ((s_1 E u_1) & (t_1 E u_1)))
>>         => ((((s_1 ; t_1) <<= u_1) Fixform
>>         (Ug([(x1_10:obj) => (Ded([(xev1_11:
>>               that (x1_10 E (s_1 ; t_1))) =>
>>               (Cases((xev1_11 Iff1 (x1_10 Ui
>>               (s_1 Pair t_1))),[(case1_13:that
>>                  (x1_10 = s_1)) => ((Eqsymm(case1_13)
>>                  Subs1 Simp1(dir2_7)):that
>>                  (x1_10 E u_1))]
>>               ,[(case2_15:that (x1_10 = t_1))
>>                  => ((Eqsymm(case2_15) Subs1
>>                  Simp2(dir2_7)):that (x1_10
>>                  E u_1))])
>>               :that (x1_10 E u_1))])
>>            :that ((x1_10 E (s_1 ; t_1)) ->
>>            (x1_10 E u_1)))])
>>         Conj (Inhabited((s_1 Inpair1 t_1))
>>         Conj Inhabited(Simp1(dir2_7))))):that
>>         ((s_1 ; t_1) <<= u_1))])
>>      :that (((s_1 ; t_1) <<= u_1) == ((s_1
>>      E u_1) & (t_1 E u_1))))]
>>   {move 0}



open

   declare dir1 that Usc s <<= t

>>    dir1: that (Usc(s) <<= t) {move 2}



   define linea8 dir1: Simp1 (Iff1 dir1, \
      Pairsubs s s t)

>>    linea8: [(dir1_1:that (Usc(s) <<= t))
>>         => (---:that (s E t))]
>>      {move 1}



   declare dir2 that s E t

>>    dir2: that (s E t) {move 2}



   define linea9 dir2: Fixform(Usc s <<= \
      t, Iff2(Conj dir2 dir2,Pairsubs s \
      s t))

>>    linea9: [(dir2_1:that (s E t)) => (---:
>>         that (Usc(s) <<= t))]
>>      {move 1}



   close

define Uscsubs s t: Dediff linea8, linea9


>> Uscsubs: [(s_1:obj),(t_1:obj) => (Dediff([(dir1_2:
>>         that (Usc(s_1) <<= t_1)) => (Simp1((dir1_2
>>         Iff1 Pairsubs(s_1,s_1,t_1))):that (s_1
>>         E t_1))]
>>      ,[(dir2_3:that (s_1 E t_1)) => (((Usc(s_1)
>>         <<= t_1) Fixform ((dir2_3 Conj dir2_3)
>>         Iff2 Pairsubs(s_1,s_1,t_1))):that (Usc(s_1)
>>         <<= t_1))])
>>      :that ((Usc(s_1) <<= t_1) == (s_1 E t_1)))]
>>   {move 0}



define Pairinhabited s t: Ei s, [u=>u \
      E s;t] \
   , Inpair1 s t

>> Pairinhabited: [(s_1:obj),(t_1:obj) => (Ei(s_1,
>>      [(u_2:obj) => ((u_2 E (s_1 ; t_1)):prop)]
>>      ,(s_1 Inpair1 t_1)):that Exists([(u_3:
>>         obj) => ((u_3 E (s_1 ; t_1)):prop)]))
>>      ]
>>   {move 0}


\end{verbatim}

This is a batch of axioms relating unordered pairs and singletons to subset which were brought to my attention by the actual Zermelo development.

\begin{verbatim}Lestrade execution:

%% Zermelo point 6.  We declare his notion of part, though we probably
% won't use it.

clearcurrent


declare x obj

>> x: obj {move 1}



declare sethyp that Isset x

>> sethyp: that Isset(x) {move 1}



open

   declare W obj

>>    W: obj {move 2}



   open

      declare absurdhyp that W E 0

>>       absurdhyp: that (W E 0) {move 3}



      define line1 absurdhyp: Giveup(W E \
         x, Mp absurdhyp Empty W)

>>       line1: [(absurdhyp_1:that (W E 0))
>>            => (---:that (W E x))]
>>         {move 2}



      close

   define lineb2 W: Ded line1

>>    lineb2: [(W_1:obj) => (---:that ((W_1
>>         E 0) -> (W_1 E x)))]
>>      {move 1}



   close

define Zeroissubset sethyp: Fixform(0 <<= \
   x,Conj(Ug lineb2,Conj(Zeroisset,sethyp))) \
   



>> Zeroissubset: [(.x_1:obj),(sethyp_1:that
>>      Isset(.x_1)) => (((0 <<= .x_1) Fixform
>>      (Ug([(W_4:obj) => (Ded([(absurdhyp_5:that
>>            (W_4 E 0)) => (((W_4 E .x_1) Giveup
>>            (absurdhyp_5 Mp Empty(W_4))):that
>>            (W_4 E .x_1))])
>>         :that ((W_4 E 0) -> (W_4 E .x_1)))])
>>      Conj (Zeroisset Conj sethyp_1))):that
>>      (0 <<= .x_1))]
>>   {move 0}


\end{verbatim}

The empty set is a subset of every set.

\begin{verbatim}Lestrade execution:

% Zermelo's part relation (strict subset)


declare y obj

>> y: obj {move 1}



define <=/= x y: (x <<= y) & x =/= y

>> <=/=: [(x_1:obj),(y_1:obj) => (((x_1 <<=
>>      y_1) & (x_1 =/= y_1)):prop)]
>>   {move 0}


\end{verbatim}

\section{The axiom scheme of separation}

We now develop the signature axiom scheme of Zermelo set theory, which may be thought of as its solution to the ``paradoxes of naive set theory".  An arbitrary predicate of untyped objects can be converted to a set, if restricted to an already given set.

Our development follows the order in the axiomatics paper.  In Zermelo's treatment, this is the third axiom, after extensionality and the axiom of elementary sets
(empty set, singleton, and pairing).  Zermelo does assert that the object witnessing an instance of separation is a subset of the bounding set, and so a set:  we merely provide
an additional axiom that $\{x \in A : \phi(x)\}$, from which the assertion that it is a subset of $A$ can be proved.

\begin{verbatim}Lestrade execution:

clearcurrent

% also part of point 6.


declare A obj

>> A: obj {move 1}



declare x obj

>> x: obj {move 1}



declare pred [x=>prop] \
   



>> pred: [(x_1:obj) => (---:prop)]
>>   {move 1}



postulate Set A pred obj

>> Set: [(A_1:obj),(pred_1:[(x_2:obj) => (---:
>>         prop)])
>>      => (---:obj)]
>>   {move 0}



postulate Separation A pred that Forall[x=>(x \
      E Set A pred)==(x E A) & pred x] \
   



>> Separation: [(A_1:obj),(pred_1:[(x_2:obj)
>>         => (---:prop)])
>>      => (---:that Forall([(x_3:obj) => (((x_3
>>         E (A_1 Set pred_1)) == ((x_3 E A_1)
>>         & pred_1(x_3))):prop)]))
>>      ]
>>   {move 0}


\end{verbatim}

We present the axiom of separation and the constructor implementing it.  Like the deduction theorem, this is a constructor taking constructions to objects.
Its argument of type {\tt [x:obj => prop]} is a general predicate of objects, and may be thought of as a proper class.

The fact that any property of objects however formulated generates a set when restricted to a previously given set implements Zermelo's intention.  We do not thereby automatically find ourselves in a second order theory, because we have not provided ourselves with quantifiers over proper classes.  We could declare quantifiers over proper classes easily enough, but we have not done so.  It is worth noting that in Automath (at least in later versions) quantification over any type, including function types such as the type of predicates of sets we are considering here, is automatically provided:  as soon as one axiomatizes Zermelo set theory along these lines in Automath, one has thereby axiomatized second order Zermelo set theory, which is a bit stronger.  Weakness in a logical framework can be an advantage.

\begin{verbatim}Lestrade execution:


postulate Separation2 A pred that Isset (Set \
   A pred)

>> Separation2: [(A_1:obj),(pred_1:[(x_2:obj)
>>         => (---:prop)])
>>      => (---:that Isset((A_1 Set pred_1)))]
>>   {move 0}


\end{verbatim}

We provide the additional axiom that $\{x \in A:\phi(x)\}$ is always a set (which is only relevant to empty extensions).  Like {\tt Scthm2}. this is implicit in Zermelo's statement of his axioms.

\begin{verbatim}Lestrade execution:


declare sillyeq that x = Set A pred

>> sillyeq: that (x = (A Set pred)) {move 1}



define Separation3 sillyeq : Separation2 \
   A pred

>> Separation3: [(.x_1:obj),(.A_1:obj),(.pred_1:
>>      [(x_2:obj) => (---:prop)]),
>>      (sillyeq_1:that (.x_1 = (.A_1 Set .pred_1)))
>>      => ((.A_1 Separation2 .pred_1):that Isset((.A_1
>>      Set .pred_1)))]
>>   {move 0}


\end{verbatim}

This is a tricky ``theorem" which allows the deduction that $x$ is a set from the proof of $x=x$ if $x$ happens to be ultimately defined using the separation constructor, without the user needing to specify the predicate defining the set.  This is a diabolical perhaps unintended use of the implicit argument mechanism.

\begin{verbatim}Lestrade execution:


define Separation4 sillyeq : Separation A \
   pred

>> Separation4: [(.x_1:obj),(.A_1:obj),(.pred_1:
>>      [(x_2:obj) => (---:prop)]),
>>      (sillyeq_1:that (.x_1 = (.A_1 Set .pred_1)))
>>      => ((.A_1 Separation .pred_1):that Forall([(x_3:
>>         obj) => (((x_3 E (.A_1 Set .pred_1))
>>         == ((x_3 E .A_1) & .pred_1(x_3))):prop)]))
>>      ]
>>   {move 0}


\end{verbatim}

This is a tricky ``theorem" which allows the instance of separation defining $x$ to be extracted from the proof of $x=x$.  This is a diabolical perhaps unintended use of the implicit argument mechanism.

\begin{verbatim}Lestrade execution:


declare X7 obj

>> X7: obj {move 1}



declare Y7 obj

>> Y7: obj {move 1}



declare Z7 obj

>> Z7: obj {move 1}



declare xinyev that X7 E Y7

>> xinyev: that (X7 E Y7) {move 1}



declare pred7 [Z7 => prop] \
   



>> pred7: [(Z7_1:obj) => (---:prop)]
>>   {move 1}



declare univev that Forall[Z7 => (Z7 E Y7) \
      -> pred7 Z7] \
   



>> univev: that Forall([(Z7_1:obj) => (((Z7_1
>>      E Y7) -> pred7(Z7_1)):prop)])
>>   {move 1}



define Univcheat xinyev univev: Mp xinyev, \
   Ui X7 univev

>> Univcheat: [(.X7_1:obj),(.Y7_1:obj),(xinyev_1:
>>      that (.X7_1 E .Y7_1)),(.pred7_1:[(Z7_2:
>>         obj) => (---:prop)]),
>>      (univev_1:that Forall([(Z7_3:obj) => (((Z7_3
>>         E .Y7_1) -> .pred7_1(Z7_3)):prop)]))
>>      => ((xinyev_1 Mp (.X7_1 Ui univev_1)):
>>      that .pred7_1(.X7_1))]
>>   {move 0}


\end{verbatim}

This is another implicit argument trick.  From evidence for $x \in y$ and $(\forall z: z \in y \rightarrow \phi(z))$, get evidence for $\phi(x)$.  The advantage is
that the second parameter may be a complex defined notion which is only universal when expanded:  the implicit argument mechanism handles the expansion without the user's attention being needed.

\begin{verbatim}Lestrade execution:


declare inev7 that X7 E Set Y7, pred7

>> inev7: that (X7 E (Y7 Set pred7)) {move 1}



define Separation5 inev7: Iff1 inev7, Ui \
   X7, Separation4 Refleq Set Y7, pred7

>> Separation5: [(.X7_1:obj),(.Y7_1:obj),(.pred7_1:
>>      [(Z7_2:obj) => (---:prop)]),
>>      (inev7_1:that (.X7_1 E (.Y7_1 Set .pred7_1)))
>>      => ((inev7_1 Iff1 (.X7_1 Ui Separation4(Refleq((.Y7_1
>>      Set .pred7_1))))):that ((.X7_1 E .Y7_1)
>>      & .pred7_1(.X7_1)))]
>>   {move 0}


\end{verbatim}

This is a tricky method to get a proof of $a \in A \wedge \phi(a)$ from a proof of $a \in \{x \mid \phi(x)\}$.  The numbers attached to the various flavors of separation are arbitrary, basically in order of discovery of the need for them.

\begin{verbatim}Lestrade execution:


declare y obj

>> y: obj {move 1}



declare z obj

>> z: obj {move 1}



declare Aisset that Isset A

>> Aisset: that Isset(A) {move 1}



open

   declare X obj

>>    X: obj {move 2}



   open

      declare Xinev that X E (Set A pred)


>>       Xinev: that (X E (A Set pred)) {move
>>         3}



      define line1 Xinev: Simp1 Iff1 Xinev, \
         Ui X, Separation A pred

>>       line1: [(Xinev_1:that (X E (A Set pred)))
>>            => (---:that (X E A))]
>>         {move 2}



      close

   define line2 X: Ded line1

>>    line2: [(X_1:obj) => (---:that ((X_1 E
>>         (A Set pred)) -> (X_1 E A)))]
>>      {move 1}



   close

define Sepsub A pred, Aisset: Fixform((Set \
   A pred)<<=A,Conj(Ug line2,Conj(Separation2 \
   A pred,Aisset)))

>> Sepsub: [(A_1:obj),(pred_1:[(x_2:obj) =>
>>         (---:prop)]),
>>      (Aisset_1:that Isset(A_1)) => ((((A_1
>>      Set pred_1) <<= A_1) Fixform (Ug([(X_5:
>>         obj) => (Ded([(Xinev_6:that (X_5 E
>>            (A_1 Set pred_1))) => (Simp1((Xinev_6
>>            Iff1 (X_5 Ui (A_1 Separation pred_1)))):
>>            that (X_5 E A_1))])
>>         :that ((X_5 E (A_1 Set pred_1)) ->
>>         (X_5 E A_1)))])
>>      Conj ((A_1 Separation2 pred_1) Conj Aisset_1))):
>>      that ((A_1 Set pred_1) <<= A_1))]
>>   {move 0}


\end{verbatim}

This uses the implicit argument mechanism to extract a proof that $\{x \in A:\phi(x)\}$ is a subset of $A$ (if $A$ is a set) from the proof that $\{x \in A:\phi(x)\}$ is equal to itself.
The magic is that this works if the form used for $\{x : \phi(x)\}$ is a definition from which we do not want to extract the predicate.

\begin{verbatim}Lestrade execution:


declare eqev that (Set A pred)=Set A pred


>> eqev: that ((A Set pred) = (A Set pred))
>>   {move 1}



define Sepsub2 Aisset eqev:Sepsub A pred, \
   Aisset

>> Sepsub2: [(.A_1:obj),(Aisset_1:that Isset(.A_1)),
>>      (.pred_1:[(x_2:obj) => (---:prop)]),
>>      (eqev_1:that ((.A_1 Set .pred_1) = (.A_1
>>      Set .pred_1))) => (Sepsub(.A_1,.pred_1,
>>      Aisset_1):that ((.A_1 Set .pred_1) <<=
>>      .A_1))]
>>   {move 0}


\end{verbatim}

This uses the implicit argument mechanism to extract a proof that $\{x \in A:\phi(x)\}$ is a subset of $A$ (if $A$ is a set) from the proof that $\{x \in A:\phi(x)\}$ is equal to itself.
The magic is that this works if the form used for $\{x : \phi(x)\}$ is a definition from which we do not want to extract the predicate.

\begin{verbatim}Lestrade execution:

clearcurrent


declare M obj

>> M: obj {move 1}



declare M1 obj

>> M1: obj {move 1}



declare x obj

>> x: obj {move 1}


% Zermelo point 7:  relative complements.


define Complement M M1 : Set M [x => ~(x \
      E M1)] \
   



>> Complement: [(M_1:obj),(M1_1:obj) => ((M_1
>>      Set [(x_2:obj) => (~((x_2 E M1_1)):prop)])
>>      :obj)]
>>   {move 0}



define Compax M M1 : Fixform(Forall[x=>(x \
      E Complement M M1)==(x E M)& ~(x E M1)] \
   ,Separation M [x => ~(x E M1)]) \
   



>> Compax: [(M_1:obj),(M1_1:obj) => ((Forall([(x_2:
>>         obj) => (((x_2 E (M_1 Complement M1_1))
>>         == ((x_2 E M_1) & ~((x_2 E M1_1)))):
>>         prop)])
>>      Fixform (M_1 Separation [(x_3:obj) =>
>>         (~((x_3 E M1_1)):prop)]))
>>      :that Forall([(x_4:obj) => (((x_4 E (M_1
>>         Complement M1_1)) == ((x_4 E M_1) &
>>         ~((x_4 E M1_1)))):prop)]))
>>      ]
>>   {move 0}


% unfinished point 7 business to be done if needed

% prove A Complement (A Complement B) = B

% prove B <=/= A -> (A Complement B) <=/= A

\end{verbatim}

Above we implement the relative complement and its defining axiom.

\begin{verbatim}Lestrade execution:

clearcurrent

% Zermelo point 8, binary intersection

% prove (x ** x) = x

% prove (x ** 0) = 0

% prove (x disjoint y) -> (x **y) = 0


declare x obj

>> x: obj {move 1}



declare y obj

>> y: obj {move 1}



declare z obj

>> z: obj {move 1}



define ** x y: Set x [z => z E y] \
   



>> **: [(x_1:obj),(y_1:obj) => ((x_1 Set [(z_2:
>>         obj) => ((z_2 E y_1):prop)])
>>      :obj)]
>>   {move 0}


\end{verbatim}

\begin{verbatim}Lestrade execution:

clearcurrent

% Zermelo point 9, intersection of a nonempty set of sets


declare T obj

>> T: obj {move 1}



declare A obj

>> A: obj {move 1}



declare x obj

>> x: obj {move 1}



declare B obj

>> B: obj {move 1}



define Intersection T A : Set A [x => Forall[B \
         => (B E T) -> x E B] \
      ] \
   



>> Intersection: [(T_1:obj),(A_1:obj) => ((A_1
>>      Set [(x_2:obj) => (Forall([(B_3:obj) =>
>>            (((B_3 E T_1) -> (x_2 E B_3)):prop)])
>>         :prop)])
>>      :obj)]
>>   {move 0}



open

   declare inev that A E T

>>    inev: that (A E T) {move 2}



   open

      declare u obj

>>       u: obj {move 3}



      open

         declare hyp1 that u E Intersection \
            T A

>>          hyp1: that (u E (T Intersection
>>            A)) {move 4}



         declare x1 obj

>>          x1: obj {move 4}



         declare B1 obj

>>          B1: obj {move 4}



         declare hyp2 that Forall[B1 =>(B1 \
               E T)-> u E B1] \
            



>>          hyp2: that Forall([(B1_1:obj) =>
>>               (((B1_1 E T) -> (u E B1_1)):prop)])
>>            {move 4}



         define line1 hyp2: Ui A hyp2

>>          line1: [(hyp2_1:that Forall([(B1_2:
>>                  obj) => (((B1_2 E T) -> (u
>>                  E B1_2)):prop)]))
>>               => (---:that ((A E T) -> (u E
>>               A)))]
>>            {move 3}



         define line2 hyp2: Mp inev line1 \
            hyp2

>>          line2: [(hyp2_1:that Forall([(B1_2:
>>                  obj) => (((B1_2 E T) -> (u
>>                  E B1_2)):prop)]))
>>               => (---:that (u E A))]
>>            {move 3}



         define line3 hyp2: Conj (line2 hyp2, \
            hyp2)

>>          line3: [(hyp2_1:that Forall([(B1_2:
>>                  obj) => (((B1_2 E T) -> (u
>>                  E B1_2)):prop)]))
>>               => (---:that ((u E A) & Forall([(B1_4:
>>                  obj) => (((B1_4 E T) -> (u
>>                  E B1_4)):prop)]))
>>               )]
>>            {move 3}



         define line4 hyp2: Fixform(u E Intersection \
            T A,Iff2(line3 hyp2,Ui(u,Separation \
            A [x1 => Forall[B1 => (B1 E T) \
                  -> x1 E B1] \
               ])) \
            )

>>          line4: [(hyp2_1:that Forall([(B1_2:
>>                  obj) => (((B1_2 E T) -> (u
>>                  E B1_2)):prop)]))
>>               => (---:that (u E (T Intersection
>>               A)))]
>>            {move 3}



         define line5 hyp1 : Simp2(Iff1(hyp1, \
            Ui(u, Separation A [x1 => Forall[B1 \
                  => (B1 E T) -> x1 E B1] \
               ])) \
            )

>>          line5: [(hyp1_1:that (u E (T Intersection
>>               A))) => (---:that Forall([(B1_10:
>>                  obj) => (((B1_10 E T) -> (u
>>                  E B1_10)):prop)]))
>>               ]
>>            {move 3}



         close

      define bothways u : Dediff line5,line4


>>       bothways: [(u_1:obj) => (---:that ((u_1
>>            E (T Intersection A)) == Forall([(B1_26:
>>               obj) => (((B1_26 E T) -> (u_1
>>               E B1_26)):prop)]))
>>            )]
>>         {move 2}



      close

   define Intax1 inev: Ug bothways

>>    Intax1: [(inev_1:that (A E T)) => (---:
>>         that Forall([(u_30:obj) => (((u_30
>>            E (T Intersection A)) == Forall([(B1_31:
>>               obj) => (((B1_31 E T) -> (u_30
>>               E B1_31)):prop)]))
>>            :prop)]))
>>         ]
>>      {move 1}



   close

define Intax T A: Ded Intax1

>> Intax: [(T_1:obj),(A_1:obj) => (Ded([(inev_4:
>>         that (A_1 E T_1)) => (Ug([(u_7:obj)
>>            => (Dediff([(hyp1_9:that (u_7 E
>>               (T_1 Intersection A_1))) => (Simp2((hyp1_9
>>               Iff1 (u_7 Ui (A_1 Separation
>>               [(x1_16:obj) => (Forall([(B1_17:
>>                     obj) => (((B1_17 E T_1)
>>                     -> (x1_16 E B1_17)):prop)])
>>                  :prop)]))
>>               )):that Forall([(B1_18:obj) =>
>>                  (((B1_18 E T_1) -> (u_7 E
>>                  B1_18)):prop)]))
>>               ]
>>            ,[(hyp2_19:that Forall([(B1_20:obj)
>>                  => (((B1_20 E T_1) -> (u_7
>>                  E B1_20)):prop)]))
>>               => (((u_7 E (T_1 Intersection
>>               A_1)) Fixform (((inev_4 Mp (A_1
>>               Ui hyp2_19)) Conj hyp2_19) Iff2
>>               (u_7 Ui (A_1 Separation [(x1_30:
>>                  obj) => (Forall([(B1_31:obj)
>>                     => (((B1_31 E T_1) -> (x1_30
>>                     E B1_31)):prop)])
>>                  :prop)]))
>>               )):that (u_7 E (T_1 Intersection
>>               A_1)))])
>>            :that ((u_7 E (T_1 Intersection
>>            A_1)) == Forall([(B1_32:obj) =>
>>               (((B1_32 E T_1) -> (u_7 E B1_32)):
>>               prop)]))
>>            )])
>>         :that Forall([(u_33:obj) => (((u_33
>>            E (T_1 Intersection A_1)) == Forall([(B1_34:
>>               obj) => (((B1_34 E T_1) -> (u_33
>>               E B1_34)):prop)]))
>>            :prop)]))
>>         ])
>>      :that ((A_1 E T_1) -> Forall([(u_35:obj)
>>         => (((u_35 E (T_1 Intersection A_1))
>>         == Forall([(B1_36:obj) => (((B1_36
>>            E T_1) -> (u_35 E B1_36)):prop)]))
>>         :prop)]))
>>      )]
>>   {move 0}


\end{verbatim}

Above we develop the set intersection operation and prove the natural symmetric form of its associated comprehension axiom (without the asymmetric special role of $A$).

%%  Zermelo point 10:  
%% Russell's argument, 
% we can uniformly select an object in the complement of any set.

The following development makes use of the reasoning in Russell's paradox to show that for every set there is some object not belonging to it.

\begin{verbatim}Lestrade execution:

clearcurrent


declare x1 obj

>> x1: obj {move 1}



declare y obj

>> y: obj {move 1}



define Russell x1 : Set x1 [y=> ~(y E y)] \
   



>> Russell: [(x1_1:obj) => ((x1_1 Set [(y_2:
>>         obj) => (~((y_2 E y_2)):prop)])
>>      :obj)]
>>   {move 0}



define Russellax x1 : Fixform(Forall[y=>(y \
      E Russell x1) == (y E x1) & ~(y E y)] \
   ,Separation x1 [y=> ~(y E y)]) \
   



>> Russellax: [(x1_1:obj) => ((Forall([(y_2:
>>         obj) => (((y_2 E Russell(x1_1)) ==
>>         ((y_2 E x1_1) & ~((y_2 E y_2)))):prop)])
>>      Fixform (x1_1 Separation [(y_3:obj) =>
>>         (~((y_3 E y_3)):prop)]))
>>      :that Forall([(y_4:obj) => (((y_4 E Russell(x1_1))
>>         == ((y_4 E x1_1) & ~((y_4 E y_4)))):
>>         prop)]))
>>      ]
>>   {move 0}



open

   declare x obj

>>    x: obj {move 2}



   open

      declare rhyp1 that (Russell x) E x


>>       rhyp1: that (Russell(x) E x) {move
>>         3}



      open

         declare rhyp2 that (Russell x) E \
            Russell x

>>          rhyp2: that (Russell(x) E Russell(x))
>>            {move 4}



         open

            declare y1 obj

>>             y1: obj {move 5}



            define line1 : Ui (Russell x, \
               Russellax x)

>>             line1: [(---:that ((Russell(x)
>>                  E Russell(x)) == ((Russell(x)
>>                  E x) & ~((Russell(x) E Russell(x))))))]
>>               {move 4}



            define linea1: Ui(Russell x,Separation \
               x [y1 => ~(y1 E y1)]) \
               



>>             linea1: [(---:that ((Russell(x)
>>                  E (x Set [(y1_4:obj) => (~((y1_4
>>                     E y1_4)):prop)]))
>>                  == ((Russell(x) E x) & ~((Russell(x)
>>                  E Russell(x))))))]
>>               {move 4}



            define line2: Iff1 rhyp2 linea1


>>             line2: [(---:that ((Russell(x)
>>                  E x) & ~((Russell(x) E Russell(x)))))]
>>               {move 4}



            define line3: Simp2 line2

>>             line3: [(---:that ~((Russell(x)
>>                  E Russell(x))))]
>>               {move 4}



            define line4: Mp rhyp2 line3


>>             line4: [(---:that ??)]
>>               {move 4}



            close

         define line5 rhyp2: line4

>>          line5: [(rhyp2_1:that (Russell(x)
>>               E Russell(x))) => (---:that ??)]
>>            {move 3}



         define line6: Negintro line5

>>          line6: [(---:that ~((Russell(x)
>>               E Russell(x))))]
>>            {move 3}



         define line7: Ui (Russell x,Russellax \
            x)

>>          line7: [(---:that ((Russell(x) E
>>               Russell(x)) == ((Russell(x) E
>>               x) & ~((Russell(x) E Russell(x))))))]
>>            {move 3}



         declare z obj

>>          z: obj {move 4}



         define linea7: Ui(Russell x,Separation \
            x [z=> ~(z E z)]) \
            



>>          linea7: [(---:that ((Russell(x)
>>               E (x Set [(z_4:obj) => (~((z_4
>>                  E z_4)):prop)]))
>>               == ((Russell(x) E x) & ~((Russell(x)
>>               E Russell(x))))))]
>>            {move 3}



         define line8: Iff2(Conj(rhyp1,line6), \
            linea7)

>>          line8: [(---:that (Russell(x) E
>>               (x Set [(z_2:obj) => (~((z_2
>>                  E z_2)):prop)]))
>>               )]
>>            {move 3}



         define line9: Mp line8 line6

>>          line9: [(---:that ??)]
>>            {move 3}



         close

      define notin rhyp1:line9

>>       notin: [(rhyp1_1:that (Russell(x) E
>>            x)) => (---:that ??)]
>>         {move 2}



      define Notin1:Negintro notin

>>       Notin1: [(---:that ~((Russell(x) E
>>            x)))]
>>         {move 2}



      define Enotin1: Ei1 (Russell x,Notin1)


>>       Enotin1: [(---:that Exists([(x_2:obj)
>>               => (~((x_2 E x)):prop)]))
>>            ]
>>         {move 2}



      close

   define Notin2 x: Notin1

>>    Notin2: [(x_1:obj) => (---:that ~((Russell(x_1)
>>         E x_1)))]
>>      {move 1}



   define Enotin x : Enotin1

>>    Enotin: [(x_1:obj) => (---:that Exists([(x_17:
>>            obj) => (~((x_17 E x_1)):prop)]))
>>         ]
>>      {move 1}



   close

define Notin x1: Notin2 x1

>> Notin: [(x1_1:obj) => (Negintro([(rhyp1_2:
>>         that (Russell(x1_1) E x1_1)) => ((((rhyp1_2
>>         Conj Negintro([(rhyp2_4:that (Russell(x1_1)
>>            E Russell(x1_1))) => ((rhyp2_4 Mp
>>            Simp2((rhyp2_4 Iff1 (Russell(x1_1)
>>            Ui (x1_1 Separation [(y1_7:obj)
>>               => (~((y1_7 E y1_7)):prop)]))
>>            ))):that ??)]))
>>         Iff2 (Russell(x1_1) Ui (x1_1 Separation
>>         [(z_11:obj) => (~((z_11 E z_11)):prop)]))
>>         ) Mp Negintro([(rhyp2_12:that (Russell(x1_1)
>>            E Russell(x1_1))) => ((rhyp2_12
>>            Mp Simp2((rhyp2_12 Iff1 (Russell(x1_1)
>>            Ui (x1_1 Separation [(y1_15:obj)
>>               => (~((y1_15 E y1_15)):prop)]))
>>            ))):that ??)]))
>>         :that ??)])
>>      :that ~((Russell(x1_1) E x1_1)))]
>>   {move 0}



define Uenotin: Ug Enotin

>> Uenotin: [(Ug([(x_1:obj) => ((Russell(x_1)
>>         Ei1 Negintro([(rhyp1_4:that (Russell(x_1)
>>            E x_1)) => ((((rhyp1_4 Conj Negintro([(rhyp2_6:
>>               that (Russell(x_1) E Russell(x_1)))
>>               => ((rhyp2_6 Mp Simp2((rhyp2_6
>>               Iff1 (Russell(x_1) Ui (x_1 Separation
>>               [(y1_9:obj) => (~((y1_9 E y1_9)):
>>                  prop)]))
>>               ))):that ??)]))
>>            Iff2 (Russell(x_1) Ui (x_1 Separation
>>            [(z_13:obj) => (~((z_13 E z_13)):
>>               prop)]))
>>            ) Mp Negintro([(rhyp2_14:that (Russell(x_1)
>>               E Russell(x_1))) => ((rhyp2_14
>>               Mp Simp2((rhyp2_14 Iff1 (Russell(x_1)
>>               Ui (x_1 Separation [(y1_17:obj)
>>                  => (~((y1_17 E y1_17)):prop)]))
>>               ))):that ??)]))
>>            :that ??)]))
>>         :that Exists([(x_2:obj) => (~((x_2
>>            E x_1)):prop)]))
>>         ])
>>      :that Forall([(x_18:obj) => (Exists([(x_19:
>>            obj) => (~((x_19 E x_18)):prop)])
>>         :prop)]))
>>      ]
>>   {move 0}


\end{verbatim}

By a diagonalization similar to that in the Russell argument, we are able to uniformly select an element from the complement of each set.

The use of the definitions {\tt linea1} and {\tt linea7} (which eliminate the need to define {\t Russellax}) are a test of the matching capabilities of Lestrade.  But the formulation of something like {\tt Russellax} for a defined set construction is probably a good idea.

I believe I may use the constructions here to implement some of Zermelo's constructions where he speaks generally of choosing something not in a set.

\section{The axioms of power set and union}

In this section, we introduce the axioms of power set and union, which allow construction of more specific sets.

\begin{verbatim}Lestrade execution:

clearcurrent

%% the axioms of power set and union seem for all the world to fall under point 10, which
%% suggests to me that axioms are really intended to be separate points outside the 
% numbering system.


declare x obj

>> x: obj {move 1}



declare y obj

>> y: obj {move 1}



declare z obj

>> z: obj {move 1}



postulate Sc x obj

>> Sc: [(x_1:obj) => (---:obj)]
>>   {move 0}



postulate Scthm x : that Forall [z=>(z E \
      Sc x) == z <<= x] \
   



>> Scthm: [(x_1:obj) => (---:that Forall([(z_2:
>>         obj) => (((z_2 E Sc(x_1)) == (z_2 <<=
>>         x_1)):prop)]))
>>      ]
>>   {move 0}


\end{verbatim}

Here is the declaration of the power set operation (for which we use a variant of Rosser's notaiton ${\tt SC}(x)$) and its main axiom.

\begin{verbatim}Lestrade execution:


open

   declare X obj

>>    X: obj {move 2}



   open

      declare Xisset that Isset X

>>       Xisset: that Isset(X) {move 3}



      define line1 : Ui X Subsetrefl

>>       line1: [(---:that (Isset(X) -> (X <<=
>>            X)))]
>>         {move 2}



      define line2 Xisset: Xisset Mp line1


>>       line2: [(Xisset_1:that Isset(X)) =>
>>            (---:that (X <<= X))]
>>         {move 2}



      define line3 : Scthm X

>>       line3: [(---:that Forall([(z_1:obj)
>>               => (((z_1 E Sc(X)) == (z_1 <<=
>>               X)):prop)]))
>>            ]
>>         {move 2}



      define line4 : Ui X line3

>>       line4: [(---:that ((X E Sc(X)) == (X
>>            <<= X)))]
>>         {move 2}



      define linea5 Xisset: line2 Xisset \
         Iff2 line4

>>       linea5: [(Xisset_1:that Isset(X)) =>
>>            (---:that (X E Sc(X)))]
>>         {move 2}



      declare v obj

>>       v: obj {move 3}



      define line6 Xisset: Fixform(Isset \
         Sc X,Add2((Sc X)=0,Ei(X, \
         [v=>v E (Sc X)] \
         ,linea5 Xisset)))

>>       line6: [(Xisset_1:that Isset(X)) =>
>>            (---:that Isset(Sc(X)))]
>>         {move 2}



      close

   define line7 X: Ded line6

>>    line7: [(X_1:obj) => (---:that (Isset(X_1)
>>         -> Isset(Sc(X_1))))]
>>      {move 1}



   define linea7 X: Ded linea5

>>    linea7: [(X_1:obj) => (---:that (Isset(X_1)
>>         -> (X_1 E Sc(X_1))))]
>>      {move 1}



   close

define Scofsetisset: Ug line7

>> Scofsetisset: [(Ug([(X_1:obj) => (Ded([(Xisset_2:
>>            that Isset(X_1)) => ((Isset(Sc(X_1))
>>            Fixform ((Sc(X_1) = 0) Add2 Ei(X_1,
>>            [(v_4:obj) => ((v_4 E Sc(X_1)):prop)]
>>            ,((Xisset_2 Mp (X_1 Ui Subsetrefl))
>>            Iff2 (X_1 Ui Scthm(X_1)))))):that
>>            Isset(Sc(X_1)))])
>>         :that (Isset(X_1) -> Isset(Sc(X_1))))])
>>      :that Forall([(X_7:obj) => ((Isset(X_7)
>>         -> Isset(Sc(X_7))):prop)]))
>>      ]
>>   {move 0}


\end{verbatim}

The power set of a set is a set.

\begin{verbatim}Lestrade execution:


define Inownpowerset: Ug linea7

>> Inownpowerset: [(Ug([(X_1:obj) => (Ded([(Xisset_2:
>>            that Isset(X_1)) => (((Xisset_2
>>            Mp (X_1 Ui Subsetrefl)) Iff2 (X_1
>>            Ui Scthm(X_1))):that (X_1 E Sc(X_1)))])
>>         :that (Isset(X_1) -> (X_1 E Sc(X_1))))])
>>      :that Forall([(X_5:obj) => ((Isset(X_5)
>>         -> (X_5 E Sc(X_5))):prop)]))
>>      ]
>>   {move 0}


\end{verbatim}

Each set belongs to its own power set.

\begin{verbatim}Lestrade execution:


postulate Sc2 x: that Isset Sc x

>> Sc2: [(x_1:obj) => (---:that Isset(Sc(x_1)))]
>>   {move 0}


\end{verbatim}

This is an additional axiom implicit in Zermelo's treatment but natural in any case:  the power set of an atom is empty by the axioms given, but we further
specify that it is the empty set.  The axiom is stated in the convenient general form that all power sets are sets (which is what Zermelo actually says), but the case of atoms (and the empty set itself)
is the only case in which it is actually needed.  Careful reading of Zermelo's axiom may reveal that he says that power sets are actually sets, which would fully justify this.



\begin{verbatim}Lestrade execution:


declare w obj

>> w: obj {move 1}



postulate Union x obj

>> Union: [(x_1:obj) => (---:obj)]
>>   {move 0}



postulate Uthm x: that Forall [z=> (z E Union \
      x) == Exists [w=>(z E w) & w E x] \
      ] \
   



>> Uthm: [(x_1:obj) => (---:that Forall([(z_2:
>>         obj) => (((z_2 E Union(x_1)) == Exists([(w_3:
>>            obj) => (((z_2 E w_3) & (w_3 E x_1)):
>>            prop)]))
>>         :prop)]))
>>      ]
>>   {move 0}



postulate Uthm2 x: that Isset Union x

>> Uthm2: [(x_1:obj) => (---:that Isset(Union(x_1)))]
>>   {move 0}



open

   declare unioninhyp that z E Union y

>>    unioninhyp: that (z E Union(y)) {move
>>      2}



   declare unionsubshyp that y <<= x

>>    unionsubshyp: that (y <<= x) {move 2}



   define line1 unioninhyp : Iff1 unioninhyp, \
      Ui z Uthm y

>>    line1: [(unioninhyp_1:that (z E Union(y)))
>>         => (---:that Exists([(w_5:obj) => (((z
>>            E w_5) & (w_5 E y)):prop)]))
>>         ]
>>      {move 1}



   open

      declare w1 obj

>>       w1: obj {move 3}



      declare wev that (z E w1) & w1 E y


>>       wev: that ((z E w1) & (w1 E y)) {move
>>         3}



      define line2 wev: Mpsubs Simp2 wev \
         unionsubshyp

>>       line2: [(.w1_1:obj),(wev_1:that ((z
>>            E .w1_1) & (.w1_1 E y))) => (---:
>>            that (.w1_1 E x))]
>>         {move 2}



      define line3 wev: Conj Simp1 wev line2 \
         wev

>>       line3: [(.w1_1:obj),(wev_1:that ((z
>>            E .w1_1) & (.w1_1 E y))) => (---:
>>            that ((z E .w1_1) & (.w1_1 E x)))]
>>         {move 2}



      define line4 wev: Ei1 w1 line3 wev


>>       line4: [(.w1_1:obj),(wev_1:that ((z
>>            E .w1_1) & (.w1_1 E y))) => (---:
>>            that Exists([(x_3:obj) => (((z E
>>               x_3) & (x_3 E x)):prop)]))
>>            ]
>>         {move 2}



      close

   define line5 unioninhyp unionsubshyp: \
      Eg(line1 unioninhyp, line4)

>>    line5: [(unioninhyp_1:that (z E Union(y))),
>>         (unionsubshyp_1:that (y <<= x)) =>
>>         (---:that Exists([(x_7:obj) => (((z
>>            E x_7) & (x_7 E x)):prop)]))
>>         ]
>>      {move 1}



   define line6 unioninhyp unionsubshyp: \
      Iff2(line5 unioninhyp unionsubshyp,Ui \
      z Uthm x)

>>    line6: [(unioninhyp_1:that (z E Union(y))),
>>         (unionsubshyp_1:that (y <<= x)) =>
>>         (---:that (z E Union(x)))]
>>      {move 1}



   close

declare uihyp that z E Union y

>> uihyp: that (z E Union(y)) {move 1}



declare ushyp that y <<= x

>> ushyp: that (y <<= x) {move 1}



define Unionmonotone uihyp ushyp: line6 uihyp \
   ushyp

>> Unionmonotone: [(.z_1:obj),(.y_1:obj),(uihyp_1:
>>      that (.z_1 E Union(.y_1))),(.x_1:obj),
>>      (ushyp_1:that (.y_1 <<= .x_1)) => ((((uihyp_1
>>      Iff1 (.z_1 Ui Uthm(.y_1))) Eg [(.w1_8:
>>         obj),(wev_8:that ((.z_1 E .w1_8) &
>>         (.w1_8 E .y_1))) => ((.w1_8 Ei1 (Simp1(wev_8)
>>         Conj (Simp2(wev_8) Mpsubs ushyp_1))):
>>         that Exists([(x_10:obj) => (((.z_1
>>            E x_10) & (x_10 E .x_1)):prop)]))
>>         ])
>>      Iff2 (.z_1 Ui Uthm(.x_1))):that (.z_1
>>      E Union(.x_1)))]
>>   {move 0}


% Zermelo point 11, binary union


define ++ x y: Union (x ; y)

>> ++: [(x_1:obj),(y_1:obj) => (Union((x_1 ;
>>      y_1)):obj)]
>>   {move 0}



goal that (z E x ++ y) == (z E x) V z E y


>> Goal: that ((z E (x ++ y)) == ((z E x) V
>>   (z E y)))

open

   declare dir1 that z E x ++ y

>>    dir1: that (z E (x ++ y)) {move 2}



   define linec1 dir1: Iff1 dir1, Ui z, Uthm(x;y)


>>    linec1: [(dir1_1:that (z E (x ++ y)))
>>         => (---:that Exists([(w_5:obj) => (((z
>>            E w_5) & (w_5 E (x ; y))):prop)]))
>>         ]
>>      {move 1}



   open

      declare w83 obj

>>       w83: obj {move 3}



      declare wev83 that (z E w83) & w83 \
         E x;y

>>       wev83: that ((z E w83) & (w83 E (x
>>         ; y))) {move 3}



      define linec2 wev83: Iff1 Simp2 wev83, \
         Ui w83, Pair x y

>>       linec2: [(.w83_1:obj),(wev83_1:that
>>            ((z E .w83_1) & (.w83_1 E (x ; y))))
>>            => (---:that ((.w83_1 = x) V (.w83_1
>>            = y)))]
>>         {move 2}



      open

         declare case1 that w83 = x

>>          case1: that (w83 = x) {move 4}



         declare case2 that w83 = y

>>          case2: that (w83 = y) {move 4}



         define linec3 case1: Add1(z E y, \
            Subs1 case1 Simp1 wev83)

>>          linec3: [(case1_1:that (w83 = x))
>>               => (---:that ((z E x) V (z E
>>               y)))]
>>            {move 3}



         define linec4 case2: Add2(z E x, \
            Subs1 case2 Simp1 wev83)

>>          linec4: [(case2_1:that (w83 = y))
>>               => (---:that ((z E x) V (z E
>>               y)))]
>>            {move 3}



         close

      define linec5 wev83: Cases linec2 wev83, \
         linec3,linec4

>>       linec5: [(.w83_1:obj),(wev83_1:that
>>            ((z E .w83_1) & (.w83_1 E (x ; y))))
>>            => (---:that ((z E x) V (z E y)))]
>>         {move 2}



      close

   define linec6 dir1: Eg linec1 dir1, linec5


>>    linec6: [(dir1_1:that (z E (x ++ y)))
>>         => (---:that ((z E x) V (z E y)))]
>>      {move 1}



   declare dir2 that (z E x) V z E y

>>    dir2: that ((z E x) V (z E y)) {move 2}



   open

      declare case1 that z E x

>>       case1: that (z E x) {move 3}



      declare case2 that z E y

>>       case2: that (z E y) {move 3}



      define linec7: Inpair1 x y

>>       linec7: [(---:that (x E (x ; y)))]
>>         {move 2}



      define linec8: Inpair2 x y

>>       linec8: [(---:that (y E (x ; y)))]
>>         {move 2}



      declare z1 obj

>>       z1: obj {move 3}



      define linec9 case1: Ei x,[z1=>(z E \
            z1) & z1 E x;y] \
         ,Conj(case1,linec7)

>>       linec9: [(case1_1:that (z E x)) =>
>>            (---:that Exists([(z1_3:obj) =>
>>               (((z E z1_3) & (z1_3 E (x ; y))):
>>               prop)]))
>>            ]
>>         {move 2}



      define linec10 case2: Ei y,[z1=>(z \
            E z1) & z1 E x;y] \
         ,Conj(case2,linec8)

>>       linec10: [(case2_1:that (z E y)) =>
>>            (---:that Exists([(z1_3:obj) =>
>>               (((z E z1_3) & (z1_3 E (x ; y))):
>>               prop)]))
>>            ]
>>         {move 2}



      close

   define linec11 dir2: Cases dir2, linec9, \
      linec10

>>    linec11: [(dir2_1:that ((z E x) V (z E
>>         y))) => (---:that Exists([(z1_9:obj)
>>            => (((z E z1_9) & (z1_9 E (x ; y))):
>>            prop)]))
>>         ]
>>      {move 1}



   define linec12 dir2: Iff2 linec11 dir2, \
      Ui z,Uthm (x;y)

>>    linec12: [(dir2_1:that ((z E x) V (z E
>>         y))) => (---:that (z E Union((x ; y))))]
>>      {move 1}



   close

define Binaryunion x y z: Dediff linec6, \
   linec12

>> Binaryunion: [(x_1:obj),(y_1:obj),(z_1:obj)
>>      => (Dediff([(dir1_2:that (z_1 E (x_1 ++
>>         y_1))) => (((dir1_2 Iff1 (z_1 Ui Uthm((x_1
>>         ; y_1)))) Eg [(.w83_7:obj),(wev83_7:
>>            that ((z_1 E .w83_7) & (.w83_7 E
>>            (x_1 ; y_1)))) => (Cases((Simp2(wev83_7)
>>            Iff1 (.w83_7 Ui (x_1 Pair y_1))),
>>            [(case1_9:that (.w83_7 = x_1)) =>
>>               (((z_1 E y_1) Add1 (case1_9 Subs1
>>               Simp1(wev83_7))):that ((z_1 E
>>               x_1) V (z_1 E y_1)))]
>>            ,[(case2_11:that (.w83_7 = y_1))
>>               => (((z_1 E x_1) Add2 (case2_11
>>               Subs1 Simp1(wev83_7))):that ((z_1
>>               E x_1) V (z_1 E y_1)))])
>>            :that ((z_1 E x_1) V (z_1 E y_1)))])
>>         :that ((z_1 E x_1) V (z_1 E y_1)))]
>>      ,[(dir2_13:that ((z_1 E x_1) V (z_1 E
>>         y_1))) => ((Cases(dir2_13,[(case1_16:
>>            that (z_1 E x_1)) => (Ei(x_1,[(z1_17:
>>               obj) => (((z_1 E z1_17) & (z1_17
>>               E (x_1 ; y_1))):prop)]
>>            ,(case1_16 Conj (x_1 Inpair1 y_1))):
>>            that Exists([(z1_18:obj) => (((z_1
>>               E z1_18) & (z1_18 E (x_1 ; y_1))):
>>               prop)]))
>>            ]
>>         ,[(case2_19:that (z_1 E y_1)) => (Ei(y_1,
>>            [(z1_20:obj) => (((z_1 E z1_20)
>>               & (z1_20 E (x_1 ; y_1))):prop)]
>>            ,(case2_19 Conj (x_1 Inpair2 y_1))):
>>            that Exists([(z1_21:obj) => (((z_1
>>               E z1_21) & (z1_21 E (x_1 ; y_1))):
>>               prop)]))
>>            ])
>>         Iff2 (z_1 Ui Uthm((x_1 ; y_1)))):that
>>         (z_1 E Union((x_1 ; y_1))))])
>>      :that ((z_1 E (x_1 ++ y_1)) == ((z_1 E
>>      x_1) V (z_1 E y_1))))]
>>   {move 0}


%% Zermelo point 12:  
%%  laws of boolean algebra, to be done
% when needed.

\end{verbatim}


Here we declare the set union operation and its defining theorem, and define binary union.  Various utilities need to be developed, for example the theorem
{\tt Unionmonotone} needed in the proof below that a subset of a partition is a partition.

\section{The axiom of choice}

Here we state the axiom of choice in its original form:  each partition has a choice set.

\begin{verbatim}Lestrade execution:

clearcurrent

% Zermelo point 13


declare x obj

>> x: obj {move 1}



declare y obj

>> y: obj {move 1}



declare z obj

>> z: obj {move 1}



declare w obj

>> w: obj {move 1}



define Ispartition x : (Forall [y => (y E \
      x) -> Exists[z => z E y] \
      ]) \
   & Forall [y => (y E (Union x)) -> One[z \
         => (y E z) & z E x] \
      ] \
   



>> Ispartition: [(x_1:obj) => ((Forall([(y_2:
>>         obj) => (((y_2 E x_1) -> Exists([(z_3:
>>            obj) => ((z_3 E y_2):prop)]))
>>         :prop)])
>>      & Forall([(y_4:obj) => (((y_4 E Union(x_1))
>>         -> One([(z_5:obj) => (((y_4 E z_5)
>>            & (z_5 E x_1)):prop)]))
>>         :prop)]))
>>      :prop)]
>>   {move 0}


%% a subset of a partition is a partition:
% proof still in development


open

   declare partev that Ispartition x

>>    partev: that Ispartition(x) {move 2}



   declare subpartev that y <<= x

>>    subpartev: that (y <<= x) {move 2}



   goal that Ispartition y

>>    Goal: that Ispartition(y)

   declare x17 obj

>>    x17: obj {move 2}



   declare z17 obj

>>    z17: obj {move 2}



   goal that Forall [z17 => (z17 E y) -> \
         Exists [x17 => x17 E z17] \
         ] \
      



>>    Goal: that Forall([(z17_159:obj) => (((z17_159
>>         E y) -> Exists([(x17_160:obj) => ((x17_160
>>            E z17_159):prop)]))
>>         :prop)])
>>      

   open

      declare z1 obj

>>       z1: obj {move 3}



      open

         declare inev that z1 E y

>>          inev: that (z1 E y) {move 4}



         define line1 inev: Mpsubs inev, \
            subpartev

>>          line1: [(inev_1:that (z1 E y)) =>
>>               (---:that (z1 E x))]
>>            {move 3}



         define line2 inev: Mp line1 inev \
            , Ui z1 Simp1 partev

>>          line2: [(inev_1:that (z1 E y)) =>
>>               (---:that Exists([(z_9:obj) =>
>>                  ((z_9 E z1):prop)]))
>>               ]
>>            {move 3}



         close

      define line3 z1: Ded line2

>>       line3: [(z1_1:obj) => (---:that ((z1_1
>>            E y) -> Exists([(z_12:obj) => ((z_12
>>               E z1_1):prop)]))
>>            )]
>>         {move 2}



      close

   define line4 partev subpartev: Ug line3


>>    line4: [(partev_1:that Ispartition(x)),
>>         (subpartev_1:that (y <<= x)) => (---:
>>         that Forall([(z1_16:obj) => (((z1_16
>>            E y) -> Exists([(z_17:obj) => ((z_17
>>               E z1_16):prop)]))
>>            :prop)]))
>>         ]
>>      {move 1}



   goal that Forall [z17 => (z17 E Union \
         y) -> One[x17 => (z17 E x17) & x17 \
            E y] \
         ] \
      



>>    Goal: that Forall([(z17_295:obj) => (((z17_295
>>         E Union(y)) -> One([(x17_296:obj) =>
>>            (((z17_295 E x17_296) & (x17_296
>>            E y)):prop)]))
>>         :prop)])
>>      

   open

      declare z1 obj

>>       z1: obj {move 3}



      open

         declare thehyp that z1 E Union y


>>          thehyp: that (z1 E Union(y)) {move
>>            4}



         define line5 thehyp: Unionmonotone \
            thehyp subpartev

>>          line5: [(thehyp_1:that (z1 E Union(y)))
>>               => (---:that (z1 E Union(x)))]
>>            {move 3}



         define line6 thehyp: Mp line5 thehyp, \
            Ui z1 Simp2 partev

>>          line6: [(thehyp_1:that (z1 E Union(y)))
>>               => (---:that One([(z_9:obj) =>
>>                  (((z1 E z_9) & (z_9 E x)):
>>                  prop)]))
>>               ]
>>            {move 3}



         declare w1 obj

>>          w1: obj {move 4}



         goal that Forall[w1 => ((z1 E w1) \
               & w1 E x) == (z1 \
               E w1) & w1 E y] \
            



>>          Goal: that Forall([(w1_384:obj)
>>               => ((((z1 E w1_384) & (w1_384
>>               E x)) == ((z1 E w1_384) & (w1_384
>>               E y))):prop)])
>>            

         open

            declare w2 obj

>>             w2: obj {move 5}



            open

               declare dir1 that (z1 E w2) \
                  & w2 E x

>>                dir1: that ((z1 E w2) & (w2
>>                  E x)) {move 6}



               define line7 dir1: Simp2 dir1


>>                line7: [(dir1_1:that ((z1
>>                     E w2) & (w2 E x))) => (---:
>>                     that (w2 E x))]
>>                  {move 5}



               define line8 dir1: Iff1 thehyp, \
                  Ui z1 Uthm y

>>                line8: [(dir1_1:that ((z1
>>                     E w2) & (w2 E x))) => (---:
>>                     that Exists([(w_5:obj)
>>                        => (((z1 E w_5) & (w_5
>>                        E y)):prop)]))
>>                     ]
>>                  {move 5}



               define line9 dir1: Ui z1 Simp2 \
                  partev

>>                line9: [(dir1_1:that ((z1
>>                     E w2) & (w2 E x))) => (---:
>>                     that ((z1 E Union(x)) ->
>>                     One([(z_8:obj) => (((z1
>>                        E z_8) & (z_8 E x)):
>>                        prop)]))
>>                     )]
>>                  {move 5}



               define line10 dir1: Unionmonotone \
                  thehyp subpartev

>>                line10: [(dir1_1:that ((z1
>>                     E w2) & (w2 E x))) => (---:
>>                     that (z1 E Union(x)))]
>>                  {move 5}



               define line11 dir1: Mp line10 \
                  dir1, line9 dir1

>>                line11: [(dir1_1:that ((z1
>>                     E w2) & (w2 E x))) => (---:
>>                     that One([(z_3:obj) =>
>>                        (((z1 E z_3) & (z_3
>>                        E x)):prop)]))
>>                     ]
>>                  {move 5}



               open

                  declare w3 obj

>>                   w3: obj {move 7}



                  declare u obj

>>                   u: obj {move 7}



                  declare whyp3 that Forall[ \
                        u=>((z1 E u) & u E \
                        x) == u=w3] \
                     



>>                   whyp3: that Forall([(u_1:
>>                        obj) => ((((z1 E u_1)
>>                        & (u_1 E x)) == (u_1
>>                        = w3)):prop)])
>>                     {move 7}



                  define line12 whyp3: Iff1 \
                     dir1,Ui w2 whyp3

>>                   line12: [(.w3_1:obj),(whyp3_1:
>>                        that Forall([(u_2:obj)
>>                           => ((((z1 E u_2)
>>                           & (u_2 E x)) == (u_2
>>                           = .w3_1)):prop)]))
>>                        => (---:that (w2 = .w3_1))]
>>                     {move 6}



                  open

                     declare w4 obj

>>                      w4: obj {move 8}



                     declare whyp4 that (z1 \
                        E w4) & w4 E y

>>                      whyp4: that ((z1 E w4)
>>                        & (w4 E y)) {move 8}



                     define line13 whyp4: \
                        Mpsubs Simp2 whyp4 \
                        subpartev

>>                      line13: [(.w4_1:obj),
>>                           (whyp4_1:that ((z1
>>                           E .w4_1) & (.w4_1
>>                           E y))) => (---:that
>>                           (.w4_1 E x))]
>>                        {move 7}



                     define line14 whyp4: \
                        Iff1 (Conj Simp1 whyp4 \
                        line13 whyp4, Ui w4 \
                        whyp3)

>>                      line14: [(.w4_1:obj),
>>                           (whyp4_1:that ((z1
>>                           E .w4_1) & (.w4_1
>>                           E y))) => (---:that
>>                           (.w4_1 = w3))]
>>                        {move 7}



                     define line15 whyp4 \
                        : Subs1 line14 whyp4 \
                        Simp2 whyp4

>>                      line15: [(.w4_1:obj),
>>                           (whyp4_1:that ((z1
>>                           E .w4_1) & (.w4_1
>>                           E y))) => (---:that
>>                           (w3 E y))]
>>                        {move 7}



                     define line16 whyp4: \
                        Subs1(Eqsymm line12 \
                        whyp3,line15 whyp4)


>>                      line16: [(.w4_1:obj),
>>                           (whyp4_1:that ((z1
>>                           E .w4_1) & (.w4_1
>>                           E y))) => (---:that
>>                           (w2 E y))]
>>                        {move 7}



                     close

                  define line17 whyp3: Eg \
                     line8 dir1 line16

>>                   line17: [(.w3_1:obj),(whyp3_1:
>>                        that Forall([(u_2:obj)
>>                           => ((((z1 E u_2)
>>                           & (u_2 E x)) == (u_2
>>                           = .w3_1)):prop)]))
>>                        => (---:that (w2 E y))]
>>                     {move 6}



                  close

               define line18 dir1: Eg line11 \
                  dir1 line17

>>                line18: [(dir1_1:that ((z1
>>                     E w2) & (w2 E x))) => (---:
>>                     that (w2 E y))]
>>                  {move 5}



               define line19 dir1 : Conj \
                  Simp1 dir1, line18 dir1

>>                line19: [(dir1_1:that ((z1
>>                     E w2) & (w2 E x))) => (---:
>>                     that ((z1 E w2) & (w2 E
>>                     y)))]
>>                  {move 5}



               declare dir2 that (z1 E w2) \
                  & w2 E y

>>                dir2: that ((z1 E w2) & (w2
>>                  E y)) {move 6}



               define line20 dir2: Conj(Simp1 \
                  dir2,Mpsubs Simp2 dir2 subpartev)


>>                line20: [(dir2_1:that ((z1
>>                     E w2) & (w2 E y))) => (---:
>>                     that ((z1 E w2) & (w2 E
>>                     x)))]
>>                  {move 5}



               close

            define line21 w2: Dediff line19, \
               line20

>>             line21: [(w2_1:obj) => (---:that
>>                  (((z1 E w2_1) & (w2_1 E x))
>>                  == ((z1 E w2_1) & (w2_1 E
>>                  y))))]
>>               {move 4}



            close

         define line22 thehyp: Ug line21


>>          line22: [(thehyp_1:that (z1 E Union(y)))
>>               => (---:that Forall([(w2_26:obj)
>>                  => ((((z1 E w2_26) & (w2_26
>>                  E x)) == ((z1 E w2_26) & (w2_26
>>                  E y))):prop)]))
>>               ]
>>            {move 3}



         define line23 thehyp: Onequiv line6 \
            thehyp line22 thehyp

>>          line23: [(thehyp_1:that (z1 E Union(y)))
>>               => (---:that One([(w2_4:obj)
>>                  => (((z1 E w2_4) & (w2_4 E
>>                  y)):prop)]))
>>               ]
>>            {move 3}



         close

      define line24 z1: Ded line23

>>       line24: [(z1_1:obj) => (---:that ((z1_1
>>            E Union(y)) -> One([(w2_38:obj)
>>               => (((z1_1 E w2_38) & (w2_38
>>               E y)):prop)]))
>>            )]
>>         {move 2}



      close

   define line25 partev subpartev: Ug line24


>>    line25: [(partev_1:that Ispartition(x)),
>>         (subpartev_1:that (y <<= x)) => (---:
>>         that Forall([(z1_42:obj) => (((z1_42
>>            E Union(y)) -> One([(w2_43:obj)
>>               => (((z1_42 E w2_43) & (w2_43
>>               E y)):prop)]))
>>            :prop)]))
>>         ]
>>      {move 1}



   close

declare partev2 that Ispartition x

>> partev2: that Ispartition(x) {move 1}



declare subpartev2 that y <<= x

>> subpartev2: that (y <<= x) {move 1}



define Subpartition partev2 subpartev2: Fixform(Ispartition \
   y,Conj(line4 partev2 subpartev2,line25 partev2 \
   subpartev2))

>> Subpartition: [(.x_1:obj),(partev2_1:that
>>      Ispartition(.x_1)),(.y_1:obj),(subpartev2_1:
>>      that (.y_1 <<= .x_1)) => ((Ispartition(.y_1)
>>      Fixform (Ug([(z1_6:obj) => (Ded([(inev_8:
>>            that (z1_6 E .y_1)) => (((inev_8
>>            Mpsubs subpartev2_1) Mp (z1_6 Ui
>>            Simp1(partev2_1))):that Exists([(z_16:
>>               obj) => ((z_16 E z1_6):prop)]))
>>            ])
>>         :that ((z1_6 E .y_1) -> Exists([(z_17:
>>            obj) => ((z_17 E z1_6):prop)]))
>>         )])
>>      Conj Ug([(z1_22:obj) => (Ded([(thehyp_24:
>>            that (z1_22 E Union(.y_1))) => ((((thehyp_24
>>            Unionmonotone subpartev2_1) Mp (z1_22
>>            Ui Simp2(partev2_1))) Onequiv Ug([(w2_35:
>>               obj) => (Dediff([(dir1_36:that
>>                  ((z1_22 E w2_35) & (w2_35
>>                  E .x_1))) => ((Simp1(dir1_36)
>>                  Conj (((thehyp_24 Unionmonotone
>>                  subpartev2_1) Mp (z1_22 Ui
>>                  Simp2(partev2_1))) Eg [(.w3_46:
>>                     obj),(whyp3_46:that Forall([(u_47:
>>                        obj) => ((((z1_22 E
>>                        u_47) & (u_47 E .x_1))
>>                        == (u_47 = .w3_46)):
>>                        prop)]))
>>                     => (((thehyp_24 Iff1 (z1_22
>>                     Ui Uthm(.y_1))) Eg [(.w4_52:
>>                        obj),(whyp4_52:that
>>                        ((z1_22 E .w4_52) &
>>                        (.w4_52 E .y_1))) =>
>>                        ((Eqsymm((dir1_36 Iff1
>>                        (w2_35 Ui whyp3_46)))
>>                        Subs1 (((Simp1(whyp4_52)
>>                        Conj (Simp2(whyp4_52)
>>                        Mpsubs subpartev2_1))
>>                        Iff1 (.w4_52 Ui whyp3_46))
>>                        Subs1 Simp2(whyp4_52))):
>>                        that (w2_35 E .y_1))])
>>                     :that (w2_35 E .y_1))]))
>>                  :that ((z1_22 E w2_35) & (w2_35
>>                  E .y_1)))]
>>               ,[(dir2_57:that ((z1_22 E w2_35)
>>                  & (w2_35 E .y_1))) => ((Simp1(dir2_57)
>>                  Conj (Simp2(dir2_57) Mpsubs
>>                  subpartev2_1)):that ((z1_22
>>                  E w2_35) & (w2_35 E .x_1)))])
>>               :that (((z1_22 E w2_35) & (w2_35
>>               E .x_1)) == ((z1_22 E w2_35)
>>               & (w2_35 E .y_1))))]))
>>            :that One([(w2_58:obj) => (((z1_22
>>               E w2_58) & (w2_58 E .y_1)):prop)]))
>>            ])
>>         :that ((z1_22 E Union(.y_1)) -> One([(w2_59:
>>            obj) => (((z1_22 E w2_59) & (w2_59
>>            E .y_1)):prop)]))
>>         )]))
>>      ):that Ispartition(.y_1))]
>>   {move 0}


\end{verbatim}

We prove above that a subset of a partition is a partition.

\begin{verbatim}Lestrade execution:


postulate Ac that Forall [x => (Ispartition \
      x) -> Exists[y => (y <<= Union x) & Forall \
         [z=> (z E x) -> One [w => (w E y) \
               & w E z] \
            ] \
         ] \
      ] \
   



>> Ac: that Forall([(x_1:obj) => ((Ispartition(x_1)
>>      -> Exists([(y_2:obj) => (((y_2 <<= Union(x_1))
>>         & Forall([(z_3:obj) => (((z_3 E x_1)
>>            -> One([(w_4:obj) => (((w_4 E y_2)
>>               & (w_4 E z_3)):prop)]))
>>            :prop)]))
>>         :prop)]))
>>      :prop)])
>>   {move 0}



declare partx that Ispartition x

>> partx: that Ispartition(x) {move 1}



define Product partx: Set Sc(Union x) [y \
      => Forall [z=> (z E x) -> One \
         [w => (w E y) & w E z] \
         ] \
      ] \
   



>> Product: [(.x_1:obj),(partx_1:that Ispartition(.x_1))
>>      => ((Sc(Union(.x_1)) Set [(y_2:obj) =>
>>         (Forall([(z_3:obj) => (((z_3 E .x_1)
>>            -> One([(w_4:obj) => (((w_4 E y_2)
>>               & (w_4 E z_3)):prop)]))
>>            :prop)])
>>         :prop)])
>>      :obj)]
>>   {move 0}


\end{verbatim}

Examples of use of this axiom are needed.  I should add the development of binary product.


\section{The axiom of infinity}

The axiom of infinity is introduced in the original form used by Zermelo.  0 is implemented as $\emptyset$ and the successor operation is implemented as the singleton operation.

\begin{verbatim}Lestrade execution:

clearcurrent


declare x obj

>> x: obj {move 1}



declare pred [x=> prop] \
   



>> pred: [(x_1:obj) => (---:prop)]
>>   {move 1}



define inductive pred : (Forall [x=> pred \
      x -> pred Usc x]) \
   & pred 0

>> inductive: [(pred_1:[(x_2:obj) => (---:prop)])
>>      => ((Forall([(x_3:obj) => ((pred_1(x_3)
>>         -> pred_1(Usc(x_3))):prop)])
>>      & pred_1(0)):prop)]
>>   {move 0}



postulate N obj

>> N: obj {move 0}



postulate Nax1 that inductive [x => x E N] \
   



>> Nax1: that inductive([(x_1:obj) => ((x_1
>>      E N):prop)])
>>   {move 0}



declare predindev that inductive pred

>> predindev: that inductive(pred) {move 1}



declare isnatev that x E N

>> isnatev: that (x E N) {move 1}



postulate Nax2 predindev isnatev : that x \
   E N

>> Nax2: [(.pred_1:[(x_2:obj) => (---:prop)]),
>>      (predindev_1:that inductive(.pred_1)),
>>      (.x_1:obj),(isnatev_1:that (.x_1 E N))
>>      => (---:that (.x_1 E N))]
>>   {move 0}


\end{verbatim}

Natural numbers are defined as those objects which have all inductive properties, and it is declared that the collection of natural numbers is a set (and that belonging to this set
is an inductive property).  We cannot prove that having all inductive properties is an inductive property, because we have not equipped ourselves with second order quantification.

This is not exactly the same as Zermelo's development:  he simply asserts the existence of a set ${\mathbb N}_0$ membership in which is inductive, then defines
$\mathbb N$ as the intersection of all inductive subsets of ${\mathbb N}_0$, and shows that the latter set is uniquely determined by this procedure.  But this approach is
equivalent, and one is asserting the existence of a definite object.

Some declarations related to arithmetic and finite sets should appear here.

\section{Commencing the theory of equivalence}

This completes the development of the axioms of 1908 Zermelo set theory under Lestrade.  It remains to develop the theory of equivalence following the Zermelo paper.

\begin{verbatim}Lestrade execution:

clearcurrent

% Zermelo point 15


declare x obj

>> x: obj {move 1}



declare y obj

>> y: obj {move 1}



declare z obj

>> z: obj {move 1}



declare A obj

>> A: obj {move 1}



declare B obj

>> B: obj {move 1}



declare disjev that A disjoint B

>> disjev: that (A disjoint B) {move 1}



define product disjev: Set Sc(A ++ B) [z \
      => Exists[x => (x E A) & Exists [y => \
            (y E B) & z=x;y] \
         ] \
      ] \
   



>> product: [(.A_1:obj),(.B_1:obj),(disjev_1:
>>      that (.A_1 disjoint .B_1)) => ((Sc((.A_1
>>      ++ .B_1)) Set [(z_2:obj) => (Exists([(x_3:
>>            obj) => (((x_3 E .A_1) & Exists([(y_4:
>>               obj) => (((y_4 E .B_1) & (z_2
>>               = (x_3 ; y_4))):prop)]))
>>            :prop)])
>>         :prop)])
>>      :obj)]
>>   {move 0}


\end{verbatim}

Above I saved myself a little work by defining binary product independently of the infinitary product defined with AC.  The missing ingredient would be the proof of equivalence
of disjointness of $A,B$ with $\{A,B\}$ being a partition (which would probably be good for me).

\begin{verbatim}Lestrade execution:

% paragraph 15


declare F obj

>> F: obj {move 1}



declare s obj

>> s: obj {move 1}



declare t obj

>> t: obj {move 1}



define Equivalent disjev: Exists[F => (F \
      <<= product disjev) & Forall[s => (s \
         E A ++ B) -> One[t => (t \
            E F) & s E t] \
         ] \
      ] \
   



>> Equivalent: [(.A_1:obj),(.B_1:obj),(disjev_1:
>>      that (.A_1 disjoint .B_1)) => (Exists([(F_2:
>>         obj) => (((F_2 <<= product(disjev_1))
>>         & Forall([(s_3:obj) => (((s_3 E (.A_1
>>            ++ .B_1)) -> One([(t_4:obj) => (((t_4
>>               E F_2) & (s_3 E t_4)):prop)]))
>>            :prop)]))
>>         :prop)])
>>      :prop)]
>>   {move 0}



define Mapping disjev F: (F <<= product disjev) \
   & Forall[s => (s E A ++ B) -> One[t => (t \
         E F) & s E t] \
      ] \
   



>> Mapping: [(.A_1:obj),(.B_1:obj),(disjev_1:
>>      that (.A_1 disjoint .B_1)),(F_1:obj) =>
>>      (((F_1 <<= product(disjev_1)) & Forall([(s_2:
>>         obj) => (((s_2 E (.A_1 ++ .B_1)) ->
>>         One([(t_3:obj) => (((t_3 E F_1) & (s_2
>>            E t_3)):prop)]))
>>         :prop)]))
>>      :prop)]
>>   {move 0}



declare ismap that Mapping disjev F

>> ismap: that (disjev Mapping F) {move 1}



declare c obj

>> c: obj {move 1}



declare d obj

>> d: obj {move 1}



define corresponds ismap c d: (c;d) E F

>> corresponds: [(.A_1:obj),(.B_1:obj),(.disjev_1:
>>      that (.A_1 disjoint .B_1)),(.F_1:obj),
>>      (ismap_1:that (.disjev_1 Mapping .F_1)),
>>      (c_1:obj),(d_1:obj) => (((c_1 ; d_1) E
>>      .F_1):prop)]
>>   {move 0}



declare infield that s E A ++ B

>> infield: that (s E (A ++ B)) {move 1}



open

   define line1 : Mp infield, Ui s Simp2 \
      ismap

>>    line1: [(---:that One([(t_6:obj) => (((t_6
>>            E F) & (s E t_6)):prop)]))
>>         ]
>>      {move 1}



   define theimage: The line1

>>    theimage: [(---:obj)]
>>      {move 1}



   define theimagefact: Fixform ((theimage \
      E F) & s E theimage, Theax line1)

>>    theimagefact: [(---:that ((theimage E
>>         F) & (s E theimage)))]
>>      {move 1}



   declare u obj

>>    u: obj {move 2}



   goal that One [u=> (u E theimage) & ~(u \
         = s)] \
      



>>    Goal: that One([(u_199:obj) => (((u_199
>>         E theimage) & ~((u_199 = s))):prop)])
>>      

   define line2 : Fixform theimage E product \
      disjev, Mpsubs Simp1 theimagefact, Simp1 \
      ismap

>>    line2: [(---:that (theimage E product(disjev)))]
>>      {move 1}



   define line3: Simp2 Iff1 line2, Ui theimage \
      Separation4 Refleq product disjev

>>    line3: [(---:that Exists([(x_14:obj) =>
>>            (((x_14 E A) & Exists([(y_15:obj)
>>               => (((y_15 E B) & (theimage =
>>               (x_14 ; y_15))):prop)]))
>>            :prop)]))
>>         ]
>>      {move 1}



   open

      declare u1 obj

>>       u1: obj {move 3}



      declare witnessev1 that Witnesses line3 \
         u1

>>       witnessev1: that (line3 Witnesses u1)
>>         {move 3}



      define line4 witnessev1: Simp1 witnessev1


>>       line4: [(.u1_1:obj),(witnessev1_1:that
>>            (line3 Witnesses .u1_1)) => (---:
>>            that (.u1_1 E A))]
>>         {move 2}



      define line5 witnessev1: Simp2 witnessev1


>>       line5: [(.u1_1:obj),(witnessev1_1:that
>>            (line3 Witnesses .u1_1)) => (---:
>>            that Exists([(y_5:obj) => (((y_5
>>               E B) & (theimage = (.u1_1 ; y_5))):
>>               prop)]))
>>            ]
>>         {move 2}



      open

         declare v1 obj

>>          v1: obj {move 4}



         declare witnessev2 that Witnesses \
            line5 witnessev1 v1

>>          witnessev2: that (line5(witnessev1)
>>            Witnesses v1) {move 4}



         define line6 witnessev2: Simp1 witnessev2


>>          line6: [(.v1_1:obj),(witnessev2_1:
>>               that (line5(witnessev1) Witnesses
>>               .v1_1)) => (---:that (.v1_1 E
>>               B))]
>>            {move 3}



         define line7 witnessev2: Simp2 witnessev2


>>          line7: [(.v1_1:obj),(witnessev2_1:
>>               that (line5(witnessev1) Witnesses
>>               .v1_1)) => (---:that (theimage
>>               = (u1 ; .v1_1)))]
>>            {move 3}



         define line8 witnessev2: Iff1 Subs1 \
            line7 witnessev2 Simp2 theimagefact, \
            Ui s, Pair u1 v1

>>          line8: [(.v1_1:obj),(witnessev2_1:
>>               that (line5(witnessev1) Witnesses
>>               .v1_1)) => (---:that ((s = u1)
>>               V (s = .v1_1)))]
>>            {move 3}



         open

            declare case1 that s=u1

>>             case1: that (s = u1) {move 5}



            declare u2 obj

>>             u2: obj {move 5}



            goal that One [u2=> (u2 E theimage) \
                  & ~(u2 = s)] \
               



>>             Goal: that One([(u2_75:obj) =>
>>                  (((u2_75 E theimage) & ~((u2_75
>>                  = s))):prop)])
>>               

            declare case2 that s=v1

>>             case2: that (s = v1) {move 5}



            goal that One [u2=> (u2 E theimage) \
                  & ~(u2 = s)] \
               



>>             Goal: that One([(u2_110:obj)
>>                  => (((u2_110 E theimage) &
>>                  ~((u2_110 = s))):prop)])
>>               
% pause


            close

         close

      close

   close
\end{verbatim}

Above is Zermelo's definition of the ``immediate equivalence" of disjoint sets $A$ and $B$.  We also define the notion of a mapping from $A$ to $B$, where $A,B$ are disjoint, 
realized as a subset $F$ of the binary product of $A$ and $B$ defined above with the property that each element of $A \cup B$ belongs to exactly one element of $F$, and the notion of correspondence of $c$ in one of the sets with a $d$ in the other set via $F$.

\begin{verbatim}Lestrade execution:

% paragraph 16


define Mappings disjev: Set(Sc product disjev, \
   Mapping(disjev))

>> Mappings: [(.A_1:obj),(.B_1:obj),(disjev_1:
>>      that (.A_1 disjoint .B_1)) => ((Sc(product(disjev_1))
>>      Set [(F_2:obj) => ((disjev_1 Mapping F_2):
>>         prop)])
>>      :obj)]
>>   {move 0}


\end{verbatim}

Here we define the set of all mappngs witnessing the equivalence of disjoint $A$ and $B$.  This set is nonempty iff $A \sim B$ holds.  Note the use of the curried abstraction
{\tt Mapping(disjev)} as an argument.


\end{document}

quit
