\documentclass[12pt]{article}

\usepackage{amssymb}

\title{Implementation of Zermelo's work of 1908 in Lestrade:  Part III, opening of Zermelo well-ordering theorem argument}

\author{M. Randall Holmes}

\begin{document}

\maketitle

\section{Introduction}
 
This document was originally titled as an essay on the proposition that mathematics is what can be done in Automath (as opposed to what can be done in ZFC, for example).  Such an essay is still in in my mind, but this particular document has transformed itself into the large project of implementing Zermelo's two important set theory papers of 1908 in Lestrade, with the further purpose of exploring the actual capabilities of Zermelo's system of 1908 as a mathematical foundation, which we think are perhaps underrated.

This is a new version of this document in modules, designed to make it possible to work more efficiently without repeated execution of slow log files when they do not need to be revisited.

\section{Zermelo's 1908 proof of the well-ordering theorem}

I am now going to carry out or at least attempt a significant piece of mathematics in Lestrade.  I shall attempt to directly translate Zermelo's 1908 proof of the well-ordering theorem
(which was published at about the same time as the axiomatization implemented above:  they are intimately connected) into a Lestrade proof.

Zermelo starts by stating prerequisites which are found in the axiomatization.  Point I:  he requires the axiom of Separation, stated above.  He points out as an important corollary the existence of
relative complements.

As point II he requires the existence of power sets, provided by us in the axiomatization above. 

As point III, he notes that Separation implies the existence of intersections of (nonempty) sets.

In earlier versions, declarations of the above points appeared here, but we have moved them back into the second file, which implements the axiomatics paper.

Zermelo states the following Theorem, the central result in his argument (the well ordering theorem follows immediately from this theorem and the axiom of choice, as we will verify below).

\begin{description}

\item[Theorem:]  If with every nonempty subset of a set $M$ an element of the subset is associated by some law as ``distinguished element", then ${\cal P}(x)$, the set of all subsets of $M$, possesses one and only one subset ${\bf M}$ such that to every arbitrary subset $P$ of $M$  there always corresponds one and only one element $P_0$ of $M$ that includes $P$ as a subset and contains an element of $P$ as its distinguished element.  The set $M$ is well-ordered by ${\bf M}$.

\end{description}

The apparent second-order quality of this theorem is dispelled by the proof in Lestrade, in which we see how we can introduce the ``law" referred to as a hypothetical without in fact allowing quantification over such laws (and nonetheless successfully carry out the proof of the corollary).



We declare the hypotheses of the theorem, the set $M$ and the unspecified law that, given a subset $S$ of $M$, allows us to select a distinguished element of $S$.

\begin{verbatim}Lestrade execution:


load whatismath2

>> Inspector Lestrade says:  No such theory to load:
whatismath2.lti must be read before this file can be read
>> Hit return to continue

quit
