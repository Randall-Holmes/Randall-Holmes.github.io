\documentclass[12pt]{article}

\title{Introduction to the foundations of mathematics, using the Lestrade Type Inspector}

\author{Randall Holmes}

\date{10/31/2017 10:30 am:  Systematic introduction of terms with bound variables.}

\usepackage{amssymb}

\begin{document}

\maketitle

\tableofcontents

\newpage

The purpose of this document is to introduce a reader to the foundations of logic and mathematics using the Lestrade Type Inspector, a piece of software designed to allow the specification of mathematical objects in a very general way.  It could also be used as an introduction to the software for someone familiar with the foundational subject matter.

Lestrade implements a particular very general framework for the implementation of mathematical objects, statements, and proofs of statements.  Part of the underpinning of the approach is that in this framework the statements and their proofs are viewed as  particular kinds of mathematical object themselves.

The actual implementation of foundational concepts of logic and mathematics here is not dictated by Lestrade:  there is considerable latitude for different design decisions in the implementation of logic and mathematics in the framework.  We may sometimes indicate alternative approaches.

\section{Initial examples.  Conjunction, implication, and their rules.}

We begin with the implementation of the very simple concepts of logical conjunction, the use of the word ``and" to link sentences, and logical implication, the use of ``if$\ldots$then$\ldots$" to link sentences.

\begin{verbatim}Lestrade execution:


declare A prop

>> A: prop {move 1}



declare B prop

>> B: prop {move 1}


\end{verbatim}

Here is a bit of initial dialogue with Lestrade.  Here we use the {\tt declare} command to introduce two variables, $A$ and $B$, of type {\tt prop},
the type inhabited by mathematical statements.

Lines starting with Lestrade command names such as {\tt declare}, {\tt postulate}, {\tt define} are entered by the user.  Lines starting with {\tt >>} are Lestrade responses to commands typed by the user.

\begin{verbatim}Lestrade execution:


postulate & A B : prop

>> &: [(A_1:prop),(B_1:prop) => (---:prop)]
>>   {move 0}



postulate -> A B : prop

>> ->: [(A_1:prop),(B_1:prop) => (---:prop)]
>>   {move 0}


\end{verbatim}

Here we declare the operations of conjunction and implication.  At the moment, they look just the same:  the only thing Lestrade knows about them so far is that they are operations taking two proposition inputs to a proposition output.  Details of the input and output of Lestrade itself (the things the user enters and the replies that Lestrade produces) will be analyzed more carefully as we go forward.

\begin{verbatim}Lestrade execution:


define proptest A B : (A & B) -> A

>> proptest: [(A_1:prop),(B_1:prop) => (((A_1
>>      & B_1) -> A_1):prop)]
>>   {move 0}


\end{verbatim}

We illustrate another Lestrade command, using {\tt define} to introduce a defined operation.  The main point here is to notice that Lestrade supports
infix use of the conjunction and implication operators, though the Lestrade declaration commands requires their use in prefix position when they are newly declared.  The Lestrade user should get used to typing lots of parentheses, though she does not need to use as many as are displayed in the output:  she does need to be aware that in general terms all infix (or mixfix) operations have the same precedence and group to the right if explicit parentheses are not provided, and unary operations bind more tightly than binary or infix operations.

\begin{verbatim}Lestrade execution:


open

   declare A1 prop

>>    A1: prop {move 2}



   declare B1 prop

>>    B1: prop {move 2}



   define proptest2 A1 B1 : (A1 & B1) -> \
      A1

>>    proptest2: [(A1_1:prop),(B1_1:prop) =>
>>         (---:prop)]
>>      {move 1}



   close
\end{verbatim}

Here we do something subtle in the Lestrade declaration environment which we don't explain fully for now:  the {\tt open}$\ldots${\tt close} environment creates
a separate little Lestrade context.  The alternative version {\tt proptest2} of our defined notion will behave a little differently as we see at once.

\begin{verbatim}Lestrade execution:


declare C prop

>> C: prop {move 1}



declare D prop

>> D: prop {move 1}



define zorch C D: proptest C & D, D -> C


>> zorch: [(C_1:prop),(D_1:prop) => (((C_1 &
>>      D_1) proptest (D_1 -> C_1)):prop)]
>>   {move 0}



define zorch2 C D: proptest2 C & D, D -> \
   C

>> zorch2: [(C_1:prop),(D_1:prop) => ((((C_1
>>      & D_1) & (D_1 -> C_1)) -> (C_1 & D_1)):
>>      prop)]
>>   {move 0}


\end{verbatim}

Here we use {\tt proptest} and {\tt proptest2} to define new operations {\tt zorch} and {\tt zorch2}.  The interesting thing which happens is that the operation
{\tt proptest2} which was defined in its own little  local context gets expanded when it is used, while {\tt proptest} (which ``means" the same thing) is left unexpanded.  Expansion of definitions is the main kind of ``calculation" that Lestrade does, though we may detect it doing more complex things as we go forward.

Now we will return to our main line of development, introducing the machinery of proof in Lestrade.

\begin{verbatim}Lestrade execution:


declare aa that A

>> aa: that A {move 1}



declare bb that B

>> bb: that B {move 1}


\end{verbatim}

We declare new variables {\tt aa}  and {\tt bb}.  The sorts of these variables require special explanation.  With each proposition $p$ of sort
{\tt prop}, we associate a new sort {\tt that} $p$ inhabited by proofs of $p$, or, perhaps better, evidence that $p$ is true.

\begin{verbatim}Lestrade execution:


postulate Andproof0 A B aa bb:that A & B


>> Andproof0: [(A_1:prop),(B_1:prop),(aa_1:that
>>      A_1),(bb_1:that B_1) => (---:that (A_1
>>      & B_1))]
>>   {move 0}



postulate Andproof aa bb:that A & B

>> Andproof: [(.A_1:prop),(aa_1:that .A_1),(.B_1:
>>      prop),(bb_1:that .B_1) => (---:that (.A_1
>>      & .B_1))]
>>   {move 0}



define Selfand aa : Andproof aa aa

>> Selfand: [(.A_1:prop),(aa_1:that .A_1) =>
>>      ((aa_1 Andproof aa_1):that (.A_1 & .A_1))]
>>   {move 0}


\end{verbatim}

And now we introduce a rule of proof:  if we have evidence that $A$ and evidence that $B$, we can conclude $A \wedge B$:  to conclude $A \wedge B$ is equivalent to postulateing or defining an object of sort {\tt that A \& B}.  The symbol $\wedge$ is the standard representation of ``and" in formal logic; Lestrade uses {\tt \&} because of the limitations of the typewriter keyboard.

The fully verbose version {\tt Andproof0} takes the arguments $A$, $B$, {\tt aa} and {\tt bb}, and the Lestrade framework requires these arguments officially.  Notice though that from the arguments {\tt aa} and {\tt bb} we can deduce what $A$ and $B$ have to be:  the second version {\tt Andproof} uses the
``implicit argument inference" feature of Lestrade to allow the user to enter just the names of the proofs, deducing the names of the propositions proved.  The declaration that Lestrade gives as a response makes it clear that it knows about the hidden arguments.

{\tt Selfand} is a defined operation on proofs:  from a proof of $A$ it generates a proof of $A \wedge A$.  We might think that this is a proof of ``If $A$ then $A\wedge A$, or $A  \rightarrow (A\wedge A)$:  the fact that we might think this is a hint as to how Lestrade represents proofs of implications.  In fact, {\tt Selfand} is a rule of inference, not a proof of a conditional, but it can be used to prove the conditional as we will see below.

\begin{verbatim}Lestrade execution:


declare xx that A & B

>> xx: that (A & B) {move 1}



postulate Simplification1 xx : that A

>> Simplification1: [(.A_1:prop),(.B_1:prop),
>>      (xx_1:that (.A_1 & .B_1)) => (---:that
>>      .A_1)]
>>   {move 0}



postulate Simplification2 xx : that B

>> Simplification2: [(.A_1:prop),(.B_1:prop),
>>      (xx_1:that (.A_1 & .B_1)) => (---:that
>>      .B_1)]
>>   {move 0}


\end{verbatim}

For completeness, we introduce the other two (quite obvious) rules of conjunction:  from evidence {\tt xx} for $A \wedge B$, we can extract evidence for $A$ and evidence for $B$.  We introduce them in forms which hide implicit arguments.

\begin{verbatim}Lestrade execution:


declare cc that A->B

>> cc: that (A -> B) {move 1}



postulate Mp aa cc: that B

>> Mp: [(.A_1:prop),(aa_1:that .A_1),(.B_1:prop),
>>      (cc_1:that (.A_1 -> .B_1)) => (---:that
>>      .B_1)]
>>   {move 0}


\end{verbatim}

This snippet of code embodies the traditional rule of {\em modus ponens}:  given evidence for $A$ and evidence for $A \rightarrow B$, we have evidence for $B$.  We have only given the version with implicit arguments.  It is interesting to note that the order of the arguments of {\tt Mp} is probably not what we would choose if we were writing all the arguments explicitly:  but it works.  In general terms, Lestrade places deduced implicit arguments as late as possible in  the argument list.

    \begin{verbatim}
open

     declare aaa that A

>>      aaa: that A {move 2}


     postulate ded aaa that B

>>      ded: [(aaa_1:that A) => (---:that B)]
>>        {move 1}


     close

postulate Deduction ded : that A -> B

>> Deduction: [(.A_1:prop),(.B_1:prop),(ded_1:
>>        [(aaa_2:that .A_1) => (---:that .B_1)])
>>        => (---:that (.A_1 -> .B_1))]
>>   {move 0}


\end{verbatim}

Above is the old style implementation of the proof of the deduction theorem without variable binding terms.

\begin{verbatim}Lestrade execution:


declare aaa1 that A

>> aaa1: that A {move 1}



declare ded [aaa1 => that B] \
   



>> ded: [(aaa1_1:that A) => (---:that B)]
>>   {move 1}



postulate Deduction ded : that A -> B

>> Deduction: [(.A_1:prop),(.B_1:prop),(ded_1:
>>      [(aaa1_2:that .A_1) => (---:that .B_1)])
>>      => (---:that (.A_1 -> .B_1))]
>>   {move 0}


\end{verbatim}


This piece of code implements a standard strategy for proving implications, in the more compact style which variable binding terms allow (it is not necessary to open a new move to declare {\tt ded} as in the code given above):  if assuming $A$ allows us to deduce $B$, we can conclude $A \rightarrow B$.  What is quite tricky is how Lestrade represents this.  We open a little environment in which we postulate the function {\tt ded} which takes 
evidence {\tt aaa} for $A$ to evidence for $B$:  we close this environment, and the symbol {\tt ded} remains as a variable representing a function of this type.  We are then able to postulate a function which takes any such function to evidence for $A \rightarrow B$.  We {\em will\/} in due course have a careful discussion of Lestrade environments.  For the moment, we will content ourselves with giving an example of how this is used.

A side remark to those in the know:  it is important to notice that a proof of an implication is not identified with a function from proofs of its antecedent to proofs of its consequent, but obtained from such a function by applying a constructor casting from a function sort to an object sort (see the next section on metaphysics of Lestrade for a discussion of object vs. function sorts).

   \begin{verbatim}
open

     declare aaa that A

>>      aaa: that A {move 2}


     define selfand aaa : Andproof aaa aaa

>>      selfand: [(aaa_1:that A) => ((aaa_1 
>>             Andproof aaa_1):that (A & A))]
>>        {move 1}


     close

define Selfand2 A : Deduction selfand

>> Selfand2: [(A_1:prop) => (Deduction([(aaa_2:
>>             that A_1) => ((aaa_2 Andproof aaa_2):
>>             that (A_1 & A_1))])
>>        :that (A_1 -> (A_1 & A_1)))]
>>   {move 0}

\end{verbatim}

A proof given in the old style, discussed below.

\begin{verbatim}Lestrade execution:


define Selfand2 A : Deduction [aaa1 => Andproof \
      aaa1 aaa1] \
   



>> Selfand2: [(A_1:prop) => (Deduction([(aaa1_2:
>>         that A_1) => ((aaa1_2 Andproof aaa1_2):
>>         that (A_1 & A_1))])
>>      :that (A_1 -> (A_1 & A_1)))]
>>   {move 0}


\end{verbatim}

Here we actually prove the theorem $A \rightarrow (A \wedge A)$ for any proposition $A$, in a very compact form allowed by use of the term {\tt [aaa1 => Andproof aaa1 aaa1]} for the function declared as {\tt selfand} in the open/close block in the original proof.   It is interesting to observe that this is actually a function of the proposition $A$ rather than of a proof of $A$:  whether $A$ itself is true or not, this theorem is true, and the definition of the function {\tt Selfand2} encapsulates reasoning justifying this:  from a proposition $A$, we can postulate evidence for the proposition $A \wedge A$.

An interesting feature of the Lestrade output is that it contains a mathematical expression $${\tt [(A_1:prop) => (Deduction([(aaa_2: that \,A_1) => ((aaa_2 \,Andproof\, aaa_2): that (A_1 \& A_1))])}$$

standing for the proof as a mathematical object.  Lestrade allows itself output notation significantly more complex that the user input notation, but with experience we will be able to read this.

\newpage

\section{The Curry Howard isomorphism:  defining type constructors analogous to propositional connectives}

We recapitulate the basic declarations for the propositional connectives of conjunction and implication, and in parallel implement the type constructors which build Cartesian products and function spaces, along with the basic operations on the complex types.  The analogy between the propositional constructions on the one hand and the type constructions on the other is known as the Curry-Howard isomorphism.

  \begin{verbatim}

declare A prop

>> A: prop {move 1}


declare B prop

>> B: prop {move 1}

postulate & A B : prop

>> &: [(A_1:prop),(B_1:prop) => (---:prop)]
>>   {move 0}

postulate -> A B : prop

>> ->: [(A_1:prop),(B_1:prop) => (---:prop)]
>>   {move 0}

\end{verbatim}

We declare the Cartesian product and function space constructors.

\begin{verbatim}Lestrade execution:


declare At type

>> At: type {move 1}



declare Bt type

>> Bt: type {move 1}



postulate X At Bt : type

>> X: [(At_1:type),(Bt_1:type) => (---:type)]
>>   {move 0}



postulate ->> At Bt : type

>> ->>: [(At_1:type),(Bt_1:type) => (---:type)]
>>   {move 0}


\end{verbatim}

  \begin{verbatim}

declare aa that A

>> aa: that A {move 1}


declare bb that B

>> bb: that B {move 1}

postulate Andproof aa bb:that A & B

>> Andproof: [(.A_1:prop),(aa_1:that .A_1),(.B_1:
>>        prop),(bb_1:that .B_1) => (---:that 
>>        (.A_1 & .B_1))]
>>   {move 0}

declare xx that A & B

>> xx: that (A & B) {move 1}


postulate Simplification1 xx : that A

>> Simplification1: [(.A_1:prop),(.B_1:prop),
>>        (xx_1:that (.A_1 & .B_1)) => (---:that 
>>        .A_1)]
>>   {move 0}


postulate Simplification2 xx : that B

>> Simplification2: [(.A_1:prop),(.B_1:prop),
>>        (xx_1:that (.A_1 & .B_1)) => (---:that 
>>        .B_1)]
>>   {move 0}

\end{verbatim}

We introduce the pair operation and projection functions, which we see are formally analogous to the logical rules of conjunction and simplification.

\begin{verbatim}Lestrade execution:


declare aat in At

>> aat: in At {move 1}



declare bbt in Bt

>> bbt: in Bt {move 1}



postulate Pair aat bbt in At X Bt

>> Pair: [(.At_1:type),(aat_1:in .At_1),(.Bt_1:
>>      type),(bbt_1:in .Bt_1) => (---:in (.At_1
>>      X .Bt_1))]
>>   {move 0}



declare xxt in At X Bt

>> xxt: in (At X Bt) {move 1}



postulate proj1 xxt in At

>> proj1: [(.At_1:type),(.Bt_1:type),(xxt_1:
>>      in (.At_1 X .Bt_1)) => (---:in .At_1)]
>>   {move 0}



postulate proj2 xxt in Bt

>> proj2: [(.At_1:type),(.Bt_1:type),(xxt_1:
>>      in (.At_1 X .Bt_1)) => (---:in .Bt_1)]
>>   {move 0}


\end{verbatim}

     \begin{verbatim}

declare cc that A->B

>> cc: that (A -> B) {move 1}


postulate Mp aa cc: that B

>> Mp: [(.A_1:prop),(aa_1:that .A_1),(.B_1:prop),
>>        (cc_1:that (.A_1 -> .B_1)) => (---:that 
>>        .B_1)]
>>   {move 0}

open

     declare aaa that A

>>      aaa: that A {move 2}


     postulate ded aaa that B

>>      ded: [(aaa_1:that A) => (---:that B)]
>>        {move 1}


     close

postulate Deduction ded : that A -> B

>> Deduction: [(.A_1:prop),(.B_1:prop),(ded_1:
>>        [(aaa_2:that .A_1) => (---:that .B_1)])
>>        => (---:that (.A_1 -> .B_1))]
>>   {move 0}

\end{verbatim}

We introduce function application and the formation of function objects from functions (lambda abstraction), which we see are formally analogous to modus ponens and the deduction theorem.  The arguments of function application are supplied in converse order to those of modus ponens (not because there is any virtue to this but because existing text below written earlier would have had to be extensively revised to reverse the order of the arguments of {\tt Mp})!)

\begin{verbatim}Lestrade execution:


declare cct in At ->> Bt

>> cct: in (At ->> Bt) {move 1}



declare aat2 in At

>> aat2: in At {move 1}



postulate Apply cct aat2 in Bt

>> Apply: [(.At_1:type),(.Bt_1:type),(cct_1:
>>      in (.At_1 ->> .Bt_1)),(aat2_1:in .At_1)
>>      => (---:in .Bt_1)]
>>   {move 0}



declare dedt [aat2 => in Bt] \
   



>> dedt: [(aat2_1:in At) => (---:in Bt)]
>>   {move 1}



postulate Lambda dedt in At ->> Bt

>> Lambda: [(.At_1:type),(.Bt_1:type),(dedt_1:
>>      [(aat2_2:in .At_1) => (---:in .Bt_1)])
>>      => (---:in (.At_1 ->> .Bt_1))]
>>   {move 0}


\end{verbatim}

In the section on equality, we will introduce more primitives for the case of types, which would have analogues for propositions if we were working in a constructive logic and wanted to carry out formal operations on proofs.

\newpage

\section{A brief discussion of the metaphysics of Lestrade}

Probably we should explain ourselves a bit more.

The most general word used for things we talk about in Lestrade is {\em entity\/}.  Entities are further partitioned into {\em objects} and {\em functions}.

Entities have sorts:  the sort indicates what kind of thing we are talking about.  

The sorts of object can be reviewed quickly:

\begin{enumerate}

\item {\tt prop} is the sort of propositions, i.e., mathematical statements.

\item For each proposition $p$, we provide a sort {\tt that} $p$ inhabited by evidence that $p$ is true.  A proof of $p$ is such evidence, and explicitly constructed objects of sort {\tt that} $p$ will be referred to as ``proofs of $p$";  but to suppose that $p$ is true (to postulate an object of the sort {\tt that} $p$) is not the same thing as to suppose that $p$ has actually been proved or even can be proved.\footnote{A constructivist might presume that to suppose that $X$ is true is the same thing as to suppose that $X$ has been proved, but we do not presume a constructivist view here.}

\item {\tt obj} is a sort inhabited by untyped mathematical objects.

\item {\tt type} is a sort inhabited by ``type labels".  An example of an object of sort {\tt type} would be the label {\tt Nat} for the sort ``natural number".

\item For each $\tau$ of sort {\tt type} we provide a sort {\tt in} $\tau$ inhabited by objects of type $\tau$.  If $n$ is a natural number, it might be construed as of sort {\tt in Nat}.  Something of sort {\tt in} $\tau$ may more briefly be said to be of type $\tau$.

\end{enumerate}

If the reader notices an analogy between {\tt prop}/{\tt that} and {\tt type}/{\tt in}, she is perceptive.

Functions are more complicated and their sorts are more complicated.  A Lestrade function takes a list of arguments of a fixed length, each item of which is of a sort possibly determined by earlier arguments in the list, and yields output of a sort which may depend on its arguments.  A lot of the logical power of this framework comes from the fact that the sort of an argument of a function may depend on the values of earlier arguments, and the sort of the output may depend on the values of the inputs.  One mechanism which makes such dependencies possible is the fact that the object sorts of the form {\tt that} $p$ and {\tt in} $\tau$ may contain quite complex expressions abbreviated here by $p$, $\tau$;  we have already seen this in Lestrade output above;  function sorts also have complex internal structure which supports such dependencies.

The general notation for a function sort is $$(x_1:\tau_1),\ldots,(x_n:\tau_n) \Rightarrow (-,\tau)$$

The variables $x_i$ representing the arguments are dummy variables (they are ``bound" in this expression)  Distinct function sort expressions (including ones which might appear as $\tau_i$'s or parts of $\tau_i$'s) have different dummy variables.  Each $\tau_i$ is an expression representing the sort of $x_i$, which may be an object or a function sort and  is allowed to include $x_j$'s only for $j<i$.  The output sort $\tau$ will be an object sort, not a function sort, and may include any of the $x_i$.

A species of notation for a function used in Lestrade output is $$(x_1:\tau_1),\ldots,(x_n:\tau_n) \Rightarrow (y,\tau),$$ where $y$ is an expression for the value of the function which may of course include any or all of the $x_i$'s (and which must be of the object sort $\tau$):  the formation rules for such an expression are the same as for function sorts:  the function sort expression $(x_1:\tau_1),\ldots,(x_n,\tau_n) \Rightarrow (-,\tau)$  must be well-formed for the expression above to be well-formed.

When a function is declared in Lestrade and explicitly defined, the sort reported for it is actually this notation for it.  The user will always refer to it using its name (the identifier declared with this type):  users do not enter function sort notations or function notations.

The account given here should allow the reader to make a stab at interpreting details of Lestrade responses to user commands which we skated over above.

\section{The care and feeding of declarations:  the system of possible worlds or ``moves"}

We have to give an account of the declaration environments of Lestrade.  We'll do this in the simplest way (in which all declared environments are anonymous and in a sense ephemeral:  we will look at the consequences of allowing environments to be named and saved in an appended subsection).

In the simplest model of what we are doing, the Lestrade user is working in a finite sequence of environments indexed by natural numbers, called
``move 0", ``move 1",\ldots,``move $i$", ``move $i+1$".   Move $i$ is called ``the last move" and move $i+1$ is called ``the next move" (elsewhere sometimes ``the current move").  There are always at least two moves, so all four explicitly given items are present, though they may not all be distinct.  Each move contains an ordered list of declarations of identifiers as representing entities of given sorts.  The sort of an identifier declared at a given sort will not mention identifiers declared at moves of higher index or declared at the same move but later in the list of declarations.  Entities declared at the last move or earlier moves are to be thought of as constant;
entities declared at the next move are to be thought of as variable.

By a fresh identifier we mean an identifier not declared at the moment in any move.  It will never be the case that the same identifier is declared more than once.

The {\tt declare} command takes a fresh identifier and an object or function sort as its two arguments (in that order) and declares the identifier as a variable
of the given sort in the next move (placed last in the order on the move).   Object sorts are represented by {\tt prop}, {\tt obj}, {\tt type}, or by {\tt that} $p$ when $p$ is of sort {\tt prop}, or {\tt in} $\tau$ when $\tau$ is of sort {\tt type}.  Function sorts are represented by terms $[x_1,\ldots,x_n \Rightarrow \tau]$
where the $x_i$'s are variables declared in the next move and $\tau$ is an object term.

The {\tt postulate} command takes a fresh identifier followed by zero or more arguments (variables declared previously at the next move, appearing in the order in which their declarations appear in the next move), followed
by an object sort  [optionally separated from the previous arguments by a colon ``{\tt :}";  this is sometimes mandatory for the sake of the parser].  If there are zero arguments, the identifier is declared as being of the given sort, but at (the end of) the last move rather than the next move.  This can be thought of as declaring a constant (relatively speaking, as we will see).  If there are arguments $x_i$ of types $\tau_i$
and the output type is $\tau$, the identifier is declared at the last move (not at the next move!) and appearing finally in the order on the last move, as a function of sort

$$(x_1:\tau_1),\ldots,(x_n:\tau_n) \Rightarrow (-,\tau)$$ (with the refinement that the names of the parameters, since they become bound, are systematically changed).

The {\tt postulate} command can be thought of as declaring axioms and primitive notions, when it is used when $i=0$.  At higher indexed moves, what it is doing is subtler, but will become evident with experience:  we will see that in combination with the {\tt open} and {\tt close} commands it allows
declaration of function variables.

The {\tt define} command is a sister command of the {\tt postulate} command:  the keyword is followed by an identifier, then by zero or more arguments, variables $x_i$ of type $\tau_i$ appearing in the same order in which they were declared, then by a Lestrade expression $y$ of an object type [always separated from the previous arguments by a colon {\tt :}].  The identifier is defined at the last move (not the next move), and finally in the order on the last move,  as $$(x_1:\tau_1),\ldots,(x_n,\tau_n) \Rightarrow (y,\tau),$$ as long as sort checking reports that this is possible [in the case where there are no arguments, it is just defined as $y$].   Identifiers declared in this way are not eligible to serve as arguments of functions (they  are not variables).  

The {\tt open} command introduces a new move with the index $i+2$:  as it were, the parameter $i$ is incremented, so that the old ``next move", move $i+1$,
becomes the new ``last move", and the new move $i+2$ is the new next move.  We call this action ``opening move $i+2$".

The {\tt close} command erases all information in move $i+1$ and decrements the parameter $i$, if $i>1$;  it is not possible to close move 1.
The old ``last move" move $i$ becomes the new next move, and move $i-1$ becomes the new last move.  We call this action ``closing move $i+1$".

The {\tt clearcurrent} command removes all declarations from move $i+1$ but does not decrement the counter:  at the end of this action,
move $i$ is unchanged and move $i+1$ is empty.  This amounts to clearing accumulated variable declarations;  it is needed because there is no other way
to remove declarations from move 1.  It will be a while before we see uses of this command:  over a large initial segment of the document, we will suppose that the program remembers all previous move 1 declarations (which can cause the namespace to get rather cluttered!)

There are devices whereby moves can be saved and then reopened after being closed, which lead to some complexities, but these can be ignored for the present.

It may seem that we cannot create a function variable (recall that we said above that functions can have functions as arguments) but we can and in fact we have already illustrated this in an example above.  One creates a function variable in move $i+1$ by opening move $i+2$, declaring desired variable parameters, postulateing a function of the desired type in move $i+1$ in its then role as the last move, then closing move $i+2$ whereupon the constructed function is now a variable.   We did this (and the reader may now review the example to see that it conforms with our account) in postulateing {\tt Deduction} above, which needed the function parameter {\tt ded}.

Functions found in the next move which were  introduced by the {\tt define} command when there were more moves do not become variables:  they are as it were ``variable expressions", and a distinctive point about these is that where they are used in the final argument of a {\tt define} command they must be expanded out (as {\tt proptest2} was in an example above)
as a defined identifier at move $i+1$ cannot appear in a declaration at move $i$.  Where a defined operator declared at the last move is used in applied position, its application is carried out (suitable substitutions are made) as in the example above;  where it appears as an argument it is replaced by its anonymous formal notation.

A further point about declarations of functions which must be noted, though its details are nasty, is the permission we give ourselves to not give all arguments of a function under certain circumstances.  In fact, any non-defined identifier declared at the next move appearing in the sort of a variable appearing as an argument of a function must itself be an earlier argument of that function:  the input/output mechanism of Lestrade itself allows us to hide this, omitting arguments when their presence can be deduced.  If we did not do this, we would have a lot of arguments in argument lists which ``felt" redundant, like $A$ and $B$ as arguments of {\tt Andproof0} (it being evident from the sorts of {\tt aa} and {\tt bb} what $A$ and $B$ must be).

We make a philosophical remark at this point.  The currently popular view of the nature of functions is that they are as it were actually infinite tables containing all their values.  We resist this.  We regard a function as determined by a  specification of how a value is to be obtained (or, in the case of a primitive notion,  simply {\em that\/} a value of given sort can be obtained) from {\em any given} sequence of inputs of appropriate sorts which may happen to be presented now or in the future, not from all possible such sequences in a way given all at once.  The arbitrary objects used as inputs in a function definition can each be viewed as a single object drawn from a ``possible world" (``the next move") accessible from the world which is our current standing point (``the last move").  Another metaphor which might be helpful is that objects at the next move are things to be chosen in the future;  we do not know anything about them except what is given in their sort.  When we declare a function as a primitive, we declare that there is a construction principle which for
any given inputs of given types will give an output of that type:  we do not presume that we have given such outputs for all possible inputs (such outputs are produced on demand when we apply the constructed function to specific inputs).  In this way we preserve the possibility of the view that all infinities are potential, never completely realized.  Nonetheless, the mathematical consequences of the particular Lestrade theory we present are fully classical.

\subsection{Namespace management refined:  saving and retrieving environments}

With the limited environment handling given above, there is no way to remove or revise declarations of variables and variable expressions in move 1 other than clearing all of them.  After a while, it is quite hard to remember what sorts have been assigned to parameters and variable expressions, and for that matter what order they appear in (recalling that parameters in {\tt postulate} and {\tt define} commands must appear in order of declaration).  We have already noted
that the {\tt clearcurrent} command will clear all declarations at the next move.

More intelligent namespace management is supported by the full specification of the {\tt open}, {\tt clearcurrent}, and {\tt save} commands.

Each move is assigned a name.  The default name is its numeral index (the $j$ such that it is move $j$).  The command {\tt save envname} will save the next move with the name
{\tt envname}, associated with the list of names of preceding moves at the time it is saved (a saved move is actually identified by the sequence of names of all moves at the time it is saved, and this is how it is identified internally;  this means that moves saved in different contexts can quite safely be tagged with the same name).  The command {\tt open envname} will open an already existing move (of the right index, wth the same preceding moves) with the name
{\tt envname} or if there is no such move, or create a new blank move with that name.  The command {\tt clearcurrent envname} will clear the net move and replace it with a move named {\tt envname} if there is such a move with the appropriate preceding names of moves associated with it or replace it with a blank move of that name otherwise.  A move cannot be saved or opened with its default numeral name:  the reason for this is that we do not want the
parameterless {\tt open} or {\tt clearcurrent} command to unexpectedly invoke declarations from a saved environment.  For the same reason, no move other than move 0 in the sequence of moves associated with a named move created by an application of {\tt save}, {\tt open}, or {\tt clearcurrent} may have its default numeral name.

Any identifiers in a saved environment which conflict with identifiers declared in earlier moves since it was saved have {\tt '} or {\tt \$} appended to them, depending on whether they are alphanumeric or special character identifiers.  Identifiers ending in these characters cannot be declared by the user.

The effect of all of this is that that instead of having a linear sequence of moves, which we can think of as times or possible worlds, we have a tree structure.\footnote{which can still be thought of using a temporal metaphor as working out the consequences of different choices on alternative timelines, as it were.}

\section{A proof as an example $A \wedge B \rightarrow B \wedge A$.}

We give the proof of a simple theorem of propositional logic, then present the proof in the form of Lestrade declarations.

\begin{description}

\item[Theorem:]  $A \wedge B \rightarrow B \wedge A$

Assume $A \wedge B$ for the sake of argument:  our goal is to show that $B \wedge A$ follows.

$B$ follows from $A \wedge B$ by simplification.  $A$ follows from $A \wedge B$ by simplification.  

The local conclusion $B \wedge A$ follows by conjunction from $B$ and $A$.

By deduction, we can conclude $A \wedge B \rightarrow B \wedge A$.

\end{description}

\begin{verbatim}Lestrade execution:


open

   declare yy that A & B

>>    yy: that (A & B) {move 2}



   define zz yy : Simplification1 yy

>>    zz: [(yy_1:that (A & B)) => (---:that
>>         A)]
>>      {move 1}



   define ww yy : Simplification2 yy

>>    ww: [(yy_1:that (A & B)) => (---:that
>>         B)]
>>      {move 1}



   define uu yy : Andproof (ww yy, zz yy)


>>    uu: [(yy_1:that (A & B)) => (---:that
>>         (B & A))]
>>      {move 1}



   close

define Andconj A B: Deduction uu

>> Andconj: [(A_1:prop),(B_1:prop) => (Deduction([(yy_2:
>>         that (A_1 & B_1)) => ((Simplification2(yy_2)
>>         Andproof Simplification1(yy_2)):that
>>         (B_1 & A_1))])
>>      :that ((A_1 & B_1) -> (B_1 & A_1)))]
>>   {move 0}


\end{verbatim}

The Lestrade declarations given embody the proof given.  One very subtle point is that the functions {\tt ww} and {\tt yy} are
distinct from {\tt Simplification1} and {\tt Simplification2}, because the latter functions take additional arguments which are not visible.

A point to note is that the argument under the hypothesis $A \wedge B$, assumed for the sake of argument, corresponds to the introduction of a new environment by the {\tt open} command in which the variable {\tt yy} of sort {\tt that} $(A \& B)$ is declared.

\section{The Lestrade user input language}


We discuss practical details of entering mathematical expressions in the language of Lestrade.  This section concentrates on what users can enter at the keyboard. 

Lestrade identifiers are the first detail of the syntax.  An identifier is a string of characters of positive length, consisting of zero or one capital letters,
followed by zero or more lower case letters, followed by zero or more numerals.

A Lestrade object expression is either an identifier declared of an object sort, or an application expression $f(t_1,\ldots,t_n)$  where $f$ is
an identifier declared as of function type with $n$ arguments, and $t_1,\ldots,t_n$ are expressions of the correct sorts (some may be function expressions).
A mixfix expression $(t_1 \,f \,t_2, \ldots, t_n)$ is well-formed under the same conditions and has the same referent.  

The parentheses and commas in these expressions may be omitted under some circumstances.  All infix and mixfix operators have the same precedence and group to the right (in the absence of restrictive punctuation they will take as many arguments as they can).  A function symbol used as an argument must be followed by a comma or parenthesis to avoid it attempting to take the next expression as an argument.  A parenthesis following a function symbol will always be taken as opening an argument list (so if one wants to enclose the first argument in parentheses one must also enclose the entire argument list in parentheses).  A function symbol representing a function taking more than one argument
must be preceded by a comma when it might otherwise take a preceding object expression as a first argument [reading a mixfix expression].  A function symbol appearing as the first argument of a mixfix expression must be enclosed in parentheses to avoid the function symbol trying to eat the mixfix.

Function expressions include identifiers declared as of function type, and expressions $f(t_1,\ldots,t_m)$ where $m<n$, the number of arguments taken by $f$.  Such expressions are understood as functions of $(x_{m+1},\ldots,x_n)$, and may only appear as arguments, not function or mixfix symbols.  The parentheses around the argument list in such a function expression are mandatory.  In addition, there are $\lambda$-terms of quite general form, $[x_1,\ldots,x_n \Rightarrow T]$ where the $x_i$'s are variables declared at the next move and $T$ is an object term.

An additional important punctuation device is the use of a colon {\tt :} to separate the final argument of a {\tt postulate} or {\tt define} command from the preceding arguments.  The colon is optional in the {\tt postulate} command (in earlier versions it was sometimes needed if the final preceding argument was a function identifier; it is now (we believe) always optional); it is mandatory in the {\tt define} command.  (The colon is neither needed nor allowed in the {\tt declare} command).

Lestrade output will use infix form for functions of two arguments where the first argument is not of function type.  Lestrade output will never use mixfix notation for functions of more than two arguments.

In general, problems with parsing of input notation can be solved by explicitly writing more parentheses and commas.  In Lestrade  output, all parentheses and commas are shown.

\section{We begin considering ontology:  equality primitives introduced.  The biconditional as equality on propositions.  Identification of proofs of the same proposition.}

We are by no means through with logic, but we will begin to consider the treatment of objects.  In this section we introduce the notion of equality and its basic primitives.  Equality is defined for typed mathematical objects:  related notions applying to propositions and their proofs are discussed, and defining equality for untyped mathematical objects of sort {\tt obj} is straightforward but not discussed here.

\begin{verbatim}Lestrade execution:


declare T type

>> T: type {move 1}



declare tt1 in T

>> tt1: in T {move 1}



declare tpred [tt1 => prop] \
   



>> tpred: [(tt1_1:in T) => (---:prop)]
>>   {move 1}


\end{verbatim}


We introduce a general object type $T$ which will be a hidden parameter of our notions of equality.\footnote{The implicit argument feature in effect allows overloading of equality as an operation on each type;  it was very annoying in earlier versions without this feature that equality on typed objects was a ternary relation and so had quite unexpected syntax.  Genuine overloading could be achieved by for example declaring an addition operator {\tt +} in every type without exception and providing for its properties by axioms in each type where it is to be used independently.  Things that one expects to be uniformly true of all {\tt +} operations (commutativlty, for example) could be stipulated as axioms.}  We then introduce  a predicate {\tt tpred} of objects of type $T$ (i.e, of sort {\tt in} $T$).  

\begin{verbatim}Lestrade execution:


declare t in T

>> t: in T {move 1}



declare u in T

>> u: in T {move 1}



postulate = t u : prop

>> =: [(.T_1:type),(t_1:in .T_1),(u_1:in .T_1)
>>      => (---:prop)]
>>   {move 0}



declare eqev that t=u

>> eqev: that (t = u) {move 1}


\end{verbatim}

We introduce the primitive notion of equality and evidence of equality $t=u$.

\begin{verbatim}Lestrade execution:


declare tpredev that tpred t

>> tpredev: that tpred(t) {move 1}



postulate Substitution0 tpred, eqev tpredev \
   that tpred u

>> Substitution0: [(.T_1:type),(tpred_1:[(tt1_2:
>>         in .T_1) => (---:prop)]),
>>      (.t_1:in .T_1),(.u_1:in .T_1),(eqev_1:
>>      that (.t_1 = .u_1)),(tpredev_1:that tpred_1(.t_1))
>>      => (---:that tpred_1(.u_1))]
>>   {move 0}



define Substitution eqev tpredev : Substitution0 \
   tpred, eqev tpredev

>> Substitution: [(.T_1:type),(.t_1:in .T_1),
>>      (.u_1:in .T_1),(eqev_1:that (.t_1 = .u_1)),
>>      (.tpred_1:[(tt1_2:in .T_1) => (---:prop)]),
>>      (tpredev_1:that .tpred_1(.t_1)) => (Substitution0(.tpred_1,
>>      eqev_1,tpredev_1):that .tpred_1(.u_1))]
>>   {move 0}


\end{verbatim}

We introduce the substitution rule of equality, whose type is perhaps the most complex yet introduced.   There are two different versions with different choices of explicitly given arguments.

\begin{verbatim}Lestrade execution:


postulate Reflexeq t : that t=t

>> Reflexeq: [(.T_1:type),(t_1:in .T_1) => (---:
>>      that (t_1 = t_1))]
>>   {move 0}


\end{verbatim}

The other primitive rule of equality is the reflexivity rule $t=t$.  We will see that other familiar rules of equality such as symmetry and transitivity can be proved.

\begin{verbatim}Lestrade execution:


open

   declare t17 in T

>>    t17: in T {move 2}



   declare u17 in T

>>    u17: in T {move 2}



   declare v17 in T

>>    v17: in T {move 2}



   declare eqev17 that t17=u17

>>    eqev17: that (t17 = u17) {move 2}



   define eqsymm0 eqev17: Substitution0 [v17 \
         => v17=t17] \
      , eqev17 Reflexeq t17

>>    eqsymm0: [(.t17_1:in T),(.u17_1:in T),
>>         (eqev17_1:that (.t17_1 = .u17_1)) =>
>>         (---:that (.u17_1 = .t17_1))]
>>      {move 1}



   close

define Eqsymm eqev : eqsymm0 eqev

>> Eqsymm: [(.T_1:type),(.t_1:in .T_1),(.u_1:
>>      in .T_1),(eqev_1:that (.t_1 = .u_1)) =>
>>      (Substitution0([(v17_2:in .T_1) => ((v17_2
>>         = .t_1):prop)]
>>      ,eqev_1,Reflexeq(.t_1)):that (.u_1 = .t_1))]
>>   {move 0}


\end{verbatim}

We present the proof of symmetry of equality.  Notice the use of a variable binding term for the predicate in the crucial substitution step.

Other notions of equality for sorts of functions may be introduced, as well as equality for untyped objects of sort {\tt obj}.

We introduce the biconditional, which plays the role of equality for propositions.

\begin{verbatim}Lestrade execution:


define <-> A B : (A -> B) & (B -> A)

>> <->: [(A_1:prop),(B_1:prop) => (((A_1 ->
>>      B_1) & (B_1 -> A_1)):prop)]
>>   {move 0}



declare ppred [A => prop] \
   



>> ppred: [(A_1:prop) => (---:prop)]
>>   {move 1}



declare iffev that A <-> B

>> iffev: that (A <-> B) {move 1}



declare ppredev that ppred A

>> ppredev: that ppred(A) {move 1}



postulate Substitutionp0 ppred, iffev ppredev: \
   that ppred B

>> Substitutionp0: [(ppred_1:[(A_2:prop) =>
>>         (---:prop)]),
>>      (.A_1:prop),(.B_1:prop),(iffev_1:that
>>      (.A_1 <-> .B_1)),(ppredev_1:that ppred_1(.A_1))
>>      => (---:that ppred_1(.B_1))]
>>   {move 0}



define Substitutionp iffev ppredev : Substitutionp0 \
   ppred, iffev ppredev

>> Substitutionp: [(.A_1:prop),(.B_1:prop),(iffev_1:
>>      that (.A_1 <-> .B_1)),(.ppred_1:[(A_2:
>>         prop) => (---:prop)]),
>>      (ppredev_1:that .ppred_1(.A_1)) => (Substitutionp0(.ppred_1,
>>      iffev_1,ppredev_1):that .ppred_1(.B_1))]
>>   {move 0}



define Reflexp0 A : Deduction [aaa1 => aaa1] \
   



>> Reflexp0: [(A_1:prop) => (Deduction([(aaa1_2:
>>         that A_1) => (aaa1_2:that A_1)])
>>      :that (A_1 -> A_1))]
>>   {move 0}



declare afix that A

>> afix: that A {move 1}



define propfixform A afix : afix

>> propfixform: [(A_1:prop),(afix_1:that A_1)
>>      => (afix_1:that A_1)]
>>   {move 0}



define Reflexp A : propfixform (A<->A,Andproof(Reflexp0 \
   A,Reflexp0 A))

>> Reflexp: [(A_1:prop) => (((A_1 <-> A_1) propfixform
>>      (Reflexp0(A_1) Andproof Reflexp0(A_1))):
>>      that (A_1 <-> A_1))]
>>   {move 0}



define Reflexp1 A: Andproof(Reflexp0 A,Reflexp0 \
   A)

>> Reflexp1: [(A_1:prop) => ((Reflexp0(A_1)
>>      Andproof Reflexp0(A_1)):that ((A_1 ->
>>      A_1) & (A_1 -> A_1)))]
>>   {move 0}


\end{verbatim}

We make some observations about the biconditional development.  A primitive {\tt Substitutionp} is needed to justify substitution of logically equivalent propositions in general contexts, but the reflexivity property {\tt Reflexp} is a theorem derivable from primitives we have already.  Notice the use of {\tt propfixform} to force the type of the output of {\tt Reflexp} into the correct form:  what happens if we don't use it is exhibited in  the declaration of {\tt Reflexp1}.  The Lestrade matching facility is good enough that in fact {\tt Reflexp1} would be usable for exactly the same purposes as {\tt Reflexp};  the two functions match in type because Lestrade recognizes that the type of one is a definitional expansion of the type of the other.  The pragmatic advantages of {\tt Reflexp} for user understanding of what is going on are clear.

A notion of equality for objects of sorts {\tt that} $p$ (proofs or evidence) could be defined by analogy with what is given above for objects of sorts
{\tt in} $p$, and such a development could be given.  A radical alternative (not appropriate for example for a constructive logic) is the following:

\begin{verbatim}Lestrade execution:


declare proofpred [aaa1 => prop] \
   



>> proofpred: [(aaa1_1:that A) => (---:prop)]
>>   {move 1}



declare proofpredev that proofpred aa

>> proofpredev: that proofpred(aa) {move 1}



declare aax that A

>> aax: that A {move 1}



postulate Indifference proofpredev aax : \
   that proofpred aax

>> Indifference: [(.A_1:prop),(.proofpred_1:
>>      [(aaa1_2:that .A_1) => (---:prop)]),
>>      (.aa_1:that .A_1),(proofpredev_1:that
>>      .proofpred_1(.aa_1)),(aax_1:that .A_1)
>>      => (---:that .proofpred_1(aax_1))]
>>   {move 0}


\end{verbatim}

The primitive {\tt Indifference} takes a proof that a first proof of $p$ satisfies a predicate of proofs, and another proof of $p$, to a proof that the second proof of $p$ satisfies the same predicate.  In other words, {\tt Indifference} witnesses the fact that each type {\tt that} $p$ is in effect inhabited by no more than one object.

To assume such an axiom is optional.  If a constructive logic were preferred, in which information could be extracted from proofs, one would certainly not want such an axiom.  It should be noted in general that Lestrade is a very flexible framework in which many different logical approaches can be implemented:  our particular development of logical and mathematical concepts is in no way dictated by the framework.

\subsection{Equality and type constructions}

In this section we complete the primitives needed for Cartesian products and function spaces.  Analogous constructions for propositions would be wanted
in a constructive logic in which one wanted to extract information from proofs.

We implement the identities $\pi_1(x,y)=x; \pi_2(x,y)=y; (\pi_1(x),\pi_2(x))=x$.

\begin{verbatim}Lestrade execution:


postulate Proj1 aat bbt : that proj1 (Pair \
   aat bbt) = aat

>> Proj1: [(.At_1:type),(aat_1:in .At_1),(.Bt_1:
>>      type),(bbt_1:in .Bt_1) => (---:that (proj1((aat_1
>>      Pair bbt_1)) = aat_1))]
>>   {move 0}



postulate Proj2 aat bbt : that proj2 (Pair \
   aat bbt) = bbt

>> Proj2: [(.At_1:type),(aat_1:in .At_1),(.Bt_1:
>>      type),(bbt_1:in .Bt_1) => (---:that (proj2((aat_1
>>      Pair bbt_1)) = bbt_1))]
>>   {move 0}



postulate Proj3 xxt : that Pair(proj1 xxt, \
   proj2 xxt) = xxt

>> Proj3: [(.At_1:type),(.Bt_1:type),(xxt_1:
>>      in (.At_1 X .Bt_1)) => (---:that ((proj1(xxt_1)
>>      Pair proj2(xxt_1)) = xxt_1))]
>>   {move 0}


\end{verbatim}

We implement the identities $(\lambda x.T)(a) = T[a/x]$ ($\beta$-reduction) and extensionality for function objects.

\begin{verbatim}Lestrade execution:


declare aat3 in At

>> aat3: in At {move 1}



postulate Betared dedt, aat3 : that Apply(Lambda \
   dedt, aat3) = dedt aat3

>> Betared: [(.At_1:type),(.Bt_1:type),(dedt_1:
>>      [(aat2_2:in .At_1) => (---:in .Bt_1)]),
>>      (aat3_1:in .At_1) => (---:that ((Lambda(dedt_1)
>>      Apply aat3_1) = dedt_1(aat3_1)))]
>>   {move 0}



declare dedt2 [aat3 => in Bt] \
   



>> dedt2: [(aat3_1:in At) => (---:in Bt)]
>>   {move 1}



declare fnext [aat3 => that dedt aat3 = dedt2 \
      aat3] \
   



>> fnext: [(aat3_1:in At) => (---:that (dedt(aat3_1)
>>      = dedt2(aat3_1)))]
>>   {move 1}



postulate Fnext fnext that (Lambda dedt) \
   = Lambda dedt2

>> Fnext: [(.At_1:type),(.Bt_1:type),(.dedt_1:
>>      [(aat2_2:in .At_1) => (---:in .Bt_1)]),
>>      (.dedt2_1:[(aat3_3:in .At_1) => (---:in
>>         .Bt_1)]),
>>      (fnext_1:[(aat3_4:in .At_1) => (---:that
>>         (.dedt_1(aat3_4) = .dedt2_1(aat3_4)))])
>>      => (---:that (Lambda(.dedt_1) = Lambda(.dedt2_1)))]
>>   {move 0}


\end{verbatim}

\section{Natural numbers introduced}




In this section, we introduce the natural numbers, via the concept of iterated application of functions.

\begin{verbatim}Lestrade execution:


postulate Nat type

>> Nat: type {move 0}



postulate 0 in Nat

>> 0: in Nat {move 0}



declare n1 in Nat

>> n1: in Nat {move 1}



postulate Succ n1 in Nat

>> Succ: [(n1_1:in Nat) => (---:in Nat)]
>>   {move 0}


\end{verbatim}

The primitive notions of arithmetic are introduced.  These are the type of natural numbers, the number zero, and the successor operation.  We will next define iteration of a function a number of times, and we will see later that addition and multiplication are then definable.

\begin{verbatim}Lestrade execution:


declare nnn2 in Nat

>> nnn2: in Nat {move 1}



declare Tt [nnn2 => type] \
   



>> Tt: [(nnn2_1:in Nat) => (---:type)]
>>   {move 1}



open

   declare nn2 in Nat

>>    nn2: in Nat {move 2}



   declare ttt1 in Tt nn2

>>    ttt1: in Tt(nn2) {move 2}



   postulate F ttt1 in Tt (Succ nn2)

>>    F: [(.nn2_1:in Nat),(ttt1_1:in Tt(.nn2_1))
>>         => (---:in Tt(Succ(.nn2_1)))]
>>      {move 1}



   close

declare init in Tt 0

>> init: in Tt(0) {move 1}



declare n in Nat

>> n: in Nat {move 1}



postulate Iterate F, init n : in Tt n

>> Iterate: [(.Tt_1:[(nnn2_2:in Nat) => (---:
>>         type)]),
>>      (F_1:[(.nn2_3:in Nat),(ttt1_3:in .Tt_1(.nn2_3))
>>         => (---:in .Tt_1(Succ(.nn2_3)))]),
>>      (init_1:in .Tt_1(0)),(n_1:in Nat) => (---:
>>      in .Tt_1(n_1))]
>>   {move 0}



define Iterate0 Tt, F, init n: Iterate F, \
   init n

>> Iterate0: [(Tt_1:[(nnn2_2:in Nat) => (---:
>>         type)]),
>>      (F_1:[(.nn2_3:in Nat),(ttt1_3:in Tt_1(.nn2_3))
>>         => (---:in Tt_1(Succ(.nn2_3)))]),
>>      (init_1:in Tt_1(0)),(n_1:in Nat) => (Iterate(F_1,
>>      init_1,n_1):in Tt_1(n_1))]
>>   {move 0}



postulate Initialize F, init : that (Iterate \
   F, init 0) = init

>> Initialize: [(.Tt_1:[(nnn2_2:in Nat) => (---:
>>         type)]),
>>      (F_1:[(.nn2_3:in Nat),(ttt1_3:in .Tt_1(.nn2_3))
>>         => (---:in .Tt_1(Succ(.nn2_3)))]),
>>      (init_1:in .Tt_1(0)) => (---:that (Iterate(F_1,
>>      init_1,0) = init_1))]
>>   {move 0}



postulate Iterstep F, init n : that \
   (Iterate F, init (Succ n)) = F(Iterate F, \
   init n)

>> Iterstep: [(.Tt_1:[(nnn2_2:in Nat) => (---:
>>         type)]),
>>      (F_1:[(.nn2_3:in Nat),(ttt1_3:in .Tt_1(.nn2_3))
>>         => (---:in .Tt_1(Succ(.nn2_3)))]),
>>      (init_1:in .Tt_1(0)),(n_1:in Nat) => (---:
>>      that (Iterate(F_1,init_1,Succ(n_1)) =
>>      (n_1 F_1 Iterate(F_1,init_1,n_1))))]
>>   {move 0}


\end{verbatim}

We introduce the basic equations governing iterated application of a function.  The fact that the type of the output can depend on a numerical argument
will be used below in exhibiting the proof of the principle of mathematical induction.  The type valued function {\tt Tt} can be taken to be constant and
the function {\tt F} to be not dependent on the numerical argument to support simple iteration.

Note that the function variable {\tt F} really needs to be declared in an open/close block if it is to have the syntax with which it is presented, because it has an implicit argument; there is no provision for function variables
declared in one-line declarations to have implicit arguments.


\begin{verbatim}Lestrade execution:


declare nn99 in Nat

>> nn99: in Nat {move 1}



declare tt99 in T

>> tt99: in T {move 1}



declare F99 [tt99 => in T] \
   



>> F99: [(tt99_1:in T) => (---:in T)]
>>   {move 1}



declare init98 in T

>> init98: in T {move 1}



declare n98 in Nat

>> n98: in Nat {move 1}



define Simpleiter F99, init98 n98 : \
   Iterate [nn99,tt99=>F99 tt99] \
   , init98 n98

>> Simpleiter: [(.T_1:type),(F99_1:[(tt99_2:
>>         in .T_1) => (---:in .T_1)]),
>>      (init98_1:in .T_1),(n98_1:in Nat) => (Iterate([(nn99_4:
>>         in Nat),(tt99_4:in .T_1) => (F99_1(tt99_4):
>>         in .T_1)]
>>      ,init98_1,n98_1):in .T_1)]
>>   {move 0}



define Simpleiter2 F99, init98 n98 : Iterate \
   [nn99,tt99=>F99 tt99] \
   , init98 n98

>> Simpleiter2: [(.T_1:type),(F99_1:[(tt99_2:
>>         in .T_1) => (---:in .T_1)]),
>>      (init98_1:in .T_1),(n98_1:in Nat) => (Iterate([(nn99_4:
>>         in Nat),(tt99_4:in .T_1) => (F99_1(tt99_4):
>>         in .T_1)]
>>      ,init98_1,n98_1):in .T_1)]
>>   {move 0}



define Simpleinit F99,init98 : Initialize\
 \
   [nn99,tt99=>F99 tt99] \
   ,init98

>> Simpleinit: [(.T_1:type),(F99_1:[(tt99_2:
>>         in .T_1) => (---:in .T_1)]),
>>      (init98_1:in .T_1) => (Initialize([(nn99_4:
>>         in Nat),(tt99_4:in .T_1) => (F99_1(tt99_4):
>>         in .T_1)]
>>      ,init98_1):that (Iterate([(nn99_6:in Nat),
>>         (tt99_6:in .T_1) => (F99_1(tt99_6):
>>         in .T_1)]
>>      ,init98_1,0) = init98_1))]
>>   {move 0}



define Simpleiterstep F99,init98,n98 : Iterstep\
 \
   [nn99,tt99=>F99 tt99] \
   , init98 n98

>> Simpleiterstep: [(.T_1:type),(F99_1:[(tt99_2:
>>         in .T_1) => (---:in .T_1)]),
>>      (init98_1:in .T_1),(n98_1:in Nat) => (Iterstep([(nn99_4:
>>         in Nat),(tt99_4:in .T_1) => (F99_1(tt99_4):
>>         in .T_1)]
>>      ,init98_1,n98_1):that (Iterate([(nn99_6:
>>         in Nat),(tt99_6:in .T_1) => (F99_1(tt99_6):
>>         in .T_1)]
>>      ,init98_1,Succ(n98_1)) = F99_1(Iterate([(nn99_8:
>>         in Nat),(tt99_8:in .T_1) => (F99_1(tt99_8):
>>         in .T_1)]
>>      ,init98_1,n98_1))))]
>>   {move 0}


\end{verbatim}

We define simple iteration over a single type in terms of the more complex notion of iteration which we take as primitive.  The alternative version {\tt Simpleiter2} will be set up for automatic rewriting in an example below.  Note the use of a bound variable term to refer to the form of {\tt F99} which has an additional dummy natural number argument.

The very similar declarations which support the principle of mathematical induction follow.  These are entirely analogous to the declarations for iteration of a function through a sequence of types above, but working with types of proofs or evidence rather than types of object, and analogues of {\tt Initialize} and
{\tt Iterstep} do not seem to be required as we do not generally consider equations between proofs.

{\tt Fp} is declared in an open/close block so that it can have an implicit argument.

\begin{verbatim}Lestrade execution:


declare nn2 in Nat

>> nn2: in Nat {move 1}



declare Pp [nn2 => prop] \
   



>> Pp: [(nn2_1:in Nat) => (---:prop)]
>>   {move 1}



open

   declare n2 in Nat

>>    n2: in Nat {move 2}



   declare t1 that Pp n2

>>    t1: that Pp(n2) {move 2}



   postulate Fp t1 that Pp (Succ n2)

>>    Fp: [(.n2_1:in Nat),(t1_1:that Pp(.n2_1))
>>         => (---:that Pp(Succ(.n2_1)))]
>>      {move 1}



   close

declare initp that Pp 0

>> initp: that Pp(0) {move 1}



declare np in Nat

>> np: in Nat {move 1}



postulate Iteratep Fp, initp np : that Pp \
   np

>> Iteratep: [(.Pp_1:[(nn2_2:in Nat) => (---:
>>         prop)]),
>>      (Fp_1:[(.n2_3:in Nat),(t1_3:that .Pp_1(.n2_3))
>>         => (---:that .Pp_1(Succ(.n2_3)))]),
>>      (initp_1:that .Pp_1(0)),(np_1:in Nat)
>>      => (---:that .Pp_1(np_1))]
>>   {move 0}


\end{verbatim}

{\bf Technical note:} We discuss the question of the most general form an iteration operator can take in the Lestrade sort system.  If $f$ takes an argument $t$ of type $\tau_1$ to type $\tau(t)$,
there is no latitude for $\tau(t)$ to be anything but $\tau_1$ for iteration to be possible.  Suppose that $f$ actually takes an additional hidden argument, so its actual form
is $f(u,t)$, where $t$ is of type $\tau_1(u)$ and the output is of type $\tau(u,t)$.  For iteration to be possible, it must be the case that $\tau(u,t) = \tau_1(g(u,t))$, where
$g(u,t)$ is of the same constant sort as $u$.   Now we want to define {\tt Iterate f, init, n} so that {\tt Iterate f, init, 0} is {\tt init} and
{\tt Iterate f, init, Succ n} is {\tt f(Iterate f, init, n}).   
The sorts of $f({\tt Iterate(f,init,n)})$ and ${\tt Iterate(f,init,Succ\, n)}$ have to match.  The first must take the form
$\tau(g(u,{\tt Iterate(f,init,n)})$, where the sort of {\tt Iterate(f,init,n))} is $\tau_1(u)$.  This tells us something about the output type
of {\tt Iterate f,init,n}:  its output type must be a fixed function $\tau_1$ of a parameter $u$ extractible from the argument list {\tt f,init,n}:
write this {\tt U(f,init,n)}.  From this it follows that the type of {\tt Iterate f, init, Succ n} is $\tau_1({\tt U(f,init,Succ\, n}))$.  So $\tau_1({\tt U(f,init,Succ\, n})$ must match $\tau(g({\tt U(f,init,n)},{\tt Iterate(f,init,n)})$.  It can readily be seen that there is no actual dependence on {\tt Iterate(f,init,n)} in the second term, since there is none in the first term.
It appears in fact that the only way to achieve this compatible with the type matching facilities we have so far, which are entirely based on literal matching of terms supplemented with definitional expansion, is $\tau=\tau_1$, $g(f,{\tt init},n) = {\tt Succ\, n}$, whence $u=n$, which yields the form of {\tt Iterate} given above.

Under the rewriting facilities of Lestrade not yet described, it may be possible to implement a more general form of iteration; we will revisit this later.
In fact it seems pretty clear to us that the rewrite facility would handle iteration of a function $f(u,t)$ with $t$ of type $\tau(u)$ and output type $\tau(g(u))$, where
the type of $f^n(u,t)$ would be $\tau(g^n(u))$, for general $\tau$ and $g$:  rewriting would allow the matching of types $\tau(g({\tt Simpleiter( g,init,n}))$
and $\tau({\tt Simpleiter(g,init,Succ\, n)})$, and considerations above indicate that this is the most general form of iteration we can expect to support.  It appears that this would not require any additional primitives:  the more general iteration would be definable in terms of the primitives already given.  Additional primitives would be needed, precisely analogous to the ones we have, if we wanted to iterate functions applied to the sorts {\tt prop}, {\tt type}, or {\tt obj} (the last being a case we would be very likely to want).

The code implementing the abstract iteration just discussed now appears here, but without comment!  Real comments would require an introduction to the rewriting feature of Lestrade.  It would also be of interest (if an application of this form of iteration comes into view) to see if this form is actually usable (does the rewriting feature actually support sort inference where things are more concrete?).

\begin{verbatim}Lestrade execution:


declare U30 type

>> U30: type {move 1}



declare u30 in U30

>> u30: in U30 {move 1}



declare n30 in Nat

>> n30: in Nat {move 1}



open

   declare u31 in U30

>>    u31: in U30 {move 2}



   postulate g30 u31 in U30

>>    g30: [(u31_1:in U30) => (---:in U30)]
>>      {move 1}



   postulate tau30 u31 type

>>    tau30: [(u31_1:in U30) => (---:type)]
>>      {move 1}



   declare t31 in tau30 u31

>>    t31: in tau30(u31) {move 2}



   postulate f30 u31 t31 in tau30 g30 u31


>>    f30: [(u31_1:in U30),(t31_1:in tau30(u31_1))
>>         => (---:in tau30(g30(u31_1)))]
>>      {move 1}



   close

declare t30 in tau30 u30

>> t30: in tau30(u30) {move 1}



rewritep Iterwrite u30, n30, g30, Simpleiter2 \
   g30, u30, Succ n30, g30(Simpleiter2 g30, \
   u30, n30)

>> Iterwrite'': [(Iterwrite'''_1:in U30) =>
>>      (---:prop)]
>>   {move 1}



>> Iterwrite': that Iterwrite''(Simpleiter2(g30,
>>   u30,Succ(n30))) {move 1}



>> Iterwrite: [(.U30_1:type),(u30_1:in .U30_1),
>>      (n30_1:in Nat),(g30_1:[(u31_2:in .U30_1)
>>         => (---:in .U30_1)]),
>>      (Iterwrite''_1:[(Iterwrite'''_3:in .U30_1)
>>         => (---:prop)]),
>>      (Iterwrite'_1:that Iterwrite''_1(Simpleiter2(g30_1,
>>      u30_1,Succ(n30_1)))) => (---:that Iterwrite''_1(g30_1(Simpleiter2(g30_1,
>>      u30_1,n30_1))))]
>>   {move 0}



rewritep Iterwrite2 u30, g30, Simpleiter2 \
   g30, u30, 0, u30

>> Iterwrite2'': [(Iterwrite2'''_1:in U30) =>
>>      (---:prop)]
>>   {move 1}



>> Iterwrite2': that Iterwrite2''(Simpleiter2(g30,
>>   u30,0)) {move 1}



>> Iterwrite2: [(.U30_1:type),(u30_1:in .U30_1),
>>      (g30_1:[(u31_2:in .U30_1) => (---:in .U30_1)]),
>>      (Iterwrite2''_1:[(Iterwrite2'''_3:in .U30_1)
>>         => (---:prop)]),
>>      (Iterwrite2'_1:that Iterwrite2''_1(Simpleiter2(g30_1,
>>      u30_1,0))) => (---:that Iterwrite2''_1(u30_1))]
>>   {move 0}



declare init32 in tau30 u30

>> init32: in tau30(u30) {move 1}



declare n32 in Nat

>> n32: in Nat {move 1}



open

   declare n33 in Nat

>>    n33: in Nat {move 2}



   define Tt30 n33 : (Simpleiter2 g30, u30, \
      n33)

>>    Tt30: [(n33_1:in Nat) => (---:in U30)]
>>      {move 1}



   define Tt31 n33: tau30(Tt30 n33)

>>    Tt31: [(n33_1:in Nat) => (---:type)]
>>      {move 1}



   declare n35 in Nat

>>    n35: in Nat {move 2}



   declare t35 in tau30(Tt30 n35)

>>    t35: in tau30(Tt30(n35)) {move 2}



   define f35 t35 : f30 (Tt30 n35,t35)

>>    f35: [(.n35_1:in Nat),(t35_1:in tau30(Tt30(.n35_1)))
>>         => (---:in tau30(g30(Tt30(.n35_1))))]
>>      {move 1}



   close

define Abstractiter f30, init32, n32: Iterate \
   f35,init32,n32

>> Abstractiter: [(.U30_1:type),(.tau30_1:[(u31_2:
>>         in .U30_1) => (---:type)]),
>>      (.g30_1:[(u31_3:in .U30_1) => (---:in
>>         .U30_1)]),
>>      (f30_1:[(u31_4:in .U30_1),(t31_4:in .tau30_1(u31_4))
>>         => (---:in .tau30_1(.g30_1(u31_4)))]),
>>      (.u30_1:in .U30_1),(init32_1:in .tau30_1(.u30_1)),
>>      (n32_1:in Nat) => (Iterate([(.n35_6:in
>>         Nat),(t35_6:in .tau30_1(Simpleiter2(.g30_1,
>>         .u30_1,.n35_6))) => ((Simpleiter2(.g30_1,
>>         .u30_1,.n35_6) f30_1 t35_6):in .tau30_1(.g30_1(Simpleiter2(.g30_1,
>>         .u30_1,.n35_6))))]
>>      ,init32_1,n32_1):in .tau30_1(Simpleiter2(.g30_1,
>>      .u30_1,n32_1)))]
>>   {move 0}


\end{verbatim}

\section{The universal quantifier.  Principle of mathematical induction.}

In this section we introduce the notion of universal quantification (over types of mathematical object)  and develop the familiar form of the principle of mathematical induction.

\begin{verbatim}Lestrade execution:


postulate Forall tpred : prop

>> Forall: [(.T_1:type),(tpred_1:[(tt1_2:in
>>         .T_1) => (---:prop)])
>>      => (---:prop)]
>>   {move 0}



declare univev that Forall tpred

>> univev: that Forall(tpred) {move 1}



declare ttt in T

>> ttt: in T {move 1}



postulate Uinst univev ttt : that tpred ttt


>> Uinst: [(.T_1:type),(.tpred_1:[(tt1_2:in
>>         .T_1) => (---:prop)]),
>>      (univev_1:that Forall(.tpred_1)),(ttt_1:
>>      in .T_1) => (---:that .tpred_1(ttt_1))]
>>   {move 0}



declare ugen [ttt => that tpred ttt] \
   



>> ugen: [(ttt_1:in T) => (---:that tpred(ttt_1))]
>>   {move 1}



postulate Ugen ugen : that Forall tpred

>> Ugen: [(.T_1:type),(.tpred_1:[(tt1_2:in .T_1)
>>         => (---:prop)]),
>>      (ugen_1:[(ttt_3:in .T_1) => (---:that
>>         .tpred_1(ttt_3))])
>>      => (---:that Forall(.tpred_1))]
>>   {move 0}


\end{verbatim}

Here is the development of the universal quantifier (over a type) and its basic rules.  The usual notation for {\tt Forall(tpred)} in mathematical text is $(\forall x\in T:{\tt tpred}(x))$, where ${\tt tpred}(x)$ may be expanded out.  This is read ``for all $x$ in $T$, ${\tt tpred}(x)$".  We should note that we are being bad here, conflating $x$ being of type $T$ with $x$ belonging to a set $T$.  Our excuse for this is that mathematical reasoning is usually done in an officially untyped language, where actual types of mathematical object are usually referred to via sets.

In contrast with Automath and other dependent type provers, evidence for a universal statement is not identified with a suitable dependently typed function, but is obtained by applying a suitable constructor to such a function to get an object of the appropriate object type.  This means that Lestrade, unlike Automath, does not automatically support quantification over all sorts.  This weakness of the framework will turn out to be useful in the formulation of an ambiguous version of the simple theory of types below.

\begin{verbatim}Lestrade execution:


declare natpred [nn2 => prop] \
   



>> natpred: [(nn2_1:in Nat) => (---:prop)]
>>   {move 1}



declare ind that Forall [nn2 => (natpred \
      nn2) -> natpred (Succ nn2)] \
   



>> ind: that Forall([(nn2_1:in Nat) => ((natpred(nn2_1)
>>      -> natpred(Succ(nn2_1))):prop)])
>>   {move 1}



declare basis that natpred 0

>> basis: that natpred(0) {move 1}


\end{verbatim}

Here are familiar prerequisites for mathematical induction, the basis step, evidence for ${\tt natpred}(0)$, and the induction step, evidence for $$(\forall n \in {\tt Nat}:{\tt natpred}(n) \rightarrow {\tt natpred}(n+1) ).$$

\begin{verbatim}Lestrade execution:


open

   declare n2 in Nat

>>    n2: in Nat {move 2}



   declare indhyp that natpred n2

>>    indhyp: that natpred(n2) {move 2}



   define step1 n2 : Uinst ind n2

>>    step1: [(n2_1:in Nat) => (---:that (natpred(n2_1)
>>         -> natpred(Succ(n2_1))))]
>>      {move 1}



   define step2 n2 indhyp : Mp (indhyp,step1 \
      n2)

>>    step2: [(n2_1:in Nat),(indhyp_1:that natpred(n2_1))
>>         => (---:that natpred(Succ(n2_1)))]
>>      {move 1}



   close

declare nq in Nat

>> nq: in Nat {move 1}



define Induction1 ind basis nq : Iteratep \
   step2, basis, nq

>> Induction1: [(.natpred_1:[(nn2_2:in Nat)
>>         => (---:prop)]),
>>      (ind_1:that Forall([(nn2_3:in Nat) =>
>>         ((.natpred_1(nn2_3) -> .natpred_1(Succ(nn2_3))):
>>         prop)]))
>>      ,(basis_1:that .natpred_1(0)),(nq_1:in
>>      Nat) => (Iteratep([(n2_4:in Nat),(indhyp_4:
>>         that .natpred_1(n2_4)) => ((indhyp_4
>>         Mp (ind_1 Uinst n2_4)):that .natpred_1(Succ(n2_4)))]
>>      ,basis_1,nq_1):that .natpred_1(nq_1))]
>>   {move 0}

define Inductiona1 natpred, ind basis nq: Iteratep \
   step2, basis, nq

declare nq2 in Nat

define Inductiona natpred, ind basis: Ugen([nq2=> Inductiona1  natpred,ind, basis,nq2])

define Induction ind basis : Ugen(Induction1 \
   (ind, basis))

>> Induction: [(.natpred_1:[(nn2_2:in Nat) =>
>>         (---:prop)]),
>>      (ind_1:that Forall([(nn2_3:in Nat) =>
>>         ((.natpred_1(nn2_3) -> .natpred_1(Succ(nn2_3))):
>>         prop)]))
>>      ,(basis_1:that .natpred_1(0)) => (Ugen([(nq_4:
>>         in Nat) => (Induction1(ind_1,basis_1,
>>         nq_4):that .natpred_1(nq_4))])
>>      :that Forall(.natpred_1))]
>>   {move 0}


\end{verbatim}

Here is the proof of a standard form of mathematical induction.   {\tt Induction1} generates instances of theorems proved by induction:  {\tt Induction} generates universally quantified theorems derived by induction.  The meat of the proof lies in showing that the existence of a proof of {\tt Forall(indimp)} yields a function taking proofs of {\tt natpred(n)} to proofs of {\tt natpred(Succ(n))}, which is what is required as input to {\tt Iteratep}.  The declaration of {\tt Induction} is a nice example of the use as an argument of a function defined by giving another function a truncated argument list.

We think that it is interesting to contemplate the mathematical object presented as the referent of {\tt Induction1} in the Lestrade reply to its declaration.

It may seem odd that the induction step is the first argument rather than the basis step:  the reason for this is that Lestrade can reliably read the hidden argument {\tt natpred} from the induction step, but not so reliably from the basis step.

\section{Definitions and basic axioms for addition and multiplication}

In this section we define the notions of addition and multiplication and prove the usual Peano ``axioms" governing these operations.  No new axioms are actually required:  addition and multiplication are defined by iterating suitable functions,
and here natural numbers are entirely defined in terms of iteration of abstract functions.

\begin{verbatim}Lestrade execution:


declare N1 in Nat

>> N1: in Nat {move 1}



declare N2 in Nat

>> N2: in Nat {move 1}



define + N1 N2 : Simpleiter Succ, N1 N2

>> +: [(N1_1:in Nat),(N2_1:in Nat) => (Simpleiter(Succ,
>>      N1_1,N2_1):in Nat)]
>>   {move 0}


\end{verbatim}

The sum {\tt N1 + N2} is defined as the result of iterating successor {\tt N2} times starting at {\tt N1}.  The function {\tt Succ1} is needed because the function iterated in the fully abstract case
has an additional natural number argument which can qualify types.  Note that Lestrade does not need to be told that the function {\tt Tt} from natural numbers to types which is a hidden parameter of {\tt Iterate} is here the constant function whose
value is {\tt Nat}:  its type inference is smart enough to figure this out.

\begin{verbatim}Lestrade execution:


define Addid N1: propfixform ((N1+0)=N1,Simpleinit \
   Succ, N1)

>> Addid: [(N1_1:in Nat) => ((((N1_1 + 0) =
>>      N1_1) propfixform Simpleinit(Succ,N1_1)):
>>      that ((N1_1 + 0) = N1_1))]
>>   {move 0}



define Additer N1 N2 : propfixform ((N1 + \
   Succ N2)=Succ(N1 + N2), Simpleiterstep Succ, \
   N1 N2)

>> Additer: [(N1_1:in Nat),(N2_1:in Nat) =>
>>      ((((N1_1 + Succ(N2_1)) = Succ((N1_1 +
>>      N2_1))) propfixform Simpleiterstep(Succ,
>>      N1_1,N2_1)):that ((N1_1 + Succ(N2_1))
>>      = Succ((N1_1 + N2_1))))]
>>   {move 0}


\end{verbatim}

Here the usual Peano axioms for addition are proved as instances of the basic equations governing simple iteration.

\begin{verbatim}Lestrade execution:


open

   declare n2 in Nat

>>    n2: in Nat {move 2}



   declare n3 in Nat

>>    n3: in Nat {move 2}



   define addenone n3: n3+N1

>>    addenone: [(n3_1:in Nat) => (---:in Nat)]
>>      {move 1}



   close

define * N1 N2 : Simpleiter addenone, 0, \
   N2

>> *: [(N1_1:in Nat),(N2_1:in Nat) => (Simpleiter([(n3_2:
>>         in Nat) => ((n3_2 + N1_1):in Nat)]
>>      ,0,N2_1):in Nat)]
>>   {move 0}



define Multzero N1 : propfixform ((N1*0)=0, \
   Simpleinit addenone, 0)

>> Multzero: [(N1_1:in Nat) => ((((N1_1 * 0)
>>      = 0) propfixform Simpleinit([(n3_2:in
>>         Nat) => ((n3_2 + N1_1):in Nat)]
>>      ,0)):that ((N1_1 * 0) = 0))]
>>   {move 0}



define Multiter N1 N2 : propfixform \
   ((N1*Succ N2)=(N1*N2)+N1, Simpleiterstep \
   addenone,0,N2)

>> Multiter: [(N1_1:in Nat),(N2_1:in Nat) =>
>>      ((((N1_1 * Succ(N2_1)) = ((N1_1 * N2_1)
>>      + N1_1)) propfixform Simpleiterstep([(n3_2:
>>         in Nat) => ((n3_2 + N1_1):in Nat)]
>>      ,0,N2_1)):that ((N1_1 * Succ(N2_1)) =
>>      ((N1_1 * N2_1) + N1_1)))]
>>   {move 0}


\end{verbatim}

The development of multiplication is very similar to that of addition, subject to the additional complication that the operation ``add {\tt N1}" which is iterated has to be given
a nonce name {\tt addenone} (when this was first written:  we could replace it with {\tt [nn2 => nn2 + N1]}, but we see no compelling reason to do so).

\section{Addition is commutative}

In this section, we prove from the axioms for addition given in the previous section that addition is commutative, narrating our motivations
as we go.



\begin{verbatim}Lestrade execution:


open

   declare M3 in Nat

>>    M3: in Nat {move 2}



   declare N3 in Nat

>>    N3: in Nat {move 2}



   define commuteswithall M3: Forall [N3 \
         => (M3 + N3) = N3 + M3] \
      



>>    commuteswithall: [(M3_1:in Nat) => (---:
>>         prop)]
>>      {move 1}



   close
\end{verbatim}

Open a working environment, in which we declare a natural number {\tt M3}, and introduce the property of commuting with {\tt M3}, and then the property of {\tt M3} of commuting with every natural number.

We first show {\tt commuteswithall 0} by induction.

\begin{verbatim}Lestrade execution:

comment The basis step


define zerocommuteswithzero : Reflexeq (0+0)


>> zerocommuteswithzero: [(Reflexeq((0 + 0)):
>>      that ((0 + 0) = (0 + 0)))]
>>   {move 0}



open

   declare M3 in Nat

>>    M3: in Nat {move 2}



   open

      declare indhyp that (0 + M3) = M3 + \
         0

>>       indhyp: that ((0 + M3) = (M3 + 0))
>>         {move 3}



      define commzero1 : Additer 0 M3

>>       commzero1: [(---:that ((0 + Succ(M3))
>>            = Succ((0 + M3))))]
>>         {move 2}



      define commzero2 indhyp : Substitution \
         indhyp commzero1

>>       commzero2: [(indhyp_1:that ((0 + M3)
>>            = (M3 + 0))) => (---:that ((0 +
>>            Succ(M3)) = Succ((M3 + 0))))]
>>         {move 2}



      define commzero3 : Addid M3

>>       commzero3: [(---:that ((M3 + 0) = M3))]
>>         {move 2}



      define commzero4 indhyp : Substitution \
         commzero3 commzero2 indhyp

>>       commzero4: [(indhyp_1:that ((0 + M3)
>>            = (M3 + 0))) => (---:that ((0 +
>>            Succ(M3)) = Succ(M3)))]
>>         {move 2}



      declare M4 in Nat

>>       M4: in Nat {move 3}



      define commzero5 indhyp: Substitution0 \
         ([M4=>(0+Succ M3)=M4] \
         ,Eqsymm Addid Succ M3, \
         commzero4 indhyp)

>>       commzero5: [(indhyp_1:that ((0 + M3)
>>            = (M3 + 0))) => (---:that ((0 +
>>            Succ(M3)) = (Succ(M3) + 0)))]
>>         {move 2}



      close

   define indstep1 M3 : Deduction commzero5


>>    indstep1: [(M3_1:in Nat) => (---:that
>>         (((0 + M3_1) = (M3_1 + 0)) -> ((0 +
>>         Succ(M3_1)) = (Succ(M3_1) + 0))))]
>>      {move 1}



   close

define commzerobasisindstep : Ugen indstep1


>> commzerobasisindstep: [(Ugen([(M3_1:in Nat)
>>         => (Deduction([(indhyp_2:that ((0 +
>>            M3_1) = (M3_1 + 0))) => (Substitution0([(M4_3:
>>               in Nat) => (((0 + Succ(M3_1))
>>               = M4_3):prop)]
>>            ,Eqsymm(Addid(Succ(M3_1))),(Addid(M3_1)
>>            Substitution (indhyp_2 Substitution
>>            (0 Additer M3_1)))):that ((0 + Succ(M3_1))
>>            = (Succ(M3_1) + 0)))])
>>         :that (((0 + M3_1) = (M3_1 + 0)) ->
>>         ((0 + Succ(M3_1)) = (Succ(M3_1) + 0))))])
>>      :that Forall([(M3_6:in Nat) => ((((0 +
>>         M3_6) = (M3_6 + 0)) -> ((0 + Succ(M3_6))
>>         = (Succ(M3_6) + 0))):prop)]))
>>      ]
>>   {move 0}



define commzerobasis : Induction commzerobasisindstep \
   zerocommuteswithzero

>> commzerobasis: [((commzerobasisindstep Induction
>>      zerocommuteswithzero):that Forall([(M3_2:
>>         in Nat) => (((0 + M3_2) = (M3_2 + 0)):
>>         prop)]))
>>      ]
>>   {move 0}


\end{verbatim}

We have now proved the basis step (commutativity of addition with zero).

We commence the induction step.

\begin{verbatim}Lestrade execution:


declare M3 in Nat

>> M3: in Nat {move 1}



open

   declare commindhyp that commuteswithall \
      M3

>>    commindhyp: that commuteswithall(M3) {move
>>      2}



   open

      declare N3 in Nat

>>       N3: in Nat {move 3}



      define commind1 N3 : Reflexeq (Succ \
         M3 + N3)

>>       commind1: [(N3_1:in Nat) => (---:that
>>            ((Succ(M3) + N3_1) = (Succ(M3) +
>>            N3_1)))]
>>         {move 2}


\end{verbatim}

At this point we pause and remark that we immediately need the lemma $\sigma(m)+m = \sigma(m+n)$.  We prove the lemma inline right here.

\begin{verbatim}Lestrade execution:


      define commindlemma1 : Addid Succ M3


>>       commindlemma1: [(---:that ((Succ(M3)
>>            + 0) = Succ(M3)))]
>>         {move 2}



      declare N4 in Nat

>>       N4: in Nat {move 3}



      define commindlemma2: Substitution0([N4 \
            => (Succ M3 +0)= Succ \
            N4] \
         , Eqsymm (Addid M3),commindlemma1)


>>       commindlemma2: [(---:that ((Succ(M3)
>>            + 0) = Succ((M3 + 0))))]
>>         {move 2}


\end{verbatim}

The object {\tt commindlemma2} is evidence for the basis of the lemma.

\begin{verbatim}Lestrade execution:


      open

         declare commindlemmaindhyp \
            that (Succ M3 + N3) = Succ(M3 + \
            N3)

>>          commindlemmaindhyp: that ((Succ(M3)
>>            + N3) = Succ((M3 + N3))) {move 4}



         define commindlemma3 : Additer Succ \
            M3 N3

>>          commindlemma3: [(---:that ((Succ(M3)
>>               + Succ(N3)) = Succ((Succ(M3)
>>               + N3))))]
>>            {move 3}



         define commindlemma4 commindlemmaindhyp \
            : Substitution commindlemmaindhyp \
            commindlemma3

>>          commindlemma4: [(commindlemmaindhyp_1:
>>               that ((Succ(M3) + N3) = Succ((M3
>>               + N3)))) => (---:that ((Succ(M3)
>>               + Succ(N3)) = Succ(Succ((M3 +
>>               N3)))))]
>>            {move 3}



         define commindlemma5 commindlemmaindhyp \
            : Substitution (Eqsymm(Additer \
            M3 N3), commindlemma4 \
            commindlemmaindhyp)

>>          commindlemma5: [(commindlemmaindhyp_1:
>>               that ((Succ(M3) + N3) = Succ((M3
>>               + N3)))) => (---:that ((Succ(M3)
>>               + Succ(N3)) = Succ((M3 + Succ(N3)))))]
>>            {move 3}



         close

      define commindlemma6 N3 : Deduction \
         (commindlemma5)

>>       commindlemma6: [(N3_1:in Nat) => (---:
>>            that (((Succ(M3) + N3_1) = Succ((M3
>>            + N3_1))) -> ((Succ(M3) + Succ(N3_1))
>>            = Succ((M3 + Succ(N3_1))))))]
>>         {move 2}



      define commindlemma7 : Ugen commindlemma6


>>       commindlemma7: [(---:that Forall([(N3_2:
>>               in Nat) => ((((Succ(M3) + N3_2)
>>               = Succ((M3 + N3_2))) -> ((Succ(M3)
>>               + Succ(N3_2)) = Succ((M3 + Succ(N3_2))))):
>>               prop)]))
>>            ]
>>         {move 2}



      define commindlemma : Induction commindlemma7, \
         commindlemma2

>>       commindlemma: [(---:that Forall([(N3_2:
>>               in Nat) => (((Succ(M3) + N3_2)
>>               = Succ((M3 + N3_2))):prop)]))
>>            ]
>>         {move 2}



      declare M4 in Nat

>>       M4: in Nat {move 3}



      define commind2 N3 : Substitution0([ \
            M4 => (Succ M3 + N3)=M4] \
         , Uinst commindlemma N3, \
         commind1 N3)

>>       commind2: [(N3_1:in Nat) => (---:that
>>            ((Succ(M3) + N3_1) = Succ((M3 +
>>            N3_1))))]
>>         {move 2}



      define commind3 N3 : Substitution(Uinst \
         commindhyp N3,commind2 N3)

>>       commind3: [(N3_1:in Nat) => (---:that
>>            ((Succ(M3) + N3_1) = Succ((N3_1
>>            + M3))))]
>>         {move 2}



      define commind4 N3 : Substitution(Eqsymm(Additer \
         N3 M3),commind3 N3)

>>       commind4: [(N3_1:in Nat) => (---:that
>>            ((Succ(M3) + N3_1) = (N3_1 + Succ(M3))))]
>>         {move 2}



      close

   define commind5 commindhyp : propfixform(commuteswithall \
      (Succ M3), Ugen commind4)

>>    commind5: [(commindhyp_1:that commuteswithall(M3))
>>         => (---:that commuteswithall(Succ(M3)))]
>>      {move 1}



   close

define commind6 M3:Deduction commind5

>> commind6: [(M3_1:in Nat) => (Deduction([(commindhyp_4:
>>         that Forall([(N3_5:in Nat) => (((M3_1
>>            + N3_5) = (N3_5 + M3_1)):prop)]))
>>         => ((Forall([(N3_6:in Nat) => (((Succ(M3_1)
>>            + N3_6) = (N3_6 + Succ(M3_1))):prop)])
>>         propfixform Ugen([(N3_8:in Nat) =>
>>            ((Eqsymm((N3_8 Additer M3_1)) Substitution
>>            ((commindhyp_4 Uinst N3_8) Substitution
>>            Substitution0([(M4_12:in Nat) =>
>>               (((Succ(M3_1) + N3_8) = M4_12):
>>               prop)]
>>            ,((Ugen([(N3_16:in Nat) => (Deduction([(commindlemmaindhyp_17:
>>                  that ((Succ(M3_1) + N3_16)
>>                  = Succ((M3_1 + N3_16)))) =>
>>                  ((Eqsymm((M3_1 Additer N3_16))
>>                  Substitution (commindlemmaindhyp_17
>>                  Substitution (Succ(M3_1) Additer
>>                  N3_16))):that ((Succ(M3_1)
>>                  + Succ(N3_16)) = Succ((M3_1
>>                  + Succ(N3_16)))))])
>>               :that (((Succ(M3_1) + N3_16)
>>               = Succ((M3_1 + N3_16))) -> ((Succ(M3_1)
>>               + Succ(N3_16)) = Succ((M3_1 +
>>               Succ(N3_16))))))])
>>            Induction Substitution0([(N4_20:
>>               in Nat) => (((Succ(M3_1) + 0)
>>               = Succ(N4_20)):prop)]
>>            ,Eqsymm(Addid(M3_1)),Addid(Succ(M3_1))))
>>            Uinst N3_8),Reflexeq((Succ(M3_1)
>>            + N3_8))))):that ((Succ(M3_1) +
>>            N3_8) = (N3_8 + Succ(M3_1))))]))
>>         :that Forall([(N3_21:in Nat) => (((Succ(M3_1)
>>            + N3_21) = (N3_21 + Succ(M3_1))):
>>            prop)]))
>>         ])
>>      :that (Forall([(N3_22:in Nat) => (((M3_1
>>         + N3_22) = (N3_22 + M3_1)):prop)])
>>      -> Forall([(N3_23:in Nat) => (((Succ(M3_1)
>>         + N3_23) = (N3_23 + Succ(M3_1))):prop)]))
>>      )]
>>   {move 0}



define commind7 : Ugen commind6

>> commind7: [(Ugen(commind6):that Forall([(M3_4:
>>         in Nat) => ((Forall([(N3_5:in Nat)
>>            => (((M3_4 + N3_5) = (N3_5 + M3_4)):
>>            prop)])
>>         -> Forall([(N3_6:in Nat) => (((Succ(M3_4)
>>            + N3_6) = (N3_6 + Succ(M3_4))):prop)]))
>>         :prop)]))
>>      ]
>>   {move 0}



define Addcomm : Induction commind7 commzerobasis


>> Addcomm: [((commind7 Induction commzerobasis):
>>      that Forall([(M3_3:in Nat) => (Forall([(N3_4:
>>            in Nat) => (((M3_3 + N3_4) = (N3_4
>>            + M3_3)):prop)])
>>         :prop)]))
>>      ]
>>   {move 0}



declare term1 in Nat

>> term1: in Nat {move 1}



declare term2 in Nat

>> term2: in Nat {move 1}



define Addcomm2 term1 term2 : Uinst(Uinst \
   Addcomm term1,term2)

>> Addcomm2: [(term1_1:in Nat),(term2_1:in Nat)
>>      => (((Addcomm Uinst term1_1) Uinst term2_1):
>>      that ((term1_1 + term2_1) = (term2_1 +
>>      term1_1)))]
>>   {move 0}


\end{verbatim}

At this point the commutativity of addition is proved.  The method of proof is entirely standard.  Moreover, it is not nearly as verbose as the length of the text above would seem to suggest:  the correct measure is the length of the text consisting only of user-entered lines.  These lines are closely analogous to the lines in a usual proof of this result from the axioms of Peano arithmetic, complicated by a fine-grained approach to application of rules and careful notation of dependencies and levels of hypothesis.

We shall probably clean up this proof, with attention to better use of namespace and better mnemonics for proof line objects.

\section{Power set types introduced}

\begin{verbatim}Lestrade execution:


postulate setsof T: type

>> setsof: [(T_1:type) => (---:type)]
>>   {move 0}



postulate setof tpred: in setsof T

>> setof: [(.T_1:type),(tpred_1:[(tt1_2:in .T_1)
>>         => (---:prop)])
>>      => (---:in setsof(.T_1))]
>>   {move 0}


\end{verbatim}

A more usual notation for {\tt setsof T} might be ${\cal P}(T)$, the ``power set type" of $T$.  The terminology here relates to the conceptual abuse confusing a type $T$ with the set of its elements.  The more usual mathematical notation for {\tt setof tpred} would be $\{x \in T:{\tt tpred}(x)\}$, subject to the same remark about abuse of terminology for types and sets.

\begin{verbatim}Lestrade execution:


declare t6 in T

>> t6: in T {move 1}



declare s6 in setsof T

>> s6: in setsof(T) {move 1}



postulate E t6 s6 : prop

>> E: [(.T_1:type),(t6_1:in .T_1),(s6_1:in setsof(.T_1))
>>      => (---:prop)]
>>   {move 0}


\end{verbatim}

We declare the membership relation.

\begin{verbatim}Lestrade execution:


declare elementev1 that tpred t6

>> elementev1: that tpred(t6) {move 1}



declare elementev2 that t6 E setof tpred


>> elementev2: that (t6 E setof(tpred)) {move
>>   1}



postulate Comprehension10 tpred, t6 elementev1 \
   that t6 E setof tpred

>> Comprehension10: [(.T_1:type),(tpred_1:[(tt1_2:
>>         in .T_1) => (---:prop)]),
>>      (t6_1:in .T_1),(elementev1_1:that tpred_1(t6_1))
>>      => (---:that (t6_1 E setof(tpred_1)))]
>>   {move 0}



define Comprehension11 tpred, elementev1 \
   : Comprehension10 tpred, t6 elementev1

>> Comprehension11: [(.T_1:type),(tpred_1:[(tt1_2:
>>         in .T_1) => (---:prop)]),
>>      (.t6_1:in .T_1),(elementev1_1:that tpred_1(.t6_1))
>>      => (Comprehension10(tpred_1,.t6_1,elementev1_1):
>>      that (.t6_1 E setof(tpred_1)))]
>>   {move 0}



define Comprehension12 t6 elementev1 : Comprehension10 \
   tpred, t6 elementev1

>> Comprehension12: [(.T_1:type),(t6_1:in .T_1),
>>      (.tpred_1:[(tt1_2:in .T_1) => (---:prop)]),
>>      (elementev1_1:that .tpred_1(t6_1)) =>
>>      (Comprehension10(.tpred_1,t6_1,elementev1_1):
>>      that (t6_1 E setof(.tpred_1)))]
>>   {move 0}



postulate Comprehension2 elementev2 that \
   tpred t6

>> Comprehension2: [(.T_1:type),(.t6_1:in .T_1),
>>      (.tpred_1:[(tt1_2:in .T_1) => (---:prop)]),
>>      (elementev2_1:that (.t6_1 E setof(.tpred_1)))
>>      => (---:that .tpred_1(.t6_1))]
>>   {move 0}


\end{verbatim}

We implement the comprehension axiom, the equivalence of $$a \in \{x\in T:{\tt tpred}(x)\}$$ and ${\tt tpred}(a)$, via the declaration of the functions
{\tt Comprehension1x} (where {\tt x} is 0,1,2) and {\tt Comprehension2}.

\begin{verbatim}Lestrade execution:


open

   declare t5 in T

>>    t5: in T {move 2}



   postulate tpred1 t5 prop

>>    tpred1: [(t5_1:in T) => (---:prop)]
>>      {move 1}



   postulate tpred2 t5 prop

>>    tpred2: [(t5_1:in T) => (---:prop)]
>>      {move 1}



   declare tpredev1 that tpred1 t5

>>    tpredev1: that tpred1(t5) {move 2}



   declare tpredev2 that tpred1 t5

>>    tpredev2: that tpred1(t5) {move 2}



   postulate ext1 tpredev1 : that tpred2 \
      t5

>>    ext1: [(.t5_1:in T),(tpredev1_1:that tpred1(.t5_1))
>>         => (---:that tpred2(.t5_1))]
>>      {move 1}



   postulate ext2 tpredev2 : that tpred1 \
      t5

>>    ext2: [(.t5_1:in T),(tpredev2_1:that tpred1(.t5_1))
>>         => (---:that tpred1(.t5_1))]
>>      {move 1}



   close

postulate Extensionality ext1, ext2 : \
   that (setof tpred1) = setof tpred2

>> Extensionality: [(.T_1:type),(.tpred1_1:[(t5_2:
>>         in .T_1) => (---:prop)]),
>>      (.tpred2_1:[(t5_3:in .T_1) => (---:prop)]),
>>      (ext1_1:[(.t5_4:in .T_1),(tpredev1_4:that
>>         .tpred1_1(.t5_4)) => (---:that .tpred2_1(.t5_4))]),
>>      (ext2_1:[(.t5_5:in .T_1),(tpredev2_5:that
>>         .tpred1_1(.t5_5)) => (---:that .tpred1_1(.t5_5))])
>>      => (---:that (setof(.tpred1_1) = setof(.tpred2_1)))]
>>   {move 0}



declare s7 in setsof T

>> s7: in setsof(T) {move 1}



declare t5 in T

>> t5: in T {move 1}



postulate Extensionality2 s7 that s7 = setof \
   [t5 => t5 E s7] \
   



>> Extensionality2: [(.T_1:type),(s7_1:in setsof(.T_1))
>>      => (---:that (s7_1 = setof([(t5_2:in .T_1)
>>         => ((t5_2 E s7_1):prop)]))
>>      )]
>>   {move 0}


\end{verbatim}

The functions {\tt Extensionality1} and {\tt Extensionality2} implement the axiom of extensionality.  There is something to note about how this is done (and we ought to prove some theorems later to show equivalence of this approach to other possible approaches).  In effect, we postulate
equivalence of $\{x \in T:{\tt tpred}(x)\} = \{x \in T:{\tt tpred}(x)\}$ and $(\forall x:{\tt tpred}(x) \leftrightarrow {\tt tpred}(x))$:  this is what {\tt Extensionality1} does.  To get full extensionality in the usual sense, we also postulate $S = \{x \in T:x \in S\}$ (this is what {\tt Extensionality2} does):  for each $S$ of type ${\cal P}(T)$:  this prevents existence of additional objects
of type ${\cal P}(T)$ with the same extension as sets defined in the usual way using set builder notation from predicates, but not themselves defined using set builder notation.

We have a philosophical reason for taking this approach.  We have general metaphysical reasons for avoiding conflation of functions and objects,
on which we may expand later.  The function {\tt setof} enables implementation of predicates of objects of type $T$ (functions from $T$ to {\tt prop}) as objects of type ${\cal P}(T)$:  {\tt Extensionality1} thus expresses identity criteria for predicates (indirectly).  It can be further noted that it is perfectly possible to define an equality predicate directly on the function sort of predicates of type $T$ objects, and explicitly state extensional identity criteria for such functions, and we may do this later.  But in any event, we regard the assertion of identity criteria for predicates implemented as objects of a power set type as distinguishable from the assertion that all objects of the power set type actually are implementations of predicates.

A theory of sets as untyped mathematical objects (in sort {\tt obj}) could be implemented similarly, and we may present this later.

\section{Naive set theory and Russell's paradox (without even using negation!)}

In this section we develop naive set theory (in which any property of untyped mathematical objects defines a set, and sets are untyped mathematical objects)
and develop something like the paradox of Russell.  The way in which we do this is a little strange since we do not have negation yet, but implication is enough:  the function {\tt Russell} which is our final product takes any proposition $A$ and returns a proof of $A$:  the existence of a such a function would at the very least make mathematics uninteresting.

\begin{verbatim}Lestrade execution:


open

   declare A1 prop

>>    A1: prop {move 2}



   declare ao obj

>>    ao: obj {move 2}



   declare bo obj

>>    bo: obj {move 2}



   open

      declare xo obj

>>       xo: obj {move 3}



      postulate opred xo prop

>>       opred: [(xo_1:obj) => (---:prop)]
>>         {move 2}



      close

   postulate osetof opred obj

>>    osetof: [(opred_1:[(xo_2:obj) => (---:
>>            prop)])
>>         => (---:obj)]
>>      {move 1}


\end{verbatim}

We introduce the set builder operation {\tt osetof} which takes a predicate of untyped objects to an untyped object.

\begin{verbatim}Lestrade execution:


   postulate Eo ao bo prop

>>    Eo: [(ao_1:obj),(bo_1:obj) => (---:prop)]
>>      {move 1}



   declare oelementev1 that ao Eo osetof \
      opred

>>    oelementev1: that (ao Eo osetof(opred))
>>      {move 2}



   declare oelementev2 that opred ao

>>    oelementev2: that opred(ao) {move 2}



   postulate Ocomp1 oelementev1 that opred \
      ao

>>    Ocomp1: [(.ao_1:obj),(.opred_1:[(xo_2:
>>            obj) => (---:prop)]),
>>         (oelementev1_1:that (.ao_1 Eo osetof(.opred_1)))
>>         => (---:that .opred_1(.ao_1))]
>>      {move 1}



   postulate Ocomp2 ao opred, oelementev2 \
      that ao Eo osetof opred

>>    Ocomp2: [(ao_1:obj),(opred_1:[(xo_2:obj)
>>            => (---:prop)]),
>>         (oelementev2_1:that opred_1(ao_1))
>>         => (---:that (ao_1 Eo osetof(opred_1)))]
>>      {move 1}


\end{verbatim}

We introduce the membership relation {\tt Eo} and the two functions implementing its comprehension axiom, which are precisely analogous to the functions implementing the comprehension scheme in typed set theory above.

\begin{verbatim}Lestrade execution:


   open

      declare yo obj

>>       yo: obj {move 3}



      define R yo : (yo Eo yo) -> A1

>>       R: [(yo_1:obj) => (---:prop)]
>>         {move 2}



      close

   define r A1 : osetof R

>>    r: [(A1_1:prop) => (---:obj)]
>>      {move 1}


\end{verbatim}

This is our paradoxical set {\tt r(A1)} , which we would write in ordinary notation as $\{x : x \in x \rightarrow A1\}$.


\begin{verbatim}Lestrade execution:


   open

      declare rhyp that (r A1) Eo r A1

>>       rhyp: that (r(A1) Eo r(A1)) {move 3}



      define rstep1 rhyp: Ocomp1 rhyp

>>       rstep1: [(rhyp_1:that (r(A1) Eo r(A1)))
>>            => (---:that ((r(A1) Eo r(A1)) ->
>>            A1))]
>>         {move 2}



      define rstep2 rhyp: Mp rhyp (rstep1 \
         rhyp)

>>       rstep2: [(rhyp_1:that (r(A1) Eo r(A1)))
>>            => (---:that A1)]
>>         {move 2}



      define rstep3 rhyp: Deduction rstep2


>>       rstep3: [(rhyp_1:that (r(A1) Eo r(A1)))
>>            => (---:that ((r(A1) Eo r(A1)) ->
>>            A1))]
>>         {move 2}



      define rstep4 rhyp: Mp rhyp rstep3 \
         rhyp

>>       rstep4: [(rhyp_1:that (r(A1) Eo r(A1)))
>>            => (---:that A1)]
>>         {move 2}



      close

   define Russell1 A1 : Deduction rstep4


>>    Russell1: [(A1_1:prop) => (---:that ((r(A1_1)
>>         Eo r(A1_1)) -> A1_1))]
>>      {move 1}



   define Ocomp22 ao oelementev2 : Ocomp2 \
      ao opred, oelementev2

>>    Ocomp22: [(ao_1:obj),(.opred_1:[(xo_2:
>>            obj) => (---:prop)]),
>>         (oelementev2_1:that .opred_1(ao_1))
>>         => (---:that (ao_1 Eo osetof(.opred_1)))]
>>      {move 1}



   define Russell2 A1: propfixform ((r A1) \
      Eo (r A1),Ocomp22 ((r A1),(Russell1 A1)))


>>    Russell2: [(A1_1:prop) => (---:that (r(A1_1)
>>         Eo r(A1_1)))]
>>      {move 1}



   define Russell A1: Mp (Russell2 A1, Russell1 \
      A1)

>>    Russell: [(A1_1:prop) => (---:that A1_1)]
>>      {move 1}



   close
\end{verbatim}



The argument here is perfectly mad, of course.  We review it since this is not the form usually given.

Let $R$ denote the set $\{x:x \in x \rightarrow A\}$.

Our goal is to prove $R \in R$.  To prove $R \in R$, that is $R \in \{x \in x \rightarrow A\}$, it suffices to prove $R \in R \rightarrow A$.

Suppose $R \in R$ for the sake of argument.  Our goal is $A$.  $R \in R$ as already noted is equivalent to $R \in R \rightarrow A$.
Modus ponens gives us our goal $A$, so we have established $R \in R \rightarrow A$ by deduction, and so we have established $R \in R$,
as already discussed.

Since we have both $R \in R$ and $R \in R \rightarrow A$, we have $A$ by modus ponens.

But $A$ was any proposition at all.

A Lestrade technicality to note is that it was convenient to introduce a version {\tt Ocomp22} of {\tt Ocomp2}  which did not take an explicit predicate argument.

One should always have something philosophical to say after introducing something reputed to be a paradox, a threat to the foundations of reason.  Our remark is that one should look carefully at the hypotheses before concluding that the foundations of reason are threatened.  The Lestrade framework does nothing to encourage us to think it likely that the function sort of predicates of objects of sort {\tt obj} can be embedded into the sort {\tt obj} itself.  The proof simply shows that this cannot be done (in the presence of implication, at any rate).

The observant reader may notice that we packed the whole preceding argument in an extra Lestrade environment, so that we do not actually have primitives at move 0 which allow us to deduce that any proposition $A$ is true.  What we can prove at move 0 we now unveil (if objects with types of the primitives in the development above exist, contradiction follows).  It is also a frightening example of the effects of definition unpacking!

\begin{verbatim}Lestrade execution:


define Russellthm A,osetof, Eo, Ocomp1, Ocomp2: \
   Russell A

>> Russellthm: [(A_1:prop),(osetof_1:[(opred_2:
>>         [(xo_3:obj) => (---:prop)])
>>         => (---:obj)]),
>>      (Eo_1:[(ao_4:obj),(bo_4:obj) => (---:prop)]),
>>      (Ocomp1_1:[(.ao_5:obj),(.opred_5:[(xo_6:
>>            obj) => (---:prop)]),
>>         (oelementev1_5:that (.ao_5 Eo_1 osetof_1(.opred_5)))
>>         => (---:that .opred_5(.ao_5))]),
>>      (Ocomp2_1:[(ao_7:obj),(opred_7:[(xo_8:
>>            obj) => (---:prop)]),
>>         (oelementev2_7:that opred_7(ao_7))
>>         => (---:that (ao_7 Eo_1 osetof_1(opred_7)))])
>>      => ((((osetof_1([(yo_11:obj) => (((yo_11
>>         Eo_1 yo_11) -> A_1):prop)])
>>      Eo_1 osetof_1([(yo_12:obj) => (((yo_12
>>         Eo_1 yo_12) -> A_1):prop)]))
>>      propfixform Ocomp2_1(osetof_1([(yo_13:
>>         obj) => (((yo_13 Eo_1 yo_13) -> A_1):
>>         prop)]),
>>      [(xo_14:obj) => (((xo_14 Eo_1 xo_14) ->
>>         A_1):prop)]
>>      ,Deduction([(rhyp_17:that (osetof_1([(yo_18:
>>            obj) => (((yo_18 Eo_1 yo_18) ->
>>            A_1):prop)])
>>         Eo_1 osetof_1([(yo_19:obj) => (((yo_19
>>            Eo_1 yo_19) -> A_1):prop)]))
>>         ) => ((rhyp_17 Mp Deduction([(rhyp_24:
>>            that (osetof_1([(yo_25:obj) => (((yo_25
>>               Eo_1 yo_25) -> A_1):prop)])
>>            Eo_1 osetof_1([(yo_26:obj) => (((yo_26
>>               Eo_1 yo_26) -> A_1):prop)]))
>>            ) => ((rhyp_24 Mp Ocomp1_1(osetof_1([(yo_29:
>>               obj) => (((yo_29 Eo_1 yo_29)
>>               -> A_1):prop)]),
>>            [(yo_30:obj) => (((yo_30 Eo_1 yo_30)
>>               -> A_1):prop)]
>>            ,rhyp_24)):that A_1)]))
>>         :that A_1)]))
>>      ) Mp Deduction([(rhyp_33:that (osetof_1([(yo_34:
>>            obj) => (((yo_34 Eo_1 yo_34) ->
>>            A_1):prop)])
>>         Eo_1 osetof_1([(yo_35:obj) => (((yo_35
>>            Eo_1 yo_35) -> A_1):prop)]))
>>         ) => ((rhyp_33 Mp Deduction([(rhyp_40:
>>            that (osetof_1([(yo_41:obj) => (((yo_41
>>               Eo_1 yo_41) -> A_1):prop)])
>>            Eo_1 osetof_1([(yo_42:obj) => (((yo_42
>>               Eo_1 yo_42) -> A_1):prop)]))
>>            ) => ((rhyp_40 Mp Ocomp1_1(osetof_1([(yo_45:
>>               obj) => (((yo_45 Eo_1 yo_45)
>>               -> A_1):prop)]),
>>            [(yo_46:obj) => (((yo_46 Eo_1 yo_46)
>>               -> A_1):prop)]
>>            ,rhyp_40)):that A_1)]))
>>         :that A_1)]))
>>      :that A_1)]
>>   {move 0}


\end{verbatim}

\section{Constructive forms of negation, disjunction, and the existential quantifier}

We resume the development of logical primitives.  Here we give the constructive rules for negation, disjunction and existential quantification.

\begin{verbatim}Lestrade execution:


postulate ?? prop

>> ??: prop {move 0}



declare absurd that ??

>> absurd: that ?? {move 1}



declare Dd prop

>> Dd: prop {move 1}



postulate Panic absurd Dd that Dd

>> Panic: [(absurd_1:that ??),(Dd_1:prop) =>
>>      (---:that Dd_1)]
>>   {move 0}


\end{verbatim}

We introduce the false statement {\tt ??} (typeset notation for this is $\perp$) and introduce the rule that any proposition may be deduced from a false statement.

\begin{verbatim}Lestrade execution:


define ~ Dd : Dd -> ??

>> ~: [(Dd_1:prop) => ((Dd_1 -> ??):prop)]
>>   {move 0}


\end{verbatim}

We define negation.

\begin{verbatim}Lestrade execution:


postulate v A B prop

>> v: [(A_1:prop),(B_1:prop) => (---:prop)]
>>   {move 0}



postulate Addition1 B aa that A v B

>> Addition1: [(B_1:prop),(.A_1:prop),(aa_1:
>>      that .A_1) => (---:that (.A_1 v B_1))]
>>   {move 0}



postulate Addition2 A bb that A v B

>> Addition2: [(A_1:prop),(.B_1:prop),(bb_1:
>>      that .B_1) => (---:that (A_1 v .B_1))]
>>   {move 0}



declare cases that A v B

>> cases: that (A v B) {move 1}



open

   declare aa1 that A

>>    aa1: that A {move 2}



   declare bb1 that B

>>    bb1: that B {move 2}



   postulate case1 aa1 that Dd

>>    case1: [(aa1_1:that A) => (---:that Dd)]
>>      {move 1}



   postulate case2 bb1 that Dd

>>    case2: [(bb1_1:that B) => (---:that Dd)]
>>      {move 1}



   close

postulate Cases cases, case1, case2 that \
   Dd

>> Cases: [(.A_1:prop),(.B_1:prop),(cases_1:
>>      that (.A_1 v .B_1)),(.Dd_1:prop),(case1_1:
>>      [(aa1_2:that .A_1) => (---:that .Dd_1)]),
>>      (case2_1:[(bb1_3:that .B_1) => (---:that
>>         .Dd_1)])
>>      => (---:that .Dd_1)]
>>   {move 0}


\end{verbatim}

We introduce disjunction and its basic rules, addition and proof by cases.

\begin{verbatim}Lestrade execution:


postulate Exists tpred prop

>> Exists: [(.T_1:type),(tpred_1:[(tt1_2:in
>>         .T_1) => (---:prop)])
>>      => (---:prop)]
>>   {move 0}



declare existsev that tpred t

>> existsev: that tpred(t) {move 1}



postulate Egen0 tpred, t existsev : that \
   Exists tpred

>> Egen0: [(.T_1:type),(tpred_1:[(tt1_2:in .T_1)
>>         => (---:prop)]),
>>      (t_1:in .T_1),(existsev_1:that tpred_1(t_1))
>>      => (---:that Exists(tpred_1))]
>>   {move 0}



define Egen1 t existsev : Egen0 tpred, t \
   existsev

>> Egen1: [(.T_1:type),(t_1:in .T_1),(.tpred_1:
>>      [(tt1_2:in .T_1) => (---:prop)]),
>>      (existsev_1:that .tpred_1(t_1)) => (Egen0(.tpred_1,
>>      t_1,existsev_1):that Exists(.tpred_1))]
>>   {move 0}



define Egen2 tpred, existsev : Egen0 tpred, \
   t existsev

>> Egen2: [(.T_1:type),(tpred_1:[(tt1_2:in .T_1)
>>         => (---:prop)]),
>>      (.t_1:in .T_1),(existsev_1:that tpred_1(.t_1))
>>      => (Egen0(tpred_1,.t_1,existsev_1):that
>>      Exists(tpred_1))]
>>   {move 0}



declare existsev2 that Exists tpred

>> existsev2: that Exists(tpred) {move 1}



declare witness in T

>> witness: in T {move 1}



declare witnessev that tpred witness

>> witnessev: that tpred(witness) {move 1}



declare witnessprf [witness,witnessev => \
      that Dd] \
   



>> witnessprf: [(witness_1:in T),(witnessev_1:
>>      that tpred(witness_1)) => (---:that Dd)]
>>   {move 1}



postulate Einst existsev2, witnessprf that \
   Dd

>> Einst: [(.T_1:type),(.tpred_1:[(tt1_2:in
>>         .T_1) => (---:prop)]),
>>      (existsev2_1:that Exists(.tpred_1)),(.Dd_1:
>>      prop),(witnessprf_1:[(witness_3:in .T_1),
>>         (witnessev_3:that .tpred_1(witness_3))
>>         => (---:that .Dd_1)])
>>      => (---:that .Dd_1)]
>>   {move 0}


\end{verbatim}

We introduce the existential quantifier and its basic rules.  At this point we have introduced all operations and rules of constructive (intuitionist) logic.

Note that three different additional versions of existential instantiation with different choices of explicit arguments are given.

\section{Classical logic completed with double negation.  Proofs of some classical theorems.}

\begin{verbatim}Lestrade execution:


declare maybe that ~ ~ A

>> maybe: that ~(~(A)) {move 1}



postulate Dneg maybe that A

>> Dneg: [(.A_1:prop),(maybe_1:that ~(~(.A_1)))
>>      => (---:that .A_1)]
>>   {move 0}



open

   declare nega1 that ~Dd

>>    nega1: that ~(Dd) {move 2}



   define howler nega1 :absurd

>>    howler: [(nega1_1:that ~(Dd)) => (---:
>>         that ??)]
>>      {move 1}



   close

define Panic0 absurd Dd: Dneg(Deduction howler)


>> Panic0: [(absurd_1:that ??),(Dd_1:prop) =>
>>      (Dneg(Deduction([(nega1_2:that ~(Dd_1))
>>         => (absurd_1:that ??)]))
>>      :that Dd_1)]
>>   {move 0}


\end{verbatim}

We introduce the rule of double negation $\neg \neg P \vdash P$, and we show that the constructive rule {\tt Panic} can be implemented using
{\tt Dneg}.

What follows below is the full proof of the classically valid equivalence of $\neg A \rightarrow B$ and $A \vee B$, which we ought to comment
line by line with a parallel proof in English.  Notice how indentation in Lestrade output signals the depth of the nest of environments one is working in.

\begin{verbatim}Lestrade execution:


open

   declare side1 that (~A) -> B

>>    side1: that (~(A) -> B) {move 2}


\end{verbatim}

Suppose that $\neg A \rightarrow B$.  Our aim is to prove $A \vee B$.

\begin{verbatim}Lestrade execution:


   open

      declare contrahyp that ~(A v B)

>>       contrahyp: that ~((A v B)) {move 3}


\end{verbatim}

Our strategy for proving $A \vee B$ is to suppose $\neg(A \vee B)$ and reason to a contradiction.

\begin{verbatim}Lestrade execution:


      open

         declare howabouta that A

>>          howabouta: that A {move 4}



         define noa1 howabouta : Mp (Addition1 \
            B howabouta,contrahyp)

>>          noa1: [(howabouta_1:that A) => (---:
>>               that ??)]
>>            {move 3}



         close

      define thusnota contrahyp: propfixform(~A, \
         Deduction noa1)

>>       thusnota: [(contrahyp_1:that ~((A v
>>            B))) => (---:that ~(A))]
>>         {move 2}


\end{verbatim}

In the block of text above we prove $\neg A$ from the local hypotheses.  The strategy is to suppose that $A$, deduce $A \vee B$ from this by 
the rule of addition, then note the contradiction with the assumption $\neg (A \vee B)$ made above.  To follow this, it is useful to recall that the 
deduction of a contradiction when we have both $X$ and $\neg X$ is actually an instance of {\em modus ponens\/}, since $\neg X$ is defined
as $X \rightarrow \,\perp$.

\begin{verbatim}Lestrade execution:


      define thusb contrahyp: Mp (thusnota \
         contrahyp,side1)

>>       thusb: [(contrahyp_1:that ~((A v B)))
>>            => (---:that B)]
>>         {move 2}



      define thusaorb contrahyp: Addition2 \
         A thusb contrahyp

>>       thusaorb: [(contrahyp_1:that ~((A v
>>            B))) => (---:that (A v B))]
>>         {move 2}



      define thuscontra1 contrahyp: Mp (thusaorb \
         contrahyp,contrahyp)

>>       thuscontra1: [(contrahyp_1:that ~((A
>>            v B))) => (---:that ??)]
>>         {move 2}


\end{verbatim}

In the three lines above we deduce a contradiction:  we first deduce $B$ by modus ponens from previous lines $\neg A$ and $\neg A \rightarrow B$,
then we deduce $A \vee B$ from $B$ by the rule of addition, then we obtain a contradiction.

\begin{verbatim}Lestrade execution:


      close

   define classicalor1 side1 : Dneg(Deduction \
      thuscontra1)

>>    classicalor1: [(side1_1:that (~(A) ->
>>         B)) => (---:that (A v B))]
>>      {move 1}


\end{verbatim}

Applying {\tt Deduction} to the function {\tt thuscontra1} above gives a proof that $\neg \neg (A \vee B)$.  Applying {\tt Dneg} to this gives a proof
of $A \vee B$.  What we have actually done is constructed a function from the original assumption that $\neg A \rightarrow B$ to evidence that $A \vee B$.

\begin{verbatim}Lestrade execution:


   declare side2 that A v B

>>    side2: that (A v B) {move 2}


\end{verbatim}

Now we assume that $A \vee B$ and argue to the conclusion $\neg A \rightarrow B$.

\begin{verbatim}Lestrade execution:


   open

      declare ahyp1 that ~A

>>       ahyp1: that ~(A) {move 3}


\end{verbatim}

We assume $\neg A$ and our goal is now $B$.  Our strategy is to prove this by cases on our hypothesis $A \vee B$, first showing that $B$ follows from $A$,
then showing that $B$ follows from $B$.

\begin{verbatim}Lestrade execution:


      open

         declare ifa2 that A

>>          ifa2: that A {move 4}



         define ifa21 ifa2 : Mp ifa2 ahyp1


>>          ifa21: [(ifa2_1:that A) => (---:
>>               that ??)]
>>            {move 3}



         define ifa22 ifa2 : Panic (ifa21 \
            ifa2,B)

>>          ifa22: [(ifa2_1:that A) => (---:
>>               that B)]
>>            {move 3}


\end{verbatim}

A function from proofs of $A$ to proofs of $B$ is defined:  from a proof of $A$ we get a proof of $\perp$ because we
have a constant proof of $\neg A$ given.  From a proof of $\perp$ we get a proof of anything, in particular $B$.

\begin{verbatim}Lestrade execution:


         declare ifb2 that B

>>          ifb2: that B {move 4}



         define ifb21 ifb2 : ifb2

>>          ifb21: [(ifb2_1:that B) => (---:
>>               that B)]
>>            {move 3}


\end{verbatim}

The identity function takes proofs of $B$ to proofs of $B$.

\begin{verbatim}Lestrade execution:


         close

      define thusb2 ahyp1 : Cases side2, \
         ifa22, ifb21

>>       thusb2: [(ahyp1_1:that ~(A)) => (---:
>>            that B)]
>>         {move 2}


\end{verbatim}

We complete the proof of the conclusion $B$ from the hypothesis $\neg A$ by cases outlined above.

\begin{verbatim}Lestrade execution:


      close

   define classicalor2 side2 : Deduction \
      thusb2

>>    classicalor2: [(side2_1:that (A v B))
>>         => (---:that (~(A) -> B))]
>>      {move 1}



   close

define Classicalor1 A B : Deduction classicalor1


>> Classicalor1: [(A_1:prop),(B_1:prop) => (Deduction([(side1_2:
>>         that (~(A_1) -> B_1)) => (Dneg(Deduction([(contrahyp_3:
>>            that ~((A_1 v B_1))) => (((A_1 Addition2
>>            ((~(A_1) propfixform Deduction([(howabouta_4:
>>               that A_1) => (((B_1 Addition1
>>               howabouta_4) Mp contrahyp_3):
>>               that ??)]))
>>            Mp side1_2)) Mp contrahyp_3):that
>>            ??)]))
>>         :that (A_1 v B_1))])
>>      :that ((~(A_1) -> B_1) -> (A_1 v B_1)))]
>>   {move 0}



define Classicalor2 A B : Deduction classicalor2


>> Classicalor2: [(A_1:prop),(B_1:prop) => (Deduction([(side2_2:
>>         that (A_1 v B_1)) => (Deduction([(ahyp1_3:
>>            that ~(A_1)) => (Cases(side2_2,[(ifa2_4:
>>               that A_1) => (((ifa2_4 Mp ahyp1_3)
>>               Panic B_1):that B_1)]
>>            ,[(ifb2_5:that B_1) => (ifb2_5:that
>>               B_1)])
>>            :that B_1)])
>>         :that (~(A_1) -> B_1))])
>>      :that ((A_1 v B_1) -> (~(A_1) -> B_1)))]
>>   {move 0}



define Classicalor A B: propfixform \
   (((~A)->B)<->(A v B), Andproof (Classicalor1 \
   A B,Classicalor2 A B))

>> Classicalor: [(A_1:prop),(B_1:prop) => ((((~(A_1)
>>      -> B_1) <-> (A_1 v B_1)) propfixform ((A_1
>>      Classicalor1 B_1) Andproof (A_1 Classicalor2
>>      B_1))):that ((~(A_1) -> B_1) <-> (A_1
>>      v B_1)))]
>>   {move 0}


\end{verbatim}

Finally we exit to the outermost environment and prove our three theorems, two conditionals and a biconditional.  The conditionals are proved by applying
{\tt Deduction} to the appropriate functions developed above, and the biconditional is proved using {\tt Andproof}.

The following block of so far uncommented text proves the equivalence of $\neg(A \rightarrow B)$ and $A \wedge \neg B$ in the same style.

\begin{verbatim}Lestrade execution:


open

   declare side1 that ~(A -> B)

>>    side1: that ~((A -> B)) {move 2}



   open

      declare nota that ~A

>>       nota: that ~(A) {move 3}



      open

         declare buta that A

>>          buta: that A {move 4}



         define step10 buta : Mp buta nota


>>          step10: [(buta_1:that A) => (---:
>>               that ??)]
>>            {move 3}



         define step20 buta : Panic (step10 \
            buta, B)

>>          step20: [(buta_1:that A) => (---:
>>               that B)]
>>            {move 3}



         close

      define athenb nota : Deduction step20


>>       athenb: [(nota_1:that ~(A)) => (---:
>>            that (A -> B))]
>>         {move 2}



      define iscontra nota : Mp (athenb nota, \
         side1)

>>       iscontra: [(nota_1:that ~(A)) => (---:
>>            that ??)]
>>         {move 2}



      close

   define yesa side1 : Dneg(Deduction iscontra)


>>    yesa: [(side1_1:that ~((A -> B))) => (---:
>>         that A)]
>>      {move 1}



   open

      declare butb that B

>>       butb: that B {move 3}



      open

         declare supposea that A

>>          supposea: that A {move 4}



         define indeedb supposea : butb

>>          indeedb: [(supposea_1:that A) =>
>>               (---:that B)]
>>            {move 3}



         close

      define ahenceb butb : Deduction indeedb


>>       ahenceb: [(butb_1:that B) => (---:that
>>            (A -> B))]
>>         {move 2}



      define iscontra2 butb : Mp (ahenceb \
         butb,side1)

>>       iscontra2: [(butb_1:that B) => (---:
>>            that ??)]
>>         {move 2}



      close

   define notob side1 : propfixform(~B,Deduction \
      iscontra2)

>>    notob: [(side1_1:that ~((A -> B))) =>
>>         (---:that ~(B))]
>>      {move 1}



   define negimp1 side1 : Andproof(yesa side1, \
      notob side1)

>>    negimp1: [(side1_1:that ~((A -> B))) =>
>>         (---:that (A & ~(B)))]
>>      {move 1}



   declare side2 that A & ~B

>>    side2: that (A & ~(B)) {move 2}



   open

      declare ifathenb that A -> B

>>       ifathenb: that (A -> B) {move 3}



      define step11 ifathenb : Mp(Simplification1 \
         side2,ifathenb)

>>       step11: [(ifathenb_1:that (A -> B))
>>            => (---:that B)]
>>         {move 2}



      define step21 ifathenb : Mp(step11 \
         ifathenb,Simplification2 side2)

>>       step21: [(ifathenb_1:that (A -> B))
>>            => (---:that ??)]
>>         {move 2}



      close

   define negimp2 side2: propfixform(~(A \
      -> B),Deduction step21)

>>    negimp2: [(side2_1:that (A & ~(B))) =>
>>         (---:that ~((A -> B)))]
>>      {move 1}



   close

define Negimp1 A B : Deduction negimp1

>> Negimp1: [(A_1:prop),(B_1:prop) => (Deduction([(side1_2:
>>         that ~((A_1 -> B_1))) => ((Dneg(Deduction([(nota_3:
>>            that ~(A_1)) => ((Deduction([(buta_4:
>>               that A_1) => (((buta_4 Mp nota_3)
>>               Panic B_1):that B_1)])
>>            Mp side1_2):that ??)]))
>>         Andproof (~(B_1) propfixform Deduction([(butb_5:
>>            that B_1) => ((Deduction([(supposea_6:
>>               that A_1) => (butb_5:that B_1)])
>>            Mp side1_2):that ??)]))
>>         ):that (A_1 & ~(B_1)))])
>>      :that (~((A_1 -> B_1)) -> (A_1 & ~(B_1))))]
>>   {move 0}



define Negimp2 A B : Deduction negimp2

>> Negimp2: [(A_1:prop),(B_1:prop) => (Deduction([(side2_2:
>>         that (A_1 & ~(B_1))) => ((~((A_1 ->
>>         B_1)) propfixform Deduction([(ifathenb_3:
>>            that (A_1 -> B_1)) => (((Simplification1(side2_2)
>>            Mp ifathenb_3) Mp Simplification2(side2_2)):
>>            that ??)]))
>>         :that ~((A_1 -> B_1)))])
>>      :that ((A_1 & ~(B_1)) -> ~((A_1 -> B_1))))]
>>   {move 0}



define Negimp A B : propfixform((~(A -> B))<->A \
   & ~B, Andproof(Negimp1 A B, Negimp2 A B))


>> Negimp: [(A_1:prop),(B_1:prop) => (((~((A_1
>>      -> B_1)) <-> (A_1 & ~(B_1))) propfixform
>>      ((A_1 Negimp1 B_1) Andproof (A_1 Negimp2
>>      B_1))):that (~((A_1 -> B_1)) <-> (A_1
>>      & ~(B_1))))]
>>   {move 0}


\end{verbatim}

We note that the more sophisticated namespace management made possible by the ability to save environments (moves) has applications relative to our philosophical motivations.  If in world $j$ we have not decided the truth value of a proposition $p$, we can in different moves with index $j+1$ postulate objects of sorts  ${\tt that}\, p$ and ${\tt that} \,\neg p$, and develop further declarations and definitions in these two contexts, switching back and forth at will betweeen the two developments.  If one of $p$ or $\neg p$ leads to contradiction, we will then have proved the other (if we are using classical logic) in the original world $j$, and we will be able to import any definitions with that premise from the appropriate world $j+1$ down into the original world $j$.  Even more interesting, perhaps, is what happens if the question cannot be decided:  we can continue to develop two different alternative pictures of the world, and anything that we can prove in both, we can import into the original world $j$ (again, on the assumption that we have implemented classical logic).  Our philosophical view supports classical logic, but it does not support the view that there is a fact of the matter with respect to (for example) the Continuum Hypothesis (a statement in set theory which is known to be undecidable);  it suggests that we can explore mathematical universes in which CH holds, and mathematical universes in which $\neg$CH holds, and anything which follows from both hypotheses we can conclude is true in any universe satisfying our basic assumptions apart from CH, without presuming that we can or even should decide the question of CH one way or the other.  We are not thereby supposing that there is a God's-eye view in which every question is resolved, though in some sense we may be providing support for the coherence of the latter view.

\section{Basic declarations for a version of Quine's New Foundations}

\begin{verbatim}Lestrade execution:


postulate V type

>> V: type {move 0}



open

   declare Tt3 type

>>    Tt3: type {move 2}



   postulate typepred Tt3 prop

>>    typepred: [(Tt3_1:type) => (---:prop)]
>>      {move 1}



   close

declare typepredev1 that typepred V

>> typepredev1: that typepred(V) {move 1}



postulate Ambiguity typepredev1 that typepred \
   setsof V

>> Ambiguity: [(.typepred_1:[(Tt3_2:type) =>
>>         (---:prop)]),
>>      (typepredev1_1:that .typepred_1(V)) =>
>>      (---:that .typepred_1(setsof(V)))]
>>   {move 0}


\end{verbatim}

This is a conjectural formulation of the simple theory of types with Specker's axiom scheme of Ambiguity, which is equiconsistent with Quine's New Foundations.



We first declare a type {\tt V} as a primitive notion:  this is type 0 in a model of the simple theory of types.

The idea is that we declare a function {\tt Ambiguity} which will send evidence {\tt typepredev1} that a predicate {\tt typepred} of types holds of {\tt V}
to evidence that the same predicate holds of {\tt setsof V}, type 1 of the same model.  

We would want an inverse operation for {\tt Ambiguity} as well if we did not have double negation.

The reason that it appears that this might work is that the primitives we have given seem to allow formulation of predicates of types only under very limited circumstances:  basically the predicates of types that can be formulated are limited to assertions that formulas of the usual first order language of TST hold in the model of TST with {\tt V} as type 0 (with the additional point that the universal applicability of our natural number type for indexing functions on different types may imply that consequences of the Axiom of Counting hold in our ambiguous type theory).    We suspect that adding equality of types and quantification over types to this theory would lead to contradiction (and so it is important that quantification over the sort of type labels is not automatically supported by our framework).  We intend to supply proofs of this point if we are able to postulate them.

Another point worth noting is that the ``Axiom" of Infinity is provable in this system (without any use of Ambiguity) by use of the fact that our notion of iteration is applicable to any type using the same type of natural numbers.  I'll supply a proof of this at some point.

There is a further issue.  The Ambiguity rule as stated would allow a rewrite rule to be declared with {\tt rewrited} which would allow a rewrite of {\tt setsof V} to {\tt V}, which is disastrous.  There are two approaches:  we could continue to investigate the theory articulated above but with the constraint
that we cannot freely use the rewrite logic, or formulate this a little differently.

\begin{verbatim}Lestrade execution:


declare V1 type

>> V1: type {move 1}



declare V2 type

>> V2: type {move 1}



declare V3 type

>> V3: type {move 1}



postulate << V1 V2 prop

>> <<: [(V1_1:type),(V2_1:type) => (---:prop)]
>>   {move 0}



postulate Order1 V1 that V1 << setsof V1


>> Order1: [(V1_1:type) => (---:that (V1_1 <<
>>      setsof(V1_1)))]
>>   {move 0}



declare order1 that V1 << V2

>> order1: that (V1 << V2) {move 1}



declare order2 that V2 << V3

>> order2: that (V2 << V3) {move 1}



postulate Ordertrans order1 order2 that V1 \
   << V3

>> Ordertrans: [(.V1_1:type),(.V2_1:type),(order1_1:
>>      that (.V1_1 << .V2_1)),(.V3_1:type),(order2_1:
>>      that (.V2_1 << .V3_1)) => (---:that (.V1_1
>>      << .V3_1))]
>>   {move 0}



postulate maxtype V1 V2 type

>> maxtype: [(V1_1:type),(V2_1:type) => (---:
>>      type)]
>>   {move 0}



postulate Max1 V1 V2 that V1 << maxtype V1 \
   V2

>> Max1: [(V1_1:type),(V2_1:type) => (---:that
>>      (V1_1 << (V1_1 maxtype V2_1)))]
>>   {move 0}



postulate Max2 V1 V2 that V2 << maxtype V1 \
   V2

>> Max2: [(V1_1:type),(V2_1:type) => (---:that
>>      (V2_1 << (V1_1 maxtype V2_1)))]
>>   {move 0}



declare order3 that V1 << V3

>> order3: that (V1 << V3) {move 1}



declare order4 that V2 << V3

>> order4: that (V2 << V3) {move 1}



postulate Max3 order3 order4 that (maxtype \
   V1 V2) << V3

>> Max3: [(.V1_1:type),(.V3_1:type),(order3_1:
>>      that (.V1_1 << .V3_1)),(.V2_1:type),(order4_1:
>>      that (.V2_1 << .V3_1)) => (---:that ((.V1_1
>>      maxtype .V2_1) << .V3_1))]
>>   {move 0}



postulate ambtype typepred type

>> ambtype: [(typepred_1:[(Tt3_2:type) => (---:
>>         prop)])
>>      => (---:type)]
>>   {move 0}



declare evid1 that typepred (ambtype typepred)


>> evid1: that typepred(ambtype(typepred)) {move
>>   1}



declare evid2 that (ambtype typepred) << \
   V1

>> evid2: that (ambtype(typepred) << V1) {move
>>   1}



postulate Ambiguity2 evid1 evid2 that typepred \
   V1

>> Ambiguity2: [(.typepred_1:[(Tt3_2:type) =>
>>         (---:prop)]),
>>      (evid1_1:that .typepred_1(ambtype(.typepred_1))),
>>      (.V1_1:type),(evid2_1:that (ambtype(.typepred_1)
>>      << .V1_1)) => (---:that .typepred_1(.V1_1))]
>>   {move 0}


\end{verbatim}

The development above is consistent with Lestrade logic with rewriting and proves existence of types ambiguous for any desired concrete finite set of properties of types.  The idea is that there is a transitive relation on types under which a type is below its power type and two types always have a maximum, 
and for every predicate of types there is an associated type such that the predicate has the same value on every type above the given type.  Now for any
collection of predicates of types (sentences true of those types) we can go to the maximum of their associated types, and above that point we have ambiguity for those sentences.  We do not incur rewriting opportunities because we do not assert that any two types satisfy exactly the same predicates.

\section{The third and fourth Peano axioms}

For the moment, just an outline.  The third Peano axiom can be proved using the operation $A \mapsto A \cup V$ in any double power type.  Applying this function 0 times to the empty set gives the empty set, and applying this function $n+1$ times for any $n$ will give $V$, which is provably nonempty in a double power type, so $0=n+1$ is false.

The fourth Peano axiom is best shown by considering the type of natural numbers and the Frege natural numbers over the power type of the natural numbers.  If numbers $1,\ldots,n$ are distinct, the Frege natural number containing $\{\{1\},\ldots,\{n\}\}$ is the result of iterating the Frege successor operation $n$ times on the Frege zero in the appropriate type.  Now, the Frege natural number containing $\{\{1\},\ldots,\{n\},\emptyset\}$ is a new one, and the result of iterating the Frege successor operation $n+1$ times on the Frege zero, which establishes that $n \neq n+1$.   This establishes that Infinity holds in the model of type theory based on the natural numbers, which is enough to show that Axiom 4 holds.

These are going to be tricky arguments with lots of preliminaries under Lestrade.

In the interpretation of NF, if the size of $V$ is the result of applying the Frege successor operation $n$ times to the Frege zero, than the same is true of ${\cal  P}(V)$, and this is readily shown not to be true.  The fact that the natural numbers are type free in this interpretation of NF (being defined in a way independent of the Frege natural numbers in each high enough type) suggests that the stratified consequences of the axiom of counting ought to hold.

\section{A note on polymorphic typing}

We provide the standard function type constructor.

\begin{verbatim}Lestrade execution:


declare tau8 type

>> tau8: type {move 1}



declare tau9 type

>> tau9: type {move 1}



open

   declare x8 in tau8

>>    x8: in tau8 {move 2}



   postulate f8 x8 in tau9

>>    f8: [(x8_1:in tau8) => (---:in tau9)]
>>      {move 1}



   close

postulate Arrowtype tau8 tau9 type

>> Arrowtype: [(tau8_1:type),(tau9_1:type) =>
>>      (---:type)]
>>   {move 0}



declare ff8 in Arrowtype tau8 tau9

>> ff8: in (tau8 Arrowtype tau9) {move 1}



declare xx8 in tau8

>> xx8: in tau8 {move 1}



postulate Arrowapp ff8 xx8 in tau9

>> Arrowapp: [(.tau8_1:type),(.tau9_1:type),
>>      (ff8_1:in (.tau8_1 Arrowtype .tau9_1)),
>>      (xx8_1:in .tau8_1) => (---:in .tau9_1)]
>>   {move 0}



postulate Fun f8 in Arrowtype tau8 tau9

>> Fun: [(.tau8_1:type),(.tau9_1:type),(f8_1:
>>      [(x8_2:in .tau8_1) => (---:in .tau9_1)])
>>      => (---:in (.tau8_1 Arrowtype .tau9_1))]
>>   {move 0}



postulate Arrowcomp f8, xx8 that Arrowapp(Fun \
   f8,xx8)= f8 xx8

>> Arrowcomp: [(.tau8_1:type),(.tau9_1:type),
>>      (f8_1:[(x8_2:in .tau8_1) => (---:in .tau9_1)]),
>>      (xx8_1:in .tau8_1) => (---:that ((Fun(f8_1)
>>      Arrowapp xx8_1) = f8_1(xx8_1)))]
>>   {move 0}


\end{verbatim}

The following is a set of tentative Lestrade declarations supporting a popular brand of polymorphic type.  We supply the arrow type above
because any application of this polymorphic type scheme requires some prior type constructors to work with.  This section may be expanded with some actual constructions to illustrate this.

The declarations here are very bare:  for example, no extensionality principles are given.  The arrow type declarations above are made at move 0
(they may be postulated and used later);  the polymorphism constructors are kept in a sandbox environment, declared at move 1.

8/16/2017 entirely new polymorphism declarations

\begin{verbatim}Lestrade execution:


open

   open

      declare tau50 type

>>       tau50: type {move 3}



      postulate taufun50 tau50 type

>>       taufun50: [(tau50_1:type) => (---:type)]
>>         {move 2}



      postulate polyfun50 tau50 in taufun50 \
         tau50

>>       polyfun50: [(tau50_1:type) => (---:
>>            in taufun50(tau50_1))]
>>         {move 2}



      close

   postulate Polytype taufun50 type

>>    Polytype: [(taufun50_1:[(tau50_2:type)
>>            => (---:type)])
>>         => (---:type)]
>>      {move 1}



   postulate Polyfun polyfun50 in Polytype \
      taufun50

>>    Polyfun: [(.taufun50_1:[(tau50_2:type)
>>            => (---:type)]),
>>         (polyfun50_1:[(tau50_3:type) => (---:
>>            in .taufun50_1(tau50_3))])
>>         => (---:in Polytype(.taufun50_1))]
>>      {move 1}



   declare polyobj50 in Polytype taufun50


>>    polyobj50: in Polytype(taufun50) {move
>>      2}



   declare tau51 type

>>    tau51: type {move 2}



   postulate Polyapp polyobj50 tau51 : in \
      taufun50 tau51

>>    Polyapp: [(.taufun50_1:[(tau50_2:type)
>>            => (---:type)]),
>>         (polyobj50_1:in Polytype(.taufun50_1)),
>>         (tau51_1:type) => (---:in .taufun50_1(tau51_1))]
>>      {move 1}



   postulate Polycomp polyfun50, tau51 that \
      ((Polyfun polyfun50) Polyapp tau51) = \
      polyfun50 tau51

>>    Polycomp: [(.taufun50_1:[(tau50_2:type)
>>            => (---:type)]),
>>         (polyfun50_1:[(tau50_3:type) => (---:
>>            in .taufun50_1(tau50_3))]),
>>         (tau51_1:type) => (---:that ((Polyfun(polyfun50_1)
>>         Polyapp tau51_1) = polyfun50_1(tau51_1)))]
>>      {move 1}



   open

      declare tau50 type

>>       tau50: type {move 3}



      define taufuntest tau50 :tau50 Arrowtype \
         tau50

>>       taufuntest: [(tau50_1:type) => (---:
>>            type)]
>>         {move 2}



      open

         declare x50 in tau50

>>          x50: in tau50 {move 4}



         define testid x50:x50

>>          testid: [(x50_1:in tau50) => (---:
>>               in tau50)]
>>            {move 3}



         close

      define polyfuntest tau50 : Fun testid


>>       polyfuntest: [(tau50_1:type) => (---:
>>            in (tau50_1 Arrowtype tau50_1))]
>>         {move 2}



      close

   define Polyfuntest : Polyfun polyfuntest


>>    Polyfuntest: [(---:in Polytype([(tau50_3:
>>            type) => ((tau50_3 Arrowtype tau50_3):
>>            type)]))
>>         ]
>>      {move 1}



   define Polyapptest : (Polyfuntest Polyapp \
      Nat) Arrowapp 0

>>    Polyapptest: [(---:in Nat)]
>>      {move 1}



   define Polycomptest : Polycomp polyfuntest, \
      Nat

>>    Polycomptest: [(---:that ((Polyfun([(tau50_5:
>>            type) => (Fun([(x50_6:in tau50_5)
>>               => (x50_6:in tau50_5)])
>>            :in (tau50_5 Arrowtype tau50_5))])
>>         Polyapp Nat) = Fun([(x50_7:in Nat)
>>            => (x50_7:in Nat)]))
>>         )]
>>      {move 1}



   open

      declare Y50 in Nat Arrowtype Nat

>>       Y50: in (Nat Arrowtype Nat) {move 3}



      define noncepred Y50 : ( (Polyfuntest \
         Polyapp Nat) Arrowapp 0)= Y50 Arrowapp \
         0

>>       noncepred: [(Y50_1:in (Nat Arrowtype
>>            Nat)) => (---:prop)]
>>         {move 2}



      close

   define Polycomptest2: Substitution0 noncepred, \
      Polycomptest, Reflexeq( (Polyfuntest \
      Polyapp Nat) Arrowapp 0)

>>    Polycomptest2: [(---:that (((Polyfuntest
>>         Polyapp Nat) Arrowapp 0) = (Fun([(x50_9:
>>            in Nat) => (x50_9:in Nat)])
>>         Arrowapp 0)))]
>>      {move 1}



   open

      declare n50 in Nat

>>       n50: in Nat {move 3}



      define natid n50:n50

>>       natid: [(n50_1:in Nat) => (---:in Nat)]
>>         {move 2}



      close

   define Polycomptest3 : Substitution (Arrowcomp \
      natid,0,Polycomptest2)

>>    Polycomptest3: [(---:that (((Polyfuntest
>>         Polyapp Nat) Arrowapp 0) = 0))]
>>      {move 1}



   close
\end{verbatim}


\section{Introduction to Lestrade}

This section contains a formal discussion of Lestrade (the framework and the software), versions of parts of which are already embedded in the discussion of the development of a particular Lestrade theory which precedes this.  The discussion here is generally more detailed and can be consulted for reference.  Note that the phrase ``the current move" is used for move $i$ in this section, where ``the last move" is used above.  These usages are equivalent:  in terms of our temporal metaphor, ``the present" is right after the last declaration in move $i$ and right before the first declaration in move $i+1$.

\subsection{Introduction}

Lestrade is a general purpose logical framework for mathematics.  It is motivated by a philosophical premise:  contrary to the statements of its founders, practitioners, and detractors (when they say anything about philosophical matters at all), modern foundational mathematics, with classical logic and including Cantorian set theory, does not depend on actual infinities.  All activities in mathematics can be viewed as finitary, or at least as involving no more than potential infinities, and this does not make classical logic or the Cantorian transfinite mathematics illegitimate.

Such a claim might be taken to be rather startling.  We support it with the development of a computer implementation of the framework (the Lestrade Type Inspector) in which one can actually conduct mathematical investigations in the style suggested by the underlying philosophical view.  Some aspects of the formal framework have been strongly shaped by what might seem accidental consequences of the way we have implemented the software.  Some of these features are not in our opinion accidental.  Of course, some features of the software and of the framework as formally presented are the results of design choices which could have been made differently:  we will indicate some choices we have made which were not essential.

Lestrade uses the idea that mathematical propositions and their proofs are among the objects of mathematics.  It uses a version of the Curry-Howard isomorphism, which is usually associated with constructive logic (which can be implemented in Lestrade) but can, as here, be used to motivate classical logic.
As the mention of the Curry-Howard isomorphism should indicate, Lestrade is a dependent type system.

The view taken of functions in Lestrade is not the standard one.  We do not view functions as completely given infinite tables of values (this would quite defeat our basic philosophical premise!); instead, we regard a function as a machine which will return an output (of a type given in advance, possibly depending on input values) given a sequence of inputs (later ones possibly of types depending on earlier ones).  When a function is given by an expression, we can reasonably say that we have finitely described the action of the function given any inputs that may later be presented to us.  We do not not, however, assume that an arbitrarily given function is determined by some unknown expression;  our language is not regarded as constraining what functions are possible.  Theorems about functions provable by induction on the structure of the expressions we are able to define so far may suggest themselves as further axioms, but our framework does not presume that such principles are true.

Lestrade can be viewed as a version of Automath, though there are considerable differences.  Since it is related to Automath, it is also more distantly related to other systems such as Coq with this genealogy.

\subsection{Metaphysics of Lestrade}

The things we talk about in Lestrade we will refer to as {\em entities} when we are being completely general.  The entities fall into two large categories, 
{\em objects\/} and {\em functions\/}.  Each object or function has a sort (the word {\em type\/} is reserved for a specific variety of sert, as we will see shortly).

The sorts of object can be reviewed quickly.

\begin{enumerate}

\item  There is a sort {\tt prop} intended to be inhabited by mathematical propositions.

\item  For each $p$ of sort {\tt prop}, there is a sort {\tt that} $p$ inhabited by evidence for $p$.  A proof of $p$ is evidence for $p$, of course, but we do not hold that evidence for $p$ must be a formal proof as such.  If {\tt that} $p$ is inhabited, we do take $p$ to be true:  there is no probable evidence here.

\item  There is a sort {\tt obj} intended to be inhabited by ``untyped mathematical objects".  In an implementation of ZFC, for example, the sets would be of sort {\tt obj}.  Note that other sorts would be in use, in spite of the fact that ZFC is an untyped theory, because its propositions and their proofs (or evidence) would be objects of other sorts.

\item  There is a sort {\tt type} inhabited by ``type labels".  A typical example would be {\tt Nat}, the type of natural numbers.  We find it useful to call these objects type labels in order to resist the temptation to view them as (necessarily usually infinite) collections containing all the objects of the indicated type, which would tell against our philosophical premise.

\item  For each type (label) $\tau$, there is a sort {\tt in} $\tau$ inhabited by the objects of type $\tau$.  A natural number $n$ would have sort
{\tt in Nat}.  We say that an object of sort {\tt in} $\tau$ is of type $\tau$.  

\end{enumerate}




And that is all.  But it turns out to be quite a lot, once the additional apparatus of functions is introduced.

The analogy between {\tt prop/that} and {\tt type/in} is almost perfect in Lestrade:  the analogy is perfect in the core functions of the Type Inspector and could be made perfect throughout with a slight modification to the software which we will note later;  though the symmetry is likely to be collapsed by
postulates in particular theories, where we may not want to treat proofs/evidence for propositions in the same way as mathematical objects of various types.


A function in Lestrade takes a fixed positive number of arguments.  The sort of a function is determined by the sorts of its arguments and its output,
with the further subtlety that the sort of each argument may depend on the values of earlier arguments and the sort of the output may depend on the values of the arguments.

The general notation for a function sort is $$(x_1,\tau_1),\ldots,(x_n,\tau_n)\Rightarrow (-,\tau).$$  The variables $x_i$ are bound in this notation, and bound variables in distinct function sort notations are viewed as distinct.  Each $\tau_i$ is the sort of $x_i$, and may contain $x_j$'s only for $j<i$.  The $\tau_i$'s may be object or function sorts.  The notation $\tau$ stands for the output sort, which must be an object sort and may contain any of the $x_i$'s.

A user of the Lestrade Type Inspector never writes a function sort notation (or the  kind of notation for specific functions presented just below):  Lestrade does present such notations in output.  The purpose of presenting the notation at this point is to explain what sorts of function there are.

A general notation for a function given by an explicit definition $y=f(x_1,\ldots,x_n)$ is

$$(x_1,\tau_1),\ldots,(x_n,\tau_n)\Rightarrow (y,\tau),$$  where of course each $x_i$ has type $\tau_i$ and $y$ must have sort $\tau$ and $$(x_1,\tau_1),\ldots,(x_n,\tau_n)\Rightarrow (-,\tau)$$ will be its sort (which imposes conditions on the $\tau_i$'s and $\tau$ which are described above).

That is the complete sort system of Lestrade.  The rest is user postulation of objects and functions of different types, and user development of useful definitions from the primitive constructions, where ``user" may indifferently mean ``user of the Lestrade Type Inspector" or ``developer of a mathematical theory in the Lestrade framework".

\subsection{The Lestrade Environment:  a metaphor for mathematical activity}

The Lestrade environment is in concrete terms a finite sequence of finite lists of declarations of identifiers.  Each declaration is a pair whose first component is  an identifier and whose second component is a function sort notation or a function notation\footnote{The second component of the form $[(y,\tau)]$ in the declaration of a defined object can be thought of as a notation for a constant function to be applied to an empty argument list, though Lestrade does not explicitly treat it that way.}.  This sequence always has at least two lists of declarations in it, and we refer to its length as $i+2$.  The $j$th list will be referred to as ``move $j$" (because we have a temporal metaphor for what is going on, though we are certainly not talking about physical time).  Move $i+1$ is called ``the next move" and move $i$ is called ``the current move" (or ``the last move", in some documents;  the idea is that the ``present" is right after the end of the current/last move, and just before all the declarations in the next move).

We should think of each move as a possible world in some modality.  We use a temporal metaphor:  the current move and the previous moves are as it were ``past" and the items declared there are constant:  the next move is ``future" and the items declared there are variable.  The concrete mathematical actions we carry out should make it clearer what is going on.

All identifiers declared in all the moves are distinct.  We think of the entities represented by those identifiers as having been discovered in the order of the moves, and within each move in the order in which they are listed.  This is useful for getting dependencies to work correctly without rather laborious checks.

We introduce the six core commands of Lestrade, the declaration commands {\tt declare}, {\tt postulate}, and {\tt define}, and the environment handling commands {\tt open}, {\tt close}, and {\tt clearcurrent}.

\begin{description}

\item[declare:] An instance of the {\tt declare} command is of the form {\tt declare} $i$ $\tau$, where $i$ is a fresh identifier and $\tau$ is an expression for an object sort (which of course may not contain any identifier which has not been declared previously).  The effect of the command is to introduce a variable $i$ of sort $\tau$, an item in the next move.

\item[postulate:] An instance of the {\tt postulate} command is of the form 

{\tt postulate} $f$ {\tt args} (:) $\tau$

The component $f$ is a fresh identifier.

The component {\tt args} is an argument list, which may be empty, or may be of the form $x_1,\ldots,x_n$, where the $x_i$'s are identifiers previously declared in the next move, none of which were introduced using the {\tt define} command, appearing in the order in which they appear in the next move.  This order restriction automatically enforces dependency relations
on the $x_i$'s.  

The component $\tau$ must be an expression for an object sort.  

All non-defined identifiers declared at the next move on which $\tau$ or the sort $\tau_i$ of each $x_i$ depends must be either among the $x_k$'s in {\tt args} or should appear in the sort of some $x_k$.  Lestrade will insert any non-defined identifiers found in $\tau$ or some $\tau_i$ into its internal representation of the argument list of $f$ at the latest possible point;  when Lestrade evaluates $f$ at particular lists of arguments, it will attempt to deduce the values of these implicit arguments [and may fail:  this is not a failure of the type system, but a failure of Lestrade input/output, as it were, and such failures can always be avoided by re-declaring the function with more arguments given explicitly].  The details of the implicit argument scheme are a complication here, but must be mentioned as the implicit argument scheme is very useful and is used immediately in examples.

The effect of the command is to enhance the argument list
to an argument list $x'_1,\ldots,x'_m$, appearing in the order in which they are declared in the next move and with every identifier declared at the next
move and appearing in $\tau$ or any of the $\tau_i$'s (or indeed any of the sorts $\tau'_i$ of $x'_i$'s) appearing as an $x'_j$, and then declare the fresh identifier $f$
as having sort $$(x'_1,\tau'_1),\ldots,(x'_m,\tau)\Rightarrow (-,\tau),$$ appending this declaration to {\em the current move\/}.    

If {\tt args} is empty, the effect of the command is to declare $f$ of type $\tau$ at the current move rather than the next move:  this is a declaration of a constant.  

The colon before the $\tau$ is now (we believe) always optional:  earlier it was sometimes required for things to parse correctly.

When executed when the next move is move 1, the {\tt postulate} command should be thought of as introducing  axioms and primitive notions.  When used at later moves in combination with the {\tt open} and {\tt close} commands to be discussed below, it is used to introduce function variables (and for other purposes, as for example to introduce hypothetical primitives or axioms).  The reader should note that the {\tt declare} command only allows us to introduce function parameters of object sorts,  but our description of function sorts above allows the possibility of parameters of function sorts.

\item[define:]  An instance of the {\tt define} command is of the form 

{\tt define} $f$ {\tt args} : $y$

The component $f$ is a fresh identifier.

The component {\tt args} is an argument list, which may be empty, or may be of the form $x_1,\ldots,x_n$, where the $x_i$'s are identifiers previously declared in the next move, none of which were introduced using the {\tt define} command [we will see here that such declarations have distinctive features], appearing in the order in which they appear in the next move.  This order restriction automatically enforces dependency relations
on the $x_i$'s. 
 
The component $y$ must be an expression representing an object, whose sort we denote by $\tau$.  

All non-defined identifiers declared at the next move on which $y, \tau$ or the sort $\tau_i$ of each $x_i$ depend must be either among the $x_k$'s in {\tt args} or should appear in the sort of some $x_k$.  Lestrade will insert any non-defined identifiers found in $y,\tau$, or some $\tau_i$ and not found among the arguments into its internal representation of the argument list of $f$ at the latest possible point;  when Lestrade evaluates $f$ at particular lists of arguments, it will attempt to deduce the values of these implicit arguments [and may fail:  implicit argument handling is not a failure of the type system, but a failure of Lestrade input/output, as it were, and such failures can always be avoided by re-declaring the function with more arguments given explicitly].  The details of the implicit argument scheme are a complication here, but must be mentioned as the implicit argument inference scheme is very useful and is used immediately in examples.

  The effect of the command is to enhance the argument list
to an argument list $x'_1,\ldots,x'_m$, appearing in the order in which they are declared in the next move and with every identifier declared at the next
move and appearing in $y, \tau$, or any of the $\tau_i$'s (or indeed any of the sorts $\tau'_i$ of $x'_i$'s) appearing as an $x'_j$, and then declare the fresh identifier $f$
as having sort $$(x'_1,\tau'_1),\ldots,(x'_m,\tau)\Rightarrow (y,\tau),$$ appending this declaration to {\em the current move\/}.  

Of course, $f$ is actually being declared as of sort  $$(x'_1,\tau'_1),\ldots,(x'_m,\tau)\Rightarrow (-,\tau),$$ with $y$ serving as an annotation that it is a specific function.

If {\tt args} is empty, the effect of the command is to declare $f$ of type $(y,\tau)$ [really, as being of type $\tau$ with the additional data of its particular identity as $y$] at the current move rather than the next move:  this is a declaration of a defined constant.  

The colon before the $\tau$ is required.

\item[remarks on postulate and define commands:]  In either the {\tt postulate} or {\tt define} commands, in terms of the metaphor,
$f$ is declared as a presently given object, which, whatever $x_i$'s of type $\tau_i$ may be given in the future, will return an output of type $\tau$.  In the case of the {\tt define} command, we are specifically told what object will be returned in each case (we have a template into which to insert the arguments);  in the case of the {\tt postulate} command we suppose that the values will be presented on demand on presentation of appropriately sorted inputs.  In neither case are we obliged to suppose that we know all the values at once.  Even in the case of the {\tt define} command, executing the definition with a particular list of arguments may cause us to request as yet unknown values of primitive functions introduced by the {\tt postulate} command (indeed, this will almost certainly be the case).  

Part of the commitment is that if we are given $x_1,\ldots,x_n$, we can deduce the implied values of $x'_1,\ldots x'_m$ from the sorts of the explicitly given arguments:  it should be noted that this can fail at run-time as it were, as we may see in examples. The implicit argument inference feature is actually a feature of the input/output of Lestrade:  as far as the sort checker (which is the heart of Lestrade)  is concerned, all functions have all of their required arguments.  A problem with implicit argument inference can always be solved by using a version of the offending function with all arguments given explicitly.

\item[remarks on the environment commands:]  The commands {\tt open}, {\tt close}, and {\tt clearcurrent} manipulate the environment.  Only their core behavior is described here.   There is a system for saving and restoring moves, and in this context some of these commands may appear with names of saved moves as arguments:  these uses are not described here.

\item[open:]  The {\tt open} command adds a new empty list of declarations to the end of the environment.  This has the effect of incrementing the parameter $i$
and changing the identities of the current move and the next move:  the old next move becomes the new current move, and the new empty move is the new next move.

\item[close, clearcurrent:]  The {\tt close} command can only be executed if $i>0$.  It simply deletes the next move.  The environment is shortened by one, the old current move becomes the next move and the old move $i-1$ becomes the current move (again;  it must have been at some time before).  The {\tt clearcurrent} command is a variant of {\tt close}:  it replaces the next move with an empty list (clearing all variable declarations, as it were) but leaves the length of the environment unchanged.  The command {\tt clearcurrent} is required as an independent command because {\tt close} cannot be used to clear declarations  in move 1.

\item[function variables:]  Now we can explain how to generate function parameters.   To introduce a function parameter $f$, execute the {\tt open} command, introduce parameters for $f$ of the desired types, then use the {\tt postulate} command to declare $f$, followed by the {\tt close} command, which leaves us with $f$ of the desired type declared at the next move.  If a function parameter itself requires parameters of function sorts, this may require repeated use of {\tt open} and  {\tt close}, but it can be done.

\item[variable expressions:]  Defined identifiers declared at the next move are as it were complex variable expressions.  When they are used in the expression $y$ in a {\tt define} command, they must be expanded out:  where defined constants appear, they are expanded in the obvious way;  where defined functions appear in applied position, their application is carried out;  where defined functions appear as arguments they are replaced with their function notation with bound variables given above:  this must be done because a declaration at the current move cannot depend on a defined identifier at the next move, whose declaration disappears when the {\tt close} command is executed.  The variable parameters of a {\tt postulate} command are, as we will see, replaced by bound variables [differentiated by applying a fresh numerical index to each variable, preserving the condition on Lestrade expressions that identical bound variables are always associated with the same instance of that variable as a binder]  and their types are supplied as part of the information in the function sort reported for the constructed identifier:  in any case, no declaration in a particular move can depend on a declaration appearing in a later move or later in the same move.  Defined identifiers declared in the current move or previous moves do not need to be expanded out.

\end{description}

\subsubsection{Namespace management refined:  saving and retrieving environments}

With the limited environment handling given above, there is no way to remove or revise declarations of variables and variable expressions in move 1 other than clearing all of them.  After a while, it is quite hard to remember what sorts have been assigned to parameters and variable expressions, and for that matter what order they appear in (recalling that parameters in {\tt postulate} and {\tt define} commands must appear in order of declaration).  We have already noted
that the {\tt clearcurrent} command will clear all declarations at the next move.

More intelligent namespace management is supported by the full specification of the {\tt open}, {\tt clearcurrent}, and {\tt save} commands.

Each move is assigned a name.  The default name is its numeral index (the $j$ such that it is move $j$).  The command {\tt save envname} will save the next move with the name
{\tt envname}, associated with the list of names of preceding moves at the time it is saved (a saved move is actually identified by the sequence of names of all moves at the time it is saved, and this is how it is identified internally;  this means that moves saved in different contexts can quite safely be tagged with the same name).  The command {\tt open envname} will open an already existing move (of the right index, wth the same preceding moves) with the name
{\tt envname} or if there is no such move, or create a new blank move with that name.  The command {\tt clearcurrent envname} will clear the net move and replace it with a move named {\tt envname} if there is such a move with the appropriate preceding names of moves associated with it or replace it with a blank move of that name otherwise.  A move cannot be saved or opened with its default numeral name:  the reason for this is that we do not want the
parameterless {\tt open} or {\tt clearcurrent} command to unexpectedly invoke declarations from a saved environment.  Further, when an environment is named using the {\tt open}, {\tt clearcurrent}, or {\tt save} command, none of the preceding moves associated with it, other than move 0, may have its default numeral name.

Any identifiers in a saved environment which conflict with identifiers declared in earlier moves since it was saved have {\tt '} or {\tt \$} appended to them, depending on whether they are alphanumeric or special character identifiers.  Identifiers ending in these characters cannot be declared by the user.

This means that instead of having a linear sequence of moves, which we can think of as times or possible worlds, we have a tree structure.  Each node in the tree of moves has a name;  different nodes may actually have the same name, since the identity of a node is determined by the sequence of names of the nodes on the branch leading to it (including its own name), not the name attached to it by itself.  Every such sequence of nodes contains move 0 in the first position:  for any $n>1$, if a move with the name `$n$' appears in position $n+1$ of the sequence then the entire sequence is labelled with successive numerals.  The shorthand way of putting this is that no move may be saved with its default numeral name (even as part of the sequence of previous moves saved along with a move explicitly saved) with its default numeral name, except move 0.


\subsection{Lestrade Notation}

In this section we describe the notation which the user can enter for objects, functions, and object sorts.

An identifier is a string of characters of positive length which consists of zero or one upper case letters followed by zero or more lower case characters followed by zero or more numerals, or consists of zero or one special characters taken from a list      \begin{verbatim}
~!@#$%^&*-+=/<>|?
\end{verbatim}

An object notation is either an identifier declared of object type or an object notation enclosed in parentheses or a notation $f(t_1,\ldots,t_n)$  or $t_1\, f \,t_2,\ldots,t_n$ (synonymous) [in the mixfix notation  $t_1$ must be enclosed in parentheses if it is a function identifier and $n>1$ is required]  where
$f$ is declared as a function identifier of appropriate type, or a variant of the latter two notations obtained by dropping some of the parentheses and commas.  The rule is that the parser will read as long an expression as possible before performing any sort checking.  Thus, commas (or close parentheses, or colons) must appear after function
identifiers used as arguments to keep them from being applied to following terms, and preceded with commas [or sometimes other punctuation]  to keep them from absorbing previous terms as an infix.  The parser will read a parenthesis following a non-infix function identifier as starting an argument list, so if it is desired to enclose a first argument in parentheses, it is also necessary to enclose the entire argument list.  It should be remembered that in effect all operations have the same precedence and group to the right as in the ancient language APL.  Problems with user input can always be avoided by putting in more parentheses and commas.  Lestrade output will show as many parentheses and commas as possible, and will use infix notation for operators of arity 2 which have first arguments which are objects.  Lestrade output will never use mixfix notation with more than one argument after the function symbol.

Lestrade user function notations are either identifiers declared as functions or or function notations enclosed in parentheses or notations $f(t_1,\ldots,t_m)$ where $m$ is less than the number of arguments which $f$ requires:  such a term represents the function $$(x_{m+1},\ldots,x_n) \Rightarrow f(t_1,\ldots,t_m,x_{m+1},\ldots,x_n).$$  Such a term can appear only as an argument (not in applied or infix position), and the parentheses around the argument list are required.

Lestrade user object sort notations are then {\tt prop}, {\tt that} $p$ where $p$ is an expression for an object of sort {\tt prop}, {\tt obj},
{\tt type}, or {\tt in} $\tau$ where $\tau$ is an expression for an object  of sort {\tt type}.

In a Lestrade declaration line (with any of the first three commands) the identifier is always the second component of the command:  declarations always present the declared identifier in prefix notation.  The arguments
may be separated with commas as required, and the argument list (which is never enclosed in parentheses) may be terminated with a colon {\tt :} if required.  This is mandatory in a {\tt define} command and we believe now always optional (but allowed) in a {\tt postulate} command.  A colon will never appear in a {\tt declare} command, which does not contain an argument list.

\subsection{Lestrade Sort Checking and Definition Expansion}

An identifier will have a sort determined by lookup in the environment.

We use the notation $T[t/x]$ for the result of replacing the variable $x$ with the term $t$ in the term $T$.  Of course a formal definition of substitution in the presence of variable binding constructions requires care (and is not given here).

A function application term $f(t_1,\ldots,t_n)$ or $t_1\, f\, t_2,\ldots,t_n$, for simplicity supposed supplied with all needed arguments (we suppose that the implicit argument inference algorithm has already been carried out), where $f$ has type $$((x_1,\tau_1),\ldots,(x_n,\tau_n) \Rightarrow (-,\tau)$$ will sort check if $t_1$ is of type $\tau_1$
and if either $n=1$ (in which case its sort is $\tau[t_1/x_1]$) or if $n>1$ and if we introduce $f'$, a new function symbol of type
$$(x_2,\tau_2[t_1/x_1]),\ldots,(x_n,\tau_n[t_1/x_1]) \Rightarrow (-,\tau[t_1/x_1])$$ and $f'(t_2,\ldots,t_n)$ is well-typed [in this case, we return the sort of $f'(t_2,\ldots,t_n)$ as the sort of the original expression].

Where $f$ is defined as  $$((x_1,\tau_1),\ldots,(x_n,\tau_n) \Rightarrow (y,\tau)$$ we evaluate $f(t_1,\ldots,t_n)$ very similarly.  If $n=1$ we return $y[t_1,x_1]$.  If $n>1$, we define a new function symbol $f'$ for $$(x_2,\tau_2[t_1/x_1]),\ldots,(x_n,\tau_n[t_1/x_1]) \Rightarrow (y[t_1,x_1],\tau[t_1/x_1])$$ and return $f'(t_2,\ldots,t_n)$.

Expansion of definitions is employed in two different contexts.  Whenever an identifier passes out of scope (when a defined identifier in the next move is used in the definition of an entity introduced at the current move), this identifier will be expanded.  If it is a defined object, the identifier is replaced with its definition.
If it is a defined function applied to arguments, the appropriate values of the arguments are substituted into its definition.  If it is a function appearing as an argument, it is replaced by the anonymous notation for that function used in its declaration (something similar happens if a function with a truncated argument list appears as an argument).  The second use of definitional expansion is in matching (determining when two expressions are the same):  the matching facility will check by carrying out definitional expansions behind the scenes to determine whether two expressions being compared have the same meaning.  This use of definitional expansion does not lead to the expansion of defined terms in Lestrade output.

\end{document}

quit
