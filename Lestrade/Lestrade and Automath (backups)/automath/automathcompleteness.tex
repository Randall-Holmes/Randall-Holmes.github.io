\documentclass{article}

\usepackage{amssymb}

\title{Notes toward a completeness theorem for Automath as a logic}

\author{M. Randall Holmes}

\begin{document}

\maketitle

We consider the proposition that (a certain version of) Automath is a logic in the proper sense which enjoys a Completeness Theorem.

The version of Automath for which we make this argument has as a primitive a type $\perp$ which is to be uninhabited.  We start with a theory with some declared primitives:  our aim is to show that if it is consistent (no term of type $\perp$ can be constructed), it has a model.  By a model we mean a set theoretical structure in which each type that the model thinks is inhabited is inhabited, and in which elements of function types can be applied to elements of the domain of their types as one would expect, and so forth.

The strategy is to enumerate the types, and postulate an inhabitant of each one in turn.  We generate terms with the newly postulated inhabitants.  If a new term of type $\tau$ allows us to
generate a term of type $\perp$, we abandon the postulated term and note the presence of a term of type $[x:\tau]\perp$ witnessed by that bad term construction (these terms may of course depend on terms postulated earlier).  At the end of the process, we
have a term model of the theory, in which each type asserted to be inhabited actually has an inhabitant, and in which the type $\perp$ never acquires an inhabitant unless the primitives of the theory actually allow such a construction.  This model will satisfy classical logic, though we are not adding classical logical primitives.

The function elements of the term model have their function behavior explicitly given (computation by reductions as usual).

We probably want AUT-QE so that we have implication and quantification over all types as primitives.  AUT-QE2, which has equality on entity types as primitive, is even better.

Is a proof of the completeness theorem along these lines better than the usual approach?

Of course, one would need to support this with expressibility of the usual logic in Automath.  This is desirable anyway.

Does such an approach have pedagogical merits?

Should I carry out the proof (with attendant specification of Automath semantics..) in Automath...?

\end{document}