\documentclass{slides}

\usepackage{amssymb}

\title{How did Zermelo get along without an ordered pair in 1908 (or did he?)}

\author{Randall Holmes}

\date{Boise State Logic Seminar, Sept. 10, 2019\\ with slight corrections after the presentation}

\begin{document}

\begin{slide}

\maketitle

\end{slide}

\begin{slide}

{\tiny

Abstract: Ernst Zermelo published two papers in 1908 which could be regarded as the real starting point of the now generally accepted set theoretical foundation for mathematics..  The first was a new version of his proof of the well-ordering theorem from the Axiom of Choice.  The second was a paper on the axiomatics of set theory with some further investigations of equinumerousness of sets, which arguably was largely motivated by the need to make the underpinnings of the first paper explicit.  We note that the modern system ZFC regarded as the working foundation of mathematics by most mathematicians now is an extension of the system described in the second paper.  There is a question about both of these papers which should naturally arise just from a general familiarity with their contents:  the representation of sets and functions in modern set theory depends on a set theoretical implementation of the ordered pair, and no such implementation was available until the work of Norbert Wiener in 1914.  There is a brief answer to this question for the well-ordering theorem paper (very much worthy of further discussion!):  an order can be represented by the set of its initial segments.  The answer for the development of equinumerousness is different and is the subject of this talk.  It can be said briefly that Zermelo made clever use of unordered pairs in his discussion of equinumerousness.  But a very close examination suggests that a definition in terms of a notion of ordered pair is implicit in his development, a fact he may not have realized.}



\end{slide}

\begin{slide}

{\Large Overview}

The work I am reporting on here is a small corner of a large project I did this summer, of which I will try to give some impression in a talk to the graduate student seminar on the 27th,
and which may generate further talks for this seminar.

Ernst Zermelo published two important papers in set theory in 1908.  The first was a version of his proof of the well-ordering theorem (a version had appeared in 1904).  The second was a paper on axiomatics of set theory and some further investigations in set theory (notably into the notion of cardinality of sets).  This second paper is the index paper on the system of set theory which is now the generally accepted foundation of mathematics, though further axioms of Replacement and Foundation were not adjoined until 1920.
\end{slide}

\begin{slide}
It is worth observing that the axiomatics in the second paper can be viewed as motivated by the need to provide an underpinning for the first paper.

(Discuss the axioms and how they differ from the current presentation briefly.  Point out the local application of Russell's paradox which is used later.)
\end{slide}

\begin{slide}

{\Large Zermelo's 1908 axiomatics}

Zermelo is quite clear that some of the objects in the domain of his theory may not be sets, and that any  objects which are not sets have no elements.

\begin{description}

\item[Extensionality:]  Sets with the same elements are the same.  Notice that this implies that there is at most one set with no elements, but there may be many non-sets with no elements.

\item[Elementary sets:]  There is an empty set $\emptyset$.  If $x,y$ are distinct objects, the sets $\{x\}$ and $\{x,y\}$ exist.  Zermelo does not think of singleton sets as special cases of the unordered pair, as we tend to do.

\item[Separation:]  For any set $A$ and property $P$, $\{x : x \in A \wedge P(x)\}$ exists.   Zermelo did not phrase this as we do as a scheme, a separate axiom for each
$P(x)$ expressed in the language of first-order logic.  In fact, he objected to this approach.

\item[Power set:]  For any set $A$, the set $${\cal P}(A)=\{x : x \subseteq A\}$$ exists.

\item[Set union:]  For any set $A$, the set $$\bigcup A = \{x:(\exists y:x \in y \wedge y \in A)\}$$ exists.

\item[Infinity:]  There is a set $I$ such that $\emptyset \in I$ and $(\forall x:x \in I \rightarrow \{x\} \in I)$.  Separation gives us the intersection of all such sets as a set of there is one of them, and this intersection is (for Zermelo) the set of natural numbers.  Modern presentations of Infinity, even in Zermelo set theory, follow von Neumann in using $(x \mapsto x \cup \{x\})$ as successor instead of the singleton operation.  It is an interesting fact that the two formulations of Infinity are {\em not\/} equivalent in the absence of Replacement.

\item[Choice]:  Any collection of pairwise disjont sets has a choice set.

\end{description}

There are many talks possible on ways in which Zermelo's 1908 set theory differs from the current ZFC.

\end{slide}

\begin{slide}

{\Large The local application of the ``paradox" of Russelll}

Zermelo proves a theorem, used below, which is in effect a practical application of the reasoning in the paradox of Russell.

Let $S$ be a set.  $\{x : x \in S \wedge x \not\in x\}$ is a set by Separation.  We show further that $$\{x : x \in S \wedge x \not\in x\} \not\in S.$$

Define $R$ as $\{x : x \in S \wedge x \not\in x\}$.  Suppose for the sake of a contradiction that $R \in S$.  By Separation, $R \in R$ is logically equivalent to
$R \in S \wedge R \not\in R$, which is logically equivalent to $R \not\in R$ under the assumption that $R \in S$ is true.  But this is absurd.

So there is a uniform way to choose an element of the complement of any set.

\end{slide}


\begin{slide}

This summer, I implemented the axiomatics paper and most of the well-ordering theorem proof under my dependent types-based theorem prover, Lestrade.  I will not even say what this means for this talk, except to observe that computer implementation of the proofs in a paper enforces very careful reading of the paper.  This is a very good thing, as I think these papers are now more cited than read.  The observation reported here is something that I noticed in reading the axiomatics paper which I would likely not have noticed in a cursory reading.

\end{slide}

\begin{slide}

{\Large The Question}

A question which is raised by mere familiarity with the contents of the two Zermelo papers without looking carefully at them is this.

The fact that the ordered paper can be defined in pure set theory was not known until it was shown by Norbert Wiener in 1914 (using a definition different from the one due to Kuratowski now universally used).  So how did Zermelo prove (or even state) the well-ordering theorem or investigate the notion of cardinality without an implementation of the ordered pair?


\end{slide}

\begin{slide}

{\Large An aside}

Wiener's definition of the pair $(x,y)$, for anyone who might be interested, was $\{\{\{x\},\emptyset\},\{\{y\}\}\}$.  This definition has some technical advantages.

\end{slide}

\begin{slide}

The reason why as moderns we feel the lack of the ordered pair is that we all learned early in our careers that a relation or a function is implemented as a set of ordered pairs, and a little later we learn that the ordered pair $(x,y)$ is the set $\{\{x\},\{x,y\}\}$.  Now of course the well-ordering theorem is a proposition about relations on an arbitrary set, so how did Zermelo even state it, much less prove it?

The answer re the well-ordering theorem is brief (and we have explored this at length in other work).  Zermelo implements a well-ordering as the set of its initial segments.  This is not a general purpose way to implement {\em all\/} relations, but it does work for orders and indeed for a more general class of relations.  This is not our focus here:  I have a recent published paper on the effectiveness of representations of functions and relations of this kind and some related kinds.

\end{slide}

\begin{slide}

{\Large Equinumerousness without ordered pairs}

Our focus here is Zermelo's technique for representing and reasoning about equinumerousness of sets.

He starts by defining equinumerousness for {\em disjoint\/} sets.  Disjoint sets $A$ and $B$ are equinumerous iff there is  a set $F$ such that each element of $F$ is an unordered pair meeting both $A$ and $B$ and each element of $A \cup B$ belongs to exactly one element of $F$.  We understand $a \in A$  as corresponding to $b \in B$ iff $\{a,b\} \in F$.

He does not say this, but this can be adapted to support a general representation of relations whose domain and codomain are disjoint.  Very likely he knows this, but he has a specific application in mind.


\end{slide}

\begin{slide}

Of course Zermelo wants to be able to talk about equinumerousness of sets which are not necessarily disjoint.

He gives a more general definition:  sets $A$ and $B$ are equinumerous iff there is a set $C$ which is disjoint from $A$ and equinumerous with  $A$ and disjoint from $B$ and equinumerous with  $B$.

Then he proves that for any sets $A, B$ there is a set $C$ which is disjoint from $A$ and $B$ and equinumerous to $A$.  This is where things get interesting.

\end{slide}

\begin{slide}

We look at this argument in detail.  Let $A,B$ be any two sets.  Let $x$ be an object which is not an element of $A \cup \bigcup(A \cup B)$.  Then the set
$A'= \{\{a,x\} : a \in A\}$ has the desired properties.  This set is disjoint from $A$ because every element of it has an element $x$ not belonging to $\bigcup A$.
It is disjoint from $B$ because every element of it has an element $x$ which is not an element of $\bigcup B$.   It is equinumerous with $A$:  this is witnessed by the
set $\{\{a,\{a,x\}\}: a \in A\}$.

\end{slide}

\begin{slide}

{\Large $\ldots$ or is it without ordered pairs?}

We can then see (modulo some additional work, which appears in the investigations paper) that $A$ is equinumerous with $B$ iff $A'$ is equinumerous with $B$, with the set $A'$ playing the role of
the set $C$ in the general definition of equinumerousness.  The set of pairs witnessing equivalence of $A$ and $A'$ is chosen in a way not depending on the size of $B$; so in fact
equinumerousness of $A$ and $B$  is witnessed by a single set of unordered pairs.

The punchline is that this set of unordered pairs witnessing equinumerousness of $A'$ and $B$, and so of $A$ and $B$, can in fact be interpreted as a set of {\em ordered\/} pairs with first projection in $A$ and second projection in $B$.
There is a kind of definition of the ordered pair implicit in Zermelo's development!

\end{slide}

\begin{slide}

First we observe that we can fix exactly what the object $x$ should be, and this was certainly in Zermelo's mind.  He points out for any set $S$, the set $\{s \in S: s \not\in s\}$ provided by his axiom of separation does not belong to $S$.  So we can very naturally explicitly define $x_{A,B}$ as $$\{z \in A \cup \bigcup(A \cup B):z \not\in z\}$$ and
define $A'$ as $$\{\{a,x_{A,B}\}:a \in A\}.$$

We can then assert that $A$ and $B$ are equivalent iff there is a set $F$ each of whose elements is an unordered pair, with each element of $F$ an unordered pair meeting both $A'$ and $B$  and each
element of $A' \bigcup B$ belonging to exactly one element of $F$.

\end{slide}

\begin{slide}

A typical element of this $F$ is of the form $\{\{a,x_{A,B}\},b\}$.

We now observe that we could define $(a,b)$ (for $a \in A$ and $b\in B$) as $\{\{a,x_{A,B}\},b\}$, and from such a ``pair" $(a,b)$ we can extract its first and second projections:
the second projection $b$ is the unique element of $(a,b)$ which belongs to $B$ (we know that $\{a,x_{A,B}\}$, an element of $A'$, does not belong to $B$) and $a$ is
the unique element of $A$ belonging to the element of $(a,b)$ not belonging to $B$ (we know that $x_{A,B}$ is not an element of $A$).

\end{slide}

\begin{slide}

This is an odd definition of the ordered pair because it is in a sense ``local":  define $A \times B$ as $\{\{\{a,x_{A,B}\},b\}:a \in A \wedge b \in B\}$ and we can make two observations:
we have provided a general framework for representing not only bijections from $A$ to $B$ but any relation with $A$ as domain and $B$ as codomain (extracted from Zermelo's papers but possibly  not noticed by him) as a subset of $A \times B$.  This is familiar to us.  What is less familiar (the ``locality" of the definition) is that the definition of the
ordered pair $(a,b)$ which is used depends on the domain and codomain, $A$ and $B$, via the choice of $x_{A,B}$.


\end{slide}

\begin{slide}

In various places earlier in this development I have noted things implicit in Zermelo's development which I think he knew perfectly well though he did not state them (because he didn't  need to state them to realize his intentions).  I suspect that Zermelo did not know that implicit in his development of equinumerousness of possibly overlapping sets was a complete solution to the problem of representing relations and functions in set theory quite close to the modern solution.

Isn't that interesting?


\end{slide}


\end{document}