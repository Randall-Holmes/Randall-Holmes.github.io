\documentstyle{article}

\title{Review of ``Finsler Set Theory:  Platonism and Circularity'', David Booth and Renatus Ziegler, eds.}

\author{M. Randall Holmes}

\begin{document}

\maketitle

\section{Introduction}

This book contains translations of papers on set theory by Paul
Finsler with supporting materials prepared by the editors.

The book is divided into three parts, a philosophical part, a
foundational part, and a combinatorial part.

We summarize our conclusions at the outset.  Finsler has interesting
things to say about the philosophy of set theory and of mathematics in
general.  He has an interesting analysis of the notion of set, and of
the reasons why the paradoxes of self-reference present themselves and
the nature of an appropriate response to the paradoxes.  He defends a
Platonist view of mathematics using arguments which we find
congenial.

Finsler's contributions to foundations are interesting, but ultimately
unsatisfactory in the form in which he presents them.  We agree with
the apparent consensus of modern set theorists that the Finsler set
theory is incoherent as presented.  In asserting that Finsler's full
set theory is incoherent, we are taking issue not only with Finsler
himself, but with the editors of the volume. However, Finsler's
concept of ``circle-free'' sets can be successfully implemented; this
is seen in the set theory of Ackermann, a complete description of
which can be extracted from Finsler's papers presented in this book;
discovering this fact made reading the book worthwhile for the
reviewer.  Finsler's notion of allowing non-well-founded sets but
strengthening the notion of extensionality on non-well-founded sets
has been carried further by Scott and Aczel, among others, and systems
with ``anti-foundation axioms'' resembling Axiom II of Finsler's
system are now popular and are finding applications.

Finally, the combinatorial aspects of Finsler's set theoretical work
issue from a specific line of investigation into the properties of
small non-well-founded sets which can be carried out in the various
set theories defined in Aczel's book on non-well-founded set theory as
well as in the Finsler set theory.  No profoundly interesting
combinatorial results are given in this book, but it seems that there
might be potentially interesting problems in finite combinatorics in
the investigation of hereditarily finite non-well-founded sets.  There
is a philosophically interesting idea to this section: Finsler
enunciates a view of sets as ``generalized numbers'', regarding a set
as having its elements as ``predecessors''.  This idea may be of some
use in trying to understand Finsler's intuitions about set theory.
Finsler goes on to define analogues of arithmetic operations on small
sets.  We will have nothing further to say about this aspect of the
book in this review, but the article by Booth introducing this part
and at least the first article by Finsler are worthwhile.

\section{Finsler's Platonism}

There are many things which we find extremely problematic in Finsler's
work, but his defense of mathematical Platonism is not one of them.
It is Finsler's Platonism which leads him to the conviction that the
paradoxes of set theory must (and should be able to be) {\em
solved\/}, not endured (as somehow unavoidable) or merely avoided (as
by schemes of type theory).  Finsler believed that the paradoxes arose
from errors in reasoning which it should be possible to identify
clearly.  He resists the idea that the mathematical world is
``constructed'' by human thought.  We quote from p. 79: ``Now an
antinomy would result if $\ldots$ we were to see ourselves being
compelled to `form' mathematical objects (for example, the set of all
transfinite ordinal numbers) that is, to {\em affirm\/} their
existence.  This idea confuses the {\em class\/} of the ordinal
numbers and the differently defined {\em set\/}.  In an exact set
theory sets do not arise through an act of collecting, but are
mathematical objects with definite properties, in exactly the same way
as the natural numbers are objects with definite properties.  From
p. 80: ``In order to be able to know what is true and what is false in
set theory, one has to have won back the certainty lost through the
antinomies, however, nothing stands in the way.  This does not mean
that one could then solve every single question; this is not the case
elsewhere in mathematics, either. $\ldots$ By ``Platonism'' one can of
course understand various things.  In the case before us this
expression only means that in the realm of set theory too, objective
relationships do exist; it is not meant that sets would have to be
given to us in some way other than our knowledge of their existence.
Consequently, quite another question is how we know these objective
relationships and how are we to base classical mathematics on them?''
Later, same page, referring to some disagreement with Bernays about
the significance of Ackermann's formalization of part of Finsler's
theory (which was probably independent of Finsler's work): ``The
actual existence of infinitely many things cannot be guaranteed in
this way.  This turns mathematics into a `doing as if', pretending
that there are infinitely many things.  I cannot accept this.''
Finsler does state a criterion for the existence of mathematical
objects: on p. 169: ``consistent things can always be taken to exist;
in pure mathematics, existence means nothing more than freedom from
contradiction''.  The reviewer finds nothing here to disagree with.

\section{Paradoxes of Self-Reference}

Paradoxes of self-reference are addressed by Renatus Ziegler, one of
the editors, in an introductory article, ``Intrinsic Analysis of
Antinomies and Self-Reference''.  We found this article not to be
helpful.  We summarize our objections to its approach by considering
an example and then considering the consequences of this approach for
an algebraic example.

The examples that Ziegler discusses are of the form

\begin{description}

\item[$a$:]  $b$ is true

\item[$b$:]  $a$ is false

\end{description}

This pair of sentences is circularly defined and certainly
paradoxical.  Ziegler alleges that the reason that the paradox arises
is that the two occurrences of $a$ and $b$ respectively are
identified; if the identification were shown to be unsound, the
paradoxical character of the pair of sentences would be resolved.  We
actually agree with this assertion.  But Ziegler goes on to find the
difference between the two occurrences of $a$ (for example) in the
fact that the second occurrence of $a$ is presented to us as the
subject of a proposition while the first occurrence of $a$ is
presented to us as an entire proposition.  On this basis, he tells us
that the two occurrences of $a$ (and likewise the two occurrences of
$b$) should be differnetiated (say by adding superscripts) producing a
situation something like this:


\begin{description}

\item[$a^{(1)}$:]  $b^{(1)}$ is true

\item[$b^{(2)}$:]  $a^{(2)}$ is false

\end{description}

Of course, there is no longer any contradiction in this set of
sentences.  Ziegler says that this is the result of ``strictly
following the {\em principle of identity\/}, which says that only
objects possessing identical properties may be identified'' (p. 17).

We present a {\em reductio ad absurdum\/} of this line of argument.
Consider the familiar identity $$x + x = 2x$$.  The three $x$'s here
are presented to us in quite different ways: the first is the left
term of a sum, the second is the right term of a sum, and the third is
not a term at all but a factor.  Thus, they must be distinct: the
correct way of presenting this identity is $$x^{(1)}+x^{(2)} =
2x^{(3)}$$.  Of course, this is entirely absurd: the whole point of
the identity (and any other algebraic equation) is that a letter such
as $x$ {\em can\/} refer to the same object in different contexts.
Indeed, it must refer to the same object in our algebraic context; the
modified identity is no longer valid (it is the same as asserting that
$$x + y = 2z$$ for all $x$, $y$, and $z$)!  The differences between
the contexts in which different occurrences of a letter like $x$ occur
do not make the referents of the different tokens different.

The sentences must be understood as intended: the two occurrences of
$a$ and the two occurrences of $b$ are indeed to be understood as
having identical referents if they can be regarded as intelligible at
all.  If we cannot understand different occurrences of the same symbol
as having the same referent, we cannot engage in mathematical
discourse of even the simplest kind.

Something can be made of what Ziegler is trying to say if we
explicitly take into account the fact that the constructions involved
in paradoxes of self-reference are ``token-reflexive'', so where a
token (of certain specific kinds) occurs can affect its meaning;
Finsler does something like this with his idea of ``implicit content''
below.

Finsler himself has much more reasonable things to say about paradoxes
of self-reference; the reader should look at Finsler's articles in the
book and disregard Ziegler's treatment.

Finsler states (we believe correctly) that the difficulty with
examples like the one above is that they involve {\em circular
definition\/}, and that a circular definition does not need to be
satisfied.  It is more usual to assert that a ``circular definition''
is not a {\em definition\/} at all, but given that Finsler allows such
things to be called definitions, what he has to say about them is
unexceptionable.  A trio of examples taken from his paper ``Are there
contradictions in mathematics'' (presented in this volume) should make
this clear: $$x = a-b,$$ where $a$ and $b$ are previously given
numbers not depending on $x$, defines a number $x$.  $$x = a-x$$ is
``circular'', and is not a definition in the usual sense, but it is
nonetheless successful in specifying a unique value $\frac a2$ for
$x$.  In the final example $$x = a+x,$$ in which $a$ is understood to
be nonzero, we see a ``circular definition'' which is not satisfiable.
Notice that Ziegler's treatment would entirely destroy the sense of
either of the last two equations.

To see how Finsler approaches an example more similar to Ziegler's, of
a circularly defined {\em proposition\/}, consider his treatment of
the Liar in the paper ``Are there undecidable propositions?''  His
conclusion is that the Liar sentence ``This sentence is false'', which
in Ziegler's notation might be presented as

\begin{description}
\item[s:] The sentence $s$ is false
\end{description}

is actually non-paradoxically false.  For he notes that $s$ has not
only an explicit content ``sentence $s$ is false'' but an implicit
content ``sentence $s$ is true'' deriving from the fact that it {\em
is\/} sentence $s$.  The assertion ``sentence $s$ is false'' is true,
if it is written anywhere but next to the label $s$, where it has an
implicit conjunct ``$s$ is true'' which renders it contradictory.  We
are actually paraphrasing his argument, because we are using the
labelling idiom from Ziegler, but we believe, and the reader may check
to his own satisfaction, that we are much closer to Finsler's own
assertion. 

Another example, taken from the paper ``Are there contradictions in
mathematics'', is that of a blackboard on which are written the
symbols 1, 2, 3, and ``the smallest natural number not represented on
this blackboard''.  If we call the blackboard $B$, Finsler's
conclusion is that the smallest natural number not represented on
blackboard $B$ is 4, but that the occurrence of the string ``the
smallest natural number not represented on this blackboard'' actually
written on $B$ does not represent any natural number; the naive
argument that it represents 4 (which leads to a paradox!) disregards
the implicit content ``I am a number written on $B$'' of the string,
which contradicts the explicit content and causes the string to fail
to refer.

The idea of implicit content which Finsler uses here seems to be
supplementary to his argument that circular definitions are not
necessarily satisfied; the notion that circular definitions are not
necessarily satisfied can be used by itself to conclude that the
labels $a$ and $b$ in Ziegler's example do not necessarily refer to
any sentences, and the same for the Liar sentence and the purported
number on Finsler's blackboard.  The occurrence of paradoxes if they
are understood as referring can be used to draw the stronger
conclusion that they actually do not refer.  Notice that our
conclusion with regard to the Liar differs from Finsler's: we would
prefer to regard the Liar sentence as not being a sentence rather than
as being false.

Now we consider Finsler's application of these ideas to the paradoxes
of set theory.  He asserts that the Russell paradox of ``the set of
all sets which are not elements of themselves'' is in fact a circular
definition.  The reason for this is that Cantor's original definition
of the notion of a set is understood by Finsler to be a circular
definition.  He says that there would be no such problem with Cantor's
definition if it spoke of collections of e.g., concrete material
things, which are themselves understood without reference to the
notion of ``set'', but as soon as we allow sets themselves and the
relation of membership to enter into the definitions of further sets,
we are in the realm of circular definition.

We clarify this point by considering two examples.  We consider a set
$A$, the set of all stars in the Milky Way Galaxy which are brighter
than our Sun.  The stars in our Galaxy are objects which we understand
without any reference to sets.  We may consider sets of stars
(collections of stars considered as one thing) without fear of
circularity.  For any star $s$, we have $s \in A$ ($s$ is an element
of $A$) defined as ``$s$ is a star in the Milky Way brighter than the
Sun''; note that the definition of $s\in A$ eliminates all reference
to set theoretical concepts.  Now consider a set $R$, defined as ``the
set of all sets which are not elements of themselves'': for any set
$x$, $x \in R$ is defined as ``$x$ is not an element of $x$'', or,
more briefly, $x \not\in x$.  Notice that the set theoretical relation
$\in$ is not eliminated here; membership in the set $R$ is defined in
terms of further information about membership in sets.  Now observe
that the sentence $R \in R$ is defined as $R \not\in R$, a
contradiction!

Finsler does not believe that the solution to the paradoxes is to be
found in an attempt to eliminate the circular definitions; he thinks,
and mathematical experience reveals as well, that the ``circularity''
here (the dependence of some assertions about membership in sets on
further assertions about membership in sets) is {\em essential\/}.
For example, the set $\cal N$ of natural numbers is defined as the
intersection of all sets which contain 0 and are closed under the
successor operation (in the usual set theory {\em ZFC\/}, 0 is defined
as the empty set and the successor operation applied to a general set
$x$ is $x \cup \{x\}$).  The assertion that a given object $n$ is a
natural number is expanded via the definition of $\cal N$ into the
assertion that $n$ belongs to each of a very large class of further
sets (and, even worse, this class includes $\cal N$ itself!)  The
viewpoint adopted in modern set theory is that comprehension axioms
which assert the existence of sets defined by properties of their
elements are not definitions at all; Finsler's view was that they are
definitions, but, since they are circular, only some of them will
succeed; this may ultimately be no more than a difference in
terminology.

We have now summarized Finsler's view of the nature of the paradoxes
of self-reference.  Equally worthy of note is his attitude toward the
paradoxes.  He is confident that the paradoxes represent, not some
essential deficiency in human understanding, but an {\em error\/} in
reasoning that can be identified and corrected.  He dislikes solutions
to the paradoxes along the lines of type theory, which seem to him
only to avoid the contradictions, not to explain them.  The reviewer
finds this attitude entirely congenial.  His confidence that the
paradoxes represent mistakes which can be corrected rather than
fundamental limitations of reason goes along with his Platonist
philosophy of mathematics.  We also think that he correctly identifies
the nature of the mistake, in the case of the paradoxes of set theory,
as having to do with the distinction between a class (an arbitrary
collection of objects) and a set (a particular single object which we
have associated with some collection); in fact, the reviewer has
independently articulated an explanation of the paradoxes resting on
the same distinction.  A further discussion of this distinction
belongs to the ``foundational part'' of the review.

\section{Formally Undecidable Propositions}

Two papers by Finsler on undecidable propositions are included.  One
of these predates G\"odel's first paper on the subject.  We briefly
describe Finsler's construction of a sentence which is formally
undecidable.  We are given a fixed language $L$ consisting of finitely
(or countably) many symbols.  A fixed alphabetical order on these
symbols is also given.  A fixed dictionary $D$ giving meanings for
finite combinations of the symbols taken from $L$ is also given.  An
object will be called ``finitely definable'' if there is a finite
collection of symbols from $L$ with which it is associated in $D$.
Finsler now reasons about binary sequences (countable sequences of 0's
and 1's); he points out that diagonalization over the finitely
definable binary sequences gives a binary sequence which can have no
definition in $D$.

Finsler defines a {\em formal proof\/} as a finite combination of
symbols of $L$ whose translation via $D$ is a correct proof.  We
consider all formal proofs for the fact that the number 0 occurs
infinitely often in a given binary sequence or for the fact that it
does not occur infinitely often.  We list these proofs
lexicographically.  Each such proof determines a binary sequence (the
one in which it proves that there are infinitely many 0's); we form
the sequence whose $n$th term is different from the $n$th term of the
binary sequence associated with the $n$th proof for each $n$.  Call
this sequence $s$.  We now consider the statement ``the number 0 does
not occur infinitely often in $s$'' (or its negation!)  This sentence
cannot be decided by any formal proof (by construction).  It is,
though, clearly the case that 0 occurs infinitely many times in the
sequence $s$ (there are, for example, infinitely many proofs that the
sequence consistning entirely of 1's does not contain infinitely many
0's), so the statement we have given is false, though formally
undecidable -- and we have given a ``proof'' that it is false!

Finsler has not anticipated G\"odel here; the contribution that
G\"odel made was to provide a completely formal definition of the
notion of a proof, and to see what really could be done formally with
this notion.  We can formally define the predicate of strings ``is a
proof of a sentence of the form ``$b$ is a binary sequence and there
are infinitely many $n$ such that $b(n)=0$'' or of the negation of
such a sentence''.  Given this, we can try formally defining the
notion ``if the $n$th string satisfying the predicate above has
$b(n)=0$, where $b$ is the sequence to which the $n$th proof refers,
then 1 else 0''.  This would be Finsler's $s$.  But this definition
does not succeed, because it requires a truth predicate: we need to be
able to determine whether the sentence $b(n)=0$ holds.  Tarski's
theorem tells us that it is not possible to define the truth predicate
of a language inside that language.  Finsler has not achieved what
G\"odel achieved, because the sequence $s$ cannot actually be defined
in his language, and so his ``formally undecidable sentence'' is not a
sentence expressible in his language at all.

We outline a correct development of a formally undecidable sentence
along similar lines. Define $P_n$ as the $n$th proof deciding a
sentence ``$b$ is a binary sequence and there are infinitely many $n$
such that $b(n)=0$''.  Define $b_n$ as the name of the sequence
referred to in $P_n$.  Define $S$ as the sequence whose $n$th term is
1 if we can prove that $b_n(n)=0$ and 0 otherwise.  Certainly $S$ is a
binary sequence (this can be proven).  Further, we can prove that
there are infinitely many $n$ such that $S(n)=0$; consider $n$ such
that $P_n$ proves that the constant sequence 1 does not have
infinitely many 0's: there are infinitely many such $n$, in each case
we can prove that $b_n(n)=1$, so $S(n)=0$.  This proof can be carried
out (details omitted).  This proof $P_m$ has $b_m = S$.  Now consider
$S(m)$: this is 1 if we can prove that $b_m(m)=0$ and 0 otherwise; but
$b_m = S$!  We can see that the sentence $S(m) = 0$ is formally
undecidable; $S(m)=0$ exactly if we cannot prove $S(m)=0$!  We can
also see that $S(m) = 0$ is true; if it were false, that is, if
$S(m)=1$, we would be able to prove $S(m)=0$, which is absurd.  But
$S(m)=0$ is not the analogue to Finsler's sentence: the analogue to
Finsler's sentence is ``$S(n)=0$ does not hold for infinitely many
$n$'', which is both false and provably false.

The editors of the book are clearly aware that the distinction between
Finsler's and G\"odel's approach is that G\"odel has expended more
effort on the ``arithmetization of syntax''.  They do not appreciate
the fact that Finsler's approach, though more cursory in the area of
formal syntax, does require a correct understanding of what can be
referred to in the dictionary $D$.  Finsler is {\em wrong\/} when he
asserts (p. 54) that the formal proof he gives there can be expressed
in words taken from $D$; the truth predicate for his language is
required in a full definition of his ``anti-diagonal sequence'', and
this predicate cannot be defined in $D$.  It {\em is\/} necessary,
even from Finsler's more philosophical standpoint, to concern oneself
with what one really can express in one's language and what one
cannot.  A modern editorial treatment should have included a formal
analysis of what Finsler actually did prove.

\section{Finsler set theory}

\subsection{Naive set theory; paradoxes; sets and classes}

We now turn to Finsler's development of the foundations of
mathematics.  We review his development at the beginning of the paper
``On the foundations of set theory, part 1''.  He begins by remarking
that the assumption that arbitrarily specified things can always be
combined into a set is the foundation of naive set theory, and that it
inevitably leads to contradiction.  He gives the definition of the
Russell class as an example of why the naive assumption is false.

He gives two reasons why the naive approach is in error.  The first is
the same consideration as above: the assumption that we can collect
arbitrary objects together is safe as long as it is made in a
``circle-free'' context, that is, as long as we do not admit (or at
least somehow restrict) the formation of collections whose elements
are collections in their turn.  Definitions of collections like the
universe, which must contain themselves, and of other collections
whose definitions depend in more complicated ways on their own
presence, are ``circular definitions'' which can sometimes be
satisfied and sometimes cannot be satisfied.

The second reason he gives is that the naive approach fails to
distinguish between ``sets'' and ``classes''.  In the definition of a
set, there are two components: the specification of a collection of
objects and the association of a unique single object with that
collection.  What Finsler suggests (quite in line with modern
thinking) is that it is always admissable to discuss a collection of
objects (a ``class'') of objects of our universe of discourse defined
in whatever manner (as long as it is defined precisely) but that it is
a further step to assert that there is a unique object in our universe
of discourse (a ``set'') which we can identify with this collection,
and which can itself participate in further collections.  Finsler says
``$\ldots$sets are things which {\em correspond\/} to collections, in
so far as this is consistent.  It is in general better not to refer to
collections as `things'\,''.  We can then say that the error of the
naive approach is that it confuses class and set: we can define a
collection of sets (or other objects) freely (this much of the naive
approach is sound) but we cannot then freely assume that the
collection (a class) is associated with an object in our universe of
discourse (a set).  As in Russell's paradox, we can define a
collection of sets in such a way as to frustrate the possibility of
this collection being identified with any of the sets we have
available.

Finsler suggests (again in line with modern thinking) that we should
consider only ``pure sets'', those whose elements, elements of
elements, etc. include only sets.  He observes (in line with the
discussion of the previous paragraph) that we can freely construct
classes or ``systems'' of sets, but that we cannot assume without
restriction that these will be sets in their turn.  Systems are not
really objects of our universe of discourse, so we do not investigate
the possibility of forming systems of systems, or higher iterations.

\subsection{Finsler's axioms}

We now state Finsler's axioms.  We must at this point note Finsler's
preference for the converse of the usual membership relation: he
writes $x \beta y$ where we would usually write $y \in x$, and $\beta$
is primitive for him.

\begin{description}

\item ``We consider a system of things, which we call {\em sets\/}, and
a relation, which we symbolize by $\beta$. The exact and complete
description is achieved by means of the following axioms.

\item[I.  Axiom of Relation:] For arbitrary sets $M$ and $N$ it is
always uniquely determined whether $M$ possesses the relation $\beta$
to $N$ or not.

\item[II.  Axiom of Identity:]  Isomorphic sets are identical.

\item[III.  Axiom of Completeness:] The sets form a system of things
which, by strict adherence to the axioms I and II, is no longer
capable of extension.

That is, it is not possible to adjoin further things in such a way
that the axioms I and II are satisfied.''

\end{description}

Finsler's axiom I asserts that sets are well-defined collections:
everything is either in a given set or not in it.

Finsler's axiom II (which appears in various forms) needs explanation.
To understand its place in his theory, it is sufficient to understand
it as saying that the identity of a set is determined by the
isomorphism type of the relation $\beta$ restricted to the transitive
closure of the set under the relation.  One of the consequences of
axiom II is extensionality: sets with the same elements are the same.
But it is stronger than that: it is easy to see, for example, that any
two sets which are their own sole elements will have isomorphic
transitive closures and so will be the same.  Axiom II is an
``anti-foundation axiom'' in the sense of Aczel.  The precise form of
isomorphism needed is a subject for technical adjustment, and Finsler
did have occasion to change it.

The reviewer (and many earlier workers) have serious difficulties with
axiom III.  We have struggled to understand what is meant by this
axiom and how Finsler could draw his stated conclusions from it, and
we have been unable to come up with a coherent explanation.

Finsler claims that the following are consequences of axiom III:

\begin{description}

\item[Proposition 6:] For any well-defined class of sets, there exists
a set which contains each member of the class, if and only if the
assumption that such a set exists does not contradict axiom I.

Finsler asserts that proposition 6 is a consequence of axiom III but
not equivalent to it; he points out that a set which is its own sole
element is not provided by proposition 6 alone.

\item[Proposition 7:] An arbitrarily defined set $M$ exists$\ldots$if
the assumption that such a set $M$ exists does not contradict the
first two axioms.

Finsler asserts that proposition 7 is equivalent to axiom III.

\end{description}

An objection to Finsler's theory expressed by Specker is that
proposition 7 can clearly be seen to be false if ``arbitrarily
defined'' really means what it says.  It is possible to have the
universe, the set of all $x$ such that $x=x$, in a model of Finsler's
axioms I and II (consider a system considering only of a set and its
own sole element).  In fact, Finsler claims that the existence of the
universe follows from his axioms.  It is also possible to have the set
of all $x$ such that $x$ is an element of some set $y$ and $x$ is not
an element of $x$: consider the system of all sets of hereditarily
finite sets--the set of all hereditarily finite sets is the collection
of all elements which are not elements of themselves in this system.
No model of axioms I and II can contain both of these sets, so
proposition 7, naively understood, cannot be true in any system.
Finsler and the editors of this volume articulate objections to
Specker's counterexample, but I do not understand the objections.
They can only be understood if there is some kind of restriction to be
placed on acceptable definitions of sets $M$ which has not been made
clear.

We must observe in this context that the example the editors give on
p. 97, the set $W$ = $\{x \mid x=0$ and $x=y$ for every set $y\}$ does
not have the properties they ascribe to it (they claim that it is
paradoxical).  $W$ is the empty set, pure and simple (if it exists);
there can be an $x$ with the properties ascribed to an element of $W$
only if 0 is the only set (we assume that the authors follow the usual
convention that 0 is the empty set), in which case $W$ could not
exist, but nothing precludes $W$ existing, being empty, and there
being other sets than the empty set (which we think is the actual
situation).  Note that the definition of $W$ clearly succeeds in
defining a set in Zermelo set theory, since it is a subset of the
natural numbers defined by a first-order formula!

\subsection{Model theory of Finsler's axioms}

Finsler seems to believe (reading the preamble to his axioms) that
they are categorical.  They are not.  We state some definitions of our own:

\begin{description}

\item[Definition:] A {\em Finsler premodel\/} is a (class) relation
$R$ the union of whose domain and range is a class $X$ with the
property that for all $x$ and $y$ in $X$, if the restriction of $R$ to
the transitive closure of $x$ under $R$ and the restriction of $R$ to
the transitive closure of $y$ under $R$ are isomorphic relations (in a
suitable sense whose details do not matter here) then $x=y$.

A Finsler premodel is a model of axioms I and II.

\item[Theorem (Baer):] If the union $X$ of the domain and range of the
defining relation $R$ of a Finsler premodel is not the universal
class, then the Finsler premodel can be properly extended.

\item[Proof of Baer's Theorem:] Choose an object $y$ not an element of
$X$ and extend $R$ to a relation $R'$ by stipulating that $y R' x$
holds precisely if $x R x$ does not hold, for each $x$ in $X$.  If the
restriction of $R'$ to the transitive closure of $y$ were isomorphic
to the restriction of $R$ to any $x$ in $X$, that element $x$ would be
the Russell class in the original Finsler premodel, which is
impossible.

\end{description}

Finsler (and the editors) object to Baer's assumption that he can
introduce a new object.  This is fine; Baer's theorem is understood by
us as well as by them to require that a maximal Finsler premodel must
have the union of the domain and range of its membership relation be
the universal class.  It is quite easy to believe that there are
Finsler premodels in which the entire universe participates; but these
fail to be extendible only in the quite trivial sense that there is no
new object which can be adjoined to them.
\begin{description}
\item[Theorem (ours):] Given a maximal Finsler premodel in which the
universal set does exist, it is possible to construct a maximal
Finsler premodel in which the universe does not exist.

\item[Proof:] Take the object $v$ such that $v R x$ holds for all $x$,
and modify the definition of $R$ to $R'$: ``$x R' y$ iff either $x
\neq v$ and $x R y$ or $x = v$ and $R$ restricted to the transitive
closure of $y$ with respect to $R$ is well-founded''.  This amounts to
replacing the extension of the erstwhile universal set with the
extension of the erstwhile class of well-founded sets (which cannot be
a set).  It is straightforward to check that this is still a maximal
Finsler premodel but no longer has a universal set.

\end{description}

The last theorem is sufficient to establish that Propositions 6 and 7
do {\em not\/} follow from axioms I-III.  This establishes that
Finsler's treatment of set theory is basically incoherent.

Having given our own development of the model theory of Finsler's
axioms I-III, we turn to Finsler's own.

Finlser proposes to construct a model of his theory in the following
way: given any collection (however large) of models of axioms I-II, we
can take their union, identifying those elements of distinct models
which have isomorphic transitive closures under the local membership
relation.  He proposes to construct a maximal model of axioms I-III by
carrying this out for {\em all\/} models of axioms I-II.  An objection
which he notes to this is that he is not restricting his models of
axioms I-II to be sets, so he is considering the union of a system of
proper classes!  I will allow him this construction for the sake of
argument, however.  The punchline is that this construction can
``almost'' be carried out in standard set theory with proper classes
(as, for example, by Aczel), and it gives what we would call a maximal
Finsler premodel with rather nice properties, but without a universal
set (and so not satisfying his propositions 6 or 7).  The object
produced by this construction is an extension of the model of {\em
ZFC-\/} with Finsler's anti-foundation axiom that Aczel constructs; it
is an extension because the class of relations that are used is larger.
The reason I say ``almost'' above is that there is a technical
problem: the construction of this model appears to require
superclasses -- it will be larger than the original model of the
theory of sets and classes one started with.

\subsection{Finsler's mistake}

We discuss the error in reasoning behind Finsler's axiom III.  Finsler
states that the criterion for mathematical existence is consistency,
which is a reasonable criterion for a Platonist.  On p. 169:
``consistent things can always be taken to exist; in pure mathematics,
existence means nothing more than freedom from contradiction''.  We
agree with this criterion, on the whole; we agree that every structure
that we can describe consistently is a legitimate object of
mathematical study and must be taken to exist from a Platonist
standpoint.  But we do not agree that it follows from this that every
consistently satisfiable set definition can be satisfied at once
(Finsler's axiom III).  Moreover, we believe that we can identify the
mistake.  The set of all sets is a satisfiable object; we can present
a model with a relation $\beta_1$ of converse membership in which this
object is found.  The set of all elements which are not elements of
themselves is also a satisfiable object; we can present a model with a
relation $\beta_2$ of converse membership in which this object is
found.  The illicit further step which Finsler takes is to think that
we can identify the membership relations on these two structures.  The
reason that we cannot identify them is that each of these set
definitions places restrictions on what other sets there can be in the
model which includes it.  Each of these definitions has consequences
for the kinds of sets there must be which preclude the satisfaction of
the other definition.  Each ``set'' must be possible to discover in
the Platonist universe, but they will not be found in the same set
theory.  Another point against the intelligibility of axiom III is
that we can at least entertain doubts that the totality of all
consistent definitions is a consistent totality.  

A corrective to the reasoning behind axiom III would be to say that
all those sets can be taken to exist simultaneously whose definitions
do not depend on the question of what sets exist in general; it is
reasonable to suppose that such definitions would be compatible with
one another.  This is a vague idea, but it is made much more precise
in the motivation for the set theory of Ackermann which I will quote
later. In Ackermann's theory, there are sets and classes--a class of
sets is defined by any condition, and a class all of whose elements
are sets is itself a set if it can be defined by a condition which
{\em does not depend on the notion of sethood\/}.  Curiously, the set
theory of Ackermann can be understood as an implementation of
Finsler's notion of ``circle-free'' sets, our next topic!

\subsection{Circle-free sets and the set theory of Ackermann}

We now discuss Finsler's concept of ``circle-free'' sets.  Certainly a
set which is included in its own transitive closure is ``circular'';
all circle-free sets must be well-founded.  Finsler assumes further
that any set whose definition leads to paradox is ``circular'' (in the
sense of being ``circularly defined'').  The set of all well-founded
sets, for example, is a paradoxical object; since it is a collection
of sets none of which are contained in their own transitive closures,
it should not appear in its own transitive closure -- but then it
should belong to itself, and so to its own transitive closure!

Finsler is led to the conclusion that the correctly defined class of
circle-free sets is itself a circular set (in the sense of
``circularly defined'').

\begin{description}

\item[Definition:] A set $M$ which is said to be {\em circle-free\/}
if $M$ together with every set in the transitive closure of $M$ is
independent of the concept ``circle-free''.

\end{description}

This definition is somewhat paraphrased from the original paper.

Sets not in their own transitive closures which are not circle-free
are said to be ``circular''.

Finsler asserts the following propositions about circle-free sets:

\begin{description}

\item[Proposition 9:] If $M$ is circle-free then every set in the
transitive closure of $M$ is circle-free and distinct from $M$.

\item[Proposition 10:] Every well-defined class of circle-free sets
forms a set.  This can be either cirlce-free or circular, but it is
disinct from every set which is contained in it.

\item[Proposition 12:] A well-defined class of circle-free sets forms
a circle-free set iff it is independent of the concept
``circle-free'', i.e., iff it can be so defined that the definition
always yields the same class regardless of which sets are classified
as being circle-free.

\item[Proposition 16:] Each well-defined class of elements of a
circle-free set $M$ forms a circle-free subsetof $M$.

The editors seem to believe that the axiom that every subclass of a
circle-free set is a set is not found in Finsler (p. 101)!  It is
found, as we indicate here, and essential use of it is made (as in
Ackermann's later work) to prove the existence of power sets.

\end{description}

It should be noted that Finsler merely asserts the basic propositions
about circle-free sets; he does not attempt to deduce them from axioms
I-III.

Compare this to a subtheory of the set theory of Ackermann, which we
present as a theory with sets and classes.  We warn the reader that
``set'' does not coincide with ``element'' in Ackermann's system; it
is a theorem of Ackermann's system that there are non-sets which are
elements of classes.

\begin{description}

\item[1.]  Any condition whatever on sets defines a class (not
necessarily a set).  Note that we do {\em not\/} require that all
elements of classes are sets.

\item[3.]  Any element of a set is a set.

\item[4.]  Any subclass of a set is a set.

\item[5.]  Any class all of whose elements are sets is a set if the
class can be defined without reference to the property of being a set
or to any non-set parameter.

\end{description}

All of the axioms we give are found in Finsler, if we interpret the
class of ``sets'' as Finsler's class of circle-free sets. It is known
that Ackermann's theory is essentially equivalent to Zermelo-Frankel
set theory (with all sets of {\em ZFC\/} being understood to be sets
in Ackermann's sense, and classes of sets being understood to be
proper classes).  The mathematically interesting work in Finsler's
paper is all derived from these axioms (not from Axioms I-III), and in
essentially the same way that Ackermann later derived the same
propositions from his axioms.  The axiom of foundation is sometimes
added to Ackermann's axioms (as for instance by Levy), and certainly
holds for circle-free sets.  The axiom of infinity is provable from
these axioms; an analogue of Ackermann's proof was given earlier by
Finsler (in the paper ``The existence of the natural numbers and the
continuum'').

Just for fun, we prove that there must be an ordinal in Ackermann's
system which contains a non-set ordinal as an element.  The class of
set ordinals exists by Ackermann's axioms; it is itself an ordinal,
which we will call $\Omega$, and cannot be a set.  If all elements of
ordinals were sets, then we could use the predicate ``is an element of
some ordinal'', which does not mention sethood, to define $\Omega$,
and, since all of its elements are sets, it would then be a set by
axiom 5, which is absurd.  Thus, there must be an ordinal which has
non-set elements, which implies in particular that $\Omega+1$ must
exist.

Finsler asserts that the theory of circle-free sets is adequate for
applications (e.g., on pp. 49, 150).  This is fortunate; for this
means that Finsler's work in applied areas can be understood as work
in a correct theory.  One can then leave aside his claims that the
universal set, the set of all singletons of ordinals, or a largest
ordinal must exist; these assertions are based on an unsound
intuition, whereas his proof of the axiom of infinity and the
existence of the continuum, for example, can be regarded as being
based on an intuition as sound as that on which the usual set theory
is based, given what we now know about the set theory of Ackermann.

It is worth describing the motivation for axiom 5 of Ackermann's
system as reported by Azriel Levy: ``Let us consider the sets to be
the ``real'' objects of set theory.  Not all the sets are given at
once when one starts to handle set theory--the sets are to be thought
of as obtained in some constructive process.  Thus at no moment during
this process can one consider the predicate $[$of sethood$]$ as a
``well-defined''predicate, since the process of constructing the sets
still goes on and it is not yet determined whether a given class $X$
will eventually be constructed as a set or not.  As a consequence, a
condition$\ldots$ can be regarded as ``well-defined'' only if it
avoids using the predicate $[$of sethood$]$.  Also, parameters are
allowed in such a definition only to the extent that they stand for
``well-defined'' objects, i.e. sets.''  Finsler would not agree with
the idea that the sets are ``constructed'' in quite the sense that
Ackermann thinks they are, but there seems to be some relation between
Ackermann's idea that the sethood predicate is not ``well-defined''
and Finsler's assertion that it is ``circular''.  It is interesting to
see how quite different intuitions on the surface can lead to an
axiomatization of basically the same form.

\section{Conclusion}

In conclusion, I find Finsler's papers extremely interesting.
Finsler's intuition is clearly profound, though the notion behind
Axiom III does not work out.  A presentation of Finsler's papers is a
contribution at least to historical scholarship about set theory.  The
editorial apparatus with which the papers are encumbered has serious
deficiencies.  The papers could have used the services of editors who
understood the relevant mathematics better themselves.  This is
painfully evident in the treatment of Finsler's axioms, which cannot
be defended as the editors have attempted here, and it is unfortunate
that the editors were not able to bring out the actual nature of
Finsler's derivation of a ``formally undecidable proposition''.

Aczel, Peter, {\em Non-well-founded sets\/}, CSLI, Stanford, 1988.

Booth, David and Ziegler, Renatus, {\em Finsler Set Theory:  Platonism and Circularity\/}, Birk\"auser-Verlag, Basel, 1996.

Holmes, M. Randall, ``Review of ``Finsler Set Theory:
Platonism and Circularity'', David Booth and Renatus Ziegler, eds.'', unpublished, available at {\tt
http://math.idbsu.edu/faculty/holmes.html}

Levy, Azriel, ``The role of classes in set theory'', in M\"uller,
Gert, ed., {\em Sets and Classes\/}, North Holland, Amsterdam, 1976.
See pp. 207-212 on Ackermann's theory.

\end{document}










