\documentclass{article}

\usepackage{amssymb}

\title{The symmetric theory of sets and classes with a stronger symmetry condition has a model}

\author{M. Randall Holmes}

\date{complete version more or less (alpha release):  6/5/2020}

\begin{document}

\maketitle

\section{Introduction}

This paper continues a recent paper of the  author in which a theory of sets and classes was defined with a criterion for sethood of classes which
caused the universe of sets to satisfy Quine's New Foundations.  In this paper, we describe a similar theory of sets and classes with a stronger (more restrictive) criterion based on symmetry determining sethood of classes, under which the universe of sets satisfies a fragment of NF which we describe, and which has a model which we describe.

\section{The theory of sets and classes}

The predicative theory of sets and classes is a first-order theory with equality and membership as primitive predicates.

We state axioms and basic definitions.

\begin{description}

\item[definition of class:]  All objects of the theory are called classes.

\item[axiom of extensionality:]  We assert $(\forall xy.x=y \leftrightarrow (\forall z:z \in x \leftrightarrow z \in y))$ as an axiom.  Classes with the same elements are equal.

\item[definition of set:]  We define ${\tt set}(x)$, read ``$x$ is a set" as $(\exists y:x \in y)$:  elements are sets.

\item[axiom scheme of class comprehension:]  For each formula $\phi$ in which $A$ does not appear, we provide the universal closure of $(\exists A:(\forall x:{\tt set}(x) \rightarrow (x \in A \leftrightarrow \phi)))$ as an axiom.

\item[definition of set builder notation:]  We define $\{x \in V:\phi\}$ as the unique $A$ such that $(\forall x:{\tt set}(x) \rightarrow (x \in A \leftrightarrow \phi))$.  
There is at least one such $A$ by comprehension and at most one by extensionality.  We define $V$ as $\{x \in V:x=x\}$ (the class of all sets).

\item[definitions of pairs:]  For sets $a,b$ we define $\{a,b\}$ as $\{x \in V:x=a \vee x=b\}$.  We define $\{a\}$ as $\{a,a\}$.  We define $(a,b)$ as
$\{\{a\},\{a,b\}\}$ as usual.  The standard theorem $(a,b) = (c,d) \rightarrow a=c \wedge b=d$ has its usual proof.

\item[axiom of elementary sets:]  The empty set $\emptyset = \{x \in V:x \neq x\}$ is a set.  For each set $a$, $\{a\}$ is a set.  For any sets $a,b$, $a \cup b = \{x \in V:x \in a \vee x \in b\}$ is a set.

\item[definition of union of a class:]  $\bigcup C$ is defined as $\{x \in V:(\exists y:x \in y \wedge y \in V)\}$.

\end{description}

This completes the formulation of the elementary theory of sets and classes.  It should be noted that in the previous paper we had to restrict the scheme of comprehension to formulas in which every quantifier was bounded in $V$.  This restriction is not needed (or wanted) in the weaker theory presented here.  The stronger theory had just the axiom of pairing to support relations and functions, whereas in this theory we need to ability to construct finite sets of all sizes.

\section{Relations, functions, and symmetry}

We present a series of definitions which should seem familiar.

\begin{description}

\item[definition of finiteness:]   A class $C$ is said to be finite-inductive iff $\emptyset \in C$ and $(\forall xy:x \in C \rightarrow x \cup \{y\} \in C)$.  A set $x$ is said to be finite iff it belongs to every finite-inductive class.  The axiom of elementary sets tells us that $V$ is a finite-inductive class.

\item[definition of relation:]  A relation is a set of ordered pairs.  We write $x \,R\, y$ for $(x,y) \in R$.  We define ${\tt dom}[R]$ as $\{x \in V:(\exists y:x\,R\,y)\}$.  We write $R^{-1}$ for $\{(x,y) \in V:y\,R\,x\}$.  We write ${\tt rng}[R]$ for ${\tt dom}(R^{-1})$.  For any relation $R$ and set $A$, we define $R``A$ as $\{y:(\exists x \in A:x \, R\,y)\}$.

\item[definition of function:]  A relation $f$ is a function iff $(\forall xyz:x \,f\,y \wedge x\,f\,z \rightarrow y=z)$.   We define $f(x)$ as $\bigcup f``\{x\}$.   We say that $f$ is a {\em permutation\/} iff $f$ is a function, $f^{-1}$ is a function, and ${\tt dom}[f] = {\tt rng}[f]=V$.

\item[definition of $j$ operation:]  For any permutation $f$, we define $j[f]$ as $\{(x,y):y=f``x\}$.  $j[f]$ is necessarily a function.  It might fail to be a function of universal domain.  If it is a function of universal domain, it is also a permutation.  We define $j^0[f]$ as $f$ in all cases and $j^{n+1}[f]$ as
$j[j^n[f]]$, if this is a permutation,  for each natural number $n$ (of the metatheory:  we do not have natural numbers in our theory so far).

\item[axiom of symmetric set comprehension:]  We assert as an axiom that a class $C$ is a set iff there is a finite set $S$ (called a {\em support\/} for $C$)
such that for any permutation $f$ such that $j[f]$ is a permutation and $(\forall s \in S:j[f](s)=s)$ we have  $j[f]``(C) = C$.

This completes the formulation of the theory of symmetric  sets and classes which is the subject of this paper.
We have two further aims in the paper:  we will define a fragment of Quine's set theory New Foundations and show that it is satisfied by the sets of any model of this theory, and then exhibit a model of this theory constructed in the usual set theory ZFC (we do not need very much ZFC to construct this model:  the theory itself is no stronger than third order arithmetic).

\section{The theory NFSI$_2$}

The theory NFSI$_2$ is a first order theory with equality and membership as primitive notions.  General objects of the theory are called sets.

\begin{description}

\item[axiom of extensionality:]  The first axiom is $(\forall xy.x=y \leftrightarrow (\forall z:z \in x \leftrightarrow z \in y))$.  Sets with the same elements are equal.

\end{description}

Before stating the second axiom (scheme), we extend our language by adorning some variables with subscripts:  for each natural number $i$ we
provide a countable supply of variables with subscript $i$, and further we have a countable supply of variables with no subscript.

\begin{description}

\item[definition of well-typed formula:]  A well-typed atomic formula is a formula which either contains an unsubscripted variable or is of one of the forms $x_i = y_i$
or $x_i \in y_{i+1}$.  A well-typed formula is one in which all bound variables are subscripted and all atomic formulas are well-typed.

\item[axiom scheme of stratified comprehension]  The axiom scheme of stratified comprehension provides that for each well-typed formula $\phi$ in which $A$ does not occur,
the universal closure of $(\exists A:(\forall x_i:x_i \in A \leftrightarrow \phi)$ is an axiom.  As stated so far, we have presented the full theory NF.
The restriction which gives NFSI$_2$ is that the subscript $i$ will be 0 or 1:  in fact, we can require that $i=1$: uniformly raising all
subscripts when $i=0$ obtains an equivalent instance of the comprehension axiom with $i=1$.

\end{description}

It is important to notice that the subscripts on variables have no effect on their range of variation or their logic:  the only effect is on the shape of comprehension axioms.  Moreover, subscripts can be eliminated from these axioms by renaming bound variables in the usual way to obtain theorems not using subscripted variables.

\begin{description}

\item[Theorem 1:]  The universe of sets of a model of our theory of symmetric sets and classes satisfies the axioms of NFSI$_2$.  


\item[Proof of Theorem 1:]
That the universe of sets
satisfies extensionality is obvious by the extensionality axiom for classes.

Let $\{x_1 :\phi\}$ be a set abstract in the language of NFSI$_2$ with $\phi$ well-typed (in fact, we want all variables in $\phi$ subscripted).  We convert it to a set abstract of the theory of sets and classes
by bounding $x_i$ and each variable bound by a quantifier in $V$:  these variables are intended to range only over sets.  We argue that if any
parameters in $\{x_1:\phi\}$ are sets and so have supports, we can compute a support for $\{x_1:\phi\}$, establishing that it is a set.

We want to argue for each permutation $f$ that if $j[f]$ is a permutation, that $j^2[f]$ is also a permutation.  What needs to be shown is that
for any set $a$, $j[f]``a$ is a set (and so can be read $j^2[f](a)$).  Let $f$ be a permutation with $j[f]$ a permutation.  Let $a$ be a set with support $S$.
Our aim is to show that $j[f]``a$ is a set by showing that it has support $j[f]``S$.  Suppose that $g$ is a permutation, $j[g]$ is a permutation,
and $j[g]$ fixes each element of $j[f]``S$, that is, $j[g \circ f](s) = j[f](s)$ for each $s \in S$.  Thus $j[f^{-1} \circ g \circ f](s) = s$ for
each $s \in S$, so $j[f^{-1} \circ g \circ f]``a = a$, whence $j[g] ``(j[f]``a) = j[f]``a$ which is what was to be shown.

This means that in fact if $j[f]$ is a permutation, so is $j^n[f]$ for every concrete $n$.  This is a very useful result:  one thing to note is that we don't
even really have a way to say this in our set theoretical language (but we do not need to).

Now observe that $x = y \leftrightarrow j^i[f](x) = j^i[f](y)$ and $x \in y \leftrightarrow j^i[f](x) \in j^{i+1}[f](y)$ allow us to assert that
each well-typed formula in which all variables are subscripted is unaffected in truth value if each $z_i$ appearing in it is replaced with
$j^i[f](z_i)$.  We can further again replace each $j^i[f](z_i)$ with $z_i$ if $z_i$ is bounded by a quantifier, because $j^i[f]$ is a permutation of the universe of sets.   If $\phi(x_1,a^1_{\tau_1},\ldots,a^n_{\tau_n})$ is a formula with $x_1$ and the $a^i_{\tau_i}$'s as its only free variables,  then it
has the same truth value as $\phi(j^i[f](x_1),j^{\tau_1}[f](a^1_{\tau_1}),\ldots,j^{\tau_n}[f](a^n_{\tau_n}))$, so
$j[f]``(\{x_1:\phi(x_i,a^1_{\tau_1},\ldots,a^n_{\tau_n})\}) = \{x_i:\phi(x_i,j^{\tau_1}[f](a^1_{\tau_1}),\ldots,j^{\tau_n}[f](a^n_{\tau_n}))\}$, so
if we can enforce the conditions $j^{\tau_i}[f](a^i_{\tau_i})=a_{\tau_i}$ for each $i$ we can enforce $j[f]``(\{x_1:\phi(x_i,a^1_{\tau_1},\ldots,a^n_{\tau_n})\}) = \{x_1:\phi(x_i,a^1_{\tau_1},\ldots,a^n_{\tau_n})\}$.

At this point we need a Lemma.

\begin{description}

\item[Lemma:]  For any finite set $A$, there is a support $S$ which is a support for each element of $A$.

\item[Proof:]  This is proved by induction on finite sets.  Consider the class of sets for which this assertion is true.  Clearly $\emptyset$ belongs to this class.  If $x$ belongs to this class, then there is a support $S_1$ which is a support of each element of $S$, and a support $S_2$ which is a support of $y$:  it is then clear that $S_1 \cup S_2$ is a support of each element of $x \cup \{y\}$.  It then follows that every finite set belongs to this class by the definition of the class of finite sets.  Note that it is crucially important here that our class comprehension axiom is impredicative, as the definition of support involves quantification over permutations which are not expected to be sets.

\end{description}

The conditions $j^{\tau_i}[f](a_{\tau_i})=a_{\tau_i}$  will be enforced by considering the supports of the $a^i_{\tau_i}$'s, and they do not have quite the right form, but a little tweaking does it.
We need to convert the condtions $j^{\tau_i}[f](a_{\tau_i})=a_{\tau_i}$ into a finite set of conditions with all the exponents equal to 1, then
collect the arguments of the new conditions to form the desired support. A condition with $\tau_i = 1$ requires no action.  A condition
$j^0[f](a^i_0) = a^i_0$ is equivalent to $j^1(f)(\{a^i_0\}) = \{a^i_0\}$, so replace such an item with its singleton.  Now consider a condition
$j^k[f](a^i_k)$ where $k>1$.  Suppose $a^i_k$ has support $S_i$.  By the definition of support, if $j^{k-1}[f]$ fixes each element $s$ of $S_i$ then
$j^k[f]$ will fix $a^i_k$.  Now we apply the Lemma.  Let $S^2_i$ be a common support for all elements of $S_i$.  It then follows that
if $j^{k-2}[f]$ fixes each element of  $S^2_i$, each condition  $j^{k-1}[f](s) = s$ holds for $s \in S_i$, and so $j^k[f]$ will fix $a^i_k$.  Applying
the Lemma repeatedly will give $S^{n+1}_i$ a common support for all elements of $S^n_i$ and if $j^{k-n}[f]$ fixes each element of  $S^n_i$, then the original condition holds:  this need only be iterated until $k-n=2$ (a concrete finite number of times taken from the formula).

A support for $\{x_1:\phi(x_i,a^1_{\tau_1},\ldots,a^n_{\tau_n})\}$ is obtained as the union of a concrete finite collection of supports, one for each of the original conditions (individual ones of which obtained using the Lemma need not be concrete finite) which will be a finite set.

This completes the proof that the sets of our theory of sets and classes satisfy {\em NFSI$_2$\/}.

\end{description}

The theory NFSI$_2$ is an extension of two weak fragments of NF already studied:  these are NF$_3$ studied by Grishin and NFSI studied by Tupailo.

\section{A Model of Our Theory of Sets and Classes (and so of {\em NFSI$_2$\/})}

In this section, our metatheory is ZFC.  We do not use very much of it.  The theory of symmetric sets and classes is in fact mutually interpretable with third order arithmetic.

We build a structure {\bf M} containing codes for sets of the model, arbitary subsets of which implement classes of the model.

We define a number of notions by simultaneous recursion.

\begin{description}

\item[splitting function code:]  A splitting function code is a function $f$  whose domain is a set of set codes with the property that $f(s) = (\kappa,\lambda)$ where $\kappa, \lambda$ are elements of ${\mathbb N} \cup \{c\}$ and  $\kappa+\lambda$ is the formal cardinality of the set code $s$.  The domain of a splitting function code must either be $\{V\}$ or a finite partition of the set of all splitting function codes with a certain fixed domain $T$.

\item[set code:]  A set code is either a special code {\bf V} for the universe or a set of splitting function codes all with the same domain $S$, where $S$ is called the support of the set code.

\item[combinatorial coefficent:]  The combinatorial coefficient of  a pair $(\kappa,\lambda)$ is the minimum of $c$ and the number of subsets $B$ of a set $A$ of size $\kappa+\lambda$ which have $B$ of size $\kappa$ and $A \setminus B$ of size $\lambda$.

\item[formal cardinal:]  The formal cardinal of ${\bf V}$ is $c$.  The formal cardinal of a splitting function code is the product of the combinatorial coefficients of its range elements.  The formal cardinal of a set code other than ${\bf V}$ is the sum of the formal cardinals of its elements.

\item[extending the support of a code:]  To extend a code $X$ with support $S$ with another support $T$ is to construct a code $U$ consisting of all nonempty intersections of elements of $S$ with elements of $T$ and collect all splitting function codes $f$ with domain $U$ which have the property that there is a splitting functon code $g$ in $X$ such that the projectionwise sum of the values of $f$ on subsets
of any $s \in S$ is $g(s)$ to make a set code $Y$ with domain $U$ equivalent to $X$.  To extend {\bf V} to support $S$ is to collect all splitting function codes with domain $S$.

\item[equivalence of codes:]  Two codes are equivalent iff extending each of them with the support of the other gives the same code.  That equivalence of codes is an equivalence relation should be evident from simple infinite combinatorics.

\item[formal intersections of codes:]  The formal intersection of two set  codes is obtained by extending each of them with the support of the other and taking the intersection of the resulting set codes.  The formal complement of a code $x$ is the set of all splitting code functions with the same domain as the elements of $x$ which do not belong to $x$.

\item[formal membership of a code $x$ in a code $y$:]  A code $x$ is formally a member of $y$ iff there is a splitting function code $f$ belonging to $y$ such that for each $s$ in the domain of $f$,
$f(s)$ has first projection the formal cardinality of the formal intersection of $x$ and $s$ and second projection the formal cardinality of the formal intersection of $s$ and the formal complement of $x$.

\end{description} 

The structure {\bf M} can be supposed to be constructed in $\omega$ stages, at stage 0 having just ${\bf V}$ and at stage $n+1$ adding codes with each suitable finite domain taken
from the codes constructed at previous stages.

The cardinality of stage 0 is 1.  The cardinality of stage 1 is $c$:  there are $\omega$ splitting function codes with domain $\{V\}$, and $c$ possible subsets of these.

If the cardinality of the set of codes of rank $\leq n$ is $c$, then there are no more than $c$ suitable sets $S$ to serve as supports, for each set $S$ no more than $\omega$ possible splitting function codes,
and thus for each of no more than $c$ supports, no more than $c$ sets of splitting functions codes with domain that support, so no more than $c$ objects of rank $\leq n+1$.  The size of {\bf M} is $c$.

Let $\sim$ denote the relationship of equivalence.  The structure ${\bf M}/\sim$ is our actual model.

Proving things directly about this model is tricky.  We present a concrete structure which will help us with this process.

For any set $X$ with a finite partition $S$ where all elements of $S$ are either singleton sets or of cardinality $c$ we define a splitting function as a function $f$ from $S$ to $({\mathbb N} \cup \{c\})^2$ such that 
for any $s \in S$, $\pi_1(f(s)) + \pi_2(f(s)) = |s|$.  An $S$-cardinal over $X$ is determined by a splitting function $f$:   an element $Y$ of ${\cal P}^2(X)$ is an $S$-cardinal over $X$ iff
there is a splitting function $f$ such that $Y = \{Z:Z \subseteq X \wedge (\forall s \in S:  f(s) = (|s \cap Z|,|s \setminus Z|))\}$.

We define a sequence of sets $X_i$ with a not entirely misleading resemblance to the sequence of types in a typed theory of sets.  We choose $X_0$ to be a set of cardinality $c$ which does not
meet any ${\cal P}^n(X_0)$ for $n$ a positive natural number.  We define $X_1$ as a subset of ${\cal P}(X_0)$ which is constructed by partitioning $X_0$ into two sets of cardinality $c$ and
defining $X_1$ as the collection of all sets with finite symmetric difference from one of the two sets of the partition, or from $\emptyset$, or from $X_0$.

We then define $X_{i+2}$ for each $i$ as the collection of all intersections with $X_{i-1}$ with arbitrary  unions of $S$-cardinals over $X_i$, for all finite partitions $S$ of $X_i$ into singletons of elements of $X_i$ (if $i \geq 2$) and sets of cardinality 1 or $c$ belonging to $X_{i+1}$ (proper subsets of $X_{i+1}$ allowed only if $i+1 \geq 2$); we further require that any element of a partition which is infinite is either an $S$-cardinal or an infinite union of $S$-cardinals all of which are infinite sets.  Notice that a more general finite partition can be refined to one which meets the additional conditions on $S$-cardinals.

We allow ourselves after this point to refer to intersections with $X_{i+1}$ of $S$-cardinals over $X_i$ simply as $S$-cardinals over $X_i$.

   Notice that this means that all elements of $X_2$ are unions of $\{X_1\}$-cardinals and the supports of elements of $X_3$ do not include singletons of elements of $X_1$.  All information
about specific objects in $X_0$ or $X_1$ disappears from view.  It should be clear that each $\{X_1\}$-cardinal other than the trivial $\{X_1\}$ and $\{\emptyset\}$ is of size $c$.

An element $x$ of any $X_{i+2}$ is naturally associated with set codes:  for any $S$ such that $x$ is a union of $S$-cardinals, we assume by inductive hypothesis that we have coded all elements of
$S$ with codes with common domain and that we have provided a map $\chi$ sending each of these codes to the element of $S$ that it codes.  The code for $x$ will be the set of all
maps $f \circ \chi$ where $f$ is a splitting function associated with an $S$-cardinal included in $x$.  If we have a finite collection $\{x_1,\ldots,x_n\}$ of elements of $X_{i+2}$, with each $x_i$ a union of $S_i$-cardinals, we can construct a $T$ such that each $x_i$ is a union of $T$ cardinals:  $T$ is the set of nonempty intersections of collections of sets containing one element of each $S_i$.  This justifies
the use of codes with common domains above.  Of course this must bottom out at some point with  domains  $\{X_1\}$.  Note that we do not consider codes which have supports violating our technical restrictions:  it is easy to see that every code not meeting the technical restrictions is equivalent to one which does, by refinement of supports, as long as elements of $X_0$ and $X_1$ are not involved.

We define an equivalence relations $\approx$ on $\bigcup_{i \geq 2} X_i$.  $x \approx y$ holds iff $x$ and $y$ are associated with a common set code.  It should be clear that if $x$ and $y$ belong to the same
$X_{i+2}$ and are associated with equivalent codes in the sense defined above then they are in fact equal:  the concrete interpretation of the codes which we are given here assures us that $x$ and
$y$ are assigned exactly the same elements.

We argue that the cardinality of each $X_i$ is $c$, and each $X_i$ has elements which are either finite or of cardinality $c$.  The number of elements of an $S$-cardinal in $X_{i+2}$ will be the product of the sizes
of sets $[s]^{f(s)} \cap X_{i+1}$, where we define $[X^{(\kappa,\lambda)}$ as the set of all subsets $Y \subseteq X$ such that $|X \cap Y| = \kappa$ and $|X \setminus Y| = \lambda$.  This is immediately evident once we observe that finite unions of elements of $X_{i+2}$ are elements of $X_{i+2}$ (for which the result above that elements of a finite subset of $X_{i+2}$ can be taken to be unions of $S$-cardinals for the same $S$ intersected with $X_{i+1}$ is relevant).

The issue then is counting ways to split sets of size $c$ (since ways to split sets of size 1 present no difficulties).  Note that for each $i \geq 3$, singletons of elements of $X_{i-1}$ appear as elements of
$X_{i}$ using the splitting function sending $\{x\}$ to $(1,0)$ and $X_{i-2} \setminus \{x\}$ to $(0,c)$.  This means that when we count splits of sets of size $c$ belonging to $X_i$ for $i \geq 4$,
we can find $c$ splits of the forms $(n,c)$ and $(c,n)$ for positive $n$ by refining the support of the original set with the set of size $c$ in its support $S$ to include $n$ singleton subsets of that set
and using these to define the split sets -- and these sets of $n$ singletons can be chosen in $c$ different ways, so $c$ different splits are possible.  In the $(c,c)$ case, if there is any split, there
are $c$ distinct splits which can be achieved by moving a single element from one side of the split to the other, and there are $c$ elements which can be moved.

This means that the number of elements of any $S$-cardinal intersected with an $X_{i+1}$ is either 0 (in case there is a split of an infinite set which cannot be realized), 1, or $c$, and the case of $c$ only occurs if an infinite element of $S$ is nontrivially split.  The numbers of sets which are unions of $S$-cardinals intersected with $X_{i+2}$ is either finite or $c$, because the number of $S$-cardinals which are finite sets is finite, and the total number of $S$-cardinals is no more than countably infinite (no more than countably many possible splits of finitely many sets).  So there are no more than $c$ unions of $S$-cardinals;
a union of finitely many $S$-cardinals which are finite sets is of finite size;  any union above a certain finite size includes an $S$-cardinal with $c$ elements and so has $c$ elements.

Bad splitting cases do occur.  Infinite $S$-cardinals in $X_2$ cannot be split into disjoint sets belonging to $X_2$.  This is basically the only bad case, though effects of it cascade through the entire hierarchy, in a sense.  We analyze the problem of splitting an infinite element of $X_{i+2}$ into two infinite sets.  If it is a union of infinitely many $S$-cardinals, then we can split it into two unions
of infinitely many $S$-cardinals.  If it is the union of finitely many $S$-cardinals at least two of which are infinite sets, there is again no problem.  We consider the problem of splitting a single $S$-cardinal which is an infinite set.  Its support $S$ must include a nontrivially split infinite element.  If the split is not $(1,c)$ or $(c,1)$, and the set being split has singleton subsets, then we can choose a singleton set from
one side or the other of the split, and refine the $S$ cardinal to the collection of elements of that $S$-cardinal which include that singleton as a subset and those which do not, two infinite sets.
If the split is $(1,c)$ or $(c,1)$ and the set in $S$ being split itself admits an infinite split, then we can split the $S$ cardinal to those elements in which the single element distinguished by the $(1,c)$
or $(c,1)$ cut is on one side or the other of the split of the support element into two infinite sets.  We further observe that every infinite set in $X_3$ permits an infinite split,
because each such set has a support which must have an element which is an infinite union of $\{X_1\}$-cardinals in $X_2$ which are infinite sets.  The only possible splits of this domain element
are $(c,c)$ splits into two sets with finite symmetric difference from such infinite unions (there are a lot of them).   Choose one such split of the support element and use it to refine the support of the element of $X_3$.  We can choose one $\{X_1\}$-cardinal and partition the original $S$-cardinal into elements including that $\{X_1\}$-cardinal as a subset and elements which do not.

We have thus shown that every set in $X_i$ with $i \geq 4$ has actual cardinality exactly equal to the formal cardinality of any code associated with it, because all of its support elements
can be split as many ways as we expect.

This further means that every code has a representative in an $X_i$ for large enough $i$.  We show this by taking the index representative of ${\bf V}$ to be $X_3$.  When we have constructed index
representatives for all elements of the domain of a code in some $X_i$, we can construct the index representative for the entire code in $X_{i+1}$.  A code is represented at every level above
the level of its index representative, because we can start by representing ${\bf V}$ using any $X_{3+k}$.

For every $X_{2+k}$, the code for the set of $k$-fold singletons of elements of the Frege cardinal 2 is an example of a set code not represented in $X_{2+k}$.  So we do not ever have all codes at once.

What does happen in this structure is that we can compute formal membership between any two codes:  $x \in_{new} y$ can be defined as holding if there are $x',y'$ such that
$x \approx x' \in y' \approx y$.  The formal cardinal of a code is the cardinality in the usual sense of every representative of that code (this needs to be checked in some weird low indexed cases:  the reason that it works is that our restrictions on supports ensure that if there are any splits of a support element of a particular flavor there are exactly the expected number of splits).
The computation of the formal cardinal of a set is exactly realized in every implementation in high enough levels.  If two codes are equivalent, representatives of those codes at levels high enough to contain representatives of both (an important qualification) are equal.   The relation $\approx$ is an equivalence relation, because if $x \approx y$ and $y \approx z$ are witnessed by different codes
and $y$ is at the lowest indexed $X_i$ of the three, we can use a code based on the common refinement of the two supports of $y$ to witness $x \approx z$, whereas if
$x$ or $z$ is the lowest indexed of the three (wlog $x \in X_i$) we know that $y$ has a support made up of sets coded analogously to sets from $X_{i-1}$, and we can argue that $z$ has such a support as well.  It might be necessary to move $x,y,z$ up the same sufficient number of levels to ensure that all elements of the supports under consideration have index representatives:  then a desired
support of the translated version of $z$ can be constructed directly.

Elements which have exactly the same formal elements also have exactly the same actual elements and so are equal if they are at the same level:  they will be equivalent to the same
code in some sufficiently high level.  Codes represented by the same object are also clearly equivalent.

We have at this point checked the details required to see that ${\bf M} / \sim$ is the same up to isomorphism as $(\bigcup _{i \geq 2}X_i)/\approx$.

An element of the model which is a union of $S$-cardinals is clearly  symmetric with support $S$ in the desired sense.  We do need to establish (the hardest part of the proof) that
every subset of the model which is symmetric in the correct sense is actually a set of the model.



What we need more work to show is the converse:  that any set with support $S$ which is invariant in the suitable sense is actually of this kind.  What is needed to show this is that any infinite split
of an element of $S$ can be mapped to any other split of an element of $S$ by a setlike permutation.  To show this, it is enough to show that $V$ can be mapped to any set by a setlike map.


Note first that every set $X$  that we have constructed so far is $n$-symmetric, in the sense that there is an $n$ such that for any permutation $f$ such that $j^n[f]$ is a permutation, $j^n[f]$ fixes
$X$.  For ${\bf V}$ and $\emptyset$, $n$ is 1.  For any set with support $S$, if each element of $S$ is $n$-symmetric, the set with support $S$ is $(n+1)$-symmetric.


We fix an infinite set $A$.  We first describe a setlike injection into $A$.  $A$ will have support $S$.  Because $A$ is infinite, there is a splitting function coded by an element of $A$ which nontrivially splits
an infinite element $s \in S$.  By inductive hypothesis there is an injection $g$ from $V$ into $s$.  Choose a single element $A$ of $X$ which splits elements of $S$ as described by $f$, with the property
that either $s \cap A$ or $s \setminus A$ is a set $B$ of images under $g$ of non-singletons.  The injection $h$ from $V$ into $S$ which we define is executed by mapping $x \in V$ to the
result of transposing $g(\iota^k(x))$ for a suitable fixed $k$ with a fixed element $b$ of $B$.  $k$ should be chosen large enough that $A$ is something like $k$-symmetric:  the only effect
of $j^{k+1}(xy)$ on $h(x)$ will be to move it to $h(y)$.  This holds partly because $g$, by an inductive hypothesis, has the same characteristic.

For any set $C$, $h``C$ is a set:  its elements have full intersection with $A \setminus \{b\}$ or $A \cup \{b\}$ (as appropriate) and contain or fail to contain (as appropriate) a single element of $g``\iota^k``\{V\}$.  For any set $C$, $h^{-1}(C)$ is a set:  take the union of the set, remove all elements of $A \setminus \{b\}$ or $A \cup \{b\}$ as appropriate, possibly take the complement elementwise,
apply the inverse of $g$ elementwise, then take unions repeatedly.  So $h$ is setlike.

In order to verify that the conditions described above make sense, we need to verify that a transposition $(xy)$ is a setlike permutation in the sense local to {\bf M}, and that in fact $j^n[(xy)]$ is a permutation for 
each $x,y,n$.  It's easy to see that $j[(xy)]$ is a permutation:  extend the support of a set to include $\{x\}$ and $\{y\}$ and then replace each splitting function describing an element appropriately
so thst it includes $x$ if it formerly included $y$, and so forth.  We give a general argument for any permutation $f$ that if $f$ and $j[f]$ are permutations, so is $j^2[f]$:
to apply $j^2[f]$ to a set, one needs to apply $j[f]$ to each element of the support, which is possible by hypothesis, producing a code for $j^2[f]$.  One can then iterate this to show
that $j^n[f]$ is a permutation for each $n$.

Now we define the desired setlike bijection $h^*$ from $V$ to $A$ by applying $h$ to all elements of $V \setminus A$ and to each $h^n``(V \setminus A)$, and applying the
identity $A \setminus h``(V \setminus A)$ and each $h^n``(A \setminus h``(V \setminus  A))$.  This is a bijection from $V$ exactly onto $A$, but it is not clear that it is a setlike map,
because we cannot rely on countable union of setlike maps being a setlike map.  Let $Y$ be any subset of $A$.  For large enough $n$, the sets $Y \cap h^n``(V \setminus A)$
and  $Y \cap h^n``(A \setminus h``(V \setminus  A))$ are not moved by elementwise application of $h$ and so are not moved by elementwise applications of $h^*$, by symmetry.
$Y$ itself is symmetric to some fixed degree so is fixed by some $j^[m](h)$.  The action of $h$ on the other factors in this intersections is the same as the action of $j^{n+1}(h)$.  So if $n$ is taken large enough the application of $h$ or the identity to such a compartment can make no difference to whether a value is moved by $h^*$ into or out of $Y$.
Thus the computation of $h^*``Y$ requires the computation of images under $h$ or $h^{-1}$ on  finitely many sets (the compartments where elements might be mapped in or out of $Y$ by $h^*$), and since $h$ and the identity are both setlike, $h^*$ is setlike, so there is a setlike bijection from $V$ onto $A$.

So we have proved

\begin{description}

\item[Theorem 2:]  The theory of sets and classes described above is consistent.  Moreover, it has a model in which all sets are either finite or of cardinality $c$ from the standpoint of the metatheory.

\end{description}

\section{Conclusions and Further Observation}

This result is not progress toward a model of NF.  This theory uses impredicative class comprehension, which is inconsistent with the symmetric theory of the previous paper which yields NF.

The theory NFSI$_2$ and the symmetric theory of classes presented here are not mathematically entirely trivial.  Either theory can define Frege natural numbers and can define real numbers
as arbitrary unions of Frege natural numbers.  There is very likely a converse result:  we think that with care the theory can be interpreted in second order arithmetic.  To be exact, we think that it is reasonably clear that 
elements of the model we describe can be coded by real numbers and the  equality and membership relations of the model are expressible in terms of second order arithmetic, so that
second order arithmetic admits an interpretation of NFSI$_2$ + Infinity, but the details would be complicated.

The theory NFSI$_2$ extends Grishin's theory NF$_3$ (New Foundations with stratified set definitions using only three types) and Tupailo's odd theory NFSI (stratified set definitions
$\{x : \phi\}$ in which $x$ is of lowest type).  Tupailo's theory is very weak, but it is not a fragment of NF$_3$.  The symmetric set theory proves infinity;  it is not clear to us whether
NFSI$_2$ proves Infinity.

It is an interesting incidental remark that NF$_3$ has an internal theory of cardinality and a partial theory of functions, but the models of NF$_3$ generated by the symmetric theory
of sets and classes have very defective internal notions of cardinality and function.  Since there are no countable sets in these models, all orbits in any functions we manage to represent must
be finite.  Our suspicion is that there are lots of incomparable infinite cardinals in the models in the internal sense:  relative to setlike maps all infinite sets are the same size, and we can see externally that this is the cardinality of the continuum, but setlike maps do not have to be representable functions in any internal sense.  Further remark:  this has to be true.  If a bijection between $A$ and
$A \cup \{x\}$ could be represented, with $x \not\in a$, we could define a countable set (the orbit of $x$ under the bijection).  We have Infinity in the symmetric model but
we cannot have a Dedekind-infinite set.

Just for fun, we outline the development of mathematics up to the level of calculus in NFSI$_2$.

0 can be defined as $\{\emptyset\}$.  $\sigma(x_2)$ can be defined as $\{y_1 \cup \{z_0\}:y_1 \in x_2 \wedge z_0 \not\in y_1\}$.  So each natural number can be defined, and the predicate ``is a natural number"
is definable, but its extension is not granted as a set.   The predicate ``belongs to some natural number" (is finite) is definable:  ${\tt Fin} = \{x_1:(\forall I_2:\emptyset \in I_2 \wedge (\forall y_1 \in I_2:(\forall z_0 \not\in y_1:y_1 \cup \{z_0\} \in I_2)) \rightarrow x_1 \in I_2)\}$.
 For any set $X$, we can define $X^*$ as the set of all $x$ such that for some $y$, $y \in X$ and there is a natural number to which $x$ and $y$ both belong.
$X^* = \{x_1:(\exists y_1:(\exists z_2:(\forall I_3:0 \in I \wedge (\forall n_2 \in I_3:\sigma(n_2) \in I_3) \rightarrow z_2 \in I_3) \wedge y_1 \in X \wedge y_1 \in z_2 \wedge x_1 \in z_2))\}$.  This is interesting as an example of a native definition in NFSI$_2$ which does not work in NF$_3$.  This gives us an implementation of sets of natural numbers.

We can naturally represent a tuple of natural numbers $(x_1,\ldots,x_n)$ as the set which contains all $x_i$ element subsets of the Frege natural number $i$ for each $i \leq n$.  Of course, this is done concretely for each $n$.  We can define sets of $n$-tuples of natural numbers using a closure operation basically as above.

We can then present mathematical objects representing any desired functions and relations on the natural numbers.  Rational numbers can be represented as pairs of natural numbers, real numbers can be presented as Dedekind cuts or Cauchy sequences,  and (for example)
a continuous function $f$ from the reals to the reals can be coded as the set of pairs of rationals $(r,s)$ (coded as quadruples of natural numbers) such that $r < f(s)$.  We are ready for calculus!

The representations of things for elementary analysis are thus seen not to be notably cumbersome.  There is a philosophical outlook coded into NFSI$_2$ or our theory of sets and classes:  sets are formed by abstraction from properties of sets
looking to the level of elements, but not elements of elements, with finitely many given sets in hand.  The theory of sets and classes is stricter in saying that this is the {\em only\/} way sets are formed.  This is an outlook which avoids paradox and supports quite a lot of mathematics quite directly.  We certainly do not advocate such a view, but its formal possibility is interesting.

Here is an incidental puzzle:  in the model, we can define the set of complements of elements of a set $A$.  Is this actually definable in the symmetric theory of sets and classes or in NFSI$_2$?  Let $A$ be a set.  $\{b_1 : (\exists a_1 \in A:(\forall x_0:  x \in b_1 \leftrightarrow x_0 \not\in a_1))\}$ is provided as a set by NFSI$_2$ comprehension (in fact by NF$_3$ comprehension):  puzzle solved.





















\end{description}


\end{document}