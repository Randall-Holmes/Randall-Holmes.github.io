\documentclass[12pt]{article}

\title{Type Theories with General Variables}

\author{Thomas Forster and Randall Holmes}

\usepackage{amssymb}

\begin{document}

\maketitle

\tableofcontents

\subsection{Version Notes}

\begin{description}

\item[7/4/2024:]  Creating the file.

\item[7/10/2024:]  Some comments by Thomas and my replies.


\item[7/18/2024:]  Mathematical constructions and type hierarchy added -- starting to talk about typed language.

\item[7/21/2024:]  Revision of terminology and resolution of the axiom of comprehension into two axioms.  Proved the theorem that set abstracts are equivalent to typed set abstracts, all without introducing type theory.

\item[7/22/2024:]  The equivalence relation is called cohabitation.  Later revisions here and there.

\item[7/23/2024:]  Starting the discussion of typed theories of sets and older proposals.

\item[7/24/2024:]  Making a version to maintain on the web page (no visible link, just a URL for Thomas).

This version contains a statement of the untyped system which you (Thomas) originally communicated to me (in the very last subsection).  It is cribbed from a summary I made in one of my own versions:  I do not have the original document you sent me;  would like to see it if you can find it.

I have now extended this to cover the correspondences between sorted and unsorted theories.  Some of this needs firming up and organization.
I envision another section about the general topic of set theory without the axiom of pairing.

\item[7/25/2024:]  I think I have now recorded everything I know about these systems at least cursorily somewhere in this document.  It may need reorganization.
Thomas, do I have a correct description of the first proposal you made to me of ``TST(U) without sorts".  I do think your original version allowed for urelements [as the account I give here does], though I don't think this was your intention.  I wish I could track down the first mail you sent me about this :-)

I have added an untitled final section on set theory without pairing.  I do think this subject culminates in consideration of that issue;  it wasn't what Quine had in mind but it's clearly related.  One should reference Bolzano's remarks about this.

I know I need to find the reference to Boffa on typed properties.

\end{description}

\newpage

\section{Introduction}

The title of the paper refers to a proposal of Quine (in
\cite{quiettgv}) for a presentation of the typed theory of sets as an
unsorted theory.  A number of theories of this kind will be presented
in this paper.  We do not start with Quine's own proposal, for
rhetorical reasons (we want to make a point).

\subsection{TTGV:  a simple theory of types with general variables}

We present as part of our introduction a specific theory TTGV of this
kind\footnote{This theory was defined by Holmes as a modification of Forster's proposal which will be discussed later; at that time Holmes had no acquaintance with the earlier work of Quine or Resnik (\cite{resnikttgv}):  it is a slight (?) weakening of Resnik's theory, however, which confirms the naturalness of the approach.}.  We will be interested in other related theories, and we will
also be interested in different axiomatizations of this one; but we
want to present something concrete to illustrate what we are talking
about.

TTGV is a one-sorted first order theory with equality and membership as primitive relations.  It has some other primitive notions introduced later.

\begin{description}

\item[Definition (nonempty set):]  We say that $x$ is a {\em nonempty set\/} iff $$(\exists y:y \in x).$$ Nonempty sets are objects with elements.

\item[Remark:]  There is a simple identity criterion for nonempty sets expressed in the following axiom.

\item[Axiom of (weak) extensionality:]  $$(\forall xyz:z \in x \rightarrow ((\forall w:w \in x \leftrightarrow w \in y) \rightarrow x=y)).$$  Nonempty sets are the same if and only if they have the same elements.

\item[Remark:]  We prefer the axiom of weak extensionality because we think for full generality one should allow atoms, but in this particular context, as we will see, empty sets of different types will also be distinct.

\item[Remark:]  What follows is the central concept of this development, going right back to Quine.  In an unsorted theory, implementing the typed theory of sets, we do want to express the idea that two objects are of the same type.  The following definition does this neatly.

\item[Definition (cohabitation):]  We say that objects $x$ and $y$ cohabit (written $x \sim y$) iff $(\exists z:x \in z \wedge y \in z)$.  Objects cohabit if and only if there is a set to which both belong.

\item[Axiom of types:]  For each object $x$, there is an object $\tau(x)$ (which we call the type of $x$) such that $x \in \tau(x)$ and $(\forall y:y \in \tau(x) \leftrightarrow y \sim x)$.  The type of $x$ is the collection of things that cohabit with $x$ (and $x$ cohabits with $x$).  Objects of the form $\tau(x)$ are called types.

\item[Theorem:]  $\sim$ is an equivalence relation. 

\item[Proof:]  $x \sim x$ follows from $x \in \tau(x)$, part of the axiom of types.

$x \sim y\rightarrow y \sim x$ is a theorem of first order logic.

Notice that symmetry of $\sim$ implies that $x \in \tau(y)$ is equivalent to $y \in \tau(x)$.  Now suppose that $x \sim y$ and $y \sim z$.  It follows that
$x \in \tau(y)$ and $z \in \tau(y)$, so $x$ and $z$ are of the same type:  $x \sim z$ as desired.

\item[Remark:]  If when $x$ is an object and $\tau$ is a type, we read $x \in \tau$ as ``$x$ has type $\tau$", note that $x \sim y$ can be read as ``$x$ and $y$ have the same type".

\item[Type hierarchy:]  We define $\tau^1(x)$ as $\tau(x)$ and $\tau^{n+1}(x)$ as $\tau(\tau^n(x))$ for each concrete $n$.  Notice that the variable $n$ here is a variable of the metatheory which we cannot quantify over.

\item[Remark:]  We ought to say something about empty sets.

\item[Primitive notion (empty set):]  We introduce an primitive notion, for every nonempty set $a$ providing an object $\emptyset_{a}$, the empty set cohabiting with $a$.  Our general convention when we provide an object notation as subscript to a set notation is that the set denoted cohabits with the object denoted by the subscript:  the reader should note that such subscripts do not signal any difference of sort, just provide an indication of an equivalence class under $\sim$ which which the object can be found.

\item[Axiom of empty sets:]  For any nonempty sets $a,b$, $\emptyset_a \sim a$ and $a \sim b \leftrightarrow \emptyset_a = \emptyset_b$.  An object which is $\emptyset_a$ for some $a$ is called an empty set.  We further define $\emptyset_x$ for any object $x$ that cohabits with with a nonempty set $a$ as $\emptyset_a$.

\item[Definition (sethood):]  We say that $x$ is a set (written ${\tt set}(x)$) iff $$(\exists y:y \in x) \vee (\exists a:x = \emptyset_{a}), $$  that is, if $x$ is a nonempty set or $x$ is an empty set.

\item[Definition (classification of empty objects):]  An object with no elements which is not an empty set and cohabits with some set is called an {\em atom\/};  an object with no elements which does not cohabit with any set is called an {\em individual\/}.   TTGV does not prove the existence of either atoms or individuals, but they are an interesting possibility.

Asserting that there are no atoms would give the appropriate version of strong extensionality for this theory (there would still be many elementless empty sets, and might be many individuals).  Asserting that there are no individuals would have less familiar effects which we will discuss later.

\item[Definition (subset relation):]  We define $x \subseteq^+ y$ as $$(\exists z \in x) \wedge (\forall z:z \in x \rightarrow z \in y):$$ this is read ``$x$ is a nonempty subset of $y$".   We define $x \subseteq y$ as $$x \subseteq^+ y \vee ({\tt set}(y) \wedge x = \emptyset_y).$$

\item[Remark:]  We now provide the comprehension axiom of this theory, which has the simple motivation that any subcollection of a type ought to be a set, with an auxiliary axiom first.

\item[Axiom (subsets):]  $(\forall xy:x \in y \rightarrow y \sim \tau(x))$.  

This is equivalent to the assertion that sets which meet cohabit, that is, $(\forall xyz:x \in y \wedge x \in z \rightarrow y \sim z)$.  If $x \in y$ and $x \in z$, then by the axiom of subsets, $y \sim \tau(x)$ and $z \sim \tau(x)$, so $y \sim z$.  If sets which meet cohabit, then $x \in y$ implies that $y$ meets $\tau(x)$, so $y \sim \tau(x)$.

The axiom is called the axiom of subsets because it is equivalent to the assertion that a subset of a type must cohabit with that type.

A consequence of this is that $x \in y \rightarrow \emptyset_y = \emptyset_{\tau(x)}$:  subscripts on empty sets can be supposed to be types without loss of generality.

\item[Axiom (separation):]  For each set $B$ and each formula $\phi$ in which $A$ is not free, $(\exists A:A \in B \wedge (\forall x:x \in A \leftrightarrow y \in B\wedge \phi))$.  The object $A$, if it has elements, is unique and may be denoted by $\{x \in B:\phi\}$; if there is no $x$ such that $y \in B$ and $\phi$, we define $\{x \in \tau(y):\phi\}$ as
$\emptyset_{B}$.

\item[Remark:]  Note that if $\{x \in B:\phi\}$ has elements it cohabits with $B$ by the axiom of subsets; otherwise it cohabits with $B$ because it is $\emptyset_B$:  in either case, $\{x \in B:\phi\} \sim B$.

\item[Corollary (comprehension):]  For any object $y$ and formula $\phi$ in which $A$ does not occur,
$(\exists A:A \sim \tau(y) \wedge (\forall x:x \in A \leftrightarrow x \sim y \wedge \phi))$.  The object $\{x \in \tau(y):\phi\}$ witnesses this.  \footnote{This was our original formulation, and we provide it to make any surviving subsequent references to ``comprehension" make sense.   Comprehension is equivalent to the axiom of subsets and separation together.}

\item[Remark:]  One more axiom is required for technical reasons.  The exact axiom that is used can vary;  this one is very natural.   The reason that it is needed may not be so evident.

\item[Axiom of binary union:]  If $A \sim B$ are nonempty sets, $$(\exists C:(\forall x:x \in C \leftrightarrow x \in A \vee x \in B)).$$  The object $C$ is uniquely determined and may be denoted by $A \cup B$.  

The definition of union can be extended to empty sets:  $A \cup \emptyset_A = \emptyset_A \cup A = A$.

\end{description}

\subsubsection{Variants of the axiomatization}

\begin{description}

\item[Theorem:]  In the presence of the other axioms, the action of binary union is equivalent to the axiom of set union, which we state as $$(\forall A:\exists U:(\forall x:x \in U \leftrightarrow (\exists y:x \in y \wedge y \in A))).$$  The witness $U$ is uniquely determined if nonempty and can be written $\bigcup A$;  if it is empty and $A \in \tau^2(u)$, we can let $\bigcup A$ denote the empty set of the same type as $\tau(u)$ [a set of individuals is not expected to have a set union].

\item[Proof:]  Suppose binary union holds.  Let $A$ be a set.  If there are no $x,y$ such that $x \in y \in A$, then any elementless object will serve as the set union (strictly speaking, we would want $A \in \tau^2(u)$ for some $u$, and the empty set of the same type as $\tau(u)$ would be the desired set union:  a set of individuals is not expected to have a union).  If $x \in y \in A$ and $z \in w \in A$ then $y \sim w$, so $y \cup w$ exists and contains both $x$ and $z$, so $x \sim z$, so $\{z \in \tau(x):(\exists w:z \in w \wedge w \in A\}$ is the set union of $A$.

Suppose set union holds.  Let $x$ and $y$ be nonempty sets of the same type.  Then $$\{x,y\} = \{z \in \tau(x):z=x \vee z=y\}$$ exists by separation, and its set union is $x \cup y$.

\item[Theorem:]  The axiom of subsets is equivalent in the presence of the other axioms to the assertion that for each $x$, there is a set $B(x)$ such that $$(\forall y:y \in B(x)\leftrightarrow x \in y).$$

\item[Proof:]   Suppose that $B(x)$ exists, and suppose that the axioms other than subsets hold (only part of the axiom of types is used).  Then if $x \in y$, it follows that $y \in B(x)$ and of course
$\tau(x) \in B(x)$, whence $y \in \tau(x)$.

Existence of $B(x)$ follows from our usual axiom set, because $x \in y$ implies $y \sim \tau(x)$ by subsets,
so $y \in \tau^2(x)$, so $$B(x) = \{y \in \tau^2(x):x \in y\}.$$

\item[Theorem:]  The axiom of subsets is equivalent in the presence of the other axioms to the assertion that for each set $x$, there is a set ${\cal P}(x)$ (the power set of $x$) such that $$(\forall y:y \in {\cal P}(x)\leftrightarrow y \subseteq x).$$

\item[Proof:]  Suppose that power sets exist and $x \in y$.  Notice that every element of $y$ belongs to $\tau(x)$, so $y \subseteq \tau(x)$, so $y \in {\cal P}(\tau(x))$, and of course $\tau(x) \in {\cal P}(\tau(x))$, so $y \sim \tau(x)$.

The existence of power sets follows from our full axiom set.  Let $x$ be a set.  Suppose $y \subseteq x$.
If $x$ is an empty set, then $y=x=\emptyset_x$ and ${\cal P}(x) = \{y \in \tau(x):y=x\}$.  Suppose $z \in x$.  Then $x \sim \tau(z)$, so $x \in \tau^2(z)$, by the axiom of subsets.  We claim that ${\cal P}(x) = \{y \in \tau^2(z):y \subseteq x\}$.  To verify this, we need to show that if $y \subseteq x$, $y \in \tau^2(z)$.  If $y$ is empty, then $y = \emptyset_x \sim x$ so $y \in \tau^2(z)$.  If $w \in y$, then $w \in x$ (subset hypothesis) and thus $w \sim z$.  Also $y \sim \tau(w)$ by the axiom of subsets, but $\tau(w)=\tau(z)$, so $y \sim \tau(z)$, so $y \in \tau^2(z)$, completing what we need.

It is easy to adapt this argument to show that the axiom of subsets is equivalent to existence for each set $x$ of ${\cal P}^+(x)$  such that $$(\forall y:y \in {\cal P}(x)\leftrightarrow y \subseteq^+ x).$$

\item[Remark:]  The result just previous depends for its success on defining the subset relation very carefully.  The quite natural definition $$x \subseteq y \equiv_{\tt def} {\tt set}(x) \wedge {\tt set}(y) \wedge x \sim y \wedge (\forall z:z \in x \rightarrow z \in y)$$ would not serve here, though it is an adequate definition of the subset relation in the presence of the full axiom set.

\item[Discussion of strong extensionality:]  The assertion that there are no atoms gives a stronger version of extensionality in this theory which we will see gives the usual extensionality conditions in an interpreted type theory.  An object is an atom if it has no elements, is of the same type as a type, and is not an empty set.   The assertion that there are no atoms implies
that for any $x,y,u$, if $x$ and $y$ are both of the same type as $\tau(u)$ (and so belong to $\tau^2(u)$) and have the same extension, they are equal (they either have the same nonempty extension or are both equal to the empty set for type $\tau(u)$).  Thus the only failures of extensionality for objects of the same type are for objects which belong to no type $\tau^2(u)$, that is, individuals.  Empty sets of distinct type will still be coextensional with and distinct from individuals and
empty sets of different types\footnote{The theory of Resnik, which we were not aware of when we defined this theory, differs from ours only in two additional assumptions which we can state at this point in terms of concepts we have explained:  strong extensionality is assumed (objects with the same extension which cohabit with a nonempty set are equal), and an axiom provides that there are individuals (there is an object which does not cohabit with any nonempty set).  Resnik apparently believed that he could prove that all individuals are of the same type, so this may be taken to be his intention, but it does not follow from his axioms.}.

\end{description}

\newpage

\subsection{Basic set constructions and type hierarchy in TTGV}

In this section, we work on basic set theory machinery in this context.  We freely use notations defined in the axiomatics section.

\begin{description}

\item[Extending definitions to the empty sets:]  If $B$ is a set and $$(\forall x \in B:\neg \phi),$$ we define $\{x \in B:\phi\}$ as $\emptyset_{B}$.  If $A$ is a set of the same type as  $\emptyset_{\tau(x)}$, we define both $A \cup  \emptyset_{\tau(x)}$ and  $\emptyset_{\tau(x)}\cup A$ as $A$.



\item[Coercion of types:]  Suppose $x \in y$.   The axiom of subsets tells us that $y \sim \tau(x)$, so $y \in \tau^2(x)$. 
So $x \in y \rightarrow y \in \tau^2(x)$.

Further, suppose $x \in y$ and $y \in \tau^2(z)$.  It follows that $y \sim \tau(z)$, so  that  $y \cup \tau(z)$ exists, and both $x$ and $z$ belong to this set, so $x \sim z$, that is,
$x \in \tau(z)$.  We have established $x \in y \wedge y \in \tau^2(z) \rightarrow x \in \tau(z)$.

More generally, if $x \in y$ and $x \in \tau^n(u)$ then $y \in \tau^{n+1}(u)$ and if $x \in y$ and $y \in \tau^{n+1}(u)$ it follows
that $x \in \tau^n(u)$. 

Notice that it follows from type coercion that $\bigcup(\tau^{n+1}(x)) = \tau^n(x)$.  We adopt the convention that for any concrete integer $i$, we can define $\tau^{i-1}(x)$ as $\bigcup \tau^i(x)$ if this is a type.

We will show that the types $\tau^n(x)$ for a fixed $x$ are distinct (therefore disjoint) for distinct $n$.



\item[Theorem (following Russell):]  For each $x$, $\tau^2(x) \neq \tau(x)$.

\item[Proof:]  Let $x$ be any object.  Suppose that $\tau^2(x)=\tau(x)$. 

Define $R_{\tau(x)}$ as $\{y \in \tau(x):y \not\in y\}$.  $R_{\tau(x)} \in \tau^2(x)$, so $R_{\tau(x)} \in \tau(x)$ by hypothesis, so $R_{\tau(x)} \in R_{\tau(x)}$ is equivalent to $$R_{\tau(x)} \in \tau(x) \wedge R_{\tau(x)} \not\in R_{\tau(x)},$$ which is equivalent to $R_{\tau(x)} \not\in R_{\tau(x)}$, which is absurd.

The proof by contradiction that $\tau^2(x)\neq \tau(x)$ is complete.

\end{description}

We adapt this proof to show that $\tau^{n+2}(x) \neq \tau(x)$ for any $x$, for each concrete natural number $n$.  Resnik proved this theorem in the same way in his related earlier work (which we will discuss below).
\begin{description}
\item[Theorem (following Russell and Resnik):]  For each $x$, $\tau^{n+2}(x) \neq \tau(x)$, for each concrete natural number $n$.

\item[Proof:]  Let $x$ be any object.  Suppose that $\tau^{n+2}(x)=\tau(x)$. 

Define $R_x$ as $\{\iota^n(y)\in \tau^{n+1}(x):\iota^n(y) \not\in y\}$.  It is straightforward to show that $\iota^n(y) \in \tau^{n+1}(y)$ for each $y$ and concrete $n$ (by type coercion) so $R_x \in \tau^{n+2}(x) = \tau(x)$ by hypothesis, so $\iota^n(R_x) \in \tau^{n+1}(x)$.    Now $\iota^n(R_x) \in R_x$ iff $\iota^n(R_x) \in \tau^{n+1}(x) \wedge \iota^n(R_x) \not\in R_x$, which we have seen is equivalent to $\iota^n(R_x) \not\in R_x$, which is absurd.

\item[Remark:]  We remind the reader that distinct types are disjoint, since they are equivalence classes.  The types $\tau(x)$ and $\tau^{n+2}(x)$ are not only distinct, but they have no common  elements.

\item[Constructing finite sets:]  The singleton set of $x$, $\{x\}$, can be defined as $\{y \in \tau(x):y=x\}$.  This exists and has $x$ as its only element by the axioms of types and separation.  It is convenient to define $\iota(x)$ as $\{x\}$, $\iota^0(x)$ as $x$,
and $\iota^{n+1}(x)$ as $\{\iota^n(x)\}$ for each concrete natural number $n$.

If $x \sim y$, we have $\{x\} \sim \{y\}$ by type coercion so $\{x,y\}$ can be defined
as $\{z \in \tau(x):z =x \vee z=y\}$ and has exactly $x$ and $y$ as its elements.

We can define $\{x_1,x_2,\ldots,x_n\}$ as $\{x_1\} \cup \{x_2,\ldots,x_n\}$, completing the definition of the list notation for finite sets by metatheoretic recursion.

\item[Definiton (ordered pair, basics of relations):]  For $x \sim y$, we define $(x,y)$ as $\{\{x\},\{x,y\}\}$.  

Notice that $x$ is the unique object which belongs to every element of $(x,y)$, and $y$ is the unique object which belongs to exactly one element of $(x,y)$.  It follows from this that $(x,y)=(z,w)$ implies $x=z$ and $y=w$.  

Notice that by type coercion, $(x,y)$ belongs
to $\tau^3(x) = \tau^3(y)$.  For $A , B$, both sets and of the same type, we can define $A \times B$, the cartesian product of $A$ and $B$,  as $$\{(a,b)\in \tau^2(A):a \in A \wedge b \in B\}.$$

We define a relation as a set $R$ of ordered pairs (further requiring, in case $R$ is empty, that it belong to some type $\tau^3(u)$).  If $R$ is a relation, we write $x \, R\, y$ for $(x,y) \in R$.  For any relation $R$, we define $R^{-1}$ as \newline $\{(y,x) \in \tau(R):x \, R\, y\}$ and ${\tt dom}(R)$ as $\{x \in \bigcup^2 \tau(R):(\exists y:x \, R\, y)\}$.  Notice that
$\bigcup^2 \tau(R)$ will be $\tau(x) = \tau(y)$ for any $(x,y)\in R$, by existence of set unions and type coercion. We can then define ${\tt rng}(R)$ as ${\tt dom}(R^{-1})$.

We define $R``A$, for $R$ a relation and $A$ a set with $\tau^2(A) = \tau(R)$, as $\{b \in \bigcup \tau(A):(\exists a:a \in A \wedge a \, R \, b)\}$.

\item[Definition (basics of functions):]  We say that $f$ is a function (written ${\tt function}(f)$) iff it is a relation and for every $x,y,z$, $x \,f\,y \wedge x \,f\,z \rightarrow y=z$.  We say $f:A \rightarrow B$ ($f$ is a function from $A$ to $B$) iff $${\tt function}(f) \wedge f \subseteq A \times B.$$  

We say that $f$ is one to one or an injection iff $f$ and $f^{-1}$ are both functions.

We say that $f:A \rightarrow B$ is onto $B$ or a surjection from $A$ to $B$ iff ${\tt rng}(f)=B$.  

We say for sets $A,B$ of the same type that $A \approx B$ ($A$ and $B$ are of the same cardinality) iff there is $f:A \rightarrow B$ which is an injection and onto $B$ (such a function is called a bijection from $A$ to $B$).  

We define $|A|$, the cardinality of $A$,
as $\{B \in \tau(A):A \approx B\}$.  Note that $|A| \in \tau^2(A)$.

\item[Lemma:]  For any sets $A\sim B \sim \{x\} \sim \{y\}$ with $x \not\in A$ and $y \not\in B$, $A \approx B \leftrightarrow A \cup \{x\} \approx B \cup\{y\}$

\item[Proof:]  Standard, and omitted for the moment.

\item[Definition:]  We define $\sigma(|A|)$ as $|A \cup \{x\}|$ where $\{x\} \sim A$ and $x \not\in A$.  The previous lemma tells us that this definition works.  Note that $\sigma(|\tau(x)|)$ is not defined, unless there is a set $A \subseteq \tau(x)$ such that $A \approx \tau(x)$, in which case it is straightforward to establish that $\sigma(|A|) = |A| = |\tau(x)|$, so [in this case] $\sigma(|\tau(x)|)$ can safely be defined as $|\tau(x)|$.

\item[Definition:]  We define $0_{\tau^2(x)}$ as $|\emptyset_{\tau(x)}|$.  

For each concrete numeral $n$ for which $n_{\tau^2(x)}$ has been defined,
we define $(n+1)_{\tau^2(x)}$ as $\sigma(n_{\tau^2(x)})$.

We define ${\mathbb N}_{\tau^3(x)}$ as $$\{n \in \tau^2(x):(\forall I:(0_{\tau^2(x)} \in I \wedge (\forall m:m \in I \rightarrow \sigma(m) \in I)) \rightarrow n \in I)\}.$$

We define ${\mathbb F}_{\tau^2(x)}$ as $\bigcup {\mathbb N}_{\tau^3(x)}$.

Thus we have defined the natural number zero, successor for natural numbers, each concrete natural number for counting elements of each given type,
the natural numbers for counting elements of each given type, and the set of finite sets of each given type.

\item[Axiom of Infinity:]  $(\forall x:\tau(x) \not\in {\mathbb F}_{\tau^2(x)})$

We will generally assume this axiom.

\item[Remark:]  We have enough machinery to do the standard mathematics of the natural numbers.  It is odd (and reminiscent of what happens in the typed theory of sets not yet mentioned) that the natural numbers used to count elements of different types are different.  It is not the case that arithmetic over different types is necessarily the same, though for any fixed $x$ (and assuming Infinity) arithmetic in each ${\mathbb N}_{\tau^{i+3}(x)}$ will be the same.  The constructions of the real numbers and of the usual spaces constructed from the reals go through with similar remarks about reduplication over different types.

\item[ operations of cardinal arithmetic:]  If $A$ and $B$ are disjoint sets, we define $|A|+|B|$ as $|A \cup B|$.  It is straightforward to show that if $A \approx A'$, $B \approx B'$ and $A'$ and $B'$ are disjoint, it follows that $A \cup B \approx A' \cup B'$, so the definition of addition of cardinals does not depend on the choice of the sets $A$ and $B$ from the cardinals.
This definition applies to all cardinal numbers, but of course it specializes to the naturals.   The formal possibility is present that some sums of cardinals might not be defined, though this will not happen under reasonable assumptions.

If $A$ is a set, we define $\iota``A$ as $\{\{a\}:a \in A\}$ and $T(|A|) = |\iota``A|$.  It is straightforward to show that the definition of this operation on cardinals does not depend on $A$, and further that $T(|A|) = T(|B|) \rightarrow |A|=|B|$, so the operation $T^{-1}(\kappa)$ makes sense for cardinals $\kappa$, though it may be partial.  The cardinal
$T(|A|)$ might seem to be the same cardinal but it is of a different type, and so not the same object.

If $A$ and $B$ are sets, we define $|A| \cdot |B|$ as $T^{-2}(A \times B)$.  This operation may seem to be partial, but under reasonable assumptions it is total.  Again, this definition does not depend on the choice of the sets $A$ and $B$, and it specializes to the natural numbers.

If $A$ and $B$ are sets, we define $|B|^{|A|}$ as $T^{-3}(|\{f:(f:A \rightarrow B)\}|)$, the cardinality of the set of functions from $A$ to $B$ shifted downward suitably in type.  This operation is not total, as we will see.

Given the Axiom of Infinity, there are straightforward proofs by induction that the addition, multiplication, and exponentiation operations are total on the natural numbers.

\item[well-orderings and ordinal numbers:]  We define linear orders and well-orderings in the usual way:  a well-ordering for us is reflexive ($\leq$ rather than $<$).  The notion of isomorphism of well-orderings is defined as usual.  If $\leq$ is a well-ordering, the order type of $\leq$, written ${\tt ot}(\leq)$, is the isomorpshim class of $\leq$;  an object is an ordinal number iff it is the order type of some well-ordering.  For any relation $R$, we define $R^{\iota}$ as $\{(\{x\},\{y\}) \in \tau^2(R):x \, R \, y\}$ and for any ordinal $\alpha ={\tt ot}(\leq)$ define $T(\alpha)$ as ${\tt ot}(\leq^\iota)$.  The ordinal $T(\alpha)$ seems in some sense to be the same order type as $\alpha$, but it is a distinct object because it belongs to a different type.

\item[Remark:]  The knowledgeable reader will recognize the mathematical constructions here as the same ones done in New Foundations or NFU (or in the simple typed theory of sets);  the point we are making by exhibiting them here is that they are entirely natural here without any reference to the typed theories or to the notion of stratification.

\item[Remark:]  One of the earlier proposers of a theory of this kind asserted that what Russell called systematic ambiguity, which is more recently called polymorphism or typical ambuity, has no role here.  The construction of the natural numbers indicates that this is false.  The hall of mirrors effect found in the typed theory of sets
is clearly manifest:  we end up, for example, with a number $3_{\tau(t)}$ that cohabits with $\tau(t)$ for every type $t$.  Stating an ambiguity property of provable statements is a bit trickier because the untyped language of TTGV has more resources.

\end{description}

\subsection{Typed formulas}

We introduce language which is in effect sorted (though not formally so), but is naturally motivated by the type coercion and type hierarchy results of the previous section

\begin{description}

\item[Definition:]  We say that a formula in the language of TTGV is {\em typed\/} if each bound variable $v$ is restricted to a type $\tau^n(x)$ where $x$ is a parameter of the formula.  The values of $n$ and $x$ may be different for different variables.  We refer to $\tau^n(x)$ as the type of $v$, written ${\tt type}(v)$ , in this context.  The type of a parameter $u$ is simply the actual set $\tau(u)$, the type in the usual sense to which it belongs.

\item[Observation and further definition:]  A formula $u=v$ where $u$ and $v$ are both bound can be reduced to the False if ${\tt type}(u) \neq {\tt type}(v)$.  A formula $u \in v$ where $u$ and $v$ are both bound can be reduced to the False if $\tau({\tt type}(u)) \neq {\tt type}(v)$.   We define a well-typed formula as a typed formula in which
each atomic fomula $u=v$ satisfies ${\tt type}(u) = {\tt type}(v)$ and each atomic formula $u \in v$ satisfies $\tau({\tt type}(u)) = {\tt type}(v)$.  Observe that any typed formula in which we presume that we know the values or at least the types of any parameters is equivalent to a well-typed fomula, because all subformulas which fail these conditions can be eliminated.

\item[Definition:]  A variable $u$ is said to be connected to a variable $v$ in a formula $\phi$ if and only if $v$ belongs to every set of variables appearing in $\phi$ which contains $u$ and is closed under the relation of occurring together in an atomic subformula of $\phi$. 

\item[Observation:]  Notice that in a well-typed formula $\phi$, if a variable $x$ has type $\tau(u)$, every variable connected to $x$ has type $\tau^i(u)$ for some integer $i$ (review our definition above of $\tau^i(u)$ for nonpositive $i$).

\item[Segregation Lemma:]  
In what follows, we may view $(\forall u:\phi \rightarrow \psi)$, where $\phi$ may contain $u$ but nothing else but parameters, as a restricted quantifier over $u$ with scope $\psi$, and we regard each occurrence of a quantifier as having a restriction (or lack of restriction) understood.  In a typed formula, there is an obvious understood restriction of each quantifier.

For any formula $\phi$ and variable $x$, it is possible to present an equivalent formula $\phi^*$ in which any quantifier (possibly restricted) over a variable connected to $x$ has only variables connected to $x$ in its scope,
and any quantifier (possibly restricted) over a variable not connected to $x$ has only variables not connected to $x$ in its scope.  Such a formula is said to be segregated for $x$.  

\item[Proof:]  We may assume that we use only universal quantifiers, for simplicity.

We indicate how to export any atomic subformula $u \, R \, v$ in which $u$ and $v$ are connected to $x$ from the scope of all quantifiers (unrestricted or restricted) over a variable $w$ not connected to $x$.   We include as
restricted quantified formulas those of the form $(\forall x:\phi \rightarrow \psi)$ in which $\phi$ may contain $x$ and parameters, as noted above.

Consider a formula $(\forall w:\chi \rightarrow \psi)$, supposing that $\chi$ contains no variables but $w$ and parameters and $\psi$ can be converted to a segregated form $\psi^*$.

Let $\phi$ be the largest proper subformula of $\psi^*$ containing a given instance of $u \, R \, v$ which is either $u\,R\, v$ itself or a quantified formula (restricted or otherwise).  $\phi$ is not in the scope of any quantifier
not connected to $x$ other than the given quantifier over $w$, nor is it in the scope of any quantifier over a variable connected to $w$ or it would not be largest.  So we can convert  $(\forall w:\chi \rightarrow \psi)$ to the form
$$(\forall w:\chi \rightarrow (\phi \rightarrow \psi_1^*) \wedge \chi \rightarrow (\neg \phi \rightarrow \psi_2^*)),$$ where $\psi_1^*$ and $\psi_2^*$ are obtained by replacing $\phi$ with truth values in $\psi^*$, which is equivalent to $$\phi \rightarrow (\forall w:\chi \rightarrow \psi^*_1) \wedge \neg\phi \rightarrow (\forall w:\chi \rightarrow \psi^*_2),$$ in which the occurrence of $u \, R\, v$ has beeen moved out of the scope of the quantifier over $w$.  We have shown that this works if the quantifier over $w$ is restricted; clearly it also works for unrestricted quantifiers. 

In the same way, export formulas involving variables not connected to $x$ past quantifiers over variables connected to $x$.   

This process can be iterated until all undesired occurrences of atomic subformulas in scopes of quantifiers with understood restrictions have been removed.


\item[Theorem:]  Every set abstract $\{x \in \tau(u):\phi\}$ is equivalent, for each fixed assignment of values to its parameters, to a set abstraction \newline $\{x \in \tau(u):\phi^*\}$ in which $\phi^*$ is a typed formula.

\item[Proof:]  The strategy of the argument is to show that unrestricted universal quantifiers (and so, by duality, unrestricted existential quantifiers) can be eliminated in favor of quantifiers restricted to types.

We describe this process for a subformula $(\forall y:\psi)$ in which $\psi$ is a typed formula.  The variable $y$ is free in $\psi$.  Each subformula in which it appears suggests a type to which it might belong:  a subformula $y = z$ in which $z$ is connected to a parameter in $\psi$ suggests that
$y \in \tau(u)$, where $\tau(u)$ is the type of $z$, a subformula $y \in z$ suggests that $y \in \tau^2(u)$, and a subformula $z \in y$ suggests that $y \in \tau^0(u) = \bigcup \tau(u)$.  Let $\tau(z_1),\ldots,\tau(z_n)$ be the types conjecturable for $y$ in this way.  $(\forall y:\psi)$ is equivalent to $$(\forall y_1 \in \tau(z_1):\psi[y_1/y]) \wedge \ldots (\forall y_n \in \tau(z_n):\psi[y_n/y]) $$ $$\wedge (\forall y_{n+1}:\tau(y_{n+1}) \neq \tau(z_1) \wedge \ldots \tau(y_{n+1}) \neq \tau(z_n) \rightarrow \psi[y_{n+1}/y]).$$

Each of the conjuncts $(\forall y_1 \in \tau(z_1):\psi[y_1/y])$ is unproblematic because it is a typed formula (and can further be transformed to be well-typed).

The alarming conjunct is the final one, $$(\forall y_{n+1}:\tau(y_{n+1}) \neq \tau(z_1) \wedge \ldots \tau(y_{n+1}) \neq \tau(z_n) \rightarrow \psi[y_{n+1}/y]).$$

We view the hypothesis of the quantified implication as a restriction for purposes of the Segregation Lemma.

The key here is that we know from the hypotheses about the type of $y_{n+1}$ that $\psi[y_{n+1}/y]$ can be converted
to a form in which $y_{n+1}$ is not connected to any parameter in $\psi$, and in particular it is not connected to the variable
$x$ which is bound by the set abstract.  Any assertion about variables not connected to $y_{n+1}$, which include parameters in $\psi$ (including $x$),  which is included in
$\psi[y_{n+1}/y]$ can be pulled out of the scope of the restricted quantifier over $y_{n+1}$ using the Segregation Lemma (and will be typed).  The possibly multiple formulas with quantifiers over $y_{n+1}$ which remain are not necessarily  typed formulas [entirely because of the assertion $\tau(y_{n+1}) \neq \tau(z_1) \wedge \ldots \tau(y_{n+1}) \neq \tau(z_n)$ serving as restriction]  , but  since each of them does not depend on any variable not connected to $y_{n+1}$ other than parameters in the restricting clause, each is closed and equivalent simply to a truth value (note that while $x$ is variable,
its type [which we do have to be concerned about because it might occur in the hypothesis $$\tau(y_{n+1}) \neq \tau(z_1) \wedge \ldots \tau(y_{n+1}) \neq \tau(z_n)$$ which we are using as a bound to the quantifier] is not:  $\tau(u)$ is its type and $u$ can be taken to be a parameter whose value is fixed).

Thus for any fixed values of parameters in $\{x \in \tau(u):\phi\}$, it is the same collection as $\{x \in \tau(u):\phi^*\}$ for some typed formula $\phi^*$.

\item[Remark:]  One might want to prove the more dramatic statement that every formula $\phi$ in which each parameter is assigned a fixed value is equivalent to a typed formula.  Unfortunately, it appears unlikely that this is the case.  The method of proof above can be attempted:  the problem is the presence of formulas
which are closed typed formulas asserted for all but a given finite collection of types.  The restricted quantifiers over all types that are involved can be exported all the way to the outside as we work through the construction.  But the restriction to all but a finite collection of types is intractable.

Two incompatible additional schemes allow one to conclude that every formula is equivalent to a partial universal closure of a typed formula.  Both attack the
issue of the strange restriction on the universal quantifiers over types, in different ways.  If for each type $t$ there is a typed formula $\phi_t(x)$ such that $\phi_t(x) \leftrightarrow x \in t$, then one can convert the restricted universal quantifiers over types to unrestricted universal quantifiers over types\footnote{This is true in a model derived directly from a model of the usual typed theory of sets with a type of individuals.}.  On the other hand,
if one has a scheme asserting that any formula $\phi(t)$ (with no free variables other than $t$) which holds of all types $t$ but those in a concrete finite list $t_1,\ldots,t_n$ in fact holds of all types,
then again the restricted quantifiers over types convert directly to unrestricted quantifiers over all types.

We do not see any general method of removing this failure of equivalence to typed formulas without special assumptions.

\item[Remark:]  For any typed sentence $\phi$, define $\phi^+$ as the result of replacing each type bound $\tau(u)$ with $\tau^2(u)$.  If the universal closure of $\phi$ is a theorem,
so is the universal closure of $\phi^+$;  the converse is not true.  This is a clarification of the polymorphism of this theory.  Now the following scheme strengthens this to $\phi \leftrightarrow \phi^+$:  the Ambiguity Scheme for TTGV asserts for each formula in which only $x$ is free, $(\forall uv:\phi[\tau(u)/x] \leftrightarrow \phi[\tau(v)/x])$.  This asserts
that all types look the same, in effect.  Consistency with TTGV of the ambiguity scheme for TTGV follows from the consistency of NFU;  we will review this in the next section.

An entertaining variant which seems to have the strength of the Axiom of Counting is, for any formula $\phi$ in which $x,y$ are the only free variables,
$$(\forall u:(\forall n \in {\mathbb N}_{\tau^3(u)}:\phi[\tau(u)/x;n/y] \leftrightarrow \phi[\tau^2(u)/x;T(n)/y])).$$  This is unsatisfying in that we can't talk about all types at once in the same way as in the other scheme of ambiguity:  the problem is that we have no way to port natural numbers from one type to another if the types  are not connected by iterated applications of $\tau$.  So one would want the unrestricted scheme of ambiguity as well.

Notice that the Ambiguity Scheme implies that every formula is equivalent to the partial universal closure of a typed formula, since it is a strengthening of one of the schemes which we know implies this.

\end{description}


\section{Typed theories of sets introduced;  older proposals for type theory with general variables}

We presented TTGV in the previous section as if it were an independent proposal for the foundation of mathematics.  The knowledgeable reader should be able to divine a lot about where it came from from what we have said so far;  in this section we will make the historical background of this proposal clear.  Our aim in organizing things this way is to make it clear that a theory of this kind can be presented without explicitly or implicitly supposing knowledge of the typed theories at all.  We have used the word ``type" in the previous section, but to denote a kind of set, not a sort in the strict logical sense.

\subsection{Typed theories of sets:  TST and variants}

The original theory of this kind, which appears to have been implicitly proposed by Norbert Wiener in 1914 and explicitly described by Tarski in the 1930s, has been called TST by the Belgian school of logicians who studied NF, and this is what we will call it.  We note for historical accuracy that this is {\em not\/} the theory of types of Russell and Whitehead's {\em Principia\/};  it is considerably simpler, and Russell and Whitehead did not have mathematical knowledge required to simplify their system to this form.

TST is a multi-sorted first order theory with equality and membership.  The sort of a variable $v$ will be written ${\tt type}(v)$.  We provide a countable supply of variables of each sort.  Using ++ to denote concatenation of strings, the formation rules for atomic formulas are that $v{\tt ++}\verb|`|=\verb|'|{\tt ++}w$ is a well formed atomic formula iff ${\tt type}(v)={\tt type}(w)$, and 
$v{\tt ++}\verb|`|\in\verb|'|{\tt ++}w$ is a well-formed atomic formula iff ${\tt type}(v)+1 ={\tt type}(w)$.  All atomic formulas are formed in this way.  Writing this out in a way which manages use and mention correctly is a technical challenge!

We do not follow the convention of equipping variables with type superscripts in TST, which makes for very cluttered notation, though if we do provide a variable with a numeral superscript, one may expect that the type of that variable is as indicated.

The axioms of TST are a scheme of extensionality and a scheme of comprehension.  The scheme of extensionality provides that each well-formed formula of the shape $$(\forall xy:x=y \leftrightarrow (\forall z:z\in x \leftrightarrow z \in y))$$ is an axiom.  This asserts that objects of type $n+1$ with the same extension (consisting of type $n$ objects) are the same.  The scheme of comprehension provides that for each well-formed formula $\phi$ in which the variable $A$ is not free, 
$(\exists A:\forall x:(x \in A \leftrightarrow \phi))$ is an axiom if it is well-formed (the only additional requirement being that the type of $A$ is the successor of the type of $x$).

It is usual to adjoin an axiom of infinity (whose form can be deduced from the development of mathematics in TTGV in the previous section) and often the axiom of choice to this theory, but they are not part of the formal definition of the theory we give here.

Some variants of this theory are worth noticing.  Hao Wang proposed the variant TZT which differs simply by indexing the sorts by all integers instead of just the nonnegative integers.  The consistency of TZT follows from the consistency of TST by a simple compactness argument.  Wang himself called the theory TNT for ``theory of negative types";  Forster prefers TZT because the theory in fact has all integer types, not just the negative ones.

The variant TSTU differs from TST in allowing urelements.  Its extensionality scheme is $$(\forall xyz:z\in x \rightarrow(x=y \leftrightarrow (\forall z:z\in x \leftrightarrow z \in y)))$$ (providing that nonempty sets with the same extension are equal) and it is convenient to supply a primitive constant $\emptyset^{i+1}$ of each type $i$ with the axiom scheme consisting of $(\forall x:x \not\in \emptyset^{i+1})$ for each concrete natural number $i$.  An object of type $i+1$ is a set if it has elements or is equal to $\emptyset^{i+1}$.

In any of these theories, one can provide a term construction $\{x^i:\phi\}$ of a term of type $i+1$ representing the unique set $A$ such that $(\forall x:(x \in A \leftrightarrow \phi)$.  The type rules for term constructions are straightforward to adapt from those for variables.

It is straightforward to show that TST is interpretable in the usual set theory ZFC.  Let $X_0$ be an arbitrarily chosen set. Define $X_{n+1}$ as ${\cal P}(X_n)$ for each $n$.  In any formula of the language of TST, assign each parameter
of type $i$ a value in $X_i$ and interpret each quantifier over type $i$ as a quantifier restricted to $X_i$.  It is straightforward to check that each interpretation of an axiom of TST is true.   The fact that the sets representing the types are not disjoint is harmless.

We further note that TSTU is interpretable in TTGV, the theory we defined in the first section.  Let $\tau(x)$ be a type.  Interpret each parameter of type $i$ as an element of $\tau^{i+1}(x)$.  Interpret each quantifier over type $i$ as a quantifier restricted to $\tau^{i+1}(x)$.  That the axioms of TSTU hold is immediate from the axioms of TTGV:  the weak extensionality of TSTU has the same form as the extensionality of TST, and the interpreted comprehension axiom of TSTU follows from
the separation axiom of TTGV.  Of course TST is interpretable in TTGV with the additional assumption of strong extensionality.

Our reasons for preferring to frame our flagship theory of types with general variables with weak extensionality will become evident shortly.

\subsection{Quine's original proposal of type theory with general variables}

Quine's original proposal of a type theory with general variables equivalent to TST is the subject of this subsection.


\begin{description}

\item[definition of ``being of the previous type":]  $x\, {\tt PT}\, y$ is defined as $$(\exists zw: x \in w \wedge w \in z \wedge y \in z).$$

\item[definition of type 0:]  $T_0(x)$ is defined as $(\forall y: \neg y\, {\tt PT}\, x)$.

\item[definition of next type:]  For each concrete natural number $n$, $T_{n+1}(x)$ is
defined as $(\forall y:  T_n(y) \rightarrow y\, {\tt PT}\, x)$

\end{description}

He then stated his axioms schematically.


\begin{description}

\item[Quine's comprehension axiom:]  For any formula $\phi$, $$(\exists A:  T_{n+1}(A) \wedge (x\in A \leftrightarrow (T_n(x) \wedge  \phi)))).$$

\item[Quine's extensionality axiom:]  $$(\forall xyz:T_{n+1}(x) \wedge  T_{n+1}(y) \wedge (\forall w:T_n(w) \rightarrow (w \in x \leftrightarrow w \in y)) \wedge x \in z \rightarrow y \in z).$$  I preserve the form of this axiom, which reflects defining equality in terms of membership, but it could be phrased differently.

\end{description}

These are actually not the axioms as he first states them:  this is the original extensionality axiom together with a modified version of the comprehension axiom which he states later as a consequence of the assumption that all elements of type $n+1$ objects belong to type $n$, which his original axioms (astonishingly) do not imply.

Quine does recognize the importance of the relation of cohabitation as representing the notion of belonging to the same type, though he does not use it in the statement of his axioms.

This theory is not quite the same as ours.  To begin with, it has what we regard as a formal defect:  there is no need to axiomatize the theory with schemata with concrete natural numbers as indices, as we have demonstrated with our axiomatization (and as Resnik did previously and very similarly).  Quine does observe that he cannot prove and cannot even actually say that every object belongs to some type.  Further, his theory says nothing at all about objects which do not belong to a type. In our theory and Resnik's, it is immediate that every object belongs to a type, but the types may not be restricted to the familiar ones.

Quine says more about individuals than we do.  Quine asserts that all individuals belong to the same type.  Resnik also asserted this as an axiom.  We have not felt the need to do this, but we could.  We also want to be free to explore the possibility there are no individuals at all.

We think that our presentation is superior to Quine's for a number of reasons.  Our presentation does not allude to the simple typed theory of sets at all in its formulation [or much less obviously]:  the fact that it is actually a presentation of the simple typed theory of sets unfolds in the development, as the reader should see in our first section.  We dispute something that Quine says:  there is a strong place for systematic ambiguity in this theory; we do not escape this phenomenon when we transition to a one-sorted theory.  But this also comes out in the development in the first section.

The axioms as selected above from Quine's treatment allow us to prove that
all elements of a type $n+1$ object are of type $n$:  for any $x$ of type $n+1$ there
is $x^*$ of type $n+1$ containing exactly the type $n$ elements of $x$, and then by his original formulation of extensionality, $x^*=x$, so in fact all elements of $x$ are of type $n$.  In our formulation, as will be seen, the axiom of binary union is used to prove the analogous assertion.

The fundamental point here is that Quine's theory is not intellectually independent from TST:  Resnik's theory and mine are independent of TST(U) in their formulation, though related notions naturally develop as these theories unfold.

Finally, our theory differs from Quine's quite deliberately in allowing atoms as well as empty sets, for reasons to be discussed soon.

NOTE:  this system supports a proof that every formula is equivalent to the partial universal closure of a well-typed formula, because every type is associated with a formula, so the ``all but finitely many types" restricted quantifiers can be handled.  Details are different enough that they need to be written up.  This discussion is needed because the comprehension axiom is stated for arbitrary formulas, which might suggest additional power over TST.  In fact there is none.


\subsection{The system of Resnik}

What Quine did was a kludge.  The presence of meta theoretic natural number parameters corresponding exactly to the types
reveals that he is not really describing an autonomously motivated system.

Resnik gives a genuine one-sorted theory with one-sorted motivation from which type theory falls out, as we do, and his theory is very close to ours.

We list his seven axioms, staying closer to our own notation.

\begin{description}

\item[Definition:]  $x \sim y$ means $(\exists z:x \in z \wedge y \in z)$.  Resnik defines
$x=y$ as $(\forall z:x \in z \leftrightarrow y \in z)$.  So does Quine; for us equality is a logical primitive, but the comprehension axiom
of any of these theories should make this definition harmless.

\item[Ax 1:]  $(\forall x:(\exists y:(\forall z:z \in y \leftrightarrow z \sim x)))$.  This is almost the same as our axiom of types:  ours has the extra clause $x \in y$ to ensure that $\sim$ is reflexive.  Strangely, the axiom of comprehension has to be used to fill in this detail in Resnik's system.

\item[Ax 2:] $(\forall xyw:y \in x \wedge y \in w \rightarrow x \sim w)$.  Sets which meet have the same type.   This is equivalent to our axiom of subsets.  I'm wondering whether this is provable from Resnik's other axioms.  

\item[Ax 3:]  $(\forall uvwxy:  y \in x \wedge u \in x \wedge y \in w \wedge v \in w \rightarrow (\exists t:y \in t \wedge u \in t \wedge v \in t))$.  This axiom is used to support transitivity of $\sim$.  I believe that it is redundant.  If $y \in x \wedge u \in x$ then we have $y \in \tau(u)$, where $\tau(u)$ witnesses
Ax 1 with $x := u$.  Similarly we have $v \in \tau(u)$.  $u \in \tau(u)$ is not a consequence of Ax 1 (as it is in our formulation) but it does hold here because
$u$ belongs to some set by the hypotheses.  So we can choose $\tau(u)$ as $t$.

\item[Ax 4:]  $(\forall vwxyz:y \in x \wedge v \in w \wedge x \in z \wedge w \in z \rightarrow y \sim v)$  This does what the axiom of binary union does for us.  We say that because $x$ and $w$ have the same type, they have a union, and of course this union will contain $y$ and $v$.  We have considered this exact statement as an axiom, but union seemed more clearly a natural axiom.

\item[Definition:]  $x {\tt PT} y$ is defined (following Quine) as $(\exists zw:x \in w \wedge w \in z \wedge y \in z)$.  $T_0(x)$ ($x$ is an individual)
is defined as meaning $\neg(\exists y:y {\tt PT} x)$:  nothing belonging to the same type as $x$ has elements.

\item[Ax 5:]  $(\exists x:T_0(x))$  We do not commit ourselves to the existence of any individuals.  But it is natural to do so given the historical origin of this theory.

\item[Ax 6:]  $(\neg T_0(x) \wedge x \sim y \wedge (\forall z:z \in x \leftrightarrow z \in y) \wedge x \in w) \rightarrow y \in w$.  The form of this looks peculiar to us because Resnik treats equality is a defined notion, but it is the axiom of extensionality.  It is a bit different from ours:  it is weaker in that it does not force equality of nonempty sets with the same extension (Axiom 2 assists with this);  it allows individuals with the same empty extension to be distinct but any empty object in a type is the only empty object in that type.  This is natural;  we are more liberal in allowing many atoms in each type.

\item[Ax 7:] For any formula $\phi$, $(\forall z:(\exists y:w \sim z \wedge (\forall x:x \in y \leftrightarrow (\phi \wedge x \in z))))$.  This is Zermelo's axiom scheme of separation, with the extra proviso that the set defined is of the same type as the bounding object (which actually doesnt not have to be a set, and this has a use).

\end{description}

The maneuver for showing that a general object belongs to a set is rather strange here, and I want to be sure that Resnik actually realizes that he has to do it.
For an arbitrary $x$, there is $w \sim x$ with empty extension, by axiom 7...and incidentally, some set contains both $x$ and $w$, so $x$ belongs to a set.

That said, this theory is the same as ours with stronger extensionality and the positive assertion that there are individuals.  I think that my axiomatics are cleaner, and that there are really good reasons to consider the possibility of atoms in addition to empty sets.

There is an error in Resnik.  He claims that he can prove that all individuals are of the same type.  This does not follow from his axioms.  I think part of the issue is that he defines {\tt ST} (the relation of being the same type, which we denote by $\sim$) in two different ways and does not seem to realize that they are not equivalent.  It wouldn't be unreasonable given his evident intention to strength his axiom asserting the existence of individuals to assert as well that all individuals are of the same type.

We acknowledge this system as prior to ours, and as doing basically the same work:  we were not aware of this work when we framed our system of the first section.  We do think that there are formal advantages to our slightly weaker system, which will come out in further discussion.  Proofs of useful results from the first section port easily to this system.


\subsection{Ambiguity and stratification:  NF and NFU}

TST exhibits a stronger form of a symmetry that Russell noted in the more complicated system of
Principia Mathematica and called ``systematic ambiguity".  This symmetry led to another proposal by Quine of an untyped version of TST which we describe because it is relevant to our project here.

In TST, provide a map $(x \mapsto x^+)$ on variables which is an injection and raises type by one.
For any formula $\phi$, define $\phi^+$ as the formula which results if each variable $x$ is replaced with $x^+$.

It is straightforward to see that if $\phi$ is provable, so is $\phi^+$.  The converse is not true.

One could then reasonably conjecture the consistency of the Ambiguity Scheme, which asserts
$\phi \leftrightarrow \phi^+$ for each closed formula $\phi$.

Quine made the apparently stronger proposal that the types can simply be identified.  The resulting theory is called NF (New Foundations) after the name of the paper in which it appeared.

NF is a one-sorted first order theory with equality and membership with the axiom of strong extensionality
(objects with the same extension are equal) and the axiom scheme of stratified comprehension:  $\{x:\phi\}$ exists if there is a function $\sigma$ from variables to natural numbers such that each atomic subformula
$v++\verb|`|=\verb|'|++w$ appearing in $\phi$ has $\sigma(v)=\sigma(w)$ and each atomic subformula
$v++\verb|`|\in\verb|'|++w$ appearing in $\phi$ has $\sigma(v)+1=\sigma(w)$: such a formula is said to be stratified and the function $\sigma$ is called a stratification.  Clearly it is equivalent to say that we are asserting that $\{x:\phi\}$ exists if $\phi$ could be turned into a well formed formula of TST by an appropriate assignment of sorts to variables.

NF presents difficulties:  it was shown in 1953 to disprove Choice, and its consistency remained an open question until very recently.

Specker showed in 1962 that NF is equiconsistent with TST + Ambiguity, and with the existence of a model of TST in which there is a type raising endomorphism.  This justifies Quine's jump from the temptation of the Ambiguity Scheme to the temptation of simply identifying the types.

Jensen showed in 1969 that NFU, the system with weak extensionality and stratified comprehension, is consistent and not even very strong.  It is consistent with but does not prove Infinity, and it is consistent with Choice.  This formal advantage of NFU over NF is the main reason that we choose to use weak extensionality in the definition of TTGV.

\subsection{The proposal of Forster}

Thomas Forster proposed the following type theory of general variables, which was the first one I encountered.  [NOTE:  Thomas, I am lacking a copy of your initial communication with me about this:  I surely have it but I cant find it in my email;  this is a summary I put at the head of one of my versions].

This is a first order one sorted theory with equality and membership

\begin{description}

\item[Definition:]  $x \sim y$ is defined as $(\exists z:x \in z \wedge y \in z)$.

\item[Axiom of weak extensionality:]  Objects with elements are equal if they have the same extension.

\item[$\sim$ is an equivalence relation:]  $\sim$ is an equivalence relation in which the equivalence classes are sets:  the equivalence class containing $x$ is denoted by $\tau(x)$.

\item[set union:]  The usual axiom of set union is asserted:  for every $A$, $\bigcup A$ exists where $x \in \bigcup A \leftrightarrow (\exists y:x \in y \wedge y \in A)$

\item[comprehension:]  $\{x \in \tau^n(u):\phi\}$ exists
where each variable appearing in $\phi$ is typed in the sense that
(if it is a parameter) it belongs to some $\tau^m(u)$ and if it is bound, it is bound by a quantifier restricted to a $\tau^m(u)$, $x$ is assigned type $n$ of course, and further that in each subformula $u=v$ the types of $u$ and $v$ are the same and in each subformula $u \in v$, the type of $v$ is the image under $\tau$ of the type of $u$.

This is something like the assertion that all sets determined by well typed set abstracts exist (using the terminology of the first section) though it is a bit less general.


\end{description}

The second author is somewhat critical of this proposal of the first author, though it does have substantial interest.  Like the proposal of Quine, it appears to depend philosophically on prior awareness of TST.  It does have the interesting feature that it does not prove that the types are disjoint:  if NF is consistent, a model of NF is a model of this theory in which there is only one type.  Forster is also interested in the possibilty of cycles in the types, in which $\tau^n(x)$ might be equal to $\tau(x)$ for some $n>2$.  These would correspond to type theories with loops in the types.

\subsection{Interpretation of the theories with general variables in the typed theories}

We now argue that a model of TST provides an interpretation of TTGV.  These results will extend
to the other theories, possibly under special assumptions.

Given a model of TSTU in which the sets implementing the types are disjoint (a model not satisfying this condition is readily modifiable to one which does), extend its language to a one-sorted language with the same variables
by the device of assigning the value False to each ill typed atomic formula and interpreting complex formulas in the natural way.

All of the axioms of TTGV are obviously true in this structure for the language of TTGV except the axiom of separation.  The problem with separation is that it asserts the existence of $\{x \in A:\phi\}$ for formulas
which do not correspond to formulas for which this set is provided by the comprehension of TST.

We have foresightfully provided for this by proving in the first section that every set $\{x \in \tau(u):\phi\}$ with fixed values for its parameters is provably equal to a set $\{x \in \tau(u):\phi\}$ in which $\phi^*$ is well-typed, and in this context a well-typed formula is exactly equivalent to a well-formed formula of the underlying TSTU.

The system of Resnik is interpretable if the model of TSTU is a model of TST.  This is direct, as Resnik's system differs very little from ours.

The system of Quine is close to TST in allowing only the types that TST itself has.  Some application of a theorem similar to our well-typedness theorem will be needed, because the comprehension axiom of Quine's system is not restricted to well-typed formulas.

The system of Forster shines here, because its comprehension axiom provides for exactly the sets which the comprehension scheme of TST provides for, and also there is no need for the condition that the types are disjoint which is important in our proof above.

Quine's system is in some sense exactly equivalent to TST (apart from the possibility of objects not satisfying any of the predicates $T_n$, about which the theory says nothing).

The other theories cannot be said to be exactly equivalent to TST, because they do not restrict themselves to the hierarchy of types indexed by the natural numbers which TST supports, and in fact their language cannot even express such a restriction.

We describe a theory TSTG which is typed and in some sense exactly equivalent to TTGV.

TSTG is a first order multisorted theory (or family of theories) with sorts of two kinds, $\tau^+(l,n)$ where $l$ is a label and $n$ is a natural number, and $\tau(l,i)$ where $l$ is a label and $i$ is an integer (a version of the theory might have only one of these kinds of type).  For any type $t$ we define $t^+$ as $\tau^+(l,n+1)$ if $t=\tau^+(i,n)$ and as $\tau(l,i+1)$ if $t=\tau(l,i)$.  

An atomic subformula $u=v$ iff the types of $u$ and $v$ are the same; $u \in v$ in which $u$ is of type $t$ is well-formed if and only if the type of $v$ is $t^+$.

The weak extensionality and comprehension axioms of TSTG are the complete schemes of formulas of the same shapes given for TSTU, with the additional latitude afforded by having more types.

In effect, we are providing for an arbitrary large collection of models of TST and an arbitrarily large collection of models of TZT.  This is a family of theories because we have not stipulated how many labels there are for types of each kind.

Now there is a direct translation between models of TTGV and models of TTGV.  From a model of TTGV obtain a model of TSTG in which the sets implementing the types of TTGV are the extensions of the sets $\tau(x)$ in the model of TTGV, and the $(t \mapsto t^+)$ operation on type labels parallels the $\tau$ operation on types in the sense of TTGV (to realize the type labels, one needs to make a choice of ``base type" in each orbit in $\tau$ without a minimal element;  this is not an essential use of choice because we could also allow many interconvertible notations for each type in a sequence of types indexed by all integers).  This is readily seen to be a model of TSTG.

A model of TSTG is converted to a model of TTGV by assigning values to all ill-typed atomic formulas of False
(ensuring first that the sets implementing the types are pairwise disjoint) and extending the definition of truth values of general formulas appropriately.  Again, the only axiom of TTGV which requires care to verify in the resulting structure is separation, and its validity follows from the fact
that general set abstracts in TTGV are equivalent to well-typed set abstracts.

This result adapts to the theory of Resnik which has the added assumptions of strong extensionality and existence of individuals.

Consideration of the theory TSTG can be useful in thinking out things about TTGV.  Notice that
an arbitrary set of models of TSTG can easily be made a pairwise disjoint set, and the union of a pairwise disjoint collection of models of TSTG is a model of TSTG.  TTGV has similar properties.

One way to get the condition that any well-typed formula $\phi(\tau(y)/x)$ in which $x$ is the only free variable in $\phi$
will hold for all values of $\tau(y)$ if it holds for all but a concrete finite collection of values of $\tau(y)$, which is sufficient to make every formula equivalent to the partial universal closure of a well-typed formula, is to replace the model of TTGV in which one works with a countable union of pairwise disjoint copies of the model one starts with.

Another observation is that we cannot (verifying a claim we made above) establish that arithmetic is the same everywhere in a model of TTGV, because we can take unions of models of TST with different arithmetic facts and convert them to a model of TTGV.

Note that consistency of NFU (and of NF) implies consistency of TTGV (even with strong extensionality) with the Ambiguity Scheme which asserts that for any  formula $\phi$ in which $x$ is the only free variable, $$(\forall uv:\phi[\tau(u)/x]\leftrightarrow \phi[\tau(x)/x]).$$  A model of TSTU with a type shifting endomorphism, which exists by the results of Jensen and Specker, converts to a model of TTGV in which this is true.

An interesting footnote to this section is that TTGV is not finitely axiomatizable.  The separation axiom for typed formulas is finitely axiomatizable:  this can be done for example by converting the axioms of Hailperin's finite axiomatization of NF to well-typed formulas closed with a quantifier over all types.  The equivalence of set abstracts over general formulas to set abstracts over well-typed formulas depends also on the disjointness of the types, which depends on a countable collection
of instances of TTGV comprehension (the ones defining the sets $R_x$ in the proof).  Suppose that TTGV was finitely axiomatizable.  Each of the axioms in this finite axiomatization would be provable using the axioms implemented well-typed separation and finitely many of the axioms providing for $R_x$'s (and the other axioms of TTGV).  Thus there would be a finite axiomatization consisting of the well-typed separation axioms, the other axioms of TTGV and finitely many of the axioms providing $R_x$'s.  But all of these axioms hold in a model of cyclic type theory with weak extensionality with a large enough finite number of axioms:  because NFU is consistent, this theory is consistent.  And this theory does not cover TTGV because it does not prove existence of one of the $R_x$'s.

\section{Set theory without the axiom of pairing}



\end{document}





