\chapter[The Set Concept]{The Set Concept.\\
Extensionality. Atoms}\index{set}\index{extensionality}


If the reader glances at the Introduction, he may
expect an unusual treatment of set theory and its use as a foundation
for mathematics in these pages.  If our premise is correct, this will
not be the case.  The basic fact that mathematics is founded on
the undefined concepts of {\itshape set\/} and {\itshape
membership\index{membership}\/} is unchanged.  The notion of {\itshape
ordered pair\index{ordered pair}\/} is also treated as primitive, but
we will indicate how it could be defined using the set concept. The
techniques which are used to achieve this end differ only in technical
detail from the techniques used in a more familiar treatment.  We
believe that the constructions given here are if anything more natural
than the traditional constructions; that is why this book was written.

What is a set?  This is not a question that we will answer
directly.  ``Set" is an undefined notion.  The reader will have to
count on her intuitive notion of what a ``set" or ``collection" is at
the start.  She may find that some of the properties of the set concept
that we develop here will be somewhat unfamiliar.

A set or collection has members.  The basic relationship between
objects in our theory, written $a \in b$, can be translated ``$a$ is
an element of $b$" or ``$a$ is a member\index{membership} of $b$" or ``$a$
belongs to $b$".  One way in which it should {\itshape not\/} be translated is
``$a$ is a part of $b$".  The subset\index{subset} or
inclusion\index{inclusion} relation, to be introduced
later, is a better translation for the intuitive relation of part to
whole\index{relation!of part to whole}.  This is the major intuitive pitfall
with set theory; a set does not have its elements as parts.  An important
difference between the membership relation and the relation of part to whole is
that the latter is transitive but the former is not: if $A$ is a part of $B$
and $B$ is a part of $C$, then $A$ is a part of $C$; but it is not the
case that given $A \in B$ and $B \in C$, $A \in C$ follows.  We will
see counterexamples to this shortly.

A set is exactly determined by its members.  This can be
summarized in our first axiom:

\begin{axiom}{Axiom of Extensionality}
 If $A$ and $B$ are sets, and for each $x$, $x$ is an
 element of $A$ if and only if $x$ is an element of $B$, then $A = B$.
\end{axiom}

The Axiom of Extensionality\index{extensionality!axiom} can be paraphrased in
more colloquial English:  ``Sets with the same elements are the same".

Not all objects in our universe are sets.  Objects which are
not sets are called ``atoms\index{atoms}".  You can think of ordinary physical
objects, for instance, as being atoms.  We certainly do not think of
them as being sets!  Atoms have no elements, since they are not sets:

\begin{axiom}{Axiom of Atoms\index{atoms!axiom}}
  If $x$ is an atom, then for all $y$, $y
  \not\in x$ (read ``$y$ is not an element of $x$'').
\end{axiom}

An advantage of the presence of atoms is that we can suppose
that the objects of any theory (or the objects of the usual physical
universe) are available for discussion, even if we do not know how to
describe them as sets or do not believe that they are sets.  It turns
out that our axioms will allow us to prove the existence of atoms\index{atoms},
which is a rather surprising result!

Observe that the converse is not necessarily true:  if an
object has no elements, we cannot conclude that it is an atom.  It is
possible for there to be sets with no elements.  What we can prove
using the Axiom of Extensionality\index{extensionality} is that there is no
more than one set with no elements:

\begin{thm}
 If $A$ and $B$ are sets, and for all $x$, $x$ is not
 an element of $A$ and $x$ is not an element of $B$, then $A = B$.
\end{thm}

\preuve  For all $x$, if $x$ is an element of $A$, it is an element of $B$
and\linebreak

\pagebreak

vice versa (a false statement implies anything\ldots).  By
Extensionality\index{extensionality}, $A$ and $B$ must be equal.
\finpreuve

We will state axioms shortly which will guarantee the
existence of a set with no elements.  This unique set is called {\itshape the
empty set\index{empty set}\/} and denoted by the symbol $\vide$.

A natural way to specify a set is to take the collection of
all objects with some property\index{properties}.  For example, we could
consider the 
collection of all prime numbers greater than 17.  This is a technique
of specifying sets which has pitfalls (we will see some of these in
later chapters), and we will not use it as our technique of choice for
building sets at first, but it does inspire a very useful
notation for sets.  The set of all prime numbers greater than 17 can
be written $\{x \st x$ is a prime number and $x$ is greater than
17$\}$; in general, the collection of all objects with property $\phi$
can be written $\{x \st x$ has property $\phi\}$, which is read ``The
set (or class, or collection) of all $x$ such that $x$ has property
$\phi$".  The choice of the letter representing elements of the
proposed set is indifferent: the expressions $\{x \st x$ has property
$\phi\}$, $\{y \st y$ has property $\phi\}$ and $\{A \st A$ has
property $\phi\}$ are exactly the same (as long as appropriate
substitutions of $y$ or $A$ for $x$ are made in the description of the
property $\phi$: $\{x \st x$ is a prime number and $x$ is greater than 17$\}$
is equivalent to $\{A \st A$ is a prime number and $A$ is greater than
17$\}$).

It will turn out that not every set of the form $\{x \st x$
has property\index{properties} $\phi\}$ actually exists; there are
properties which {\itshape cannot\/} define sets.  We will introduce such a
property eventually.  As a consequence, we will not use the general
technique of collecting all objects with a given property to build
sets; we will use a number of basic (and natural) constructions which
experience indicates are safe to build sets.  We will eventually use
properties of these basic constructions to prove a theorem showing
that a large class of properties (the
``stratified\index{stratification}" properties) do in fact define
sets, and after the proof of this theorem we will use sets of the form
$\{x \st x$ has property $\phi\}$ much more freely.  Some of the
basic constructions which we use are incorrectly considered to be
``dangerous" because they lead to problems in the context of the
usual\index{Zermelo--Fraenkel set theory} set theory; for instance,
this set theory has a universal set\index{universe, universal set}.
Illustrating the fact that such constructions are not dangerous is one
of the aims of this work.  

If the reader feels that arbitrary collections of objects of our
theory {\itshape must\/} exist in some sense, he can understand the
sentence ``the set $\{x \st x$ has property $\phi\}$ does not exist'' as
meaning not that there is no collection of all objects $x$ of our
theory such that $\phi$, but that this collection cannot be regarded as a
set in our theory; such collections which are not sets will sometimes
be discussed (we call them ``proper classes'').



