\chapter{The Structure of the Transfinite}

\section[Cardinals, Ordinals, and $\aleph$]{Cardinals, Ordinals, and
\mathversion{bold}$\aleph$}
\index{cardinal numbers}\index{ordinal numbers}

We begin with some notions and results relating cardinal and ordinal
numbers.

It is an obvious observation that two similar\index{similarity}
well-order-\linebreak
ings\index{well-orderings} are 
well-orderings of a set of the same cardinality, since a similarity
between well-order\-ings induces a bijection on the underlying sets.
Thus, each ordinal is associated with a uniquely determined
cardinal\index{cardinal numbers}, 
and each cardinal is associated with a class of ordinals\index{ordinal numbers}
which are 
order types\index{order type} of well-orderings of sets of that cardinality.
For any 
infinite\index{infinite!cardinal} cardinal, this class of
ordinals\index{ordinal numbers} has many members.  This 
discussion motivates the following

\begin{definition}
 For each ordinal $\alpha$, we use the notation
 $\card(\alpha)$ to represent the cardinal\index{cardinal numbers} of sets
 well-orderable\index{well-orderings} with 
 order type\index{order type} $\alpha$, and for each cardinal $\kappa$, we use
 the notation $\init(\kappa)$ to denote the smallest ordinal $\alpha$ such
 that $\card(\alpha) = \kappa$.  Ordinals\index{ordinal numbers}
 $\init(\kappa)$ are called {\upshape initial ordinals\index{initial
 ordinal|textbf}\/}.
\end{definition}

There is a one-to-one correspondence between cardinals\index{cardinal numbers}
and initial ordinals\index{ordinal numbers}.  In the
usual\index{Zermelo--Fraenkel set theory} set theory, initial (von
Neumann\index{von Neumann ordinal}) ordinals and 
cardinals are identified.  The notations introduced in the definition
are not standard, but something like them is necessary in the absence
of the usual identification of cardinals with initial ordinals.

We have already encountered one (necessarily partial) function\index{function}
(the exponential map) which sends each cardinal in its domain\index{domain} to
a strictly larger cardinal\index{cardinal numbers}.  We now introduce another
natural such function.

\begin{definition}
 For each cardinal $\kappa$, we define $\kappa+$ as
 the smallest cardinal greater than $\kappa$, i.e., the
 successor\index{successor!cardinal} of 
 $\kappa$.  That cardinals (other than $|V|$) have successors is a
 consequence of the Axiom of Choice\index{axiom of choice}, which implies that
 the natural order on cardinals\index{cardinal numbers} is a
 well-ordering\index{well-orderings}. 
\end{definition}

The assumption that this function\index{function} is actually identical to the
exponential map, called the Generalized Continuum Hypothesis
(GCH)\index{Generalized Continuum Hypothesis}, is 
consistent with our set theory, but not a consequence of our axioms.
Cantor's original Continuum Hypothesis is the assertion
$\aleph_0+=2^{\aleph_0}$.  Although we cannot prove that the
successor\index{successor!cardinal} 
operation on cardinals\index{cardinal numbers} is equivalent to the exponential
map, we can prove that it is equivalent (in the presence of the Axiom of
Choice\index{axiom of choice}) 
to a certain natural operation on cardinals.

\begin{definition}
 Let $\kappa$ be a cardinal.  We define
 $\aleph(\kappa)$ as the result of applying $T^{-2}$ to the
 cardinality\index{cardinal numbers} of the collection of
 ordinals\index{ordinal numbers} $\alpha$ such that $\card(\alpha)\leq\kappa$.
 The use of $T^{-2}$ ensures that $\aleph$ 
 is a function, known as the {\upshape Hartogs aleph function\index{Hartogs
 aleph function}}.
\end{definition}

\begin{Thm}{Theorem (Hartogs)\index{Hartogs Theorem}}
 For all cardinals $\kappa$, the cardinality of
 the collection of ordinals $\alpha$ such that
 $\card(\alpha)\leq\kappa$ is $T^2\{\kappa\}+$.  
\end{Thm}

\begin{cor}
 For all cardinals\index{cardinal numbers} $\kappa\neq|V|$, $\kappa+ =
 \aleph(\kappa)$.  The proof of the Corollary requires nothing more
 than the observation that the successor\index{successor!cardinal} and Hartogs
 aleph operations commute with $T$.
\end{cor}

{\sc Proof of the Theorem.~---} $|\seg\{\alpha\}| =
T^2\{\card(\alpha)\}$\index{segment}
(to see this, look at the discussion of order types\index{order type} of
segments of the ordinals\index{ordinal numbers} above).  It is clear from this
that the cardinality of the 
set of interest is $\geq T^2\{\kappa\}$.  It is also clear that the
cardinality\index{cardinal numbers} of every proper initial segment of the set
of interest (so every cardinal less than the cardinality of the set of
interest) is $\leq T^2\{\kappa\}$.

The only thing that remains to be shown is that the cardinality of the
set of interest is not equal to $T^2\{\kappa\}$.  Suppose it were.
Let $\beta$ be the first ordinal not in the set of interest.  We are
assuming that $|\seg\{\beta\}| = T^2\{\kappa\}$\index{segment}, from which it
follows that $\card(\beta) = \kappa$, which implies that $\beta$
is in the set of interest, which is absurd.  We need to consider the
special case $\kappa=|V|$, for which the ordinal $\beta$ does not
exist; note that $\Omega$ is the first ordinal which is not the order
type of a proper initial segment of the ordinals\index{ordinal numbers}, and
that the order types of proper initial segments of the ordinals are exactly the
ordinals of the form $T^2\{\alpha\}$; this is enough to establish that
$\card(\Omega)$ must be the first cardinal\index{cardinal numbers} greater than
all cardinals of the form $T^2\{\kappa\}$, that is, $T^2\{|V|\}+$. The proof of
the Theorem is complete.
\finpreuve

The Hartogs operation makes sense in the absence of the Axiom of
Choice\index{axiom of choice}, but no longer coincides with the notion of
successor\index{successor!cardinal}.  Without 
the Axiom of Choice, the argument above proves that $\aleph(\kappa)$
is greater than any cardinal\index{cardinal numbers} of a
well-orderable\index{well-orderings} subset of a set of 
size $\kappa$; it cannot be $\leq\kappa$, but it may be incomparable
with $\kappa$ rather than $>\kappa$.  It should be noted that
$\aleph(\kappa)$ is certainly the cardinal of a well-orderable set!

We define the usual notation for cardinals, which also uses the letter
$\aleph$:

\begin{definition}
 We define $\aleph_{\alpha}$ as the cardinal\index{cardinal numbers}
 $\kappa$ such that the natural order on infinite\index{infinite!cardinal}
 cardinal numbers restricted to the cardinals $< \kappa$ has order
 type\index{order type} $\alpha$.
\end{definition}

Note that the type\index{types (relative)} of $\alpha$ is two types higher than
the type of 
$\aleph_{\alpha}$.  If $\aleph$ were a symbol for the natural
well-ordering\index{well-orderings} on infinite cardinals\index{cardinal
numbers} (which it is not) this would be an 
example of the definition of ordinal indexed
sequences\index{sequence!transfinite}. 

The notation $\aleph_0$ already introduced for $|{\cal N}|$ is a
special case of this.  We see from Hartogs's theorem that $\aleph_1$
is the cardinality of the set of all order types\index{order type} of
well-orderings of 
countable\index{countable} sets (countable ordinals\index{ordinal numbers}):
$\aleph_1 = \aleph_0+$ = the 
image under $T^{-2}$ of the cardinality of the set of countable
ordinals; the latter set is Cantorian\index{Cantorian!set}, so this is simply
the 
cardinality\index{cardinal numbers} of the set of countable ordinals.  For each
natural number\index{natural number} 
$n$, we have that $\aleph_{n+1}$ is the cardinality of the set of
order types\index{order type} of sets of cardinality $T^{-2}\{\aleph_n\} =
\aleph_{T^{-2}\{n\}}$ (if this exists!).  For each concrete natural
number $n$, we can establish without an appeal to the Axiom of
Counting\index{Axiom of Counting} or the Axiom of Small Ordinals\index{Axiom of
Small Ordinals} that $T^{-2}\{\aleph_n\} = 
\aleph_n$.  Without at least the Axiom of Counting we cannot conclude
that $\aleph_n$ exists for each $n$; it is consistent with our other
axioms that $\aleph_n = |V|$ for some $n$, in which case the
cardinality\index{cardinal numbers} of the set of order types of
well-orderings\index{well-orderings} of sets of 
size $|V|$ ($\card(\Omega)$) would be $\aleph_{T^2\{n\}+1}$, from which
we would see clearly that $T^2\{n\} < n$.  This argument in the
presence of the Axiom of Counting (or of the Axiom of Small Ordinals)
demonstrates that $|V|$ is not of the form $\aleph_n$ for any natural
number $n$, and incidentally demonstrates the existence of
$\aleph_{\omega}$.

Although such pathologies are averted at low levels by the Axiom of
Small Ordinals\index{Axiom of Small Ordinals}, it is the case that $|V| =
\aleph_{\alpha}$ for {\itshape some\/} 
ordinal $\alpha$; $\card(\Omega) = \aleph_{T^2\{\alpha\}+1}$ for this
ordinal $\alpha$, which implies $T^2\{\alpha\} < \alpha$; this ordinal
is a non-Cantorian ordinal\index{Cantorian!ordinal}.  The Axiom of Small
Ordinals allows us to deduce that for each Cantorian $\beta$, $\aleph_{\beta}$
exists, since we see that the index of the last cardinal\index{cardinal
numbers} is non-Cantorian, and we know that all Cantorian
ordinals\index{ordinal numbers} are less than any non-Cantorian 
ordinal.

We define another sequence\index{sequence!transfinite} of cardinals, which is
identical with the sequence of alephs if the Generalized Continuum
Hypothesis\index{Generalized Continuum Hypothesis} holds.

\begin{definition}
 We define $\beth_{\alpha}$ as $W_{\alpha}$ for $W$
 a certain well-ordering\index{well-orderings} which we now describe (this use
 of $W$ is a nonce notation!).  $W$ is the restriction of the natural
 well-ordering on cardinal\index{cardinal numbers} numbers to the
 intersection\index{intersection!set} of all sets $B$ of cardinal 
 numbers which contain $\aleph_0$ and are closed under the function\index{function} $\exp$ (if $\kappa
 \in B$, 
 $\exp(\kappa) \in B$) and the operation of taking suprema of
 subsets\index{subset} (if $A \subset B$ and $A$ is bounded above in $B$, the
 least upper bound of $A$ in the natural order on the cardinals is an element
 of $B$).
\end{definition}

This sequence\index{sequence!transfinite} looks like the sequence of
$\aleph_{\alpha}$'s, except that the ``successor\index{successor!cardinal}''
operation is exp instead of the Hartogs aleph. 
As is the case for any ordinal-indexed sequence, the type\index{types
(relative)} of $\beth_{\alpha}$ is two types lower than the type of $\alpha$.



\section[Solovay's Theorem]{Classification of Cardinals and Solovay's Theorem
on Inaccessibles} 
\index{cardinal numbers}\index{inaccessible cardinals}

We classify cardinals.  A cardinal of the form $\aleph_{\alpha+1}$ is
called a {\itshape successor\index{successor!cardinal} cardinal\index{successor
cardinal}\/}.  A cardinal of the form $\aleph_{\lambda}$, $\lambda$ a limit
ordinal, is called a {\itshape limit
cardinal\index{limit cardinal}\/}.

The next step requires a 

\begin{definition}
 Let $\leq$ be a well-ordering\index{well-orderings}.  A subrelation
 $\leq^*$ of $\leq$ which is also a well-ordering is said to be {\upshape
 cofinal} in $\leq$ iff for each element $x$ of $\dom(\leq)$ there is
 an element $y$ of $\dom(\leq^*)$ such that $x \leq y$.  The {\upshape
 cofinality\index{cofinality}} of $\leq$ is the smallest order
 type\index{order type} of a subrelation cofinal in $\leq$.

 If $\kappa$ is a cardinal\index{cardinal numbers}, we define the {\upshape
 cofinality} of $\kappa$, written $\cf(\kappa)$ as the smallest
 cardinal $\mu$ which contains the domain of some well-ordering cofinal in
 a well-ordering of order type $\init(\kappa)$.
\end{definition}

We use the notion of cofinality to further classify cardinals:

\begin{definition}
 A cardinal\index{cardinal numbers} $\kappa$ is said to be {\upshape
 regular\index{regular cardinal}} 
 if $\cf(\kappa) = \kappa$.  Cardinals which are not regular are said
 to be {\upshape singular\index{singular cardinal}}.
\end{definition}


\begin{thm}
 Cofinalities of cardinals are regular cardinals.
\end{thm}

\preuve  Let $\kappa$ be a cardinal.  Let $\leq$ be a well-ordering of order type $\init(\kappa)$.  Let $\leq_2$ be a well-ordering cofinal
in $\leq$ of minimal order type.   For each element $w$ of the domain of $\leq$, define $w'$ as the $\leq_1$-least element of the domain
of $\leq_2$ such that $w \leq_1 w'$.   Let $\leq_3$ be a well-ordering of the domain of $\leq_2$ with order type $\init(\cf\kappa))$.
For $x,y$ in the domain of $\leq_1$, define $x \leq_4 y$ as holding iff $x' \leq_3 y'$ or ($x'=y'$ and $x\leq_1 y$).   The order $\leq_4$ has domain of size $\kappa$ and a cofinal well-ordering $\leq_3$ of order type $\init(\cf(\kappa))$.  Further, every initial segment of this well-ordering is of size
$<\kappa$.   Fix $x$ in the domain of $\leq_1$:  we will show that the segment of $x$ in $\leq_4$ is of size $<\kappa$.  Let $\lambda$ be the size of the segment in $\leq_3$ of $x'$:  $\lambda<\cf(\kappa)$.   Let $\mu$ be the least cardinal which is $\geq$ the cardinality of the segment in $\leq_1$ of each $y \leq_3 x'$:  $\mu <\kappa$ as otherwise we would have a sub-well-ordering of $\leq_2$ with domain of size $<\cf(\kappa)$ (and so certainly of smaller order type than $\leq_2$) which was cofinal in $\leq_1$.  Each $y \leq_4 x$
satisfies $y \leq_1 z$ for some $z \leq_3 x'$ ($z$ being $y'$), so the segment of $x$ in $\leq_4$ is no larger than the union of $T\{\lambda\}$ segments
each of size no larger than $\mu$, so is of size no larger than $\lambda\cdot\mu <\kappa\cdot\kappa=\kappa$.  Since each segment in $\leq_4$
has domain of cardinality $<\kappa$, the order type of $\leq_4$ is also $\init(\kappa)$, from which it follows that $\alpha=\init(\cf(\kappa))$,
from which it follows that any order type of a sub-well-ordering cofinal in a well-ordering of order type $\init(\cf(\kappa))$ is also the order type of a sub-well-ordering cofinal in a well-ordering of order type $\init(\kappa)$, whence $\cf(\cf(\kappa))=\cf(\kappa)$, so $\cf(\kappa)$ is regular.

\finpreuve


\begin{thm}
 Successor\index{successor!cardinal} cardinals\index{cardinal numbers} are
 regular.
\end{thm}

\preuve\ Given Hartogs's Theorem, it is sufficient to prove that
for any $\kappa$, any sequence\index{sequence!transfinite} cofinal in the
natural well-ordering\index{well-orderings} on ordinals\index{ordinal numbers}
$\alpha$ with card$(\alpha) \leq \kappa$ has the same 
cardinality as the set of such ordinals $\alpha$ (that is,
$T^2\{\kappa+\}$).  Suppose otherwise.  We would then have a
well-ordering with full domain\index{domain!full} of size $\leq
T^2\{\kappa\}$
purportedly cofinal in this sequence of ordinals.  But this is
impossible: the size of the set of ordinals less than or equal to some
element of this cofinal sequence is clearly $\leq T^2\{\kappa\}
\times T^2\{\kappa\}=  T^2\{\kappa\}$, which is incompatible with
the assumption that the sequence is cofinal in a segment\index{segment} of the 
ordinals\index{ordinal numbers} of size $T^2\{\kappa+\}$.  The proof of the
theorem is complete.
\finpreuve

An example of a singular limit cardinal\index{cardinal numbers} is
$\aleph_{\omega}$, which has cofinality $\aleph_0$.


For our next definition, we need a stronger notion of ``limit
cardinal''.

\begin{definition}
 A cardinal\index{cardinal numbers} $\kappa$ such that for each $\mu <
 \kappa$, $2^{\mu} < \kappa$ is said to be a {\upshape strong limit}
 cardinal.  It should be clear that a strong limit cardinal is a
 limit\index{strong limit cardinal} cardinal.
\end{definition}

\begin{definition}
 An uncountable\index{uncountable} regular strong limit cardinal is
 called an {\upshape inaccessible\index{inaccessible cardinals!definition}}
 cardinal.  (Of course $\aleph_0$ is a regular strong limit cardinal!)
\end{definition}

\begin{thm}
 A complete rank of $Z_0$ indexed by $\init(\kappa)$,
 $\kappa$ inaccessible, is a model of ZFC\index{Zermelo--Fraenkel set
 theory} (using the relation 
 $E$ to code membership\index{membership} as usual).
\end{thm}

\preuve\ The only axioms which require real
attention are Power Set\index{power set} and
Replacement\index{Replacement(axiom of ZFC)}.  Power Set holds because the 
ordinal init($\kappa$) is a limit ordinal (a set appearing at one
stage has its power set appearing at the next stage, and there is no
last stage).  We see that Replacement holds by considering a set $A$
of sets realized in the rank which can be placed in one-to-one
correspondence with a set $B$ realized in the rank; the common size of
these sets will be less than $\kappa$, and the collection of ranks at
which elements of $A$ appear cannot be cofinal in the ordering of
ranks below rank $\init(\kappa)$, because $\kappa$ is regular; thus
there is a rank below rank $\init(\kappa)$ at which all the elements of
$A$ are present, and the set $A$ is realized at that rank.
\finpreuve

The rest of this section is devoted to the proof of the following
theorem of Robert Solovay:

%It is possible to use Solovay's trick to prove the theorem without any
%reference to $Z_0$.

\begin{Thm}{Theorem (Solovay)}
 There is an inaccessible\index{inaccessible
 cardinals} cardinal\index{cardinal numbers}.
\end{Thm}

The only consequence of the Axiom of Small Ordinals\index{Axiom of
Small Ordinals} needed for the proof of Solovay's theorem is the
assertion that each Cantorian\index{Cantorian!set} set (resp. ordinal,
cardinal) is strongly Cantorian.  Note that the Axiom of Counting\index{Axiom
of Counting} follows from this directly.

{\sc Proof of Solovay's Theorem.~---}
Fix a well-ordering $\leq^*$ of the set of sub-well-orderings
of the natural well-ordering on the cardinals.  For any
sub-well-ordering $\leq_0$ of the natural order on the cardinals, we
define $T[\leq_0]$ as $\{(T\{\kappa\},T\{\lambda\})\st \kappa
\leq_0 \lambda\}$.  We define $T[\leq^*]$ as
$\{(T[\leq_0],T[\leq_1])\st [\leq_0] \leq^* [\leq_1]\}$.

We define a function $F$ with domain a subset of the set of pairs of
cardinal numbers as follows:

\begin{enumerate}
 \item If $\alpha_0$ and $\beta_0$ are distinct cardinals which are not
   strong limit\index{strong limit cardinal}, then $F(\alpha_0,\beta_0) =
   (\alpha_{1},\beta_{1})$, 
   where $\alpha_{1}$ and $\beta_{1}$ are the least cardinals
   such that exp($\alpha_{1}$) $\geq$ $\alpha_0$ and
   exp($\beta_{1}$) $\geq$ $\beta_0$.

 \item If $\alpha_0$ and $\beta_0$ are distinct singular strong limit
   cardinals, and $\cf(\alpha_0) \neq \cf(\beta_0)$, define
   $F(\alpha_0,\beta_0)$ as $(\cf(\alpha_1),\cf(\beta_1))$.

 \item If $\alpha_0$ and $\beta_0$ are distinct singular strong limit
   cardinals\index{cardinal numbers}, and $\cf(\alpha_0) = \cf(\beta_0)$,
   we proceed as follows.  Let $\leq^0$ be the first well-ordering of
   minimum possible length cofinal in the natural order on the cardinals
   less than $\alpha_0$, where ``first'' is in the sense of the order
   $\leq^*$ fixed above.  Let $\leq^1$ be the first well-ordering of
   minimum possible length cofinal in the natural order on the cardinals
   less than $\beta_0$, where ``first'' is in the sense of the order
   $T[\leq^*]$ defined above (if $\beta_0$ is not of the form
   $T\{\gamma\}$ for some cardinal $\gamma$, this clause of the definition
   does not succeed, and $F(\alpha_0,\beta_0)$ is undefined; this will
   not occur in cases of interest to us).  These two well-orderings will
   be of the same order type $T^2\{\init(\cf(\alpha_0))\} =
   T^2\{\init(\cf(\beta_0))\}$ (the $T^2$ operator appears because
   cardinals and ordinals are two types higher than the objects they
   ``count''), and will be distinct, since they are cofinal in the
   natural order on different initial segments of the cardinals.  Let
   $\delta$ be the smallest ordinal such that $[\leq^0]_{\delta} \neq
   [\leq^1]_{\delta}$ (there must be such a $\delta$ by the preceding
   considerations); define $F(\alpha_0,\beta_0)$ as
   $([\leq^0]_{\delta},[\leq^1]_{\delta})$.

 \item In any other case, $(\alpha_{0},\beta_{0})$ is excluded from
   the domain of $F$.  This includes in particular the case where
   $\alpha_{0}$ and $\beta_{0}$ are inaccessible cardinals\index{inaccessible
   cardinals}.
\end{enumerate}

The function $F$ has a stratified definition, and so is a set.  By a
slight modification of the Recursion Theorem for natural numbers, we
can define a (possibly finite) sequence\index{sequence} $p$ of pairs of
cardinals which has any desired pair of cardinals as $p_0$ and satisfies the
condition that for each $i \in {\cal N}$ we have $p_{i+1}=F(p_i)$ if
$p_i \in $ dom$(F)$ and undefined otherwise.  We write $p_i$ as
$(\alpha_i,\beta_i)$.

To completely determine $p$, we need to choose its starting point $p_0
= (\alpha_0,\beta_0)$.  Let $\alpha_0$ be any
non-Cantorian\index{Cantorian!non-, cardinal} cardinal $\alpha$, and let
$\beta_0$ be $T\{\alpha\}$. 

We want to argue by induction that the conditions $\beta_i = 
T\{\alpha_i\} \neq \alpha_i$ hold for each $i$ in the domain of $p$.  This
condition is unstratified:  however, an appeal to the Axiom of Counting shows
that it is equivalent to the stratified condition $\beta_{T\{i\}} =
T\{\alpha_i\} \neq \alpha_{T\{i\}}$, on which induction is permitted (strictly
speaking, this condition is still not stratified; it is a substitution instance
of a stratified sentence in which the same object $\alpha = \pi_1 \circ p$
replaces two parameters of different relative type; but this does not prevent
it from defining a set).

The step that requires particular care is the step handling a pair of
singular strong limit cardinals\index{strong limit cardinals}\index{cardinal
numbers} of the same cofinality.

\begin{enumerate}
 \item The condition clearly holds in the basis case.

 \item If $\alpha_i$ and $\beta_i$ are cardinals\index{cardinal
   numbers} which are not strong limit\index{strong limit cardinals}, and
   $\beta_i =  T\{\alpha_i\} 
   \neq \alpha_i$, it is clear that $\beta_{i+1} =  T\{\alpha_{i+1}\}
   \neq \alpha_{i+1}$; the operation described clearly commutes with $T$,
   and so if the cardinals $\alpha_{i+1}$ and $\beta_{i+1}$ chosen were
   equal, they would have to be Cantorian\index{Cantorian!cardinal}, so the
   images under exp of the cardinals would be Cantorian and dominate the
   non-Cantorian  $\alpha_i$ and $\beta_i$, which is absurd.

 \item If $\alpha_i$ and $\beta_i$ are singular strong limit cardinals,
   $\cf(\alpha_i) \neq \cf(\beta_i)$, and $\beta_i =  T\{\alpha_i\}
   \neq \alpha_i$, it is clear that $\beta_{i+1} =  T\{\alpha_{i+1}\}
   \neq \alpha_{i+1}$; all that is required is to note that cf commutes
   with $T$ (i.e., $T\{\cf(\kappa)\} = \cf(T\{\kappa\}$) for any
   cardinal\index{cardinal numbers} $\kappa$).

 \item If $\alpha_i$ and $\beta_i$ are singular strong limit cardinals,
   $\cf(\alpha_i) = \cf(\beta_i)$, and $\beta_i =  T\{\alpha_i\} \neq
   \alpha_i$, we choose the first well-orderings\index{well-orderings} of
   shortest length cofinal in the natural order on cardinals below
   $\alpha_i$ and $\beta_i$ respectively with respect to the orders
   $\leq^*$, $T[\leq^*]$, respectively; the operation described
   ``commutes with T'' in a suitable sense by the definition of
   $T[\leq^*]$.  The common cofinality \index{cofinality} of $\alpha_i$ and
   $\beta_i$ must be Cantorian\index{Cantorian!cardinal}, since the cf operator
   commutes with $T$, and 
   $\init(\cf(\alpha_i)) = \init(\cf(\beta_i))$ will be the common length
   of the chosen well-orderings of the cardinals below $\alpha_i$ and
   $\beta_i$ (the fact that the cofinality and thus its initial
   ordinal\index{initial ordinal} are seen to be
   Cantorian\index{Cantorian!cardinal} allows us to omit applying the $T^2$
   operator to the order types as we had to do in the discussion in the
   corresponding case of the recursive definition). $\alpha_{i+1}$ and
   $\beta_{i+1}$ are chosen to be the first pair of corresponding
   elements of the domain of these well-orderings\index{well-orderings}
   that differ; suppose that they are the $\delta$th elements of these
   well-orderings (i.e., $\delta$ is the order type\index{order type} of
   the segments\index{segment} they determine).  $T\{\alpha_{i+1}\}$ will
   be the $T\{\delta\}$th element of the cofinal well-ordering associated
   with $\beta_i$, because the cofinal well-ordering associated with
   $\beta_i = T\{\alpha_i\}$ is the image under $T$ of the cofinal
   well-ordering associated with $\alpha_i$ (in the sense of image under
   $T$ defined above for well-orderings); but $T\{\delta\}  = \delta$,
   because $\delta$ is less than the initial ordinal\index{initial ordinal}
   associated with the Cantorian\index{Cantorian!cardinal} common cofinality of
   the two original cardinals\index{ordinal numbers} and so is Cantorian.  Thus
   $T\{\alpha_{i+1}\} = \beta_{i+1}$ as desired.

 \item In any other case, $\alpha_{i+1}$ and $\beta_{i+1}$ will be
   undefined, which poses no problem for our induction.
\end{enumerate}

It is an immediate consequence of this result that each pair of
cardinals $(\alpha_i,\beta_i)$ is a pair of cardinals of the same kind
(both not strong limit, both singular strong limit, or both
inaccessible\index{inaccessible cardinals}), since the $T$ operation preserves
each of these properties of cardinals.

Since corresponding projections\index{projections} of the pairs of
cardinals\index{cardinal numbers} strictly decrease as the index
increases, the sequence\index{sequence} $p$ must be finite.  The only way it
can terminate is with a pair of distinct non-Cantorian\index{Cantorian!non-,
cardinal} inaccessible\index{inaccessible cardinals} cardinals\index{cardinal
numbers}.  The proof of Solovay's theorem is complete.
\finpreuve

The theorem establishes that our set theory proves the existence of
many inaccessible cardinals, and so is considerably stronger than
ZFC\index{Zermelo--Fraenkel set theory}.



\section[Stronger Results]{Stronger Results. The Axiom of Large\\ Ordinals}

Solovay has proved more than this.  We introduce some new concepts:

\begin{definition}
 A subset\index{subset!closed, of the domain\index{domain} of a well-ordering}
 of the domain of a well-ordering\index{well-orderings} is said 
 to be {\upshape closed} if it contains the least upper bound\index{bound!least
 upper} of each of its non-cofinal subsets.  A subset of the domain of a
 well-ordering is said to be a {\upshape club\index{club}} if it is closed and
 cofinal in the well-ordering.  A subset of the domain of a well-ordering is
 said to be {\upshape stationary\index{stationary set}} if it meets every club
 in the domain of the well-ordering.
\end{definition}

\begin{definition}
 A cardinal\index{cardinal numbers} $\kappa$ is said to be {\upshape Mahlo} if
 the set of inaccessible\index{inaccessible cardinals} cardinals less than
 $\kappa$ is stationary in the 
 natural well-ordering\index{well-orderings} of cardinals $< \kappa$.  A Mahlo
 cardinal is also said to be {\upshape 1-Mahlo\index{Mahlo cardinals}}; we
 define an {\upshape $(n+1)$-Mahlo 
 cardinal} (for each natural\index{natural number} number $n$, by induction) as
 a cardinal $\kappa$ such that the set of $n$-Mahlo cardinals less than
 $\kappa$ is stationary in the natural well-ordering of the cardinals $<
 \kappa$.
\end{definition}

\begin{Thm}{Theorem (Solovay)\index{Solovay's Theorem}}
 The strength of the set theory whose axioms
 are Extensionality\index{extensionality}, Ordered Pairs\index{ordered pair},
 Stratified Comprehension\index{Stratified Comprehension Theorem},
 Choice\index{axiom of choice}, 
 and ``each Cantorian\index{Cantorian, strongly!set} set is strongly
 Cantorian'' is exactly the same 
 as the strength of ZFC\index{Zermelo--Fraenkel set theory} $+$ ``there is an
 $n$-Mahlo cardinal\index{cardinal numbers}'' 
 for each $n$ (notice that this is a list of axioms for each concrete
 $n$; the assertion ``for each $n$, there is an $n$-Mahlo cardinal'' is
 {\upshape not} a consequence of this theory).  This means that each of
 these theories can interpret the other.  It is also the case that the
 subset of our set theory described proves the existence of $n$-Mahlo
 cardinals for each concrete $n$.
\end{Thm}


The proof of the Theorem is not given here.

We close by introducing and motivating a final axiom.

\begin{definition}
 A {\upshape T-sequence\index{T-sequence}} is a
 finite\index{finite!sequence}
 sequence $s$ of ordinals\index{ordinal numbers} such that for each $i$,
 $s_{i+1}= T\{s_i\}$ iff $s_{i+1}$\index{$T$ operation!on ordinals} 
 is defined.  We define $T^{\,n}\{\alpha\}$ for any $n \in {\cal N}$ as the
 value of $s_n$ for any T-sequence $s$ such that $s_0 = \alpha$.
\end{definition}

This definition is needed because we have no guarantee that the
natural\index{natural number} numbers of our theory correspond to the concrete
natural numbers (for which we already know how to define $T^{\,n}\{\alpha\}$,
of
course).  The existence and uniqueness of $T^{\,n}\{\alpha\}$ is easily
proved using mathematical induction\index{induction!mathematical, on an
unstratified condition} (on horribly unstratified\index{stratification}
conditions).

\begin{axiom}{Axiom of Large Ordinals\index{Axiom of Large Ordinals}}
 For each non-Cantorian ordinal\index{Cantorian!ordinal}\index{ordinal numbers}
 $\alpha$, there is a natural number $n$ such that $T^{\,n}\{\Omega\} <
 \alpha$.
\end{axiom}

The Axiom of Large Ordinals asserts that the structure of the
``large'' ordinals is as simple and clean as we can manage.  It
clearly implies the assertion ``each Cantorian ordinal is strongly
Cantorian'' and it turns out that it has exactly the same strength as
this assertion in combination with our other axioms (exclusive of the
Axiom of Small Ordinals\index{Axiom of Small Ordinals}).

An immediate corollary of the Axiom of Large Ordinals is

\begin{cor}
 $T\{\alpha\} \leq \alpha$ for all ordinals\index{$T$ operation!on ordinals}
 $\alpha$. 
\end{cor}

The reason that we introduce this final axiom is that it allows a very nice
treatment of proper classes of small ordinals\index{ordinal numbers}.

\begin{definition}
 A {\upshape natural set\index{natural set}} is a set of ordinals $A$ with
 the property that $\alpha \in A$ iff $T\{\alpha\} \in A$ for all
 ordinals $\alpha$.
\end{definition}

\begin{thm}
 Two natural sets have the same elements iff they have
 the same Cantorian\index{Cantorian!ordinal} elements.
\end{thm}

\preuve\ Let $A \neq B$ be natural sets.  Let $\alpha$ be the
smallest element of $A \Delta B$ (the symmetric difference\index{symmetric
difference} of the two 
sets).  Suppose that $\alpha$ is non-Cantorian.  This implies that
$T\{\alpha\}< \alpha$\index{$T$ operation!on ordinals} is not in $A \Delta B$,
i.e., it is either in 
both sets or in neither.  It follows by naturality of $A$ and $B$ that
$\alpha$ is either in both sets or in neither, contrary to assumption.
(the assumption that $T\{\alpha\}< \alpha$, a consequence of the Axiom
of Large Ordinals, is {\itshape not\/} necessary for this proof; if we did
not assume Large Ordinals and had $T\{\alpha\}> \alpha$, we would then
consider the status of $T^{-1}\{\alpha\}$; it would exist and cause
the same problem).
\finpreuve

\begin{thm}
 For any sentence $\phi$, there is a natural set $A$ of
 ordinals\index{ordinal numbers} such that the Cantorian ordinals $x$ belonging
 to $A$ are exactly the Cantorian\index{Cantorian!ordinal} ordinals $x$ such
 that $\phi$.
\end{thm}

\preuve\ By the Axiom of Small Ordinals\index{Axiom of Small Ordinals}, there
is a set $B$ satisfying all the required conditions except possibly naturality.
Consider the smallest element $x$ of the set $B \Delta T[B]$; $x$ is
obviously non-Cantorian.  Let $n$ be a natural\index{natural number} number
such that $T^{-n}\{x\}$ does not exist (the existence of such an $n$ is a
consequence of the Axiom of Large Ordinals). $T^{-n}[B]$ is the
desired natural set.
\finpreuve


The effect of the two preceding Theorems is that there is a precise
one-to-one correspondence between natural sets (which are objects of
our theory) and definable proper classes of Cantorian\index{Cantorian!ordinal}
ordinals\index{ordinal numbers}, which 
are not objects of our theory!  This means that we can
define proper classes of small ordinals (and corresponding natural
sets) in ways which involve quantification over all proper classes of
small ordinals, by replacing the quantifications over proper classes
with quantifications over the corresponding natural sets.  Conditions
involving quantification over all natural sets will of course be
unstratified\index{stratification}.  This suggests that our set theory has
considerable 
strength; the set theory QM (Quine--Morse set theory\index{Quine--Morse set
theory|see {Morse--Kelley set theory} }, also
known as Morse--Kelley set theory\index{Morse--Kelley set theory} ) which adds
quantification over 
proper classes to the machinery of ZFC\index{Zermelo--Fraenkel set theory} is
stronger than ZFC, for example.

We describe an interpretation of a version of ZFC with proper
classes which turns out to have the precise strength of our set
theory.  The interpretation of the {\itshape sets\/} of ZFC will be
the interpretation in the Cantorian\index{Cantorian!ordinal}
ordinals\index{ordinal numbers} which we gave above.
Proper classes of ZFC\index{Zermelo--Fraenkel set theory} will be interpreted
using the natural 
sets extending the corresponding classes of Cantorian ordinals of our
set theory.  A technical point is that a set of the interpreted
ZFC will be coded by a certain Cantorian ordinal, while the class
with the same elements will be coded by a natural set which is
actually a different object.  It is convenient and harmless to pass
over this distinction in the subsequent discussion.

There are von Neumann\index{von Neumann ordinal} ordinals in the interpreted
ZFC\index{Zermelo--Fraenkel set theory}
corresponding to each Cantorian ordinal\index{ordinal numbers} of our set
theory.  There is 
in addition one proper class ordinal, the collection of all ordinals
(which is implemented as the natural set of all ordinals coding von
Neumann ordinals of ZFC).  Let us refer to this ``ordinal'',
the order type\index{order type} of the Cantorian ordinals of our theory, as
$\kappa$. 
The universe\index{universe, universal set} of sets of our interpreted
ZFC\index{Zermelo--Fraenkel set theory} is the stage of
the construction of the cumulative hierarchy of sets indexed by
$\kappa$; call this $V_{\kappa}$.  The classes which we admit in
addition may be interpreted as adding one more stage to the cumulative
hierarchy; we can view the universe of classes (including sets and
proper classes) as representing $V_{\kappa+1}$.  We do not have any
objects constructed at later stages available to us.

Observe that we can define proper classes and sets in our interpreted
ZFC\index{Zermelo--Fraenkel set theory} using sentences which involve
quantifiers not only over 
sets, but over proper classes.  This is a consequence of our ability
to code proper classes of Cantorian\index{Cantorian!ordinal}
ordinals\index{ordinal numbers} in our set theory as 
natural sets, which are objects in our theory; sentences involving
quantification over natural sets can be used to define sets
implementing classes of Cantorian ordinals (an application of the
Axiom of Small Ordinals\index{Axiom of Small Ordinals}, since such sentences
are clearly 
unstratified\index{stratification}!).  The axioms of
Separation\index{Separation (axiom of ZFC)} and
Replacement\index{Replacement(axiom of ZFC)}, which 
concern sets rather than classes, are valid for sentences involving
quantification over classes, because any class of Cantorian
ordinals\index{ordinal numbers} 
which is a subclass of a set or which can be placed in an external
one-to-one correspondence with the elements of a set of Cantorian
ordinals, however defined, is a set (this was shown above).  An
extension of ZFC\index{Zermelo--Fraenkel set theory} which admits proper
classes  and allows quantification over proper classes in definitions of classes and of
sets strengthens ZFC significantly.

Though we do not admit elements of the stage $V_{\kappa+2}$ into our
interpreted ZFC, it is possible to code
``sequences\index{sequence!transfinite}'' of 
classes indexed by set ordinals\index{ordinal numbers} as single classes.  The
trick is to 
code a sequence $A$ of classes indexed by ordinals $\beta$ less than
some $\alpha \leq \kappa$ as $\{\left<\left<\beta,x\right>\right>\st
x \in A_{\beta}\}$, where the brackets signify the coding of
pairs\index{ordered pair!of ordinals, coded as an ordinal} of 
ordinals into the ordinals introduced earlier.  This trick can be
carried out in our set theory to implement sequences of natural sets
indexed by Cantorian\index{Cantorian!ordinal} ordinals as natural sets.  It is
useful to recall 
that our coding of pairs of ordinals into the ordinals ``commutes with
$T$'', so will behave nicely in the context of reasoning about natural
sets.

There is a further strengthening of ZFC\index{Zermelo--Fraenkel set theory}
possible.  Consider 
the proper class $\Upsilon$ of natural sets which contain $\Omega$ as
an element.  We cannot introduce $\Upsilon$ as an object of our
interpreted ZFC with classes, but we can introduce membership\index{membership}
in $\Upsilon$ as a predicate of classes of ordinals\index{ordinal numbers}.
The ``superclass'' $\Upsilon$ over our interpreted ZFC will have
the following properties:

\begin{enumerate}
 \item Any superclass of an element of $\Upsilon$ is an element of
   $\Upsilon$ (a natural set which includes a natural subset containing
   $\Omega$ contains $\Omega$).

 \item Any element of $\Upsilon$ is of size $\kappa$ (a natural set
   containing $\Omega$, or any non-Cantorian element, has a proper class
   of Cantorian ordinal\index{Cantorian!ordinal} elements which can be placed
   in an external one-to-one correspondence with the class of all Cantorian
   ordinals\index{ordinal numbers}). 

 \item Any intersection\index{intersection!of classes} of fewer than $\kappa$
   elements of $\Upsilon$ 
   belongs to $\Upsilon$ (collections of fewer than $\kappa$ elements of
   $\Upsilon$ are coded as described above; the intersection of natural
   sets implemented as ``rows'' in a Cartesian\index{Cartesian product} product
   with ``rows'' 
   indexed by a Cantorian initial segment\index{segment} of the ordinals will
   be natural and contain $\Omega$ if each of the ``rows'' contains $\Omega$).

 \item Each class of ordinals\index{ordinal numbers} or its
   complement\index{complement} belongs to $\Upsilon$. 
   (Each natural set either contains $\Omega$ or has a natural
   complement\index{complement} 
   (relative to the set of all ordinals) which contains $\Omega$).

 \item  Membership\index{membership} of classes in $\Upsilon$ may be used
   freely in definitions of sets and classes. 
\end{enumerate}

Such a ``superclass'' $\Upsilon$ is called a {\itshape $\kappa$-complete
nonprincipal ultrafilter\index{ultrafilter}\/} over $\kappa$.  The existence of
a $\kappa$-complete nonprincipal ultrafilter over an
uncountable\index{uncountable} set ordinal $\kappa$ is a very strong condition
in ZFC\index{Zermelo--Fraenkel set theory} (the 
existence of a measurable cardinal\index{cardinal numbers}); the strength of
the situation we 
describe here is not as great, because $\kappa$ is not a set ordinal
and we do not have access to levels of the hierarchy of sets above
$V_{\kappa+1}$, but it is still considerable.  It turns out that the
extension of ZFC\index{Zermelo--Fraenkel set theory} with proper classes and a
nonprincipal 
ultrafilter over the proper class ordinal is sufficient to describe
our set theory as well as to be described by it; there is an exact
equivalence in strength.  Since the existence of a measurable $\kappa$
and the corresponding ultrafilter is a stronger condition, this
implies that our theory can be interpreted in the presence of a
measurable cardinal\index{cardinal numbers}, but this is almost certainly more
strength than is needed.

Some further technical observations for the non-naive reader: the
construction of the constructible universe $L$ can be carried out in
$Z_0$; a subset of our set theory can be interpreted in $L$ as defined
in $Z_0$ in the same way we interpret our set theory in $Z_0$.  We say
``a subset'' because there is no apparent reason why the full Axiom of
Small Ordinals\index{Axiom of Small Ordinals} would be satisfied in the
interpretation via $L$, 
though many of its consequences would be.  Forcing can be carried out
in this theory: we think the best way to do this is to build a
structure analogous to $Z_0$ for well-founded relations in which each
item is paired with an element of a fixed Cantorian\index{Cantorian!partial
order} partial order\index{order (partial)} 
indicating the conditions under which it is to be an element of the
set being constructed; the resulting structure would have a $T$
operation which could be used to interpret our set theory as we
interpreted it in $Z_0$.  Neither constructibility nor forcing are
showcases for the independence of our approach from that of the
usual\index{Zermelo--Fraenkel set theory} 
set theory; this is hardly surprising, as both are closely bound up
with the cumulative hierarchy of sets in ZFC\index{Zermelo--Fraenkel set
theory}.
