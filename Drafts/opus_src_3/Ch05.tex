\chapter[The Theory of Relations]{The Theory of Relations}

Just as we associate sets with (some) properties\index{properties} of objects,
so we can associate sets with relations between two or more objects,
with the assistance of the ordered pair\index{ordered pair} construction.  For
instance, we interpret the relation ``is less than" among numbers as the set of
ordered pairs $(m,n)$ of natural\index{natural number} numbers such that $m <
n$.  The use
of the pair is that it allows us to reinterpret a relation pertaining
to a ``configuration" of two or more objects as a property of a single
``abstract" object associated with that configuration.

\begin{definition}
 A {\upshape relation\index{relation!definition}\index{relation|textbf}\/} is a
 set of ordered pairs\index{ordered pair}.  An {\upshape $n$-ary 
 relation} is a set of ordered $n$-tuples (note that any $n$-ary relation
 is also an $(n-1)$-ary relation, because all $n$-tuples are also 
 $(n-1)$-tuples).
\end{definition}
 
We will use a distinctive notation for relations: if the name
of a relation is $R$, we will write ``$x \rR y$" for ``$x$ stands in
the relation $R$ to $y$".  The exactly equivalent statement ``$(x,y)$
belongs to $R$" will be written ``$(x,y) \in R$".

We will want to use certain infix operators which represent
relations (for example, ``='' and ``$\subseteq$'') as names of sets as
well.  To avoid confusion, we enclose an infix operator symbol in
brackets (as for example ``$[=]$'' or ``$[\subseteq]$'') when it is
being used as the name of a set rather than as an infix.

We provide some specific relations\index{relation} of interest:

\begin{axiom}{Axiom of the Diagonal\index{diagonal, axiom of the}}
 The set $[=] = \{(x,x) \st x \in V\}$ exists (this is
 the {\upshape equality relation}).
\end{axiom}

\begin{axiom}{Axiom of Projections\index{projections!axiom of}}
 The sets $\pi_1$ = $\{((x,y),x) \st x,y \in V\}$ and $\pi_2$ =
 $\{((x,y),y) \st x,y \in V\}$ exist (these are the relations which an ordered
 pair\index{ordered pair} has to its first and second terms, which are
 technically referred to as its {\upshape projections}).  These names may be
 used as relation symbols.
\end{axiom}

The names of the last two axioms reflect geometrical concepts;
the graph of $y = x$ on the plane (the Cartesian product\index{Cartesian
product} of the real\index{real numbers!as a line}
line with itself) is a diagonal, and the coordinates $x$ and $y$ of a
point $(x,y)$ in the plane are its projections\index{projections} onto the
coordinate 
axes.  The Axiom of Projections has a special role which is not
immediately obvious; it allows us to prove that there are infinitely
many objects, as we will see below.

The Boolean\index{Boolean algebra, operations} operations give us operations on
relations "ready-made"\hspace{-2pt}.
Observe that $\vide$ is a
relation\index{relation!empty} (it has no elements 
which are not ordered pairs\index{ordered pair}!), and clearly the ``smallest"
possible (no object stands in this relation to any other).  The ``largest"
possible relation is $V \times V$,\index{relation!universal} the Cartesian
product\index{Cartesian product} of the
universe\index{universe, universal set} and itself, i. e., the set of all
ordered pairs (each object
stands in this relation to each other).  The relation symbol we will
use for $V \times V$ will be $V^2$.  (It is possible that $V^2=V$, and
it would be technically convenient if we postulated this, but we
prefer not to add axioms for reasons of pure technical convenience.)
The natural notion of complement\index{complement!of a relation} for a relation
is relative to $V^2$, 
so if $R$ is a relation\index{relation} symbol, we will use $R^c$ as the
relation 
symbol corresponding to the set $V^2 - R$; we can write this as an
equation: $R^c = V^2 - R$.  Note that the relation\index{relation} symbol for
the 
relation that never holds is $(V^2)^c$.  The definitions of relation
symbols $R \cap S$ and $R \cup S$ do not require comment.  We
summarize:

\begin{definition}
 $V^2 = V \times V; R^c = V^2 - R$.
\end{definition}

We now introduce some additional operations on relations\index{relation}.
Both are natural constructions.  Suppose that $L$ stands for the
relation ``loves"; ``$x \mathrel{L} y$" stands for ``$x$ loves $y$".  In
English, we can 
also say ``$y$ is loved by $x$".  The construction in the theory of
relations which corresponds to the passive voice in English is called
the ``converse":  the converse of $L$ is written $L^{-1}$, and we write
``$y \mathrel{L^{-1}} x$" to mean ``$y$ is loved by $x$".  The converse
operation can be applied both to relation\index{relation} symbols and to set
names: for instance, $(A \times B)^{-1} = (B \times A)$.
We state the following

\begin{axiom}{Axiom of Converses\index{converse (of a
relation)!axiom}\index{converse (of a relation)|textbf}}
 For each relation $R$, the set $R^{-1} = \{(x,y) \st y \rR x\}$
 exists; observe that $x \mathrel{R^{-1}} y$ exactly if $y \rR x$.
\end{axiom}

Consider the relation\index{relation} ``is the paternal grandmother
of\hspace{1pt}''.  The
fact that $x$ is the paternal grandmother of $y$ is mediated by a third
party; $x$ is the paternal grandmother of $y$ exactly if there is a $z$ such
that $x$ is the mother of $z$ and $z$ is the father of $y$.  This is the model
for the second primitive operation on relations that we introduce.  If
we write ``$x \mathrel{M} y$" for ``$x$ is the mother of $y$" and ``$x
\mathrel{F} y$" for ``$x$ is the
father of $y$", the operation we will introduce will enable us to write
``$x \mathrel{M|F} y$" for ``$x$ is the paternal grandmother of $y$".  The
operation defined here is called ``relative product", and relative products are
provided by an \index{product, relative|see {relative product}}

\begin{axiom}{Axiom of Relative Products\index{relative
product!axiom}\index{relative product|textbf}}
 If $R,S$ are relations\index{relation}, the set
 $$
  (R|S) =
   \{(x,y) \st \mbox{for some $z$, $x \rR z$ and $z \rS y$}\},
 $$
 called the
 {\upshape relative product} of $R$ and $S$, exists.
\end{axiom}

A useful notion extending this is the notion of a ``power'' of a relation:

\begin{definition}
  We define $R^0$ as $[=]$ and $R^1$ as $R$ for each relation\index{relation}
  $R$; for each positive natural\index{natural number} number $n$, we define
  $R^n$ as $R|(R^{n-1})$.
\end{definition}

There is no problem with grouping in this
definition, since the relative product is associative.  There is the
potential difficulty of ascertaining whether a Cartesian\index{Cartesian
power} or relation power\index{relation!power of a} is intended, which one
hopes will always be resolvable from the context. 

Next, we consider how to get information about relations back
down to the level of sets.  If it is true that ``$x \mathrel{L} y$" ($x$ loves
$y$), 
we would like to say that $x$ is a lover; we would like to be able to
define the set of objects which have the relation\index{relation} $L$ to
something, 
without reference to what that something might be.  The set of things
which stand in the relation $L$ to something is called the {\itshape
domain} of $L$ and represented by dom$(L)$; this is summarized in
the

\begin{axiom}{Axiom of Domains\index{domain!axiom}\index{domain|textbf}}
 If $R$ is a relation, the set
 $$
  \dom(R) = \{x \st \mbox{for some $y$, $x \rR y$}\},
 $$
 called the {\upshape domain} of $R$, exists.
\end{axiom}

If $\dom(L)$ stands for the set of lovers, we may complain that
we have not provided for the set of beloveds, but this is simply the
domain of $L^{-1}$.  We call the domain of the converse\index{converse (of a
relation)} of a relation\index{relation} the {\itshape range} of the relation:

\begin{thm}
 If $R$ is a relation, the set $\rng(R)$ = 
 $\{x \st$ for some $y, y \rR x\}$, called the {\upshape range} of $x$,
 exists.
\end{thm}

\preuve  rng$(R)$ = dom$(R^{-1})$.
\finpreuve

We introduce a related notion:

\begin{definition}
 If $R$ is a relation\index{relation}, we define the {\upshape field}
 or {\upshape full domain\index{domain!full, defined}} of $R$, written
 $\Dom(R)$, as $\dom(R)\,\cup\, \rng(R)$.
\end{definition}

The final construction we provide here is the construction of
``singleton images" of relations.  We would certainly like to believe
that for every $R$ there is a relation $R^{\iota}$ such that $\{x\}
\mathrel{R^{\iota}} \{y\}$ exactly if $x \rR y$; it turns out that
such relations\index{relation} are important.

\begin{axiom}{Axiom of Singleton Images\index{singleton
image!axiom}\index{singleton image|textbf}}
 For any relation $R$, the set 
 $$
  R^{\iota} =
  \{(\{x\},\{y\}) \st x \rR y\},
 $$
 called the {\upshape singleton image\/} of $R$, exists.
\end{axiom}

\begin{definition}
We similarly define $R^{\iota^n}$ as   $\{(\iota^n(x),\iota^n(y)) \st x \rR y\}$.

\end{definition}

The existence of ``singleton images\index{singleton image!of a set}" of sets is
an easy theorem:

\begin{thm}
 For each set $A$, the set ${\cal P}_1\{A\}$ = $\{\{x\} \st x \in A\}$ exists.
\end{thm}

\preuve  ${\cal P}_1\{A\}$ = dom($(\{A \times A\})^{\iota}$).
\finpreuve

The reader might want to prove this 

\begin{thm}
 $\bigcup[{\cal P}_1\{A\}] = A$
\end{thm}


With these axioms for sets and relations\index{relation}, we have arrived at a
certain ``critical mass" of primitive concepts and propositions.  We
will spend the next chapters elaborating the consequences of these
axioms; it turns out that they have a great deal of logical power.  We
will not introduce more axioms until we have ``digested" some of the
consequences of the axioms we already have.

We close with a list of axioms for the ``algebra of
relations\index{relation!algebra of relations}"
taken from a book by Tarski and Givant; the ``algebra of relations" is
generally a less familiar subject than Boolean\index{Boolean algebra,
operations} algebra.  Understanding
and verifying these axioms could be a useful exercise.

\begin{displaymath}
\begin{array}{l}
  A \cup B = B \cup A;\\[1mm]
  (A \cup B) \cup C = A \cup (B \cup C);\\[1mm]
  (A^c \cup B)^c \cup (A^c \cup B^c)^c = A;\\[1mm]
  A|(B|C) = (A|B)|C;\\[1mm]
  (A \cup B)|C = (A|C) \cup (B|C);\\[1mm]
  A|[=] = A;\\[1mm]
  (A^{-1})^{-1} = A;\\[1mm]
  (A \cup B)^{-1} = A^{-1} \cup B^{-1};\\[1mm]
  (A|B)^{-1} = (B^{-1}|A^{-1});\\[1mm]
  (A^{-1}|((A|B)^c)) \cup B^c = B^c.
\end{array}
\end{displaymath}


\Exercises

\begin{enumerate}

\item  Verify some or all of the axioms for relation\index{relation!algebra of
  relations} algebra given at the end of the chapter.  In particular, prove
  that the relative product\index{relative product} is associative. 

\item (open ended)  What difficulties would you encounter in trying to develop
  a system for reasoning about relations using diagrams (analogous to the
  method of Venn diagrams\index{Venn diagrams} which you were asked to read
  about in an earlier exercise)?  Do you think that such a system could be
  developed?

\end{enumerate}
                                   
