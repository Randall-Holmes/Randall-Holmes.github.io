\begin{thebibliography}{99}

\bibitem{conway}  J. H. Conway, {\em On Numbers and Games\/}, Academic
Press, London, 1976.

\bibitem{curry}  H. B. Curry and R. Feys, {\em Combinatory Logic}, Vol.
I, North Holland, Amsterdam, 1958.

\bibitem{forster}T. E. Forster, {\em Set Theory with a Universal
Set\/}, second edition, Oxford logic guides, no. 31, Cambridge
University Press, 1995.

\bibitem{grishin}  V. N. Grishin, ``Consistency of a fragment of
Quine's {\em NF} system,'' {\em Sov. Math. Dokl.}, 10 (1969), pp. 1387--90.

\bibitem{hailperin} T. Hailperin, ``A set of axioms for logic".  {\em
Journal of Symbolic Logic\/}, 9 (1944), pp. 1--19.

\bibitem{halmos} Paul Halmos, {\em Naive Set Theory\/}, van Nostrand,
Princeton, 1960.

\bibitem{henson} C. W. Henson,
``Type-raising operations in {\em NF\/}'',
{\em Journal of Symbolic Logic\/} 38 (1973), pp. 59-68.

\bibitem{hinnion}  Roland Hinnion,
``Sur la th\'eorie des ensembles de Quine'',
Ph.D. thesis, ULB Brussels, 1975.

\bibitem{hinnion2}  Roland Hinnion, ``Extensional quotients of structures and applications to the study of the axiom of extensionality'', {\em Bulletin de la Societ\'e Math\'ematique de Belgique s\'erie B\/}, vol. 31 (1981), pp. 3-11.

\bibitem{thesis} M. Randall Holmes, ``Systems of Combinatory Logic
Related to Quine's `New Foundations'\,,'', Ph. D. thesis, SUNY
Binghamton (now Binghamton University), 1990.

\bibitem{thesispaper}  M. Randall Holmes, ``Systems of Combinatory Logic Related to
Quine's `New Foundations'\,,'' {\em Annals of Pure and Applied Logic},
53 (1991), pp. 103--33.

\bibitem{afa} M. Randall Holmes, ``The Axiom of Anti-Foundation in
Jensen's `New Foundations with Ur-Elements'\,,'' {\em Bulletin de la
Soci\'et\'e Math\'ematique de Belgique, s\'erie~B}, 43 (1991), no. 2,
pp. 167--79.

\bibitem{modernlogic} M. Randall Holmes, ``The set-theoretical program of
Quine succeeded, but nobody noticed''. {\em Modern Logic\/}, vol. 4,
no. 1 (1994), pp. 1-47.

\bibitem{stratification} M. Randall Holmes, ``Untyped $\lambda
$-calculus with relative typing'', in {\em Typed Lambda-Calculi and
Applications\/} (proceedings of TLCA '95), Springer, 1995, pp. 235-48.

\bibitem{rewrite} M. Randall Holmes, ``Disguising recursively chained
rewrite rules as equational theorems'', in Hsiang, ed., {\em Rewriting
Techniques and Applications\/}, the proceedings of RTA '95, Springer,
1995.

\bibitem{hrbacek} K. Hrbacek, and T. Jech, {\em Introduction to Set
Theory\/}, second edition, Marcel Dekker, New York, 1984.

\bibitem{jensen} Ronald Bjorn Jensen, ``On the consistency of a slight
(?) modification of Quine's `New Foundations'\,,'' {\em Synthese}, 19
(1969), pp. 250--63.

\bibitem{kunen}  K. Kunen, {\em Set Theory: an introduction to independence proofs\/}, North-Holland, Amsterdam, 1980.

\bibitem{lewis} David Lewis, {\em Parts of Classes\/}, Basil Blackwell Ltd., Oxford, 1991.

\bibitem{orey} S. Orey, ``New Foundations and the axiom of counting,''
{\em Duke Mathematical Journal\/}, 31 (1964), pp. 655--60.

\bibitem{quine} W. V. O. Quine, ``New Foundations for Mathematical
 Logic,'' {\em American Mathematical Monthly}, 44 (1937), pp. 70--80.

\bibitem{ml} W. V. O. Quine, {\em Mathematical Logic\/}, second
edition, Harvard, 1951.

\bibitem{nonstandard}  Abraham Robinson, {\em Nonstandard Analysis\/}, North-Holland, Amsterdam, 1966.

\bibitem{rosser} J. Barkley Rosser, {\em Logic for Mathematicians},
second edition, Chelsea, New York, 1973.

\bibitem{russell}  Alfred North Whitehead and Bertrand Russell, {\em
Principia Mathematica (to *56)}, Cambridge University Press, 1967.

\bibitem{specker} E. P. Specker, ``The axiom of choice in Quine's `New
Foundations for Mathematical Logic'\,,'' {\em Proceedings of the
National Academy of Sciences of the U. S. A.}, 39 (1953), pp. 972--5.

\bibitem{tarski-givant} Alfred Tarski and Steven Givant, {\em A Formalization
of Set Theory Without Variables}, American Mathematical Society,
Providence, 1988.

\end{thebibliography}
