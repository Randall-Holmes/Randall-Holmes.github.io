\chapter[Introducing Functions]{Introducing Functions}


\begin{definition}
 A {\upshape function\index{function|textbf}} is a
 relation\index{relation!kind of} $F$ with the property that for each $x$ in
 $\dom(F)$, there is exactly one $y$ such that $x \mathrel{F} y$.  We introduce
 $F(x)$ as  a name for this unique $y$; we 
 define $F(x)$ as (the $y$ such that ($F$ is a function and $x \mathrel{F}
 y$)). 
 Notice that this definition causes values of non-functions and values
 of functions outside their domains\index{domain} to become $\vide$.
\end{definition}

Observe that the type\index{types (relative)} of $F$ is one higher than the
type of $x$ when $F(x)$ appears in a stratified\index{stratification} sentence, while the type of the whole expression $F(x)$ is the same as the type of $x$.
The word ``map" is used as a synonym for ``function"; $F$ is said to ``map" $x$
to $F(x)$.

We will generally write $F(x) = y$ instead of $(x,y) \in F$.
$y$ is called the value of $F$ at $x$; we also say that $F$ sends $x$
to $y$; $x$ is called the {\itshape argument\/} of $F$ in the expression
$F(x)$ The symbol $f:X \rightarrow Y$ may be used to mean ``$f$ is a
function from $X$ to $Y$", or, more explicitly, ``$f$ is a function,
dom($F$) = $X$ and rng$(F) \subseteq Y$".

\begin{definition}
 The set $[X \rightarrow Y]$ = $\{f \st f$ is a function
 from $X$ to $Y\}$ exists by the Stratified Comprehension
 Theorem\index{Stratified Comprehension Theorem}. Another notation frequently
 used for this set is $Y^X$.
 Observe that the type\index{types (relative)} of $[X \rightarrow Y]$ is one
 higher than the type of $X$ or $Y$ in a stratified sentence.
\end{definition}

Since functions are relations, we have already defined the
notions of {\itshape domain\index{domain}} and {\itshape range\/} for
functions.  We define some related notions.

\begin{definitions}
 A function in $[X \rightarrow  Y]$
 with the property\index{properties!of functions} that $\rng(f) = Y$ is said
 to be {\upshape onto\index{surjection|see {onto map}}} $Y$ and is said 
 to be a {\upshape surjection\index{onto map|textbf}} (relative to $Y$).  

 A function $f$ with the property
 that $f^{-1}$ is also a function is said to be {\upshape
 one-to-one\index{injection|see {one-to-one map}}} or an
 {\upshape injection\index{one-to-one map|textbf}} (not only does $x
 \mathrel{F} y$ uniquely determine $y$, given $x$, but it
 also uniquely determines $x$, given $y$; $F$ associates each $x$ in $\dom(F)$
 with one and only one $y$).  

 A map in $[X \rightarrow  Y]$ which is one-to-one and
 onto\index{one-to-one and onto map|see {bijection}} $Y$ is called a {\upshape
 bijection\index{bijection|textbf}} from $X$ onto $Y$ (or {\upshape between}
 $X$ and $Y$). 
 (Any injection from $X$ to $Y$ might be referred to as a bijection from $X$
 {\upshape into\/} Y).
\end{definitions}

We pause to breathe before introducing more
\begin{definitions}        
 Suppose that $\dom(F) = X$ and $A \subseteq X$.
 We define the {\upshape restriction} of $F$ to $A$, $F\lceil A = F \cap
 (A \times V)$. $F$ itself is said to be an {\upshape extension\/} of
 $F\lceil A$.

 The {\upshape image\index{image!defined}} of $A$ under $F$, $F[A]$, is defined
 as $\rng(F\lceil A$); this is the set of all values of $F$ at elements of $A$. 
 The {\upshape inverse image\index{inverse image!defined}} $f^{-1}[A]$ is
 defined as $\dom(F \cap (V \times A)$); although this may seem a less natural
 concept than the image, it turns out to have much nicer properties (notice
 that $f^{-1}$ is not always a function).

 The notions of image and inverse image make sense for general
 relations\index{relation}, not just for functions\index{function}.

 A related notion is the {\upshape preimage\index{preimage!defined}} of an
 element $x$ of the range of a relation $R$ (not necessarily a function) under
 that relation: it is defined as the inverse image of $\{x\}$ under $R$.
\end{definitions}

If $X \subseteq Y$, there is an obvious function\index{function} in $[X
\rightarrow Y]$ which takes each element of $X$ to itself; this is
called the {\itshape inclusion\index{inclusion!map} map\/} of $X$ (and actually
does not depend on $Y$ at all; it is simply $[=] \cap (X \times X)$).  The
inclusion map of $X$ into $X$ (the same object) is called the {\itshape
identity map\index{identity map!defined}\/} on $X$.  Note that $[=]$ is the
identity map on the universe\index{universe, universal set}.
The identity map on $Y$, restricted to $X \subseteq Y$, yields the
inclusion map from $X$ into $Y$.

Observe that $[=]$ is not alone among our primitive relations\index{relation}
in being a function\index{function}; the projection relations $\pi_1$ and
$\pi_2$ are also functions, with $\pi_1(x,y) = x$ and $\pi_2(x,y) = y$ for all
$x,y$.  Note that we write $f(x,y)$ instead of $f((x,y))$.  We define
{\itshape projections\index{projections!of sets}} of subsets\index{subset} of
Cartesian products\index{Cartesian product} as their images
under the projection functions; of course, there is a conflict here,
since we also call the values under the projection functions
``projections\index{projections}".  It should always be possible to resolve
this issue in context.  The words ``domain\index{domain}" and ``range" are also
available for these images, of course.

Any function\index{function} $F$ from $X$ to $Y$ can be used to define an
equivalence\index{equivalence relations, equivalence classes} relation (and a
partition\index{partition}) on $X$: the equivalence 
relation is $R  = \{(x,y) \st x,y \in X$ and $F(x) = F(y)\}$, and
the corresponding partition is the collection of sets
$F^{-1}[\{y\}]$, the inverse images\index{inverse image} under $F$ of singletons\index{singleton} of
elements $y$ of $Y$.  The map which sends $F^{-1}[\{y\}]$ to
$\{y\}$ for each $y$ is a one-to-one\index{one-to-one map} map from $X/R$ to
${\cal P}_1\{Y\}$; we can use the Axiom of Choice\index{axiom of choice} to
define a one-to-one map 
from representatives of the equivalence classes\index{equivalence relations,
equivalence classes} in $X/R$\linebreak onto $Y$.

We can associate a function\index{function} with each set in a natural way.
We introduce the definitions of the numbers 0 and 1: 0 is defined as
$\{\vide\}$ (the set of all sets with zero elements) and 1 is defined as ${\cal
P}_1\{V\}$, 
the set of all sets with one element.  The {\itshape characteristic
function\index{function!characteristic, of a set|textbf}\/}
$\Char(A)$ of a set $A$ is the function $\{(x,y) \st$ if $x \in A$ then $y = 1$
and if not$(x \in A)$ then $y = 0\}$; $\Char(A)(x) = 1$ if $x$ is an element of
$A$ and 0 otherwise.  The operation $\Char$ is a function\index{function},
since the type\index{types (relative)} of $A$ is the same as the type of
$\Char(A)$ in a stratified\index{stratification} sentence.

The collection $[X \rightarrow  \{0,1\}]$ of restrictions of characteristic
functions to $X$, which is more usually called the collection of
characteristic functions on $X$, encodes all the subsets\index{subset} of $X$.
We introduce the set of all subsets\linebreak of $X$:

\begin{definition}
 ${\cal P}\{X\}$ is defined as $\dom([\subseteq] \cap (V \times \{X\}))$, the
 collection of all subsets of $X$.  We call ${\cal P}\{X\}$ the {\upshape power
 set\index{power set|textbf}\/} of X.  Notice that the type\index{types
 (relative)} of ${\cal P}\{X\}$ in a stratified\index{stratification} sentence
 would be one higher than the type of $X$.
\end{definition}

\begin{thm}
 There is a natural bijection\index{bijection} between ${\cal P}\{X\}$ and $[X
 \rightarrow  \{0,1\}]$.
\end{thm}

\preuve\ The map which takes each subset $A$ of $X$ to Char$(A)\lceil X$ can be
shown to exist using Stratified Comprehension\index{Stratified Comprehension
Theorem}.
\finpreuve

We introduce ``function-builder\index{function-builder notation|textbf}''
notation analogous to the ``set builder'' notation $\{x \st \phi\}$:


\begin{definition}
 If $T$ is an expression and $x$ is a variable, we
 define $(x \mapsto T)$ as (the function $F$ such that for all $x$,
 $F(x)=T$).  There must be no occurrences of $F$ in $T$.
\end{definition}

An example is the typical function\index{function} $f(x) = 2x+1$ from algebra,
which would be written $(x \mapsto 2x+1)$ in this notation.  $(x \mapsto T)$
will be defined if the expressions $x$ and $T$ would be assigned the
same type\index{types (relative)} in a stratified\index{stratification}
sentence (which includes the implicit 
condition that $T$ {\itshape can\/} appear in such a sentence; some
expressions, like $x \cap \{x\}$ or $x(x)$, cannot appear in such a
sentence at all, so certainly cannot appear with the same type as
$x$).  Another notation for $(x \mapsto T)$ is the notation $(\lambda
x.T)$ of the ``lambda-calculus".  The construction here is called
``function\index{function!abstraction}\index{abstraction!see {function}}
abstraction" from the name or ``term'' $T$, and is formally 
analogous to set comprehension\index{comprehension} $\{x \st \phi\}$ from
sentences $\phi$.

The notation $(x \mapsto T)$ will also sometimes be used informally
for ``functions'' which actually do not exist as sets in our theory.
We will for example refer from time to time to the ``map'' $(x \mapsto
\{x\})$, which, as we will see below, is not a set.  You should be
able to see now that its definition is not stratified\index{stratification}.
We also occasionally refer to restrictions of proper class
functions\index{function!proper class} and  images under proper class functions
(which may be proper classes) in the sequel.  All instances of such notations
should be possible to eliminate in principle, as we do not formally admit that
there are such objects as proper classes or proper class functions.

\Exercises

\begin{enumerate}
\item  Verify our assertions about the stratification\index{stratification} and
  relative type\index{types (relative)} of\linebreak $[X \rightarrow Y]$.

\item  Develop a definition of the number 2 by analogy with the definitions of
  0 and 1 given in the chapter.  Give the definition in set-builder notation,
  then give it in terms of the primitive operations for constructing sets.

\item  What is the type of $(x \mapsto T)$ relative to the term $T$?  To take a
  specific example, is the definition of the function\index{function} $(a
  \mapsto (x \mapsto ax))$ which takes a number $a$ to the function ``multiply
  by $a$'' a stratified\index{stratification} definition (on reasonable
  assumptions about multiplication)?

\item  Verify our assertion that $(x \mapsto \{x\})$ cannot be shown to exist
  using Stratified Comprehension\index{Stratified Comprehension Theorem}, by
  expanding its definition into set-builder notation and verifying that it is
  not stratified.
\end{enumerate}
