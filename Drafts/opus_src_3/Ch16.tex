\chapter{Cardinal Numbers}\index{cardinal numbers}

We defined the relationship of ``equivalence\index{equivalence
(of cardinality)}" above as holding between sets $A$ and $B$ whenever
there is a bijection\index{bijection} $f$ between $A$ and $B$.  We proved that
``equivalence" is an equivalence relation\index{equivalence relations,
equivalence classes}.  We showed that two finite\index{finite!set} sets belong
to the same natural\index{natural number} number $n$ (i.e., are ``the same
size") exactly if they are equivalent\index{equivalence (of cardinality)}.
What we will do in this chapter is generalize the application of ``equivalence"
as meaning ``the same size" to all sets, and use the notion of
``equivalence" to define the notion of cardinal\index{cardinal
numbers} number for general sets.  The intuitive idea that two sets
are the same size if we can set up a one-to-one correspondence\index{one-to-one
correspondence|see {bijection}} between them, taken from our experience with
finite sets, extends to infinite\index{infinite!set} 
sets fairly well.




\section{Size of Sets and Cardinals}\index{cardinal numbers}

One counterintuitive consequence of adopting this notion of
``size" of sets is that a proper part of a set need not be strictly
smaller than the set:  for instance, to use a classical example due to
Galileo Galilei, there is a bijection\index{bijection} between the set of
natural numbers ${\cal N}$ and its proper subset\index{subset!proper}, the set
of perfect squares.  In fact, in the presence of the Axiom of
Choice\index{axiom of choice}, this failure of our intuition 
for an infinite set defines the notion of ``infinite set" exactly:

\begin{Thm}{Theorem (Dedekind)\index{Dedekind criterion for infinity}}
 A set is infinite\index{infinite!set} exactly if it is
 equivalent\index{equivalence (of cardinality)} to one of its proper subsets.
\end{Thm}\label{thm-dedekind}

\preuve\ Each set belonging to 0 is not equivalent to any of its
proper subsets.  That is, $\vide$ is not equivalent to any of its
(nonexistent) proper subsets.  Suppose that no element of $n$ is
equivalent to any of its proper subsets.  An element $A$ of $n+1$ is
of the form $A' \cup \{x\}$, where $A' \in n$ and not($x \in A'$).
Suppose that $A$ were equivalent\index{equivalence (of cardinality)} to one of
its proper subsets. 
Remove $x$ from $A$ and the corresponding element from the proper
subset; we may suppose without loss of generality that $x$ does not
belong to the proper subset.  A one-to-one
correspondence\index{bijection} between $A'$ and a proper subset of $A'$ is
constructed, 
contrary to hypothesis.  Thus no finite\index{finite!set} set is equivalent to
a   proper subset of itself.  Now suppose that $X$ is infinite.  It is
sufficient to prove that $X$ contains a subset equivalent\index{equivalence (of
cardinality)} to ${\cal N}$, since 
we can then use a bijection\index{bijection} between ${\cal N}$ and one of its
proper
subsets to construct a bijection\index{bijection} between $X$ and one of its
proper subsets.  Consider the partial order\index{order (partial)} of
inclusion\index{inclusion} on one-to-one maps 
from subsets of ${\cal N}$ into $X$.  This order satisfies the conditions of
Zorn's Lemma\index{Zorn's Lemma} (since $X$ is not finite); a maximal element
in this order is a one-to-one\index{one-to-one map} map from ${\cal N}$ into
$X$.
\finpreuve

We would like to say that a set $X$ is less than or equal to a
set $Y$ in size if there is a one-to-one\index{one-to-one map} map from the set
$X$ into the set $Y$.  This requires a little work; it is easy to see that $X$
is less than or equal to $X$ in size (identity map) and that if $X$ is less
than or equal to $Y$ in
size and $Y$ is less than or equal to $Z$ in size, $X$ is less than or
equal to $Z$ in size (use a
composition\index{composition}).  We also expect, however, that if $X$ is less
than or equal to $Y$ in
size and $Y$ is less than or equal to $X$ in size, that $X$ and $Y$ will be the
same size, that is, equivalent\index{equivalence (of cardinality)}.  This is
the not entirely trivial

\begin{Thm}{Schr\"oder--Bernstein Theorem\index{Schr\"oder--Bernstein theorem}}
  If $X$ and $Y$ are sets, and $f$ is a one-to-one\index{one-to-one map} map
  from $X$ into $Y$ and $g$ is a one-to-one map from $Y$ into $X$, then $X$ and
  $Y$ are equivalent\index{equivalence (of cardinality)}.
\end{Thm}

\preuve\ We need to construct a bijection\index{bijection} $h$ between $X$ and
$Y$.  It is sufficient to show that $X$ is equivalent\index{equivalence (of
cardinality)} to $g[Y] = g[f[X]] \subseteq X$.  The map $g\circ f$ is a
one-to-one map from $X$ into $g[Y]$.
Let the set Pushset be defined as the smallest set containing $X - g[Y]$ and
closed under $g\circ f$.
Define Push as the union of the restriction of $(g\circ f)$ to Pushset and the
restriction of\linebreak

\pagebreak

the identity map to $X - \mbox{Pushset}$.
It is easy to see that Push is a bijection\index{bijection} from
$X$ onto $g[Y]$.
$g^{-1} \circ \mbox{Push}$ is the desired $h$.
\finpreuve

The Axiom of Choice\index{axiom of choice} enables us to show that this is a
total order\index{order (linear)} on size:

\begin{thm}
 For any sets $X,Y$, either $X$ is less than or equal to $Y$ in size
 or $Y$ is less than or equal to $X$ in size.
\end{thm}

\preuve\ Consider well-orderings\index{well-orderings} $R,S$ on $X,Y$
respectively.  If $R$ and $S$ 
are similar\index{similarity}, $X$ and $Y$ are of the same size; if $S$ is the
continuation of a relation similar to $R$, $X$ is less than or equal to $Y$ in
size; if $R$ is the continuation of a relation similar to $S$, $Y$ is less than
or equal to $X$ in size; there are no other alternatives.
\finpreuve

We also need the Axiom of Choice\index{axiom of choice} to show the following:

\begin{thm}
 If $f$ is a map from $X$ onto\index{onto map} $Y$, then $Y$ is less than or
 equal to $X$ in size.
\end{thm}

\preuve\ Build a choice set $C$ for the collection of inverse\index{inverse
image} images of singletons\index{singleton} $\{y\}$ of elements $y$ of $Y$;
map each $y$ to the unique element $x$ of $C$ such that $f(x) = y$.
\finpreuve

This removes some possible doubts about the following

\begin{definition}
 Let $A$ be a set.  We define the {\upshape cardinal\index{cardinal
 numbers!definition of}\index{cardinal numbers|textbf} number} of $A$, written
 $|A|$, as $\{B \st B$ is equivalent\index{equivalence (of cardinality)} to 
 $A\}$, which exists by the Stratified Comprehension\index{Stratified
 Comprehension Theorem} Theorem.  Note that the type\index{types (relative)} of
 $|A|$ is one higher than the type of $A$.
\end{definition}

\begin{ThmEtc}{Observation.}
 The natural\index{natural number} numbers are cardinal numbers.  We call them
 ``finite\index{finite!cardinal} cardinal numbers''.
\end{ThmEtc}

\begin{definition}
 We define $|A| \leq |B|$ iff $A$ is equivalent\index{equivalence (of
 cardinality)} to a subset\index{subset} of $B$.
\end{definition}

We have already shown that $\leq$ is a linear order on cardinal
numbers, but we can do better.

\begin{thm}
 The order $\leq$ on cardinal numbers is a well-ordering\index{well-orderings}.
\end{thm}

\preuve\ Take any set $A$ of cardinals, and choose a cardinal
$\kappa$ from the set.  Choose a set $W \in \kappa$ and choose a
well-ordering $R$ of $W$.  We look at the cardinal numbers of
initial segments\index{segment} of $R$; consider the set of all $a$ in $W$ such
that $|\seg_R\{a\}| \geq \lambda$ for some $\lambda$ in $A$.  If
this set is empty, $\kappa$ is the smallest cardinal in $A$; if it
is nonempty, it has a least element $m$, and $|\seg_R\{m\}|$ is the
smallest cardinal in $A$.
\finpreuve


\section{Simple Arithmetic of Infinite Cardinals}\index{infinite!cardinal}

We define the arithmetic operations of addition\index{addition!of cardinals}
and multiplication\index{multiplication!of cardinals} just as we defined them
for the natural\index{natural number} numbers.

\begin{definition}
 If $A$ and $B$ are sets, $|A| + |B| = |A \oplus B|$ and $|A|.|B| =
 |A \times B|$.
\end{definition}

It is easy to show that the results of these operations do not
depend on the choice of $A$ and $B$.  We will now introduce our first
infinite cardinal\index{cardinal numbers} number.

\begin{definition}
 $\aleph_0 = |{\cal N}|$.  A set of cardinality $\aleph_0$ is
 said to be ``countable\index{countable|textbf}".
\end{definition}

We prove some theorems about $\aleph_0$ (read ``aleph-null'' ---\,$\aleph$ is
the first letter of the Hebrew alphabet) and general 
cardinal\index{cardinal numbers} numbers which will enable us to show that a
great many sets are countable; in fact, we will begin to wonder if any infinite
sets are {\itshape not\/} countable.

\begin{thm}
 If $\kappa$ is an infinite\index{infinite!cardinal} cardinal number and $n$ is a finite\index{finite!cardinal}
 cardinal number (a natural\index{natural number} number), then $\kappa + n =
 \kappa$. 
\end{thm}

\preuve\ Let $\kappa = |A|$.  $A$ has a countable subset\index{subset} (a 
subset equivalent\index{equivalence (of cardinality)} to ${\cal N}$) by the
theorem of Dedekind given 
above (p.~\pageref{thm-dedekind}).  Let $f$ be a bijection\index{bijection}
from ${\cal N}$ into $A$.
$\kappa +   n$ is the cardinality\index{cardinal numbers} of $A \times \{0\}
\cup \{1,\ldots,n\} \times \{1\}$, applying Rosser's Counting\index{Axiom of
Counting} Theorem to get a set with $n$ elements (the use of the Counting
Theorem is easily avoided).  Map each element $(a,0)$ of $A \times \{0\}$ to
$f(f^{-1}(a) + n)$ if $a$ is in the range of $f$ and to $a$ itself otherwise,
then map $(i,1)$ to $f(i-1)$ for $1 \leq i \leq n$, obtaining a
bijection\index{bijection} onto  $A$.
\finpreuve

\begin{thm}
 $\aleph_0 + \aleph_0 = \aleph_0$.
\end{thm}

\preuve\ Map ${\cal N} \times \{0\}$ to the even numbers and ${\cal N} \times
\{1\}$ to the odd numbers in the natural way.
\finpreuve

\begin{thm}
 $\kappa + \kappa = \kappa$, if $\kappa$ is an
 infinite\index{infinite!cardinal} cardinal\index{cardinal numbers}.
\end{thm}

\preuve\ Let $\kappa = |A|$.  Apply Zorn's Lemma\index{Zorn's Lemma} to the
relation of inclusion\index{inclusion} 
restricted to the set of bijections\index{bijection} between $X \oplus X$ and
$X$ for $X \subseteq A$. 
There are such maps because $A$ has countable subsets.  These are easily
seen to satisfy the conditions of the Lemma.  A maximal element will
be a bijection from $Y \oplus Y$ to $Y$ for some $Y \subseteq A$.  We prove
that the set $A - Y$ is finite\index{finite!set}.  If it were not, then $A - Y$
would have a countable subset $M$; it would then be easy to define a bijection
from $(Y \oplus M) \oplus (Y \oplus M)$ to $(Y \oplus M)$ extending the
``maximal" bijection by exploiting the bijection from $(M \oplus M)$ to $M$.
Now we have shown that $|Y| + |Y| = |Y|$, but, since $A - Y$ is finite, $|Y| =
|A|$ by an earlier theorem.
\finpreuve

\begin{thm}
 $\kappa + \lambda = \lambda + \kappa = \kappa$, if $\lambda$ is an
 infinite\index{infinite!cardinal} cardinal\index{cardinal numbers} and $\kappa
 \geq \lambda$.
\end{thm}

\preuve\ Commutativity of addition\index{addition!commutative property of} is
obvious.  If $\kappa = |A|$ and $\lambda = |B|$,
choose a subset $C$ of $A$ equivalent\index{equivalence (of cardinality)} to
$B$, then define a bijection\index{bijection} 
between $C \oplus B$ and $B$ and use it in the obvious way to build a
bijection from $A \oplus B$ to $A$.
\finpreuve

\begin{thm}
 $\aleph_0\aleph_0 = \aleph_0$.
\end{thm}

\preuve\ First list the ordered pairs\index{ordered pair} of
natural\index{natural number} numbers which add up to
0, then the ordered pairs which add up to 1 (increasing the first
projection and decreasing the second), then the ordered pairs which
add up to 2, and so forth.  This sets up a bijection between ${\cal N}
\times {\cal N}$ and
${\cal N}$.
\finpreuve

\begin{thm}
 $\kappa\kappa = \kappa$, for $\kappa$ an infinite\index{infinite!cardinal}
 cardinal\index{cardinal numbers}.
\end{thm}

\preuve\ Let $\kappa = |A|$.  Apply Zorn's Lemma\index{Zorn's Lemma} to the
inclusion\index{inclusion} order\index{order (partial)} 
restricted to bijections\index{bijection} between $X \times X$ and $X$, where
$X \subseteq A$.  Since $A$ has a countable\index{countable} subset, there are
such maps.  A maximal element of this order is a bijection from $Y \times Y$ to
$Y$ for some $Y \subseteq A$.  Now suppose that $|Y| < |A|$.  It would follow
that $|A - Y| = |A|$, since $|A - Y| + |Y| = |A|$ and the sum of two
infinite\index{infinite!cardinal} cardinals\index{cardinal numbers} is their 
maximum.  It follows that $A - Y$ has a subset $Z$ equivalent\index{equivalence
(of cardinality)} to $Y$.  Now consider the set $(Y \oplus Z) \times (Y \oplus
Z)$; this is the union of four Cartesian\index{Cartesian product} products
(take one term of each disjoint\index{disjoint!disjoint sum} sum in each
possible way), each having cardinality $|Y| \times |Y| = |Y| = |Z|$.  The 
union of the three Cartesian products not in the domain\index{domain} of our
``maximal" function\index{function} must have cardinality $|Y| = |Z|$ by our
theorems on addition, and so there is a bijection from this set to $Z$, which
can be used then to extend the ``maximal" bijection to a bijection from $(Y
\oplus Z) \times (Y \oplus Z)$ onto $(Y \oplus Z)$, contradicting maximality.
Thus $|Y| = |A|$, and we are done.
\finpreuve

\begin{thm}
 $\kappa\lambda = \lambda\kappa = \kappa$ if $\lambda$ is an
 infinite\index{infinite!cardinal} cardinal\index{cardinal numbers} and $\kappa
 \geq \lambda$.
\end{thm}

\preuve\ $\kappa \leq \kappa\lambda \leq \kappa\kappa = \kappa$; apply the
Schr\"oder--Bernstein\index{Schr\"oder--Bernstein theorem} Theorem.\linebreak
\mbox{}\finpreuve

\begin{cor}
 For any infinite cardinals $\kappa,\lambda$,
 $\kappa + \lambda = \kappa\lambda = \max\{\kappa,\lambda\}$.
\end{cor}

This is all very well, and remarkably simple, but it does not
give us any way of getting any infinite cardinal\index{cardinal numbers} larger
than $\aleph_0$!  The results of this chapter can be used to show that
the set ${\cal Z}$ of integers is countable\index{countable} ($({\cal N} \oplus
\{0\}) \oplus {\cal N}$ is clearly of the same cardinality as the set of
integers), that the set of rational\index{rational numbers} numbers is
countable (!) (the set ${\cal Q}$ of rational numbers has a subset
equivalent\index{equivalence (of cardinality)} to ${\cal Z}$ and is equivalent
in an obvious way to a subset of ${\cal Z} \times
{\cal Z}$); the same results apply to intuitively ``bigger" sets such as the
set of all points with rational coordinates in three-dimensional
space.  Are all infinities the same?

\Exercises

\begin{enumerate}
\item  Give an algebraic formula for a bijection\index{bijection} $p \in
  [({\cal N} \times {\cal N}) \rightarrow {\cal N}]$ (an explicit witness for
  $\aleph_0\aleph_0 = \aleph_0$).

\item  Construct a bijection between the natural\index{natural number} numbers
  and the integers.

\item  Construct a bijection between the natural numbers and the
  rational\index{rational numbers} numbers (your description should enable me
  to compute the rational number corresponding to each natural number).

\item  Prove that the set of all points with rational coordinates in
  three-dimensional space is countable\index{countable}.
\end{enumerate}
