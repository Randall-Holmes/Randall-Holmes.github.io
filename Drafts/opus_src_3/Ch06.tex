\chapter{Sentences and Sets}

We now pause to analyze the structure of the language we are
using.  We are going to define the notion ``sentence (of first-order
logic)"; we will not introduce any special symbolism, but rather a
careful technical definition of the meaning of certain English
expressions.

\section{The Definition of ``Sentence''}  

The first notion we need to define is ``atomic sentence''.
Informally, an atomic sentence asserts that an object $x$ has a
property\index{properties} $P$ (we can write this $Px$) or that objects $x, y$
stand in a relation\index{relation!as predicate} $R$ to one another (where
``relation'' does not mean ``set 
of ordered pairs'') (we can write this $Rxy$), or that a list of
objects $x_1\ldots x_n$ satisfy an ``$n$-ary relation'' $G$ (we write
this $G[x_1,\ldots,x_n]$).  Informal examples: $Gx$, meaning ``$x$ is
green''; $Lxy$, meaning ``$x$ loves $y$''; $Txyz$, meaning $x$ takes
$y$ from $z$.  An atomic sentence can be thought of as expressing a
simple fact about the world (or, at least, one that we have not
analyzed logically).

We now give a semi-formal

\begin{definition}
 An {\upshape atomic sentence} is a sentence of the
 form $P[x_1,\ldots,x_n]$, where $P$ is a predicate (property\index{properties}
 or relation\index{relation!as predicate}) applicable to a list of $n$ objects.
 The brackets and 
 commas in the sentence can be omitted where this is convenient.
\end{definition}

The first kind of complex structure we consider is the
building of complex sentences using words like ``not'', ``and'',
``or'', ``implies'', and ``if and only if'' (usually abbreviated
``iff''; we often say ``exactly if'' or ``exactly when'').  These
words are called ``propositional connectives''.  We regard negation
and conjunction as primitive, and we define other English
propositional connectives in terms of these.  Their meanings in common
English usage are less precise.

\begin{description}

\item[\fdescr disjunction:] ($\phi$ or $\psi$) is defined as not((not $\phi$)
  and
 (not $\psi$)).  Notice that this is the ``non-exclusive or'', written
 ``and/or'' in legal documents.

\item[\fdescr implication:] ($\phi$ implies $\psi$) or, equivalently, (if
 $\phi$ then $\psi$) is defined as ((not $\phi$) or $\psi$), an odd
 definition, but one which grows on you.

\item[\fdescr if and only if:] ($\phi$ iff $\psi$) is defined as (($\phi$
 implies $\psi$) and ($\psi$ implies $\phi$)).

\end{description}

Notice that we allow the non-English expedient of parentheses to
clarify the structure of sentences built with propositional
connectives.

The remaining construction of sentences is the use of ``quantifiers''
such as ``for all $x$'', ``for some $x$'', ``for exactly one $x$''.
We give examples of sentences using these: ``All men are mortal''
translates to ``For all $x$, if $x$ is a man, then $x$ is mortal''.  A
mathematical sentence using two quantifiers is ``Every number has
prime divisors'', which translates to ``For all $x$, if $x$ is a
number then there is a $y$ such that $y$ divides $x$ and $y$ is
prime''.

We take ``for some $x$'' as primitive.  ``there is an $x$ such
that$\ldots$'' and ``there exists $x$ such that $\ldots$'' and similar
constructions are regarded as synonymous.  We now give the complete
semi-formal definition of {\itshape sentence\/}:

\begin{Definition}{sentence}
 The following clauses define our
 technical meaning of {\upshape sentence\/}.  (It may be useful to be aware
 that logicians normally use the term ``formula'' for what we call a
 ``sentence'' and reserve the term ``sentence'' for formulas containing
 no free variables.)
 \begin{description}
  \item[\fdescr atomic: ]  An atomic sentence is a sentence.
  \item[\fdescr negation: ]  If $\phi$ is a sentence, ``not $\phi$'' is a
    sentence.
  \item[\fdescr conjunction: ] If $\phi$ and $\psi$ are sentences, ``$\phi$ and
    $\psi$'' is a sentence.
  \item[\fdescr existential quantifier: ] If $\phi$ is a sentence and $x$ is a
    variable, ``for some $x$, $\phi$'' is a sentence.
  \item[\fdescr completeness of definition: ] Every sentence is constructed
    using 
    these clauses alone.
 \end{description}
\end{Definition}

\begin{definition}

Notations are provided to avoid negations of atomic sentences.\linebreak
``$x \neq y$" abbreviates ``not($x=y$)".  ``$x \not\in y$" abbreviates ``not($x \in y$)".

\end{definition}

We give the definition of the universal quantifier:

\begin{Definition}{universal quantifier}
 ``for all $x$, $\phi$'' is defined as
 ``not(for some $x$, (not $\phi$))''.  ``for each $x$'' and ``for any
 $x$'' are regarded as synonymous alternatives to this.
\end{Definition}

We define some basic notions useful in discussing quantifiers:

\begin{definition}
 An occurrence of a variable $x$ is said to be {\upshape
 bound} in a sentence when it is part of a sentence ``for some $x$,
 $\phi$''.  (and so when it is part of a sentence ``for all $x$,
 $\phi$'' as well).  An occurrence of a variable which is not bound is
 said to be {\upshape free}.
\end{definition}

\begin{definition}
 When $\phi$ is a sentence and $y$ is a variable, we
 define $\phi[y/x]$ as the result of substituting $y$ for $x$
 throughout $\phi$, but only in case there are no bound occurrences of
 $x$ or $y$ in $\phi$.  (We note for later, when we allow the
 construction of complex names $a$ which might contain bound variables,
 that $\phi[a/x]$ is only defined if no bound variable of $a$ occurs in
 $\phi$ (free or bound) and {\upshape vice versa\/}).
\end{definition}

It is possible to give a more general definition of substitution, but
this one is simple.  The reason that it is adequate is that it is
always possible to rename a bound variable without changing the
meaning of a sentence.  A substitution blocked by the restriction on
the definition of substitution can be carried out by first changing
the names of bound variables as indicated in the following

\begin{fact}
 The meanings of ``for some $x$, $\phi$'' and ``for all
 $x$, $\phi$'' are unaffected by substitutions yielding ``for some $y$,
 $\phi[y/x]$'' and ``for all $y$, $\phi[y/x]$'', when the substitutions
 are meaningful.

 For example, ``All men are mortal'' is equally well translated by
 ``for all $x$, if $x$ is a man then $x$ is mortal'' and ``for all $y$,
 if $y$ is a man then $y$ is mortal''.  This should remind you of
 similar considerations regarding ``set-builder'' notation $\{x \st
 \phi\}$; the $x$ in this notation is also a bound variable, as we will
 see below.
\end{fact}

We can now give another

\begin{Definition}{uniqueness quantifier}
 ``for exactly one $x$,
 $\phi$'' is defined as ``(for some $x$, $\phi$) and (for all $x$, for
 all $y$, ($\phi$ and $\phi[y/x]$) implies $x=y$)''.  It is required
 that $\phi[y/x]$ be defined, which may require renaming of bound
 variables in $\phi$.
\end{Definition} 

We now show how to introduce names into our language.  At the moment,
there are no nouns in our language, only pronouns (variables).  The
basic construction for nouns is (the $x$ such that $\phi$), where
$\phi$ is a sentence.  The meaning of this name will be ``the unique
object $x$ such that $\phi$, if there is indeed exactly one such
object, otherwise the empty set\index{empty set!as default object}.''  A
special sentence construction 
which does not fall under this definition is ``(the $x$ such that
$\phi$) exists'', which is taken to mean ``there is exactly one $x$
such that $\phi$''.

These names cover not only the case of unique objects as in ``0 = (the
natural\index{natural number} number $n$ such that for all natural numbers $m$,
$n+m = m+n = m$)'' but also the case of definitions of {\itshape operations\/}:
the union\index{union!set} $A \cup B$ of two sets can be defined as (the $C$
such that for all $x$, $x \in C$ iff ($x\in A$ or $x \in B$)) and the set
builder notation $\{x \st \phi\}$ can be defined as (the $A$ such that for
all $x$, $x \in A$ iff $\phi$).  Notice that $x$ is a bound variable
in this ``complex name''; there is an additional requirement that
there be no occurrences of $A$ in $\phi$.  Variables free in a complex
name (as in the definition of Boolean\index{Boolean algebra, operations} union)
are called {\itshape parameters\/} of the name.

We give a semi-formal definition of complex names (this is a variation
on Bertrand Russell's Theory of Descriptions):

\begin{definition}
 A sentence $\psi[($the $y$ such that $\phi)/x]$ is
 defined as ``((there is exactly one $y$ such that $\phi$) implies (for
 all $y$, $\phi$ implies $\psi[y/x]$)) and ((not(there is exactly one
 $y$ such that $\phi$)) implies (for all $x$, ($x$ is the empty set\index{empty
 set}) implies $\psi$))''.  Renaming of bound variables may be needed.
\end{definition}

\begin{definition}
 A sentence ``(the $x$ such that $\phi$) exists'' is
 defined as ``there
is exactly one $x$ such that $\phi$''.  ``exists''
 is not to be understood as a predicate.
\end{definition}

A useful logical construction is the restriction of bound variables to
sets.  We give definitions.

\begin{definition}
 ``for all $x \in A$, $\phi$'' means ``for all $x$,
 if $x \in A$, then $\phi$''.  

 ``for some $x \in A$, $\phi$'' means
 ``for some $x$, $x \in A$ and $\phi$''.  

 ``for exactly one $x \in A$, $\phi$'' means ``for exactly one $x$, $x
 \in A$ and $\phi$''.

 ``the $x\in A$ such that $\phi$'' means ``the $x$ such that $x \in A$
 and $\phi$'' (this may serve to explain the idiom ``the natural\index{natural
 number} number $n$ such that$\ldots$'' used above without comment).
\end{definition}



\section{The Representation Theorem}

Our interest in first-order logic is to explain conditions
under which sets\index{set} $\{x \st \phi\}$ exist.  What we will do is provide
set-theoretical constructions which parallel the sentence
constructions above, using the primitive set and relation\index{relation}
constructions we have given in the last few chapters.

The ``predicates" (properties\index{properties} and relations\index{relation!as
predicate}) in atomic sentences correspond (sometimes) to sets and relations
(in the set 
theory sense ``sets of ordered pairs\index{ordered pair} or $n$-tuples").  We
note this in the context of a

\begin{definition}
 A property $G$ is said to be {\upshape realized} by
 the set $\{x \st Gx\}$ if this set exists.  We call the set $G!$ when
 confusion with the predicate needs to be avoided.  

 A relation\index{relation!as predicate} $R[x_1,...,x_n]$ among $n$ objects is
 said to be realized by the set $\{(x_1,\ldots,x_n) \st R[x_1,\ldots,x_n]\}$
 when this set exists.  We will call this set $R!$ when confusion with the
 predicate needs to be avoided.
\end{definition}

The constructions of sentences not($P$) and ($P$ and $Q$) have
obvious parallels in our arsenal of basic constructions:  if $\{x \st \phi\}$
and $\{x \st \psi\}$ exist, \linebreak $\{x \st$ not$(\phi)\}$ = $\{x \st \phi\}^c$ and 
$\{x \st (\phi$ and $\psi)\}$ =
$\{x \st \phi\} \cap \{x \st \psi\}$.  Similarly, $\{x \st \phi$ or $\psi\} =
\{x \st \phi\} \cup \{x \st \psi\}$.

The construction of sentences with quantifiers involves the
use of variables; we need to develop set-theoretical tools analogous
to the use of variables.  What we will do is restrict our attention to
sentences $\phi$ which involve only $n$ variables $x_1,\ldots,x_n$; clearly, we
can analyze any sentence if we take $n$ large enough.  We then represent any
sentence $\phi$ by the collection of tuples $(a_1,\ldots,a_n)$ such that $\phi$
would be true if $x_1 = a_1,\ldots,x_i = a_i,\ldots,x_n = a_n$.  This set can
conveniently be written $\{(x_1,...,x_n) \st \phi\}$.

We go back to atomic sentences, where we have a difficulty:
not all atomic sentences will involve all the variables, or involve
them in the given order.  We need to represent atomic sentences like
$P[x_2]$, $Q[x_2,x_1]$, or $R[x_1,x_1,x_3,x_2]$ as well as sentences
$S[x_1,\ldots,x_n]$ whose representation is obvious.  The trick here is
to provide relations\index{relation} among $m$-tuples (for varying $m$) which
manage the shuffling and reduplication of variables.  Once we deal with this,
the operation analogous to quantification will turn out to be very easy to
represent.

We consider a sentence $P[x_i]$ first.  This will be
represented by
$$
 \{(x_1,\ldots,x_i,\ldots,x_n) \st P[x_i]\}.
$$
We could define this as the domain\index{domain} of an
intersection\index{intersection!Boolean} of relations\index{relation} 
$X_i \cap (V\times P!)$, where $X_i$ is the relation between any
$(x_1,\ldots,x_i,\ldots,x_n)$ and its ``projection" $x_i$ and $V
\times P!$ is the relation between anything and an object of which $P$ is true.
The difficulty is now to construct the ``projection" relation
$X_i$. Recall that $(x_1,\ldots,x_n) =
(x_1,(x_2,(x_3,\ldots(x_{n-1},x_n))))$.  Clearly $X_1 = \pi_1$.
Analysis reveals that $X_2 = \pi_2|\pi_1, X_i = (\pi_2^{i-1})|\pi_1$
for $2 \leq i < n$, while $X_n = \pi_2^{n-1}$.  The definition of
$X_n$ is the only one which depends on the particular value of $n$.
So $P[x_i]$ is represented by $\dom(X_i
\cap (V \times P!))$, which exists by our axioms if $P!$ exists.

We now consider more complex atomic sentences.  The trick is
to find the correct relations\index{relation} to represent {\itshape lists\/}
of variables, as the $X_i$'s represent individual variables.  In order to
represent $R[x_{k_1},\ldots,x_{k_m}]$, where for each $j$, $1 \leq k_j \leq n$,
we will need the relation which holds between any $n$-tuple $(x_1,\ldots,x_n)$
and the $m$-tuple $(x_{k_1},\ldots,x_{k_m})$.  For this, we use the following

\begin{definition}
 Where $R, S$ are relations, $R \otimes S$ is defined as $(R|\pi_1^{-1}) \cap
 (S|\pi_2^{-1})$.  Analysis reveals that $R \otimes S$ = $\{(x,(y,z)) \st x
 \rR y$ and $x \mathrel{S} z\}$.  Parentheses in expressions $(R_1
 \otimes R_2 \otimes\cdots \otimes R_n)$ are supplied from the right; for
 example,
 $(R_1 \otimes R_2 \otimes R_3) = (R_1 \otimes (R_2 \otimes R_3))$; this is
 done to parallel the grouping of ordered pairs\index{ordered pair} in the 
 definition of the $n$-tuple.
\end{definition}

The desired relation\index{relation} is then $(X_{k_1} \otimes \cdots \otimes
X_{k_m})$.  We can then show that the set $\{(x_1,\ldots,x_n) \st
P[x_{k_1},\ldots,x_{k_m}]\}$ is represented by 
$\dom((X_{k_1} \otimes\cdots\otimes X_{k_m}) \cap (V \times P!)$).
Thus any atomic sentence can be represented in our current scheme if the
relation\index{relation!as predicate} (in the sense of first-order logic)
in it is represented by a (set) relation\index{relation} in the natural way.

The apparently simple problem of representing atomic sentences
turned out to be rather involved.  The problem of representing
negation and conjunction is trivial: use complement\index{complement} and
intersection\index{intersection!Boolean}
as indicated above.  The representation of quantification turns out to
be fairly simple.  The idea is that if a sentence $\phi$ is
represented by a set $\phi! = \{(x_1,\ldots,x_n) \st \phi\}$, the set
of $n$-tuples such that ``(for some $x_i$, $\phi$)" is true for the
listed values of $x_j$'s is the set of
$(x_1,\ldots,x_i',\ldots,x_n)$'s obtained by substituting any $x_i'$
at all for the specific value of $x_i$ in an $n$-tuple
$(x_1,\ldots,x_i,\ldots,x_n)$ which belongs to $\phi!$.  The point is
that the statement (for some $x_i$, $\phi$) says nothing about the
value of $x_i$; it says that {\itshape some\/} value of $x_i$ works;
whereas $\phi$ might require some specific value of $x_i$ to be true,
(for some $x_i$, $\phi$) allows $x_i$ to ``float free".  Now this set is
very easy to represent: it is dom$((X_1 \otimes \cdots \otimes X_{i-1}
\otimes V^2 \otimes X_{i+1} \otimes \cdots \otimes X_n) \cap (V \times
\phi!))$, with obvious variations for $i=1,2,n-1,n$.

These results taken together show that $\{(x_1,\ldots,x_n)
\st \phi\}$ is definable for any sentence $\phi$ involving only the
variables $x_1,\ldots,x_n$ in which every relation\index{relation!as predicate}
or property\index{properties}
mentioned in an atomic sentence is realized.  Suppose that we want to
define the set $\{x_1 \st \phi\}$, with fixed values $a_i$ for each
variable $x_i$ with $i > 1$; this can be constructed as the
domain\index{domain} of the intersection\index{intersection!Boolean} of
$\{(x_1,\ldots,x_n) \st \phi\}$ and the sets interpreting the conditions ``$x_i
= a_i$" (the condition ``$x = a$" on a variable $x$ is realized by $\{a\}$).
Thus, we can define $\{x \st \phi\}$ for any variable $x$ and sentence $\phi$
of first-order logic, 
as long as all properties\index{properties} and relations\index{relation!as predicate} mentioned in atomic sentences
are ``realized".  The operations we have defined on $n$-ary relations
are basically the operations of the ``cylindrical algebra" of Tarski,
which is to first-order logic as Boolean\index{Boolean algebra, operations} algebra is to propositional
logic (the logic of ``and", ``not" and their relatives without
quantifiers).

We state the result of our discussion as the

\pagebreak

\begin{Thm}{Representation Theorem\index{Representation Theorem}}
 For any sentence $\phi$ in which all
 predicates mentioned in atomic sentences are realized, the set $\{x
 \st \phi\}$ exists.
\end{Thm}

A technical point which ought to be made (but which should not be
belabored) is that statements like the Representation Theorem which
begin with phrases like ``for any sentence $\phi$'' should not be
regarded as involving quantification over sentences.  When we say
``for all $x$, $\phi$'', we are thinking of $\phi$ as a sentence about
objects in our domain of discourse, any of which can be substituted
for the variable $x$ to give a true statement.  We do not think of
``$\{x \st \phi\}$ exists'' as a sentence about the sentence $\phi$;
we have not considered the question as to whether we regard sentences
themselves as inhabitants of our universe of discourse.  The
``sentence'' (for all $\phi$, $\{x \st \phi\}$ exists) is not found
in our language.  This remark will apply to other theorems or axioms
stated later as well, such as the Stratified\index{stratification}
Comprehension\index{comprehension} Theorem and 
the Axiom of Small Ordinals\index{Axiom of Small Ordinals}.  One should think
of an axiom or theorem of this kind as an infinite list of sentences, one for
each specific sentence $\phi$, rather than as a single sentence of our language
at all.  Axioms and theorems of this kind are properly called ``axiom
schemes'' and ``meta-theorems'' respectively.

\Exercises

\begin{enumerate}

\item  Express the sentence ``All swans are white'' in our formalized language.

\item  Express the sentences ``Everybody is loved by somebody'' and ``Somebody
  loves everybody'' in our formalized language.  Are these sentences equivalent
  in meaning?  Explain.

\item  Define the set of women with at least two children as in the proof of
  the Representation Theorem.  You may assume that appropriate basic predicates
  are realized by sets.

\item  Express the sentence ``The set $A$ has exactly two elements'' in our
  formalized language, using the membership\index{membership} and equality
  relations as the only predicates.  Can you express it in terms of membership
  only?

\end{enumerate}

