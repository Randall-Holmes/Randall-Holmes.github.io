\chapter[Stratified Comprehension]{Russell's Paradox and the\\ Theorem of
Stratified Comprehension}
\index{paradoxes!Russell}\index{stratification}\index{comprehension}

Every set that we have defined, we have defined in the format
$\{x \st \phi\}$, where $\phi$ is a sentence of first-order logic.
The reader should be wondering by now why we bother with all these
special cases instead of adopting the following

\begin{staraxiom}
 For each sentence $\phi$, $\{x \st \phi\}$ exists.
\end{staraxiom}

This axiom is called the (naive) Axiom of Comprehension\index{comprehension}.
It is false.
Consider the relation symbol $\in$ for membership\index{membership}.  Notice
that we did not include ``the set $[\in]$ = $\{(x,y) \st x \in y\}$ exists" as
an axiom in the natural place at the end of the last chapter.  We have a good
reason for this:

\begin{thm}
 $[\in]$\index{membership} does not exist.
\end{thm}

{\sc Proof} (Russell).~--- Suppose that $[\in]$ existed.  We construct the set 
\linebreak $\dom([=] - [\in])$.  $[=] - [\in]$ would be the set of ordered
pairs\index{ordered pair} $(x,y)$ such 
that $x = y$ and $x$ is not an element of $y$, i.e., the set of ordered
pairs $(x,x)$ such that $x$ is not an element of $x$, and dom($[=] - [\in]$)
would be the set of all $x$ such that $x$ is not an element of itself,
conveniently abbreviated \linebreak $\{x \st x \not\in x\}$.  If this set is an element
of itself, it is not an element of itself; if it is not an element of
itself, it is an element of itself.  The contradiction indicates that
our original assumption that $[\in]$ existed must have been in error.
\finpreuve

The core of this argument, called Russell's Paradox\index{paradoxes!Russell},
is a {\itshape reductio ad absurdum\/} of the proposed Axiom of
Comprehension\index{comprehension}: the set
$\{x \st$ not$(x \in x)\}$ cannot exist, and not($x \in x$) is
certainly a sentence 
of first-order logic.  The specific theorem proved is an artifact of
the set theory we are working in.

Since we cannot have $[\in]$, we settle for something less which
proves adequate.  We append a last axiom to our list of axioms
asserting the existence of specific relations\index{relation}:

\begin{axiom}{Axiom of Inclusion\index{inclusion!axiom of}}
 The set $[\subseteq]$ = $\{(x,y) \st x \subseteq y\}$ exists.
\end{axiom}

Using the results of the previous chapter, we see that we can
define\linebreak
$\{x \st \phi\}$ if $\phi$ mentions no primitive relation other than =,
$\pi_1$, $\pi_2$, or $\subseteq$.  It turns out that a wide class of sentences
$\phi$ which mention the relation $\in$ of membership\index{membership}, which
is {\itshape not\/} realized, as we 
saw above, are nonetheless realized by sets; there is a technique of
translation which converts some sentences with $\in$ to sentences without
$\in$.

We now define the class of ``stratified\index{stratification}" sentences of
first-order logic.  

\begin{definition}
 A sentence $\phi$ of first-order logic involving no relation\index{relation!as
 predicate} other than $\in$, $\pi_1$, $\pi_2$, or = is said to be
 {\upshape stratified} if it is possible to assign a non-negative integer to
 each
 variable $x$ in $\phi$, called the {\upshape type}\index{types 
 (relative)!definition of}\index{types (relative)|textbf} of $x$, in such a way
 that:
 \begin{enumerate}
  \item  Each variable has the same ``type'' wherever it appears.
  \item  In each atomic sentence $x = y$, $x \mathrel{\pi_1} y$,
    $x \mathrel{\pi_2} y$ in
    $\phi$, the ``types'' of the variables $x$ and $y$ are the same.
  \item  In each atomic sentence $x \in y$ in $\phi$, the ``type'' of $y$
    is one higher than the ``type" of $x$.
\end{enumerate}
\end{definition}

\begin{Thm}{Stratified Comprehension Theorem\index{Stratified Comprehension
Theorem}}
 For each stratified sentence $\phi$, the set $\{x \st \phi\}$ exists.
\end{Thm}

\preuve\ We would like to apply the results of the previous
chapter, but $\in$ is not realized.  We observe that ``$y \in z$" is
precisely equivalent to ``$\{y\} \subseteq z$".  We choose a natural\index{natural number}
number $N$ larger than the type assigned to any variable in $\phi$.  Our
strategy is to replace all reference to each variable $y$ of type $n$
with reference to the $(N - n)$-fold iterated
singleton\index{singleton!iterated} of $y$.  All 
occurrences of the relations $\pi_1, \pi_2$ and = are thus replaced by
equivalent statements involving iterated singleton\index{singleton image!iterated} images of
these relations, which are 
realized; an occurrence of $x \in y$ in which $x$ has type $n-1$ and $y$ has type $n$
is replaced by the equivalent $\iota^{N-n+1}(x) \subseteq^{\iota^{N-n}} \iota^{N-n}(y)$;
all occurrences of $\in$ are thus replaced by occurrences of
iterated singleton images of the relation $\subseteq$, which is realized.  Since each variable occurs with only one type\index{types
(relative)}, references 
to the $(N-n)$-fold iterated singleton\index{singleton!iterated} of the
variable $y$ of type $n$ 
(for example) can be replaced with references to a variable $y$
``restricted" to the collection of $(N-n)$-fold singletons, a
collection whose existence follows from our axioms (it is the result
of the singleton image operation\index{singleton image} on sets applied $N - n$
times to $V$).  The restriction is achieved by replacing each ``(for some $y,
\phi$)" with ``(for some $y$ ``$y$ is an $(N - n)$-fold singleton" and
$\phi'$)" ($n$ being the type\index{types (relative)} of $y$, and $\phi'$ being
the result of replacing references to the $(N - n)$-fold
singleton\index{singleton!iterated} of $y$ in $\phi$
with references to $y$) and similarly replacing references to the
$(N-m)$-fold singleton of $x$ (where $m$ is the type of $x$) with
references to $x$ and imposing the condition that the new $x$ is an
$(N-m)$-fold singleton. Since the set of $(N-n)$-fold singletons
exists, the property of being an $(N - n)$-fold singleton is
``realized" for each $n$.  Recall that the type of the variable $x$ in the set
definition $\{x \st \phi\}$ has been taken to be $m$; by applying the
result of the previous chapter at this point (all properties\index{properties}
and relations\index{relation!as predicate} appearing in the modified $\phi$ are
now ``realized"), we succeed in defining the set of $(N - m)$-fold
singletons\index{singleton!iterated} of objects $x$ for which $\phi$ is true;
we apply the axiom of set union\index{union!set} $N-m$ times to obtain the
desired set $\{x \st \phi\}$.  The proof of the Stratified Comprehension
Theorem \index{Stratified Comprehension Theorem} is complete.
\finpreuve

One question which might remain is that we have not explicitly
referred to the presence of atoms\index{atoms} in our theory in this
discussion.
It is sufficient to note that atoms are exactly those objects which
have no subsets\index{subset}; atomhood is expressible in terms of $\subseteq$.

It is worth mentioning that the Axiom of Singleton Images\index{singleton
image} and the Axiom of Set Union\index{union!set}, though they were introduced
in earlier chapters, were not needed for the proof of the Representation
Theorem and have found their first major application in the proof of the
Stratified Comprehension Theorem\index{Stratified Comprehension Theorem}.


\Exercises

\begin{enumerate}
\item Construct the set ${\cal P}\{A\}$ of all subsets\index{subset} of a given
  set using the primitive  operations  for constructing sets now available.
  (The solution appears later in the book).

\item  Verify that the intersection\index{intersection!set} $\bigcap[A]$ of all
  elements of a set $A$ (the set of all $x$ which belong to each element $B$ of
  $A$) has a stratified\index{stratification} definition, and so exists by the
  theorem of Stratified Comprehension\index{Stratified Comprehension Theorem}.
  Then construct it using the primitive set constructions.
\end{enumerate}
