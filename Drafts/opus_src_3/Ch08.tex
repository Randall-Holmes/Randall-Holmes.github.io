\chapter[Philosophical Interlude]{Philosophical
Interlude:\\  What is a Set, Anyway?\\ Motivating Stratification}\index{set}

In contrast with the rather natural basic constructions given in the
axioms, the Stratified Comprehension Theorem\index{Stratified Comprehension
Theorem} is not at all intuitive.
The usual way in which the set theory we are developing is initially
presented is to take as axioms Extensionality\index{extensionality}, the axiom
of Atoms\index{atoms}, and 
Stratified Comprehension\index{Stratified Comprehension Theorem}.  Our axiom of
Ordered Pairs\index{ordered pair} would also be 
needed if our precise development were to be followed.  The reason we
have not done this is that Stratified Comprehension\index{Stratified
Comprehension Theorem} is not
particularly easy to motivate.  However, Stratified
Comprehension\index{Stratified Comprehension Theorem} is a
very powerful tool, which we will want to be able to use.  In this
section, we will try to provide motivation for stratified\index{stratification}
comprehension\index{comprehension} as a criterion for set existence.



\section{What is a Set, Anyway?}\index{set}

We begin by returning to the vexed question of set existence on a more
fundamental level; what is a set supposed to be, anyway?  We outline
an informal answer to this question.

We don't intend to give any kind of formal definition of what a set
is; we stated at the outset that ``set" is an undefined notion of our
theory.  Nonetheless, we should be now be enough acquainted with a set
to say something about this on an informal level.

How do sets come into human experience?  An obvious example is that of
herds of animals or sets of tools.  Collections of discrete physical
objects are our usual first model for the notion of ``set''.  

But we need to be careful.  We think of a herd of animals as a large
physical object of which the individual animals are parts.  We have
learned that we cannot view the relation of element\index{membership} to set as
the relation of part to whole\index{relation!of part to whole}.  For example,
$a$ is an element of $\{a\}$,
and $\{a\}$ is an element of $\{\{a\}\}$, but $a$ is not an element of
$\{\{a\}\}$; this makes no sense if ``element'' is to be interpreted
as ``part'', because the relation of part to whole is transitive.  The
occurrence of singleton\index{singleton} sets in our example is telling; notice
that we do not distinguish between a herd of one animal and that single animal
(though we do regard the former as a very peculiar notion!).  

We give an example where we can tell in a mostly pre-mathematical
context that the relation of element\index{membership} to set must be
distinguished from the relation of whole to part\index{relation!of part to
whole}.  Suppose that John, Sam, and Tim are 
members of a committee.  We ask how many subcommittee rosters of two
members there can be; we comfortably answer ``three''.  We might be
asked for the set of all possible commmittee rosters; we would
probably list seven elements (the empty\index{empty set} committee can't get
much done).  But observe that, while we might be reasonably comfortable
viewing each committee as an object formed by conglomerating its
members, if we try to view sets of committee rosters as
conglomerations of their elements, we cannot draw distinctions that we
recognize as real; the set $\{\{$John, Sam$\},\{$John, Tim$\}\}$ of
two subcommittees could not be distinguished from the set $\{\{$John,
Sam$\},\{$John, Tim$\},\{$Sam, Tim$\}\}$ of three subcommittees.

Even collections of disjoint\index{disjoint} physical objects present a
difficulty. 
Consider the set of all human beings.  This is a set we all ``know'',
whose elements we can count, more or less.  Now consider the set of
all human cells!  This is a set we all ``know'', can count, more or
less, and which has (more or less) the same physical extent as the
previous set -- but it is not the same set because it does not have
the same number of elements by many orders of magnitude!  The ``sets''
of our experience, whose elements are discrete physical objects, need
for their exact identification not only their physical extent but the
kind of individual object into which they should be partitioned (they
can be thought of as ``typed'' collections).  Not all parts\index{relation!of
part to whole} of a ``set'' thus understood are its ``elements''; only those
distinguished parts of the understood ``type''.

We noted above that the correct analogue of the relation of part to
whole in the world of sets is the subset\index{subset} relation rather than the
membership relation\index{membership}.  Now use the fact (which we used in the
proof of the Stratified Comprehension Theorem\index{Stratified Comprehension
Theorem}) that the membership relation can be reduced to the
inclusion\index{inclusion} relation and the singleton\index{singleton}
operator: $x \in y$ iff $\{x\} \subseteq y$.  We have explained inclusion as
the intuitively familiar relation of part to whole\index{relation!of part to
whole}.  It remains to explain the singleton set construction.

The singletons are a collection of pairwise disjoint\index{disjoint} sets; if
we understand inclusion as the relation of part to whole, we can put this
informally by saying that the singletons\index{singleton}, whatever they are,
are a collection of discrete, non-overlapping objects.  Singletons are
disjoint objects like the physical objects which are ``elements''
(actually distinguished parts\index{relation!of part to whole}) of the ``sets''
of our everyday experience.

The singleton construction allows us to associate with each object in
the universe of our set theory an object taken from a domain of
discrete, non-overlapping objects.

The construction of a ``conglomerate'' object from a collection of
arbitrary sets corresponds to taking the union\index{union!set} of the
collection (the minimal collection which has every element of the original
collection as a {\itshape subset\index{subset}\/}).  Quite different
collections of arbitrary sets can have the same union.  But distinct
collections of singletons\index{singleton} do have distinct unions.

We interpret the singleton construction as choosing a ``token'' or
``counter'' to correspond to each object in our universe of discourse
(the universe of things of which we want to build sets).  The counters
are disjoint\index{disjoint} objects, which makes them satisfactory for
building sets by aggregation.  The singleton considered as a counter ``stands
for'' its element in the construction of sets.  So we can construe the
relation between singleton\index{singleton!as a ``name''} and element as a
special case of the familiar relation of symbol and referent; a singleton is a
kind of {\itshape name\/} for its element.  If we focus on the extension, we
can think of the singleton as a kind of label affixed to it to make it
possible for the extension to be itself an element of further sets.

In terms of this metaphor, a set will be understood to be an object
which has as its parts\index{relation!of part to whole} names (or tokens,
counters, labels) for its elements.  (In everyday terms, it can be thought of
as a {\itshape list\/} or {\itshape catalogue\/}.)  The
relation\index{membership} ``$x$ is an element of $y$'' will 
mean ``the name of $x$ is part of $y$, and $y$ is an aggregate of
names''.  Objects which are not aggregates of names will be
``atoms\index{atoms}'' in the sense of our set theory (though they will not be
``atomic'' objects in terms of this metaphor).

This does not complete the informal definition of ``set'', though.  It
seems intuitively reasonable that every aggregate of names, however
defined, should exist as an object of some sort, and we already know
that certain of these cannot be sets.  Consider the aggregate $R$ made
up of all sets $x$ such that $x \not\in x$.  This would be Russell's
impossible ``set'', if it were a set!  This aggregate has as
parts\index{relation!of part to whole} all 
``names'' $\{x\}$ such that $x \not\in x$, that is, such that $\{x\}$
is not part of $x$.  It seems reasonable to suppose that this object
would exist, though it might not be in our universe\index{universe, universal
set} of discourse.  We maintain informally that it {\itshape does\/} exist.  Is
$R \in R$?  $R \in R$ is defined as holding if and only if $\{R\}$ is part of
$R$ (by the informal definition of $\in$), but also as holding if and only if
$\{R\}$ is not a part\index{relation!of part to whole} of $R$ (by the
definition of $R$ itself)!  But notice that we have not made any provision for
an object $\{R\}$ (a ``name'' of $R$)!  We can't: for if $\{R\}$ existed, we
would be forced to conclude that $R \in R$ iff $R \not\in R$.

We have not arrived at a contradiction; we have arrived at the
conclusion that $R$ cannot have been assigned a singleton\index{singleton!as a
``name''} (a ``name'', in terms of our metaphor), which means that it is not in
our universe of discourse.  This shows us how to qualify the definition of
``set''.
This is the missing element of the definition of ``set'': a set\index{set} is
to be understood as an aggregation of names {\itshape which itself has a
name\/}; our conclusion about $R$ is that it exists (as an aggregate
object) but is not a set.  Other aggregations of names may be
understood as real objects (our metaphor certainly suggests that they
exist) but they are not sets.  We may informally define ``class'' to
mean ``aggregation of names, not necessarily having a name itself''.
Classes which are not sets may be termed ``proper classes''.  Proper
classes have elements but are not eligible to be elements.

Because we accept informally that classes which are not sets have some
sort of existence, we will not be {\itshape too\/} wary in the sequel about
using set-builder notation $\{x \st \phi\}$ (and
``function-builder\index{function-builder notation} notation'' $(x
\mapsto T)$ when this is introduced) to refer to classes which might
not be sets.  But we will always attempt to warn the reader when this
may be happening; we will not formally admit proper classes as objects
of our theory.

Our discussion so far is a philosophical analysis of the term ``set''
(and related terms) which should not prejudice us in our choice of a
formal system of set theory.  The Russell paradox\index{paradoxes!Russell} may
lose some of its ``paradoxical'' flavor as a result of this informal analysis
(we hope so).  We think of the paradox as not representing any kind of
difficulty with the nature of reality or fundamental limitation on
human understanding (as some seem to think), but as representing a
mistake in reasoning: the failure to appreciate the difference between
``arbitrary collection of labels of objects'' and ``arbitrary
collection of labels of objects which itself has a label''.  It is
quite reasonable that any condition at all should define an arbitrary
collection of labels of objects; but it is also quite reasonable that
a collection defined in terms of one's scheme for labelling
collections may prove impossible to label!



\section[Motivating Stratified Comprehension]{Motivating
Stratified Comprehension}\index{stratification}\index{comprehension}

The choice of a particular set theory is determined by what conditions
one thinks {\itshape ought\/} to define sets.  We turn to a motivation of
our particular choice (stratified comprehension).

The motivation we propose here is related to the discipline of
abstract data typing\index{data types!safety of abstract} in computer science.
Properties\index{properties} of the concrete implementation of an abstract
object which do not reflect properties of the object being implemented should
not be used by the programmer when manipulating the implemented abstraction.
The abstraction that we are trying to implement with sets is the notion of a
collection, a pure extension.  This coincides with the notion of {\itshape
class\/} in our intuitive picture.  A set implements a class, but it has an
additional feature, the name or label associated with it.  A
``type-safe''\index{type safety} property of sets or operation on sets should
not depend on details of which objects ``label'' which collections of objects.
Any relationship between the ``label'' attached to an extension and
the elements of the extension is an ``implementation-dependent''
feature of a set, and not a feature of the class it implements.
Self-membership\index{membership} of a set, for example, is the
property\index{properties} of identity 
between the label and one of the parts of the extension, which is
clearly not a property of the extension of the set; if the same
extension were labelled differently, it would cease to be an element
of itself.

This gives us an {\itshape a priori\/} reason to reject $\{x \st x \in
x\}$, for example, as a set definition, prior to any consideration of
paradox\index{paradoxes} (in this case, there is no paradox to consider unless
we go on to allow other set constructions, such as
complement\index{complement}).  The same
reasoning allows us to reject the definition of the Russell class as a
reasonable specification for a set {\itshape before\/} discovering that it
is impossible to implement.

One possible analysis of the reason for rejecting $\{x \st x \in x\}$
(or its paradoxical\index{paradoxes} complement) is that the same object $x$
appears in the sentence $x \in x$, further analyzed as $\{x\} \subseteq x$, in
two different roles, as a label (in which role we are concerned with
it only as an unanalyzed bare object) and as representing an
extension.  Further analysis reveals that there is a hierarchy of
further roles.  An object can be considered as a bare object (as a
label), or as the associated extension, a class.  But, further, we may
consider an object via the associated extensions of the members of its
associated extension, viewing it as a class of classes of bare
objects.  We may consider classes of classes of classes of objects,
and so forth.  These roles can be indexed by the natural\index{natural number}
numbers: an object being considered by itself is taking role 0, while an object
being considered as a collection of objects themselves taking role $n$
is taking role $n+1$.  The relationship between any two distinct roles
of one and the same object can be perturbed by permuting the
underlying ``labelling'' scheme; we should reject any definition of a
set in which any object appears in more than one role as
``implementation-dependent''.  This is exactly the criterion of
stratification\index{stratification}: in any sentence ``$x \in y$'', the role
taken by $y$ should be expected to have index one higher than that of the role
taken by $x$ if a coherent assignment of roles to objects is possible;
the relations of equality and left and right projection ($\pi_1$ and
$\pi_2$) are understood as relations between bare objects, and so make
sense only for pairs of objects playing the same role.

Historically, the motivation of stratification was as a simplification
of Russell's type\index{types (relative)} theory \footnote{Actually, this is a simplification of a quite streamlined version of type theory:  the type theory in Russell and Whitehead's famous {\em Principia Mathematica\/} is far more complicated.}, in which the different
``roles'' (object, class of objects, class of classes of objects, etc.) are
played by objects of different sorts or ``types''.  Our motivation here is
different from the historical motivation in assuming from the outset
that there is only one sort of object.  We will use the word ``type''
to refer to the different ``roles'', sometimes qualifying it as
``relative type\index{types (relative)}'' to remind ourselves that we do not
assume the existence of different kinds of object, as in Russell's original
theory, but of different levels of permitted access to one kind of
object.

There is some formal mathematical support for our motivation; it has
been shown that the stratified\index{stratification} conditions are exactly
those which are invariant under all redefinitions of the
membership\index{membership} relation by permutation which preserve the
totality of represented sets in a certain technical sense.

There is nothing in our metaphor to suggest that every object that has
a label (and so is capable of being an element of our universe) is an
aggregate of labels (a set).  In fact, the Axiom of Choice\index{axiom of
choice} will allow us to prove that there must be elements which are not sets.


\section{Technical Remarks}

It needs to be noted that just because a set is defined by an
unstratified\index{stratification} condition does not mean that it {\itshape
cannot\/} exist; it simply means that it {\itshape might not\/} exist; we are
free to reject such a set if it turns out that its existence would lead to
problems.  An example of a set which is impossible to define in a stratified
manner, but which can (but need not) exist in our set theory is the collection
having as members exactly the von Neumann numerals\index{von Neumann numeral}
`0', `1', `2',\ldots\ defined earlier.

The Stratified Comprehension Theorem\index{Stratified Comprehension Theorem} is
stated for sentences $\phi$ in which the only names of objects used are
variables.  But recall that we can use the Theory of Descriptions to introduce
names (the $x$ such that $\phi$), where $\phi$ is a sentence.  

It is straightforward to determine by examining the way in which names
(the $x$ such that $\phi$) are eliminated that a sentence containing
one or more such descriptions will be stratified after the term is
eliminated under the following refinement of the usual conditions:
relative types should be assignable to all terms (descriptions as well
as variables) with the same rules for atomic sentences, and, in
addition, the same type must be assigned to the variable $x$ as to any
name (the $x$ such that $\phi$) in which it is bound (since instances
of (the $x$ such that $\phi$) will actually be replaced with instances
of $x$ when the description is eliminated).  Notice that a name (the
$x$ such that $\phi$) cannot appear in a stratified sentence unless it
is itself stratified (since the sentence $\phi$ will still appear as
part of the larger sentence when the name is eliminated).

Operations in our set theory are theoretically implemented by complex
names with free variable parameters.  If a name has variables in it,
it is important to be aware of the difference in type (``role")
between the name and the free variables which appear in it.  Examples:
$(x,y)$ means ``The $z$ such that $z
\mathrel{\pi_1} x$ and $z \mathrel{\pi_2} y$"; here $z$, and so the whole name,
has the same type as $x$ or $y$; on the other hand, $\{x\}$ means ``the $y$
such that $x
\in y$ and for all $z$, if $z \in y$ then $z = x$"; here $y$, and so
$\{x\}$ must have type\index{types (relative)} one higher than that of $x$.
Both of these examples are easily handled using the intuitive concept of
``roles"; 
the projections\index{projections!relative type of} of a pair\index{ordered
pair} play the same ``role" as the pair itself (we regard pairing as a
construction acting on bare objects), while if $x$ is playing the role of a
bare object, it should be clear that $\{x\}$ is playing the role of a
collection of bare objects!  This, in turn, would enable us to conclude that
the relation which every $x$ has to its singleton\index{singleton} $\{x\}$ is
``implementation-dependent" and may turn out not to be a set (as indeed we will find is the case).  We are not planning to make
explicit use of the Theory of Descriptions; we noted its existence to
justify the use of names in general and their role in relation to the
notion of stratification\index{stratification}.

Another fact about stratification restrictions should be noted: a
variable free in $\phi$ in a set definition $\{x \st \phi\}$ may
appear with more than one type without preventing the set from
existing, as long as this set definition is not itself embedded in a
further set definition.  The reason for this is that such a set
definition can be made stratified by distinguishing all free
variables: the resulting definition is supposed to work for {\itshape
all\/} assignments of values to those free variables, including those
in which some of the free variables (even ones of different type) are
identified with one another.  For example, the set $\{x,\{y\}\}$ has a
stratified definition; the set $\{x,\{x\}\}$ has an unstratified
definition, but the existence of $\{x,\{y\}\}$ for all values of $x$
and $y$ ensures the existence of $\{x,\{x\}\}$ for any $x$.  But a
term $\{x,\{x\}\}$ {\itshape cannot\/} appear in the definition of a
further set in which $x$ is bound.

The Stratified Comprehension Theorem\index{Stratified Comprehension Theorem}
can be used as an axiom (technically, an axiom scheme) of this set theory.  If
it is used as an axiom, the only other axioms which are needed (of those
introduced so far) are those of Extensionality\index{extensionality},
Atoms\index{atoms}, and Ordered Pairs\index{ordered pair}.  Look
at the definitions given in set-builder notation for the objects
stated to exist by each other axiom, and you will see that each of the
other axioms is actually a case of Stratified Comprehension\index{Stratified
Comprehension Theorem}.

The use of a primitive ordered pair is not strictly necessary for the
development up to this point, although it is far more convenient than
the alternative.  The Kuratowski pair\index{ordered pair!Kuratowski}
$\left<x,y\right>$ = $\{\{x\},\{x,y\}\}$ could have been used instead, but the
forms of certain axioms would have been very hard to motivate without the
notion of stratification having already been introduced.  The main
point is that the relations $\pi_1$ and $\pi_2$ do not exist for the
Kuratowski pair, because if $x$ and $y$ are assigned type\index{types
(relative)} $n$, the Kuratowski pair $\left<x,y\right>$ is assigned type $n+2$.
The relations $\kappa_1$ and $\kappa_2$ which $\{\{x\}\}$ =
$\left<x,x\right>$ and $\{\{y\}\}$ = $\left<y,y\right>$, respectively,
have to $\left<x,y\right>$ have to be used instead.  The other axioms
are true for the Kuratowski pair\index{ordered pair!Kuratowski}.  The results
on representation of
sentences of first-order logic are also true for the Kuratowski pair,
but the Axiom of Set Union\index{union!set} has to be used to collapse sets of
objects $\left<x,x\right>$ (for example) down to the corresponding sets of
objects $x$ at certain points in the proof, and the Axiom of
Singleton\index{singleton image}
Images has to be used to reverse this process.  The use of a primitive
ordered pair satisfying our form of the Axiom of
Projections\index{projections!relative type of}\index{types (relative)} does
have some additional strength; it allows one to prove that the universe is
infinite, which we could not do with the other axioms we have up to
this point.  We will not consider the use of the Kuratowski pair\index{ordered
pair!Kuratowski} further.
