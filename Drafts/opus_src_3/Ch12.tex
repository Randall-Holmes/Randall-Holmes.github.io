\chapter{The Natural Numbers}


\section[Induction and Infinity]{Definition of the Natural Numbers.\\
Induction and Infinity}\index{infinity}

We noted earlier that the numbers are defined in a more usual
treatment of set theory as follows: `0' = $\vide$, `1' = $\{\mbox{`0'}\}$,
`2' = $\{\mbox{`0'},\mbox{`1'}\}$, and so forth.  These are called the von
Neumann\index{von Neumann numeral}
numerals, and the existence of each von Neumann numeral is a
consequence of our axioms.  Observe that the successor `$n$'$^+$ of
each von Neumann numeral `$n$' is $\mbox{`$n$'} \cup \{\mbox{`$n$'}\}$, and you
will see why these numerals are unsatisfactory from our standpoint.
The expression $x^+ = x \cup \{x\}$ cannot appear in a
stratified\index{stratification}
sentence, since it contains $x$ with two different types\index{types
(relative)}!  If we cannot define the successor\index{successor!of a von
Neumann natural number} operation for these numerals, we cannot use them
effectively.

Fortunately, there is a much more natural definition of the
natural numbers (actually the original set-theoretic definition of
number) which succeeds here (and fails in the more
usual\index{Zermelo--Fraenkel set theory} treatment of set theory).

We have already defined 0 as $\{\vide\}$ and 1 as ${\cal P}_1\{V\}$.  0 is
defined as the set of all sets with 0 elements, and 1 is defined as
the set of all sets with 1 element (but without the circularity
suggested by the English phrases).  This seems quite natural.  The
number 3, for example, can be thought of as a property\index{properties} shared
by all sets with three elements; properties are represented by sets, so 3
might be expected to be the set of all sets with three elements.

Now we need to address the question of whether the number 3 as
defined above actually exists.  What we will do is define the
successor\index{successor!natural number} operation: 

\begin{definition}
  For any set of sets $A$, we define $A + 1$ as the
  collection
  $$
   \{a \cup \{x\} \st \mbox{$a \in A$ and $x \not\in a$}\}
  $$
  of unions of elements $a$ of $A$ and singletons\index{singleton} of
  objects $x$ not in $a$.
\end{definition}

If the set of all sets with $n$ elements exists (call it ``$n$" for
the moment), then it should be clear that ``$n$" + 1 will be the
collection of all sets with $n+1$ elements, for any concrete natural
number $n$.  Since we have 0 and 1, we can construct $2 = 1 + 1$, $3 = 2
+ 1$, $4 = 3 + 1$, and so forth.

We are now in a better position than with the von Neumann\index{von Neumann
numeral} numerals in two ways: the numerals we are using are more natural
objects, and (as can easily be checked) the function\index{function}
$\Inc = (A \mapsto  A + 1)$ which realizes the successor operation actually
exists.  We can go 
further: we can define the set of natural numbers (or something which
looks as if it {\itshape ought\/} to be the set of natural numbers!): it would
appear that any set which contains 0 and is closed under the function
Inc will contain all natural numbers (by the principle of mathematical
induction) and it seems unreasonable that any other objects would
belong to all such sets.  We make some definitions:

\begin{definitions}
 $\Inc = (A \mapsto  A + 1)$.

 $\Ind$, the collection of\/ {\upshape inductive sets\index{inductive
   set}}, is  defined as
   $\{A \st 0 \in A$ and $\Inc[A] \subseteq A\}$; an inductive set is one which
   contains 0 and contains the ``successor\index{successor!natural number}" of
   each of its elements. 

 ${\cal N}$, the collection of {\upshape natural numbers}\index{natural
   number|textbf}, is  defined as $\bigcap [$Ind$]$, the collection of objects
   which belong to all inductive sets.

 $\Fin$, the collection of {\upshape finite\index{finite!sets, definition of}
   sets},
   is defined as $\bigcup[{\cal N}]$.

 A set which is an element of a natural number $n$
   may be referred to as a {\upshape set of size $n$\/} or {\upshape a set with
   $n$ elements}; the natural number to which a finite set $A$ belongs may
   be referred to as the {\upshape size of $A$\/}.
\end{definitions}

It is straightforward to check that the existence of each of the sets
defined above follows from Stratified Comprehension\index{Stratified
Comprehension Theorem}.  The definition 
of Fin may require a little explanation: the union\index{union!set} of ${\cal
N}$ consists of all the elements of the elements of ${\cal N}$; the
elements of a natural number $n$ are the sets with $n$ elements, so
$\bigcup[{\cal N}]$ consists of the sets $A$ such that for some
natural number $n$ (depending on $A$), $A$ has $n$ elements, which
seems a reasonable definition for the collection of finite sets.

An immediate corollary of the definition of the set ${\cal N}$ is the
familiar


\begin{ThmEtc}{Principle of Mathematical
Induction.\index{induction!mathematical}}\slshape
\index{induction!transfinite|see {transfinite}}
 Suppose that $S$ is a set
 of natural numbers.  If $0\in S$ and for all $n$, $n+1 \in S$ if $n
 \in S$, it follows that $S={\cal N}$; all natural numbers are in $S$.
\end{ThmEtc}

A formulation in terms of properties\index{properties} of numbers instead of
sets of numbers will work as well, as long as the properties are required to
be stratified\index{stratification}, so that they define sets.  A later axiom
will have as a consequence that induction\index{induction!mathematical, on an
unstratified condition} holds for unstratified\index{stratification} conditions
as well, 
but induction on unstratified\index{stratification} conditions is not needed
for arithmetic (it has very strong consequences in set theory).

We prove a perfectly obvious

\begin{thm}
 ${\cal N}$ is a partition\index{partition} of\/ $\Fin$.
\end{thm}
using the following
\begin{definition}
 Sets $A$ and $B$ are said to be {\upshape equivalent\index{equivalence (of
 cardinality)!defined}\/}\index{equivalence (of cardinality)|textbf} if there
 is a bijection\index{bijection} $f$ between $A$ and $B$.
\end{definition}
and
\begin{lemme}
 Equivalence is an equivalence relation\index{equivalence relations,
 equivalence classes} on sets.
\end{lemme}

{\sc Proof of the Lemma.~---} Equivalence exists as a relation by Stratified
Comprehension\index{Stratified Comprehension Theorem}.  To prove reflexivity,
observe that identity functions\index{function}
are bijections\index{bijection}.  To prove symmetry, observe that
inverses\index{inverse function} of bijections exist and are bijections.  To
prove transitivity, observe that compositions\index{composition} of bijections
are bijections.\finpreuve

{\sc Proof of the Theorem.~---} We prove that for any $A \in n$, $B \in n$ iff
$A$ and $B$ are equivalent.
We prove this by induction\index{induction!mathematical}: consider the set of
all $n$ such that for any $A \in n$, $B \in n$ iff there exists a
bijection\index{bijection} $f$ between $A$ and $B$.
Certainly 0 belongs to this set: it has only one element, $\vide$, and
there is a bijection between a set and $\vide$ only if the set is itself
empty\index{empty set}.  Suppose that $n$ belongs to this set: then any $A \in
n+1$ equals $ A' \cup \{x\}$, where $A' \in n$ and $x$ is not in $A'$.
Similarly, any $B \in n+1$ equals $B' \cup \{y\}$ with parallel conditions.  By
inductive hypothesis, there is a bijection between $A'$ and $B'$, which we
extend by mapping $x$ to $y$.  Now suppose that there is a bijection $f$
between an element $A$ of $n+1$ and a set $B$: $A = A' \cup \{x\}$, and $B =
f[A'] \cup \{f(x)\}$. $f[A']$ belongs to $n$ by hypothesis, so $B$ belongs to
$n+1$.  It follows that if two natural\index{natural number} numbers share an
element, they are equal (since equivalence is an equivalence
relation\index{equivalence relations, equivalence classes}), so the collection
${\cal N}$ of natural numbers is pairwise disjoint\index{disjoint} as a
collection of sets, so partitions\index{partition} its set union Fin.
\finpreuve

There is one possible problem.  Suppose that the universe\index{universe,
universal set} actually contained only $n$ objects for some natural number $n$.
We would then find that the numeral for $n+1$ and each subsequent numeral
would be $\vide$, the empty set\index{empty set}!  Fortunately, we can prove
that this is not the case.  Usually, a set theory has an Axiom of
Infinity\index{infinity} to take care of this case; ours is no exception, but
we have already smuggled in the Axiom of Infinity\index{infinity} in the guise
of the Axiom of Projections\index{projections} (this axiom ensures that the
relative types\index{types (relative)} of the projections of an ordered
pair\index{ordered pair} are the same as the relative type of the pair itself).
We can see from above that if the universe were finite\index{finite!size of
universe?},
$\vide$ would be a natural\index{natural number} number.  We prove the

\begin{Thm}{Theorem of Infinity\index{infinity!theorem of}}
 $\vide$ is not a natural number.
\end{Thm}

\preuve\ We show that for any set $A$ belonging to a natural number $n$, $A
\times \{0\}$ belongs to the same natural number.  We do this by
induction\index{induction!mathematical} on the property of $n$ ``for all $n \in
{\cal N}$, $A \in n$ implies $A \times \{0\} \in n$".  This is clearly true of
0, since $\vide$ (the sole element of 0) is equal to $\vide \times \{0\}$.
Suppose that it is true of $n$; then the elements of $n+1$ are exactly the
objects $A \cup \{x\}$ for $A \in n$ and $x$ not in $A$; but $A \times \{0\}$
is in $n$ for each such $A$ by inductive hypothesis, and so $(A \cup \{x\})
\times \{0\} = (A \times \{0\}) \cup \{(x,0)\}$ belongs to $n+1$ (since $(x,0)$
cannot belong to $A \times \{0\}$, $x$ not being in $A$).  Now our condition on
$n$ defines a set, which we have just shown to be inductive\index{inductive
set}, so all elements of ${\cal N}$ belong to it.
Now we show that the collection of nonempty elements of ${\cal N}$ is
inductive; certainly 0 belongs to it; if $n$ is nonempty, take an
element $A$ of $n$, and form $(A \times \{0\}) \cup \{(0,1)\}$, which must
belong to $n+1$.
\finpreuve

\begin{cor}
 Every natural\index{natural number} number $n$ has an element which is not
 $V$. 
\end{cor}

\preuve This is a corollary of the construction:  if $X$ belongs to $n$, $X
\times \{0\}$ belongs to $n$ and is not equal to $V$.  This means that whenever
we choose an element $X$ of $n$, we can do it in such a way that there is an
$x$ which is not in $X$.
\finpreuve

The following theorem gives us the capability of inductive
{\itshape definition\/} of functions\index{function}, without any relaxation of
stratification\index{stratification} restrictions:
\index{recursion!transfinite|see {transfinite}}
\begin{Thm}{Recursion\index{recursion|textbf} Theorem}
 If $X$ is a set, $x$ is an element of $X$, and $f$ is a
 function in $[X \rightarrow  X]$, there is a unique function $g$ in $[{\cal N}
 \rightarrow  X]$ such that $g(0) = x$ and $g(n+1) = f(g(n))$ for each
 natural\index{natural number} number $n$.
\end{Thm}

\preuve\ The function\index{function} $g$ is defined very much as ${\cal N}$
is:  it is the intersection\index{intersection!set} of all sets which contain
$(0,x)$ and are closed under the function $((n,y) \mapsto  (n+1,f(y)))$.
\finpreuve

\begin{definition}
 For $f$ a function, $x$ any object, and $n$ a natural number, we define
 $f^n(x)$ as $g(n)$, where $g$ is the function with domain $\cal N$ such that
 $g(0) = x$ and $g(n+1) = f(g(n))$ for each natural\index{natural number} 
 number $n$.
\end{definition}




\section{Peano Arithmetic}

We could give inductive definitions of the operations of
addition and multiplication, but we prefer to give ``natural" ones.
After defining the operations, we will prove the axioms of Peano
arithmetic.

\begin{definition}
 For $m,n \in {\cal N}$, $A,B \in m,n$ respectively, we define $m + n$ as
 the natural\index{natural number}\index{addition!of natural numbers} number
 which contains $A \oplus B$, if there is one.
\end{definition}

\begin{thm}
 The two definitions of $m+1$ agree.
\end{thm}

\preuve The typical element of $m+1$ as originally defined is $M \cup \{x\}$,
for $M \in m$ and $x$ not in $M$; the typical element of $m+1$ as defined here
is $M \times \{0\} \cup (\{x\} \times \{1\})$ = $M \times \{0\} \cup
\{(x,1)\}$; but $M \times \{0\} \in m$ and $(x,1)$ is certainly not in $M
\times \{0\}$.
\finpreuve

\begin{definition}
 For $m,n \in {\cal N}$, $A,B \in m,n$ respectively, we define
 $mn$\index{multiplication!of natural numbers} as
 the natural number which contains $A \times B$, if there is one.
\end{definition}

The numbers $m+n$ and $mn$ are necessarily unique by the result that
${\cal N}$
is pairwise disjoint\index{disjoint}.  We will prove below that they actually
exist in all cases.

\begin{thm}
 $0 \in N$.
\end{thm}

\preuve\ True by definition of ${\cal N}$.
\finpreuve

\begin{thm}
 If $n \in {\cal N}$, $n+1 \in {\cal N}$.
\end{thm}

\preuve\ True by definition of ${\cal N}$.
\finpreuve

\begin{thm}
 If $n \in {\cal N}$, $n+1 \neq 0$.
\end{thm}

\preuve\ $\vide$, the sole element of 0, is not of the form $A \cup \{x\}$ for
an $A \in n$ and $x$ not in $A$; all elements of $n+1$ must be of this form.
\finpreuve

\begin{thm}
 If $n \in {\cal N}$ and $n \neq 0$, then $n = m+1$ for some $m$.
\end{thm}

\preuve\ Trivial induction\index{induction!mathematical}.
\finpreuve

\begin{thm}
 If $n,m \in {\cal N}$ and $n+1 = m+1$, then $n = m$.
\end{thm}

\preuve\ We proceed by induction on the property of $n$ ``for all $m$, if $n+1
= m+1$, then $n = m$".  If $0+1 = m+1$, then every singleton\index{singleton}
$\{x\}$ (element of $0+1$) is of the form $A \cup \{y\}$, for $A \in m$ and $y$
not in $A$; but $y$ clearly must be $x$, and so $A$ must be empty, i.e., $m =
0$.  This shows that 0 has the property.  Suppose that for all $m$, if $n+1 =
m+1$, then $n$ = $m$.  Now suppose that $n+1+1 = m+1$.  This means that any set
of the form $A \cup \{x\} \cup \{y\}$, where $A \in n$ and $x, y$ are distinct
and both not in $A$, is also a set of the form $B \cup \{z\}$, where $B \in m$
and $z$ is not in $B$ (there are such sets because all numbers are nonempty).
We see that $B$ is nonempty (it must contain $x$ or $y$ or both) so $m = q+1$
for some $q$.
If $z$ is $x$ or $y$, we see that $q+1 = n+1$, since then $B = A \cup \{$the
other of $x$ and $y\}$, so $q = n$ and $m = n+1$ by inductive hypothesis.  If
$z$ is not $x$ or $y$, we consider $B' = B - \{x\} \cup \{z\}$; this belongs to
$m$, because there is a bijection\index{bijection} between it and $B$.  Now $B
\cup \{z\}$ = $B' \cup \{x\}$ = $A \cup \{x\} \cup \{y\}$, and, by deleting $x$
from both sets, we see that $q+1 = m = n+1$, as before, so $q = n$ by inductive
hypothesis, and $m+1 = n+1$.  We have shown that $n+1$ has the property if $n$
does, so the proof is complete by induction\index{induction!mathematical} (on a
stratified\index{stratification} condition which defines a set).
\finpreuve

\begin{thm}
 For each $m$, $m+0 = m$.
\end{thm}

\preuve\ If $A \in m$, $B \in 0$, $B$ = $\vide$ and $A \oplus \vide$ = $A
\times \{0\} \in m$.
\finpreuve

\begin{thm}
 For each $m,n$ for which $m + n$ exists, $m + (n+1) = (m + n)+1$.
\end{thm}

\preuve\ Let $M, N$ belong to $m, n$ respectively.  An element of $m + (n+1)$
is $M \times \{0\} \cup ((N \cup \{x\}) \times \{1\})$ for some $x$ not in $N$,
or $M \times \{0\} \cup N \times \{1\} \cup \{(x,1)\}$.  But $M \times \{0\}
\cup N \times \{1\}$ belongs to $m + n$, and $(x,1)$ belongs to neither $M
\times \{0\}$ nor $N \times \{1\}$, so this set also belongs to $(m + n) +
1$.\linebreak
\mbox{}\finpreuve

\begin{cor}
 $m + n$ exists for all $m,n$.
\end{cor}

\preuve\ By induction\index{induction!mathematical} on the condition on $n$,
``$m+n$ exists for all $m$".  This condition, the closure property of
addition\index{addition!closure property of}, is important for the next
section.
\finpreuve

\begin{thm}
 For all $m \in N$, $m0 = 0$.
\end{thm}

\preuve Let $M$ be an element of $m$.  $M \times \vide = \vide$.
\finpreuve

\begin{thm}
 For all $m,n \in {\cal N}$, $m(n+1) = mn + m$.
\end{thm}

\preuve\ Let $M,N$ belong to $m,n$ and let $x$ not belong to $N$.  $M \times (N
\cup \{x\})$ belongs to $m(n+1)$, and is equal to $(M \times N) \cup (M \times
\{x\})$.  The function\index{function} which sends $(y,z)$ to $((y,z),0)$ for
$y \in M$, $z \in N$ and sends $(y,x)$ to $(y,1)$ for $y \in M$ is well-defined
and a bijection\index{bijection} between this set and $(M \times N) \oplus M$,
an element of $mn + m$, so these two numbers are equal.
\finpreuve

\begin{cor}
 $mn$ exists for each pair of natural\index{natural number} numbers $m,n$.
\end{cor}

\preuve\ By induction\index{induction!mathematical} on the condition on $n$,
``$mn$ exists for all $m$''.\linebreak
\mbox{ }\finpreuve

All conditions expressed in the language of first-order logic
with equality, addition\index{addition!of natural numbers} and
multiplication\index{multiplication! of natural numbers}, with all variables 
restricted to ${\cal N}$, are stratified\index{stratification}, so the
definition of ${\cal N}$ provides us with the Axiom of
Induction\index{induction!axiom} for Peano arithmetic (we
do not need the stronger theorem provable for our full set theory).
We have proven all the axioms of Peano arithmetic as interpreted in
set theory.  Of course, with our ``natural" definitions of operations,
the usual algebraic properties of the system of natural numbers are
fairly easy to prove.






\section{Finite Sets. The Axiom of Counting}\index{finite!set}

\begin{thm}
 Any subset\index{subset} of a finite set is finite.
\end{thm}

\preuve\ We restate the theorem in the form ``for each natural\index{natural
number} number $n$, any subset of an element of $n$ is finite''.  We prove 
this theorem by induction\index{induction!mathematical}.  It is clearly true in
the case $n=0$.  Now let $k$ be a natural number such that all subsets of any
set of size $k$ are finite.  Let $A$ be a set of size $k+1$ and let $x$ be any
element of $A$.  Let $B$ be any subset of $A$.  If $x \not\in B$, then
$B \subseteq A-\{x\}$, a set of size $k$, and $B$ is finite by
inductive hypothesis.  If $x\in B$, then $B- \{x\}$ is finite by
inductive hypothesis, belonging to some natural number $m$, and $B$
itself is then seen to belong to $m+1$, and so is seen to be finite.
The proof of the theorem is complete.
\finpreuve

\begin{thm}
 If $A$ and $B$ are finite\index{finite!set} sets, $A\cup B$ is finite.
\end{thm}

\preuve There is a bijection\index{bijection} between $A \cup B$ and $A \oplus
(B-A)$.  $A$ is finite by hypothesis, $B-A$ is finite by the preceding
theorem (it is a subset of $B$), and disjoint sums\index{disjoint!disjoint sum}
of finite sets are finite by the closure property of
addition\index{addition!closure property of}, proved in the previous
subsection.
\finpreuve

\begin{thm}
 The union\index{union!set} of any finite set of finite sets is finite.
\end{thm}

\preuve\ We restate this as ``for each natural\index{natural number} number
$n$, the union of any collection of $n$ finite sets is finite''.  We prove this
by induction\index{induction!mathematical} on $n$.  The case $n=0$ is trivial;
the union of the empty\index{empty set} set is empty.  Let $k$ be a natural
number such that any union of $k$ finite\index{finite!set} sets is finite.
Consider a set $\cal A$ of $k+1$ finite sets, and let $A \in {\cal A}$.  The
union of ${\cal A}-\{A\}$ is finite by hypothesis, and $A$ is finite by
hypothesis.  The union of $\cal A$ is then the Boolean\index{Boolean algebra,
operations} union of two finite sets, which we know to be finite by the
preceding theorem.  The proof of the theorem is complete.
\finpreuve

\begin{thm}
 The power set\index{power set} of any finite set is finite.
\end{thm}

\preuve\ We restate this as ``for each natural number $n$, the
power set of any element of $n$ is a finite set''.  We prove this by
induction on $n$.  The case $n=0$ is obvious; the power set of the
element of 0 has one element.  Let $k$ be a natural\index{natural number}
number such that the power set of any element of $k$ is finite.  Let $A$ be an
element of $k+1$ and let $x$ be an element of $A$.  ${\cal P}\{A-\{x\}\}$ is 
finite\index{finite!set} by hypothesis, and so belongs to a natural number $m$.
There is an obvious bijection\index{bijection} between ${\cal P}\{A\}$ and
${\cal P}\{A-\{x\}\}\oplus{\cal P}\{A-\{x\}\}$; send each subset\index{subset}
which does not contain $x$ to its pair with 0 and each subset $B$ which does
contain $x$ to $(B-\{x\},1)$.  Thus, we see that ${\cal P}\{A\}\in
m+m$, and so is finite.  The proof of the theorem is complete.
\finpreuve

\begin{thm}
 The ``singleton\index{singleton image} image'' ${\cal P}_1\{A\}$ of a finite
 set $A$ is finite.
\end{thm}

\preuve\ The power set\index{power set} of $A$ is finite by the preceding
theorem, and the ``singleton image'' of $A$ is a subset of this finite set, so
finite.
\finpreuve

\begin{thm}
 Every natural\index{natural number} number $n$ contains a set of the form
 ${\cal P}_1\{A\}$.
\end{thm}

\preuve\ By induction\index{induction!mathematical}.  This is certainly true
for $n=0$ (the empty\index{empty set} set is its own
``singleton\index{singleton image} image'').  Suppose that it is true
for $k$; we are given a set ${\cal P}_1\{A\}\in k$, and we consider
${\cal P}_1\{A\} \cup \{\{y\}\} ={\cal P}_1\{A \cup \{y\}\} $ for $y
\not\in A$; this is clearly an element of $k+1$ of the desired form.
The proof of the theorem is complete.
\finpreuve

What we {\itshape cannot\/} prove is the following obvious

\begin{ThmEtc}{$*$Theorem}\slshape
 If $A \in n$, ${\cal P}_1\{A\} \in n$.
\end{ThmEtc}

We can prove this for each concrete natural\index{natural number} number, but
the obvious proof by induction\index{induction!mathematical} does not work,
because the condition ``$A \in n$ iff ${\cal P}_1\{A\} \in n$'' is not
stratified\index{stratification}.  Methods beyond the scope of this book show
that the purported theorem {\itshape cannot\/} be proven from our axioms given
so far.

It is straightforward to show that $A$ and $B$ are equivalent\index{equivalence
(of cardinality)} iff ${\cal P}_1\{A\}$ and ${\cal P}_1\{B\}$ are equivalent.
This justifies the following

\begin{definitions}
 $T\{n\}$\index{$T$ operation!on natural numbers} is defined as the
   natural 
   number containing ${\cal P}_1\{A\}$ for each $A \in n$.  Notice that the 
   relative type\index{types (relative)} of $T\{n\}$ is one higher than that of
   $n$.

 $T^{-1}\{n\}$ is defined as the natural\index{natural number} number $m$
   such that $T\{m\} = n$.  Notice that the relative type of
   $T^{-1}\{n\}$ is one lower than that of $n$.
\end{definitions}

\begin{thm}
 $T^{-1}\{n\}$ is well-defined for every $n$.
\end{thm}

\preuve\ By a theorem above, there is a set ${\cal P}_1\{A\} \in
n$.  $A$ is finite\index{finite!set}, because it is the union of this finite
collection of finite (one-element) sets; it clearly belongs to $T^{-1}\{n\}$.
\finpreuve

We now introduce a new axiom (temporarily; it will turn out to be a
consequence of the Axiom of Small Ordinals\index{Axiom of Small Ordinals}
introduced later):

\begin{axiom}{Axiom of Counting\index{Axiom of Counting|textbf}}
  For all natural\index{natural number} numbers $n$, $T\{n\} = n$\index{$T$
  operation!on natural numbers}.
\end{axiom}

\begin{cor}
 For all natural numbers $n$, $T^{-1}\{n\} = n$.
\end{cor}

The reader may wonder why we went to the trouble of defining the $T$
operation and its inverse.  The answer lies in the following result:

\begin{Thm}{Restricted Subversion Theorem\index{Restricted Subversion Theorem}}
 If a variable $x$ in a sentence $\phi$ is restricted to $\cal N$, then its
 type\index{types (relative)} can be freely raised and lowered; i.e., such a
 variable can safely be ignored in making type assignments for
 stratification\index{stratification}.
\end{Thm}

\preuve Any occurrence of a variable $n$ restricted to $\cal N$ can be 
replaced by $T\{n\}$ (which lowers the type of $n$ by one) or by
$T^{-1}\{n\}$ (which raises the type of $n$ by one).  This can be
repeated as needed to give any occurrence of a variable restricted to
$\cal N$ any desired type.
\finpreuve

A more general theorem of the same kind appears in a later chapter.

We prove some obviously true theorems which require the Axiom of Counting:

\begin{thm}
 $\{1,\ldots,n\} \in n$.
\end{thm}

\preuve\ It is straightforward to show that $\{1,\ldots,n\} \in
T^2\{n\}$ for each natural\index{natural number} number $n$, by
induction\index{induction!mathematical}.  The double application of $T$ is
needed 
to stratify\index{stratification} the sentence so that the 
induction can be carried out; the Axiom of Counting allows us to
eliminate it.  This obviously true statement was the original form in
which the Axiom of Counting was proposed by Rosser.
\finpreuve

\begin{thm}
 A union\index{union!set} of $m$ disjoint\index{disjoint} sets each belonging
 to $n$ belongs to $mn$ (this is shorthand for ``if $A \in m \in {\cal N}$ and
 $A \subseteq n \in {\cal N}$ then $\bigcup[A] \in mn$'').
\end{thm}

\preuve\ It is straightforward to show by induction that the
union of $m$ disjoint sets each belonging to $n$ has $T^{-1}\{m\}n$\index{$T$
operation!on natural numbers}
elements. Also notice that for $A \in m \in {\cal N}$, $B \in n\in
{\cal N}$, it is easy to see that $A \times B$ is the union of
$T\{m\}$ sets, each of size $n$; the $T\{m\}$ sets are, of course, the
``rows'' $\{(a,b)\st b \in B\}$ for each $a$: there is a
stratified\index{stratification} definition of a bijection\index{bijection}
between the set of ``rows'' and the set of singletons\index{singleton image}
$\{a\}$ of elements of $A$.  The application of $T$ or $T^{-1}$ are needed 
for stratification\index{stratification}, and are eliminable by the Axiom of
Counting\index{Axiom of Counting}.
\finpreuve


\Exercises

\begin{enumerate}
 \item  Prove that $2 + 2 = 4$.

 \item  Give a set-theoretical proof of the distributive law of
   multiplication\index{multiplication!of natural numbers} over
   addition\index{addition!of natural numbers}.

 \item  Prove that the Cartesian\index{Cartesian product} product of two
   finite\index{finite!set} sets is a finite set.

 \item  Define the function\index{function} $(n \in {\cal N}\mapsto2^n)$
   recursively\index{recursion} (give a definition of the function as a set of
   ordered pairs\index{ordered pair}).  Try to prove the theorem ``for all $n
   \in {\cal N}$, $A \in n$ implies ${\cal P}\{A\} \in 2^n$''.  If you do this
   carefully, you should find that there is an obstruction to the
   natural\index{natural number} proof involving
   stratification\index{stratification}.  Complete the proof using the Axiom of
   Counting\index{Axiom of Counting}; what is the form of the theorem you could
   prove without using the Axiom of Counting?

\item Let $A$ and $B$ be any sets:  show that $A$ and $B$ are equivalent if and only if ${\cal P}_1\{A\}$ and ${\cal P}_1\{B\}$ are equivalent.

\end{enumerate}
