\chapter[Equivalence and Order]{Special Kinds of
Relations:\\  Equivalence and Order}
\index{relation!kinds of}
\index{relation!properties of relations|see {properties}}

We now consider some interesting properties\index{properties!of relations}
which may belong to particular relations.  A relation $R$ is said to be
{\itshape reflexive\index{reflexive relation|textbf}\/} if $x \rR x$
for every $x$ in $\Dom(R)$ (recall that this 
denotes the full domain\index{domain!full} of $R$, the union of its domain and
range).
This can be readily expressed as
$$
 ([=] \cap (\Dom(R) \times \Dom(R))) \subseteq R.
$$
A relation $R$ is said to be {\itshape symmetric\/} if $x
\rR y$ implies $y \rR x$.  This can be expressed very briefly
as $R^{-1} = R$.  A relation is said to be {\itshape transitive\/} if
($x \rR y$ and $y \rR z$) implies $x \rR z$.  This can
be abbreviated $R|R \subseteq R$.



\section[Equivalence Relations and Partitions]{Equivalence
Relations and Partitions}
\index{relation!kinds of}
\index{relation!equivalence|see{equivalence relations, equivalence classes}}

A relation\index{relation} is called an {\itshape equivalence
relation\index{equivalence relations, equivalence classes!equivalence relation
defined}\index{equivalence relations, equivalence classes|textbf}\/} if it is
reflexive, symmetric\index{symmetric relation|textbf}, and
transitive\index{transitive relation|textbf}.  The ``smallest" equivalence
relations are subsets\index{subset} of $[=]$; the ``largest" equivalence
relations are of the form $X \times X$.  Each equivalence relation $R$ is
related to a ``partition'' of $\dom(R)$ into disjoint\index{disjoint} sets:

\begin{definition}
 A set $P$ of non-empty subsets of a set $X$ is called a
 {\upshape partition\index{partition|textbf}\/} of $X$ if for any sets $A, B$
 in $P$, $A$ and $B$ are disjoint ($P$ is a pairwise disjoint collection of
 non-empty sets) and $\bigcup[P] = X$ (the collection $P$ ``covers" all of
 $X$).
\end{definition}

\begin{definition}
 Let $X$ be a set.  An equivalence relation $R$ is called an
 {\upshape equivalence relation on $X$} exactly if $\dom(R) = X$.
\end{definition}

\begin{thm}
 Let $X$ be a set, and let $R$ be an equivalence relation on $X$.
 Let $[x] = \{y \st y \rR x\} = \dom(R \cap (V \times \{x\})$ for
 each $x$ in  $X$.  Then the collection $X/R$ of all sets $[x]$ exists and is a
 partition of $X$.
 Moreover, for any partition $P$ of $X$, the relation $[\sim_P] = \{(x,y) \st$
 for some $A \in P,$ $ x \in A$ and $y \in A\}$ exists, is an equivalence
 relation on $X$, and\linebreak
 $X/[\sim_P] = P$.  Each equivalence relation uniquely
 determines the associated partition and vice versa.
\end{thm}

\preuve\ The collection $X/R$ = $\{A \st$ for some $x$, for all
$y,$ $ y \in A$ iff $y \rR x\}$ exists by the Stratified Comprehension
Theorem\index{Stratified Comprehension Theorem}.  It is a
partition\index{partition} of $X$: each $x$ belongs to $[x]$, since $x
\rR x$; any $y$ in $[x]$ must belong to $X$, since $y \rR x$ and $X =
\dom(R)$; thus the union of all the $[x]$'s covers all of $X$ and no
more than $X$; if $[x]$ and $[y]$ meet, they are equal, because if $z$
is in both, we have $z \rR x$ and $z \rR y$, thus, by symmetry and
transitivity, $x \rR y$ and $y \rR x$, thus any $w \rR x$ or $w \rR y$ implies
$w \rR y$ or $w \rR x$, respectively, and $[x]$ and $[y]$ are the same
set; thus two sets $[x]$ and $[y]$ are disjoint\index{disjoint} if
they are different.  The relation $[\sim_P]$ obviously exists, by
Stratified Comprehension\index{Stratified Comprehension Theorem}, and
is easily seen to be an equivalence relation.  The one-to-one
correspondence between equivalence relations and partitions is easy to
verify.
\finpreuve

The classes $[x]$ are called ``equivalence classes\index{equivalence relations,
equivalence classes!equivalence class defined}\index{equivalence relations,
equivalence classes|textbf}"; we suggest 
the notation $[x]_R$ for them if the dependence on $R$ must be made
explicit.  An equivalence relation on a set $X$ is often used to
indicate that certain objects in $X$ are to be identified for some
purpose; if $x \rR y$, the corresponding objects $[x]$ and $[y]$ in $X/R$ are
actually equal, so a ``version" of $X$ can be constructed in which the
desired identifications are actually realized.

A caution: in a stratified\index{stratification!and equivalence classes} set
definition, $[x]$ is not at the same ``type\index{types (relative)}" as $x$.
As a result, the relation between $x$ and $[x]$ does not necessarily exist.
However, the relation between $\{x\}$ and $[x]$ is definable in
a stratified way.  The use of equivalence classes is slightly more
difficult technically for this reason than in the usual set
theory\index{Zermelo--Fraenkel set theory};
the compensation for this is that many concepts can be modelled in a
natural way using equivalence classes which cannot be modelled in this
way in the usual set theory!

A solution to this difficulty is to choose an element from each
equivalence class $[x]$ as a ``representative" of that equivalence
class.  The advantage of this is that the relation between $x$ and the
representative element of $[x]$ is stratified\index{stratification}; the
representative element of $[x]$ is playing the same ``role" as $x$ in this
construction.  Although the construction of a set by such choices\index{axiom
of choice} seems intuitively reasonable, it is not enabled by any of our axioms
so far, so we introduce the dreaded

\begin{axiom}{Axiom of Choice\index{axiom of choice!introduced}\index{axiom of
choice|textbf}}
 For each set $P$ of pairwise disjoint\index{disjoint} non-empty
 sets, there is a set $C$, called a {\upshape choice set\/} from $P$, which
 contains exactly one element of each element of $P$.
\end{axiom}

On this axiom, there will be a great deal more later.  For the
moment, we observe that the Axiom of Choice allows us to ``collapse"
equivalence classes\index{equivalence relations, equivalence classes} to single
elements without introducing a stratification\index{stratification} problem.




\section{Order}

``Orders" in a general sense are an important kind of relation\index{relation!kinds of}.  A relation $R$ is called {\itshape
antisymmetric\index{antisymmetric relation|textbf}\/} if for all $x,y$, ($x 
\rR y$ and $y \rR x$) implies $x = y$; equivalently, $R$ is antisymmetric iff
$R \cap R^{-1} \subseteq [=]$.

\begin{definition}
 A relation $R$ is called a {\upshape (partial) order\index{order
 (partial)|textbf}\/} if it is reflexive\index{reflexive relation},
 antisymmetric and transitive\index{transitive relation}.

 It is called a
 {\upshape partial order on $X$} if $\dom(R) = X$.
\end{definition}

The most natural example of an equivalence relation\index{equivalence
relations, equivalence classes} is 
equality itself; the most natural example of a partial order is
inclusion\index{inclusion}.  Another interesting partial order is the extension
relation\index{relation} on functions\index{function} (introduced below; it is
the partial order on functions induced by inclusion).

Where $R$ is a partial order and $X = \dom(R)$ (and thus 
$X = \rng(R)$ as well), one often sees mention of $R$ suppressed and $X$
termed a ``partially ordered set" with $R$ understood.  We will
attempt to avoid this; we will refer to $X$ as the domain\index{domain} of the
partial order $R$, which itself is sufficient to determine $X$.
Similarly, we consider the mathematical structure of the natural\index{natural
number} numbers (re Peano arithmetic) to be the ordered pair (addition,
multiplication); all other elements usually taken to be parts of this
structure (such as the set ${\cal N}$ itself) can be derived from these
by set-theoretical operations.

If one wonders why general orders are called ``partial", one
needs only to attend to the following 

\begin{definition}
 A {\upshape linear\index{order (linear)|textbf}} or {\upshape total} order
 is an order $R$ such that for each $x,y$ in $\dom(R)$, either $x \rR y$ or $y
 \rR x$.
\end{definition}

The most natural example of a linear order is the usual order
relation ``less than or equal to" on the real numbers\index{real numbers}.  By
analogy with this example, orders in many contexts (including partial orders)
are written $\leq$.  When $\leq$ is a general order, we say $x < y$ when $x
\leq y$ and $x \neq y$.  If $\leq$ is an order, its converse\index{converse (of
a relation)} is also an order, which we write $\geq$, and $>$ is defined
similarly.  The relations $<$ and $>$ are called ``strict orders\index{order
(strict)|textbf}''.  Words like ``smaller", ``larger", ``less",
``greater", will be used in conformity with the analogy with the usual
order from arithmetic.  An element $s$ of the domain\index{domain} of $\leq$ is
said to be ``least" if $s \leq x$ for all $x$ in the domain; $s$ is said to be
``minimal" if there is no $x$ such that $x < s$.  These notions coincide for
linear\index{order (linear)}
but not for partial orders\index{order (partial)}. The notions of ``greatest"
and ``maximal" are defined similarly.

Suppose that $A$ is a subset\index{subset} of the domain of $\leq$.  Then any
$b$ such that $b \leq a$ for all $a$ in $A$ is called a {\itshape lower
bound\index{bound!lower|textbf}\/} of A; if there is a greatest among the lower
bounds of $A$, it is called the greatest lower bound\index{bound!greatest
lower|textbf} or glb of $A$ relative to $\leq$ and written inf$_{\leq}[A]$
($\leq$ may be omitted where understood).  The notions of {\itshape upper
bound\index{bound!lower|textbf}\/}
and {\itshape least upper bound\index{bound!least upper|textbf}\/} (lub;
written sup$_{\leq}[A]$) are defined analogously.
A partial order relative to which all subsets of the domain\index{domain} have
glb's and lub's is called {\itshape complete\/}.

If $\leq$ is a partial order, $X$ is its domain and $x$ is an
element of $X$, the segment\index{segment|textbf} determined by $x$, written
$\seg_{\leq}\{x\}$, or $\seg\{x\}$ if $\leq$ is understood, is the set
$\{y \in X \st y < x\}$.  Note that the strict order is used in this
definition, and that the type\index{types (relative)} of $\seg\{x\}$ is
one higher than the type of $x$.  If the non-strict order is used
instead, we get the ``weak segment\index{segment!weak|textbf}"; we use
$\seg^+\{x\}$ to denote the weak segment for $x$.  The partial order $\leq$ on
$X$ is precisely parallel to the inclusion\index{inclusion} relation on the set
of weak segments relative to $\leq$, for any partial order $\leq$, with a type
differential; this reduces all partial orders to the canonical one of
inclusion, in some sense.

Observe that there is a ``set of all partial orders" in this theory; this will be true of many natural categories of
mathematical structures which would be ``too large" to collect in
orthodox set theory.




\Exercises


\begin{enumerate}

\item  Prove or give a counterexample:  a transitive\index{transitive
  relation}, symmetric\index{symmetric relation} relation\index{relation} must
  be reflexive\index{reflexive relation}.  (The outcome here is affected by a
  difference between my definition of reflexivity and the usual one).

\item Give an example of a collection of sets $\cal S$ such that the
  restriction of inclusion\index{inclusion} to $\cal S$ is a linear order.

\item  Verify our assertion that the definition of the set $X/R$ is
  stratified\index{stratification}.

\item  Let $\leq$ be a partial order.  Does the Stratified Comprehension
  Theorem\index{Stratified Comprehension Theorem} allow us to assert the
  existence of a set $\{(x,\seg^+_{\leq}\{x\})\st x \in \dom(\leq)\}$
  witnessing the parallelism in structure between a partial order and the
  inclusion relation on its weak segments\index{segment!weak}?  If not, how
  could this set definition be modified so that it would work?

\end{enumerate}
