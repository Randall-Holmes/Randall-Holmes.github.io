\chapter[Surreal Numbers]{Generalized Dedekind Cuts and\\ Surreal Numbers}
\index{Dedekind cut}


We develop the {\itshape surreal numbers\index{surreal numbers}\/} of Conway.
These numbers are defined by Conway using induction on the
membership\index{membership} relation, which is impossible in our set
theory, but the (von Neumann\index{von Neumann ordinal}) ordinals\index{ordinal
numbers} are also 
defined in this way in Zermelo--Fraenkel set
theory\index{Zermelo--Fraenkel set theory}, and we succeeded in
defining the ordinals using the theory of
well-orderings\index{well-orderings}, which is a more natural way to
define the ordinals in any case.  The surreal numbers will be defined
using concepts from the theory of linear orderings\index{order
(linear)} in a way analogous to the way in which the ordinals were
defined.

This chapter is really only a sketch, not a full development of the
surreal numbers.  For the reader who wants to get the full picture, we
recommend a reading of the zeroth part of John Conway's {\itshape On
Numbers and Games}, followed by a re-reading of this chapter.

The intuitive idea behind the surreal numbers is a
generalization of the concept of ``Dedekind cut\index{Dedekind cut}" originally
used to define the real\index{real numbers} numbers.

\begin{definition}
 Let $\leq$ be a linear order.  Then a {\upshape cut\index{cut
 (generalized)|textbf}}
 $(L,R)$ in $\leq$ is a partition\index{partition} of $\dom(\leq)$ into two
 sets $L$ and $R$ such that for each $x \in L$ 
 and $y \in R$, $x < y$.  Let $\leq^*$ be a linear order which extends $\leq$;
 an element $c$ of $\dom(\leq^*)$ is said to {\upshape realize} a cut $C$ of
 $\leq$ if $\{x \in \dom(\leq) \st x <^* c\} = L$ and $\{x \in \dom(\leq) \st x
 >^* c\} =  R$.  Note that $c$ cannot be in $\dom(\leq)$ if it realizes a cut
 in $\leq$.
\end{definition}

The process of filling in linear orders\index{order (linear)} by introducing
objects which realize ``cuts" can be iterated.  We formalize this idea in the
following definition:

\begin{definition}
 An {\upshape iterated cut system\index{iterated cut system}} is defined by a
 linear 
 order $\leq$ and a transitive\index{transitive relation}
 relation\index{relation} $S$, ($x \rS y$ is read ``$x$ is simpler than $y$",
 or  ``$x$ comes before $y$"), with the following properties:
 \begin{enumerate}
  \item In each nonempty subset\index{subset} $A$ of $\dom(\leq) = \dom(S)$,
    there is an element $a$ such that for no $b \in A$ is $b \rS A$ true (we
    call this a {\upshape simplest element} of $A$).

  \item Let $x$ be an element of $\dom(\leq)$; let
    $\mathrm{ancestors}\{x\}$ be the set $\{y \in \dom(\leq) \st y S x\}$ (the
    set of
    objects simpler than $x$).  $x$ obviously realizes a cut in the restriction
    of $\leq$ to $\mathrm{ancestors}\{x\}$; we require further that each cut in
    the
    restriction of $\leq$ to $\mathrm{ancestors}\{x\}$ is realized.
\end{enumerate}
\end{definition}

\begin{definitions}
 We define $\mathrm{generation}\{x\}$ as the collection of objects $c$
 such that $c$ is the simplest object realizing some cut $C$ in
 $\mathrm{ancestors}\{x\}$.  It is obvious that $x$ itself belongs to
 $\mathrm{generation}\{x\}$.

 We define $\mathrm{birthday}(x)$ as the largest object in
 $\mathrm{generation}\{x\}$, the simplest object which realizes the cut
 $(\mathrm{ancestors}\{x\}$,$\vide)$.  

 We define\/ $\On$ as the set of {\upshape birthdays}.  
\end{definitions}

\begin{ThmEtc}{Obervations.}
 Observe that for each $x,y$ in On, $x \mathrel{S} y$ exactly if $x < y$, so a
 simplest
 object in a subset\index{subset} of $\On$ is the least object in the subset;
 $\On$ is 
 strictly well-ordered\index{well-orderings} by $\leq$.  

 Observe further that for any object $x$ and
 element $a$ of $\On$ which is simpler than $x$, there is a natural
 ``approximation" $x_a$ in generation$\{a\}$, namely, the simplest
 object in generation$\{a\}$ realizing the cut in ances\-tors$\{a\}$
 between objects less than $x$ and objects greater than $x$.
\end{ThmEtc}

Let's look at the structure of an iterated cut system.  There
is at least one simplest element in $\dom(\leq)$ (if it is nonempty);
call it 0.  0 must be defined by a cut in ancestors$\{0\}$, which is
the empty set\index{empty set}; there is only one such cut,
($\vide$,$\vide$). The next generation (made up of the simplest objects
different from 0) comprises two elements, which may be called $-1$ and
1, determined by the two possible cuts ($\vide$,$\{0\}$) and
($\{0\}$,$\vide$) of generation$\{0\}$.

We expand the development of the previous paragraph into a detailed
example of an iterated cut system: the elements of its domain are the
``dyadic rationals\index{rational numbers!dyadic}'' $\displaystyle\pm\frac
a{2^b}$, for
natural numbers $a$ and $b$.  The generations are indexed by the natural
numbers.  Each generation is finite\index{finite!set} and is constructed from
the finite union of all
the preceding generations in the following way: the cut at the top end
is realized by adding one to the largest element of the previous
generation; the cut at the lower end is realized by subtracting one
from the smallest element of the previous generation.  Each cut
between two successive elements of the union of all the preceding
generations is realized by their arithmetic mean.

We exhibit the first few stages in this process:
\begin{quote}
 0:  0

 1:  $-1, (0), 1$

 2:  $\displaystyle -2, (-1), -\frac12, (0), \frac12, (1), 2$

 3:  $\displaystyle -3, (-2), -\frac32, (-1), -\frac34, (-\frac12), -\frac14,
 (0), \frac14,
 (\frac12), -\frac34, (1), \frac32, (2), 3$
\end{quote}

In each list, elements of previous generations are enclosed in parentheses.

The order $\leq$ on
this system is the usual order relation.  The simplicity relation is
easily defined if we know how to compute the generation in which a
particular dyadic rational\index{rational numbers!dyadic} appears: the dyadic
rational whose simplest form is $\displaystyle \pm\frac a{2^b}$ appears in the
same
generation as the natural number $\displaystyle\Bigl\lceil\frac a{2^b}
\Bigr\rceil +b$.

If we add one more generation to this iterated cut system, we add all
the real\index{real numbers} numbers (in the obvious way preserving the
correspondence between our relation $\leq$ and the relation of the same name on
the reals) but we also need to add new objects $\pm\infty$ as the largest
and smallest elements of the final generation and objects $r^+$ and
$r^-$ realizing the cuts immediately to the left and right of each
dyadic rational $r$.

The structure of iterated cut systems is rigidly determined by the
number of generations:

\begin{thm}
 For any two iterated cut systems $(\leq,S)$ and
 $(\leq^*,S^*)$, the order type\index{order type} of whose generations is the
 same, there is a uniquely determined bijection $f$ from $\dom(\leq)$ to
 $\dom(\leq^*)$ such that $f(x) \leq^* f(y)$ iff $x \leq y$ and $f(x)
 \rS^* f(y)$ iff $x \rS y$.  It follows from this that the structure of
 iterated cut systems is entirely determined by the order type of
 $\leq$ restricted to $\On$.
\end{thm}

\preuve\ Let $(\leq,S)$ and $(\leq^*,S^*)$ be two iterated cut
systems as specified in the conditions of the theorem.  We define a
map $f$ by transfinite\index{transfinite!recursion} recursion on the
well-ordering\index{well-orderings} of generations, 
as follows: when $f$ from dom$(\leq)$ to $\dom(\leq^*)$ is defined for
members of each previous generation, we match each element $x$ of the
current generation over $\leq$ with the element of the corresponding
generation over $\leq^*$ lying between (in the sense of $\leq^*$) the
image under $f$ of the set of elements of ancestors$\{x\}$ less than
$x$ and the image under $f$ of the set of elements of ancestors$\{x\}$
greater than $x$.  This definition succeeds in defining a unique
$f(x)$ for each $x$ in dom$(\leq)$ such that the corresponding
generation exists for $\leq^*$.
\finpreuve

Note that iterated cut systems with different order types\index{order type} on
their generations can still be compared using this result; the system with
more generations has a subsystem consisting of the union\index{union!set} of
the initial segment\index{segment} of its generations similar\index{similarity}
to the generations of the other system which is seen by this result to be
isomorphic with the system with fewer generations.

This motivates the following

\begin{definition}
 A {\upshape surreal number\index{surreal numbers|textbf}} is an equivalence
 class\index{equivalence relations, equivalence classes} of triples 
 $(\leq,S,C)$, where $(\leq,S)$ is an iterated cut system and $C$ is a cut on
 $\leq$, 
 under the equivalence relation of corresponding to one another via the
 map defined above between the domains\index{domain} of iterated cut systems.
 The set of surreal numbers is called $\No$.
\end{definition}

\begin{definition}
 For surreal numbers $r$ and $s$, we define $r \rS s$ and $r \leq s$ as
 follows: let $(\leq',S',C')$ be an element of $r$ and let $(\leq'',S'',C'')$
 be an element of $s$.  Without loss of generality, we may suppose that each
 of the two iterated cut systems is included in the same iterated cut
 system, which we will call $(\leq^*,S^*)$.  We assert that $r \rS s$ if there
 are more generations in each element of $s$ than there are in each
 element of $r$ in the obvious sense.  Suppose that one of $r$ and $s$ is
 simpler than the other; without loss of generality, let it be $r$.  Then
 $C'$ is realized by a simplest element with respect to $\leq^*$, which we
 will call $r^*$.  We have $r^* \in \pi_1(C'')$, in which case we say $r < s$,
 or $r^* \in 
 \pi_2(C'')$, in which case we say $s < r$.  If neither $r$ nor $s$ is simpler
 than the other, then $r < s$ iff the intersection\index{intersection!Boolean}
 of $\pi_2(C')$ and 
 $\pi_1(C'')$ is nonempty, and similarly for $s < r$.  The only
 remaining case is that in which $r=s$.  We have now defined
 relations\index{relation} 
 $S$ and $\leq$ ($r \leq s$ iff $r < s$ in the sense defined above or
 $r = s$) on the surreal numbers.
\end{definition}

\begin{thm}
 $(\leq,S)$ is an iterated cut system with domain\index{domain} the set $\No$
 of surreal numbers.
\end{thm}

\preuve\  A surreal number, which is an equivalence class\index{equivalence
relations, equivalence classes} of relations\index{relation} 
on elements of domains of iterated cut systems, is two types\index{types
(relative)} higher than the latter elements; there is a natural map, respecting
$\leq$ and $S$, 
from No into the domain\index{domain} of the generations indexed by
the elements of T$^2[Ord]$ in 
a large enough iterated cut system.  All objects of No have
corresponding objects in an iterated cut system of ``rank"
$\lim[$T$^2[Ord]= \Omega$ (to verify the evaluation of this limit, see the next chapter), and the local relations\index{relation} of order\index{order
(linear)} and simplicity will parallel those of the surreal numbers exactly.

\begin{ThmEtc}{Observation.}
 Conway constructed the surreal numbers in the usual\index{Zermelo--Fraenkel
 set theory} set theory using a recursion on membership\index{membership};
 $(L,R)$ being the simplest 
 surreal number between $L$ and $R$ whenever $L$ and $R$ were sets of surreal
 numbers.  This clearly will not work in our theory, since it is an
 unstratified\index{stratification} construction.  It also has the consequence
 that the 
 ``class" No of surreal numbers is not a set; otherwise, consider the
 ``surreal number" (No,$\vide$), seen to be greater than all surreal numbers!
 In our set theory, we {\itshape can\/} define an analogue to Conway's basic
 operation: Let $(L,R)$ be taken to be the surreal number containing
 $(\leq,S,(L,R))$, where $\leq$ and $S$ are the relations\index{relation} on
 the iterated cut 
 system on No itself.  But $(L,R)$ is not the simplest surreal number
 between the sets $L$ and $R$; it is the simplest surreal number between
 the images of $L$ and $R$ under the $T^2$ operation on surreal numbers (the
 type-raising\index{type-raising operation} operation induced by taking double
 singleton\index{singleton image} images of 
 iterated cut systems).  Just as $\Omega$ does not sit ``past the end"
 of the ordinals\index{ordinal numbers}, but is smaller than the largest
 ordinals, so $(\No,\vide)$
 is indeed a surreal number ---\,it is the upper limit of the set of
 surreal numbers which are of the form $T^2\{r\}$.  Larger
 or more complex surreal numbers cannot be expressed in the $(L,R)$ form, but
 for any surreal $r$, we {\itshape can\/} express $T^2\{r\}$ in this form.
\end{ThmEtc}

We close by defining arithmetic operations on $\No$.

\begin{definition}
 If $r$ is a surreal number\index{surreal numbers}, we let $L[r]$ be the set of
 surreal numbers less than $r$ and simpler than $r$, and $R[r]$ be the
 set of surreal numbers greater than $r$ and simpler than $r$.

 We define $r + s$ as the simplest number greater than all elements of
 $\{r\} + L[s]$ and $L[r] + \{s\}$ and less than all elements of
 $\{r\} + R[s]$ and $R[r] + \{s\}$.

 We define $-r$ as the simplest
 number between $-R[r]$ below and $-L[r]$ above (and $r-s$ as $r +
 (-s)$).

 We define $rs$ as the simplest number between the sets
 $L[r]\{s\} + \{r\}L[s] - L[r]L[s]$ and $R[r]\{s\} + \{r\}R[s] -
 R[r]R[s]$ below and the sets $L[r]\{s\} + \{r\}R[s] - L[r]R[s]$ and
 $R[r]\{s\} + \{r\}L[s] - R[r]L[s]$ above.
\end{definition}

\begin{ThmEtc}{Observations.}
 Sums and products of sets are interpreted as sets of all
 possible sums or products of a term or factor from one and a term or
 factor from the other.  These are definitions by
 transfinite\index{transfinite!recursion} 
 recursion; observe that degrees of simplicity are indexed by
 ordinals\index{ordinal numbers}, or develop a recursion theorem for
 definitions of functions\index{function} on the surreals.  Conway
 shows in his book {\itshape On Numbers and Games\/} that $\No$ makes up a
 field with these operations; a suitable subset\index{subset} of the
 surreals with birthday on or before $\omega$ can be regarded as the
 real numbers (it turns out that those with finite
 birthdays\index{finite!birthday} are the 
 rational\index{rational numbers!dyadic} numbers with denominator a power of
 two, as in our example). 
 No has a far more interesting elementary arithmetic than the other
 systems of transfinite\index{transfinite!numbers} numbers.  As Conway points
 out, it is very interesting that we can get the real numbers\index{real
 numbers} ``all at once" without going through the usual intermediate systems
 (a restriction on cuts 
 used in iterated cut systems which gives the reals exactly: ``in any
 cut $(L,R)$, $L$ is nonempty with no greatest element exactly if $R$
 is nonempty with no least element").  The definitions of arithmetic
 operations can be shown to be based on the principle of ``choosing the
 simplest answer compatible with the axioms of an ordered field".
\end{ThmEtc}



\Exercise

Read the zeroth part of Conway's book {\itshape On Numbers and
Games\/}.  (There really is a zeroth part!) 
