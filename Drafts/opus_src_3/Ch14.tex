\chapter[The Axiom of Choice]{Equivalents of the Axiom of Choice}\index{axiom
of choice}


We now turn to an investigation of various forms of the Axiom
of Choice and their consequences.  We do not consider the Axiom of
Choice to be a problematic mathematical assumption.  It is rather like
the axiom of parallels in Euclidean geometry; we have the ``choice" of
systems with and without the Axiom of Choice, which can be shown to be
equally ``safe".  The one curious thing about the Axiom of Choice\index{axiom
of choice} in the context of this particular kind of set theory with Stratified
Comprehension\index{Stratified Comprehension Theorem} is that (as we will see
in a later chapter) we can use the Axiom of Choice to prove the existence of
atoms\index{atoms}; we do not see any way to prove the existence of atoms
without the Axiom of Choice, but it remains an open problem whether this can be
done. 

We restate the Axiom and an equivalent of the Axiom which we
have already established above:

\begin{axiom}{Axiom of Choice}
 For each set $P$ of pairwise disjoint\index{disjoint} non-empty
 sets, there is a set $C$ of elements of elements of $P$ which contains
 exactly one element of each element of $P$.
\end{axiom}

\begin{thm}
 The Cartesian\index{Cartesian product} product of any family of non-empty sets
 is non-empty.
\end{thm}

If $A$ is a family of sets, we call an element of $\prod[A]$ (a
function\index{function} which takes each element $a$ of $A$ to a
singleton\index{singleton} $\{c\} \subseteq a$) a
{\itshape choice function\/} for $A$.  We cannot in general have a function
taking each $a$ to 
the corresponding $c$ due to stratification\index{stratification} restrictions.
We consider the special case of the nonempty subsets\index{subset} of some set
$X$; an element of $\prod[{\cal P}\{X\} - \{\vide\}]$, the
Cartesian\index{Cartesian product} product of the collection of nonempty 
subsets of $X$, is called a choice function\index{function!choice|textbf} for
the set $X$ -- the choice function sends each nonempty subset of $X$ to the
singleton\index{singleton} of one of its elements.

Our discussion of equivalents for the Axiom of
Choice\index{axiom of choice} involves the introduction of a new kind
of order:

\begin{definition}
 A linear order\index{order (linear)} $\leq$ is a {\upshape
 well-ordering\index{well-orderings!definition
 of}\index{well-orderings|textbf}} if every nonempty subset\index{subset} of
 $\dom(\leq)$ has a least element relative to $\leq$.
 If $X = \dom(\leq)$, $\leq$ is said to be a {\upshape well-ordering of
 $X$}.  If $\leq$ is a linear order which is a well-ordering, the
 corresponding strict linear order\index{order (strict)} $<$ is said to be a
 {\upshape strict well-ordering}.
\end{definition}

Note that the usual order relation on the natural\index{natural number} numbers
is a well-ordering, but that the usual order on the integers is not a
well-ordering (the whole set of integers does not have a least
element) so the usual order of the reals\index{real numbers}, for example, is
not a well-ordering.

We now introduce and prove another theorem equivalent to the
Axiom of Choice\index{axiom of choice}, with a preliminary definition:

\begin{definition}
 If $\leq$ is a partial order\index{order (partial)}, a {\upshape
 chain\index{chain}\/} in $\leq$ is the domain\index{domain} of
 a subset\index{subset} of $\leq$ which is a linear order.
\end{definition}

\begin{Thm}{Zorn's Lemma\index{Zorn's Lemma}}
 If $\leq$ is a partial order such that each chain in $\leq$ has
 an upper bound\index{bound!upper} relative to $\leq$, then the domain of
 $\leq$ has a maximal element relative to $\leq$.
\end{Thm}

\preuve\ We first motivate the proof.  The Axiom of Choice\index{axiom of
choice} allows us to make
many simultaneous, independent choices.  What is needed in the proof
of Zorn's Lemma (or, for example, for a direct proof of the result
that the universe is well-ordered given below), is a technique for
making a long sequence\index{sequence!transfinite} of choices each of which
depends on choices 
made previously.  The way we do this is to consider the set of all
situations in which we might find ourselves in the course of such a
sequence of choices, and all possible ``next choices'' in each such
situation.  We make independent choices of ``next choices'' in all
possible situations, and from this we are able to construct a sequence
of choices depending on choices made earlier.

For each chain\index{chain} $c$ in $\leq$, we define a set $X_c$ as follows: if
there is any upper bound\index{bound!upper} $x$ of $c$ in $\leq$ such that $x
\not\in c$, then $X_c$ is the collection of sets $c \cup \{x\}$ for all such
upper bounds $x$.  If there is no such upper bound, then $X_c = \{c\}$.
Note that in either case, each element of $X_c$ is a chain, and, in
the latter case, any upper bound for $c$ (there is only one) belongs
to $c$ and is a maximal element for $\leq$.

The collection of sets $\{c\} \times X_c$ for $c$ a chain in $\leq$ is
a set of pairwise disjoint sets.  A choice set for this
collection will be a function\index{function} taking each chain $c$ to a
selected element of $X_c$.  Choose such a function and call it $G$.

We call a chain\index{chain!authentic} $c$ in $\leq$ {\itshape authentic\/} if
it is well-ordered by $\leq$ and has the property that for each $x \in c$,
$G(\{y \in c \st y < x\}) = \{y \in c \st y \leq x\}$.  (We define $x<y$ as
``$x \leq y$ and $x \neq y$'' in accordance with our stated convention).
So $x = \max\, G(\{x \in c \st y < x\})$ (we use the notation $\max \,A$
here for the largest element of a set $A$ with respect to an
understood order).

The following should be obvious: any initial segment\index{segment} $\{y \in c
\st y < x\}$ or $\{y \in c \st y \leq x\}$ of an authentic
chain\index{chain!authentic}
is an authentic chain; the empty chain is authentic; if $c$ is an
authentic chain, $G(c)$ is an authentic chain.

We now need to prove a technical property.
Consider two authentic chains $c$ and $c'$ and let $a$ be an element of
$c$ such that $\{x \in c \st x < a\} = \{x \in c' \st x < a\}$.
Then $a \in c'$ or $c' = \{x \in c \st x < a\}$.
Indeed, suppose that $a \not\in c'$ and that $c' \neq \{x \in c \st x < a\}$.
So $c' \neq \{x \in c' \st x < a\}$ and there exists some $a' \in c'$
which is greater or equal to $a$.
As $c'$ is authentic, it is well-ordered and we can assume that $a'$ is the
minimum of $\{x \in c' \st x \geq a\}$.
This implies that $\{x \in c' \st x < a\} = \{x \in c' \st x < a'\}$ and also
that $\{x \in c \st x < a\} = \{x \in c' \st x < a'\}$.
But then,
\begin{displaymath}
 a = \max\, G(\{x \in c \st x < a\}) = \max\, G(\{x \in c' \st x < a'\}) = a'
 \in c'.
\end{displaymath}
So $a \in c'$, a contradiction.

Now, let $c_1$ and $c_2$ be distinct authentic chains\index{chain!authentic}.
We want to prove that one of these two chains is an initial
segment\index{segment} of the other one.
Without loss of generality, suppose that there is an element of $c_1$ which
is not in $c_2$;
let $a$ be the minimal element of $c_1$ not in $c_2$.
So $\{x \in c_1 \st x < a\} \subseteq \{x \in c_2 \st x < a\}$.

If $\{x \in c_1 \st x < a\} \neq \{x \in c_2 \st x < a\}$, then we can
find some element which is in $c_2$ but not in $c_1$ and which is
smaller than $a$.
Let $b$ be the minimum of $\{x \in c_2 \st x \not\in c_1 \mbox{ and }
x < a\}$.
As $b$ is minimal and $b < a$ and $\{x \in c_1 \st x < a\} \subseteq \{x
\in c_2 \st x < a\}$, it is easy to prove that
$\{x \in c_2 \st x < b\} = \{x \in c_1 \st x < b\}$.
By the technical property, $b\in c_1$ or $c_1 = \{x \in c_2 \st x < b\}$.
As, by definition, $b \not\in c_1$, we conclude that $c_1$ is an initial
segment\index{segment} of $c_2$.
But this contradicts the assumption that $a \in c_1 -
c_2$.

So $\{x \in c_1 \st x < a\} = \{x \in c_2 \st x < a\}$.
Then, by the technical property, $a\in c_2$ or
$c_2 = \{x \in c_1 \st x < a\}$.
As, by definition, $a \not\in c_2$, we conclude that $c_2$ is an initial
segment of $c_1$.

Thus, more generally, we have proved that authentic
chains\index{chain!authentic} are linearly
ordered by inclusion.

It is straightforward to show that the union\index{union!set} of a
collection of chains\index{chain} which is linearly ordered by
inclusion\index{inclusion} will be a chain.  It is then easy to see
that the union of any collection of authentic chains will be an
authentic chain\index{chain!authentic}.  In particular, the
union\index{union!set} of all authentic chains in $\leq$ is an authentic chain;
call it $C$.  $G(C)$ 
must also be an authentic chain, but this implies that $G(C) = C$
(since each can be seen to be a subset of the other), which further
implies that $C$ has no upper bound\index{bound!upper} not in $C$ (by
the definition of the function\index{function} $G$); $C$ has an upper
bound by the assumed properties of $\leq$, which must be a maximal
element for $\leq$.  The proof of Zorn's Lemma\index{Zorn's Lemma} is complete.
\finpreuve


\begin{thm}
 Zorn's Lemma implies the Axiom of Choice\index{axiom of choice}.
\end{thm}

\preuve Let $P$ be a pairwise disjoint\index{disjoint} collection of nonempty
sets.  Let $Q$ be the set of choice sets for subsets\index{subset} of $P$.  It
is easy to see that inclusion\index{inclusion} on $Q$ satisfies the conditions
of Zorn's Lemma (take the union of a chain\index{chain} of such sets with
respect to inclusion to get an upper bound\index{bound!upper}), and that a
maximal element with respect to this order must be a choice set for $P$.
\finpreuve





Zorn's Lemma\index{Zorn's Lemma} allows us to prove this attractive

\begin{thm}
 There is a well-ordering of $V$.
\end{thm}

\preuve Consider the collection of all well-orderings\index{well-orderings}
(this exists by Stratified Comprehension\index{Stratified Comprehension
Theorem}).  We define the following order on well-orderings: we say $R \leq S$
if $R \subseteq S$ and for each $r$ in dom$(R)$ and $s$ in dom$(S)
-$dom$(R)$, $r S s$; the well-ordering $S$ extends $R$ by adding
elements at the end only; we say that $S$ ``continues" $R$.  

Now let
$C$ be a chain of well-orderings with respect to continuation; we
claim that the union\index{union!set} of $C$ is also a
well-ordering\index{well-orderings}, and a continuation of each of the
elements of $C$.    Consider a nonempty subset $A$ of
the domain of the union of $C$.  Since it is nonempty, it has an
element $x$, which belongs to the domain of some $R \in C$.  Since $R$
is a well-ordering, $A\, \cap $ dom$ (R)$ has a minimal element $a$.  Any
element of the domain of the union of $C$ less than $a$ would belong
to dom$ (R)$, and so would not belong to $A$; thus $a$ is the least
element of $A$ with respect to the union of $C$, and the union of $C$
is a well-ordering.  That it is a continuation of each of the elements
of $C$ is easy to see.

We have shown that the collection of well-orderings\index{well-orderings}
ordered by continuation satisfies the conditions of Zorn's Lemma\index{Zorn's
Lemma}.  A maximal well-ordering must have domain\index{domain} $V$, or we
could extend it by making some single object greater than all the previously
provided objects.
\finpreuve

\begin{cor}
 For any set $X$, there is a well-ordering on $X$.
\end{cor}

\begin{thm}
 The existence of a well-ordering\index{well-orderings} on $V$ implies the
 Axiom of Choice\index{axiom of choice}.
\end{thm}

\preuve\ Let $P$ be a pairwise disjoint\index{disjoint} collection of nonempty
sets.
Let $\leq$ be a well-ordering of $V$.  Let $C$ be the set of least
elements with respect to $\leq$ of elements of $P$; this $C$ is
clearly a choice set for $P$.  Note that we can {\itshape define\/} choice
sets for any partition\index{partition} given a well-ordering of the
universe.\linebreak
\finpreuve

We have given four equivalent forms for the Axiom of Choice\index{axiom of
choice},
the original statement, the statement that Cartesian\index{Cartesian product}
products of collections of nonempty sets are nonempty, Zorn's
Lemma\index{Zorn's Lemma}, and the statement that there is a
well-ordering\index{well-orderings} of the universe\index{universe, universal
set}.

\Exercises

\begin{enumerate}
 \item  Prove that every finite\index{finite!set} collection of
   disjoint\index{disjoint} sets has a choice function\index{function!choice}
   without using the Axiom of Choice\index{axiom of choice}.

 \item  Use Zorn's Lemma to show that any infinite\index{infinite!set} set
   contains the range of some bijection\index{bijection} with
   domain\index{domain} $\cal N$.

 \item  Use Zorn's Lemma\index{Zorn's Lemma} to prove that every vector space
   has a basis.

 \item  Bertrand Russell used the following example in a philosophical
   discussion of the Axiom of Choice\index{axiom of choice}.  Suppose we have
   an infinite collection of pairs of shoes; it is easy to see that there is a
   choice function for this collection of pairs (explain).  But the case of an
   infinite collection of pairs of socks is more difficult (explain).  Use
   Zorn's Lemma to demonstrate that we can choose one sock from each of the
   pairs in the infinite collection.

 \item  Look up the results of Banach and Tarski on the decomposition of
   3-spheres using the Axiom of Choice\index{axiom of choice} in the library.
   What do you think of our assertion that the Axiom of Choice is not a
   problematic assumption?
\end{enumerate}
                                   
