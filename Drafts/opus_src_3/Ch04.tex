\chapter[Building Finite Structures]{Building Finite\index{finite!structures}
Structures}


So far, we have no assurance that there are any sets other
than $V$ and $\vide$.  All of the Boolean\index{Boolean algebra, operations}
axioms are satisfied if these are the only sets.  We provide a new axiom which
will enable us to build a wide range of finite structures:

\begin{axiom}{Axiom of
Singletons\index{singleton!axiom}\index{singleton|textbf}}
  For every object $x$, the set $\{x\}$ = $\{y \st y = x\}$
  exists, and is called the {\upshape singleton} of $x$.
\end{axiom}

\begin{definition}
We define $\iota(x)$ as $\{x\}$.  We define $\iota^0(x)$ as $x$ and $\iota^{n+1}(x)$ as $\iota(\iota^n(x))$.  The representation of iterated application of this operation is the point of this rarely used alternative notation.
\end{definition}

\begin{definition}
 If $x,y$ are objects, we define $\{x,y\}$, called the {\upshape unordered
 pair} of $x$ and $y$ as $\{x\} \cup \{y\}$;
 we define unordered triples $\{x,y,z\}$ as
 $\{x\} \cup \{y\} \cup \{z\}$ and all unordered $n$-tuples
 $\{x_1,...,x_n\}$ in the same
 way (as $\{x_1\} \cup \{x_2,\ldots,x_n\}$, an inductive definition depending
 on the previous definition of the $(n-1)$-tuple).
\end{definition}

Observe that the expression $\{x,x\}$ represents $\{x\}$, and,
similarly, a symbol like $\{x,x,y\}$ represents $\{x,y\}$; duplications in
this notation for finite\index{finite!sets, notation for} sets can be ignored.
Also, $\{x,y\}$ = $\{y,x\}$;
order of items in the name of a finite set makes no difference to its
meaning.

We can build more complicated structures in this way: for
example, we can construct the objects $\vide$, $\{\vide\}$, $\{\{\vide\}\}$,
etc, or such 
things as $\{\{\vide\},\{\{V\},\{\{V\}\}\}\}$.  In the
usual\index{Zermelo--Fraenkel set theory} set theory, the numbers 
are constructed in this way; `0' is defined as $\vide$, `1' is defined as
$\{$`0'$\}$, `2' is defined as $\{$`0'$,$`1'$\}$, `3' is defined as
$\{$`0'$,$`1'$,$`2'$\}$,
and so forth.  We use single quotes to distinguish these ``von Neumann
numerals\index{von Neumann numeral}" from the rather different objects which we
will identify as 
the natural numbers\index{natural number}.  The individual von Neumann numerals
exist in our system, but the set of all von Neumann numerals does not
necessarily exist (although it is possible for it to exist).  
        
One general kind of structure that can be constructed deserves
its own

\begin{definition}
 We define $\left<x,y\right>$, the {\upshape Kuratowski ordered
 pair\index{ordered pair!Kuratowski} of $x$ and $y$\/}
 as $\{\{x\},\{x,y\}\}$.
\end{definition}

It is straightforward to prove the following Theorem.  An indication of a proof is given.

\begin{thm}
 $\left<a,b\right> = \left<c,d\right>$ exactly if $a = c$ and $b = d$.  In
 particular, $\left<a,b\right> = \left<b,a\right>$ exactly if $a = b$ (note
 that $\left<a,a\right> = \{\{a\}\}$).
\end{thm}

\preuve  One direction of the equivalence is obvious.  We establish the other.
If $p=\left<a,b\right>=\{\{a\},\{a,b\}\}$, we can identify $a$ as the unique object which belongs to both elements of $p$,
and we can identify $b$ as the unique object which belongs to exactly one element of $p$.  It is important to notice that both of these statements remain true if $a=b$.
Now suppose that $\left<a,b\right> = \left<c,d\right>$:  it then follows that the unique object which belongs to both elements of this Kuratowski pair is $a$, and also that it is $c$, so $a=c$,
and that the unique object which belongs to exactly one element of this Kuratowski pair is $b$, and also that it is $d$, so $b=d$.  \finpreuve

It is possible to develop the theory of pairs, and thus of
functions and relations\index{relation}, using the Kuratowski pair, but it
turns out to be inconvenient to do so in this set theory.  In the
usual\index{Zermelo--Fraenkel set theory} set
theory, $\left<a,b\right>$ is used as the ordered pair; in a later
chapter, we will discuss the problems with use of the Kuratowski pair
in our set theory.

We will introduce the ordered pair as an independent primitive
notion, since use of the Kuratowski pair introduces technicalities
which we prefer to avoid.  We start this by means of an

\begin{axiom}{Axiom of Ordered Pairs\index{ordered pair!axiom}}
  For each $a, b$, the {\upshape ordered pair of $a$ and $b$},
  $(a,b)$, exists; $(a,b) = (c,d)$ exactly if $a = c$ and $b = d$.
\end{axiom}

Notice that we say nothing about whether $(a,b)$ is a set or
an atom (and thus nothing about what might be an element of an ordered
pair).  We will be able to prove some results about this later (it
will be a consequence of axioms not yet introduced that ``most" pairs
must be atoms\index{atoms}, a corollary of the fact, which we will prove, that
``most" objects are atoms; this makes an actual definition of our pair
in terms of set constructions impossible).  Pairing, like the
construction of singletons\index{singleton}, allows us to build new
finite\index{finite!structures} structures:

\begin{definition}
 The {\upshape $n$-tuple $(x_1,x_2,\ldots,x_n)$} is defined as
 $(x_1,(x_2,\ldots,x_n))$;
 this is an inductive definition, depending on the prior definition of
 the $(n-1)$-tuple.  For example, $(x_1,x_2,x_3) = (x_1,(x_2,x_3))$.  Note that
 any $n$-tuple $(x_1,\ldots,x_{n-1},x_n)$ is also an $(n-1)$-tuple,
 $(x_1,\ldots,(x_{n-1},x_n))$.
\end{definition}


We close this short Chapter by providing for the construction
of certain collections of ordered pairs\index{ordered pair}
\index{product, Cartesian|see{Cartesian product}}:

\begin{axiom}{Axiom of Cartesian Products\index{Cartesian
product!axiom}\index{Cartesian product|textbf}}
 For any sets $A,B$, the set
 $$
  A \times B = \{x \st
  \mbox{for some $a$ and $b$, $ a \in A, b \in B,$ and $x = (a,b)$}\},
 $$
 briefly written 
 $\{(a,b) \st a \in A$ and $b \in B\}$, called the {\upshape Cartesian
 product\/} of $A$ and $B$, exists.
\end{axiom}

\begin{definition}
  For any set $A$, we define $A^2$ as $A\times A$ and $A^{n+1}$ as $A \times
  (A^{n})$ for each natural number\index{natural number} $n>1$.  Grouping is
  important here because the Cartesian product\index{Cartesian power} is not
  associative.
\end{definition}

We will hereafter use without comment the kind of extension of
our set builder notation which uses finite\index{finite!structures} structures
in place of a simple variable to represent the typical element of a set,
exemplified in the notation of this axiom.  The uses of the specific sets
introduced by the axiom will become apparent in the next chapter.

\Exercises

\begin{enumerate}

\item  Prove the theorem about Kuratowski ordered pairs\index{ordered
  pair!Kuratowski}.

\item  Look up the alternative definition of the ordered pair\index{ordered
  pair!another alternative definition} in the Notes and prove that it satisfies
  the basic property of an ordered pair.

\item  Verify our observation that the Cartesian\index{Cartesian product}
  product is not associative.  Can it ever be true that $(A \times B)\times C$
  = $A \times(B \times C)$?

\item What interpretation might one place on the Cartesian
  powers\index{Cartesian power} $A^1$ and $A^0$?

\end{enumerate}
