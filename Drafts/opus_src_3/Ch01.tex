\chapter[Introduction:  Why Save the Universe?]{Introduction:\\
Why Save the Universe?}

\index{ZFC|see{Zermelo--Fraenkel set theory}}
\index{universe, universal set}
\index{usual set theory, the|see {Zermelo--Fraenkel set theory}}

This book is intended for one of two uses.  It could be used as an
introduction to set\index{set} theory.  It is roughly parallel in structure to
Halmos's classic {\itshape Naive Set Theory\/}, though more topics have
been added.  The book contains exercises in most chapters, in line
with its superficial character of being an elementary set theory text,
but no representation as to pedagogical soundness is made.  Some of
the exercises are very hard (these are marked).

The other possible use of the book is to demonstrate the naturalness
and effectiveness of an alternative set theory, Jensen's corrected
version NFU\index{NFU} of W. V. O. Quine's system ``New
Foundations'' (NF), so-called because it was proposed in Quine's paper
{\itshape New foundations for mathematical logic} in 1937.  No introduction to
set theory based on Quine's approach has appeared (to my knowledge) since
J. B. Rosser's {\itshape Logic for Mathematicians\/}, which came out in
1953 (second edition 1978).  As our title implies, NFU is a
set theory in which there is a universal set\index{universe, universal set}.

Quine's ``New Foundations\index{New Foundations}" has a bad reputation.  Its
consistency relative to ZFC\index{Zermelo--Fraenkel set theory}
remains an open question.
In 1953 (immediately after Rosser published his book {\itshape Logic for
Mathematicians\/}, which was based on ``New Foundations"), E.~Specker
proved that ``New Foundations" proves the negation of the Axiom of
Choice\index{axiom of choice}!  No one has been able to derive a contradiction
from NF, but the failure of AC makes it an unfriendly theory to work in.

None of this should be allowed to reflect on NFU.  
NFU is known to be consistent since the work of Jensen in 1969.  It can be
extended with the Axiom of Infinity\index{infinity} (this is implicit in our
Axiom of Projections\index{projections!axiom of}).  It is consistent with the
Axiom of Choice.  It can be extended with more powerful axioms of infinity in
essentially the same ways that ZFC\index{Zermelo--Fraenkel set
theory} can be extended; we introduce an Axiom of Small
Ordinals\index{Axiom of Small Ordinals} which results in a theory at least as
strong as ZFC\index{Zermelo--Fraenkel set theory}
(actually considerably stronger, as has been shown recently by Robert
Solovay).  We believe that if Jensen's result had been given before
Specker's, the subsequent history of interest in this kind of set\index{set}
theory might have been quite different.

These considerations are not enough to justify the use of NFU
instead of ZFC\index{Zermelo--Fraenkel set theory}.  What positive
advantages do we claim for this approach?  The reason that we believe that
NFU is a good
vehicle for learning set theory is that it allows most of the natural
constructions of genuinely ``naive" set theory, the set theory of
Frege with unlimited comprehension\index{comprehension} (it can be claimed that
Cantor's set theory always incorporated ``limitation of size'' in some form,
even before it was formalized).  The universe\index{universe, universal set} of
sets is a 
Boolean algebra\index{Boolean algebra, operations} (there is a universe; sets
have complements\index{complement}).  Finite\index{finite!cardinal number} 
and infinite cardinal numbers\index{cardinal numbers} can be defined as
equivalence\index{equivalence (of cardinality)} classes\index{equivalence
relations, equivalence classes}
under equipotence, following the original ideas of Cantor and Frege.
Ordinal numbers\index{ordinal numbers} can be defined as equivalence classes of
well-orderings\index{well-orderings} under similarity\index{similarity}.  The
objects which cause trouble in the paradoxes\index{paradoxes} of Cantor and
Burali-Forti (the cardinality of the universe and the order type\index{order
type} of the ordinals) actually exist in
NFU, but do not have quite the expected properties.  Many
interesting large classes are sets: the set of all groups,
the set of all topological spaces, etc.\mbox{} (most categories of interest,
actually).  The reason that the paradoxes are avoided is a restriction
on the axiom of comprehension\index{comprehension}.  These paradoxes, and the
paradox of Russell, are discussed in the text.

The restriction on the axiom of comprehension\index{comprehension} which
NFU shares with ``New Foundations" is not very easy to justify to a
naive audience, unless one starts with a presentation of the Theory of
Types\index{types (relative)}.  We do not introduce NFU in this way; we use a
finite axiomatization (Hailperin first showed that NF is finitely
axiomatizable) to introduce NFU in a way quite analogous to
the way that ZFC\index{Zermelo--Fraenkel set theory} is usually
introduced, as allowing the construction of sets and relations\index{relation}
using certain basic operations.
The basic operations used are quite intuitive.  Set-builder notation
is introduced early, and the impossibility of unrestricted
comprehension\index{comprehension} is pointed out in the usual way using
Russell's paradox\index{paradoxes!Russell}.
After the arsenal of basic constructions is assembled, the schema of
stratified\index{stratification} comprehension\index{comprehension} is proved
as a theorem.

Another point for those familiar with the usual treatment of
NF and NFU is that we use notation somewhat more
similar to the notation usually used in set theory, avoiding some of
the unusual notation going back to Rosser or Principia Mathematica
which has become traditional in NF and related theories.  Our
notation does have some eccentricities, which are discussed in the
section titled ``Parentheses, Braces and Brackets'' (p.~\pageref{sec-par}).

The usual set theory of Zermelo\index{Zermelo--Fraenkel set theory} and
Fraenkel 
is not entirely neglected; there is an introduction to the usual set
theory as an alternative, motivated in the context of NFU by a
study of the isomorphism types\index{isomorphism types} of well-founded extensional relations\index{relation!well-founded extensional relations}.

There is a study of somewhat more advanced topics in set
theory at the end, including the proof of Robert Solovay's theorem
that the existence of inaccessible\index{inaccessible cardinals}
cardinals\index{cardinal numbers} follows from our Axiom of Small
Ordinals\index{Axiom of Small Ordinals}.  There is also a discussion of the
analogous system of ``stratified\index{stratification} $\lambda$-calculus'' in
which the notion of {\itshape function\/} rather than the notion of {\itshape
set\index{set}\/} is taken as primitive.

All references are deferred to the section of Notes at the end
(Chapter~\ref{chap-notes}, p.~\pageref{chap-notes}).


\Exercise

Find and read Quine's original paper {\itshape New foundations for mathematical
logic}.
