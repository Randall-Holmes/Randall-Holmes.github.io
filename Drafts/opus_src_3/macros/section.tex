%
\makeatletter
\def\firstsection{\@startsection {section}{1}{\z@}{-0.0001pt}%
{1ex}{\fsection}}
\def\section{\@startsection {section}{1}{\z@}{-3.5ex \@plus -2pt \@minus 
 -.2ex}{1ex}{\fsection}}
\def\subsection{\@startsection{subsection}{2}{\z@}{-3.25ex \@plus -1pt \@minus 
 -.2ex}{1ex}{\fsubsection}}
\def\subsubsection{\@startsection{subsubsection}{3}{\z@}{-2ex \@plus 
 -1pt \@minus -.2ex}{.5ex}{\fsubsubsection}}
\def\paragraph{\@startsection{paragraph}{4}{\z@}{3.25ex plus1ex
minus.2ex}{-0.01pt}{\fparagraph}}
%
\let\ps@plain\ps@empty
%
\def\@seccntformat#1{\csname the#1\endcsname.\hskip 1em}
%
\def\@makechapterhead#1{%
  \vspace*{-\headsep}\vspace*{-\headheight}\vspace*{-\baselineskip}\par
  \noindent\mbox{%
\vbox to8cm{%
\hbox{\fcahiers Cahiers du Centre de logique}%
\hbox{\small\fcahiers Volume 10}%
\vfill
\hbox{\ftitre\kern\decalageG
\begin{tabular}{@{}l}
    \ifnum \c@secnumdepth >\m@ne
      \if@mainmatter
        \fchapitre \@chapapp{} \thechapter\\[1cm]
      \fi
    \fi
    #1
\end{tabular}
}\vfill}}}
%
% 10.06.97 Modifier la definition qui suit pour la faire ressembler a la
%          precedente.
%
%\def\@makeschapterhead#1{%
%  \vspace*{50\p@}%
%  {\parindent \z@ \raggedright
%    \reset@font
%    \interlinepenalty\@M
%    \Huge \bfseries  #1\par\nobreak
%    \vskip 40\p@
%  }}
\def\@makeschapterhead#1{%
  \vspace*{-\headsep}\vspace*{-\headheight}\vspace*{-\baselineskip}\par
  \noindent\mbox{%
\vbox to8cm{%
\hbox{\fcahiers Cahiers du Centre de logique}%
\hbox{\small\fcahiers Volume 10}%
\vfill
\hbox{\ftitre\kern\decalageG
\begin{tabular}{@{}l}
    #1
\end{tabular}
}\vfill}}\par}
%
\def\pourindex#1{%
%  \vspace*{-\headsep}\vspace*{-\headheight}\vspace*{-\baselineskip}\par
  \noindent\mbox{%
\vbox to8cm{%
\hbox{\fcahiers Cahiers du Centre de logique}%
\hbox{\small\fcahiers Volume 10}%
\vfill
\hbox{\ftitre\kern\decalageG
\begin{tabular}{@{}l}
    #1
\end{tabular}
}\vfill}}\par}
%
\renewenvironment{theindex}
               {\chapter*{\indexname}\markboth{\indexname}{\indexname}
	        \addcontentsline{toc}{chapter}{\string\numberline{}\indexname}
	        \parindent\z@
                \parskip\z@ \@plus .3\p@\relax
                \let\item\@idxitem
	        \begin{multicols}{2}}
               {\end{multicols}}
%
\renewcommand{\tableofcontents}{%
    \if@twocolumn
      \@restonecoltrue\onecolumn
    \else
      \@restonecolfalse
    \fi
    \chapter*{\contentsname}
        \markboth{\contentsname}{\contentsname}%
    \@starttoc{toc}%
    \if@restonecol\twocolumn\fi
    }
%
\renewenvironment{thebibliography}[1]
     {\chapter*{\bibname}
        \markboth{\bibname}{\bibname}%
        \addcontentsline{toc}{chapter}{\string\numberline{}\bibname}
      \list{\@biblabel{\arabic{enumiv}}}%
           {\settowidth\labelwidth{\@biblabel{#1}}%
            \leftmargin\labelwidth
            \advance\leftmargin\labelsep
            \if@openbib
              \advance\leftmargin\bibindent
              \itemindent -\bibindent
              \listparindent \itemindent
              \parsep \z@
            \fi
            \usecounter{enumiv}%
            \let\p@enumiv\@empty
            \renewcommand{\theenumiv}{\arabic{enumiv}}}%
      \if@openbib
        \renewcommand{\newblock}{\par}
      \else
        \renewcommand{\newblock}{\hskip .11em \@plus.33em \@minus.07em}%
      \fi
      \sloppy\clubpenalty4000\widowpenalty4000%
      \sfcode`\.=\@m}
     {\def\@noitemerr
       {\@latex@warning{Empty `thebibliography' environment}}%
      \endlist}
\makeatother
