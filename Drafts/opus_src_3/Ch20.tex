\chapter[Well-Founded Extensional Relations]{Well-Founded Extensional
Relations\\
and Conventional Set Theory}



\section[Well-Founded Extensional Relations]{Connected Well-Founded
Extensional\\
Relations} 

In this chapter we will develop the theory of ``connected well-founded
extensional relations\index{relation!well-founded extensional relations}'', and
use this to develop an alternative interpretation of the usual set theory
ZFC\index{Zermelo--Fraenkel set theory}.  Some of the
following definitions are repeated from above.

\begin{definition}
 The {\upshape full domain\index{domain!full, definition repeated}\/} of a
 relation\index{properties!of relations} $R$, denoted by $\Dom(R)$ is defined
 as $\dom(R) \cup \rng(R)$, the union of the domain and range of $R$.
\end{definition}

\begin{definition}
 Let $R$ be a relation and let $S$ be a subset\index{subset!open dense} of
 $\Dom(R)$.  An element $x$ of $S$ is said to be a {\upshape minimal
 element} of $S$ with respect to $R$ exactly when there is no $y$
 such that $y\rR x$.
\end{definition}

\begin{definition}
 A relation $R$ is said to be {\upshape well-founded}
 if each nonempty subset $S$ of $\Dom(R)$ has a minimal element with
 respect to $R$.
\end{definition}

\begin{definition}
 A relation $R$ is said to be {\upshape extensional}
 exactly when for all $x,y \in \Dom(R)$, $x = y$ if and only if for
 all $z$, $z\rR x$ iff $z\rR y$, i.e., $x=y$ if and only if
 $R^{-1}[\{x\}]= R^{-1}[\{y\}]$.
\end{definition}

\begin{definition}
 A well-founded relation $R$ is said to have $t$ as
 its {\upshape top} exactly when for each element $x$ of $\Dom(R)$, there
 is a finite\index{finite!sequence} sequence $x=x_0,\ldots,x_n=t$ with
 $x_i\rR x_{i+1}$ for each appropriate value of $i$.  A well-founded relation
 with a top is called a {\upshape connected} well-founded relation.
\end{definition}

Note that the empty relation\index{empty set} is a connected well-founded
extensional relation with any object $t$ whatsoever as its top (vacuously).

\begin{lemme}
 Let $R$ be a nonempty connected well-founded
 relation.  Then $R$ has a unique top.
\end{lemme}

\preuve\ Observe that there can be no cycles in $R$, i.e., there can be no
finite\index{finite!sequence} sequence $x=x_0,\ldots,x_n=x$ with $n \geq 1$ and
$x_i \rR x_{i+1}$ for each appropriate value of $i$.  For the set
$C$ of elements of $\dom(R)\cup \rng(R)$ belonging to cycles would
clearly have no minimal element.  By the definition of ``top'', the
existence of two distinct tops of $R$ would allow us to construct a
cycle in $R$.
\finpreuve

The motivation for these definitions lies in the fact that the
membership\index{membership} relation of the usual set
theory\index{Zermelo--Fraenkel set theory} is a well-founded
extensional relation\index{relation!well-founded extensional relations} (a
proper class, of course), and that the membership relation of the usual set
theory restricted to a set $A$, its elements, the elements of its elements, and
so forth, is a connected well-founded extensional
relation\index{relation!well-founded extensional relations}.  Sets in the usual
set theory are uniquely determined by the structure of the associated
connected well-founded extensional relation, which motivates our study
of connected well-founded extensional relations as representing the
objects of ZFC\index{Zermelo--Fraenkel set theory} in our working set theory.

\begin{definition}
 Relations $R$ and $S$ are said to be {\upshape
 isomorphic} exactly when there is a bijection $f$ from $\Dom(R)$
 onto $\Dom(S)$ such that $x\rR y$ iff $f(x) \mathrel{S} f(y)$.
\end{definition}

The relation of isomorphism is easily seen to be an equivalence
relation\index{equivalence relations, equivalence classes}.  It should also be
clear that a relation isomorphic to a relation having the properties of being
(connected) well-founded or extensional is likewise (connected) well-founded or
extensional\index{relation!well-founded extensional relations}.


\begin{definition}
 We define $Z$ as the set of equivalence classes\index{equivalence relations,
 equivalence classes} of connected well-founded extensional
 relations\index{relation!well-founded extensional relations} under
 isomorphism.  The   existence of $Z$ follows from
 stratified\index{stratification} comprehension\index{comprehension}.  The
 element of $Z$ to which a connected well-founded extensional relation
 belongs is called its {\upshape isomorphism type\index{isomorphism types}}.
\end{definition}


\begin{definition}
 Let $R$ be a well-founded relation and let $x$ be
 an element of $\Dom(R)$.  We define the {\upshape component} of $R$
 determined by $x$ as the maximal subrelation of $R$ which is a
 connected well-founded relation with $x$ as top, and denote it by
 $\mathrm{comp}_R\{x\}$.  We call a component of $R$ an {\upshape immediate
 component} if its top stands in the relation $R$ to the top of
 $R$.  Note that the component associated with an element $x$ of
 $\Dom(R)$ is at the relative type\index{types (relative)} of $R$, one type
 higher than the relative type of $x$.
\end{definition}

Notice that the empty relation\index{empty set}\index{relation!empty} will be
the component associated with a minimal element of $\Dom(R)$ relative to $R$.
The fact that any object is a top of the empty relation makes this work out
correctly.

The immediate components of the connected well-founded extensional
relation\index{relation!well-founded extensional relations} associated with
each set of the usual\index{Zermelo--Fraenkel set theory} set theory are the 
connected well-founded extensional relations associated with its
elements; thus this relation will be used to define the analogue of
membership\index{membership} in the interpretation we are constructing:

\begin{definition}
 We define a relation $E \subseteq Z \times Z$: $x
 \mathrel{E} y$ exactly when $x \in Z$, $y \in Z$, and $x$ is the
 isomorphism type\index{isomorphism types} of an immediate component of an
 element of $y$ (thus of an immediate component of each element of $y$).  The
 existence of $E$ as a set follows from stratified\index{stratification}
 comprehension\index{comprehension}.
\end{definition}

\begin{Lemme}{Induction Lemma}
  $E$ is a well-founded relation.
\end{Lemme}

\preuve\ It is sufficient to show that each
subset\index{subset} $S$ of $Z$ has a minimal element with respect to $E$.
Choose any element $s$ of $S$, and choose a connected well-founded
extensional relation $R$ from $s$.  Consider the intersection of $S$
with the set of equivalence classes\index{equivalence relations, equivalence
classes} under isomorphism\index{isomorphism types} of components 
of $R$.  The set of elements of $\Dom(R)$
determining these components has an $R$-minimal element, the
isomorphism class of whose component must be an $E$-minimal element
of $S$.  The proof is complete.\linebreak
\finpreuve

We demonstrate that the set $Z$ with the relation $E$ has properties
we would expect of a sort of set theory with $E$ as
membership\index{membership}.

\begin{Thm}{Extensionality\index{extensionality!lemma (for well-founded
relations)} Lemma} 
  For all $x,y \in Z$, $x=y$ if and only
  if for all $z$, $z\mathrel{E} x$ iff $z\mathrel{E} y$.
\end{Thm}

\preuve\ If the lemma is false, there is
$R \in x \in Z$ and $S \in y \in Z$ with $y \neq x$ such that the
immediate components of $S$ represent the same elements of $Z$ as
the immediate components of $R$.  If each isomorphism class\index{isomorphism
types} has at 
most one representative among the immediate components of each of
$R$ and $S$, then it would be easy to construct an isomorphism
between $R$ and $S$, contradicting our hypothesis.  Now choose an
component of $R$ with $R$-minimal top $t_1$ such that there is an
isomorphic component of $R$ with a different top $t_2$; each of the
elements of the preimage under $R$ of $t_1$ must have a
corresponding element in the preimage under $R$ of $t_2$ with an
isomorphic component, which by minimality of $t_1$ must be the same
object, implying equality of $t_1$ and $t_2$ by extensionality of
$R$.  The same argument applies to $S$.  We have a contradiction,
establishing the truth of the lemma.
\finpreuve

\begin{cor}
 $E$ is a well-founded extensional relation\index{relation!well-founded
 extensional relations}.
\end{cor}

The following Lemma allows us to associate extensional well-founded
relations\index{relation!well-founded extensional relations} with arbitrary well-founded relations:

\begin{Lemme}{Collapsing Lemma}
  For each well-founded relation $R$, there is
  an extensional well-founded relation $S$ and a map $g$ from $\Dom(R)$
  onto\index{onto map} $\Dom(S)$ such that for each element $x$ of
  $\Dom(R)$, and element $z$ of $\Dom(S)$, $z\mathrel{S}g(x)$ iff $z=g(y)$ for
  some $y$ such that $y\rR x$ (i.e., $\{z \st z\mathrel{S} g(x)\} =
  \{g(y) \st y\rR x\}$).  Moreover, the relation $S$ is determined
  up to isomorphism by $R$.
\end{Lemme}

\preuve\ For any equivalence relation $\sim$ on $\Dom(R)$, we define an
equivalence relation $\sim^+$ by ``$x\, \sim^+\, y$ iff for each $z \rR x$
there is a $w \rR y$ such that $w \sim z$ and for each $z \rR y$ there is a $w
\rR x$ such that $w \sim z$''.  The effect of $\sim^+$ is to identify elements
of $\Dom(R)$ which have the same 
``extension'' relative to $R$ up to the equivalence relation $\sim$.

We make the following additional observations: whenever $[\sim_1]
\subseteq [\sim_2]$, we will have $[\sim_1^+]
\subseteq [\sim_2^+]$. We call an equivalence relation $\sim$ {\itshape
nice\/} iff $[\sim] \subseteq [\sim^+]$; it follows from the previous
observation that for any nice $\sim$, $\sim^+$ is also nice.  In the discussion that follows, ``chain" will abbreviate ``chain in $\subseteq$", i.e., ``set of sets linearly ordered by inclusion".  It is
straightforward to establish that the union of any nonempty
chain of equivalence relations on $\Dom(R)$ is also an
equivalence relation on $\Dom(R)$; it follows from the previous
observations that the union of a nonempty chain of nice equivalence
relations will be a nice equivalence relation.

We define a set $\mathrm{Eq}$ as the intersection of all chains $A$ of
equivalence relations on $\Dom(R)$ such that $[=]\lceil \Dom(R) \in A$, for
each $[\sim] \in A$, $[\sim^+] \in A$, and, for each nonempty $B
\subseteq A$, $\bigcup[B] \in A$.  We have already observed that the
union of a nonempty chain of equivalence relations on $\Dom(R)$ is an
equivalence relation on $\Dom(R)$.  It is clear from the observations
above and the fact that $[=]\lceil \Dom(R)$ is a nice equivalence
relation that all relations in Eq are nice.  Eq is a chain because an intersection of a set of chains is a chain.

Now consider the relation $\bigcup[\mathrm{Eq}]$.  It must be an equivalence
relation; we denote it by $\sim_0$.  We must further have $[\sim_0^+]
= [\sim_0]$; $[\sim_0]$ is in Eq because it is a union of a chain of relations in
Eq (all of them); from this it follows that $[\sim_0^+]$ is also an
element of Eq, and so a subset of $[\sim_0]$; $[\sim_0]$ is a subset
of $[\sim_0^+]$ because it is nice, so we have the desired equation.

Select a choice set for the partition of $\Dom(R)$ determined by
$\sim_0$.  Let $g$ be the map sending each element of Dom$(R)$ to the
element of the choice set belonging to its equivalence class.  Let $S$
be the relation $\{(g(x),g(y))\st x \rR y\}$.  

We verify that $S$ is a well-founded relation.  Let $C$ be a nonempty
subset of $\Dom(S)$; the set $g^{-1}[C]$ will be nonempty and have an
$R$-minimal element $x$.  We claim that $g(x)$ is an $R$-minimal
element of $C$.  Suppose that $g(x)$ had a preimage $z$ under $S$ in
$C$; this means that $z=g(a)$ with $a\rR b$ and $b \sim_0 x$ for some
$a$ and $b$.  We also have $b \sim_0^+ x$, so each preimage under $R$
of $b$ (including $a$) stands in the relation $\sim_0$ to some
preimage under $R$ of $x$.  If we had $c \sim_0 a$ and $c \rR x$,
we would have a contradiction, because $g(c) = g(a) = z \in C$, and
$x$ was chosen $R$-minimal in $g^{-1}[C]$.  This establishes that $S$
is well-founded.

Finally, if we have $z \, S \, g(x)$, we have $z = g(a)$ for some $a$
such that $a \rR  b$ and $b \sim_0 x$ for some $b$.  Since we have
$b \sim_0 x$, we also have $b \sim_0^+ x$; each preimage of $b$ under
$R$ (and so $a$) stands in the relation $\sim_0$ to some preimage
under $R$ of $x$.  So we have $z = g(a) = g(y)$ for some $y \rR
x$, which is what we need to establish the desired property of $g$.  

It should be clear that $S$ is determined up to isomorphism by $R$.
The proof of the Collapsing Lemma is complete.
\finpreuve

It is worth noting that if the well-founded relation $R$ happens to be
extensional, the relation $S$ obtained from this construction will be
isomorphic to $R$.

A type-raising\index{type-raising operation} operation, denoted as
usual by $T$, can be defined on $Z$.  Observe that $R^{\iota}$, the singleton image of $R$, is a
connected well-founded extensional
relation\index{relation!well-founded extensional relations} if and
only if $R$ is one.

\begin{definition}
 If $R \in x \in Z$, we define $T\{x\}$\index{$T$ operation!on isomorphism
 types of well-founded relations} as the 
 uniquely determined element of $Z$ such that $R^{\iota} \in T\{x\} \in Z$.
\end{definition}

\begin{Lemme}{Isomorphism Lemma}
 For all $x,y \in Z$, $x\rE y$ iff
 $T\{x\}\rE T\{y\}$\index{$T$ operation!on isomorphism types of well-founded
 relations}.
\end{Lemme}

The truth of the Isomorphism Lemma is obvious, following from the
parallelism of structure between relations $R$ and $R^{\iota}$.

\begin{Lemme}{Comprehension\index{comprehension} Lemma}
 For each subset\index{subset} $S$ of $Z$, there is an
 element $s$ of $Z$ such that for all $x$, $T\{x\}\rE s$\index{$T$ operation!on
 isomorphism types of well-founded relations} iff $x \in S$.
\end{Lemme}

\preuve\ Choose any element $R$ of an
element $x$ of $S$.  Define a new relation
$$
 R' =
 \Bigl\{(\{a\},R),(\{b\},R)\st a \rR b\Bigr\}.
$$
It is easy to see that
$R'$ exists by stratified comprehension\index{Stratified
Comprehension Theorem} and is isomorphic to $R^{\iota}$, so belongs to
$T\{x\}$.  For each pair of elements $R_1,R_2$ of the set $S$,
$R_1'$ and $R_2'$ are disjoint\index{disjoint}.  Choose one
representative $R$ of each element of $S$, take the
union\index{union!set} of the corresponding relations $R'$ and add a
new object to the resulting relation which has as its preimage
exactly the set of tops of the relations $R'$.  The resulting
relation will be well-founded and have immediate components
representing exactly the elements of $T[S] = \{T\{x\} \st x \in
S\}$\index{$T$ operation!on isomorphism types of well-founded
relations}.  Apply the Collapsing Lemma to obtain an well-founded
extensional relation with the same properties (the isomorphism types
of its proper components will not be affected by collapse, because
these components are all extensional), whose equivalence
class\index{equivalence relations, equivalence classes} in $Z$ will
be the desired $s$.
\finpreuve

The obstruction to a stronger Comprehension\index{comprehension} Lemma allowing
us to construct $s$ with $S$ as its preimage under $E$ is that, while we can
choose a representative of each $x \in S$, we may not be able to
choose representatives of each $x \in S$ which are compatible with one
another in the sense that taking the union\index{union!set} of the
representatives will 
not introduce unintended relations between elements of their full
domains\index{domain!full}, as we were able to do with the $T\{x\}$'s by taking
pairwise disjoint\index{disjoint} representatives and applying the Collapsing
Lemma.  It is fortunate that there is such an obstruction, since we would
otherwise be able to find an $r$ such that $x \rE r$ iff not($x \rE x$),
reproducing Russell's paradox\index{paradoxes!Russell}!

Applying the Comprehension\index{comprehension} Lemma, we see that there is an
element $v$ of $Z$ such that $x\in v$ if and only if $x = T\{u\}$ for some $u$;
$v$ has immediate components covering all and exactly the
isomorphism\index{isomorphism types}
types of singleton\index{singleton image} images of connected well-founded
extensional relations\index{relation!well-founded extensional relations}.  For
example, $T\{v\}\rE v$.  But then, by repeated 
applications of the Isomorphism Lemma, $T^{n+1}\{v\}\rE T^n\{v\}$,
and there is an apparent ``descending chain'' in the relation $E$.  Of
course, the collection of $T^n\{v\}$'s is not a set (its definition is
clearly unstratified\index{stratification}) and so is not obligated to have an
$E$-minimal element.  This is reminiscent of the way that the Burali-Forti
paradox\index{paradoxes!Burali-Forti} 
is avoided in the ordinals\index{ordinal numbers}.





\section[A Hierarchy in $Z$]{A Hierarchy in \mathversion{bold}$Z$}

There is a close relationship between the structure of $Z$ and the
structure of the ordinals.  Each strict well-ordering\index{order (strict)}
with a maximum element is a connected well-founded extensional
relation\index{relation!well-founded extensional relations}; this sets up 
a natural correspondence associating each ordinal $\alpha$ with the
equivalence class\index{equivalence relations, equivalence classes} of the
strict well-orderings obtained by rendering  elements of $\alpha+1$
irreflexive.  We use the latter objects as 
ordinals\index{ordinal numbers} in the rest of this chapter.  With the
following definitions, we introduce a hierarchical structure on $Z$ which can
be indexed by the ordinals:

\begin{definition}
 If $S$ is a subset\index{subset} of $Z$, we define $P(S)$ as the
 collection of all elements of $Z$ which have preimages under $E$
 which are subsets of $S$.  Note that $P(S)$ and $S$ have the same
 relative type\index{types (relative)}; P is a function\index{function} coded
 by a set.  P codes a kind of ``power set''\index{power set} operation on
 subsets\linebreak of $Z$.
\end{definition}

\begin{definition}
 We define the set $H \subseteq {\cal P}\{Z\}$ as the
 intersection\index{intersection!set} of all sets ${\cal H} \subseteq {\cal
 P}\{Z\}$ such that 
 for each $A \in {\cal H}$, P$(A) \in {\cal H}$ and for each ${\cal A}
 \subseteq {\cal H}$, $\bigcup[{\cal A}] \in {\cal H}$.  In other words,
 $H$ is the smallest collection of subsets\index{subset} of $Z$ which is closed
 under P and under the operation of taking set unions\index{union!set} of
 subcollections.
\end{definition}

\begin{definition}
 A subset $A$ of $Z$ is said to be {\upshape
 transitive} if $A \subseteq  P(A)$, i.e., if the preimages
 under $E$ of elements of $A$ are subsets of $A$.
\end{definition}

\begin{lemme}
 Each element of $H$ is transitive.
\end{lemme}

\preuve The result of applying P to a transitive subset of
$Z$ is transitive, and so is the set union\index{union!set} of a collection of
transitive subsets\index{subset} of $Z$.  By the definition of $H$, $H$ is a
subset of the set of transitive subsets of $Z$.
\finpreuve

\begin{lemme}
 The relation of set inclusion\index{inclusion} restricted to $H$ is a
 well-ordering\index{well-orderings}, and each element of $A$ is immediately
 succeeded by $P(A)$ in this order.
\end{lemme}

\preuve\ Consider the collection of elements $A$ of $H$
such that the elements $B \subseteq A$ of $H$ are well-ordered by
inclusion, and for each such $B$ either $B=A$ or $P(B) \subseteq A$.
Observe that for any such $A$, the collection of elements of $H$
which are either subsets of $A$ or supersets of $A$ will be closed
under $P$ and under set unions\index{union!set} of its subcollections, and so
will include all elements of $H$.  Thus, the collection of such $A$'s is
well-ordered\index{well-orderings} by inclusion\index{inclusion}.  It is easy
to see from the fact that it 
is ordered that the collection of such $A$'s is closed under set
unions of subcollections, and it should be clear that $P(A)$ belongs
to this collection if $A$ does, so this collection includes all of
$H$.  From this it follows that $H$ is well-ordered by inclusion and
that $P(A)$ is the successor of $A$ in this well-ordering for each
$A \in H$.
\finpreuve

The elements of $H$ are subsets of $Z$ constructed by a process of
repeatedly taking ``power sets''\index{power set} and taking
unions\index{union!set} at limit stages.
We are interested in the question of when these ``power sets'' are
genuine:

\begin{definition}
 A rank $A$ (an element of $H$) is called a {\upshape
 complete rank} when it is the case that for each subset $B$ of
 $A$, there is an element $b$ of $P(A)$ such that the preimage
 under $E$ of $b$ is precisely $B$.  A complete rank is one whose
 full power set is represented in $Z$ in the obvious sense.
\end{definition}

\begin{lemme}
 There is an incomplete rank (a rank which is not complete).
\end{lemme}

\preuve\ Since $H$ is closed under unions\index{union!set}, $\bigcup[H]$ is an
element of $H$.  This implies that $P(\bigcup[H])$ = $\bigcup[H]$,
since $H$ is closed under $P$ and $\bigcup[H]$ must be its maximal
element in the inclusion\index{inclusion} order.  This implies further that
$\bigcup[H]$ = $Z$.  Suppose otherwise and consider an $E$-minimal
element of $Z$ not in $\bigcup[H]$; clearly this would be an element
of $P(\bigcup[H])= \bigcup[H]$, which is absurd.  So $Z$ itself is a
rank, and an incomplete one, since $Z$ itself is clearly not
represented by any element of $P(Z) = Z$ (cycles cannot occur in
well-founded relations).
\finpreuve


\begin{definition}
 We define $Z_0$ as the first incomplete rank in
 the order determined by inclusion\index{inclusion}.
\end{definition}


\begin{lemme} For any rank
$X$ at or before $Z_0$, $T[X]$ is also a rank.
\end{lemme}

\preuve\ We define $T[S]$, for any set $S$ of subsets of $Z$,
as the set of all $T[X]$ for $X$ in $S$.

Now consider $\cal H$ and $T[{\cal H}]$, where $\cal H$ is the set of
ranks.  If there is no $X$ in $\cal H$ such that $T[X]$ is not in $\cal H$ then we have
the desired result.  Suppose on the contrary that $X$ is the first rank
such that $T[X]$ is not a rank, but each $T[Y]$ for $Y$ a rank below $X$ is a
rank.  $X$ is either a successor rank, so $X = {\cal P}\{Y\}$, but $T[X]$ is not
$T[{\cal P}\{Y\}]$ (this can only be true if $Y$ is not a complete rank, which puts
$X$ above $Z_0$ as desired) or a limit rank, in which case $X$ is the union
of the $Y$'s.  This case is impossible, because $T[X]$ would then be the
union of the $T[Y]$'s and itself a rank.

Since there {\em is\/} an incomplete rank, there is in fact an $X$ in $\cal H$ (the
successor ${\cal P}\{Z_0\}$ of the first incomplete rank $Z_0$) such that $T[X]$ is
not in $\cal H$ ($T[{\cal P}\{Z_0\}]$ is a proper subset of ${\cal P}\{T[Z_0]\}$).
\finpreuve

\begin{ThmEtc}{Remark.}
 $Z_0 \neq  T[Z_0]$.  For every rank of the form
 $T[A]$ is complete by the Comprehension\index{comprehension} Lemma.  This
 implies that the sequence\index{sequence} of iterated images of $Z_0$ under
 $T$ forms an external descending chain in the inclusion order on ranks, which
 cannot, of course, be a set.
\end{ThmEtc}

\begin{lemme}
 Each complete rank contains an ordinal; each complete
 successor\index{successor!rank} 
 rank contains one ``new'' ordinal.  (We remind the reader that by
 ``ordinal'' we mean here ``isomorphism class\index{isomorphism
 types} of {\upshape strict}
 well-orderings\index{well-orderings}\index{order (strict)} with a
 largest element'').
\end{lemme}

\preuve\ Each ordinal has as its preimage under $E$ the
set of smaller ordinals\index{ordinal numbers}.  The proof proceeds by
induction: consider 
the minimal complete rank for which this is not the case, and
consider the element of this minimal rank which has as its preimage
under $E$ exactly the set of ordinals occurring in previous ranks (this
exists because the rank is complete); this is clearly a new ordinal!
The contradiction establishes that each complete rank contains an
ordinal.  It should also be clear from the construction that one new
ordinal is added at each complete successor rank.
\finpreuve

\begin{definition}
 To each complete rank in $H$, we can assign the
 ordinal which first appears in its successor\index{successor!rank} rank as an
 index (the first ordinal which does not belong to the rank).  If we want the
 index to have the same relative type\index{types (relative)} as the rank, we
 need to apply the $T$ operation.  Incomplete ranks can also be assigned
 indices in this way.
\end{definition}

The set $Z_0$ with the relation $E$ interpreted as membership\index{membership}
looks like a set theory of the same general kind as
ZFC\index{Zermelo--Fraenkel set theory}.  We will 
prove later that there are ranks which are models of ZFC.



\section[ZFC in the Cantorian part of $Z_0$]{Interpreting ZFC in the\\
 Cantorian
 part of \mathversion{bold}$Z_0$}\index{Cantorian!element of $Z$}

Our aim here is to suggest that the study of the usual\index{Zermelo--Fraenkel
set theory} set theory can 
be understood in the context of our set theory as the study of
isomorphism classes\index{isomorphism types} of strongly
Cantorian\index{Cantorian, strongly!element of $Z$} connected well-founded 
extensional relations\index{relation!well-founded extensional relations}.  The
hierarchy coded in the set $H$ looks like 
the hierarchy usually seen in ZFC\index{Zermelo--Fraenkel set theory}, except
for the fact that there is an external ``isomorphism'' $T$ from $Z_0$ into
itself sending very high levels of the hierarchy down to lower (but still very
high) 
levels.  The universe\index{universe, universal set} of ZFC is usually
understood as being 
``constructed'' in a sequence\index{sequence!transfinite} of stages indexed by
the ordinals\index{ordinal numbers}, with 
power sets\index{power set} being taken at each
successor\index{successor!ordinal} ordinal and unions\index{union!set} being 
taken at each limit ordinal, just as the sequence $H$ is constructed
here.

We summarize the status of axioms of ZFC\index{Zermelo--Fraenkel set theory} in
an interpretation in the strongly Cantorian\index{Cantorian, strongly!element
of $Z$} elements of $Z_0$:

\begin{description}
 \item[\fdescr Extensionality:]  The Extensionality
   Lemma\index{extensionality!lemma (for well-founded relations)} establishes
   this.

 \item[\fdescr Pairing:] Use the Comprehension\index{comprehension} Lemma on
   the condition defining a pair of Cantorian\index{Cantorian!element of $Z$}
   elements of $Z_0$.  The two elements are fixed under the $T$ operation, so
   one gets the right extension. 

 \item[\fdescr Union\index{union!set}:] Use the
   Comprehension\index{comprehension} Lemma on the condition of being 
   the isomorphism type\index{isomorphism types} of an immediate component of
   an immediate 
   component of an element of a fixed Cantorian\index{Cantorian!element of $Z$}
   element of $Z_0$. 
   Components of Cantorian\index{Cantorian!element of $Z$} elements of $Z_0$
   are Cantorian, thus fixed by the $T$ operation, so one gets the right
   extension.

 \item[\fdescr Power Set\index{power set!axiom of ZFC}:] Use the
   Comprehension\index{comprehension} Lemma on the condition of 
   being the isomorphism type of a relation having each immediate
   component isomorphic to an immediate component of an element of a
   fixed Cantorian element of $Z_0$. A relation isomorphic to a subrelation of
   a Cantorian\index{Cantorian!element of $Z$} relation will be Cantorian, so
   elements will be fixed under the $T$ operation.

 \item[\fdescr Infinity\index{infinity}:]  Obviously holds; use the Cantorian
   ordinal $\omega$\index{Cantorian!ordinal}.

 \item[\fdescr Separation:] A full proof of Separation\index{Separation (axiom
   of ZFC)} would require an appeal to the Axiom of Small Ordinals\index{Axiom
   of Small Ordinals}.  The difficulty is that 
   quantifications over all sets would translate to quantifications over
   the domain\index{domain} of all Cantorian\index{Cantorian!element of $Z$}
   elements of $Z_0$, which are unstratified\index{stratification}. 
   It is certainly possible to use the Axiom of Small Ordinals to prove
   that Separation holds, but we will not do it here.  We will note that
   any condition in which each quantifier is restricted to a set will
   translate to a stratified condition defining a set, and we would then
   be able to apply the Comprehension\index{comprehension} Lemma.
\end{description}

The\index{Mac Lane set theory|see {Zermelo--Fraenkel set theory (bounded)}}
subtheory of ZFC which omits the Axiom of
Replacement\index{Replacement(axiom of ZFC)} and
restricts Separation to sentences in which each quantifier is
restricted to a set is called ``bounded Zermelo\index{Zermelo--Fraenkel set theory!bounded} set theory'' or ``Mac
Lane set theory''; it is adequate for most mathematical purposes.  The
restriction of quantifiers to sets harmonizes with the restriction of
the bound variables in set definitions in the Axiom of
Separation\index{Separation (axiom of ZFC)}. The 
discussion here should show that the Theorem of Counting\index{Axiom of
Counting} is the only 
consequence of the Axiom of Small Ordinals\index{Axiom of Small Ordinals}
needed to show that the 
Cantorian\index{Cantorian!element of $Z$} part of $Z_0$ provides an
interpretation of Mac Lane set 
theory (the Theorem of Counting is needed for Infinity\index{infinity}).  The
mathematical strength of our theory without the Axiom of Small
Ordinals or the Axiom of Counting turns out to be the same as that of
Mac Lane set theory.

\begin{description}
 \item[\fdescr Replacement\index{Replacement(axiom of ZFC)}:] Verification of
   this would require the Axiom of Small Ordinals\index{Axiom of Small
   Ordinals}.  For the verification of both Separation and 
   Replacement, observe that the Cantorian elements of $Z_0$ can be coded
   into the Cantorian ordinals\index{ordinal numbers}\index{Cantorian!ordinal},
   producing essentially the same structure as the Construction of the previous
   chapter.

 \item[\fdescr Foundation\index{Foundation (axiom of ZFC)} and
   Choice\index{axiom of choice}:] These axioms are easily seen to hold; 
   Foundation is immediate, while the construction of a choice set would
   require the Comprehension\index{comprehension} Lemma.
\end{description}




\section[Our set theory in $Z_0$] {Interpreting our own set theory in
\mathversion{bold}$Z_0$.\\
The Axiom of Endomorphism}

It is further possible to interpret our own set theory (rather than
ZFC\index{Zermelo--Fraenkel set theory}) inside the structure of isomorphism
classes\index{isomorphism types}, by defining 
membership\index{membership} in a slightly different way.  The ``membership''
$E$ of $Z$ or $Z_0$ is a relation realized as a set, so there are no
stratification\index{stratification} restrictions on
``comprehension\index{comprehension}'' (instead, the 
restriction is a ``size'' restriction; collections of elements of $Z$
are not represented by elements of $Z$ if they are too large).

\begin{lemme}
 $P(T[Z_0])$ is a complete rank.
\end{lemme}

\preuve\ Since $T[Z_0]$ is a complete rank, we would
  otherwise have $P(T[Z_0]) = Z_0$, the first incomplete rank.
  Now consider the set of elements $x$ of $H$ such that $P^n(x) =
  Z_0$ for some natural\index{natural number} number $n$; this would coincide
  with the set 
  of iterated images of $Z_0$ under $T$, which cannot be a set.  This
  can be argued directly using mathematical
  induction\index{induction!mathematical, on an unstratified condition} on an 
  unstratified\index{stratification} condition, but this appeal to the Axiom of
  Small Ordinals\index{Axiom of Small Ordinals} can be avoided at the price of
  making the argument more 
  technical: roughly, the set of ranks $x$ such that $P^n(x) = Z_0$
  for some $n$ must be finite\index{finite!set}, since it is a descending
  sequence\index{sequence} of ranks; it has either an odd or an even number of
  elements.  We then observe that the image under $T$ of this sequence clearly
  has the same parity ($T$ preserves that kind of structure) and also opposite
  parity (since it is obtained from the original sequence by deleting just
  the largest element).  This is impossible.
\finpreuve

\begin{definition}
 We define another pseudo-membership\index{membership} ``relation'' on
 $Z_0$: $x \mathrel{\epsilon} y$ iff $T\{x\}\rE y$ and $y\rE
  P(T[Z_0])$ (i.e., each element of the preimage under $E$ of $y$
  is of the form $T\{x\}$\index{$T$ operation!on isomorphism types of
  well-founded relations} for some $x$.)
\end{definition}

\begin{thm}
 The theory with domain $Z_0$ and membership
 ``relation'' $\epsilon$ satisfies the axioms of our set theory.
\end{thm}

\preuve\ Elements of $P(T[Z_0])$ are regarded as
sets; other objects are regarded as atoms\index{atoms}.  The axioms of
extensionality\index{extensionality} and atoms are easily seen to be satisfied.
Stratified comprehension\index{Stratified Comprehension Theorem} holds because
the type\index{types (relative)} differential of the 
``relation'' $\epsilon$ is the same as that for membership\index{membership},
so sets of elements of $Z_0$ defined by stratified sentences in $\epsilon$
instead of $\in$ exist; the element of $Z_0$ coding such a set can
be obtained by taking the image under T of the set and using the
fact that $P(T[Z_0])$ is a complete rank.  The Axiom of Small
Ordinals\index{Axiom of Small Ordinals} holds because the sets of
ordinals\index{ordinal numbers} which it requires are 
all codable in the complete rank $T[Z_0]$.  Verification of Ordered
Pairs\index{ordered pair} is technical; one could use the Kuratowski
pair\index{ordered pair!Kuratowski} in the set 
theory with $E$ as membership\index{membership} if $Z_0$ is known not to be a
successor\index{successor!rank} type; the existence of a suitable pair is
established in any case by the fact that $Z_0$ is infinite\index{infinite!set}
and the theorem $\kappa$.$\kappa$ = $\kappa$ (for infinite $\kappa$) of
cardinal\index{cardinal numbers} arithmetic.
\finpreuve

The way in which our set theory is coded into the ``set theory'' of
$Z$ is exactly analogous to a way that it can be interpreted in
ZFC\index{Zermelo--Fraenkel set theory}; it is sufficient to show that a
nonstandard model of part of the hierarchy of ZFC can be constructed with an
external automorphism which moves a rank downward.  We find it very appealing
that this set theory's interpretation of ZFC reveals how this
set theory can in turn be interpreted inside ZFC.

The interpretation of our set theory inside $Z$ satisfies the
following additional axiom:

\begin{axiom}{Axiom of Endomorphism\index{Endomorphism, axiom of}}
 There is a one-to-one\index{one-to-one map} map\/ $\Endo$ from
 ${\cal P}_1\{V\}$ (the set of singletons\index{singleton}) into ${\cal
 P}\{V\}$ (the set of sets) such that for any set $B$, $\Endo(\{B\}) =
 \{\Endo(\{A\})\st A \in B\}$.
\end{axiom}

\begin{thm}
 The Axiom of Endomorphism is satisfied in the
 interpretation of set theory in $Z_0$ outlined above.
\end{thm}

{\sc Indication of proof.~---} Let $\Endo$ be interpreted as the
restriction to $P(T[Z_0])$ of the map which takes each element of
$Z_0$ whose preimage under $E$ has exactly one element to that one
element (the inverse of the singleton\index{singleton ``map''} map in the set
theory with membership\index{membership} $E$).  Verification that the Axiom
holds is straightforward.
\finpreuve

The Axiom of Endomorphism has the appealing feature that it allows us
to assign set-theoretical structure to atoms\index{atoms} (an atom $a$ is
associated with a set Endo$(\{a\})$).  It also allows us to define a
type-level\index{type-level operation} ``membership'' relation $x \,E\,y$ by
$\Endo(\{x\}) \in y$.  A curious fact is that, while
membership\index{membership}
in our set theory is certainly not well-founded, and the type-level membership
$E$ externally parallels the usual membership relation $\in$, it is possible
(e.g., in the interpretation of our set theory in $Z_0$) for the type level
relation $E$ to be well-founded, an analogue for our set theory of the
Axiom of Foundation\index{Foundation (axiom of ZFC)} in
ZFC\index{Zermelo--Fraenkel set theory}.




\Exercises

\begin{enumerate}

\item  We call a set $A$ {\em well-founded and transitive\/} if every element
  of $A$ is a subset of $A$ and every subset of $A$ is disjoint\index{disjoint}
  from one of its elements (i.e., every subset\index{subset} of $A$ has an
  $\in$-minimal element).  We call a set {\itshape well-founded\/} if it is an
  element of some well-founded transitive set.  Show that the class of all
  well-founded sets cannot be a set.  You need only Boolean\index{Boolean
  algebra, operations} operations on sets for the argument (if that much).
  This is called Mirimanoff's paradox\index{paradoxes!Mirimanoff} (mod the
  exact definition of ``well-founded'').  Why is this not a problem for our set
  theory? 

\item  Verify that the following definition of an ordered pair\index{ordered
  pair} on $Z_0$ satisfies the basic properties of an ordered pair and can be
  used to witness the Axiom of Ordered Pairs in the interpretation of our set
  theory in $Z_0$.  For purposes of this exercise, let numerals stand for the
  isomorphism types\index{isomorphism types} of strict
  well-orderings\index{well-orderings} of finite\index{finite!set} sets.  For
  each isomorphism type $x$, define $x'$ as the isomorphism type resulting if
  each immediate component of an immediate component of an element of $x$ which
  belongs to a numeral $n$ is replaced by an element of the numeral $n+1$,
  while $x''$ results from the same operation with the additional step of
  adding an element of the numeral $0$ as an immediate component of each
  immediate component of the element of $x$.  The ordered pair
  $\left<x,y\right>$ is defined (locally to this exercise) as the isomorphism
  type ``$x' \cup y''\,$'' an element of which has as immediate components the
  immediate components of an element of $x'$ and an element of $y''$.  This
  definition of the ordered pair (cast in terms of membership\index{membership}
  rather than immediate componenthood, of course) is due to Quine.  Notice that
  the rank of $\left<x,y\right>$ is no higher than the maximum of the ranks of
  $x$ and $y$ (if these ranks are infinite\index{infinite!rank}); this is the
  technical advantage of this pair. 
\end{enumerate}
