\chapter[Strongly Cantorian Sets]{Strongly Cantorian Sets and\\ Conventional
Set Theory}\index{Cantorian, strongly!set}


\section[Consequences of Small Ordinals]{Consequences of the Axiom of\\ Small
Ordinals}\index{Axiom of Small Ordinals}

We ``take the gloves off" and start using the Axiom of Small
Ordinals\index{Axiom of Small Ordinals}.

\begin{thm}
 All Cantorian\index{Cantorian!set} sets (thus cardinals\index{cardinal
 numbers}, ordinals) are strongly Cantorian.
\end{thm}

\preuve\ Let $A$ be a Cantorian set.  Let $R$ be a
well-ordering\index{well-orderings} of $A$; $R$ then belongs to a Cantorian
ordinal\index{Cantorian!ordinal} $\alpha$, and $R$ is similar\index{similarity}
to the natural order on seg$\{\alpha\}$\index{segment}.  The similarity which
witnesses this fact is a bijection between $A$ and seg$\{\alpha\}$.
It is sufficient to show that seg$\{\alpha\}$ is strongly Cantorian.
Define a map by transfinite\index{transfinite!recursion} recursion: $f(0) =
\{0\}$; $f(\beta+1) = \{\gamma+1\}$ if $f(\beta) = \{\gamma\}; f(\lim B)$ = the
singleton\index{singleton} of the limit of the collection of
$\bigcup[f(\beta)]$'s for $\beta \in B$.  It is straightforward to prove by
(stratified\index{stratification}) transfinite 
induction that $f(\alpha) = \{T^{-1}\{\alpha\}\}$ for each ordinal
$\alpha$ which is an image under $T$; it follows that $f(\alpha) =
\{\alpha\}$ for each Cantorian $\alpha$.  We have already established
that each ordinal less than a Cantorian ordinal is Cantorian, so
$f(\beta) = \{\beta\}$ for each $\beta<\alpha$, so $f$ witnesses the
fact that seg$\{\alpha\}$\index{segment} is strongly Cantorian\index{Cantorian,
strongly!set}.
\finpreuve

This shows that the concepts of Cantorian and strongly
Cantorian sets, ordinals\index{ordinal numbers}, or cardinals\index{cardinal
numbers} collapse together in the presence of the Axiom of Small
Ordinals\index{Axiom of Small Ordinals}.  We will use the term 
``Cantorian" hereafter for this concept.

\begin{thm}
 Every set of Cantorian\index{Cantorian!ordinal} ordinals is Cantorian.
\end{thm}

\preuve\ The map $f$ defined in the proof of the preceding
theorem sends each Cantorian ordinal\index{Cantorian!ordinal} to its
singleton\index{singleton}; its restriction to any set $A$ of Cantorian
ordinals\index{ordinal numbers} witnesses the Cantorian\index{Cantorian!set}
character of $A$.
\finpreuve

We show that the Cantorian\index{Cantorian!set} sets are closed under certain
constructions.

\begin{thm}
 Any pair $\{a,b\}$ is Cantorian.
\end{thm}

\preuve\ Obvious.
\finpreuve

\begin{thm}
 There is an infinite\index{infinite!set} Cantorian set.
\end{thm}

\preuve ${\cal N}$.
\finpreuve

\begin{thm}
 If $A$ is Cantorian, $\{x \in A \st \phi\}$ exists for any sentence $\phi$.
\end{thm}

\preuve\ If $A$ is Cantorian\index{Cantorian!set}, there is a bijection $f$
between $A$ and $\seg\{\alpha\}$\index{segment} 
for a Cantorian\index{Cantorian!ordinal} ordinal $\alpha$.  $f[\{x \in A \st
\phi\}]$ is a collection of small ordinals\index{ordinal numbers} definable by
a sentence, and so has exactly the small ordinal elements of a set $B$.  The
set $f^{-1}[B \cap \seg\{\alpha\}]$ is the desired set.
\finpreuve

This gives us comprehension\index{comprehension} for any property at all (that
we can express in our language) as long as we restrict our attention to
elements of a fixed Cantorian\index{Cantorian!set} set.

\begin{thm}
 If $A$ is Cantorian, ${\cal P}\{A\}$ is Cantorian.
\end{thm}

\preuve\ If $K$ is the map which takes $a$ to $\{a\}$ for each $a$ in $A$, the
map $(B \mapsto  K[B])$ for $B$ in ${\cal P}\{A\}$ has a
stratified\index{stratification} definition, and sends each 
$B$ to ${\cal P}_1\{B\}$.  The map $({\cal P}_1\{B\} \mapsto  \{B\})$ has a
stratified definition; 
composition\index{composition} yields the singleton\index{singleton ``map''}
map on ${\cal P}\{A\}$.
\finpreuve

Since $|{\cal P}\{A\}| > |A|$ for $A$ Cantorian\index{Cantorian!set}, this
enables us to build larger and larger Cantorian sets.

\begin{definition}
 We use transfinite\index{transfinite!recursion} recursion to code
 pairs\index{ordered pair!of ordinals, coded as an ordinal} of
 ordinals\index{ordinal numbers} as ordinals:
 \begin{itemize}
  \item define
    $\left<\left<0,0\right>\right>$ as 0;
  \item define $\left<\left<\beta,\gamma+1\right>\right>$ as
    $\left<\left<\beta,\gamma\right>\right> + 2$, where $\gamma <
    \beta$;
  \item define $\left<\left<\alpha,\beta\right>\right>$ as
    $\left<\left<\beta,\alpha\right>\right>+1$ whenever $\alpha < \beta$
    (notice that the first projection is greater than or equal to the
    second in each other case defined here);
  \item define
    $\left<\left<\alpha,\lim B\right>\right>$ as $\lim
    \{\left<\left<\alpha,\beta\right>\right> \st \beta \in B\}$ when
    $B$ has no greatest element and $\alpha$ is greater than all
    elements of $B$;
  \item define $\left<\left<\beta+1,0\right>\right>$ as
    $\left<\left<\beta,\beta\right>\right>+1$;
  \item define $\left<\left<\lim
    B,0\right>\right>$ as\newline lim $\{\left<\left<\beta,\gamma\right>\right>
    \st \beta,\gamma \in B\}$, where $B$ has no greatest element.
 \end{itemize}
\end{definition}
% Pictures for the definition of page 90
%
  This definition can be pictured as follows:

\vspace{-1.2\baselineskip}

\begin{minipage}[t]{.33\linewidth}
\hspace*{-7mm}\begin{picture}(0,0)%
\special{psfile=fig1.pstex}%
\end{picture}%
\setlength{\unitlength}{0.00087500in}%
%
\begingroup\makeatletter\ifx\SetFigFont\undefined%
\gdef\SetFigFont#1#2#3#4#5{%
  \reset@font\fontsize{#1}{#2pt}%
  \fontfamily{#3}\fontseries{#4}\fontshape{#5}%
  \selectfont}%
\fi\endgroup%
\begin{picture}(1511,1500)(26,-1990)
\put(907,-1771){\makebox(0,0)[lb]{\smash{\SetFigFont{10}{12.0}{\familydefault}{\mddefault}{\updefault}$A$}}}
\put(1267,-1276){\makebox(0,0)[lb]{\smash{\SetFigFont{10}{12.0}{\familydefault}{\mddefault}{\updefault}$B$}}}
\put(1537,-1906){\makebox(0,0)[lb]{\smash{\SetFigFont{10}{12.0}{\familydefault}{\mddefault}{\updefault}Ord}}}
\put(496,-826){\makebox(0,0)[b]{\smash{\SetFigFont{10}{12.0}{\familydefault}{\mddefault}{\updefault}Ord}}}
\end{picture}
\\[2mm]
\begin{tabular}{l}
 Each point of $B$ is\\
 greater than each\\
 point of $A$.
\end{tabular}
\end{minipage}
%
\begin{minipage}[t]{.33\linewidth}
\hspace*{-7mm}\begin{picture}(0,0)%
\special{psfile=fig2.pstex}%
\end{picture}%
\setlength{\unitlength}{0.00087500in}%
%
\begingroup\makeatletter\ifx\SetFigFont\undefined%
\gdef\SetFigFont#1#2#3#4#5{%
  \reset@font\fontsize{#1}{#2pt}%
  \fontfamily{#3}\fontseries{#4}\fontshape{#5}%
  \selectfont}%
\fi\endgroup%
\begin{picture}(1505,1500)(386,-1990)
\put(1305,-1096){\makebox(0,0)[b]{\smash{\SetFigFont{10}{12.0}{\familydefault}{\mddefault}{\updefault}$x$}}}
\put(1620,-1681){\makebox(0,0)[lb]{\smash{\SetFigFont{10}{12.0}{\familydefault}{\mddefault}{\updefault}$y$}}}
\put(1891,-1906){\makebox(0,0)[lb]{\smash{\SetFigFont{10}{12.0}{\familydefault}{\mddefault}{\updefault}Ord}}}
\put(856,-826){\makebox(0,0)[b]{\smash{\SetFigFont{10}{12.0}{\familydefault}{\mddefault}{\updefault}Ord}}}
\end{picture}
\\[2mm]
\begin{tabular}{l}
When $\pi_1(x)>\pi_2(y)$,\\
 $x>y$ because $x$ \\
is farther up the \\
diagonal than $y$ \\
($\pi_1(x)<\pi_2(y)$ \\
would imply $y>x$)
\end{tabular}
\end{minipage}
%
\begin{minipage}[t]{.33\linewidth}
\hspace*{-7mm}\begin{picture}(0,0)%
\special{psfile=fig3.pstex}%
\end{picture}%
\setlength{\unitlength}{0.00087500in}%
%
\begingroup\makeatletter\ifx\SetFigFont\undefined%
\gdef\SetFigFont#1#2#3#4#5{%
  \reset@font\fontsize{#1}{#2pt}%
  \fontfamily{#3}\fontseries{#4}\fontshape{#5}%
  \selectfont}%
\fi\endgroup%
\begin{picture}(1505,1500)(386,-1990)
\put(1891,-1906){\makebox(0,0)[lb]{\smash{\SetFigFont{10}{12.0}{\familydefault}{\mddefault}{\updefault}Ord}}}
\put(856,-826){\makebox(0,0)[b]{\smash{\SetFigFont{10}{12.0}{\familydefault}{\mddefault}{\updefault}Ord}}}
\put(1261,-1096){\makebox(0,0)[b]{\smash{\SetFigFont{10}{12.0}{\familydefault}{\mddefault}{\updefault}$x$}}}
\put(1621,-1501){\makebox(0,0)[lb]{\smash{\SetFigFont{10}{12.0}{\familydefault}{\mddefault}{\updefault}$y$}}}
\end{picture}
\\[2mm]
\begin{tabular}{l}
 If the projections \\
of $x$ and $y$ onto \\
the diagonal are equal \\
($\pi_1(x) = \pi_2(y)$) \\
then $x>y$ because $x$ \\
is higher than $y$.
\end{tabular}
\end{minipage}

Notice that the projections\index{projections!relative type of}\index{types
(relative)} of a coded pair are at the same relative 
type as the coded pair.  Observe that the order on the codes of
pairs of ordinals\index{ordinal numbers} is determined by considering first the
maximum of 
the two ordinals, then the smaller ordinal, then the position of the
larger ordinal; technically, the order on codes
$\left<\left<\alpha,\beta\right>\right>$ is induced by the
lexicographic order on triples
$(\max(\alpha,\beta),\min(\alpha,\beta),i)$, where $i = 0$ if
$\alpha > \beta$ and 1 otherwise.  It is straightforward to
demonstrate that $\left<\left<\alpha,\beta\right>\right>$ exists for
each $\alpha$ and $\beta$: let $W$ be a well-ordering\index{well-orderings}
such that $W_{\alpha}$ and $W_{\beta}$ exist; then consider the order $R$ on
$\dom(W) \times \dom(W)$ defined by considering first the maximum of
the projections\index{projections} of a pair, then its smaller projection, then
the position of the larger projection (as above): $(W_{\alpha},W_{\beta}) =
R_{\left<\left<\alpha,\beta\right>\right>}$ will hold.  It is
straightforward to show that
$T\{\left<\left<\alpha,\beta\right>\right>\} =
\left<\left<T\{\alpha\},T\{\beta\}\right>\right>$, for any
ordinals\index{ordinal numbers} $\alpha$, $\beta$.\index{$T$ operation!on
ordinals}  It follows from this that pairs of
Cantorian\index{Cantorian!ordinal} ordinals are coded exactly by the Cantorian
ordinals.

\begin{Thm}{Replacement\index{Replacement(axiom of ZFC)}
Theorem}\label{repl-thm}
 If $A$ is Cantorian\index{Cantorian!set} and $\phi$ is a sentence such that
 ``for each $x$ in $A$, there is exactly one Cantorian ordinal $y$ such that
 $\phi$" holds, then $\{y \st y$ is a Cantorian ordinal and for some $x \in A,
 \phi\}$ exists.
\end{Thm}

\preuve\ Let $f$ be a bijection from $A$ to seg$\{\alpha\}$\index{segment} for
some Cantorian ordinal $\alpha$.  Consider the collection of small
ordinals\index{ordinal numbers} $\{\left<\left<f(x),y\right>\right> \st x \in
A, y$ is a Cantorian ordinal, and $\phi\}$; there is a set $B$ which contains
exactly these small ordinals.  Now consider the set $C = \{y \st$
for some $x \in A, y$ is the smallest ordinal such that
$\left<\left<f(x),y\right>\right> \in B\}$; this is the desired set
of ordinals.  Any unwanted elements of the ``image" of $B$ are
disposed of by choosing the smallest $y$ for each $f(x)$, which must
be the unique Cantorian\index{Cantorian!ordinal}~$y$.
\finpreuve

This theorem says that if we can ``replace" each element of a
``small" set with a Cantorian ordinal, the replacements themselves make
up a set.  We cannot have such a theorem about ``replacement" with
general objects: the numbers 0,1,2$\ldots$ can be ``replaced" with $\Omega,
T\{\Omega\}, T^2\{\Omega\},\ldots$\index{$T$ operation!on ordinals} which do
not make up a set.  One consequence of the theorem is that no Cantorian
set\index{Cantorian!set} can contain all of these objects.

\begin{thm}
 If $A$ is a Cantorian\index{Cantorian!set} set, and each element of $A$ is
 a Cantorian set, $\bigcup[A]$ is Cantorian.
\end{thm}

\preuve\ Consider $\sum[ A]$, the disjoint sum\index{disjoint!disjoint
sum} of $A$ (indexed by $A$ itself); it is partitioned\index{partition} into
disjoint sets, on each of which we know how to construct the
singleton\index{singleton ``map''} map; the question is whether we 
can put all these maps together uniformly.  It is larger than ${\cal
P}_1\{\bigcup[A]\}$ (there is a natural map from $\sum[ A]$ onto\index{onto
map} ${\cal P}_1\{\bigcup[A]\}$), so if it is Cantorian\index{Cantorian!set}
${\cal P}_1\{\bigcup[A]\}$ and so $\bigcup[A]$ itself must be Cantorian.  The
trick is to use the coding of pairs of ordinals\index{ordinal numbers} by
ordinals again:
choose a bijection $f_B$ of $B$ onto seg$\{\beta\}$\index{segment} for some
ordinal $\beta$ for each element $B$ of $A$, and an embedding $g$ of $A$ onto
seg$\{\alpha\}$ for some ordinal $\alpha$, then replace each element
$(\{x\},B)$ of $\sum[ A]$ with
$\left<\left<T\{f_B(x)\},g(B)\right>\right>$ (the $T$ is actually
redundant, since $f_B(x)$ is a Cantorian ordinal\index{Cantorian!ordinal} in
each case) to get a set of ordinals of the same size; this is a set of strongly
Cantorian ordinals\index{ordinal numbers}, so strongly Cantorian; it is the
same size as $\sum[ A]$, so $\sum[ A]$ is strongly Cantorian.  The proof of the
theorem is complete.
\finpreuve

\begin{thm}
 A ``class" $\{x \st \phi\}$ of Cantorian ordinals
 fails to be realized by a set of Cantorian\index{Cantorian!ordinal} ordinals
 exactly if there is a sentence which describes a ``one-to-one correspondence"
 between $\{x \st \phi\}$ and the proper class of Cantorian ordinals.
\end{thm}

\preuve\ By the Axiom of Small Ordinals\index{Axiom of Small Ordinals}, there
is a set $B$ whose ``intersection\index{intersection!of classes}" with the
small ordinals\index{ordinal numbers} is $\{x \st \phi\}$.  Define 
a function\index{function} $f$ on Ord by
transfinite\index{transfinite!recursion} recursion which takes each 
$\beta$ to the smallest ordinal element of $B$ greater than all
$f(\gamma)$ for $\gamma < \beta$, or to $\vide$ if there is no such
element.  If some Cantorian ordinal is sent to a non-Cantorian ordinal
or to $\vide$, there is a smallest such ordinal $\alpha$ (the Axiom is
applied here), and $f[$seg$\{\alpha\}]$\index{segment} is a
Cantorian\index{Cantorian!set} set of Cantorian ordinals\index{ordinal
numbers}, certainly not in one-to-one correspondence with all
Cantorian\index{Cantorian!ordinal} ordinals (smaller than the
Cantorian\index{Cantorian!set} set ${\cal P}\{\seg\{a\}\}$).  If a set of
Cantorian ordinals could be placed in 
a ``one-to-one correspondence'' defined by a sentence with all the
Cantorian ordinals, the class of Cantorian ordinals would be a set by
the Replacement\index{Replacement(axiom of ZFC)} Theorem, which is impossible.
If each Cantorian ordinal is sent to a Cantorian ordinal, ``$f \lceil$
Cantorian ordinals" is the desired ``one-to-one correspondence" (observe that
$f(\alpha) \geq \alpha$ for any ordinal $\alpha$, so a non-Cantorian
ordinal will not be mapped to a Cantorian ordinal); in this event, no
set of Cantorian ordinals\index{ordinal numbers} can realize $\{x \st \phi\}$,
because we could then define the set of all Cantorian ordinals as $f^{-1}[\{x
\st \phi\}]$.
\finpreuve


\vspace{-.7\baselineskip}

\section[ZFC in the Cantorian Ordinals]{Interpreting ZFC in the Cantorian\\
Ordinals}\index{Zermelo--Fraenkel set theory}

\vspace{-.4\baselineskip}

The theorems we have been proving above will look familiar if
one has experience with conventional set theory.  We introduce the
usual set theory ZFC (Zermelo--Fraenkel set theory\index{Zermelo--Fraenkel set
theory}) by showing how to interpret it using the
Cantorian\index{Cantorian!ordinal} ordinals\index{ordinal numbers}, and stating
its axioms as theorems.

\subsection *{Construction}
Each Cantorian ordinal to be interpreted as a set of
Cantorian ordinals in such a way that every set of Cantorian ordinals
is interpreted by some ordinal.

We construct a relation\index{relation} $E$ on the ordinals\index{ordinal
numbers} by recursion.  We define a sequence\index{sequence!transfinite} of
ordinals $\rho_{\beta}$ indexed by the ordinals, by recursion: $\rho_0$ 
= 0, $\rho_{\beta+1}$ = the first ordinal $\gamma$ such that
$|\seg\{\gamma\}| = \exp(|\seg\{\beta\}|)$,
and $\rho_{\lim B} = \sup \{\rho_{\beta} \st \beta \in B\}$\index{segment},
where $B$ has no greatest element.
When we have defined the part $E_{\beta}$ of the relation $E$ with
domain\index{domain} $\seg\{\rho_{\beta}\}$, we want the following conditions:
$\rng(E_{\beta}) = \seg\{\rho_{\beta + 1}\} -
\{0\}$; for each subset\index{subset} $A$ of $\seg\{\rho_{\beta}\}$, there is
exactly one element $\alpha$ of
$\seg\{\rho_{\beta+1}\}$\index{segment} such that $\beta E \alpha$ iff $\beta
\in A$, for all $\beta$.  We prove by
transfinite\index{transfinite!induction} induction that this can be achieved.
It is certainly achievable for $\rho_{0}$.  Suppose it
has been achieved for $\rho_{\beta}$.  We show how to extend $E$ so that it is
achieved on $\rho_{\beta+1}$.  Let $F$ be a bijection between ${\cal
P}_1\{\seg\{\rho_{\beta+2}\}\} -
{\cal P}_1\{\seg\{\rho_{\beta+1}\}\}$ and ${\cal P}\{\seg\{\rho_{\beta+1}\}\} -
{\cal P}\{\seg\{\rho_{\beta}\}\}$ (it is straightforward
to show that the cardinalities\index{cardinal numbers} work out correctly); now
define $E$ as obtaining between $\alpha \in$
seg$\{\rho_{\beta+1}\}$\index{segment} and $\beta \in$
seg$\{\rho_{\beta+2}\}-$seg$\{\rho_{\beta+1}\}$ iff $\alpha \in F(\{\beta\})$.
Suppose it has been achieved for all $\rho_{\gamma}$ for $\gamma$ in a set $B$
with no greatest element; let $\beta = \lim B$; to achieve it for
$\rho_{\beta}$, we need a bijection $F$ between ${\cal
P}_1\{\seg\{\rho_{\beta+1}\}\} - {\cal P}_1\{\seg\{\rho_{\beta}\}\}$ and
${\cal P}\{\seg\{\rho_{\beta}\}\} - 
\bigcup[\{{\cal P}\{$seg$\{\rho_{\gamma}\}\} \st \gamma \in B\}]$, which we use
just as in the last case\index{segment}.


The relation $E$ is certainly defined for all
Cantorian\index{Cantorian!ordinal} ordinals\index{ordinal numbers}, and
determines a ``one-to-one correspondence" between Cantorian ordinals and sets
of Cantorian ordinals:  any Cantorian ordinal is associated with a
set\index{Cantorian!set} of smaller ordinals, which must thus 
be Cantorian; any set of Cantorian ordinals is bounded by a Cantorian
ordinal $\alpha$, and its ``code" is found below the first ordinal $\beta$ such
that $|\seg\{\beta\}| = \exp(|\seg\{\alpha\}|)$\index{segment}, which is
certainly Cantorian.


The following axioms are satisfied if $E$ is interpreted as
membership\index{membership} and the class of Cantorian ordinals\index{ordinal
numbers} is interpreted as the universe\index{universe, universal set} of
objects:

\begin{description}
 \item[\fdescr Extensionality\index{extensionality}:]  Sets with the same
   elements are the same.

   \preuve\ Obvious.
   \finpreuve

 \item[\fdescr Pairing:]  If $a,b$ are sets, $\{a,b\}$ is a set.

   \preuve\ $a$ and $b$ are actually Cantorian\index{Cantorian!ordinal}
   ordinals\index{ordinal numbers}; the actual set $\{a,b\}$
   is coded by an ordinal.
   \finpreuve

 \item[\fdescr Union\index{union!set}:]  If $A$ is a set, the set $\bigcup[A]$
   = $\{x \st$ for some $B, x \in B$ and $B \in A\}$ exists.

   \preuve\ If $\in$ is replaced by $E$ in the condition, we have a condition
   which defines a set of Cantorian ordinals\index{ordinal numbers}, which will
   be coded by an ordinal.
   \finpreuve

 \item[\fdescr Power Set\index{power set!axiom of ZFC}:]  If $A$ is a set, the
   power set ${\cal P}\{A\}$ = $\{x \st x \subseteq A\}$ exists. 

   \preuve Same as for Union.
   \finpreuve

 \item[\fdescr Infinity\index{infinity}:]  There is a set which contains
   $\vide$ and which contains $x \cup \{x\}$ whenever it contains $x$.

   \preuve\ Same as for Union.
   \finpreuve
\end{description}

These axioms allow us to build some specific sets.

\begin{description}
 \item[\fdescr Choice\index{axiom of choice}:]  Pairwise
   disjoint\index{disjoint} collections of nonempty sets have choice sets.

   \preuve\ An easy consequence of Choice in our theory.
   \finpreuve

 \item[\fdescr Foundation\index{Foundation (axiom of ZFC)}:] For each property
   $\phi$, (for all $A$, ((for all $x \in A, \phi$) implies $\phi[A/x]$))
   implies for all $x, \phi$. (i.e., if $\phi$ holding of all elements of a set
   implies that it hold of the set as well then $\phi$ holds for all sets).

   \preuve All elements of the set coded by an ordinal are smaller
   ordinals\index{ordinal numbers}; this follows by
   transfinite\index{transfinite!induction} induction. 
   \finpreuve
\end{description}

Choice\index{axiom of choice} and Foundation are ``structural" axioms.

Now we get to the meat of the difference between conventional
set theory and the theory we have presented.

\begin{description}
 \item[\fdescr Separation\index{Separation (axiom of ZFC)}:]  For any set $A$
   and sentence $\phi$ of the language of set theory, $\{x \in A \st \phi\}$
   exists.

   \preuve The ordinals\index{ordinal numbers} interpreting elements of $\{x
   \in A \st \phi\}$ are all less than the ordinal interpreting $A$.  The
   ordinals less than the ordinal interpreting $A$ make up a
   Cantorian\index{Cantorian!set} set, and any condition we can express on
   elements of a Cantorian set defines a set by a theorem in the previous
   subsection.
   \finpreuve
\end{description}

This is the particular restriction on the inconsistent axiom
of unrestricted comprehension\index{comprehension} which motivates this style
of set theory.  The intuitive idea is ``limitation of size"; the
``illegitimate" classes , like Russell's class, are very large.  The
previous axioms give us the material for Separation\index{Separation (axiom of
ZFC)} to work on (the particular sets $A$).  

Note that an arbitrary sentence $\phi$ about our interpreted
ZFC\index{Zermelo--Fraenkel set theory} is usually
unstratified\index{stratification}, even though $E$ is a set relation,
because it is likely to involve quantification over all sets of
ZFC, which is interpreted as quantification over all
Cantorian\index{Cantorian!ordinal} ordinals\index{ordinal numbers}: the
definition of ``Cantorian ordinal'' is unstratified.
Thus, we cannot prove Separation in the simple way we proved Power
Set, Union\index{union!set} and Infinity\index{infinity} above.  A sentence
$\phi$ in which every quantifier is restricted to a set does translate to a
stratified\index{stratification} sentence.  The same comments apply to the
Axiom of Replacement\index{Replacement(axiom of ZFC)} 
discussed below.

\begin{thm}
 There is no universe\index{universe, universal set}.
\end{thm}

\preuve\ If the universe $V$ existed, we could find $\{x \in V \st$ not$(x \in
x)\}$ using the Axiom of Separation\index{Separation (axiom of ZFC)}.
\finpreuve

The set theory with the axioms given so far is the set theory
of Zermelo\index{Zermelo--Fraenkel set theory!axioms of}, which is slightly
stronger than our set theory without the Axiom of Small Ordinals\index{Axiom of
Small Ordinals} (or Counting\index{Axiom of Counting}).  The full conventional
set theory ZFC\index{Zermelo--Fraenkel set theory} is completed with the

\begin{description}
 \item[\fdescr Axiom of Replacement\index{Replacement(axiom of ZFC)}:]  If
   $\phi$ is a sentence and $A$ is a set, and we have ``for each $x \in A$, for
   exactly one $y$, $\phi$", then $\{y \st$ for some $x \in A, \phi\}$ 
   exists.

   \preuve\ Use our Replacement Theorem above (p.~\pageref{repl-thm}).
   \finpreuve
\end{description}

We have shown that our full set theory is at least as strong as the
usual set theory ZFC\index{Zermelo--Fraenkel set theory}; in fact, it is much
stronger.
Preliminary results along these lines will be established below; the
full results are stated and their proofs sketched, but they are really
beyond the scope of this book.  The ``universe\index{universe, universal set}"
of ZFC is 
built in stages, like our model of it in the Cantorian\index{Cantorian!ordinal}
ordinals\index{ordinal numbers}: one starts with the empty set\index{empty
set}, then builds successive power sets\index{power set} indexed 
by the ordinals (the von Neumann\index{von Neumann ordinal} ordinals are used;
each ordinal is identified with the set of smaller ordinals) to get larger and
larger sets as needed.  Definitions of mathematical objects using equivalence
classes\index{equivalence relations, equivalence classes} on ``large"
relations, such as we have used for various kinds 
of numbers here, do not work in ZFC\index{Zermelo--Fraenkel set theory} because
the ``large" sets do not exist.  This is sometimes evaded in ZFC by
allowing ``large" collections as ``classes" which have elements but are not
elements.  We believe that the set theory we present here is better in
its ability to work with intuitively appealing ``large" collections.
The additional discipline of attention to stratification\index{stratification}
(usually involving references to singletons\index{singleton}, singleton
images\index{singleton image} or $T$ operators)
seems not to be that hard (in fact, it has been claimed that
stratification is the rule in mathematical practice, mod systematic
confusion of objects with their singletons\index{singleton}) and to present the
advantage of permitting access to a larger mathematical world.  A
suitable redefinition of the membership\index{membership} relation by
permutation will actually cause the membership relation $E$ of our model to
coincide with the ``real" membership $\in$ in a corner of our
universe\index{universe, universal set} which can then be taken to be the
universe of ZFC\index{Zermelo--Fraenkel set theory}.

\Exercises


\begin{enumerate}
 \item  Verify the existence of the bijections whose existence is required for
   the Construction in the chapter.

 \item  (hard) The map $F$ taking each ordinal $\alpha$ in $\Dom(E) \cap
   T[\Ord]$ (where $E$ is the simulated membership\index{membership}
   relation in the ordinals\index{ordinal numbers}) to 
   $$
    \{\beta \st \beta\mathrel{E} T^{-1}\{\alpha\}\}
   $$
   has a stratified\index{stratification} definition.
   The use of $T^{-1}$ is necessary to preserve stratification; note that
   $T^{-1}$ fixes all Cantorian\index{Cantorian!ordinal} ordinals, and that all
   Cantorian ordinals are in the domain\index{domain} of $F$.  Define a
   permutation $\pi$ as interchanging $\alpha$ and $F(\alpha)$ for all $\alpha
   \in \Dom(E) \cap T[\Ord]$ and fixing all other objects.  Define a
   new ``membership relation'' $x \in^{\pi} y$ as $x \in \pi(y)$.  Demonstrate
   that all axioms of our theory hold if $\in^{\pi}$ replaces $\in$, and that
   the axioms of ZFC\index{Zermelo--Fraenkel set theory} hold on the
   original domain of Cantorian ordinals\index{ordinal numbers} with the new
   membership\index{membership} relation.
\end{enumerate}
