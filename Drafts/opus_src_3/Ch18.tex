\chapter{Sets of Real Numbers}\index{real numbers!sets of}

In this section, we resume our treatment of the familiar systems of
rational\index{rational numbers} and real numbers.

\section{Intervals, Density, and Completeness}

We first consider a special property of the total order\index{order (linear)}
defined on the rational and real numbers.
\index{order (total)|see {order (linear)}}

\begin{definition}
 A linear order\index{order (linear)}\index{order (linear)!dense|textbf} $\leq$
 is said to be {\upshape dense} 
 iff for every pair $a < b$ of elements of $\dom(\leq)$, there is an
 element $c$ of $\dom(\leq)$ such that $a < c < b$.
\end{definition}

We also state a definition of sets important to consider in connection
with linear orders:

\begin{definition}
 Let $\leq$ be a linear order.  For any pair of
 points $a < b$ in $\dom(\leq)$, we define $[a,b]$ as $\{c \st a \leq c
 \leq b\}$.  We call this the {\upshape closed interval} determined by $a$
 and $b$.  We define $(a,b)$, the corresponding {\upshape open interval},
 as $[a,b]-\{a\}-\{b\}$.  We define $[a,b)$ and $(a,b]$ as
 $[a,b]-\{b\}$ and $[a,b]-\{a\}$, respectively.  We allow the use of
 the symbols $\infty$ and $-\infty$ as upper or lower limits:
 $[a,\infty)$ = $\{c \st c \geq a\}$, for example. The subsets\index{subset}
 of $\dom(\leq)$ described in this definition are called {\upshape
 intervals}, including the ones with open ``infinite\index{infinite!limits (of
 an interval)}'' upper and lower limits.
\end{definition}

\begin{ThmEtc}{Observation.}
 For any $a < b$ in the domain\index{domain} of a dense linear
 order\index{order (linear)} $\leq$, the interval $(a,b)$ is an infinite set.
\end{ThmEtc}

The linear order\index{order (linear)} on the rationals\index{rational
numbers!the order on} (and on the reals) is easily seen to 
be dense.  We see in the next theorem that the order on the rationals
is very special (indeed, in a sense, unique):

\begin{thm}
 Let $\leq$ be a dense linear order with a countable
 domain\index{domain} and without a greatest or least element.  Then there is a
 bijection\index{bijection} $f$ from $\dom(\leq)$ onto $\cal Q$ such that $f(x)
 \leq f(y)$ (the order here is the usual order on the rationals) iff $x \leq
 y$.  In other words, all such orders have the same structure as the
 order on the rationals.
\end{thm}

\preuve\ Choose a sequence\index{sequence} $s$ which is a bijection from $\cal
N$ onto dom$(\leq)$.  Choose a sequence $q$ which is a bijection from
$\cal N$ onto $\cal Q$.  We construct the map $f$ in stages.  At stage
0, we set $f(s_0) = q_0$.  At each subsequent stage, we will define up
to two values of $f$; we guarantee that at least $f(s_i)$ for each $i
\leq n$ and $f^{-1}(q_i)$ for each $i \leq n$ will be defined at the end of
stage $n$.  A finite number of values of $f$ will have been determined at each
stage.  At stage $n+1$, we check whether $f(s_{n+1})$ has been defined; if it
has not, we define it as $q_j$, where $j$ is the smallest index such that
$f^{-1}(q_j)$ has not already been determined and $q_j$ has the same order
relations to each $f(s_i)$ already defined that $s_{n+1}$ has to the
corresponding $s_i$ (there will be such a $q_j$ because the order is dense with
no largest or smallest element and there are only finitely many $s_i$'s to
consider).  Then check whether $f^{-1}(q_{n+1})$ has been defined; if it has
not, define it as $s_k$, where $k$ is the smallest index such that $f(s_k)$ has
not already been determined and $s_k$ has the same order relations to each
$f^{-1}(q_i)$ already defined that $q_{n+1}$ has to the corresponding $q_i$'s
(the reasons that this works have already been stated).  Observe that the
conditions which are to hold at each stage will hold at stage $n+1$ if they
held at stage $n$.  Induction shows that each stage can be carried out, giving
a complete definition of a function $f$ with the desired properties.  This kind
of proof is called a ``back-and-forth argument''. 
\finpreuve

The form of this theorem inspires a

\begin{definition}
 Two linear orders\index{order (linear)} $\leq$ and $\leq^*$ are said to
 be {\upshape isomorphic} iff there is a bijection\index{bijection} $f$ from
 $\dom(\leq)$ onto $\dom(\leq^*)$ such that $x \leq y$ iff $f(x) \leq^* f(y)$.
 It is easy to see that isomorphism is an equivalence
 relation\index{equivalence relations, equivalence classes} on linear 
 orders.  The bijection $f$ is said to be an {\upshape isomorphism}.
\end{definition}

\begin{ThmEtc}{Observation.}
 The previous theorem can be restated as ``All
 dense linear orders\index{order (linear)!dense} without endpoints and with
 countable\index{countable} domains\index{domain} are isomorphic''.
\end{ThmEtc}

The order on the real\index{real numbers!the order on} numbers is dense, and
also complete in a 
stronger sense.  The order on the rationals\index{rational numbers!the order
on} has gaps in it in a sense in which the order on the reals does not.

\begin{definition}
 A subset $A$ of the domain\index{domain} of a linear order\index{order
 (linear)} $\leq$ is said to be {\upshape convex} iff for each $a,b \in A$ such
 that $a<b$, $(a,b) \subset A$.
\end{definition}

\begin{Thm}{Interval Classification Theorem}
 Each nonempty subset of $\cal
 R$ convex with respect to the usual order is an interval.
\end{Thm}

\preuve\ If the set has no upper bound\index{bound!upper}, the upper limit of
the interval will be $\infty$; if the set has no lower
bound\index{bound!lower}, the lower 
limit of the interval will be $-\infty$.  If the interval has an upper
bound, take the upper limit of the interval to be the least upper
bound\index{bound!least upper} of the set, taking the interval to be closed or
open at the upper endpoint as appropriate.  If the interval has a lower bound,
take the lower limit of the interval to be the greatest lower
bound\index{bound!greatest lower} of 
the set, taking the interval to be open or closed at the lower
endpoint as appropriate.
\finpreuve

\begin{definition}
 A linear order\index{order (linear)!complete|textbf} with respect to which
 every convex set is an interval is said to be {\upshape complete}.
\end{definition}

This theorem can be understood as telling us that we cannot cut the
real\index{real numbers} line in a way which misses any real number; for surely
the sets above and below a cut would be convex!  The subset $\{p \in {\cal Q}
\st p^2 < 2\}$ is an example of a convex subset of the rationals\index{rational
numbers!sets of} with respect to the usual order which is not an interval.

We now propose a further extension of our use of the word {\itshape
dense\/}:

\begin{definition}
 A subset\index{subset!dense} $D$ of the domain of a linear order $\leq$
 is said to be {\upshape dense} if for every pair $a < b$ in $\dom(\leq)$, the
 interval $(a,b)$ contains an element of $D$.
\end{definition}

For example, ${\cal Q}$ is a dense subset of $\cal R$ with respect to
the usual ordering on these sets (and with respect to the usual
identification of $\cal Q$ with a certain subset of $\cal R$).

We now give a characterization of the structure of the usual order on
$\cal R$:

\begin{thm}
 Any complete dense linear order\index{order (linear)} $\leq$ with no
 endpoints and a countable\index{countable} dense subset\index{subset!countable
 dense} $D$ is isomorphic to the usual linear order\linebreak on $\cal R$.
\end{thm}

\preuve\ Choose an isomorphism $f$ between the order $\leq$
restricted to $D$ and the usual order on the rationals\index{rational
numbers!the order on}.  Each element 
$x$ of $\dom(\leq)-D$ uniquely determines the partition\index{partition} of $D$
into subsets $(-\infty,x)\cap D$ and $(x,\infty) \cap D$ (any $y \neq x$
has a different relation to elements of the nonempty set $(x,y)\cap D$
than $x$).  This determines a partition of the rationals $\cal Q$ into
two convex sets via the isomorphism $f$, which in turn determines a
unique real\index{real numbers} number $r$ between the two sets.  We define
$f(x) = r$. This process is reversible; we can determine $f^{-1}(r)$ for each
$r \in {\cal R}$ in exactly the same way.
\finpreuve




\section[Topology of ${\cal R}$]{Topology of \mathversion{bold}${\cal R}$}

We introduce elementary concepts of topology of the real\index{real numbers!as
a line} line.

\begin{definition}
 A subset of $\cal R$ is said to be {\upshape open\/} iff
 it is the union of some set of open intervals.  Equivalently, a set $A
 \subseteq {\cal R}$ is open iff for each element $x$ of $A$, there is
 an open interval $(y,z) \subseteq A$ of which $x$ is an element.  Such
 an interval $(y,z)$ is said to be a {\em neighborhood\/} of $x$; any
 open set containing $x$ may also be called a neighborhood of $x$.
\end{definition}

\begin{definition}
 A set $A \subseteq {\cal R}$ is said to be {\upshape
 closed\/} iff ${\cal R}-A$ is open.
\end{definition}

We prove some easy theorems.

\begin{thm}
 For any set $\cal O$ of open sets, $\bigcup [{\cal
 O}]$ is an open set.
\end{thm}

\preuve\ The union\index{union!set} of a collection of unions of open intervals
is still the union of a collection of open intervals.
\finpreuve

\begin{thm}
 For any finite\index{finite!set} set $\cal A$ of open sets, $\bigcap
 [{\cal A}]$ is an open set.
\end{thm}

\preuve\ For each element $x$ of $\bigcap[{\cal A}]$, we need to
find an open interval $(y,z)\subseteq\bigcap[{\cal A}]$ containing
$x$.  To do this, choose an open interval $(y_i,z_i)$ containing $x$
from each element $A_i$ of $\cal A$; the intersection\index{intersection!set}
of this finite collection of open intervals containing $x$ will be an open
interval containing $x$ and also a subset of $\bigcap[{\cal A}]$.
\finpreuve

\begin{thm}
 For any set $\cal C$ of closed sets, $\bigcap[{\cal
 C}]$ is a closed set.
\end{thm}

\preuve\ We need to show that ${\cal R} - \bigcap[{\cal C}]$ is
open.  But this is immediate: another way of writing this set is
$\bigcup [\{{\cal R}-C \st C \in {\cal C}\}]$, which is the union of
a collection of open sets, so open.
\finpreuve

\begin{thm}
 The union of any finite\index{finite!set} collection $\cal B$ of closed
 sets is closed.
\end{thm}

\preuve\ This is proved in the same way as the previous theorem;
the complement\index{complement} of the union of a finite collection of closed
sets is the intersection\index{intersection!set} of the finite collection of
(open) complements of those closed sets.
\finpreuve

We define a new concept.

\begin{definition}
 A real\index{real numbers} number $r$ is a {\upshape limit point} of a set
 $A$ iff each open interval containing $r$ meets $A$.  The {\upshape
 closure} of a set $A$ is the set of all limit points of $A$.  A
 variation on this definition: a real number $r$ is an {\upshape
 accumulation point} of a set $A$ iff each open interval containing
 $r$ meets $A - \{r\}$.  A limit point of $A$ which is not an
 accumulation point of $A$ must be an element of $A$ and is said to be
 an {\upshape isolated point} of $A$.
\end{definition}

\begin{thm}
 A set is closed iff it contains all of its limit
 points.  A set is closed iff it is equal to its closure.  The closure
 of any set is closed.
\end{thm}

\preuve\ Suppose that the set $C$ is closed.  Each point in $C$
is certainly a limit point of $C$.  Each point $y$ in ${\cal R} - C$
has an open interval containing it which lies entirely in ${\cal
R}-C$, because the complement\index{complement} of $C$ is open.  This implies
that no point not in $C$ is a limit point of $C$.  It follows that the set of
limit points of $C$ coincides with $C$.

Suppose that a set $L$ contains all of its limit points.  It follows
that each element $y$ of ${\cal R}-L$ belongs to an open interval not
meeting $L$ (because it is not a limit point of $L$).  This is exactly what it means for ${\cal R}-L$ to be open, and for $L$ to be closed!

Suppose that $C$ is the closure of a set $A$.  A limit point of $C$ is
a point any open interval about which will meet a point $x$ of $C$.
We know that any open interval containing $x \in C$ will meet $A$
(because $C$ is the closure of $A$).  Thus any limit point of $C$ is
also an element of $C$ (a limit point of $A$), and we have already
seen that a set which contains all of its own limit points is closed.

The proof of the theorem is complete.
\finpreuve

We give a result depending on the existence of the countable\index{countable}
dense subset\index{subset!countable dense} $\cal Q$ of $\cal R$.

\begin{thm}
 Any system $\cal O$ of pairwise disjoint\index{disjoint} open sets
 (or, more specifically, open intervals) is at most countable\index{countable}.
\end{thm}

\preuve\ Each element of $\cal O$ contains an open interval as a
subset, which in turn contains a rational\index{rational numbers} number as an
element. 
Choose one rational number to associate with each element of $\cal O$
(this does not require the Axiom of Choice\index{axiom of choice}; one could
choose the first rational number falling in $\cal O$ from a fixed
sequence\index{sequence} containing all the rationals).  A different rational
must be associated with each element of $\cal O$ because the elements of $\cal
O$ are disjoint\index{disjoint}; $\cal O$ can be no larger than the collection
of such rationals, and so no more than countable\index{countable}.
(Stratification\index{stratification} would require us to associate the
singleton\index{singleton} of a rational element with each element of 
$\cal O$, but the collection ${\cal P}_1\{{\cal Q}\}$ is countable as
well).
\finpreuve

Here is an important theorem depending on the least upper bound property\index{Least Upper Bound Property}:

\begin{definition}
 A sequence\index{sequence} $A$ of sets is said to be {\upshape nested}
 if $A_{i+1} \subseteq A_i$ for\linebreak each $i$.
\end{definition}

\begin{thm}
 A nested sequence of closed intervals $[r_i,s_i]$ has
 nonempty intersection\index{intersection!of nested intervals}.
\end{thm}

\preuve\ Consider sup $\{r_i \st i \in {\cal N}\}$.  This is
clearly greater than or equal to each $r_i$, and also less than or
equal to each $s_i$, and so belongs to each interval.
\finpreuve

We introduce another family of concepts:

\begin{thm}
 Any infinite subset $A$ of a closed interval $[r,s]$ has
 an accumulation point.
\end{thm}

\preuve We construct a sequence\index{sequence} of intervals $[r_i,s_i]$ as
follows:
\begin{itemize}
 \item $[r_0,s_0]$ = $[r,s]$;
 \item $[r_{i+1},s_{i+1}]$ =
   $[r_i,\frac{r_i+s_i}2]$ if this interval contains infinitely many
   elements of $A$;
 \item otherwise $[\frac{r_i+s_i}2,s_i]$
\end{itemize}
(this is a
definition by recursion\index{recursion}).
A simple
induction\index{induction!mathematical} shows that each $[r_i,s_i]$
contains infinitely many elements of $A$.  The diameter of $[r_i,s_i]$
will be $\frac{s-r}{2^i}$.  The intersection\index{intersection!of
nested intervals} of all the $r_i$'s is nonempty (by the previous
theorem) and contains no more than one point (because the diameters of
the $[r_i,s_i]$'s converge to zero).  Call this point $p$.  Any open
interval containing $p$ contains some entire interval $[r_i,s_i]$ (one
can take $i$ sufficiently large that the diameter of $[r_i,s_i]$ will
be smaller than the distance from $p$ to either endpoint of the open
interval about $p$), so contains points of $A$ distinct from $p$
itself; thus $p$ is an accumulation point of $A$ in $[r,s]$.  The
proof of the theorem is complete.
\finpreuve

\begin{definition}
 If $A$ is a subset of $\cal R$ and $\cal O$ is a
 set of open subsets of $\cal R$, $\cal O$ is said to be an {\upshape open
 cover} of $A$ iff $A \subseteq \bigcup[{\cal O}]$.
\end{definition}

\begin{definition}
 A subset of the reals\index{real numbers!sets of} is said to be {\em compact}
 if it is a closed subset of a closed interval $[r,s]$ (i.e., it is
 closed and bounded above and below).
\end{definition}

\begin{thm}
 A subset $K$ of the reals is compact iff each open
 cover of $K$ has a finite\index{finite!subcover} subcover (i.e., a finite
 subset which is also an open cover).
\end{thm}

\preuve\ If $K$ is not bounded, the set of intervals $(r,r+1)$
will provide an open cover of $K$ with no finite subcover.  If $K$ is
not closed, choose a limit point $p$ of $K$ which is not an element of
$K$, and the collection of all complements\index{complement} of closed
intervals centered at $p$ will provide an open cover of $K$ with no finite
subcover.

Now suppose that $K$ is closed and bounded.  Any open cover of $K$ has
a countable\index{countable} subcover; it suffices to choose one element of the
cover containing each open interval with rational\index{rational numbers}
endpoints that is a subset of some element of the cover.  Thus every open cover
of $K$ is the range of a sequence\index{sequence} of open sets $O$. Suppose
that $O$ is a sequence of open sets whose range has no finite subcover; we can
require further that no $O_i \cap K$ is covered by the sets $O_j \cap K$ for
$j < i$ (we can eliminate redundant sets $O_i$).  Choose a sequence of
points $p_i \in K$ such that $p_i \in O_i$ for each $i$ (note the
failure of stratification\index{stratification}, which is all right given the
Axiom of Counting\index{Axiom of Counting}) and $p_i \not\in O_j$ for any
$j<i$.  The sequence $p$ must have an accumulation point $q$ (because $K$ is a
subset of a closed interval), which must be an element of $K$ (because $K$ is
closed). $q$ cannot belong to any open set of the sequence $O$, because if it
did, so would infinitely many $p_i$'s, which is impossible by the
construction of the sequence of $p_i$'s.  But the range of $O$
purports to be a cover of $K$.  This contradiction completes the proof
of the theorem.
\finpreuve

We return to the distinction between an accumulation point and a mere
limit point.

\begin{definition}
 Let $A \subseteq {\cal R}$.  We say that $A$ is
 {\upshape perfect} iff $A$ is closed, nonempty, and has no isolated
 points.
\end{definition}

\begin{thm}
 The cardinality\index{cardinal numbers} of a perfect set is $c =
 2^{\aleph_0}$, the cardinality of the continuum.
\end{thm}

\preuve\ Let $P$ be a perfect set.  $|P| \leq c = |{\cal R}|$ is
immediate; we need to establish $|P| \geq c$.

We do this by defining an injection\index{one-to-one map} from $[{\cal N}
\rightarrow \{0,1\}]$, a set known to have cardinality $c$, to a subset of
$\cal P$.

$P$ contains at least one point $x$, because it is nonempty.  It
contains at least one other point $y$, because any open interval about
$x$ contains a point of $P$ other than $x$.  We associate with each of
$x$, $y$ an open interval in such a way that the corresponding closed
intervals do not intersect.  We now consider each of the points $x$
and $y$ and repeat this process: for example, $y$ is in an interval
$(u,v)$ chosen at the previous step; since $P$ is perfect, there is
another real\index{real numbers} number $y'$ in $(u,v)\cap P$; choose
disjoint\index{disjoint!intervals} open 
intervals\linebreak

\pagebreak

about $y$ and $y'$ in such a way that the corresponding
closed intervals do not intersect.

In this way we obtain a countable\index{countable} set $F$ with the following
properties: it is the union of finite\index{finite!set} sets $F_i$ for $i \in
{\cal N}$ 
such that $F_0$ = $\{x\}$, $|F_i|$ = $2^i$, each set $F_i$ is
associated with a set $O_i$ of open intervals such that the
corresponding closed intervals are pairwise disjoint\index{disjoint!intervals}
and each element 
of $O_i$ contains exactly one element of $F_i$, $F_i \subset F_{i+1}$
and each element of $O_i$ contains two elements of $F_{i+1}$.  We may
further require that the radius of each interval in $O_i < 2^{-i}$.
We associate a finite\index{finite!sequence} sequence with each element of each
$F_i$ for $i>0$ as follows: with the two elements of $F_1$ we associate the
one-term sequence with 0 as its sole term (this goes with $x$) and the
one-term sequence\index{sequence} with 1 as its sole term (this goes with $y$).
The map on $F_{i+1}$ is determined by assigning to each element of $F_i$
the sequence obtained by adding one more value, a 0, to the sequence
associated with it at the previous step; the new element of each
interval in $O_i$ is assigned the extension of the finite sequence
associated with the old element of the same interval with one new
value, a 1.  Assignment of relative types\index{types (relative)} to the
various occurrences of $i$ in this discussion would require some use of the $T$
operation, which the Axiom of Counting\index{Axiom of Counting} allows us to
neglect. 

The closure of the set $F$ is a subset of $P$, because $P$ is perfect
and so closed.  Each element of $[{\cal N}\rightarrow \{0,1\}]$
(infinite\index{infinite!sequence} sequence of binary digits) is associated
with a unique element of the closure of $F$ and so of $P$ by considering the
intersection\index{intersection!of nested intervals} of the nested
sequence\index{sequence} of intervals from which points 
associated with finite\index{finite!segment, initial} initial
segments\index{segment} of the sequence are chosen. 
There are $c$ such sequences, so there are at least $c$ elements in
$P$.  The proof of the theorem is complete.
\finpreuve

We continue the development with a

\begin{definition}
 For each set $A \subseteq {\cal R}$, we define the
 {\upshape derivative} $A'$ of $A$ as the set of accumulation points of
 $A$.
\end{definition}

\begin{thm}
 The derivative of a closed set $C$ is a closed subset of $C$.
\end{thm}

\preuve\ The derivative is a set of limit points of $C$, so
certainly a subset of $C$.

It is sufficient to show that for any closed set $C$, any limit point
of accumulation points of $C$ is itself an accumulation point of $C$.
A limit point of accumulation points of $C$ is a point $c$ any open
interval about which contains an accumulation point $a$ of $C$.  If
$a=c$ we are done.  If $a\neq c$, choose an open interval about $a$
included in the interval already chosen about $c$ and not containing
$c$.  This interval must contain a point of $C$, which cannot be $c$,
so the original interval about $c$ contained such a point, so $c$ is
an accumulation point of $C$.  The proof of the theorem is complete.
\finpreuve

\begin{thm}
 The set of isolated points of any set of reals\index{real numbers!sets of} $A$
 is at most\linebreak countable\index{countable}.
\end{thm}

\preuve\ For each isolated point, we can choose an open interval
containing that point which contains no other point of $A$.  We can
further refine this by halving the radius of each chosen interval; the
resulting family of intervals will be pairwise
disjoint\index{disjoint!intervals}.  By a theorem above, no set of pairwise
disjoint open intervals can be more than countable.  The proof of the theorem
is complete.
\finpreuve

We prove a special case of the Continuum Hypothesis\index{Continuum
Hypothesis}:

\begin{thm}
 Each closed subset of $\cal R$ is either finite\index{finite!set},
 countable, or of cardinality\index{cardinal numbers} $c$.
\end{thm}

The full Continuum Hypothesis would be obtained if we could drop the
word ``closed''.  The theorem follows immediately from the following

\begin{lemme}
 Each closed subset $A$ of $\cal R$ is the union $C \cup
 P$ of an at most countable\index{countable} set $C$ and a set $P$ which is
 either perfect or empty.
\end{lemme}

This is of course a ``Lemma'' only for our purposes; it is a
substantial theorem in its own right!

{\sc Proof of the Lemma.~---} Let $\leq$ be any
well-ordering\index{well-orderings} of an uncountable\index{uncountable} 
set.  We can define a function\index{function} $f$ from dom$(\leq)$ to subsets
of $\cal R$ by transfinite\index{transfinite!recursion} recursion such that
$f(w) = A \cap \bigcap[\{(f(v))'\st v < w\}]$.  The first set in the resulting 
transfinite sequence\index{sequence!transfinite} of sets is $A$; the successor
of any set in the sequence is its derivative;
intersections\index{intersection!set} are taken at limits.  It is 
easy to see that each $f(w)$ is a closed subset of $A$.

If we choose the well-ordering\index{well-orderings} $\leq$ with large enough
domain\index{domain}, the 
process described by $f$ must terminate: each set that we construct is
a subset of the previous one (because all are closed), so the number
of steps we can go through is bounded by the number of disjoint\index{disjoint}
subsets we can take from $\cal R$.  Assume that we choose $\leq$ so
that the process terminates; thus, there is a $u$ such that we have
$f(u)=f(v)$ for all $v$ such that $u \leq v$.  We define $P$ = $f(u)$
and $C$ = $A - f(u)$.  $P=P'$, so by the definition of ``perfect'',
$P$ is either perfect or empty.

It only remains to show that $C$ is countable\index{countable}.  $C$ is
partitioned\index{partition} 
into disjoint sets $f(w)-f(w+1)$ for $w \in$ dom$(\leq)$.
$f(w)-f(w+1)$ is the set of isolated points of $f(w)$ for each $w$.
Let $J_n$ be a sequence\index{sequence} whose range is all of the open
intervals with rational\index{rational numbers} endpoints (this is a countable
set).  For each element $c$ of 
$C$, define $I_c$ as the first interval in the sequence $J$ such that
for some $w \in $ dom$(\leq)$, $I_c \cap f(w)$ = $\{c\}$.  Since there
is a set $f(w)-f(w+1)$ which contains $c$, and $c$ is an
isolated point of this $f(w)$, there is such an $I_c$.  It is
straightforward to show that no two points $c,d \in C$ can have
$I_c=I_d$, and so $C$ is at most countable.  It is worth noting that
this implies that there can be no more than countably many distinct
$f(w)$'s.  The proof of the Lemma is complete.
\finpreuve

{\sc Proof of Theorem.~---} A closed subset of $\cal R$ is seen by the
Lemma to be the union of a set which is either finite\index{finite!set} or
countable\index{countable} with a set which is either empty or has
cardinality\index{cardinal numbers} $c$; the union of two such sets is either
finite, countable, or has cardinality $c$.
\finpreuve

We reluctantly leave this fascinating area of application of set theory.



\section[Ultrafilters]{Ultrafilters. An Alternative Definition of 
\mathversion{bold}$\cal R$.
Nonstandard Analysis}

We outline an elegant alternative definition of the real\index{real
numbers!alternative definition outlined} numbers, 
which gives us an occasion to introduce the notions of {\itshape filter\/}
and {\itshape ultrafilter\/} (along with the dual notions of {\itshape ideal\/}
and {\itshape prime ideal\/}).  We go on to discuss the application of
ultrafilters to construction of ``nonstandard models'' of sets, and
sketch the application of this construction to analysis.

\begin{definition}
 Let $\leq$ be a partial order\index{order (partial)}.  A {\upshape
 filter\index{filter|textbf} on $X$} is a subset of $\dom(\leq)$ with the
 following properties:
 \begin{enumerate}
  \item  If $p \in F$ and $p \leq q$, $q \in F$.
  \item For any $p,q \in F$, there is $r \in F$ such that $r \leq p$ and $r
    \leq q$.
 \end{enumerate}

 Notice that $\dom(\leq)$ can be a filter under this definition, but
 only in case every pair of elements of $\dom(\leq)$ has a lower bound.
 If there are pairs of elements of $\dom(\leq)$ which do not have a
 lower bound (as will be the case for the orders we will consider),
 then filters on $\leq$ must be proper subsets of $\dom(\leq)$.

 An {\upshape ultrafilter\index{ultrafilter|textbf}} is a filter which has no
 proper superset which is a filter.

 An {\upshape ideal (prime ideal)} on $\leq$ is a filter (ultrafilter) on
 $\geq$.
\end{definition}

\begin{thm}
 For any filter $F$ on $\leq$, there is an ultrafilter $U$
 on $\leq$ which extends $F$, i.e., such that $F \subseteq U$.
\end{thm}

\preuve Consider the order\index{order (partial)} on filters on $\leq$
determined by inclusion\index{inclusion}.  The union of any chain of filters in
this order is a filter, so the conditions of Zorn's Lemma\index{Zorn's Lemma}
are satisfied and there is a maximal filter in this order.  The proof of the
theorem is complete.
\finpreuve

Our alternative definition of the real\index{real numbers!alternative definition outlined} numbers follows:

\begin{definition}
 A {\upshape real number} is an ultrafilter on the
 inclusion\index{inclusion} order\index{order (partial)} on the set of bounded
 closed intervals $[p,q] = \{x \st p \leq x \leq q\}$ for $p \leq q$ elements
 of $\cal Q$.  The real number corresponding to a rational\index{rational
 numbers} $p$ is simply the set of closed intervals that contain $p$. $\cal R$
 is the set of all real numbers.
\end{definition}

\begin{definition}
 Let $r, s$ be real numbers.  $r + s$ is the set of
 intervals $I$ such that there are intervals $J \in r$ and $K \in s$
 such that $I = \{p+q\st p\in J$ and $q\in K\}$.  The product $rs$ is
 defined analogously.  $r \leq s$ is said to hold iff there are
 intervals $J \in r$, $K \in s$ such that for all $p \in J$, $q \in K$,
 $p \leq q$.
\end{definition}

Verification that this definition ``works'' is left to the reader.
One way of understanding it is to think of the closed interval
elements of the ultrafilter as ``estimates'' $p \pm q$ ($p$ and $q$
rational, $q \geq 0$); this estimate denotes the interval $[p-q,p+q]$.
A real\index{real numbers} number is a maximal collection of rational estimates
of this kind which are consistent with one another.  A sufficient condition
for a filter on the inclusion\index{inclusion} order on the closed intervals to
be an ultrafilter is that it contain closed intervals of every positive
diameter; the existence of ultrafilters extending any given filter on
this order can be proved without an appeal to Zorn's Lemma\index{Zorn's Lemma}!

We outline the verification of the least upper bound property\index{Least Upper
Bound Property}. 
Consider a nonempty set $A \subset {\cal R}$ which is bounded above
and below.  Now consider the set $\bigcap[A]$ of closed intervals in
$\cal Q$ which belong to all elements of $A$.  The set
$\bigcap[\bigcap[A]]$ is the set of rational\index{rational numbers} numbers
which are ``between'' elements of $A$ in a suitable sense.  Choose closed
intervals of each positive diameter which contain an element of
$\bigcap[\bigcap[A]]$ and an upper bound of $\bigcap[\bigcap[A]]$; the
unique real number containing all of these intervals as elements will
be the least upper bound\index{bound!least upper} of $A$.  It is easy to deduce
from the least upper bound property for nonempty sets bounded below that the
least upper bound property holds for all nonempty sets.

It is usual to consider filters and ideals over the inclusion\index{inclusion}
order on families of sets.  There is a nice characterization of
ultrafilters\index{ultrafilter} on the family of nonempty subsets\index{subset}
of a set $X$:

\begin{thm}
 Let $U$ be a filter on the inclusion order on ${\cal
 P}\{X\}-\{\vide\}$ for some set $X$.  $U$ is an ultrafilter iff for each
 set $A\subseteq X$, either $A \in U$ or $X-A \in U$.
\end{thm}

\preuve\ Suppose that for some set $Y$, $Y \not\in U$ and $X-Y
\not\in U$.  We show that $U$ is not maximal.  Consider the set
$$
 U' =
 U \cup \{(A \cap Y) \cup Z \st A \in U \mbox{ and } Z \subseteq X\}.
$$
It is straightforward to verify that $U'$ is a filter on nonempty subsets of
$X$ properly extending $U$.  On the other hand, if $U$ does contain
each set or its relative complement\index{complement}, it is easy to
see that $U$ is maximal; if $Y \not\in U$, it follows that $X - Y \in
U$, so we cannot adjoin $Y$ to $U$ and still satisfy the definition of
a filter on the inclusion\index{inclusion} order on nonempty
subsets\index{subset} of $X$; there is no nonempty subset of $X$ below
a set and its relative complement!
\finpreuve

\begin{ThmEtc}{Observation.} For each element $x$ of a set $X$, the set of
 subsets of $X$ which contain $x$ as an element is an
 ultrafilter\index{ultrafilter} on the inclusion order on nonempty
 subsets\index{subset} of $X$.  Such an ultrafilter 
 is called a {\itshape principal\/} ultrafilter on $X$.
\end{ThmEtc}

We describe a way to make an infinite set $X$ look like a larger
object $X^*$ with the same properties (in a suitable sense) by
adjoining some ``nonstandard'' objects using an ultrafilter.

\begin{thm}
 Let $A$ be an infinite set.  There is a nonprincipal
 ultrafilter on the inclusion\index{inclusion} order on nonempty subsets of
 $A$.
\end{thm}

\preuve The set of sets $X$ such that $X-A$ is finite\index{finite!set} (these
are called {\itshape cofinite\index{cofinite}\/} subsets of $A$) makes up a
filter on the inclusion order on nonempty subsets of $A$.  This filter can be
extended to an ultrafilter\index{ultrafilter}, and this ultrafilter cannot be
principal: each element $a$ of $A$ fails to belong to the cofinite set
$\{a\}^c$.
\finpreuve

Let $A$ be an infinite strongly Cantorian\index{Cantorian, strongly!set} set
(the Axiom of Counting\index{Axiom of Counting} 
allows us to assert that there are such sets), and let $U$ be a
nonprincipal ultrafilter on the inclusion order on nonempty subsets of
$A$.  Let $X$ be an infinite set.  We make some definitions.

\begin{definition}
 We define $X^*_U$ (or $X^*$ where $U$ is
 understood) as a partition\index{partition} of $[A \rightarrow X]$.  The
 equivalence relation\index{equivalence relations, equivalence classes}
 $\sim_U$ which determines the partition is defined thus: $f \sim_U g$ iff $\{a
 \in A \st f(a)=g(a)\} \in U$.  It is straightforward to verify that $\sim_U$
 actually is an equivalence relation.  Each element $x$ of $X$ corresponds to
 an element $x^*$ of $X^*$, the equivalence class\index{equivalence relations,
 equivalence classes} of the constant function\index{function} on $A$ with 
 value $x$.  $x^*$ is two types\index{types (relative)} higher than $x$.  The
 set $X^*_U$ is called an {\upshape ultrapower}.
\end{definition}

\begin{definition}
 For any predicate $R[x_1,\ldots,x_n]$ of objects
 $x_i$ in $X$, we define a predicate $R^*_U[[f_1],\ldots,[f_n]]$ (we
 may omit $U$ where it is understood) of objects $[f_i]$ (equivalence
 classes\index{equivalence relations, equivalence classes} of functions $f_i$)
 in $X^*_U$ as holding iff $\{a \in A \st R[f_1(a),\ldots,f_n(a)]\}\in U$.
 Here is where the strongly Cantorian\index{Cantorian, strongly!set} 
 character of $A$ becomes significant; the variable $a$ does not need
 to be assigned a consistent relative type\index{types (relative)}, because it
 is restricted to a fixed strongly Cantorian set, so it is possible to define
 the set needed for the definition of $R^*$ even when $R$ is a predicate whose
 arguments have different relative types (such as $\in$).  For any
 sentence $\phi$, we can define a sentence $\phi^*$ by replacing all
 predicates $R$ with predicates $R^*$ and references to specific
 constants $c$ with $c^*$ (it is useful to suppose here that we have
 eliminated all complex names from $\phi$ using the theory of definite
 descriptions), and restricting all quantifiers to $X^*$.
\end{definition}

\begin{Thm}{\L o\'s's Theorem\index{Los's Theorem}} Suppose either that the set
$X$ whose ultrapower we consider is strongly Cantorian or that we
restrict all sentence variables to the stratified\index{stratification}
sentences.  For any sentence $\phi$ with free variables $a_1,\ldots,a_n$,
$\phi^*$ holds for specific values of its free variables $a_i$ iff $\{a \in A
\st \phi'\}\in U$, where $\phi'$ is obtained from $\phi$ by replacing free
variables $a_i$ with terms $a_i'(a)$ with $a_i' \in X^*$ representing
the value to be assigned to the free variable $a_i$.  A special case
of this is that $\phi$ holds iff $\phi^*$ holds in the special case
where there are no free variables, or in the case where all free
variables code standard elements of $X$ (are equivalence
classes\index{equivalence relations, equivalence classes} of 
constant functions\index{function}).
\end{Thm}

\preuve\ We proceed by induction\index{induction!structural} on the structure
of $\phi$.
\begin{enumerate}
 \item If $\phi$ is atomic, this follows by the definition of
   predicates $R^*$.
 \item Suppose that the result holds for $\psi$ and $\xi$.  That it
   holds for ``not $\psi$'' follows from the fact that $U$ contains a set
   iff it does not contain its complement\index{complement} relative to $A$.
   That it holds for ``$\psi$ and $\xi$'' follows from the fact that $U$ is
   closed under Boolean\index{Boolean algebra, operations}
   intersection\index{intersection!Boolean}.
 \item Suppose that the result holds for $\psi$.  We want to determine
   whether it holds for ``for some $x$, $\psi$''.  $x$ is free in $\psi$.
   If ``for some $x \in X^*$, $\psi^*$'' holds, then $\psi^*$ holds for a
   specific $x \in X^*$, so $\{a \in A \st \psi'\}\in U$ holds by
   inductive hypothesis (using that specific value of $x \in X^*$ as
   $x'$), and $\{a \in A \st ($for some $x$, $\psi)'\}\in U$ holds
   because a superset of an element of $U$ is an element of $U$. If $\{a
   \in A \st ($for some $x$, $\psi)'\}\in U$, we want to show that for
   some $x' \in X^*$, $\{a \in A \st \psi'\} \in U$.  A specific value
   of $x'$ which works is any function\index{function} which takes each $a$ in
   $\{a \in A \st ($for some $x$, $\psi)'\}$ to an $x$ such that $\psi$ and
   takes other objects to anything; we don't care what.  For such an $x'$, the
   sets $\{a \in A \st ($for some $x$, $\psi)'\}$ and $\{a \in A \st
   \psi'\}$ are the same set, which is known to be in $U$.  Caution is
   required here: the function $x'$ can be relied upon to exist only
   under restricted conditions: one that works is ``$X$ is strongly
   Cantorian"; another that works is ``$\phi$ is
   stratified\index{stratification}".\finpreuve
 \end{enumerate}

We usually expect that there will be ``new'' objects in an ultrapower
$X^*$: e.g., the identity function on $A$ will represent an element of
$A^*$ different from all analogues of standard elements of $A$ (it
differs from each constant function on a cofinite\index{cofinite} subset of
$A$).

We present applications to ``nonstandard analysis'', a way of defining
basic concepts of the calculus using infinitesimals\index{infinitesimals}.

Our observations on ``nonstandard analysis" will be brief.  Carry out
the construction above with $A = {\cal N}$. We can work with $X=V$
here because notions of analysis can be presented in
stratified\index{stratification} form. The objects in $V^*$ will be equivalence
classes\index{equivalence relations, equivalence classes} of
sequences\index{sequence} of elements of $V$.  Consider the equivalence class
of a sequence of positive real\index{real numbers} numbers converging to 0.  It
will be an element of $({\cal R^+})^*$ less than all elements of $({\cal
R}^+)^*$ 
corresponding to standard real numbers and greater than $0^*$; in
other words, it will be a positive infinitesimal\index{infinitesimals}.

Let $s$ be a sequence of real numbers.  We assert (as usual) that
$\lim s = L$ iff for each real number $\epsilon>0$, there is a
natural\index{natural number} 
number $N$ such that for each $i \geq N$, $|s_i-L| < \epsilon$.

We present an equivalent notion in terms of ultrapowers: let $s^*$ be
the sequence of real numbers corresponding to $s$ in an ultrapower
$V^*_U$, $U$ an ultrafilter\index{ultrafilter} over the natural numbers.  We
claim that 
$\lim s = L$ iff for each nonstandard $i$ in ${\cal N}^*$, we have
$|s^*_i-L^*|$ infinitesimal\index{infinitesimals} (or zero).  The proof is
straightforward: 
if $\lim s = L$ is true, let $N = [t]$ be any nonstandard natural
number; there will be a standard function $M$ such that for any
$\epsilon > 0$, and natural number $k$, $k > M(\epsilon)$ implies
$|s_k - L| < \epsilon$; the set $\{i \st t_i > M(\epsilon)\}$ is in
$U$ for each (standard) $\epsilon > 0$, consequently the set $\{i \st
|s_{t_i} - L| <
\epsilon\}$ is in $U$ too, showing that $|s^*_N-L^*|$ is
infinitesimal\index{infinitesimals}. 
On the other hand, if $\lim s = L$ is not true, a strictly increasing
standard sequence\index{sequence} $t$ can be defined such that
$|s_{t_i}-L|>\epsilon$ 
for all $i$ with a {\itshape fixed\/} positive real\index{real numbers}
$\epsilon$; the element of the ultrafilter corresponding to the sequence $t$
will be a nonstandard integer $N$ such that $|s^*_N-L^*| > \epsilon^*$ for the
analogue $\epsilon^*$ of that fixed $\epsilon$.

This kind of reasoning can be extended to allow the presentation of
the basic notions of the calculus in terms of
infinitesimals\index{infinitesimals}.  For 
example, the derivative $f'$ of a function\index{function} $f$ from the reals
to the reals can be defined as the unique function $f'$ (if any) such that 
the absolute value of the difference between $f'^*(x)$ and
$\displaystyle\frac{f^*(x+dx)-f^*(x)}{dx}$ is
infinitesimal\index{infinitesimals} for each
nonzero infinitesimal $dx$ in ${\cal R}^*$.

\newpage

\Exercises

\begin{enumerate}
 \item (hard:  Baire\index{Baire category theorem} Category Theorem)  Prove
   that the intersection\index{intersection!set} of a
   countable\index{countable} collection of open dense
   subsets\index{subset!open dense} of the reals\index{real numbers!sets of} is
   dense.  We define a {\itshape nowhere dense\/} set as a set which is not
   dense in any interval.  Prove that the union of a countable collection of
   nowhere dense subsets\index{subset!nowhere dense} of the real\index{real
   numbers!as a line} line is not the entire real line. The union of a
   countable collection of nowhere dense subsets is called a {\itshape
   meager\/} set.

 \item (hard:  Borel\index{Borel sets} sets)  The set $\cal B$ of {\itshape
   Borel sets\/} is defined as the smallest subset of ${\cal R}$ which contains
   all intervals and is closed under unions and
   intersections\index{intersection!set} of its countable subcollections.  Show
   that all open sets and all closed sets are Borel sets.  Show that $|{\cal
   B}| = |{\cal R}|$ (so most sets of reals are not Borel).  A hierarchy of
   sets indexed by the natural\index{natural number} numbers can be obtained by
   starting with the open sets as stage 1, then adding all countable
   intersections of stage 1 sets to get stage 2 sets, all countable unions of
   stage 2 sets to get stage 3 sets, and so forth, taking countable
   intersections to get even stages and countable unions to get odd stages.
   Show that each of the stages contains new sets.  Show that there are Borel
   sets which are not constructed at stage $n$ of this process for any $n \in
   {\cal N}$.  

 \item  Find a book on nonstandard analysis and read the proofs of some
   familiar theorems of analysis.
\end{enumerate}
