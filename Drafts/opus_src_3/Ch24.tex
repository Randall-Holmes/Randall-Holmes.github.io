\chapter{Acknowledgements and Notes}\label{chap-notes}

I wish to acknowledge the help of Robert Solovay, who has shown me how
strong the system of this book is, and who has allowed me to include
his results in this area and one of his proofs.  I also wish to
acknowledge the patience of Tomek Bartoszynski, Alex Feldman, Steve
Grantham, and Marion Scheepers, who have listened to various talks
about this material in the logic seminar at Boise State University.
Thanks to G.~Aldo Antonelli for using an early draft of the book in a
graduate seminar at Yale and making useful remarks.  Last but by no
means least, I appreciate the careful and repeated reading of the text
by Paul West.

I especially acknowledge the assistance of my editor, Daniel
Dzierzgowski, and of the referees, who made many helpful suggestions.

References are excluded from the body of the text.  They may be found
here, organized by chapter.

The entire book owes a systematic debt to Halmos's {\itshape Naive Set
Theory\/} (\cite{halmos}).  The first draft was written by the author
in parallel with Halmos's book, with no thought of publication, as
training in carrying out elementary set theoretical constructions in
NFU\index{NFU}.  Of course, no work in this area can ignore Quine's
original paper \cite{quine}, in which ``New Foundations\index{New
Foundations}'' was introduced, or Jensen's paper \cite{jensen} in which
NFU was 
introduced and shown to be consistent.

We appreciate Rosser's {\itshape Logic for Mathematicians},
\cite{rosser}, the best full-length treatment of foundations of
mathematics based on NF (the only other one, Quine's
\cite{ml}, while interesting, contains some serious and distracting
errors), and Thomas Forster's {\itshape Set Theory with a Universal Set},
\cite{forster}, the only current book about this kind of set theory 
(with an exhaustive bibliography of the subject).

We acknowledge the support of the U. S. Army Research Office (grant
no.~DAAH04-94-0247); this grant supported the development of the Mark2
theorem prover alluded to in chapter 23, and chapter 23 is included to
provide guidance to users of the prover on the relationship between
the higher order logic of Mark2 and the system of the main part of the
book.

{\bf second edition:}  As we write the second edition in 2012, we wish to acknowledge the kindness of our publisher in allowing us to post and maintain 
an online version of this book in which we could conveniently provide corrections for errors in the original printed text, and by means of which we could continue to make the book available after it went out of print.  We also wish to thank readers who pointed out errata in the text.



\section*{Chapter 2}

The introduction of a distinction between sets and atoms\index{atoms} as a
primitive notion in the context of NFU\index{NFU} is due to Quine himself
in his note accompanying Jensen's original paper \cite{jensen} on
NFU (of course this distinction has been made in other set
theories!).  Jensen himself simply restricted
extensionality\index{extensionality} to 
nonempty sets.  This approach leaves us with no way to distinguish the
empty set\index{empty set} from the atoms.

{\bf second edition:}  We considered changing the notation for the empty set from $\vide$ to the more usual $\emptyset$, but decided that the notation we use is not unattested and maintains a certain uniformity in notation for finite sets.  



\section*{Chapters 3--7}

In these chapters, the aim is to develop a finite axiomatization of
NFU\index{NFU} $+$ Infinity\index{infinity} and prove that the full Stratified
Comprehension\index{Stratified Comprehension Theorem} Theorem follows from it.
The first finite\index{finite!axiomatization} axiomatization of NF, which can
be adapted to NFU, was due to Hailperin in
\cite{hailperin}; it is quite difficult to understand and use, because
Hailperin uses the Kuratowski ordered pair\index{ordered pair!Kuratowski},
whose projections\index{projections!relative type of}\index{types (relative)}
have type two lower than the type of the pair.  In NF, one can
{\itshape define\/} a type-level\index{type-level operation} pair (Quine did
this).  This cannot be done in NFU\index{NFU}, both because of the lack of
structure on atoms\index{atoms} 
and because the Axiom of Infinity\index{infinity} is not a theorem in
NFU as
it is in NF (the existence of a type-level ordered pair
implies that the universe\index{universe, universal set} is
infinite\index{infinite!size of the universe}).

{\bf second edition:}  In the second edition, we changed the notation $\SI\{R\}$ for the singleton image to the notation $R^{\iota}$ which we use elsewhere.  This note may be of use if we have inadvertantly left some instance of the older notation.  We also added the old-fashioned notation
$\iota$ for the singleton operation, recognizing its usefulness when one wants to iterate this operation.

We do know how to define a pair\index{ordered pair!another alternative definition} {\itshape one\/} type\index{types (relative)} higher than its
projections\index{projections!relative type of} in NFU without
infinity\index{infinity}: define $\left<x,y\right>$
as
$$
 \{\{x',0,1\}, \{x',2,3\},\{y',4,5\},\{y',6,7\}\st x' \in x, y'\in y\},
$$
where the objects denoted by 0--7 are any eight distinct objects
(iterated singletons\index{singleton!iterated} of the empty set\index{empty
set} would serve).  So far as we 
know, this definition is ours, but we would not be surprised if such
pairs had been defined earlier.  {\bf second edition:}  I preserve this note because the content is not without interest, but this definition does not work in NFU!

The approach of taking the type-level\index{type-level operation} ordered
pair\index{ordered pair} as a primitive is 
original with us, so far as we know.  We presented the finite
axiomatization described here in \cite{modernlogic}.  The use of
relation algebra and an ordered pair to give the expressive power of
first order logic is inspired by the book \cite{tarski-givant} of
Tarski and Givant.

A finite axiomatization of NFU\index{NFU} $+$ Infinity\index{infinity} of an
entirely different nature, also due to us, is described in the last chapter of
the book.

For the work of Bertrand Russell (the theory of descriptions and his
famous paradox\index{paradoxes!Russell}) we refer the reader to \cite{russell}.

The proof of the Stratified Comprehension\index{Stratified Comprehension
Theorem} Theorem given here follows 
ideas of Grishin found in \cite{grishin}.

{\bf second edition:}  We hope that the clarification of the translation of membership statements making explicit use of iterated singleton and singleton image notation is found useful.


\section*{Chapter 8}

We are entirely to blame for the speculations found in Chapter 8.  It
has come to our attention that David Lewis's approach to set theory in
his book \cite{lewis} is very similar in important respects.

\section*{Chapters 9--11}

The content of these chapters is motivated by Halmos's development in
\cite{halmos}.  This is somewhat masked by the fact that the
constructions here involve stratification\index{stratification} considerations
that Halmos did not have to deal with.  The natural introduction of the Axiom
of Choice\index{axiom of choice} to deal with the type\index{types (relative)}
differential between elements of partitions\index{partition} and
representatives of the elements seems serendipitous.  I 
use $f^{-1}$ in its usual sense, not in the sense in which Halmos uses
it.

We have been criticized for our eccentric use of parentheses, braces
and brackets, but those who live in glass houses should not throw
stones.  The notation used by current workers in set theory with
stratified comprehension\index{Stratified Comprehension Theorem} is peculiar
(and inconsistent from one 
worker to another).  Traditional notation derived from Russell's {\itshape
Principia Mathematica\/} (\cite{russell}) or from Rosser's {\itshape Logic
for Mathematicians\/} can be quite confusing to a newcomer to this
area.  Our aim has been to adopt a notation somewhat more similar to
that usually found in set theory; of course, the actual effect may be
only to multiply confusion!

Whether our particular notational conventions are adopted or not, we
do support the development of a common notation to be used by all who
work with this kind of set theory!

{\bf second edition:}  We have since abandoned the use of some of the odd features of the notation in this book, but it has a philosophical unity which we are still fond of, and we have preserved it in the second edition except for changing the notation for the singleton image operation.


\section*{Chapter 12}

The approach to the natural\index{natural number} numbers found here is
ultimately due to Frege, and may be found (adorned with types\index{types
(relative)}) in Russell's \cite{russell}.  The special feature that
Infinity\index{infinity} is deduced from the type-level\index{type-level
operation} nature of the ordered pair\index{ordered pair} is original with us
(if not especially laudable).

The Axiom of Counting\index{Axiom of Counting} was first introduced by Rosser
in \cite{rosser}, in the form of the theorem $|\{1,\ldots,n\}| = n$.  It has
been shown in by Orey in \cite{orey} to be independent of NF (if
NF is consistent).  It is known to strengthen NFU\index{NFU}
essentially.





\section*{Chapter 13}

The development of the reals\index{real numbers} here is very similar to that
of Quine in \cite{ml}.




\section*{Chapter 14}

The specific equivalents of the Axiom of Choice\index{axiom of choice}
proven here are motivated by Halmos's development.  Of course, it is
only because we are in NFU\index{NFU} that we can
well-order\index{well-orderings} the universe\index{universe,
universal set} (the alternatives being NF, with no
well-ordering, or ZFC\index{Zermelo--Fraenkel set theory}, with
no universe\index{universe, universal set})!




\section*{Chapters 15--16}

The development of the ordinal and cardinal\index{cardinal numbers} numbers in
this kind of 
set theory is entirely different from that in the usual\index{Zermelo--Fraenkel
set theory} set theory. 
The basic ideas are ultimately due to Frege and are found (typed\index{types
(relative)}) in 
\cite{russell}.  The description of how NF avoids the
Burali-Forti paradox\index{paradoxes!Burali-Forti} is found in the original
paper \cite{quine} of Quine.

The Axiom of Small Ordinals\index{Axiom of Small Ordinals} is an extravagance
due to the author.  The 
weaker but still very strong axiom that all Cantorian\index{Cantorian,
strongly!set} sets\index{ordinal numbers} are 
strongly Cantorian is due to C. Ward Henson in \cite{henson}.  The
precise determination of the strength of Henson's axiom (which also
applies as a lower bound to the strength of NF with Henson's
axiom) was made very recently by Robert Solovay (unpublished).  The
precise consistency strength of the full system of this book has not
been determined; its consistency follows from the existence of a
measurable cardinal\index{cardinal numbers}.

{\bf second edition:}  The author of the book demonstrated a precise equivalence of the strength
of the system of this book (formally called NFUM) with the strength of an extension of Kelly-Morse set theory;
very recently Zachiri McKenzie demonstrated its equiconsistency with a certain extension of ZFC with large cardinal axioms in his Ph.D. thesis.  [references to be supplied]

The theorems about cardinal numbers in chapter 16 are classical; the
development follows Halmos.



\section*{Chapter 17}

The definition of exponentiation\index{exponentiation!cardinal} is extended
from the original proposal of Specker in \cite{specker} following a suggestion
of Marcel Crabb\'e.  The definitions of infinite\index{infinite!sum} sums and
products\index{infinite!product} of cardinals\index{cardinal numbers} 
are the trickiest exercises in stratification\index{stratification} in the
whole book; we hope that the reader enjoys them (and their application in the
proof of the classical theorem of K\"onig\index{K\"onig's Theorem}).

{\bf second edition:}  The definitions of sums and products of indexed famlies of cardinals
in the printed version are wrong, and serve to confuse issues in reading the proof of K\"onig's Theorem (which presupposed the correct definition, though it was further injured by a typo).  These things have always been corrected in the online editions, and as far as possible the print edition has been accompanied by an errata slip (which you can get from the author's web page).

The original discussion of the resolution of Cantor's
Paradox\index{paradoxes!Cantor} in this 
kind of set theory is found in \cite{quine}.  
Specker's\index{Specker's Theorem} disproof of the Axiom of Choice\index{axiom
of choice} in {\em NF\/} (originally 
given in \cite{specker}) converts itself in the context of
NFU\index{NFU} 
with Choice to a proof of the existence of atoms\index{atoms}.  It loses none
of its surprising character.  The proof of the theorem is simplified by 
using the Axiom of Counting\index{Axiom of Counting}; we do not know whether
this form of the proof has been published by earlier workers.

The comments on
Cantorian\index{Cantorian, strongly!set} and strongly Cantorian sets are
standard in this field. 



\section*{Chapter 18}

The topological section is closely parallel to the discussion of this
subject in Hrbacek and Jech's excellent \cite{hrbacek} (except for any
defects, of course).

The very abstract definition of {\itshape filter\/} and related notions is
cribbed from Kunen's development of forcing in \cite{kunen}.  The
implementation of the reals\index{real numbers} as
ultrafilters\index{ultrafilter} over the closed intervals 
in the rationals\index{rational numbers} is not standard, but surely not novel.

The treatment of ultrapowers is standard, except for the remarks about
stratification\index{stratification} restrictions.

For nonstandard analysis one can see Abraham Robinson's \cite{nonstandard}.



\section*{Chapter 19}

All results in this section are due to the author.  Of course, the
idea of coding set theory into the ordinals\index{ordinal numbers} is not
original! 



\section*{Chapter 20}

All results in this section can be found in the work of Roland
Hinnion, except for the discussion of the Axiom of Endomorphism,
originally found in our \cite{afa}, and any discussion of the
consequences of our Axiom of Small Ordinals\index{Axiom of Small
Ordinals}.  See Hinnion's Ph.~D. thesis, \cite{hinnion}, and, for the
use of relations $[\sim^+]$ in the construction of extensional
relations, \cite{hinnion2}.  What we call a ``nice relation'' in the
proof of the Collapsing Lemma is called a ``bisimulation'' elsewhere.

{\bf second edition:}  Corrected the silly assertion that the union of a family of equivalence relations is an equivalence relation in
the proof of the Collapsing Lemma.  Of course what we meant to say was that the union of a chain (in inclusion) of equivalence relations is an equivalence relation.

\section*{Chapter 21}

This section is shamelessly parasitic on John Horton Conway's
wonderful {\itshape On Numbers and Games\/} (\cite{conway}).  It was
motivated by the desire to tackle a mathematical structure whose
original definition was clearly horribly unstratified\index{stratification}!



\section*{Chapter 22}

For the discussion of the Hartogs operation, I am indebted to Forster
(\cite{forster}).  All results are classical, except the major recent
result of Solovay (unpublished) that there are
inaccessible\index{inaccessible cardinals} cardinals,\index{cardinal
numbers} which he has graciously allowed us to present here.
Solovay's original argument is given entirely in terms of ZFC
with an external automorphism, rather than being framed in NFU
terms as we have essayed to do here.  This result is in a sense superseded
by the stronger result of Solovay that there are $n$-Mahlo cardinals
for each concrete $n$.  The Axiom of Large Ordinals\index{ordinal
numbers} and the discussion of the coding of proper classes of small
ordinals into our theory and the codings between our theory and second
order ZFC\index{Zermelo--Fraenkel set theory} with a measure on
classes are due to us (and any errors are our responsibility!)

{\bf second edition:}  We corrected the proof that cofinalities of cardinals are regular cardinals.



\section*{Chapter 23}

Any work in combinatory logic should cite H. B. Curry's masterful
\cite{curry}.

This section is based on our Ph.~D. thesis \cite{thesis} and the
subsequent paper \cite{thesispaper} of the same title.  It is further
discussed in our more recent \cite{stratification}.  The
interpretation of strongly Cantorian\index{Cantorian, strongly!set} sets as
data types is a perhaps silly idea of ours; it appeared in our
\cite{modernlogic}.  The adaptation of our ``metaphor'' to
functions\index{function} is also discussed there.

For more about the Mark2 automated reasoning system, see \cite{rewrite}.
