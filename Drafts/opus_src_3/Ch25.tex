\def\fThmEtc{\fontseries{m}\fontshape{n}\normalsize
\selectfont}
\appendix
\def\thechapter{}

\chapter[Appendix: Selected Axioms and Theorems]{Axioms and Selected Theorems
by\\ Chapter}
\markboth{Appendix: Selected Axioms and Theorems}{Appendix: Selected Axioms and
Theorems}

\section *{Chapter 2}

\begin{axiom}{Axiom of Extensionality\index{extensionality!axiom}}
 If $A$ and $B$ are sets, and for each $x$, $x$ is an
 element of $A$ if and only if $x$ is an element of $B$, then $A = B$.
\end{axiom}

\begin{axiom}{Axiom of Atoms\index{atoms!axiom}}
 If $x$ is an atom, then for all $y$, $y
 \not\in x$. (read ``$y$ is not an element of $x$'')
\end{axiom}



\section *{Chapter 3}

\begin{axiom}{Axiom of the Universal Set\index{universe, universal set!axiom}}
 $\{x \st x = x\}$, also called $V$, exists.
\end{axiom}

\begin{axiom}{Axiom of Complements\index{complement!axiom of}}
 For each set $A$, the
 set $A^c$ = $\{x \st x \not\in A\}$, called the {\upshape complement}
 of $A$, exists.
\end{axiom}

\begin{axiom}{Axiom of (Boolean\index{Boolean algebra, operations})
Unions\index{union!Boolean!axiom}}
 if $A$ and $B$ are sets, the set $A \cup B = 
 \{x \st x \in A$ or $x \in B$ or both$\}$, called the
 (Boolean) {\upshape union} of $A$ and $B$, exists.
\end{axiom}

\pagebreak
  
\begin{axiom}{Axiom of Set Union\index{union!set!axiom}}
 If $A$ is a set all of whose elements are sets, the
 set $\bigcup[A]$ = $\{x \st$ for some $B$, $x \in B$ and $B \in A\}$, called
 the (set) union of $A$, exists.
\end{axiom}


\vspace{-.6\baselineskip}

\section *{Chapter 4}


\begin{axiom}{Axiom of
Singletons\index{singleton!axiom}\index{singleton|textbf}}
 For every object $x$, the set $\{x\} = \{y \st y = x\}$
 exists, and is called the {\upshape singleton} of $x$.
\end{axiom}

\begin{axiom}{Axiom of Ordered Pairs\index{ordered pair!axiom}}
 For each $a, b$, the {\upshape ordered pair of $a$ and $b$},
 $(a,b)$, exists; $(a,b) = (c,d)$ exactly if $a = c$ and $b = d$.
\end{axiom}

\begin{axiom}{Axiom of Cartesian Products\index{Cartesian product!axiom}\index{Cartesian product|textbf}}
 For any sets $A,B$, the set $A \times B = \{x \st$
 for some $a$ and $b$, $ a \in A, b \in B,$ and $x = (a,b)\}$, briefly written
 $\{(a,b) \st a \in A$ and $b \in B\}$, called the {\upshape Cartesian product}
 of $A$ and $B$, exists.
\end{axiom}


\vspace{-.6\baselineskip}

\section *{Chapter 5}

\begin{axiom}{Axiom of Converses%
\index{converse (of a relation)!axiom}\index{converse (of a relation)|textbf}}
 For each relation $R$, the set $R^{-1} = \{(x,y) \st y R
 x\}$ exists; observe that $x \,R^{-1}\, y$ exactly if $y R x$.
\end{axiom}

\begin{axiom}{Axiom of Relative Products\index{relative product!axiom}\index{relative product|textbf}}
 If $R,S$ are relations\index{relation}, the set
 $$
   (R|S) =
   \{(x,y) \st \mbox{for some $z, x \rR z$ and $z \rS y$}\},
 $$
 called the {\upshape relative
 product} of $R$ and $S$, exists.
\end{axiom}


\begin{axiom}{Axiom of Domains\index{domain!axiom}\index{domain|textbf}}
 If $R$ is a relation, the set
 $$
  \dom(R) = \{x \st \mbox{for some $y$, $x \rR y$}\},
 $$
 called the {\upshape domain} of $R$, exists.
\end{axiom}

\begin{axiom}{Axiom of Singleton Images%
\index{singleton image!axiom}\index{singleton image|textbf}}
 For any relation $R$, the set
 $$
  R^{\iota} =
  \{(\{x\},\{y\}) \st x \rR y\},
 $$
 called the {\upshape singleton image} of $R$,
 exists.
\end{axiom}

\begin{axiom}{Axiom of the Diagonal\index{diagonal, axiom of the}}
 The set $[=] = \{(x,x) \st x \in V\}$ exists (this is
 the equality relation).
\end{axiom}

\begin{axiom}{Axiom of Projections\index{projections!axiom of}}
 The sets $\pi_1$ = $\{((x,y),x) \st x,y \in V\}$ and\linebreak $\pi_2$ =
 $\{((x,y),y) \st x,y \in V\}$ exist.
\end{axiom}



\section *{Chapter 6}

\begin{Thm}{Representation Theorem\index{Representation Theorem}}
 For any sentence $\phi$ in which all
 predicates mentioned in atomic sentences are realized, the set $\{x
 \st \phi\}$ exists.
\end{Thm}



\section *{Chapter 7}

\begin{axiom}{Axiom of Inclusion\index{inclusion!axiom of}}
 The set $[\subseteq] = \{(x,y) \st x \subseteq y\}$ exists.
\end{axiom}

\begin{Thm}{Stratified Comprehension Theorem\index{Stratified Comprehension
Theorem}} 
 For each stratified sentence $\phi$, the
 set $\{x \st \phi\}$ exists.
\end{Thm}




\section *{Chapter 9}

\begin{axiom}{Axiom of Choice\index{axiom of choice!introduced}%
\index{axiom of choice|textbf}}
 For each set $P$ of pairwise disjoint\index{disjoint} non-empty
 sets, there is a set $C$, called a {\upshape choice set} from $P$, which
 contains exactly one element of each element of $P$.
\end{axiom}




\section *{Chapter 12}

\begin{Thm}{Theorem of Infinity\index{infinity!theorem of}}
 $\vide$ is not a natural\index{natural number} number.
\end{Thm}

\begin{axiom}{Axiom of Counting\index{Axiom of Counting}}
 For all natural numbers $n$, $T\{n\} = 
 n$\index{$T$ operation!on natural numbers}. 
\end{axiom}

\begin{Thm}{Restricted Subversion Theorem\index{Restricted Subversion Theorem}}
 If a variable $x$ in a sentence $\phi$ is restricted to $\cal N$, then its
 type\index{types (relative)} can be freely raised and lowered; i.e., such a
 variable can safely be ignored in making type assignments for
 stratification\index{stratification}.
\end{Thm}



\section *{Chapter 14 (equivalents of the axiom of choice)}

\begin{thm}
 The Cartesian\index{Cartesian product} product of any family of non-empty sets
 is\linebreak non-empty.
\end{thm}

\begin{Thm}{Zorn's Lemma\index{Zorn's Lemma}}
 If $\leq$ is a partial order\index{order (partial)} such that each chain in
 $\leq$ has an upper bound\index{bound!upper} relative to $\leq$, then the
 domain of $\leq$ has a maximal element relative to $\leq$.
\end{Thm}

\begin{thm}
 There is a well-ordering of $V$.
\end{thm}





\section *{Chapter 15}

\begin{Thm}{Transfinite\index{transfinite!recursion} Recursion Theorem}
 Let $W$ be the domain\index{domain} of a well-ordering\index{well-orderings}
 $\leq$.  Let $F$ be a function\index{function} from the collection of
 functions with domains segments\index{segment} of $\leq$ to the collection of
 singletons\index{singleton}.  Then there is a unique function $f$ such that
 $\{f(a)\}$ = $F(f \lceil  \seg\{a\})$ for each $a \in W$.
\end{Thm}

\begin{thm}
 $T^2\{\Omega\} < \Omega$; i.e, $\leq$ on $\seg_{\leq}\{\Omega\}$ is not
 similar\index{similarity} to $\leq$ on all\linebreak
 ordinals\index{ordinal numbers}.
\end{thm}

\begin{axiom}{Axiom of Small Ordinals%
\index{Axiom of Small Ordinals!introduced}\index{Axiom of Small Ordinals|textbf}}
 For any sentence $\phi$ in the
 language of set theory, there is a set $A$ such that
 for all $x$, $x$ is a small ordinal such that $\phi$ iff ($x\in A$ and
 $x$ is a small ordinal).
\end{axiom}




\section *{Chapter 16}

\begin{Thm}{Theorem (Dedekind)}
 A set is infinite\index{infinite!set} exactly if it is
 equivalent\index{equivalence (of cardinality)} to 
 one of its proper subsets.
\end{Thm}

\begin{Thm}{Schr\"oder--Bernstein Theorem\index{Schr\"oder--Bernstein theorem}}
 If $X$ and $Y$ are sets, and $f$ is a 
 one-to-one\index{one-to-one map} map from $X$ into $Y$ and $g$ is a one-to-one
 map from $Y$ into $X$, 
 then $X$ and $Y$ are equivalent\index{equivalence (of cardinality)}.
\end{Thm}





\section *{Chapter 17}

\begin{Thm}{Theorem (Cantor)\index{Cantor's Theorem}}
 $T\{\kappa\} < \exp(T\{\kappa\})$\index{$T$ operation!on cardinals}.
\end{Thm}

\begin{Thm}{K\"onig's Theorem\index{K\"onig's Theorem}}
 Let $F$ and $G$ be functions\index{function} with the same
 nonempty domain\index{domain} $I$ all of whose values are
 cardinal\index{cardinal numbers} numbers.  Suppose 
 further that $F(x) < G(x)$ for all $x\in I$, where the order is the
 natural order on cardinals.  It follows that $\sum[F] < \prod[G]$.
\end{Thm}

\begin{Thm}{Theorem (Specker)\index{Specker's Theorem}}
 $|{\cal P}\{V\}| < |V|$, and thus there are atoms\index{atoms}.
\end{Thm}

\begin{Thm}{Subversion\index{Subversion Theorem} Theorem}
 If a variable $x$ in a sentence $\phi$ is restricted to a strongly
 Cantorian\index{Cantorian, strongly!set} set $A$, then its
 type\index{types (relative)} can be freely raised and lowered; i.e., 
 such a variable can safely be ignored in making type assignments for
 stratification\index{stratification}.
\end{Thm}





\section *{Chapter 20}

\begin{axiom}{Axiom of Endomorphism\index{Endomorphism, axiom of}}
 There is a one-to-one\index{one-to-one map} map $\Endo$ from
 ${\cal P}_1\{V\}$ (the set of singletons\index{singleton}) into ${\cal
 P}\{V\}$ (the set of sets) such that for any set $B$, $\Endo(\{B\}) =
 \{\Endo(\{A\})\st A \in B\}$.
\end{axiom}

The Axiom of Endomorphism is not an axiom of our theory, but it is shown to be
consistent with our theory in chapter~20.





\section *{Chapter 23}

\begin{Thm}{Theorem (Solovay)\index{Solovay's Theorem}}
 There is an inaccessible\index{inaccessible cardinals} cardinal\index{cardinal
 numbers}.
\end{Thm}

\begin{axiom}{Axiom of Large Ordinals}
 For each non-Cantorian ordinal\index{ordinal numbers}
 $\alpha$, there is a natural\index{natural number} number $n$ such that
 $T^n\{\Omega\} < \alpha$.
\end{axiom}
