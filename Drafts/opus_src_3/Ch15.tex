\chapter{Ordinal Numbers}


\section{Well-Orderings and Ordinals}\index{well-orderings}

Let $W$ be the domain\index{domain} of a well-ordering $\leq$; then we can
justify a method of proving statements about elements of $W$ somewhat analogous
to mathematical induction\index{induction!mathematical}:

\begin{Thm}{Theorem of Transfinite\index{transfinite!induction|textbf}
Induction}
  Let $W$ be the domain of a
  well-order\-ing $\leq$.  Let $A$ be a set with the property that for each
  element $a$ of $W$, $a \in A$ if $\seg_{\leq}\{a\}\index{segment} \subseteq
  A$; i.e, $A$ contains $a$ if it
  contains each element of $W$ less than $a$.  Then $W \subseteq A$; all
  elements of $W$ belong to $A$.
\end{Thm}

\preuve\ If the set $W - A$ is nonempty, then it has a least element $a$.
But if $a$ is the least element of $W - A$, then seg$\{a\} \subseteq A$, and
$a$ must be in $A$.  It follows that $W - A$ must be empty, and $W \subseteq
A$.
\finpreuve

The difference between conventional mathematical
induction\index{induction!mathematical} and 
this variety is that in conventional mathematical induction we
consider the immediate predecessor of an element to show that it is in
the set, while in transfinite induction we consider the set of all
predecessors.  This kind of induction can be used on the natural\index{natural
number} numbers as well; the greater generality is necessary because not all
elements of domains\index{domain} of general
well-orderings\index{well-orderings} have immediate 
predecessors.  The basis step of the usual mathematical induction
handles the problem that 0 does not have an immediate predecessor in
the natural numbers; the ``basis step" in a transfinite induction is
trivial, since the set of predecessors of the first element is the
empty set\index{empty set}, all elements of which have any
property\index{properties} one would care to 
propose.

We can also prove a recursion theorem:
\begin{Thm}{Transfinite\index{transfinite!recursion|textbf} Recursion Theorem}
 Let $W$ be the domain\index{domain} of a well-ordering\index{well-orderings}
 $\leq$.  Let $F$ be a function\index{function} from the collection of
 functions with domains segments\index{segment} of $\leq$ to the collection of
 singletons\index{singleton}.  Then there is a unique function $f$ such that
 $\{f(a)\} = F(f \lceil  \seg\{a\})$ for each $a \in W$.
\end{Thm}

\preuve\ By transfinite induction.  Let $A$ be the set of elements $a$ of $W$
such that there is a uniquely determined function $g$ on seg$^+\{b\}$
with the indicated property for each $b < a$.  We want to deduce that
there is a uniquely determined function $f$ on seg$^+\{a\}$\index{segment!weak}
with the indicated property.  The desired function is the union of the union
$g$ of all the functions on the seg$^+\{b\}$'s (which must agree with
one another by uniqueness) and $\{a\} \times F(g)$ (remember that
$F(g)$ is the singleton\index{singleton} of the desired value).  The function
on all of $W$ is the union of the functions on the weak segments.\linebreak
\finpreuve

The function\index{function} $F$ gives instructions on how to compute the value
of $f$ at a point in $W$ given the values already computed on the values
before that point.

We now use transfinite induction to define an equivalence
relation\index{equivalence relations, equivalence classes} on the class of
well-orderings\index{well-orderings}.

\begin{definition}
 Let $R$ and $S$ be well-orderings.  We say that $R$ is
 {\upshape similar\index{similarity!definition of}}\index{similarity|textbf} to
 $S$ if there is a bijection\index{bijection} $s$ between $\dom(R)$ and
 $\dom(S)$ such that $s(a)\, S\, s(b)$ iff $a \rR b$; we call the map $s$ a
 {\upshape similarity} between $R$ and $S$.
\end{definition}

\begin{thm}
 Similarity is an equivalence relation\index{equivalence relations, equivalence
 classes}.  Moreover, if $R$ is similar to $S$, the similarity between $R$ and
 $S$ is uniquely determined; if $R$ is not similar to $S$, then either $S$ is a
 continuation of an order similar to $R$ or $R$ is a continuation of an order
 similar to $S$ (not both).
\end{thm}

\preuve\ It is easy to check that similarity is an equivalence
relation; check that the identity map on the domain\index{domain} of a
well-ordering\index{well-orderings} is a similarity, that
inverses\index{inverse function} of similarities are similarities, and that
compositions\index{composition} of similarities are similarities.  The rest
follows from the Transfinite Recursion Theorem: define $s(a)$ as the least
element of dom$(S) - s[$seg$_R\{a\}]\index{segment}$ (if any) for each $a$ in
$\dom(R)$ -- this defines a unique function\index{function}.
Induction\index{transfinite!induction} shows that $s(a)$ is the only possible
candidate for the value at $a$ of the 
similarity between $R$ and $S$, if any.  If $s$ is defined on all of
$\dom(R)$, then $S$ is a continuation of an order similar to $R$; if
$s$ is only defined on a segment of $\dom(R)$ (because $\dom(S)$ is
exhausted) then $R$ is a continuation of an order similar to $S$; the
fact that $s$ is uniquely defined ensures that only one of the three
possible situations can occur.
\finpreuve

Since similarity\index{similarity} is an equivalence relation, it induces a
partition\index{partition} of the set of all well-orderings\index{well-orderings} into equivalence classes\index{equivalence relations, equivalence classes}.
These equivalence classes are objects of some interest to us.

\begin{definition}
 An {\upshape ordinal\index{ordinal numbers!definition of}\index{ordinal
 numbers|textbf} number} is an equivalence class\index{equivalence relations,
 equivalence classes} under similarity\index{similarity}.  $\Ord$ is the set of
 all ordinal numbers.
\end{definition}

\begin{definition}
 The {\upshape order type\index{order type!definition of}\/}\index{order
 type|textbf} of a well-ordering\index{well-orderings} is the 
 ordinal number of which it is an element.
\end{definition}

The first ordinal number is the similarity class of the empty
relation\index{empty set}\index{relation!empty}, which is certainly a
well-ordering; this is the ordinal number~$0$.  The ordinal numbers of
well-orderings\index{well-orderings} of finite\index{finite!ordinal} sets are
the finite ordinal numbers $0, 1, 2,\ldots$ (not to be confused with the
(cardinal\index{cardinal numbers}) natural\index{natural number} numbers
defined above).  An example of an infinite\index{infinite!ordinal} ordinal
number is the equivalence class\index{equivalence relations, equivalence
classes} ``$\omega$" of the usual order on the natural numbers.

We feel the need of some structure:

\begin{definition}
 If $\alpha$ and $\beta$ are ordinal  numbers, and $R \in \alpha$, $S \in
 \beta$ are well-orderings, we say that $\alpha \leq \beta$ iff $S$ is a
 continuation (possibly non-strict) of a well-ordering\index{well-orderings}
 similar\index{similarity} to $R$.
\end{definition}

\begin{thm}
 The relation $\leq$ defined on ordinal numbers is a
 well-ordering.
\end{thm}

\preuve\ It is easy to show that $\leq$ is well-defined (does not depend on
the choice of the well-orderings $R$ and $S$ representing $\alpha$ and $\beta$
in thee definition) and that it is a linear order (using the previous
theorem).  If $A$ is a set of ordinal  numbers, consider a
well-ordering\index{well-orderings} 
$R$ with domain\index{domain} $W$ belonging to an element of $A$; consider the
set of elements $a$ of $W$ such that $R$ restricted to
$\seg_R\{a\}\index{segment}$ belongs to an ordinal
in $A$.  If it is empty, then every ordinal less than the ordinal to
which $R$ belongs is not in $A$, and this ordinal is itself the smallest
ordinal in $A$; if it is nonempty, it has a smallest element $a$, and the
ordinal of $\seg_R\{a\}$ must be the smallest ordinal in $A$.
\finpreuve

We define some operations on ordinal numbers:

\begin{definition}
 If $\alpha$ and $\beta$ are ordinal numbers, and $R$ and $S$ are
 well-orderings\index{well-orderings}, $\alpha + \beta$ is the ordinal which
 contains the order on $\dom(R) \oplus \dom(S)$ produced by using the order on
 $R$ to order $\dom(R) \times \{0\}$, the order on $S$ to order $\dom(S) \times
 \{1\}$, and letting all elements of $\dom(R) \times \{0\}$ be ``less than" all
 elements of $\dom(S) \times \{1\}$.
\end{definition}

\begin{definition}
 If $\alpha$ and $\beta$ are ordinal numbers, and $R$ and $S$ are
 well-orderings\index{well-orderings}, $\alpha\beta$ is the ordinal which
 contains the order on $\dom(R) \times \dom(S)$ induced by comparing the second
 projection using $R$, then comparing the first projection using $S$ if the
 first projections\index{projections} are the same (reverse lexicographic
 order).
\end{definition}

These operations coincide with the usual arithmetic operations
on finite\index{finite!ordinal} ordinals\index{ordinal numbers}, but they do
not have desirable algebraic properties.  For instance, $\omega + 1$ is the
ordinal containing the order $0 < 1 < 2 < \ldots < x$, where a new object $x$
is added following all the numbers; $1 + \omega$ is the ordinal containing $x <
0 < 1 < 2< \ldots,$ which is simply $\omega$ itself.  An
exponentiation\index{exponentiation!ordinal}
operation can also be defined using transfinite\index{transfinite!recursion}
recursion.  Ordinal 
addition\index{addition!of ordinals} and multiplication\index{multiplication!of
ordinals} can be shown to be functions\index{function} using 
Stratified Comprehension\index{Stratified Comprehension Theorem}.

We give some examples of well-orderings\index{well-orderings}.  The usual
order\index{order (linear)} on the integers is not a well-ordering, because
there is no least element of the whole domain.  But if we redefined the order
so that negative numbers were all less than the positive numbers and zero (as
before) but now ordered among themselves with respect to absolute value, we
would have the order
$$
  -1 < -2 < -3 < \ldots < 0 < 1 < 2 <\ldots
$$
of order type\index{order type} $\omega + \omega$ = $\omega 2$.  A natural
order on linear functions with natural\index{natural number} number
coefficients would be
$$
  0 < 1 < 2 < \ldots x < x+1 < x+2 < \ldots < 2x < 2x+1 < 2x+2 < \ldots,
$$
an order
of type $\omega\omega = \omega^2$.  The natural order on all
polynomials of finite degree with natural number coefficients which
extends this order is an example of order type $\omega^\omega$ (which
we have not as yet defined).



\section[The ``Paradox'' of Burali-Forti meets
$T$]{The ``Paradox'' of Burali-Forti meets
\mathversion{bold}$T$}\index{paradoxes!Burali-Forti}

Observe that every ordinal\index{ordinal numbers} $\alpha$ has a
successor\index{successor!ordinal} $\alpha + 1$, which is strictly greater than
$\alpha$.  The ordinals have a natural well-ordering\index{well-orderings}
$\leq$, indicated above.  For any ordinal $\alpha$, the order $\leq$ restricted
to seg$\{\alpha\}$\index{segment} belongs to a uniquely determined ordinal
$T^2\{\alpha\}$; observe that
the type\index{types (relative)} of $T^{2}\{\alpha\}$ is two types higher than
the type of $\alpha$ in a stratified\index{stratification} sentence; this is
the reason for the superscript~2.

The following ``inductive" argument might convince us that
$T^{2}\{\alpha\} = \alpha$; consider the smallest ordinal\index{ordinal
numbers} $\beta$ such that $T^{2}\{\beta\}$ is not 
equal to $\beta$.  Now for each $\alpha < \beta$ it is easy to see that
$T^{2}\{\alpha\} <  T^{2}\{\beta\}$.  By examining the order on ordinals, we
can see that every ordinal less than $T^{2}\{\beta\}$ must be $T^{2}\{\gamma\}$
for some $\gamma < \beta$. 
But it follows that $T^{2}\{\beta\}$ is the smallest ordinal greater than all
$T^{2}\{\alpha\}$'s for $\alpha < \beta$, that is, $\beta$ itself ---\,so there
can be no such $\beta$.

Now consider the well-ordering\index{well-orderings} $\leq$ itself.  The
ordinal\index{ordinal numbers} number 
containing $\leq$ is called $\Omega$.  By the ``result" of the
previous paragraph, we find that $T^2\{\Omega\} = \Omega$; the ordinal
number of $\leq$ restricted to $\seg\{\Omega\}$\index{segment} is the same as
the ordinal number of $\leq$; but this is clearly impossible, since the
latter well-ordering is a strict continuation of the former, including
such additional ordinals as $\Omega + 1$.

This is the Burali-Forti paradox\index{paradoxes!Burali-Forti}, the paradox of
the largest ordinal\index{ordinal numbers} number.  If you were alert, you saw
that it does not really work in our set theory; the point is that
``$T^2\{\alpha\} = \alpha$" is not a stratified\index{stratification}
condition, so does not define a set, and we can only do
transfinite\index{transfinite!induction} induction on conditions which define
sets.  Instead of the paradox, we obtain the following (perhaps startling)

\begin{thm}
 $T^2\{\Omega\} < \Omega$; i.e, $\leq$ on
 $\seg_{\leq}\{\Omega\}\index{segment}$
 is not similar\index{similarity} to $\leq$ on all\linebreak
 ordinals\index{ordinal numbers}.
\end{thm}

A genuinely surprising result is the following:

\begin{thm}
 The singleton\index{singleton image} image of $\leq$ is not similar to $\leq$.
\end{thm}

\preuve\ The relation between $x$ and $\seg\{x\}$\index{segment} is not
stratified\index{stratification} (for $x$ in $W = \dom(R)$, $R \in \alpha$),
but the relation between $\{{x}\}$ and $\seg\{x\}$ is
stratified.  Thus, we can define a similarity\index{similarity} between $\leq$
on $\seg\{\Omega\}$ and the double singleton\index{singleton image} image
$\leq^{\iota^2}$ (the relation which obtains between $\{\{\alpha\}\}$ and
$\{\{\beta\}\}$ when $\alpha$ and $\beta$ are ordinals\index{ordinal numbers}
and $\alpha \leq \beta$).  Since $\leq^{\iota^2}$ is not similar to $\leq$,
clearly $\leq^{\iota}$ is not similar to $\leq$.
\finpreuve

\begin{definition}
 We can now define $T\{\alpha\}$\index{$T$ operation!on ordinals} as the order
 type\index{order type} of the singleton\index{singleton image} 
 image $R^{\iota}$ of an element $R$ of $\alpha$.  $T^2\{\alpha\}$ will have the
 intended meaning.  
\end{definition}

\begin{thm}
 The function\index{function} $(x \mapsto  \{x\})$ does not exist.
\end{thm}

\preuve\ If it did, it would be easy to define a similarity\index{similarity}
between $\leq$ and its singleton\index{singleton image} image.
\finpreuve

\begin{thm}
 The inclusion\index{inclusion} relation restricted to the proper
 initial segments\index{segment} of a well-ordering\index{well-orderings} of
 type $\alpha$ has order type\index{order type}
 T$\{\alpha\}$.
\end{thm}

\preuve\ The similarity is supplied by the obvious bijection\index{bijection}
between proper initial segments of the domain\index{domain} of the
well-ordering and singletons\index{singleton} of elements of the domain of the
well-ordering.
\finpreuve

$T$ is traditionally used in this kind of set theory for a class of
similar operations on classes of sets or relations\index{relation} involving
the singleton\index{singleton!related to $T$ operations}; it should always be
possible to figure out which one is intended from context.  All such operations
are ``singleton image\index{singleton image!related to $T$ operations}" 
operations in one way or another.

We define another important operation on the ordinals\index{ordinal numbers}:

\begin{definition}
 If $A$ is a set of ordinals, $\lim A$ (or $\sup A$) is the
 smallest ordinal greater than all elements of $A$.
\end{definition}

For example, if $F$ is the set of finite\index{finite!ordinal} ordinals,
$\omega = \lim F$.  Of course, some ``limits", like $\lim \Ord$, do not exist.
If we wanted a function\index{function} $\lim$, $\lim(A)$ would have to be the
singleton\index{singleton} of lim $A$.

In general, ordinals\index{ordinal numbers} fall into three classes: 0, which
has no predecessors, ordinals $\alpha +1$ (successor\index{successor!ordinal}
ordinals), which have immediate predecessors, and ordinals $\lim A$ (limit
ordinals), where $A$ is nonempty 
with no largest element.  An example of a limit ordinal is $\omega$
(the smallest one).

If $W$ is a well-ordering\index{well-orderings}, we may interpret $W$ as a
``sequence\index{sequence!transfinite|textbf}'' 
indexed by the ordinals as follows:

\begin{definition}
 Let $W$ be a well-ordering. For each ordinal\index{ordinal numbers}
 $\alpha$, $W_{\alpha}$ is defined as the element $w$ of $\dom(W)$ such
 that $\seg_W\{w\}\index{segment} \in \alpha$ (if there is such an element). 
\end{definition}

Notice that the ordinal index $\alpha$ is two types\index{types (relative)}
higher than the term $W_{\alpha}$.  By the results above, we cannot hope in
general to get a {\itshape function\index{function}\/} mapping an initial
segment of the ordinals onto $\dom(W)$; some
well-orderings\index{well-orderings} are longer than the natural
well-ordering of the ordinals\index{ordinal numbers}.

The argument for the Burali-Forti paradox\index{paradoxes!Burali-Forti} above
showed that the set of ordinals such that $T\{\alpha\} = \alpha$\index{$T$
operation!on ordinals} cannot exist (otherwise the induction could be carried
out).  Nonetheless, we are interested in this ``property\index{properties}" of
ordinals, and provide a

\begin{definition}
 A {\upshape Cantorian\index{Cantorian!ordinal|textbf}} ordinal\index{ordinal
 numbers} is an ordinal $\alpha$ such that $T\{\alpha\} = \alpha$. 
\end{definition}

\begin{thm}
 The set of Cantorian ordinals does not exist.
\end{thm}

\preuve\ From the Burali-Forti argument.
\finpreuve

The ``class" of Cantorian ordinals can be thought of as the
``small" ordinals\index{ordinal numbers} of our theory.  We could already prove
with no difficulty that each of the concrete finite\index{finite!ordinal}
ordinals is Cantorian and without undue difficulty that the
infinite\index{infinite!ordinal} ordinal $\omega$ and many of 
its relatives are Cantorian.  The ordinals which we can show {\upshape not\/}
to be Cantorian are the very large ones like $\Omega$.  The reason for
the use of the term ``Cantorian" will be given in a later chapter.  

Using $\Omega$, we can build another example of a ``class"
which, like the ``class" of Cantorian ordinals\index{ordinal numbers}, is not a
set.  We first state a theorem about the $T$ operation on ordinals:

\begin{thm}
 For $\alpha,\beta$ ordinals, $\alpha \leq \beta$ iff
 $T\{\alpha\} \leq T\{\beta\}$; $T\{\alpha+\beta\} = T\{\alpha\} +
 T\{\beta\}$; and $T\{\alpha\beta\} = T\{\alpha\} T\{\beta\}$.\index{$T$
 operation!on ordinals}
\end{thm}

\preuve\ The first fact (the one we will use here) is obvious; we omit
the proofs (easy) of the other facts.
\finpreuve

\begin{thm}
 The ``class" consisting of $\Omega$, $T\{\Omega\}$,
 $T^2\{\Omega\}$, $T^{\,3}\{\Omega\},\ldots$, constructed by iterating the $T$
 operation on $\Omega$, does not exist.
\end{thm}

\preuve\ We have proven above that $T\{\Omega\} < \Omega$.  Repeated
application of the previous Theorem yields the result that $T^{n+1}\{\Omega\}
< T^{n}\{\Omega\}$ for each natural\index{natural number} number $n$.  If the
``class" described above were actually a set, it would have no least element
relative to the natural ordering on the ordinals\index{ordinal numbers}, which
is impossible.
\finpreuve

\section{The Axiom of Small Ordinals}\index{Axiom of Small Ordinals}

The last theorem is somewhat disquieting; it means that in some
``external" sense the ordinals of our set theory are {\itshape not\/}
well-ordered\index{well-orderings}.  We will introduce an axiom which will help
allay this disquiet for the Cantorian\index{Cantorian!ordinal} or ``small"
ordinals\index{ordinal numbers}, first verifying that $\omega$ is Cantorian
(and so intuitively ``small''), as claimed above.

\begin{definition}
 A ``small" ordinal is an ordinal which is less than
 some Cantorian ordinal.
\end{definition}

\begin{thm}
 $\omega$ is Cantorian.
\end{thm}

\preuve\ One element of $\omega$ is the order $\leq$ on the
natural\index{natural number} numbers; one element of $T\{\omega\}$\index{$T$
operation!on ordinals} is the singleton\index{singleton image} image 
of $\leq$, an order on the singletons\index{singleton} of natural numbers.  A
similarity\index{similarity} between these orders is defined by $s(n) =
\{T^{-1}\{n\}\}$ (the $T$ operation on natural numbers was defined in the
section on finite\index{finite!set} sets and the Axiom of Counting\index{Axiom
of Counting}; results in that section show that this map is a
bijection\index{bijection}).  Note that we cannot conclude that $s(n) = \{n\}$
without appealing to the Axiom of Counting, but we don't need to do this to
establish the result.
\finpreuve

\begin{cor}
 Each finite\index{finite!ordinal} ordinal\index{ordinal numbers} is small.
\end{cor}

\preuve Each finite ordinal is less than the Cantorian\index{Cantorian!ordinal}
ordinal $\omega$.
\finpreuve

\begin{axiom}{Axiom of Small Ordinals\index{Axiom of Small
Ordinals!introduced}\index{Axiom of Small Ordinals|textbf}}
 For any sentence $\phi$ in the
 language of set theory, there is a set $A$ such that
 for all $x$, $x$ is a small ordinal such that $\phi$ iff ($x\in A$ and
 $x$ is a small ordinal).
\end{axiom}

An informal way of stating this (in terms of proper classes) is that
for any sentence $\phi$ there is a set $A$ such that the (possibly
proper) class $\{x \st \phi\}$ and the set $A$ have the same
intersection\index{intersection!of classes} with the (proper) class of small
ordinals\index{ordinal numbers}. 

We briefly discuss the intuition behind the Axiom of Small Ordinals\index{Axiom
of Small Ordinals}.
It is certainly true that our well-orderings\index{well-orderings} up to any
finite\index{finite!ordinal} ordinal are ``genuine"; there is no ``external"
collection which has no least element.  Up to some point, our
ordinals\index{ordinal numbers} are ``standard" in this sense.  A property of
``standard" ordinals $\alpha$ is that they 
contain $\seg\{\alpha\}$\index{segment} relative to the natural order on
ordinals; but this is an ``illegitimate" (unstratified\index{stratification})
property of ordinals in our intuitive model of set theory, and thus so is the
stronger property of being a ``standard" ordinal (with its reference to
{\itshape all\/} 
subcollections of the domain\index{domain} of a well-ordering\index{well-orderings}, rather than to those
with ``labels" ---\,i.e., those realized by sets).  The idea underlying
the Axiom of Small Ordinals\index{ordinal numbers} is that we let the
``illegitimate" notion of ``standard" (which justifies defining sets in terms
of any sentence $\phi$, not just stratified\index{stratification} sentences)
coincide with the ``illegitimate" notion ``small" (which says that the
``illegitimate" relation between ordinals and segments\index{segment} has not
yet totally slipped its leash).  We assert that all subcollections\index{subset} of
the ``standard" ordinals are realized by sets, although large subcollections are
realized only by extensions (the class of all
Cantorian\index{Cantorian!ordinal} ordinals is realized by Ord, for instance).

When we apply the Axiom of Small Ordinals, we are likely to consider
classes which may or may not exist as sets (and some which {\itshape
definitely\/} do not exist as sets).  We warn the reader that we will
sometimes use the notations $\{x \st \phi\}$ and $(x \mapsto T)$ as
notations of convenience for ``classes'' and ``functions\index{function!proper
class}'' which do not exist as sets in our theory.


\begin{Thm}{Mathematical Induction\index{induction!mathematical, on an
unstratified condition} Theorem}
 For each condition $\phi$ and variable $x$,
 if $\phi[0/x]$ holds and ``for all $n \in {\cal N}$, if $\phi[n/x]$ then
 $\phi[n+1/x]$" holds, then ``for all $n \in {\cal N}$, $\phi[n/x]$" holds.
\end{Thm}

\preuve\ It is easy to define the bijection\index{bijection} between the
finite\index{finite!ordinal} ordinals\index{ordinal numbers} and the
corresponding natural\index{natural number} numbers.  By the Axiom of Small 
Ordinals, there is a set $A$ whose small ordinal members are precisely
the small ordinals $x$ such that $x$ is finite and $\phi[n/x]$ holds
of the natural number $n$ corresponding to $x$.  Let ${\cal N}_{\tt Ord}$ be the set of
finite ordinals.  Then the image of $A \cap {\cal N}_{\tt Ord} $ in ${\cal N}$
consists of exactly the elements $n$ of ${\cal N}$ such that
$\phi[n/x]$ and is inductive\index{inductive set}, so includes all of ${\cal
N}$.
\finpreuve

\begin{Thm}{Theorem of Counting\index{Axiom of Counting!proved as a theorem}}
 For each natural number $n$, $T\{n\}=n$\index{$T$ operation!on natural
 numbers}.
\end{Thm}

\preuve\ Easy induction\index{induction!mathematical, on an unstratified
condition} on the unstratified\index{stratification} condition.
\finpreuve

We see that the Axiom of Counting is redundant once the Axiom of Small
Ordinals is introduced.

The Axiom of Small Ordinals allows induction on unstratified
conditions on general Cantorian well-orderings as well as on the
natural numbers:

\begin{Thm}{Cantorian\index{Cantorian!transfinite induction theorem}
Transfinite\index{transfinite!induction} Induction Theorem}
 Let $\leq$ be a
 well-ordering\index{well-orderings} belonging to a
 Cantorian\index{Cantorian!ordinal} ordinal $\alpha$.  Let $\phi$ 
 be any sentence.  Suppose that for all $x \in \dom(\leq)$, (if (for
 all $y \leq x$, $\phi[y/x]$) then $\phi$); $\phi$ is true of $x$ if it is
 true of all its predecessors in the well-ordering.  It follows that
 for all $x \in \dom(\leq)$, $\phi$.
\end{Thm}

\preuve\ Recall that the notation $[\leq]_{\beta}$ represents the
element (if any) of $\dom(\leq)$ determining a segment\index{segment} of order
type\index{order type} $\beta$, for each ordinal $\beta$.  The class of all
small ordinals\index{ordinal numbers} $\beta$ such that not
$\phi[[\leq]_{\beta}/x]$ would be the intersection\index{intersection!of
classes} of some set $B$ with the small ordinals.  All ordinals 
less than $\alpha$ would be small by definition of ``small'', so $B
\cap \seg\{\alpha\}$ would be a set.  If this set were nonempty, it
would have a smallest element $\gamma$.  For all $x \leq
[\leq]_{\gamma}$ we would have $\phi$, but not
$\phi[[\leq]_{\gamma}/x]$. This is impossible.
\finpreuve

The concept of a ``small" ordinal is actually a nonce concept,
as the following reveals:

\begin{thm}
 An ordinal is small iff it is Cantorian\index{Cantorian!ordinal}.
\end{thm}

\preuve\ Let $\beta$ be a small ordinal.
By the definition of ``small", there is an ordinal $\alpha$ greater
than $\beta$ which is Cantorian.  $\alpha$ is small because $\alpha
+1$ is Cantorian.  0 is Cantorian; successors\index{successor!ordinal} of
Cantorian ordinals\index{ordinal numbers} are Cantorian; limits of Cantorian
ordinals are Cantorian (because the supremum of a collection of
ordinals $T\{\beta_{\alpha}\}$\index{$T$ operation!on ordinals} will be the image
under $T$ of the supremum of the collection of $\beta_{\alpha}$'s; thus the
limit of a collection of fixed points of $T$ will itself be a fixed point of
$T$).
The order type of ordinals up to $\alpha$ is $T^2\{\alpha\} = \alpha$,
so we can prove by induction (on an unstratified\index{stratification}
condition) that all ordinals $< \alpha$ (including $\beta$)  are Cantorian.
All small ordinals are  Cantorian, so ``small" simply means ``Cantorian".
\finpreuve


\section{Von Neumann Ordinals}

We note that the difference between the representation of natural\index{natural
number} numbers in the usual\index{Zermelo--Fraenkel set theory} set theory and
our representation of natural 
numbers carries over into the infinite: just as each von Neumann\index{von
Neumann numeral} natural number is the collection of all smaller natural
numbers, so each ordinal in the usual set theory is the collection of all
smaller ordinals (with the numerals as the finite\index{finite!ordinal}
ordinals).  Each of the von Neumann numerals is a set in our set theory, but
this is not necessarily the case for the larger von Neumann ordinals; as we
have observed already, even the von Neumann ``$\omega$'', which would be
the collection of all the finite\index{finite!von Neumann ordinal} von Neumann
ordinals, is not necessarily a set in our set theory, though it is possible for
it to exist.  We give a definition:

\begin{definition}
 A {\upshape von Neumann ordinal\index{von Neumann ordinal|textbf}} is a set
 $A$ such that $\bigcup[A] \subseteq A$ and the restriction of
 $\in$\index{membership} 
 to $A$ is a strict well-ordering\index{well-orderings} of $A$.
\end{definition}

\begin{definition}
 For each von Neumann ordinal $A$, we define $A^+$, the {\upshape successor} of
 $A$, as $A \cup \{A\}$. 
\end{definition}

\begin{lemme}
 For each von Neumann ordinal $A$, $A^+$ is a von Neumann ordinal.
\end{lemme}

\preuve\ $\bigcup[A \cup \{A\}] = \bigcup[A] \cup A = A$ (because $\bigcup[A]
\subseteq A$) $\subseteq A \cup \{A\}$.  The restriction of $\in$ to $A$ is a
strict well-ordering; the restriction of $\in$ to $A^+$ extends this relation
by adding a single object ``greater'' than the elements of $A$, which will
still be a strict well-ordering.
\finpreuve

Note that this is an unstratified\index{stratification} condition!  We
would like to say that a von Neumann ordinal $A$ corresponds naturally
to the order type\index{order type} of the restriction of $\in$ to
$A$, but there is a problem with the von Neumann ordinals `0' = $\vide$
and `1' = $\{\vide\}$, on which $\in$ determines the same empty
relation.  The correct formulation is that each von Neumann ordinal
corresponds to the unique ordinal $\alpha$ such that the order type of
the restriction of $\in$ to $A^+$ is $\alpha+1$.  The same technical
point arises in our treatment of ordinals in relation to well-founded
extensional relations in a later chapter.

\begin{lemme}
 An element of a von Neumann ordinal $A$ is a ``proper
 initial segment\index{segment}'' of $A$ in the sense of the order
 induced by $\in$ and is itself a von Neumann ordinal.
\end{lemme}

\preuve\ The fact that $\in$ is a strict linear order of $A$ and
that each element of an element of $A$ is an element of $A$ (because
$\bigcup[A] \subseteq A$) ensures that an element of $A$ is a proper
initial segment of $A$ in the sense of the order induced by $\in$.
($A \in A$ is impossible because the ordering of $A$ by $\in$ must be
strict; this is why the initial segments\index{segment} are proper).
 
Let $B \in A$.  Let $D\in C\in B$ be an arbitrary element of
$\bigcup[B]$; we will show $D\in B$.  $C\in A$ because
$\bigcup[A]\subseteq A$ (elements of elements of $A$ are in $A$); a
second application of the same fact gives $D\in A$.  Since $C$ and $D$
are in $A$ and $\in$ is a linear order\index{order (linear)} on $A$, $D\in C\in
B$ implies $D \in B$.  Thus $\bigcup[B] \subseteq B$.  $B$ is strictly
well-ordered\index{well-orderings} by $\in$ because it is a subset of $A$,
which is strictly well-ordered by the same relation.  Thus any element $B$ of
$A$ is a von Neumann ordinal.
\finpreuve

It is useful to note that this result is correct in the case of the
peculiar ordinal `1'; seg$\{$`0'$\}$ defined with respect to the
(empty) order on `1' does come out correctly as `0'.

\begin{lemme}
 The intersection\index{intersection!Boolean} of two von Neumann ordinals is a
 von Neumann ordinal.
\end{lemme}

\preuve\ Let $C$ be the intersection of von Neumann ordinals $A$
and $B$.  $\bigcup[C] \subseteq (\bigcup[A]\cap\bigcup[B]) \subseteq
(A\cap B)=C$.  $\in$ clearly strictly
well-orders\index{well-orderings}\index{order (strict)} $C$, since 
it strictly well-orders its supersets $A$ and $B$.
\finpreuve

\begin{lemme}
 Each proper ``initial segment\index{segment}'' of a von Neumann ordinal
 $A$ (in the sense of $\in$ restricted to $A$) is an element of $A$.
\end{lemme}

\preuve\ There is an $\in$-least element $B$ of $A$ which is not
an element of the proper initial segment; each of its elements must be
in the proper initial segment by the fact that it is $\in$-least; if
$B$ did not contain some element $C$ of the initial segment, then $C$
would contain $B$ (because $\in$ linearly orders $A$) and $B$ would
belong to the initial segment\index{segment}, contradicting the choice of $B$.
Thus $B$ must contain exactly the elements of (and so be equal to) the
proper initial segment.
\finpreuve

\begin{lemme}
 For any two von Neumann\index{von Neumann ordinal} ordinals $A$ and $B$, we
 have either $A\in B$, $B \in A$, or $A = B$.
\end{lemme}

\preuve\ The intersection\index{intersection!Boolean} of two von Neumann
ordinals is a von Neumann ordinal.  It is also easy to see that the
intersection will be an initial segment\index{segment} of each of the two
originally chosen ordinals.  If it is a proper initial segment of each of them,
it will be an element of each of them and so of itself, which is impossible
(this would violate strict well-ordering\index{well-orderings} by
membership\index{membership}).  Thus, the intersection 
of two distinct von Neumann ordinals must be equal to one of them,
which will be an element of the other by the preceding Lemma.
\finpreuve

\begin{lemme}
 Each von Neumann ordinal $A$ is the set of all von
 Neumann ordinals $B$ such that the strict well-ordering of $B^+$ by $\in$
 has smaller order type\index{order type} than the strict well-ordering
 of $A^+$ by $\in$.
\end{lemme}

\preuve\ By the Lemmas above, each von Neumann\index{von Neumann ordinal}
ordinal $A$ has as elements exactly those von Neumann ordinals $B$ which are
proper initial segments of $A$ with respect to the order determined by $\in$.
$B$ is a proper initial segment of the order on $A$ iff the order type
of membership on $B^+$ is smaller than the order type of membership on
$A^+$; we have to consider the successors to avoid the strange case of
`0' and `1'.
\finpreuve

\begin{thm}
 The set of von Neumann ordinals does not exist.
\end{thm}

\preuve\ Suppose that there was a set $N$ whose members were
exactly the von Neumann ordinals.  Any element of a von Neumann
ordinal is a von Neumann ordinal, so $\bigcup[N] \subseteq N$ would
hold.  It follows from the lemmas above that $\in$ coincides with
inclusion\index{inclusion} on von Neumann ordinals, so the restriction of $\in$
to $N$ would be a set.  $\in$ would be a strict linear order on $N$ by a
preceding Lemma.  Every initial segment\index{segment} of $N$ is a von Neumann
ordinal, on which $\in$ is a strict well-ordering\index{well-orderings}; it is
easy to see that a strict linear order\index{order (linear)}\index{order
(strict)} every initial segment of which is a strict well-ordering is a strict
well-ordering itself.  Thus $N$ would itself be a von Neumann\index{von Neumann
ordinal} ordinal, from which it would follow that $N \in N$,
but no von Neumann ordinal can be an element of itself!
\finpreuve

\begin{thm}
 The order type\index{order type} of $\in$ restricted to a von Neumann
 ordinal is Cantorian\index{Cantorian!ordinal}.
\end{thm}

\preuve\ Let $\alpha$ be the order type of $\in$ restricted to a
von Neumann ordinal $A$.  The inclusion order on the set of proper
initial segments\index{segment} of $A$ with respect to the order $\in$
has the order type $T\{\alpha\}$\index{$T$ operation!on ordinals}.  But
this is the same set with the same order as $A$ ordered by $\in$, so
$\alpha= T\{\alpha\}$ is Cantorian.
\finpreuve

It follows that ``large'' ordinals\index{ordinal numbers} such as $\Omega$
cannot have corresponding von Neumann\index{von Neumann ordinal} ordinals.  It
is not possible to prove in our set theory that there are any
infinite\index{infinite!von Neumann ordinal} von Neumann ordinals; it
turns out to be consistent with our axioms that every Cantorian
ordinal has a corresponding von Neumann ordinal. 

\Exercises

\begin{enumerate}
\item Construct examples of well-orderings\index{well-orderings} and ordinal
  numbers.  Take sums and products of some of these order types\index{order
  type} and describe well-orderings belonging to the sums and products.

\item  Present definitions of addition\index{addition!of ordinals} and
  multiplication\index{multiplication!of ordinals} of ordinal numbers using
  transfinite\index{transfinite!recursion} recursion.  Prove that they are
  equivalent to the definitions given.  Propose a definition of ordinal
  exponentiation\index{exponentiation!ordinal} using transfinite recursion.

\item  Construct a well-ordering\index{well-orderings} $\leq$ with the property
  that for any sequence\index{sequence} $s$ of elements of $\dom(\leq$) which
  is   increasing in the sense of $\leq$, there is an element $x$ of
  $\dom(\leq)$
  with the property that $x > s_n$ for each $n \in {\cal N}$.

  {\itshape Hint:}  consider
  the class of order types\index{order type} of well-orderings of $\cal N$.

\item  (hard) Finite\index{finite!set} sets can be modelled in the
  natural\index{natural number} numbers using the relation $m \in^{{\cal N}} n$
  defined by ``the $m$th binary digit of $n$ is 1'', where the $m$th binary
  digit is the coefficient of $2^m$ in the binary expansion: the digit in the
  ``ones place'' is the 0th binary digit.  Show that the
  relation\index{membership} $\in^{{\cal N}}$ is represented by a set in our
  theory.   A permutation $\pi$ is defined as interchanging each natural number
  $n$ with $\{m \st m \in^{{\cal N}} n\}$.  Show that the permutation $\pi$
  exists (you need the Axiom of Counting\index{Axiom of Counting} for this).
  Now define the relation $x \in^{\pi} y$ as $x \in \pi(y)$.  The relation
  $\in^{\pi}$ is not a set (the relative type\index{types (relative)} of $y$ is
  one type higher than the relative type of $x$).  Prove that if $\in^{\pi}$ is
  used as the membership\index{membership} relation instead of $\in$, all of
  our axioms  will still hold (you can neglect Small Ordinals\index{Axiom of
  Small Ordinals}, though it works as well) and the set of all
  finite\index{finite!von Neumann ordinal} von Neumann ordinals\index{von
  Neumann ordinal} will exist.
\end{enumerate}
