\chapter[Boolean Operations on Sets]{Boolean\index{Boolean algebra, operations}
Operations on Sets}

We will now introduce the first of the basic set constructions
promised in the last chapter.  Two particular sets which we might expect to
find are $\vide$, the empty set\index{empty set} mentioned in the previous
chapter, and $V$,
the universe\index{universe, universal set!definition}, the set which contains
everything.  Notation for these sets might be $\{x \st x = x\}$ or
$\{x \st \mathrm{True}\}$ for $V$, and $\{x \st x \neq x\}$ or $\{x \st
\mathrm{False}\}$ for $\vide$ (another common notation for the set with no elements is $\emptyset$).  We state an axiom:

\begin{axiom}{Axiom of the Universal Set\index{universe, universal set!axiom}}
$\{x \st x = x\}$, also called $V$, exists.
\end{axiom}

We will not need a special axiom for $\vide$, for reasons which
will become evident shortly.  Be warned: this is our first departure
from the usual set theory ZFC\index{Zermelo--Fraenkel set theory}; in ZFC, the
Axiom of the Universal Set is {\itshape false\/}; the universe does not exist.
The reasons for this will be made clear in a later chapter.

Given a set $A$, an obvious set which might leap to mind would
be the set of all things not in $A$.  If $A$ were the set of beautiful
things, the complementary\index{complement} set which would come to mind is the
set of things which are not beautiful.  We assert an axiom providing the
existence of such sets (the symbol $\not\in$ is read ``is not an element of"):

\begin{axiom}{Axiom of Complements\index{complement!axiom of}}
  For each set $A$, the set $A^c$ = $\{x \st x \not\in A\}$, called the
  {\itshape complement\/} of A, exists.  
\end{axiom}

Two things to note here:  the Axiom of the Universal Set and
the Axiom of Complements together imply that the empty set\index{empty set}
$\vide$ exists:
the empty set can be constructed as the complement of the universe,
i.e., $\vide$ = $V^c$; the Axiom of Complements is uniformly false in
ZFC\index{Zermelo--Fraenkel set theory},
where {\itshape no\/} set has a complement.  We believe that the presence of
the universe and of complements together give the system of set
theory presented here a more intuitive flavour than the usual set
theory.

When we have two sets $A$ and $B$, a natural set to consider is
the set which contains all the elements of both $A$ and $B$.  If $A$ is the
set of green objects and $B$ is the set of red objects, the set we are
interested is the set of all objects which are either green or red.
In this case, the two sets do not overlap; if $A$ were the set of
college professors and $B$ were the set of absent-minded people, the
set we are interested in would be the set of people who are either
college professors or absent-minded or both.  We assert an axiom to
provide for this kind of set:

\begin{axiom}{Axiom of (Boolean\index{Boolean algebra, operations})
Unions\index{union!Boolean!axiom}}
  If $A$ and $B$ are sets, the set
  $$
    A \cup B = \{x \st \mbox{$x \in A$ or
    $x \in B$ or both}\},
  $$
  called the (Boolean) {\upshape union} of $A$ and $B$,
  exists.
\end{axiom}

These are the only new primitive constructions and axioms
required for most of this chapter.  There are several derived set
operations and relations among sets which we will need to define.

Given two sets $A$ and $B$, another set which we might want to
consider is the set of objects in the ``overlap" between $A$ and $B$; if $A$
were the set of college professors and $B$ were the set of absent-minded
persons, the set of interest would be the set of absent-minded college
professors.  We can {\itshape prove\/} that such sets exist for every $A$ and
$B$:

\begin{thm}
 For each set $A$ and $B$, the set
 $$
  A \cap B = \{x \st \mbox{$x \in A$ and $x \in
     B$}\},
 $$
 called the (Boolean\index{Boolean algebra, operations}) {\upshape
 intersection\index{intersection!Boolean, definition
 of}\index{intersection!Boolean|textbf}} of
 $A$ and $B$, exists.
\end{thm}

\preuve $A \cap B = (A^c \cup B^c)^c$.  An object is an element of $(A^c \cup
B^c)^c$ exactly if it is {\itshape not\/} an element of $A^c \cup B^c$.  An
object is {\itshape not\/} an 
element of $A^c \cup B^c$ exactly if it is not the case either that it is an
element of $A^c$ or that it is an element of $B^c$; equivalently, if it is
not the case either that it is not an element of $A$ or that it is not
an element of $B$.  But this is true exactly if it is an element of $A$
and an element of $B$.\finpreuve

Note the possible confusion caused by two different uses of
the word ``and" in English: an object belongs to $A \cap B$ if and
only if it belongs to $A$ {\itshape and\/} belongs to $B$, but the set made
up of the elements of $A$ {\itshape and\/} the elements of $B$ is $A \cup
B$.  We simply have to be careful: these two different uses of ``and"
are logically totally different from one another.

In some contexts, the ``working universe\index{universe, universal set}" is not
the whole of $V$.  This causes problems when complements\index{complement} are
to be taken.  For instance, if we are working in arithmetic we think of the
complement of the set of even numbers as being the set of odd numbers, rather
than the set of all odd numbers and non-numbers.  It is convenient to
define a new concept, introduced in the following:

\begin{thm}
 For each pair of sets $A, B$, the set
 $$
  B - A = \{x \st \mbox{$x \in B$ and $x \not\in A$}\},
 $$
 called the {\upshape relative
 complement\index{complement!relative} of $A$ with respect to $B$\/}, exists.
\end{thm}

\preuve $B - A = B \cap A^c$.\finpreuve

A final operation on sets (only occasionally used):

\begin{definition}
For $A,B$ sets, we define the {\upshape symmetric difference\index{symmetric
difference}\/} $A \Delta B$ as $(B - A) \cup (A - B)$.
\end{definition}


A very important relation between sets is the ``subset"
relation\index{subset}, or ``inclusion\index{inclusion}".  We say that ``$A$ is
a subset of $B$" or that ``$A$ is included in $B$" if every element of $A$ is
also an element of $B$.  We give a

\begin{definition}
 $A \subseteq B$ ($A$ is a {\upshape subset} of $B$, $A$ is {\upshape included}
 in $B$) exactly if
 $A$ and $B$ are sets and for every $x$ it is the case that if $x$ is an
 element of $A$, then $x$ is an element of $B$.

 $A \subset B$ ($A$ is a {\upshape proper subset}\index{subset!proper}
 of $B$, $A$ is properly included in $B$) exactly if $A$ is a subset of $B$ and
 $A$ is not equal to $B$.

 $A \supseteq B$ and $A \supset B$ are the
 converse\index{converse (of a relation)} relations, read ``$A$ is a superset
 of $B$ or $A$ contains $B$", ``$A$ is a proper superset of $B$
 or $A$ properly contains $B$".
\end{definition}


\begin{thm}
 $\vide \subseteq A$ for every set $A$.
\end{thm}

\preuve If $x$ is an element of $\vide$... (anything, including ``$x$ is an
element of $A$", follows).\finpreuve

We can see that inclusion\index{inclusion} is not equivalent to
membership\index{membership}; $\vide \subseteq \vide$ by the Theorem, but it is
not the case that $\vide$ is an element of $\vide$!  The relation between
inclusion and the operations above is expressed in the following:

\begin{thm}
 For all sets $A,B$, $A \subseteq B$ exactly if  $A \cup B  =  B$.
\end{thm}

A formal proof is left to the reader.  It should be obvious that
adding all the elements of a subset\index{subset} of $B$ to $B$ will not change
$B$!

We define another important relationship between sets:

\begin{definition}
 Sets $A$ and $B$ are said to be {\upshape disjoint\index{disjoint}\/} exactly
 if for every $x$, either $x$ is not an element of $A$ or $x$ is not an element
 of $B$ or both (i.e., no $x$ belongs to both $A$ and $B$).
\end{definition}

Sets are disjoint if they do not overlap; this insight is
equivalent to the following result, which we state without proof:

\begin{thm}
 Sets $A, B$ are disjoint\index{disjoint} exactly if  $A \cap B  =  \vide$.
\end{thm}

There is no conventional symbol for the relationship of
disjointness, so the result of this Theorem is used to represent
disjointness symbolically.  It is not correct to say that disjoint
sets $A$ and $B$ have ``no intersection\index{intersection!Boolean}"; they do
have an intersection, namely the empty set\index{empty set}, but this
intersection has no elements.  The notion of disjointness\index{disjoint}
extends to more than two sets:
for example, sets $A$, $B$, and $C$ are said to be disjoint if there
is no object which belongs to more than one of them.  The condition $A
\cap B \cap C$ = $\vide$ is weaker; this merely asserts that there is
no object which belongs to all three sets, and allows for the
possibility that an object may belong to two of them.  For this
reason, collections of disjoint sets with more than two elements are
often referred to as {\itshape pairwise\/} disjoint\index{disjoint!pairwise}
sets;
the real condition in the case above is that $A \cap B = A \cap C = B \cap C$ =
$\vide$.

The operations and relations defined in this chapter comprise
an interpretation of the field known as ``Boolean\index{Boolean algebra,
operations} algebra"; we will distinguish the operations defined here as
``Boolean operations".  We list some useful facts about them which are also
axioms or theorems of Boolean algebra, expressed in our notation:

\begin{description}
\item[\fdescr commutative laws:]
  $\begin{array}[t]{@{}l}
    A \cap B = B \cap A\\ A \cup B = B \cup A
   \end{array}$

\item[\fdescr associative laws:]
  $\begin{array}[t]{@{}l}
    (A \cap B) \cap C = A \cap (B \cap C)\\
    (A \cup B) \cup C = A \cup (B \cup C)
   \end{array}$

\item[\fdescr distributive laws:]
  $\begin{array}[t]{@{}l}
    A \cap (B \cup C) = (A \cap B) \cup (A \cap C)\\
    A \cup (B \cap C) = (A \cup B) \cap (A \cup C)
   \end{array}$

\item[\fdescr identity laws:]
  $\begin{array}[t]{@{}l}
    A \cup \vide = A\\ A \cap V = A
   \end{array}$

\item[\fdescr idempotence laws:]
  $\begin{array}[t]{@{}l}
    A \cup A = A\\ A \cap A = A
   \end{array}$

\item[\fdescr cancellation laws:]
  $\begin{array}[t]{@{}l}
    A \cup V = V\\ A \cap \vide = \vide
   \end{array}$

\item[\fdescr De Morgan's laws\index{de Morgan's laws}:]
  $\begin{array}[t]{@{}l}
   (A \cap B)^c = A^c \cup B^c\\ (A \cup B)^c = A^c \cap B^c
   \end{array}$

\item[\fdescr double complement\index{complement} law:]  $(A^c)^c = A$

\item[\fdescr other complement laws:]
  $\begin{array}[t]{@{}l}
    A^c \cap A = \vide\\ A^c \cup A = V\\ V^c = \vide\\ \vide^c = V
   \end{array}$

\item[\fdescr inclusion\index{inclusion} principles:]
  \begin{tabular}[t]{@{}l}
   $A \subseteq B$ exactly if $A = A \cap B$\\ 
   also exactly if $B = B \cup A$.
  \end{tabular}
\end{description} 

These laws should remind one to some extent of the ordinary rules of
algebra for addition and multiplication; but notice the exact symmetry
between the two operations, and that nothing in the usual algebra
corresponds to the complement\index{complement} operation (in particular, not
the additive inverse!)

The operations of union\index{union!Boolean} and
intersection\index{intersection!Boolean} as we have defined 
them allow us to combine finitely many sets, or to find the common
part of finitely many sets, by repeated application of the binary
Boolean\index{Boolean algebra, operations} operations.  We will sometimes want
to define unions and intersections of collections of sets which are not necessarily finite\index{finite!collections, union or intersection of};
this is provided by the following

\begin{axiom}{Axiom of Set Union\index{union!set!axiom}}
 If $A$ is a set all of whose elements are sets, the
 set $\bigcup[A]$ = $\{x \st$ for some $B$, $x \in B$ and $B \in A\}$, called
 the (set) {\upshape union of $A$}, exists.
\end{axiom}

The construction of set intersections\index{intersection!set} will require the
assistance of axioms not yet provided.  Since we chose Boolean\index{Boolean
algebra, operations} union as a primitive and used it to define Boolean
intersection, we will follow precedent and patiently wait until we have the
required additional machinery needed to construct set intersections.


\Exercises

\begin{enumerate}
\item  The smallest Boolean\index{Boolean algebra, operations} algebra consists
  of the sets $\vide$ and $V$.  Develop an interpretation of this Boolean
  algebra in mod 2 arithmetic, giving definitions of the operations of Boolean
  algebra in terms of the operations of the arithmetic.  What Boolean algebra
  operations correspond to your arithmetic operations?  Explain why there are
  two different nontrivial interpretations. 

\item  Verify that the symmetric difference\index{symmetric difference}
  operation is commutative and associative, and that
  intersection\index{intersection!Boolean} distributes over it. 

\item  Look up ``Venn diagrams\index{Venn diagrams}'' in another source and
  verify some of the axioms given at the end of the chapter or the results of
  the previous exercise using Venn diagrams. 
\end{enumerate}

