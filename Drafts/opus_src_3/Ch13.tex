\chapter{The Real Numbers}

In this section, we outline the construction of the continuum (the
system ${\cal R}$ of real numbers\index{real numbers}).



\section{Fractions and Rationals}\index{rational numbers}

We start by restricting ourselves to the set ${\cal N}^+$ of positive
natural numbers.  This will have the effect of simplifying our
constructions.

We will first construct the system ${\cal F}$ of {\itshape
fractions\index{fractions}\/} (a preliminary implementation of the positive
rational numbers).

The basic idea of the construction is that a fraction $\frac mn$,
where $m$ and $n$ are positive natural\index{natural number} numbers, is a
finite\index{finite!structures} structure 
which we might just as well identify as $(m,n)$.  The difficulty which
arises immediately is that there need to be identifications among
fractions.  We expect that $\frac mn = \frac rs$ will hold if $ms =
nr$.  We make the following


\begin{ThmEtc}{Observation.}
 The relation $\sim$ defined by ``$(m,n) \sim
 (r,s)$ iff $ms = nr$'' is an equivalence relation\index{equivalence relations,
 equivalence classes} on pairs\index{ordered pair} of positive
 natural numbers.  (This use of $\sim$ is a nonce notation).
\end{ThmEtc}

A typical approach now would be to define the positive rational\index{rational
numbers} numbers as equivalence classes\index{equivalence relations,
equivalence classes} under this relation.  We prefer not to do this, because of
the difference in relative type\index{types (relative)} between
equivalence classes and their representatives.  An example of the
difficulty is that the natural map taking each positive natural number
$n$ to the fraction $\frac n1$, which would be the equivalence class
$[(n,1)]_\sim$ under this scheme, would have an
unstratified\index{stratification}
definition and so could not be relied upon to be defined (the Axiom of
Counting\index{Axiom of Counting} introduced in the previous section {\itshape
does\/} imply that it is well-defined, though).

The solution to this problem in general, as we noted above, is the use
of the Axiom of Choice\index{axiom of choice} to choose a representative from
each equivalence class\index{equivalence relations, equivalence classes}.  This
is not necessary in this case, as there is a familiar way to choose a canonical
fractional representation for a positive rational\index{rational numbers}
number.


\begin{definition}
 We define the function\index{function}
 $$
  \mathrm{simplify}(m,n) = \Bigl(\frac
      m{\gcd(m,n)},\frac n{\gcd(m,n)}\Bigr)
 $$
 for $m,n \in {\cal N}^+$.  Observe
 that this function sends each element of each equivalence
 class\index{equivalence relations, equivalence classes} under $\sim$ to a
 unique representative element of that class.  Division of 
 positive natural\index{natural number} numbers (and greatest common divisors)
 are understood to be defined in the usual way.
\end{definition}

\begin{definition}
 We define the set ${\cal F}$ of {\upshape fractions\index{fractions|textbf}} 
 as $\{\mathrm{simplify}(m,n) \st m, n \in {\cal N}^+\}$.  We define $\frac
 mn$ as the fraction simplify$(m,n)$ for each pair\index{ordered pair} of
 positive natural numbers $m$, $n$.
\end{definition}

We define basic operations and relations\index{relation} on fractions.

\begin{definition}
 Let $\displaystyle\frac mn$ and $\displaystyle\frac rs$ be fractions.  We
 define $\displaystyle\frac mn + \frac rs$\index{addition!of fractions} as 
 $\displaystyle\frac{ms +
 nr}{ns}$ and $\displaystyle(\frac mn)(\frac rs)$ as
 $\displaystyle\frac{mr}{ns}$\index{multiplication!of
 fractions}\rule[-3.5mm]{0mm}{8mm}.  Notice that the definition of fraction
 notation ensures that
 simplifications are understood to be carried out.
\end{definition}

\begin{definition}
 Let $\displaystyle\frac mn$ and $\displaystyle\frac rs$ be fractions.  We
 define a relation $\displaystyle\frac mn < \frac rs$ (and $\displaystyle\frac
 rs > \frac mn$)as
 holding when $\displaystyle ms < nr$.  Observe that this relation implements
 the usual linear order\index{order (linear)} on positive
 rational\index{rational
 numbers} numbers.
\end{definition}

\begin{definition}
 Let $\displaystyle\frac mn$ and $\displaystyle\frac rs$ be fractions, with 
 $\displaystyle\frac mn > \frac rs$.  We define $\displaystyle\frac mn - \frac
 rs$ as $\displaystyle\frac{ms -
 nr}{ns}$; the order condition on the fractions is exactly what is
 needed for the subtraction of natural numbers in the numerator to be
 well-defined, and it is straightforward to verify that this
 subtraction operation has the correct relationship to addition of
 fractions.
\end{definition}

We now construct the system ${\cal Q}$ of rational\index{rational numbers}
numbers.  The idea,
similar to that underlying the construction of the system ${\cal F}$
of fractions, is to let ordered pairs\index{ordered pair} $(p,q)$ of
fractions stand for differences $p - q$.  Of course, we wish to make
certain identifications among these ordered pairs:

\begin{ThmEtc}{Observation.}
 The relation $\sim$ on pairs of fractions defined
 by ``$(p,q)\sim(r,s)$ iff $p+s = q+r$'' is an equivalence
 relation\index{equivalence relations, equivalence classes}.
\end{ThmEtc}

Once again, we want to avoid using equivalence relations on pairs\index{ordered
pair} for reasons of stratification\index{stratification}.

\begin{definition}
 We define a function\index{function}
 $$
  \mathrm{simp}(p,q) = \Bigl(\frac
  11+p-\min(p,q),\frac 11+q-\min(p,q)\Bigr).
 $$
 Observe that the function simp 
 collapses each equivalence class\index{equivalence relations,
 equivalence classes} under $\sim$ to the unique member of that class
 one of whose projections\index{projections} is $\frac 11$.
\end{definition}

\begin{definition}
 We define the set ${\cal Q}$ of {\upshape rational\index{rational
 numbers|textbf} numbers\/} as $\{\mathrm{simp}(p,q) \st p,q \in {\cal F}\}$.
 Notice
 that the rational numbers are of three kinds: there is a unique element 
 $(\frac 11,\frac 11)$ which we denote by 0 (an admitted abuse of
 notation), elements $(\frac mn + \frac11,\frac11)$ which we might
 denote by $+\frac mn$ and elements $(\frac 11, \frac mn + \frac11)$
 which we might denote by $-\frac mn$.
\end{definition}
We do not intend to be so
careful; we will use the usual notation for rational numbers, with the
attending confusion of positive rationals with
fractions\index{fractions} and postive integral fractions and rational
numbers with the corresponding positive natural\index{natural number}
numbers.  We introduce the notation $p - q$, where $p$ and $q$ are
fractions, for $\mathrm{simp}(p,q)$, to simplify the notation of the next
definitions; the notion of subtraction of fractions defined above, for
which we have the same notation, will not be used below.

We define basic operations and relations\index{relation} on the rational
numbers:

\begin{definition}
 Let $p-q$ and $r-s$ be rational numbers (where $p$,
 $q$, $r$, $s$ are fractions, at least one of $p$ and $q$ is $\frac11$, and at least one of $r$ and $s$ is $\frac11$).  We define $(p-q) + (r-s)$ as
 $(p+r)-(q+s)$.  We define $(p-q)(r-s)$ as $(pr+qs)-(ps+qr)$.  The
 notions of addition\index{addition!of rational numbers} and
 multiplication\index{multiplication!of rational numbers} on the right of each
 equation are those defined on fractions\index{fractions}, so these are
 definitions.
\end{definition}

\begin{definition}
 Let $p-q$ and $r-s$ be rational numbers (where $p$,
 $q$, $r$, $s$ are fractions, at least one of $p$ and $q$ is $\frac11$, and at least one of $r$ and $s$ is $\frac11$).  We define $(p-q) < (r-s)$ as holding
 when $p+s < q+r$ (this is a definition because the order\index{order (linear)}
 relation on the right is that defined on fractions).
\end{definition}

\begin{definition}
 The set ${\cal Z}$ of {\upshape integers} is defined
 as the collection of all rational numbers of the forms $\frac n1 -
 \frac 11$ or $\frac 11 - \frac n1$, for $n \in {\cal N}^+$.  Note that
 this usually distinguished set played no special part in our
 construction.
\end{definition}

This development is certainly not the only one possible.  All
developments of number systems in set theory are ``implementations''
of the familiar systems and must have some essentially arbitrary
features.  Nor is this development rigorous; it would be possible to
verify all familiar properties of the rational numbers starting with
the axioms for the natural\index{natural number} numbers proved in the previous
chapter, but we have not done this.  We have presumed familiarity with the
properties of the number systems being implemented.

An advantage of this implementation is that we need never worry about
zero denominators.  The price we pay is the special role of $\frac 11$
in the ``simplest form'' for differences; zero is not available to us!
Another advantage of this development is that it is more historically
accurate; positive rational\index{rational numbers} numbers were understood
before negative numbers, at least by the Greeks.






\section[Magnitudes, Real Numbers, and Dedekind Cuts]{Magnitudes, Real Numbers,
and\\ Dedekind Cuts}

Our official implementation of the real numbers will be eccentric in
not being based on the rationals\index{rational numbers}!

We return to the system ${\cal F}$ of fractions\index{fractions}.  We observe
that the positive real numbers correspond exactly to the nonempty open initial
segments\index{segment} of the set of positive rational\index{rational numbers}
numbers.  We define a set ${\cal M}$ of {\itshape magnitudes}, a preliminary
implementation of the positive real numbers.


\begin{definition}
 ${\cal M}$ is defined as $\{A \st A \subset {\cal
 F}, A \neq \vide$ and for all $p \in {\cal F}, p \in A$ iff for some $q
 \in A, q > p\}$.  An element of ${\cal M}$ is called a {\upshape
 magnitude\index{magnitudes|textbf}}.  A magnitude is a nonempty proper initial
 segment\index{segment} of ${\cal F}$ with no greatest element.  For some
 purposes, $\cal F$ might be regarded as a special ``magnitude'' $\infty$.
\end{definition}

Definitions of operations and relations on ${\cal M}$ are extremely
simple:

\begin{definition}
 If $A, B \in {\cal M}$, $A + B = \{a + b\st a \in
 A, b \in B\}$ and\newline $AB = \{ab \st a \in A, b \in B\}$.  $A < B$ iff $A
 \subset B$.\index{addition!of magnitudes}
\end{definition}

The construction of the system ${\cal R}$ of {\itshape real numbers\index{real
numbers|textbf}\/} from the system ${\cal M}$ is precisely analogous to the
construction of the system ${\cal Q}$ of rational\index{rational numbers}
numbers from the system ${\cal F}$ of fractions\index{fractions}; for the
details see the previous subsection.
Constructing a system of ``extended reals\index{real numbers!extended}''
including $\pm \infty$ would involve some technicalities.

For this nonstandard approach, we could claim once again the authority
of the ancient Greeks, who understood positive reals (or at least some
of them) without ever acknowledging negative numbers or zero.  We also
think that there is something very Greek about defining arbitrary
magnitudes as {\itshape segments\index{segment}}.

We indicate the approach of Dedekind, who based his construction of
${\cal R}$ on the full system ${\cal Q}$.

\begin{definition}
 A {\upshape Dedekind cut\index{Dedekind cut|textbf}} is a pair\index{ordered
 pair} $(L,R)$ of subsets\index{subset} of ${\cal Q}$ such that 
 \begin{enumerate}
  \item $L$ and $R$ are nonempty.
  \item Each element of $L$ is less than each element of $R$.
  \item $L$ has no greatest element and $R$ has no least element.
  \item ${\cal Q}-(L \cup R)$ has at most one element.
\end{enumerate}
\end{definition}

It is generally understood (a draft said ``intuitively clear'' here,
which is certainly not true!) that Dedekind cuts\index{Dedekind cut} correspond
exactly to individual real numbers\index{real numbers}.  The definitions of
operations on Dedekind cuts are somewhat more complicated than the very simple
definitions of operations\linebreak on ${\cal M}$.

A difficulty with type differentials which we avoided in the
constructions of fractions\index{fractions} and
rational\index{rational numbers} numbers by using representatives
instead of equivalence classes\index{equivalence relations,
equivalence classes} is unavoidable in our construction of
magnitudes\index{magnitudes} (and so of reals\index{real numbers}).
The identification of a fraction $\frac pq$ with the corresponding
magnitude $\{\frac rs \st \frac rs < \frac pq\}$ cannot be shown to
be witnessed by a function\index{function} from ${\cal F}$ to ${\cal
M}$ without an appeal to the Axiom of Counting\index{Axiom of
Counting}; its definition is unstratified\index{stratification}.  To
show that such a function exists using Counting, check that the map
taking $\frac {T\{p\}}{T\{q\}}$ to $\{\frac rs \st 
\frac rs < \frac pq\}$ does have a stratified\index{stratification} definition,
and then observe that the Axiom of Counting allows us to ignore the $T$
operation.

Magnitudes and real\index{real numbers} numbers, unlike
fractions\index{fractions} and rationals\index{rational numbers}, have an 
essentially set theoretical definition; their definition cannot be
reduced to finite\index{finite!structures} structures.  This is not surprising,
since the additional property which ${\cal R}$ has, over and above the axioms
of an ordered field which are also satisfied by ${\cal Q}$, is the Least
Upper Bound Property\index{bound!least upper}\index{Least Upper Bound
Property}, which is an essentially set theoretical property.

We state and prove the Least Upper Bound Property for $\cal M$; the
development of the property for $\cal R$ involves merely technical
complications.

\begin{definition}
 Let $A$ be a set of magnitudes\index{magnitudes} and let $m$ be a
 magnitude.  We call $m$ an {\upshape upper bound\index{bound!upper}} for $A$
 iff
 for all $n \in A$, $m \geq n$.
\end{definition}

\begin{Thm}{Least Upper Bound Property}
 Let $A$ be a nonempty set of
 magnitudes with an upper bound.  Then $A$ has a least upper bound;
 i.e., there is a magnitude $m$ which is an upper bound of $A$ such
 that for all upper bounds $n$ of $A$, $m \leq n$.
\end{Thm}

\preuve Recall throughout this proof that the order\index{order (linear)} on
$\cal M$ coincides with inclusion\index{inclusion}:  subsets\index{subset} are
less than supersets.

Let $B$ = $\{m \in {\cal M} \st m$ is an upper bound\index{bound!upper} for
$A\}$.  We claim that the least upper bound of $A$ is $\bigcap[B]$.  An
intersection\index{intersection!set} of open proper initial
segments\index{segment} of ${\cal M}$ will either 
be an open proper initial segment of ${\cal F}$ (and so an element of
${\cal M}$), a closed proper initial segment of ${\cal F}$, or the empty
set\index{empty set}.  $\bigcap[B]$ will not be empty because every element of
$B$ is a superset of any given element of $A$.  Suppose that $\bigcap[B]$ is
the closed interval determined by an element $p$ of ${\cal F}$.  $\{q
\in {\cal F} \st q < p\}$, the element of ${\cal M}$ corresponding to
$p$, would clearly be an upper bound\index{bound!upper} of $A$, so a superset
of $\bigcap[B]$, so $p \not\in\bigcap[B]$, contrary to the purported
choice of $p$.  Thus $\bigcap[B]\in {\cal M}$.  $\bigcap[B]$ will be
an upper bound of $A$ because it is a superset of each element of $A$;
it is less than or equal to each upper bound of $A$ because it is the
intersection\index{intersection!set} of all upper bounds of $A$.
\finpreuve

\begin{definition}
 If $A$ is a nonempty subset\index{subset} of ${\cal M}$ or ${\cal
 R}$ with an upper bound\index{bound!upper}, we define $\sup A$ as the least
 upper bound\index{bound!least upper} of $A$.  If $A$ is a nonempty set with a
 lower bound\index{bound!lower}, we define $\inf A$ as the greatest lower
 bound\index{bound!greatest lower} of $A$ (the existence of greatest lower
 bounds is an easy corollary of the existence of least upper bounds).
\end{definition}

We can define infinite\index{infinite!sum} sums of magnitudes\index{magnitudes}
(and of real\index{real numbers} numbers with more work); it is very convenient
to allow the ``magnitude'' $\infty$
in this context.

\begin{definition}
 A function\index{function} with domain\index{domain} a nonempty initial
 segment\index{segment} (proper or otherwise) of $\cal N$ will be called a
 {\upshape sequence\index{sequence|textbf}}.
 A sequence whose range is included in a set $A$ will be called a
 {\upshape sequence of elements of $A$}.  A sequence whose domain is a proper
 subset\index{subset!proper} of $\cal N$ will be called a
 {\upshape finite\index{finite!sequence} sequence}; a sequence which is not
 finite will be
 called an {\upshape infinite\index{infinite!sequence} sequence}.  The
 alternative notation $s_n$ is available for $s(n)$ when $s$ is a sequence
and $n \in {\cal N}$.
\end{definition}

\begin{definition}
 The sum $\sum[s]$ of a finite\index{finite!sum} sequence of fractions\index{fractions}
(the form of the definition will be the same for other number systems)
is defined inductively as follows: $\sum[s\lceil\{0\}]$ = $s(0)$ (this
defines the notion of sum for one-element sequences);
$\sum[s\lceil\{0,\ldots,n+1\}]$ = $\sum[s\lceil\{0,\ldots,n\}] +
s(n+1)$.
\end{definition}

\begin{definition}
 The sum $\sum[s]$ of an infinite\index{infinite!sequence} sequence $s$ of
 magnitudes\index{magnitudes} is defined 
 as $\{\sum[f] \st f$ is a finite\index{finite!sequence} sequence such that for
 each $n \in$ 
 $\dom(f),f(n) \in s(n)\}$.  This is unstratified\index{stratification}
 ($n$ cannot be consistently typed\index{types (relative)}); the Axiom
 of Counting\index{Axiom of Counting} assures us that this definition succeeds
 (the stratified\index{stratification} form of the definition is $\{\sum[f]
 \st f$ is a finite sequence such that for each $n \in$ dom$(f),f(n)
 \in s($T$\{n\})\}$\index{$T$ operation!on natural numbers}.

 The problem can be avoided completely by the following alternative
 definition: first define finite sums of magnitudes in the
 same way as finite sums of fractions\index{fractions}, then define (for $s$ an
 infinite sequence of magnitudes) $\sum[s]=\bigcup \{\sum [f]
 \st $ for some $ n \in {\mathcal N}, f = s\lceil
 \{0,1,\ldots, n\}\}$.
\end{definition}


\begin{ThmEtc}{Observation.}
 Any sum of magnitudes\index{magnitudes} is itself either a magnitude
 (in which case the sum is said to {\itshape converge\/})  or $\infty$.
\end{ThmEtc}


\Exercises

\begin{enumerate}
\item  Attempt proofs of some or all of the basic rules of algebra in the
  systems presented in this chapter. 

\item  Prove that for each positive natural\index{natural number} number $n$
  and magnitude $x \in {\cal M}$, there is a unique magnitude $\sqrt[n]{x}\in
  {\cal M}$ such that $(\sqrt[n]{x})^n = x$.

\item  (hard) How would you define addition\index{addition!of real\index{real
  numbers} numbers} of two real numbers defined as Dedekind cuts\index{Dedekind
  cut} in $\cal Q$?  How would you define
  multiplication\index{multiplication!of real numbers} of real numbers defined
  in this way?

\item  (hard) Present a set theoretical definition of the base $e$ of natural
  logarithms.
\end{enumerate}
