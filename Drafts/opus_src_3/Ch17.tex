\chapter[Three Theorems]{Exponentiation and\\ Three Major Theorems}

\section[Exponentiation meets $T$]{Exponentiation meets
\mathversion{bold}$T$. Infinite Sums and
Products}\index{infinite!sum}\index{infinite!product}

The arithmetic operation which we have avoided, even in the
realm of finite\index{finite!set} sets, is
exponentiation\index{exponentiation!cardinal|textbf}.  The natural definition
of exponentiation is

\begin{ThmEtc}{$*$Definition}
 $|A|^{|B|} = |[B \rightarrow  A]|$.\\
 Special case:  $2^{|A|} = 
 |[A \rightarrow  \{0,1\}]| = |{\cal P}\{A\}|$.
\end{ThmEtc}

The idea is that we must make $|A|$ independent choices from a
set of size $|B|$.  The Axiom of Choice protects us from certain
possible embarrassments\index{axiom of choice}.  This definition does not work
(in the sense that it does not define a function\index{function}) because it is
not stratified\index{stratification}; the types\index{types (relative)} of $A$
and $B$ are one higher on the right than on the left.

The solution to the problem is to introduce another $T$
operation (extending the $T$ operation already defined on natural\index{natural
number} numbers):

\begin{definition}
 $T\{|A|\} = |{\cal P}_1\{A\}|$\index{$T$ operation!on cardinals}.
\end{definition}

It is easy to show that this is a valid definition (it does not depend
on the choice of $A$).  It is less easy to see that it is nontrivial,
but we need to remember that we have already shown that the natural
bijection\index{bijection}\linebreak
$(x \mapsto \{x\}) \lceil A$ between $A$ and ${\cal
P}_1\{A\}$ (the set of one-element subsets of $A$) does not exist in
at least the case $A = V$.  Note that $T\{\kappa\}$ has type\index{types
(relative)} one higher than that of $\kappa$.  In the definitions below we will
use the inverse $T^{-1}$ of this operation, which proves to be a
well-defined partial operation on cardinals\index{cardinal numbers} (defined
only on cardinals less than or equal to $|{\cal P}_1\{V\}|$) and has the effect
of lowering relative type rather than raising it.

We can then give a genuine (partial)

\begin{definition}
 $|A|^{|B|} = T^{-1}\{|[A \rightarrow  B]|\}$;
 $2^{|A|} = T^{-1}\{|{\cal P}\{A\}|\}$.
\end{definition}

\begin{cor}
 $T\{|A|\}^{T\{|B|\}} = |[A \rightarrow B]|$;
 $2^{T\{|A|\}} = |{\cal P}\{A\}|$\index{$T$ operation!on cardinals}.
\end{cor}

We give the function\index{function} $(\kappa \mapsto  2^{\kappa})$ a name:
\begin{definition}
 $\exp = (\kappa \mapsto  2^{\kappa})$.
\end{definition}

Exponentiation\index{exponentiation!cardinal} is a special case of the more
general problem of sums and products of infinite sets of
cardinals\index{cardinal numbers}; we give relevant definitions:

\begin{definition}
 
Let $G$ be an indexed family of sets.  Let $F$ be the associated
infinite family of cardinals, defined by $F(\{i\}) = |G(i)|$ for each
$i$ in $\dom(G)$.

 We define $\sum[F]$, the sum of
 $F$, as $T^{-1}\{|\sum[ G]|\}$, the result of applying $T^{-1}$ to the
 cardinality\index{cardinal numbers} of the disjoint\index{disjoint!disjoint
 sum} sum of $G$.

 We define $\prod[F]$, the product of $F$, as
 $T^{-2}\{|\prod[G]|\}$, the result of applying 
 $T^{-2}$ to the cardinality of the Cartesian product of $G$.

 Self-indexing may be used to define sums and products of sets.
 Context cues will of course be required to distinguish these
 notations from the identical notations for Cartesian\index{Cartesian product}
 products and disjoint sums of sets!
\end{definition}

The criterion for determining the applications of $T$ in these
definitions is that the type\index{types (relative)} of individual
cardinals\index{cardinal numbers} in the infinite sums and products
should be the same as the type of the resulting cardinals.

For the sake of completeness, we develop a definition of
exponentiation\index{exponentiation!ordinal|textbf} of ordinals\index{ordinal
numbers}.  Infinite sums\index{infinite!sum|textbf} and
products\index{infinite!product|textbf} of ordinals
can also be defined, but we will not do this.  A natural operation on
ordinals should, of course, correspond to a natural operation on
well-orderings\index{well-orderings}.  The product of ordinals corresponded to
the order type of the ``reverse lexicographic'' ordering of pairs of elements
taken from well-orderings taken from the factors.  For example, $\omega^2$ is
the order type\index{order type} of the lexicographic ordering on
pairs\index{ordered pair} of elements taken from an ordering of type $\omega$
(natural\index{natural number} numbers, for example).

This suggests that $\omega^\omega$, for example, should be the
reverse lexicographic ordering on strings of numbers of length
$\omega$.  Unfortunately, one does not obtain a
well-ordering\index{well-orderings} (or even a total order) in this way.
Experiment reveals that one does obtain a 
well-ordering of items taken from an order of type $\beta$ ordered
with type $\alpha$ if one observes the restriction that all but
finitely many of the $\alpha$-ordered items are the smallest element
of the order of type $\beta$.  Here is the formal definition:

\begin{definition}
 Let $\alpha$, $\beta$ be ordinal numbers.  Choose
 well-orderings\index{well-orderings} $R$, $S$, with least elements $r$, $s$
 respectively, 
 from $\alpha$, $\beta$, respectively.  We define a well-ordering $W$
 on the set of all elements of $[\dom(R) \rightarrow  \dom(S)]$
 with the property that all but finitely many values of the
 function\index{function} 
 are $s$, the least element of $\dom(S)$.  For $f$, $g$ in this set of
 functions, we say that $f\mathrel{W} g$ if $f(x)\mathrel{S} g(x)$, where $x$
 is the
 $R$-maximal element at which the values of $f$ and $g$ differ (there
 must be such a maximal element because of the restriction on the
 functions used).  The order type\index{order type} of $W$ will be
 $T\{\beta^\alpha\}$\index{$T$ operation!on ordinals}; 
 if the order type of $W$ is not in the range of the $T$ operation,
 $\beta^\alpha$ is not defined (this actually never happens).
\end{definition}

Ordinal exponentiation can also be defined by
transfinite\index{transfinite!recursion} recursion in a fairly natural way.
The need to assign higher priority to positions in
sequences\index{sequence!transfinite} which are later in the $\alpha$-order
motivates the use of reverse lexicographic order here (where it is necessary)
and in the definition of ordinal multiplication\index{multiplication!of
ordinals} (where it seems a bit unnatural; 
I would rather write $2\omega$ than $\omega2$, but $2\omega=\omega$).

Ordinal exponentiation does not parallel cardinal\index{cardinal numbers}
exponentiation\index{exponentiation!cardinal} in 
its effects on size.  Note, for example, that $\omega^\omega$ is the
order type\index{order type} of a well-ordering\index{well-orderings} of a
countable\index{countable} set.  In fact, the 
cardinality of the full domain\index{domain!full} of a well-ordering of type
$\beta^\alpha$ is no larger than the larger of the sizes of full
domains of representatives of $\alpha$ and $\beta$.  A consequence of
this is that $\beta^\alpha$ is always defined.

It should be evident that the $T$ operation ``commutes'' with all
operations of cardinal\index{cardinal numbers} (and ordinal) arithmetic.  This
observation is recorded in the following:

\begin{lemme}
 If $\kappa$ and $\lambda$\index{$T$ operation!on cardinals} are cardinal
 numbers,

 $T\{\kappa\} \leq T\{\lambda\}$ iff $\kappa \leq \lambda$ (thus 
 $T\{\kappa\} = T\{\lambda\}$ iff $\kappa = \lambda$);

 $T\{\kappa + \lambda\}
 = T\{\kappa\} + T\{\lambda\}$;

 $T\{\kappa\lambda\} = T\{\kappa\}T\{\lambda\}$;

 $T\{\kappa\}^{T\{\lambda\}} = T\{\kappa^{\lambda}\}$ 
 \begin{tabular}[t]{@{}l}
  ---\,if the latter  exists\\
  (thus $\exp(T\{\kappa\}) = T\{\exp(\kappa)\}$ ---\,if $\exp(\kappa)$ exists).
 \end{tabular}
\end{lemme}

\preuve\ Omitted.  Use the fact that any structure on a collection of objects
corresponds to an exactly parallel structure on the collection of
singletons of those objects.
\finpreuve


\section[Cantor's Theorem and Paradox]{Cantor's Theorem.\\ Cantor's ``Paradox''
meets \mathversion{bold}${\cal P}_1$}\index{paradoxes!Cantor}

We now prove a powerful theorem, whose ``naive" form is the
last of the three classical paradoxes\index{paradoxes} of set theory.

\begin{Thm}{Theorem (Cantor)\index{Cantor's Theorem}}
 $T\{\kappa\} < \exp(T\{\kappa\})$\index{$T$ operation!on cardinals}.
\end{Thm}


\preuve\ Let $\kappa$ = $|A|$.  Clearly $T\{\kappa\} = |{\cal P}_1\{A\}| \leq
|{\cal P}\{A\}| = \exp(T\{\kappa\})$.
If we had $T\{\kappa\} = \exp(T\{\kappa\})$, we would have a
one-to-one\index{one-to-one map} map $f$ from ${\cal P}_1\{A\}$
onto ${\cal P}\{A\}$; it suffices to show that the existence of such a map is
impossible.  If such an $f$ existed, we could define the set $R = 
\{x \in A \st \mathrm{not}(\{x\} \subseteq f(\{x\})\}$.  Now $f^{-1}(R) =
\{r\}$ for
some $r \in A$; we see that $r \in R$ exactly if not$(\{r\} \subseteq
f(\{r\}))$, i.e., exactly if $r \not\in R$,
which is impossible.
\finpreuve

\begin{cor}
 $\kappa <  \exp(\kappa)$.
\end{cor}

In completely naive set theory, where we do not doubt that
${\cal P}_1\{A\}$ is the same size as $A$, we get the following

\begin{ThmEtc}{$*$Corollary (Cantor's Paradox\index{paradoxes!Cantor}).}
 $|V| = |{\cal P}_1\{V\}| < |V|$.
\end{ThmEtc}

Of course, what we actually get is the still disturbing

\begin{cor}
 $|{\cal P}_1\{V\}| < |V|$  (note that we can write this $|1| < |V|$ or
 $T\{|V|\} < |V|$\index{$T$ operation!on cardinals}, as well).
\end{cor}

which should not surprise us too much, as we have already shown that
the natural ``bijection"\index{bijection!proper class} $(x \mapsto  \{x\})$
does not exist.

Note that we have now proven that there are uncountable\index{uncountable}
sets; 
$\aleph_0 = |{\cal N}| = T\{|{\cal N}|\} < \exp(T\{|{\cal N}|\}) = |{\cal
P}\{{\cal N}\}|$; there are more
sets of natural\index{natural number} numbers than there are natural numbers.
We point out that $\exp(\aleph_0)$ is also the cardinal\index{cardinal numbers}
number of ${\cal R}$, the set of reals\index{real numbers}:  the union of the
reals between 0 and 1 and a countable\index{countable} set 
(the set of decimal expansions ending in infinitely many 9's ---\,countable
because equivalent\index{equivalence (of cardinality)} to the
rationals\index{rational numbers} between 0 and 1) is equivalent to the set of
decimal expansions with zero to the left of the decimal point, which is
equivalent to the set of characteristic
functions\index{function!characteristic} of subsets\index{subset} of ${\cal
N}$, which is equivalent to ${\cal P}\{{\cal N}\}$.  For this reason,
$\exp(\aleph_0)$ is called $c$, the cardinality\index{cardinal numbers} of the
continuum.

Since the natural order on cardinals is a well-ordering\index{well-orderings},
there is a first uncountable\index{uncountable!cardinal, the first} cardinal,
called $\aleph_1$.  It is an open problem whether $c$ is equal to $\aleph_1$;
i.e., whether there are uncountable sets of real numbers which are not
equivalent\index{equivalence (of cardinality)} to ${\cal R}$ itself. 
It is known that this question cannot be decided by the
usual\index{Zermelo--Fraenkel set theory} axioms 
of set theory (or by ours!).  This famous question is called the
Continuum Hypothesis\index{Continuum Hypothesis}; it was raised by Cantor at
the very beginning of the subject.  An even more powerful assumption, also
known to be undecidable by the axioms, is the Generalized Continuum
Hypothesis\index{Generalized Continuum Hypothesis} (GCH), which asserts that
$\exp(\kappa$) is the smallest cardinal\index{cardinal numbers} greater than 
$\kappa$ for every cardinal $\kappa$ for which exp($\kappa$) is defined.




\section{K\"onig's Theorem}

We prove a theorem relating the sizes of infinite\index{infinite!product}
products and infinite sums\index{infinite!sum}, of which Cantor's
Theorem\index{Cantor's Theorem} is only a special case.

\begin{Thm}{K\"onig's Theorem\index{K\"onig's Theorem}}
 Let $F$ and $G$ be functions\index{function} with the same
 nonempty domain\index{domain} $I$ all of whose values are
 cardinal\index{cardinal numbers} numbers.  Suppose further that $F(x) < G(x)$
 for all $x\in I$, where the order is the natural order on cardinals.  It
 follows that $\sum[F] < \prod[G]$.
\end{Thm}

\preuve\ Consider concrete sets $A$ and $B$ of cardinality
$\sum[F]$ and $\prod[G]$, respectively.

We may suppose that $A$ is the union of disjoint\index{disjoint} nonempty sets
${\cal A}(x)$ for $x$ in a domain\index{domain} of size $T^{-1}\{|I|\}$; we
assume henceforth 
that $I$ is a set of singletons\index{singleton} and that the domain of the
function ${\cal A}$ is the union of $I$, which we will call $J$.  No generality
is lost by assuming that each set ${\cal A}(x)$ is nonempty, i.e.,
that each $F(\{x\})$ is nonzero.

We choose sets ${\cal B}(x)$ of cardinality\index{cardinal numbers} $G(\{x\})$
for each $x \in J$ and a bijection\index{bijection} $h$ from ${\cal P}_1^2\{B\}$
onto the set of functions\index{function} $f:J \rightarrow {\cal P}_1\{V\}$
such that $f(x) \subseteq {\cal B}(x)$.  The functions $f$ are the elements of
the Cartesian\index{Cartesian product}
product of ${\cal B}$; they are two types\index{types (relative)} above the elements of $B$.

Consider any map $M:A \rightarrow B$.  We construct an element $p$ of
$B$ which cannot be an element of $\rng(M)$.  It should be clear that
this will be sufficient to prove the theorem.  $h(\{\{p\}\})$ needs to
differ from each $h(\{\{M(q)\}\})$ at some element of $J$.  We
describe the procedure for construction of $h(\{\{p\}\})(x)$ for a
fixed $x \in J$; once we have constructed a suitable function, the
fact that $h$ is a bijection allows us to obtain a suitable $p$.  We
want to choose one element from ${\cal B}(x)$ in such a way as to
frustrate the identification of $h(\{\{p\}\})$ with any
$h(\{\{M(q)\}\})$ for $q \in {\cal A}(x)$.  This is possible for each
$x$, because there are only $|{\cal A}(x)|= F(\{x\}) <G(\{x\})=|{\cal
B}(x)|$ values of the form ``the element of $h(\{\{M(q)\}\})(x)$'' to
avoid, and $|{\cal B}(x)|$ values to choose from.  It should be clear
that this procedure prevents the identification of $h(\{\{p\}\})$ with
any $h(\{\{M(q)\}\})$ for $q$ in any ${\cal A}(x)$, i.e., for any $q
\in A$.  The proof of K\"onig's Theorem\index{K\"onig's Theorem} is complete.
\finpreuve

The proof of K\"onig's Theorem involves an essential appeal to the
Axiom of Choice\index{axiom of choice}, as is not the case for Cantor's
Theorem\index{Cantor's Theorem}.



\section{There are Atoms:  Specker's Theorem}\index{atoms}

We now prepare to prove a further theorem, due to E. Specker.
This is the surprising result that there are atoms\index{atoms}!  It
is not surprising that there are atoms; it is surprising that we can
prove that there must be atoms, since on the face of it we have not
talked about anything that could not be a set (our ordered
pairs\index{ordered pair} are not necessarily sets, but we could have
used the pair $\left<x,y\right>$ of Kuratowski and done everything in
pure set theory).  The Axiom of Choice\index{axiom of choice} seems to
be necessary for this proof.  It is unknown whether the existence of
atoms\index{atoms} can be proven from the other axioms of our theory;
this leaves open the possibility of a consistent ``atomless" theory in
which the Axiom of Choice is false\footnote{For the non-naive reader,
this is the question as to whether Quine's original theory ``New
Foundations\index{New Foundations}'' is consistent.}.

\begin{definition}
 For each cardinal $\kappa$, we define the {\em Specker sequence\/} of
 $\kappa$ as the intersection of all sets which contain $(0,\kappa)$
 and contain $(n+1,\exp(\lambda))$ whenever they contain $(n,\lambda)$
 and $\exp(\lambda)$ is defined (the construction here is the
 construction from the proof of the Recursion Theorem modified for a
 partial function).  The Specker sequence is a (possibly finite)
 sequence; we use the notation $\exp^i(\kappa)$ for its $i$th term.  We
 call a cardinal $\kappa$ a {\upshape Specker number} iff
 $\exp^n(\kappa) = |V|$ for some natural number $n$ (a Specker
 number is usually defined as a cardinal with a finite Specker
 sequence, but our more restrictive definition is better for our
 version of the proof).
\end{definition}

\begin{Thm}{Theorem (Specker)\index{Specker's Theorem}}
 $|{\cal P}\{V\}| < |V|$, and thus there are atoms\index{atoms}.
\end{Thm}

\preuve\ Consider the set of Specker numbers.  It must have a smallest
element $\mu$, since the natural order on cardinals\index{cardinal
numbers} is a well-ordering\index{well-orderings} and the Specker
numbers make up a nonempty set.  Let $s$ be the Specker sequence of
$\mu$.  Since $\mu$ is a Specker number, there is an $N$ such that
$s(N) = |V|$.  Now define the function T$[s]$ as the Specker sequence
of $T\{\mu\}$.  An easy induction\index{induction!mathematical, on an
unstratified condition} (on an unstratified\index{stratification}
condition) using the fact that $\exp$ commutes with $T$ (where defined)
proves that $T[s](n) = T\{s(n)\}$ for each natural\index{natural
number} number $n$, and so $T[s](N) = T\{|V|\}$\index{$T$ operation!on
cardinals}.  Now the assertion that there are no atoms\index{atoms}
implies the assertion that ${\cal P}\{V\} = V$ (${\cal P}\{V\}$ is
the collection of sets), which implies that $\exp(T\{|V|\}) =
\exp(|{\cal P}_1\{V\}|) = |{\cal P}\{V\}| = |V|$.  If there were
no atoms\index{atoms}, we would have $T[s](N+1) = \exp(T[s](N)) =
\exp(T\{|V|\}) = |V|$, and $T\{\mu\}$ would be a Specker number.
But this is absurd; we would then have $T\{\mu\} \geq \mu$, since
$\mu$ is the smallest Specker number, and so, by an easy induction, we
would have $T[s](n) \geq s(n)$ for each $n$ (if $\kappa \leq \lambda$,
$\exp(\kappa) \leq \exp(\lambda)$), and so we would have $T\{|V|\}
= T[s](N) \geq s(N) = |V|$, which is impossible.
\finpreuve

It is worth noting that the use of mathematical
induction\index{induction!mathematical, on an unstratified condition}
on an unstratified\index{stratification} condition, which depends on
the Axiom of Small Ordinals\index{Axiom of Small Ordinals} (the Axiom
of Counting\index{Axiom of Counting} suffices), can be avoided; there
is a rather more complicated proof which requires nothing but
Extensionality\index{extensionality}, Stratified
Comprehension\index{Stratified Comprehension Theorem}, and the Axiom
of Choice\index{axiom of choice} (without which we could not show that
the natural order on the cardinal\index{cardinal numbers} numbers is a
well-ordering\index{well-orderings}).

Since the proof actually shows that $|{\cal P}\{V\}| < |V|$, we have
shown that most objects in $V$ are atoms\index{atoms}, which also shows that
most ordered pairs\index{ordered pair} (there are $|V|.|V| = |V|$ ordered
pairs) are atoms\index{atoms}; we 
cannot hope for a set theoretical definition of our pair $(x,y)$.



\section{Cantorian and Strongly Cantorian Sets}

In the usual\index{Zermelo--Fraenkel set theory} set theory, Cantor's
Theorem\index{Cantor's Theorem} is taken to prove that the power
set\index{power set} of a set is larger than the set.  This depends on the
intuition that the singleton\index{singleton ``map''} map $(x \mapsto  \{x\})$
is a function\index{function}, or at least that $|{\cal P}_1\{A\}| = |A|$,
which is not reliable.  As we have just seen, there is a set (the
universe\index{universe, universal set}) which is {\itshape larger\/} than its
power set!

However, the argument of Cantor will work exactly as intended
if the set $A$ to be compared with ${\cal P}\{A\}$ has certain properties:

\begin{definition}
 A set $A$ is {\upshape Cantorian\index{Cantorian!set|textbf}} if it is
 equivalent\index{equivalence (of cardinality)} to 
 ${\cal P}_1\{A\}$; $A$ is {\upshape strongly Cantorian\index{Cantorian,
 strongly!set|textbf}\/} if the singleton\index{singleton ``map''} map
 restricted to $A$ exists as a set (that is, if there is a function
 $(x \mapsto \{x\})\lceil A$ such that for each $x \in A$, $((x \mapsto
 \{x\})\lceil A)(x) = \{x\}$).
 If $A$ is a Cantorian or strongly Cantorian set,
 $|A|$ is said to be a Cantorian or strongly Cantorian
 cardinal\index{Cantorian, strongly!cardinal}\index{cardinal numbers}.  The
 order type of a strongly Cantorian well-ordering is said to be a strongly
 Cantorian ordinal\index{Cantorian, strongly!ordinal|textbf}.
\end{definition}

\begin{ThmEtc}{Observations.}
 If $A$ is strongly Cantorian, it is Cantorian; if $A$ is
 Cantorian, $|A| = |{\cal P}_1\{A\}| < |{\cal P}\{A\}|$.  If $A$ is the
 domain\index{domain} of a well-ordering\index{well-orderings} $R$ which in
 turn belongs to an ordinal $\alpha$, then the ordinal $\alpha$ is (strongly)
 Cantorian exactly if $A$ is (strongly) Cantorian.
\end{ThmEtc}

The truly powerful property is that of being strongly
Cantorian.  Without introducing the Axiom of Small Ordinals\index{Axiom of
Small Ordinals}, we can already ``subvert" stratification\index{stratification}
restrictions if we have strongly 
Cantorian sets:

\begin{Thm}{Subversion Theorem\index{Subversion Theorem}}
 If a variable $x$ in a sentence $\phi$ is restricted to a strongly
 Cantorian set\index{Cantorian, strongly!set} $A$, then its type\index{types
 (relative)} can be freely raised and lowered; i.e., such a variable can safely
 be ignored in making type assignments for
 stratification\index{stratification}.
\end{Thm}

\preuve\ Let $K$ be the singleton\index{singleton ``map''} map restricted to
$A$.  Any occurrence of $x$ can be replaced with an occurrence of
$\bigcup[K(x)]$, raising the type of $x$ by one, or with an occurrence of
$K^{-1}(\{x\})$, lowering the type of $x$ by one, without affecting the meaning
of the sentence.
\finpreuve

As a result, unstratified\index{stratification} conditions can be used
extensively where strongly Cantorian\index{Cantorian, strongly!sets} sets are
involved, but variables not subject to stratification\index{stratification}
conditions must be ``bounded" in a strongly Cantorian
set.

Unfortunately, the only sets which can be shown to be strongly
Cantorian\index{Cantorian, strongly!set} with
Extensionality\index{extensionality}, Atoms\index{atoms}, Ordered
Pairs,\index{ordered pair} and Stratified Comprehension\index{Stratified
Comprehension Theorem} alone are ``standard finite\index{finite!sets, standard}
sets"; each finite set for which we can actually list the elements can be
proven to be strongly Cantorian, but we cannot even prove that all finite sets
are strongly Cantorian, much less that ${\cal N}$ is strongly Cantorian
(although we can prove that ${\cal N}$ is Cantorian\index{Cantorian!set}). Even
adopting as an axiom our Axiom of Counting\index{Axiom of Counting} (which
shows that $\cal N$ and all finite sets are strongly Cantorian; we have
already enunciated the special case of the Subversion Theorem\index{Subversion
Theorem} which applies the Axiom of Counting) strengthens the theory
considerably.   The power of the Axiom of Small Ordinals\index{Axiom of Small
Ordinals} or a similar principle is that it enables us to evade
stratification\index{stratification} restrictions on a wider range of sets.

It is worth noting that the argument given above to show that all von
Neumann\index{von Neumann ordinal} ordinals are
Cantorian\index{Cantorian!ordinal} actually demonstrates that they are
strongly Cantorian\index{Cantorian, strongly!ordinal}.

\Exercises

\begin{enumerate}
\item  Construct a bijection\index{bijection} between ${\cal P}\{\cal N\}$ and
  $\cal R$.  Construct a bijection between $\cal R$ and ${\cal R}^2$.

\item  We indicate an analytic proof that $\cal R$ is
  uncountable\index{uncountable}.  Suppose otherwise.  Then we would have a
  sequence\index{sequence} $r$ such that each real\index{real numbers} number
  $x = r_i$ for some $i \in {\cal N}$.  Start with a closed interval
  $[a_0,b_0]$ in the reals which has $r_0$ neither in its interior nor as an
  endpoint.  We construct a sequence of intervals $[a_i,b_i]$.  At step $i$, we
  divide the current interval in the sequence into three parts of equal length
  (why do we need three?) and let the next interval in the sequence be chosen
  to have $r_{i+1}$ neither in its interior nor as an endpoint.  Complete the
  argument that this allows us to construct a real number not equal to any
  $r_i$.

\item  Show that Cantor's\index{Cantor's Theorem} Theorem is a special case of
  K\"onig's Theorem\index{K\"onig's Theorem}.

\item  Analyze the proof of K\"onig's Theorem from the standpoint of
  stratification\index{stratification}; assign relative types\index{types
  (relative)} to all the objects involved in the argument and check that the
  relative types are correct for the relationships postulated among the
  objects.

\item  Prove that there are infinitely many cardinals\index{cardinal numbers}
  $\kappa$ such that every set of size $\kappa$ contains atoms\index{atoms}.

\item  (hard) We outline the approach of Specker\index{Specker's Theorem} to
  proving that the Axiom of Choice\index{axiom of choice} implies that there
  are atoms without our appeal to Counting\index{Axiom of Counting}: assume
  that ${\cal P}\{V\} = V$, so that $\exp(T\{|V|\})$\index{$T$ operation!on
  cardinals} = $|V|$. Define ``Specker numbers'' as cardinals $\kappa$ such
  that $\exp^n(\kappa)$ exists for only finitely many $n$, and define $\mu$ as
  the smallest Specker number.  Prove that $\mu = T\{\mu\}$ by proving that
  $T\{\mu\}$ and $T^{-1}\{\mu\}$ are both Specker numbers (one has to prove
  that the latter exists), then considering the definition of $\mu$.  Let $n$
  be the largest natural\index{natural number} number such that $\exp^n(\mu)$
  is defined. Considering the fact that $\mu = T\{\mu\}$, show that $n=
  T\{n\}+1$\index{$T$ operation!on natural numbers} or $n= T\{n\}+2$.  Why is
  this impossible?  Complete the details of the proof.

\item  Suppose that $A$ and $B$ are strongly Cantorian sets\index{Cantorian,
  strongly!set}.  Prove that $A \cap B$, $A \cup B$, $A \times B$, ${\cal
  P}\{A\}$, and $[A \rightarrow B]$ are strongly Cantorian sets, without
  appealing to the Subversion Theorem\index{Subversion Theorem}.

\end{enumerate}
