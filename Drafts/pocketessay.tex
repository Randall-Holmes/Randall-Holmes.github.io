\documentclass[12pt]{article}

\title{A Pocket Foundation of Mathematics}

\author{Randall Holmes}

\usepackage{amssymb}

\begin{document}

\maketitle

\section{Introduction}

The intention here is to develop some mathematics with a minimal set theoretical foundation, and also to explore the properties of that set theoretical foundation from its own perspective
(every foundation of mathematics has a take on itself, as it were).

Many proofs which are familiar bits of mathematics are just alluded to here, but they will eventually be filled in:  the idea is that this text should be self-contained.

Part of our plan is to introduce basics of classical mathematics outside of set theory in this terms, at least arithmetic, the construction of the real numbers and the Euclidean plane, and some elementary analysis.

\section{Preliminaries}

\subsection{Basic axioms and definitions}

The system of set theory presented here as a foundation for our mathematical development will appear very minimal.  It is motivated by a pedagogical observation:  there are two sorts of infinity which appear in practice in undergraduate mathematics, and in applicable mathematics generally.  Our axioms combine the usual axioms of impredicative class theory with the assertion that there are only two infinite cardinalities, the cardinality of the infinite sets (which turns out to be countable infinity) and the cardinality of the infinite proper classes (which turns out to be the cardinality of the reals).  The approach taken here may put this purported insight to the test:  if the insight is wrong, applications will push the bounds of the system presented.

The use of sets as a foundation is traditional by now.  There are advantages to another choice of foundational abstraction, such as sequences, and in the reflective mode in this essay we may explore them.  Sets it will be, for now [with a distinction drawn between classes, which are general collections, and sets, which are in addition elements of collections].

So, some objects in our universe are classes (some classes will be said to be sets, as we will see).  We eschew the imperialistic view that {\em all\/} objects are classes, though we do not explicitly postulate any non-classes at the moment.

Classes have elements:  we write $x \in A$ to say that an object $x$ belongs to a class $A$.  We say ``$x$ belongs to $A$". ``$x$ is contained in $A$", ``$x$ is an element (or member) of $A$" for this.
The sentence ``$x$ is included in $A$" means something else.  The sentence ``$x$ is in $A$", though tempting, is ambiguous and we will try to avoid it.

\begin{description}

\item[Axiom of classhood:]  If $x \in A$, then $A$ is a class.  We write ${\tt class}(A)$ for ``$A$ is a class", and write the axiom in symbols thus:  $$(\forall A(\forall x:x \in A \rightarrow {\tt class}(A))),$$

\item[Definition:]   An object $x$ is said to be an {\em element} if there is a class $A$ such that $x \in A$:  in symbols, ${\tt element}(x) \leftrightarrow (\exists A:x \in A)$.

\item[Definition:]  An object is a {\em set\/} if and only if it is a class and an element:  ${\tt set}(x) \leftrightarrow {\tt class}(x) \wedge {\tt element}(x)$.  An object is said to be
an {\em atom\/} if it is an element and not a class:  note that atoms have no elements by the axiom of classhood.

\end{description}

We present the identity criterion for classes.

\begin{description}


\item[Axiom of extensionality:]  Classes with the same elements are equal.  In symbols $$(\forall AB:{\tt class}(A) \wedge {\tt class}(B) \wedge (\forall x:x \in A \leftrightarrow x \in B) \rightarrow A=B).$$

\end{description}

Finally, we present the very simple axiom which provides us with classes.  Any property of elements determines a class.

\begin{description}

\item[Axiom of comprehension:]  Let $P(x)$ be any sentence of our language.  There is a class $\{x:P(x)\}$ such that $$(\forall a:a \in \{x:P(x)\} \leftrightarrow {\tt element}(a) \wedge P(a)).$$

\end{description}

A familiar class can be introduced.

\begin{description}

\item[Definition:]  $\emptyset$, the empty class, is defined as $\{x:x\neq x\}$.  We also define $V$, the universal class, as $\{x:x=x\}$.  The latter is the class of all elements, not the collection of absolutely everything.  We will prove below that the empty class is a set, and thereafter may call it the empty set.

\item[Definition:]  Some basic class constructions.  We do not as yet have axioms about circumstances under which these classes might be sets.

\begin{enumerate}

\item  If $x$ is an element we define $\{x\}$, the singleton of $\{x\}$,  as $$\{y:y=x\}.$$   It is a theorem to be proved below that this is always a set.

\item  If $A$ and $B$ are classes, we define $A \cup B$, the union of $A$ and $B$, as $\{x:x \in A \vee x \in B\}$.  It is a theorem to be proved below that this is a set if $A,B$ are sets.

\item  We define $\{x,y\}$, the unordered pair of $x$ and $y$,  as $\{x\} \cup \{y\}$.  Notice that this is only defined if $x$ and $y$ are both elements, and also that $\{x,x\} = \{x\}$.
It is a theorem to be proved below that this is always a set when defined.

This notation is extended by a recursive definition:  $$\{x_1,x_2,\ldots,x_n\} = \{x_1\} \cup \{x_2,\ldots,x_n\}.$$

\item We define $A \cap B$, the intersection of $A$ and $B$, as $$\{x:x \in A \wedge x \in B\}$$ and $A \setminus B$, the set difference of $A$ and $B$,  as $$\{x:x \in A \wedge x \not\in B\}.$$  We define $\overline{A}$, the complement of $A$,  as $\{x:x \not\in A\}$ (this is of course the collection of elements not in $A$, not the collection of everything not in $A$).

\item We define $(a,b)$, the ordered pair of $a$ and $b$,  as $\{\{a\},\{a,b\}\}$;  the conditions under which this notation is defined are implicit in the definition of singleton sets ($a,b$ must be elements
and $\{a\}, \{a,b\}$ must be elements).  Note that once we prove that unordered pairs are always sets, it will be evident that $(a,b)$ is defined and a set whenever $a,b$ are elements.

\item[Theorem:]  If $(a,b) = (c,d)$ then $a=c$ and $b=d$.

\item[Proof:]  $a$ is the only element which belongs to all elements of $(a,b)$.  $b$ is the only element which belongs to exactly one element of $(a,b)$.
If $(a,b)=(c,d)$, then the only element which belongs to all elements of $(a,b)$ is equal to the only element which belongs to all elements of $(c,d)$ ($a=c$)
and similarly $b=d$.

\end{enumerate}

\item[Definition:]  We say that $A \subseteq B$ iff $$({\tt class}(A) \wedge {\tt class}(B) \wedge (\forall x:x \in A \rightarrow x \in B).$$  We say that $A$ is a subclass of $B$, or that
$A$ is a subset of $B$ if $A$ is a set.  We say that $A$ is a proper subclass of $B$ (or a proper subset if it is a set) if $A \subseteq B \wedge A \neq B$.  It is a theorem to be proved
below that subclasses of sets are sets.

We say ``$x$ is included in $A$" to express this in English.  We are aware of dangers in style and pedagogy of confusion between subset and membership.

\item[Definition:]  We define $A \times B$ as $\{(a,b):a \in A \wedge b \in B\}$ where $A$ and $B$ are classes and every $(a,b)$ with $a \in A$ and $b \in B$ is an element
(this hypothesis will subsequently be shown always to hold).

\item[Definition:]  We define ${\cal P}(A)$ as $\{B:B \subseteq A\}$ where every $B \subseteq A$ (including $A$ itself) is an element.  We will show below that this hypothesis holds whenever $A$ is a set.

\item[Definition:]  We define $\bigcup A$, the union of $A$, as $\{x: (\exists a:a \in A \wedge x \in a)\}$.  It is a theorem, appearing late in our development, that $\bigcup A$ is an element if $A$ is an element; in any case this notation is defined for any class $A$.  Note that $A \cup B = \bigcup \{A,B\}$ if $A$ and $B$ are both elements.

\end{description}

It is by now an old observation that it is a theorem that some classes are not sets.

\begin{description}

\item[Definition:]  $\cal R$, the Russell class, is defined as $\{x:x \not\in x\}$.

\item[Theorem:]  $\neg {\tt element}(\cal R)$ 

\item[Proof:]  Suppose that ${\tt element}(\cal R)$.  We have by axiom that $$\cal R \in \cal R \leftrightarrow {\tt element}(\cal R) \wedge \cal R \not\in \cal R:$$ the hypothesis makes this equivalent to
$$\cal R \in \cal R \leftrightarrow \cal R \not\in\cal  R,$$ which is impossible.

\end{description}

$\cal R$ is then a class which is not a set.
\begin{description}
\item[Definition:]  We say that a class $C$ is a {\em proper class\/} iff it is not a set.
\end{description}

$\cal R$ will not be the mad uncle we hide in the attic:  it actually plays a role as an example in basic proofs in our development.

It is worth noting that our axioms do not so far speak to the question as to whether $x \in x$ is a possible situation.  There is no way to construct such a set explicitly using our axioms, but there is also nothing which rules out the existence of such sets, and they might have uses.

The rest of our axioms govern the relationship between sets and classes.  They are all about the notion of size or cardinality of a set, for which we need to introduce some preliminary definitions.

\begin{description}





\item[Definition:]  If $A$ and $B$ are classes, we say that they are the same size (or equivalent)  and write $A \sim B$  if and only if there is a class $E$ with the following properties:  every element of $E$ is an ordered pair $(a,b)$ with $a \in A$ and $b \in B$, for each element $a$ there is exactly one $y$ such that $(a,y) \in E$, and for each $b$ there is exactly one $x$ such that $(x,b) \in E$.  Such a class is called an equivalence between $A$ and $B$.  [It is the same thing as a bijection between $A$ and $B$, but we have not introduced functions yet.]



\item[Definition:]  A class $C$ is {\em infinite\/} iff $C$ is the same size as a proper subset of $C$.

\end{description}

Please notice that we have not assumed that any ordered pairs exist in making these definitions.  This is part of the fun.

We now assert the rest of our axioms.

\begin{description}
 
\item[Axiom of infinite sets:]  There is an infinite set and all infinite sets are the same size.

\item[Axiom of proper classes:]  All proper classes are the same size, and any class the same size as a proper class is proper.

\end{description}

These axioms are perhaps deceptively vague.  In fact, our development will show that the cardinality of the infinite sets and the cardinality of the proper classes, which are given
center stage treatment here, are very familiar cardinalities indeed, and the very spare axiom set we give allows us to determine exactly what they are and derive all their familiar properties and some controversial ones.

\subsection{Theorems about finite set constructions}

We introduce the name $I$ for an arbitrary infinite set (which is given to us by the axiom of infinite sets).  We use this to prove the existence of another familiar set.

\begin{description}



\item[Theorem:]  $\emptyset$ is not infinite, and for elements $x,y$, $\{x\}$ and $\{x,y\}$ are not infinite.

\item[Proof:]  $\emptyset$ has no proper subclass and so cannot be infinite.  The only proper subclass of $\{x\}$ is $\emptyset$, and there can be no pair $(x,y)$ with $y \in \emptyset$:  such a pair would have to appear as an element
of an equivalence between $\{x\}$ and $\emptyset$.  The proper subclasses of $\{x,y\}$ are $\{x\}, \{y\}$, and $\emptyset$.  An equivalence between $\{x,y\}$ and $\emptyset$ would have to contain an ordered pair $(x,z)$ with $z \in \emptyset$, which is impossible.
An equivalence between $\{x,y\}$ and any singleton class $\{z\}$ would have to include $(x,z)$ and $(y,z)$, but then this would violate the condition that there must be exactly one $u$ such that $(u,z)$ is in the equivalence.  So $\{x,y\}$ cannot be equivalent to any of its proper subclasses.

\item[Theorem:]  $\emptyset$ is a set.

\item[Proof:]  Suppose otherwise.  Then the class $\{I\}$ would be a set, because it cannot be the same size as the supposed proper class $\emptyset$ (quite independently of any beliefs about what pairs might exist).  Now $\{I\}$ cannot be infinite, again because it cannot be the same size as its only proper subset $\emptyset$, and so $\{I\} \not\in \{I\}$,
so $\{I\} \in \cal R$, so the proper class $\cal R$ cannot be the same size as the empty set, which contradicts the axiom of proper classes.

\item[Theorem:]  $\{x\}$ is a set for every $x$ an element.

\item[Proof:]  Suppose for some $x$ that $\{x\}$ is not a set.  Then $\{I,\emptyset\}$ is a set, because it cannot be the same size as $\{x\}$ (its two elements are different).
This set cannot be an element of itself, because it is neither infinite nor empty.  This means that the proper class $\cal R$ has at least two elements $\emptyset$ and $\{I,\emptyset\}$,
and so cannot be the same size as $\{x\}$, which contradicts the axiom of proper classes.

\item[Theorem:]  $\{x,y\}$ is a set for all $x,y$ elements.

\item[Proof:]  $\cal R$ has at least three distinct elements, $\emptyset$, $\{\emptyset\}$, $\{\{\emptyset\}\}$ so cannot be the same size as $\{x,y\}$ for any elements
$x,y$.  This proof generalizes to all concretely given finite sets.

\item[Observation:]  By the previous theorem, the pair $(x,y)$ exists for any elements $x,y$, and any sentence in two variables $R(x,y)$ is represented by a set
$\{(x,y):R(x,y)\}$, so binary relations on elements are encoded as classes, just as properties of elements are.  This also untangles any perplexities about our definition of being the same size which might have been caused by doubts about the existence of pairs:  if we can describe a one to one correspondence between two classes, it will be implemented as a class, and the two classes will be seen to be the same size.

\end{description}

\subsection{Basic theorems about the infinite classes}

An alternative way to prove the existence of pairs and finite sets generally.

\begin{description}

\item[Theorem:]  $\cal R$ is infinite.

\item[Proof:]  The class $\{(x,\{x\}): x \in \cal R\}$ is an equivalence from $\cal R$ to a proper subclass of $\cal R$, the set of singletons of sets which are not elements of themselves.
$\emptyset \in R$ is not in this class.

\item[Alternative Proof (Peter Metis):]  Let $\cal R^+$ be the class of nonempty sets which are not elements of themselves.  If $\cal R^+$ were a set, it would belong to itself if and only if
it was nonempty and did not belong to itself.  And it is nonempty, it contains $\{\emptyset\}$.  So in fact it must be a proper class, and by the axiom of proper classes there is a bijection
from $\cal R$ to its proper subset $\cal R^+$ (proper because $\emptyset \in \cal R$), so $\cal R$ is infinite.

\item[Observation:]  Notice that the alternative proof does not use the theorem on existence of ordered pairs.

\item[Alternative proof that $\{x,y\}$ is a set (Peter Metis):]  Suppose that some $\{x,y\}$ was a class and not a set.  Then there would be an equivalence from $\{x,y\}$ to $\cal R$, which could be used along with the equivalence between $\cal R$ and $\cal R^+$ to define (externally, no commitment to existence of any ordered pairs) an one to one correspondence between $\{x,y\}$ and one of its proper subclasses, which is impossible.  The same argument works for each concretely given finite set.

\item[Theorem:]  Any class equivalent to an infinite class is infinite.

\item[Proof:]  If $A$ is infinite and $E$ is an equivalence from $A$ to a proper subclass of $A$ and $E'$ is an equivalence from $B$ to $A$,
then $$\{(b_1,b_2): (\exists a_1,a_2:(b_1,a_1) \in E' \wedge (a_1,a_2) \in E \wedge (b_2,a_2) \in E')\}$$ is an equivalence from $B$ to a proper subclass of $B$.

\item[Theorem:]  Any class with an infinite subclass is infinite.

\item[Proof:]  Let $A\subseteq B$ be infinite and let $E$ be an equivalence from $A$ to a proper subset $C$ of $A$.  Then the union of $E$ and $\{(x,x):x \in B \setminus A\}$ is an equivalence from $B$ to its proper subclass $(B \setminus A) \sup C$.

\item[Theorem:]  For any class $x$ and element $y$, $x \cup \{y\}$ is infinite if and only if $x$ is infinite.

\item[Proof:]  If $y \in x$ this is trivial.  Suppose $y \in x$ and $x$ is infinite.  An equivalence from $x$ to a proper subclass of $x$ can be extended by adding $(y,y)$ to give an equivalence from $x \cup \{y\}$ to a proper subclass of $x \cup \{y\}$.  Suppose $y \in x$ and $x \cup \{y\}$ is infinite.  Take a map from $x \cup \{y\}$ to a proper subset of $x \cup \{y\}$.  If the proper subset does not include $y$, simply restrict the map to $x$:  it will be an equivalence from $x$ to a proper subset of $x$ (not including the $z$ in the pair $(y,z)$ in the original equivalence).  If the proper subset does include $y$, the equivalence contains pairs $(u,y)$ and $(y,v)$:  replace these with the pair $(u,v)$ and one obtains an equivalence from
$x$ to a proper subset of $x$.

\item[Theorem:]  For any finite set $x$ and element $y$, $x \cup \{y\}$ is finite, and therefore a set (because it is not equivalent to the infinite $R$).  So for example if $x$ and $y$ are elements, $\{x\} \cup \{y\} = \{x,y\}$ is a set,
and this extends to sets presented as concrete lists of elements of any finite length.

\item[Theorem:]  For any infinite class $x$, $x \cup \{y\}$ is the same size as $x$, and so is a set if $x$ is a set.

\item[Proof:]  There is an equivalence $E$ from $x$ to a proper subclass $z$ of $x$.  Choose $u$  in $x$ which is not in $z$.  Let $U$ be the intersection of all classes which contain $u$ and if they contain $v$ and $(v,w) \in E$, also contain $w$.  Let $E'$ contain each $(v,w)\in E$ with $v \in U$, and $(v,v)$ for $v \in x$ which is not in $U$.  Add the pair $(y,u)$ to this to obtain an equivalence from $x \cup \{y\}$ to $x$.

\item[Theorem:]  For any set $x$ and element $y$, $x \cup \{y\}$ is a set.

\end{description}

We show that the universe is a proper class.

\begin{description}
\item[Lemma (Peter Metis):]  There is a subclass of $\cal R$ equivalent to $V$.

\item[Proof:]  $E = \{(x,\{\{x\},\emptyset\}):x=x\}$ is an equivalence between $V$ and $\{\{\{x\},\emptyset\}:x=x\}$, and the latter class is a subclass of $\cal R$:  a set of
the form $\{\{x\},\emptyset\}$ cannot be an element of itself because it has two elements, one of which has one element and one of which has none.

Notice that this immediately implies that $\cal R$ is infinite:  $\cal R$ is equivalent to the set of all $\{\{x\},\emptyset\}$ with $x \in \cal R$, which is a proper subclass of $\cal R$.

\item[Theorem (Peter Metis):]  $\cal R$ is the same size as $V$, so $V$ is a proper class.

\item[Proof:]  Define $h(x)$ as $\{\{x\},\emptyset\}$ for convenience in this proof.

Define $H$ as the intersection of all classes which include $\{x:x \in x\}$ as a subclass and which are closed under application of $h$.


Define $J(x)$ as $\{(x,y): x \in H \wedge y = h(x) \vee x \not\in H \wedge y=x\}$.

$J$ is an equivalence from $V$ to $\cal R$.

This is a concrete instance of the usual proof of the Schr\"oder-Bernstein theorem, which has very little use as a general result in a theory with two infinite cardinalities of classes.


\item[Observation:]  $\cal R$ has infinite subclasses of two different sizes,
because it has a subclass the same size as $V$, and so a subclass the same size as any subclass of $V$.  This observation does not depend on the proof given that $V$ and $\cal R$ are the same size.

\end{description}

\subsection{Introduction to arithmetic, and to implementation of familiar mathematics}

\begin{description}

\item[Definition of a relation:]  A relation is a class of ordered pairs.  We write $x\, R \, y$ for $(x,y) \in R$ when $R$ is a relation.  Notice that equivalences are relations.

The domain ${\tt dom}(R)$ of a relation $R$ is the class $\{x:(\exists y:x \, R \, y)\}$.   The inverse of a relation $R$ is $\{(y,x):x \, R \, y\}$.  The range ${\tt rng(A)}$ of a relation $R$ is ${\tt dom}(R^{-1})$.

\item[Definition of a function:]  A function is a class of ordered pairs $f$ such that for all $x,y,z$, $(x,y) \in f \wedge (x,z) \in f \rightarrow y=z$.  For $x$ in ${\tt dom}(x)$, define $f(x)$ as the unique $y$ such that $(x,y) \in f$.

\item[Definition:]  We say that $R$ is a relation from $A$ to $B$ if $R \subseteq A \times B$.  We write $f:A \rightarrow B$ if $f$ is a function, a relation from $A$ to $B$, and $A$ is the domain of $f$.

Notice that equivalences are functions.

\item[Definition:]  We define $Z$, the set of Zermelo natural numbers, as $$\{n:(\forall C:\emptyset \in C \wedge (\forall u:u \in C \rightarrow \{u\} \in C) \rightarrow n \in C\}.$$

This may not be our favorite representation of the natural numbers, but it is the easiest one for us to define.

\item[Theorem:]  There is a class $\mathbb N$ (which we call the class of natural numbers), an element $0 \in \mathbb N$ and an operation $\sigma$ such that $\sigma(x)$ is defined iff $x \in \mathbb N$, satisfying the following conditions (Peano's axioms for arithmetic).  The class of natural numbers is a set, but it will take us some time to prove that.

\begin{enumerate}

\item $0 \in \mathbb N$

\item $(\forall x \in \mathbb N:\sigma(x) \in \mathbb N)$

\item $(\forall x \in \mathbb N:\sigma(x) \neq 0)$

\item $(\forall x,y \in \mathbb N:\sigma(x) = \sigma(y) \rightarrow x=y)$

\item $$(\forall S \subseteq \mathbb N:0 \in S \wedge (\forall x\in \mathbb N:x \in S \rightarrow \sigma(x) \in S) \rightarrow (\forall x \in \mathbb N:x \in S))$$ [the principle of mathematical induction]

\end{enumerate}

You might expect that we will now define the natural numbers (as, say, the Zermelo natural numbers).  But we will not do that.  We will remain noncommittal about any definition of the natural numbers in terms of set theory, because we do not understand the natural numbers in terms of set theory, but via the properties above.  They might not seem like {\em enough\/} properties, but they are!  (They are only enough properties because we are working in a set theory;  we will give a longer list of properties later to define a less set theory dependent arithmetic).

\item[Proof:]  We suggest but do not adopt a choice of meanings for $\mathbb N$, 0 and $\sigma$ which makes the axioms above true.  But we leave open what classes and constructions we are actually talking about, and we will give at least one other possible assignment of meanings to these symbols before the end of this chapter.

We suggest that we take $\mathbb N$ to be $Z$, 0 to be $\emptyset$, and $\sigma(x)$ to mean $\{x\}$ for each $x \in Z$.

We verify that the translations of the axioms of arithmetic into these terms are true.

\begin{enumerate}

\item $\emptyset \in Z$:  clearly any class which contains $\emptyset$ and is closed under taking singletons has $\emptyset$ as an element, so  $\emptyset \in Z$.

\item $(\forall x \in Z:\{x\} \in Z)$:  if $x \in Z$, then $x$ belongs to every class which contains $\emptyset$ and is closed under singleton, so $\{x\}$ also belongs to any such class, so $\{x\} \in Z$.

\item $(\forall x \in Z:\{x\} \neq \emptyset):$  this is obvious.

\item $(\forall x,y \in Z:\{x\} = \{y\} \rightarrow x=y):$  this is obvious

\item  $$(\forall S \subseteq Z:\emptyset \in S \wedge (\forall x\in Z:x \in S \rightarrow \{x\} \in S) \rightarrow (\forall x \in Z:x \in S)):$$  the conditions imply immediately
that $S$ contains $\emptyset$ and is closed under the singleton operation, from which it follows that $S \supseteq Z$ by the definition of $Z$.

\end{enumerate}

The punchline here is not that we have revealed that $Z$ is $\mathbb N$, $0 = \emptyset$, and successors of natural numbers are singletons of natural numbers.  It is merely that we cannot get into trouble by assuming that there is a class $\mathbb N$, element 0 of this class, and operation $\sigma$ on elements of this class with the stated properties, because if we were challenged we {\em could\/} define them in this way.  Other definitions are possible, and any explicit definition of the familiar notion of natural numbers in terms of sets is not true to the way we actually learn about the natural numbers:  we are leaving open as a possibility the position usual in an undergraduate discrete math text that natural numbers are atoms, not sets at all.

\item[Iteration of maps:]  Suppose that $f:A \rightarrow A$ and $a \in A$.   Define a set $g \subseteq A \times \mathbb N$ as the intersection of all sets $G$ such that  $(0,a)\in G$
and $(\forall x:(n,x) \in G \rightarrow (\sigma(n),f(x)) \in G)$.  This is a function $g: \mathbb N \rightarrow A$ [this statement is the Iteration Theorem], and we define $f^n(a)$ as $g(n)$.

\item[Proof of the Iteration Theorem:]  To be supplied.

\item[Definition:]  Using iteration, we can define operations which the reader may feel are missing from our abstract description of the natural numbers.

\begin{enumerate}

\item  We define $m+n$ as $\sigma^n(m)$.

\item  We define $m \cdot n$ as $(\sigma^m)^n(0)$.

\item  We define $m \leq n$ as $(\exists k:m+k = n)$.

\end{enumerate}

\item[Theorem:]  For any nonempty subclass $S$ of $\mathbb N$, there is an element ${\tt min}(S) \in S$ such that $(\forall s \in S:{\tt min}(S) \leq s)$.

\item[Proof:]  Suppose otherwise.  Then there would be a subclass $S$ of $\mathbb N$ with no minimal element.  We consider the set $B$ of all $k$ such that for all $m \leq k$, $m \not \in A$.
Clearly $0 \in B$:  if $0 \in S$, it would be the nonexistent minimal element.  If $k \in B$, then $k+1 \not\in S$ because it would then be minimal in $S$.  [note that this requires
the lemma that there are no natural numbers between $k$ and $k+1$:  proofs of some standard arithmetic results could be added to the text.]

\item[Theorem:]  Every subset of $\mathbb N$ is either finite or the same size as $\mathbb N$.

\item[Proof:]  Prove by induction on $n \in \mathbb N$ that the class of numbers $\leq n$ is finite and a set.  Then observe that any bounded class of natural numbers
is included in a finite set and so is finite.

Suppose $A \subseteq \mathbb N$ is unbounded.  Define $F(n)$ as the smallest number $>n$ belonging to $A$.  Let $a$ be the smallest element of $A$.  Then the function sending
each $n \in \mathbb N$ to $F^n(A)$ is an equivalence from $\mathbb N$ to $A$.

\item[Theorem:]  $\mathbb N$ is a set, and so is $Z$.

\item[Proof:]  Any infinite subclass of $\mathbb N$ is the same size as $\mathbb N$.  If $\mathbb N$ were a proper class it would be the same size as $\cal R$, which has infinite subclasses of two different sizes.  Thus $\mathbb N$ must be a set.  $Z$ is a set by the translation of this argument into terms of $Z$, $\emptyset$ and the singleton operation:  all properties
we can prove of $\mathbb N$ must have their translations hold of $Z$.

\item[Theorem:]  Any subclass of $\mathbb N$ (or of $Z$) is a set.

\item[Proof:]  If any subclass of $\mathbb N$ were the same size as $\cal R$, it, and therefore $\mathbb N$ itself, would have infinite subclasses of two different sizes.  Notice that this also establishes
(by the Observation above) that $\mathbb N$ is not the same size as $V$, and so that $V$ is the same size as $\cal R$ and a proper class.

The same argument shows these things for $Z$, since $Z$ has all the properties of $\mathbb N$ required to show the result.

\item[Corollary:]  Any subclass of a set is a set, so the power class ${\cal P}(A)$ is defined for any set $A$.   This means that Zermelo's axiom of separation, which asserts
that if $A$ is a set, so is $\{x \in A:P(x)\}$ (defined as $$\{x: x \in A \wedge P(x)\}),$$ holds in our set theory, though it is not central to the way we understand sethood.

One should be careful about how the corollary is proved:  any set is either finite (and so all of its subclasses are finite and so sets) or infinite, so the same size as $\mathbb N$, so all of its subclasses
are either finite or the same size as $\mathbb N$ as shown above, and so sets.  We do {\em not\/} yet know that all finite sets are the same size as subsets of $\mathbb N$.

\end{description}
\section{Toward the real numbers and the plane}

\begin{description}

\item[Theorem:]  The class ${\cal P}(Z)$ of all subsets of $Z$ is not the same size as $Z$.

\item[Proof:]  The Cantor argument.

\item[Theorem:]  ${\cal P}(Z)$ is the same size as $R$, and so the same size as the universe $V$.

\item[Proof:]  It is infinite and not the same size as the infinite set $Z$, so it must be a proper class, so the same size as $R$ and $V$.

\item[Note:]  Lots more to appear here.

\end{description}

\section{Controversial Results}

The first sections have allowed us to identify the two kinds of infinity in this theory.  They are the countable infinity of the natural numbers and the uncountable infinity of the reals.

The results of this section establish that our axioms imply that the uncountable infinity has structural properties which are a matter of controversy.

This requires the development of the concept of infinite ordinal number, which is a bit tricky.

\begin{description}



\item[Theorem:]  If a function $f$ is a set, its domain is a set (easy), its range is a set (it can be shown to be either finite or countably infinite) and so is any subclass of the function which is a function (these are restrictions of the function to subclasses of its domain).



\item[Theorem:]  If $A$ and $B$ are sets, $A \times B$ is a set, and so is each relation from $A$ to $B$.

\item[Proof:]  standard cardinal arithmetic

\item[Definition of a linear order:]  A relation $R$ is a linear order if it is antisymmetric ($x \, R \, y \wedge y \, R \, x \rightarrow x=y$  for all $x,y$), reflexive ($x \, R \, y \rightarrow x \, R \, x \wedge y \, R \, y$ for all $x,y$) , transitive ($x\, R \, y \wedge y \, R \, z \rightarrow x \, R \, z$), and total ($x\, R \, y \vee y \,R\, x$ for all $x,y$)

\item[Definition of a well-ordering:]  A relation $R$ is a well-ordering iff it is a linear order and for any nonempty subset $S$ of ${\tt dom}(R)$ there is an element ${\tt min}_R(S)\in S$ such that $(\forall s \in S:{\tt min}_R(S) \, R \, s)$.  Any finite linear order and the usual order on the natural numbers are examples of well-orderings.

\item[Definition of an isomorphism:]  Relations $R$ and $S$ are isomorphic iff there is an equivalence $E$ between ${\tt dom}(R) \cup {\tt rng}(R)$ and ${\tt dom}(S) \cup {\tt rng}(S)$ such that if $(a,b)$ and $(c,d)$ are in $E$, it follows that $a \, R \, c$ iff $b \, S \, d$.  We write $R \approx S$  to say that $R$ and $S$ are isomorphic.

\item[Theorem:]  Isomorphism is an equivalence relation on relations which are sets.  The equivalence class of a relation $R$ under isomorphism (which is not as a rule a set) is called its isomorphism type.

\item[Construction:]  We build representatives of each equivalence class of well-orderings, which we call ordinal numbers.  We define a representation map on a well-ordering $W$ as a function $f$ such that $$f(w) = \{f(v):v \, W \, w\wedge v \neq w\}.$$  We argue that each well-ordering has one and only one representation map:  suppose that a well-ordering $W$
does not have a unique representation map, and consider the $W$-minimal $w$ such that $$\{(u,v): u \, W \, v \wedge v \, W \, w\},$$ which is a well-ordering, does not have a unique representation map.  Then observe that every $w' \, W \, w$ with $w \neq w'$ does have a unique representation map $\{(u,v): u \, W \, v \wedge v \, W \, w'\}$, from which we
can extract a value $f(w')$, and all representation maps on such sets agree on each value in the shared part of their domains,and the set of all such $f(w')$ can be assigned as $f(w)$, giving a representation map for $\{(u,v): u \, W \, v \wedge v \, W \, w\}$ which is clearly the only possible one.

We then say that $\alpha$ is an ordinal number iff it is in the range of some representation map of a well-ordering.  

Isomorphic well-orderings have representation maps with the same range.  Suppose otherwise (that $W$ and $W'$ are isomorphic and have different representation maps).  Let $E$ be the isomorphism.  Identify the $W$-minimal $w$ such that $(w,w')$ is in the isomorphism and $f(w) \neq f'(w)$.  Then the segment before $w$ in $W$ and the segment before $w'$ in $W'$
with the appropriate restricted well-orderings have representation maps with the same range, which then, contrary to assumption, forces $f(w) = f'(w)$.  Not only do the representation maps have the same range, but the stronger condition holds that the representation map tracks the isomorphism precisely as just shown:  if $(w,w')$ is in the isomorphism,
then $f(w) = f'(w')$.  Further, notice that for any $u,v$ in ${\tt dom}(W)$, $u \, W, v \rightarrow f(u) \subseteq f(v)$.  To complete this picture, argue that $u \neq v \rightarrow f(u) \neq f(v)$:  consider the $W$-minimal $u$ such that $f(u) = f(v)$ for some $v \, W\, u$:  this means that the elementwise image of the segment below $v$ under $f$
is the same as the elementwise image of the segment below $u$ under $f$:  this means that elements of the segment below $u$ which are not in the segment below $v$ must be mapped to values of $f$ which are images of elements of the segment below $v$, which contradicts minimality of $u$.  It follows that there is an isomorphism between any well-ordering $W$
and the inclusion relation on the range of its representation function.

It follows that the collection of ordinal numbers is well-ordered by inclusion.  From this it follows that the collection of ordinal numbers is not a set.  If it were, the inclusion order on it, being a well-ordering, would have a representation function, which is readily seen to be the identity function on its domain, and further a longer well-ordering could be constructed
(append one more element) which would have a representation function sending the new element to an ordinal number which could not be one of the ordinal numbers we already have.  This argument is the version of the Burali-Forti paradox native to this theory.

We can observe after the fact that any set well-ordering has a natural representative among the ordinals  in the following way:  take a well-ordering $W$ whose domain is a set.  Choose something $x$ not in its domain and extend the order to $W^+$  defined by $u \, W^+ \, v \leftrightarrow u \, W^+ \, v \vee v=x$.  That is, we are appending one new element to the end of $W$.
The image of $x$ under the representation function for $W^+$ is the ordinal number with the  inclusion order on its elements  isomorphic to $W$.  This is the sense in which ordinal numbers represent isomorphism classes of well-orderings.


\item[Theorem:]  The collection of all countable ordinals is a proper class.

\item[Proof:] Given above.

\item[Theorem:]  The universe can be well-ordered, and so the Axiom of Choice holds.  Fix a well-ordering of the universe, and let $P$ be any partition.  We can define a class containing one element of each element of $P$ by choosing the first element in the fixed well-ordering of each element of $P$.

\item[Proof:]  The universe is the same size as the proper class of ordinals, which is well-orderable.

\item[Theorem:]  The set of real numbers can be well-ordered.

\item[Theorem:]  Every infinite set of real numbers is either countable or of the cardinality of the continuum.

\item[Theorem:]  For any set $A$, $\bigcup A$ is a set.

\item[Proof:]  This follows because we can use a well-ordering of the universe to choose the first enumeration of each element of $A$, and then the usual argument
that the union of countably many finite and countable sets is finite or countable works.  The proof of this basic set theoretical fact depends on the axiom of choice in this context, so must appear late.

\end{description}

\end{document}