\documentclass[12pt]{article}

\usepackage{amssymb}

\usepackage{graphicx}
\newcommand{\riota}{\rotatebox[origin=c]{180}{$\iota$}}

\title{Parameter-free Zermelo set theory is Zermelo set theory}

\author{M. Randall Holmes}

\date{6/21/2018, 2 pm Boise time, fixed some typos and minor errors.  References to prior art provided:  the details may be original to some extent but the result is not.}

\begin{document}

\maketitle

We present an axiomatization of Zermelo set theory.  NOTE:  typos in the axioms are quite possible!  Please call any you find to my attention!

\begin{enumerate}

\item (Extensionality):  $(\forall xy:(\forall z:z\in x \leftrightarrow z \in y) \rightarrow x=y)$

\item (Pairing):  $(\forall xy:(\exists z:(\forall w: w \in z \leftrightarrow w=x\vee w=y)))$

\item (Union):  $(\forall x:(\exists y:(\forall z:z\in y \leftrightarrow (\exists w:z \in w \wedge w \in x))))$

\item (Power Set):  $(\forall x:(\exists y:(\forall z:z \in y \leftrightarrow (\forall w:w \in z \rightarrow w \in x))))$

\item (Infinity:) $$(\exists x:(\forall y:(y \in x $$ $$\leftrightarrow (\forall w:(\forall u:u \in w \leftrightarrow (\forall pq:p \in u \wedge q \in u \rightarrow p\in w \wedge p=q) \rightarrow y \in w)))))$$

\item (Separation):  For each formula $\phi$ in which $a,y$ are not free, $$(\forall a:(\exists y:(\forall x:x \in y \leftrightarrow y \in a \wedge \phi)))$$

\item (Choice):  $$(\forall x:(\forall yz:y \in x \wedge z \in x $$ $$\rightarrow (\exists u:u \in y) \wedge (\forall u:u \in y \wedge u \in z \rightarrow y=z)) $$ $$\rightarrow (\exists c:(\forall y:y \in x \rightarrow (\exists d:d \in c \wedge d \in y) $$ $$\wedge\, (\forall de: d \in c \wedge e \in c \wedge d \in y \wedge e \in y\rightarrow d=e))))$$


\end{enumerate}

Some of these axioms play no role in the discussion to follow.  We are concerned with how to read the axiom scheme of separation, and in connection with this we will use the axioms of pairing, union, and power set, and extensionality (for unique reference of set notations).

The question about separation which we will address is whether free variables are permitted to appear in the formula $\phi$ determining an instance of the scheme.  If we allow this, we may consider the axiom as implicitly universally closed over these parameters.  If not, we say that we are working in parameter-free Zermelo set theory.  The burden of this note is that this makes no difference in principle (though it certainly does in practice):  parameter-free Zermelo set theory proves the same theorems as full Zermelo set theory.

We first introduce some notational conveniences.

We employ a version of the theory of definite descriptions.  $\phi[(\riota x:\psi)]$ abbreviates $$(\exists x:(\forall y:\psi[y/x]\leftrightarrow y=x) \wedge \psi)
$$  $$\vee (\neg(\exists x:(\forall y:\psi[y/x]\leftrightarrow y=x))\wedge (\forall x:(\forall y:y \not\in x) \rightarrow \psi))$$  This version allows us to introduce additional term notations with definite referents quite freely.  Formulas obtained by eliminating descriptions in different orders will be equivalent.  Note that where a description is not satisfied (there is no unique object such that $\psi$) we take the description to denote the empty set.

We define $\{x : \phi\}$ as $(\riota z:(\forall w:w \in z \leftrightarrow \phi))$.  Note that if $\phi$ does not actually determine an extension which is a set, this will be the empty set.

Define 0 or $\emptyset$ as $\{x:x \neq x\}$.

Define $\{x,y\}$ as $\{z:z=x \vee z=y\}$.  Define $\{x\}$ as $\{x,x\}$ and $(x,y)$ as $\{\{x\},\{x,y\}\}$.

Define 1 as $\{0\}$ and 2 as $\{1\}$.

We define $(x_1,x_2,\ldots,x_n)$ as $(x_1,(x_2,\ldots,x_n))$.

We define $\pi_{1,n}(x)$ as $\pi_1(x)$ and $\pi_{i+1,n}(x)$ ($i< n$) as $\pi_{i,n-1}(\pi_2(x))$, except that $\pi_{i,2}(x)$ is simply $\pi_i(x)$ ($i=1,2$).

Define $\pi_1(x)$ as $(\riota u:(\exists v:x=(u,v)))$.  Define $\pi_2(x)$ as $$(\riota v:(\exists u:x=(u,v))).$$  Notice that a non-pair will have the empty set as both of its projections.

Define $\bigcup A$ as $\{z:(\exists w:z \in w \wedge w \in A)\}$.

Define $x \cup y$ as $\bigcup(\{x,y\})$.

Define ${\cal P}(A)$ as $\{z:(\forall w:w \in z \rightarrow w \in A)\}$.

Define $\{x \in A:\phi\}$ as $\{x:x \in A \wedge \phi\}$.  Correct behavior of a term of this form is provided by the axiom scheme of separation
as long as $x$ does not appear free in the term $A$.

It is important to note that descriptions appearing in the $\phi$ in an abstract $\{x \in A:\phi\}$ should be eliminated in the context $\phi$, not any larger context, so as not to introduce a parameter into the abstract.

We first observe that an instance of separation with parameters (existence of $$\{x \in A:\phi(x,a_1,\ldots,a_n)\})$$ can be reduced to an instance with a single parameter, existence of 
$$\{x \in A:\phi(x,\pi_{1,n}(a),\ldots,\pi_{n,n}(a))\}.$$



So our problem reduces to showing that $\{x \in A:\phi(x,a)\}$, in which $x$ and $a$ are the sole free variables in $\phi$, can be presented
in parameter-free form.

Define $\iota``A$ as $\{z \in {\cal P}(A):(\exists u:z=\{u\})\}$.  This is parameter-free.

For a fixed nonempty set $n$, define $A_n$ as $$\{z \in {\cal P}^2(\iota``A \cup \{\{0,n\}\}):(\exists uv:z=(u,v) \wedge u \neq \{0,n\} \wedge v = \{0,n\})\}.$$  This is parameter-free:  $n$ will in every case be a concrete natural number constant, eliminable as a description.  Note that $A_n$
abbreviates $(\iota``A) \times \{\{0,n\}\}$.

For any sets $A$, $B$, and distinct constants $m,n$, define $A_m \times B_n$ as  $$\{z \in {\cal P}^2(A_m \cup B_n):\pi_2(\pi_1(z)) = \{0,m\} \wedge \pi_2(\pi_2(z)) = \{0,n\}\}.$$  This is parameter-free:  again, $m,n$ are constants not variables.

Define ${\tt dom}(A)$ as $\bigcup \{z \in \bigcup A:(\exists u:z=\{u\})\}$.  This captures domain of a relation.

We can now define $\{x \in A:\phi(x,a)\}$ as $$\bigcup({\tt dom}^2(\{z \in A_1 \times \{a\}_2:\phi(\bigcup \pi_1(\pi_1(z)),\bigcup \pi_1(\pi_2(z))\})).$$


This is not a new result.  The paper at 

{\tt http://www.math.uni-bonn.de/people/schlicht/ZFC\_without\_parameters.pdf}

proves the result for Zermelo as well as for ZFC (or claims to, I have not checked the details but I see no reason to doubt them).  I developed
the present proof by looking at their proof for ZFC and thinking about how to eliminate the applications of Replacement:  I should have looked later in the document!  This paper references a paper of Levy which I conjecture probably also contains a proof of this result.



\end{document}