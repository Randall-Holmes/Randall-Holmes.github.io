\documentclass[12pt]{article}

\title{The simplest model of tangled type theory}

\author{M. Randall Holmes}

\usepackage{amssymb}

\begin{document}

\maketitle

\begin{enumerate}

\item The idea here is to give the simplest possible description of a model of tangled type theory using the smallest values of the cardinal parameters (relative to my current mastery of the techniques involved).

Our working set theory is ZFA, with $\beth_{\omega+1}$ atoms.

\item The model of tangled type theory will have as its domain a symmetric submodel of a natural model of TST (a natural model being one whose membership and equality relations are subrelations of the membership and equality relations of the ambient ZFA) with types $\tau_n$ indexed by the natural numbers, with $\tau_0$ a set of atoms
of cardinality $\beth_{\omega+1}$ and each $\tau_{n+1}$ equal to ${\cal P}(\tau_n) - \{\emptyset\} \cup \{\emptyset_n\}$, where the objects $\emptyset_n$ are atoms distinct from one another and from the atoms in $\tau_0$.  The atoms we have mentioned in the description of the model of type theory  are the only atoms we consider.

\item The aim is that in the eventual model, which will consist of types $\tau_n^* \subseteq \tau_n$ for each $n$, there will be not only the honest membership relation of $\tau^*_n$ in $\tau^*_{n+1}$, but tangled membership relations $\in_{m,n} \subseteq (\tau^*_m \times \tau^*_n)$ for $1<m<m+1<n$; $\in_{n,n+1}$ will be understood as $\in \cap (\tau^*_n \times \tau^*_{n+1})$.
Types 0 and 1 of the original model of type theory will not be types of the model of tangled type theory (just part of the framework).

\item  We describe exactly what it means for the eventual model determined by the $\tau^*_n$'s and the relations $\in_{m,n}$ for $m<n$ to be a model of tangled type theory.  For any strictly increasing sequence $s$ of natural numbers greater than one and sentence $\phi$ of the language of TST, define the sentence $\phi^s$ as obtained by replacing
each quantifier over type $n$ with a quantifier over $\tau^*_{s_n}$, reading equality as equality, and reading each occurrence of membership of a type $n$ object in a type $n+1$ object as $\in_{s_n,s_{n+1}}$.  What is required to have a model of tangled type theory is that for each sequence $s$ and each universal closure $\phi$ of an axiom of TST, we get $\phi^s$ true.

\item We achieve this by arranging an injective embedding of subsets of $\tau^*_m$ into $\tau^*_n$ for each $m,n$ where $1<m+1<n$, then further arranging an injective embedding of all of $\tau^*_n$ into
the set of representations of subsets of $\tau^*_m$ in the higher type $\tau^*_n$ with the intention that this embedding will be a set function in the final model of type theory, so that
in fact the collection of subsets of $\tau^*_m$ represented in $\tau^*_n$ is of the same size as $\tau^*_n$ in terms internal to the model of type theory, and so a bijection can be used to send the representations of subsets of
$\tau^*_m$  precisely onto $\tau^*_n$.  It is worth noting that the collections of subsets of $\tau^*_m$ represented in different $\tau^*_n$'s should not be expected to be the same, nor should any of these be expected to be the same as $\tau^*_{m+1}$.

\item We introduce machinery designed to effect these representations and embeddings.  $\tau_0=\tau^*_0$ is partitioned into $\beth_{\omega+1}$ sets of size $\omega_1$ which we call litters.
We call a set small if it is finite or countably infinite.  $\tau^*_1$ consists of all subsets of $\tau_0$ with small symmetric difference from a small or co-small union of litters.  Notice that $\tau^*_1$ is of cardinality $\beth_{\omega+1}$. For any litter $L$ in type 1,
the local cardinal $[L]$ is defined as the set of all subsets of $\tau_0$ with small symmetric difference from $L$, an element of type 2.

We now describe the scaffolding for our embedding machinery in higher types.  Notice that we already have a collection of $\beth_{\omega+1}$ local cardinals in type 2, which are pairwise disjoint sets.
We divide the collection of $\beth_{\omega+1}$ local cardinls which we assume we have constructed in type $n+2$, and which we assume is a pairwise disjoint collection,  into two compartments of the same size, the free local cardinals and the reserved local cardinals.
We fix a partition of the reserved  local cardinals into sets of size $\omega_1$ called litters.  A set of reserved local cardinals of the same type 
with small symmetric difference from a litter is called a near-litter.  The local cardinal of a litter $L$ of type $n+2$ is {\bf not} the set of near-litters with small symmetric difference from $L$
(which would be in type $n+4$) but the set of set unions of (near-litters with small symmetric difference from $L$) (which are in type $n+3$).  Note that any set of local cardinals uniquely determines and is determined by its set union, since local cardinals are assumed pairwise disjoint.  Further, observe that the local cardinals of type $n+3$ just defined are pairwise disjoint and there are $\beth_{\omega+1}$ of them, so the construction can continue.

We anticipate that local cardinals, near-litters, and set unions of near-litters in types above type 1 will all belong to the appropriate sets $\tau^*_n$.

For any near-litter $N$ we define $N^{\circ}$ as the litter with small symmetric difference from $N$.

\item  We define $K_n$ as the set of free local cardinals in type $n$ ($n\geq 2$).  We partition $K_n$ into countably many compartments $K_{m,n}$ all of the same cardinality $\beth_{\omega+1}$.

\item We partially describe our embedding of subsets of $\tau^*_m$ into $\tau^*_n$ for $1<m<m+1<n$.  We provide a bijection $\sigma_{m,n}$ from $\tau^*_m$ to $K_{m,n}$ (note that here $m<n-1$) without any actual commitment to what subsets of $\tau^*_m$  we will represent,
and provide a provisional representation of any $A \subseteq \tau^*_m$ as $\bigcup (\sigma_{m,n}``A)$ ($A$ is actually represented if $\bigcup (\sigma_{m,n}``A)$ is actually in $\tau^*_n$).  We have to expect each
$\tau^*_m$ to be of size $\beth_{\omega+1}$ in the sense of the ambient ZFA (so we have to be excluding almost all sets in the original $\tau_n$'s from the types $\tau^*_n$ of our final model).

It is useful to note in both this paragraph and the following one that the construction of these embeddings takes place before $\tau^*_n$ is defined (all $\tau^*_m$ for $m<n$ are presumed already constructed).

\item We further need to give a preliminary description of how we map every set in ${\cal P}(\tau^*_{n-1})$ (and so in $\tau^*_n$) to a set of the form $\bigcup (\sigma_{m,n}``A)$ for $A \subseteq \tau^*_m$.  We do this by providing a bijection $\tau_{n-1,m+1}$ from $\tau^*_{n-1}$ to local cardinals in a compartment $K_{n-1,m+1}$ (note that $n-1 \geq m+1$, so we will be using different compartments than in the construction of the  previous paragraph) in the local cardinals in $\tau_{m+1}$.  We then map any element
$A$ of ${\cal P}(\tau^*_{n-1})$ to $\phi_{m,n}(A)= \bigcup (\sigma_{m,n}``(\bigcup(\tau_{n-1,m+1}``A)))$, and we further define a bijection $\phi^*_{m,n}$ from ${\cal P}(\tau^*_{n-1})$ onto the collection of unions of subcollections of $K_{m,n}$:  map each element of $\tau^*_n$ which is not such a union to its image under $\phi_{m,n}$, along with each of its iterated images under $\phi_{m,n}$,
and fix all elements of $\tau^*_n$ which are not iterated images under $\phi_{m,n}$ of elements of $\tau^*_n$ which are not unions of subcollections of $K_{m,n}$.  

We do expect that iterated images of an $x \in {\cal P}(\tau^*_{n-1})$ under $\phi_{m,n}$ will be in $\tau^*_n$ iff $x$ is in $\tau^*_n$.

\item  The intention is that that for $1<m<m+1<n$, we have $x \, \in_{m,n} \, y$ defined as $\sigma_{m,n}(x) \subseteq \phi^*_{m,n}(y)$.

\item  We review our needs when we set out to define $\tau^*_n$.  We may assume that we have defined $\tau^*_m$ for each $m<n$.  Thus we can assume that we have already defined $\sigma_{m,p}$ for $m<{\tt min}(n,p-1)$, since what we need to know
to define this map is the extent of $\tau^*_m$ (which we suppose already determined) and the extent of $K_{m,p}$, all of these sets being defined before the process starts, and that we have already defined $\tau_{p-1,m+1}$ for $p \leq n$ and $m \leq p-1$, since these require knowledge
of the extent of $\tau^*_{p-1}$ (assumed already determined) and the extent of $K_{p-1,m+1}$, which was done already before the construction of types started.

\item This all looks perfectly insane.  The reason that we expect it to work has to do with the sense in which we only accept symmetric sets into our models.

The action of a permutation $\pi$ of $\tau_0$ is extended to higher types by the rule $\pi(A) = \pi``A$ (where $\pi``\emptyset_n = \emptyset_n$).

A permutation is understood for our purposes always to be extended to all types.   We may only be interested in its action on an initial segment of the types.

A $k$-allowable permutation (the sort of permutation we need to define the symmetry appropriate to elements of $\tau^*_k$) is a permutation $\pi$ which 

\begin{enumerate}

\item sends local cardinals to local cardinals.  It may be sufficient to require this for types with index $\leq n$, but we require it for the moment at all types.

\item for $m \leq n-1$ and $n\leq k$ satisfies the condition that the function $\sigma_{m,n}^{-1} \circ \pi \circ \sigma_{m,n}$ is the restriction to $\tau^*_m$ of  an $m$-allowable permutation $\pi_{m,n}$  (not necessarily agreeing with $\pi$ anywhere; note the reduction in index).  We further define $\pi_{m,m+1}$ as $\pi$.  Determination of $\pi_{m,n}$ from its action on $\tau^*_m$ will be seen to be easy because it will be evident soon that all iterated singletons of atoms in $\tau_0$ are in the appropriate $\tau^*_n$'s.

\item satisfies $\pi(\phi_{m,n}(x)) = \phi_{m,n}(\pi(x))$ for each $n \leq k$ and all appropriate $m$ (and so similarly will satisfy $\pi(\phi^*_{m,n}(x)) = \phi^*_{m,n}(\pi(x))$.

\end{enumerate}

\item  We describe the set of derivatives of a $k$-allowable permutation $\pi$.  Let $A$ be a finite set of natural numbers not containing 0 or 1 whose largest element is $k$.  We define $\pi^A$ as $\pi^{A \setminus \{{\tt min}(A)\}}_{{\tt min}(A),{\tt min}(A \setminus \{{\tt min}(A)\})}$.  Note that if the two smallest elements of $A$ are successive, $\pi^A = \pi^{A \setminus \{{\tt min}(A)\}}$.  The set of all $\pi^A$ defined in this way is the set of derivative permutations of $A$ (as a $k$-allowable permutation).

\item  We deduce from the condition on $\phi$ functions in the definition of allowable permutations an equivalent condition on $\tau$ functions.

$\pi(\phi_{m,n}(\{x\})) = \pi(\bigcup (\sigma_{m,n}``(\bigcup(\tau_{n-1,m+1}``\{x\})))) = \pi(\bigcup(\sigma_{m,n}``\tau_{n-1,m+1}(x))$

is to be equal to

$\phi_{m,n}(\{\pi(x)\}) = \bigcup (\sigma_{m,n}``(\bigcup(\tau_{n-1,m+1}``\{\pi(x)\}))) = \bigcup(\sigma_{m,n}``\tau_{n-1,m+1}(\pi(x)))$

from which it follows that $\pi(\bigcup(\sigma_{m,n}``\tau_{n-1,m+1}(x))=\bigcup(\sigma_{m,n}``\pi_{m,n}(\tau_{n-1,m+1}(x))$

whence $\tau_{n-1,m+1}(\pi(x)) = \pi_{m,n}(\tau_{n-1,m+1}(x))$.  This argument is reversible:  this condition on $\tau$ functions is equivalent to the original condition on $\phi$ functions.

\item We can now describe $\tau^*_n$.  It is the set of all elements of ${\cal P}(\tau^*_{n-1})$ which have a small $n$-support $S$ relative to the $n$-allowable permutations.  We define what this means.
An $n$-small support is a well-ordering of pairs $(x,A)$, where $x$ is an atom in $\tau_0$ or a reserved local cardinal or an element of a local cardinal and $A$ is a finite set of natural numbers whose largest element is $n$ and whose smallest element is the type of $x$ (these sets may contain 0 or 1 if we are dealing with an atom or type 1 litter, and the conventions for successive smallest elements of $A$ let us interpret $\pi^A$ in this case).  A set $B$ has $n$-support $S$ iff for each $n$-allowable permutation $\pi$, $\pi(B)=B$ if for each $(x,A) \in {\tt dom}(S)$, $\pi^A(x)=x$.

Notice that an $n$-support can easily be promoted to an $n+1$-support.

\item This completes the definition of the structure.  There remains the question of whether there is such a structure, and whether it is, as claimed, a model of tangled type theory.
To verify its existence, it is sufficient to show that there are no more than $\beth_{\omega+1}$ elements in $\tau^*_n$ for each $n$:  this is all that is required to construct the needed $\sigma$ and $\tau$ maps at each stage.  The verification that it is a model of tangled type theory is straightforward, given the existence of the structure, and we will defer this until we have established the existence of the structure.

There may need to be additional technical conditions imposed on the construction of the $\sigma$ and $\tau$ maps to get nice properties of the permutation group.





\item  Our tool for verifying the existence and properties of the structure is a careful analysis of supports and orbits.

The number of litters in any type is clearly $\beth_{\omega+1}$.  The number of near-litters is also $\beth_{\omega+1}$:  for this it is important that the cofinality of $\beth_{\omega+1}$ is uncountable.

It follows that the number of supports is $\beth_{\omega+1}$.

\item Suppose an object $x$ has support $S$.  We intend to write $x$ as $\chi(S)$ in a way which satisfies the identity $\pi(\chi(S)) = \chi(\pi(S))$ for each allowable permutation $\pi$.  The refinement is needed
here that formal application of $\pi$ to a support $S$ replaces each element $(x,A)$ of its domain with  $(\pi^A(x),A)$. 

We verify that if $x$ has support $S$, $\pi(x)$ has support $\pi(S)$ as just defined.  We need to show that if $\rho^A$ fixes $\pi^A(x)$ for each $(x,A) \in S$, then $\rho$ fixes $\pi(x)$.
$(\pi^{-1})^A \circ \rho^A \circ \pi^A$ fixes $x$ for each $(x,A) \in S$, so $\pi^{-1}\circ\rho\circ \pi$ fixes $x$, so $\rho$ fixes $\pi(x)$.

We want to verify that the function $\chi$ is well-defined.  It should be defined for any $T$ in the orbit of $S$ by $\chi(T) = \pi(x)$ where $\pi(S)=T$, and it remains to note that if $\pi(S) = \pi'(S)$ then
$\pi(x) = \pi'(x)$, because $\pi^{-1} \circ \pi'$ fixes $S$. 

We call the function $\chi$ just defined a coding function and denote the specific one by $\chi_{x,S}$.

\item We consider a special type of support (a strong support) and argue that every object has a strong support.

Define $A_1$ as $A \setminus \{{\tt min}(A)\}$, where $A$ is a nonempty finite set of natural numbers.

Recall that supports are well-orderings.  A $k$-strong support $S$ is a $k$-support in which 

\begin{enumerate}

\item each domain element $(x,A)$ where $x$ is an atom  is preceded in $S$ by $(N,A_1)$, where $N$ is a near litter containing $x$, unless $x$ is not contained in any near litter in the domain of $S$, 

\item and each domain element $(x,A)$ where $x$ is a reserved local cardinal is preceded in $S$  by $(\bigcup N,A)$, where $N$ is a near-litter containing $x$, unless $x$ is not contained in any near-litter in the domain of $S$, 

\item and each domain element $(N,A)$ where $N \in \sigma_{m,n}(x)$ $(n \leq k)$ is preceded in $S$ by each element of a collection of  pairs $(x,A_1\cup B)$ where the pairs $(y,B)$
make up an $m$-support of $x$, 

\item and each domain element $(N,A)$ where $N \in \tau_{n+1,m-1}(x)$ and $m,n$ are the two smallest elements of $A$ is preceded in $S$ by each element of a collection of  pairs $(y,A_1 \cup B)$ where
the pairs $(y,B)$ make up an $n-1$-support of $x$.  

\item A pair $(N,A)$ with $N$ belonging to a reserved local cardinal $x$ must be preceded in $S$ by $(x,A_1)$ if $(x,A_1)$ is in the domain of the support (which will for example not be true if $x$ is of type $n+1$).  

\item No obligations are incurred by elements $(N,A)$ in the domain of the support for local cardinal elements $N$ which belong to free local cardinals  but do not meet any of the conditions described above.

\end{enumerate}

We claim that every object in each $\tau^*_n$ has an $n$-strong support:  this is straightforward, using the inductive hypothesis that for each $m<n$, each element of $\tau^*_m$ has
an $m$-strong support.  We start with any support and insert the required additional items to meet the conditions above (and make necessary changes in order);  supports that have to be inserted are always of lower index.

\item  From a strong support, we get an abstract specification of a strong support.  This is obtained by replacing each element of the domain of the support with an item representing
its kind and relation to other items in the support.

\begin{enumerate}

\item each domain element $(x,A)$ where $x$ is an atom  is replaced with $(1,\alpha,A)$, where $\alpha$ is the ordinal position in the support of $(N,A_1)$ with
$x \in N$, or $\omega_1$ if there is no such item.

\item  each domain element $(x,A)$ where $x$ is a reserved local cardinal is replace with $(2,\alpha,A)$, where $\alpha$ is the ordinal position in the support of $\bigcup N,A_1)$ with
$x \in N$, or $\omega_1$ if there is no such item.

\item Each domain element $(N,A)$ where $N \in \sigma_{m,n}(x)$ $(n \leq k)$ is replaced with $(3,\chi_{x,A'},A)$, where $A'$ is the $m$-support made up of all $(y,B\setminus A_1)$ where $(y,B)$ appears earlier in the support and $B \setminus A_1$ has largest element $m$,

\item and each domain element $(N,A)$ where $N \in \tau_{n+1,m-1}(x)$ and $m,n$ are the two smallest elements of $A$ is replaced with $(3,\chi_{x,A'},A)$ where $A'$ contains all $(y,B \setminus A_1)$ where $(y,B)$ appears earlier in the support and the maximum element of $B \setminus A_1$ is $n-1$.

\item A pair $(N,A)$ with $N$ belonging to a reserved local cardinal $x$ is replaced with $(5.\alpha,A)$, where $\alpha$ is the ordinal position in the support of $(x,A_1)$ with $N \in x$ or $\omega_1$ if there is no such item.

\item No obligations are incurred by elements $(N,A)$ in the domain of the support for local cardinal elements $N$ which belong to free local cardinals  but do not meet any of the conditions described above:  such items $(N,A)$ are replaced with $(5,A)$.


\end{enumerate}

\item  Our aim is now to state and prove a result about the freedom of action of $n$-allowable permutations.

Define an $n$-local bijection as a system of maps $\pi^A_0$ on atoms and reserved local cardinals such that each map is injective, each map respects types, and each map has the same domain and range which have small intersection with each litter.

We also provide a system of maps $\pi_{L',M'}$ for each pair of co-small subsets $L'$, $M'$ of distinct litters $L,M$ of the same type.  $\pi_{L',M'}$ is a bijection from $L'$ to $M'$ in each case (this could be global, set once and for all).

Our claim is that an $n$-local bijection can always be extended to an $n$-allowable permutation.

We prove this by induction, assuming that we have already showed the result for all $m<n$.

Choose an $n$-local bijection with component maps $\pi^A_0$.  We describe our method of computing the value of $\pi^A$ at any atom or reserved local cardinal.  If the atom or reserved local cardinal is in the domain
of $\pi^A_0$ then $\pi^A$ agrees with $\pi^A_0$ at the given atom or reserved local cardinal, and we are done.  It also may be the case that $\pi^A$ at the designed atom or reserved local cardinal
has already been computed.

Otherwise, declare $\pi^A$ as fixed at the atom or reserved local cardinal under consideration, then compute the orbit of $\pi^A$ at the local cardinal $[L]$ of the litter $L$ containing the
atom or reserved cardinal under consideration, then compute the orbit of $\pi^A$ at each other atom or reserved cardinal which is in the litter.  For each integer $i$ (negative ones and zero included!)
let $L_i'$ be the co-small subset of the litter belonging to $(\pi^A)^i([L])$ on which $\pi^A$ is not yet defined and let $M_i'$ be  the co-small subset of the litter belonging to $(\pi^A)^{i+1}([L])$ on which $\pi^A$ is not yet defined, and extend $\pi^A$ to agree with $\pi_{L',M'}$.

It remains to describe our computation of $(\pi^A)^i([L])$.  If $[L]$ is a reserved local cardinal, we simply choose values (which will be reserved local cardinals)  in such a way as not to conflict with values of $\pi^A$ already chosen.  For each of these, we then compute values for the local cardinal of the litter containing them, which will iterate all the way up to free local cardinals.

If $[L]$ is $\sigma_{m,n}(x)$, we choose an $m$-support $S$ of $x$ and upgrade it to an $A_1 \cup \{m\}$-support.  We then proceed to compute all values of $\pi^B$'s needed to compute
$(\pi^{A_1 \cup \{m\}})^i(S)$ for each $i$.  Notice that all our recursive computations are of $\pi^B$'s with smaller minimum than our original $A$.  When computing values at a near-litter, handle
the atoms in $N \Delta N^{\circ}$ and the litter $N^{\circ}$ separately.  Handle $N^{\circ}$ by choosing an atom in it to handle;  our procedure handles the rest of the litter subsequently.  Then construct
a local bijection which computes all the $(\pi^{A_1 \cup \{m\}})^i(S)$'s from $S$, which we can do because we can build a local bijection forcing any strong
support to another strong support with the same abstract specification.  Extend this local bijection to an allowable permutation $\pi^*$ (we can do this because this is in effect an $m$-permutation
and $m<n$).  Define $(\pi^A)^i(\sigma_{m,n}(x))$ as $\sigma_{m,n}((\pi^*)^i(X))$.  This works because $\pi^*$ agrees with our eventual complete $\pi^{A_1 \cup \{m\}}$ on an $A_1 \cup \{m\}$-support of $x$.

If $[L]$ is $\tau_{n-1,m+1}(x)$ and $m,n$ are the two smallest elements of $A$, we do as above except that we are choosing an $n-1$-support of $x$ and upgrading it to an $A_1 \cup \{n-1\}$-support.
We eventually define $(\pi^A)^i(\tau_{n-1,m+1}(x))$ as $\tau_{m,n}((\pi^*)^i(x))$, where $\pi^*$ is in effect an $n-1$-permutation approximating our target $n$-permutation $\pi^A$, but in any case an object which exists by inductive hypothesis.

If $[L]$ doesn't meet any of the conditions above fix it.

This procedure  will define our desired $\pi$ and all of its $\pi^A$'s.  Notice that we ensure in the course of our computation that we never make more than a small number of assignments of
exceptional values in a litter, because as soon as we assign one (fixed) exceptional value in a litter we fill in all values in the litter in a non-exceptional way, and because we build entire orbits of the permutation all at once.  If we don't succeed in defining the entire permutation in the course of computing the value for one atom, simply start the process again with another atom (suppose a well-ordering of all atoms; proceed to the next one at which a value has not been computed.

The permutation we construct has a useful technical property.  All exceptional actions of $\pi^A$'s (where elements of a litter  are sent by the permutation or its inverse to an unexpected litter) are either fixed
or actions of $\pi^A_0$.












\end{enumerate}

\end{document}