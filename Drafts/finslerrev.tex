\documentstyle{article} 

\title{Review of ``Finsler Set Theory:
Platonism and Circularity'', David Booth and Renatus Ziegler, eds.}

\author{M. Randall Holmes}

\begin{document}
\maketitle
This book is a collection of translations of papers by Paul Finsler on
set theory, with some additional material prepared by the editors.

In the translated papers, Finsler expresses his views on philosophy of
set theory and philosophy of mathematics generally.  He discusses the
paradoxes of self-reference from a philosophical standpoint.  He also
analyzes the paradoxes of set theory from a more mathematical
standpoint, with the aim of identifying the error of reasoning which
led to the paradoxes.

Finsler proposes a set theory of his own which he believes will avoid
the paradoxes.  The consensus of modern set theorists is that
Finsler's set theory is incoherent; the reviewer concurs, and thereby
takes issue not only with Finsler himself but with the editors of the
volume.  We state Finsler's axioms and discuss them:


\begin{description}

\item ``We consider a system of things, which we call {\em sets\/}, and
a relation, which we symbolize by $\beta$. The exact and complete
description is achieved by means of the following axioms.

\item[I.  Axiom of Relation:] For arbitrary sets $M$ and $N$ it is
always uniquely determined whether $M$ possesses the relation $\beta$
to $N$ or not.

\item[II.  Axiom of Identity:]  Isomorphic sets are identical.

\item[III.  Axiom of Completeness:] The sets form a system of things
which, by strict adherence to the axioms I and II, is no longer
capable of extension.

That is, it is not possible to adjoin further things in such a way
that the axioms I and II are satisfied.''

\end{description}

The relation $\beta$ is the converse of the usual membership relation;
it might be read ``contains''.  Axiom I asserts that the extensions of
sets must be well-defined.  Axiom II needs explanation (it is
explained in Finsler's paper some time after it is stated).  The idea
is that the identity of a set should be completely determined by the
isomorphism class of the relation $\beta$ restricted to the transitive
closure of the set (i.e., the collection of its elements, the elements
of its elements, and so forth).  A consequence of this axiom is the
usual axiom of extensionality (sets with the same elements are the
same) but axiom II is stronger than extensionality: for example, it
implies that two sets which are their own sole elements must be equal.
It is an anti-foundation axiom of the kind studied by Peter Aczel in
his book {\em Non-well-founded sets\/}: Aczel includes Finsler's axiom
among those that he considers.

The problem with Finsler's theory is his Axiom III.  Literally, what
it says is that the universe of sets is a maximal structure satisfying
axioms I and II.  The only way that a structure satisfying axioms I
and II can be maximal is for its domain to be the universal class;
otherwise, as Baer observed, one could take a new object and assign to
it the extension of the Russell class of the purported maximal
structure, thus extending it.

Finsler reads his axiom as implying that all consistent set
definitions will be satisfied in his theory.  But this is not
possible, as Specker pointed out: the sets $\{x \mid x = x\}$ (the
universe) and $\{x \mid x \not\in x $ and for some $y, x \in y\}$
(the class of all elements which are not elements of themselves) can
each be satisfied in a model of axioms I and II, but they cannot both
be present in a single model of axioms I and II.  The second set
cannot be an element, and so cannot coexist with the set of all sets,
which must have every set as an element.

Both Finsler and the editors attempt objections to Specker's
refutation, but I find these objections unintelligible.

Further, the model theory of axioms I and II does not support
Finsler's assertions about the consequences of axioms I-III.  Finsler
claims that the existence of the universal set is a consequence of his
axioms, but the reviewer has shown that any maximal model of axioms
I-II with a universal set can be converted to a maximal model of
axioms I-II without a universal set.

Finsler goes on to introduce a further concept, which he does not
support using reasoning based on axioms I-III.  He argues convincingly
that the problem with the set-theoretical paradoxes is that the
definitions of the paradoxical sets are in some sense ``circular''.
The new concept he defines is that of a ``circle-free'' set.  He
points out that a ``circle-free'' set certainly will not appear in its
own transitive closure (it will not itself be one of the components
from which it is constructed).  But this is not enough; the collection
of all sets which do not appear in their own transitive closures is
itself paradoxical and so ``circular'' (think about whether it is an
element of itself!)  Finsler concluded that the correct definition of
``circle-free'' is

\begin{description}

\item[Definition:]  A set is {\em circle-free\/} if it does not belong to its own transitive closure and its definition does not refer to the concept ``circle-free''.

\end{description}

This is a ``circular'' definition, of course.  He proceeds to develop
the consequences of this definition, very largely independently of
axioms I-III.  The fascinating thing is that the consequences that he
develops are basically those of the set theory of Ackermann (for a
full description of this theory the reader is referred to the article
by Azriel Levy referenced below).  Ackermann's theory has sets and
classes.  The comprehension principle given for classes is that an
arbitrary condition serves to define a class {\em of sets\/} (though
there may and indeed must be classes which have non-set elements).
The comprehension principle for sets is that any class of sets which
can be defined without reference to the concept of sethood (and
without non-set parameters) is a set.  The similarity to the
definition of the circle-free sets should be clear.  The similarity
extends to detail; every axiom of Ackermann's theory is reflected by a
conclusion of Finsler's about circle-free sets, and Finsler
anticipated proofs of Ackermann in detail, including the (perhaps
surprising) proof of the axiom of infinity.  Ackermann's set theory is
consistent iff {\em ZFC\/} is consistent, so this much of Finsler's
work can be put on a sound foudnation.

Our conclusion about Finsler's set theory is that, while the entire
theory is untenable (and difficult to understand) the ideas behind
axiom II and the notion of ``circle-free'' sets prove to be sound.
Finsler appears to have had good intuition.  Space forbids discussion
of Finsler's other contributions evident from the papers in this
volume, except to say that we find his thoughts about paradoxes of
self-reference to be interesting, his thoughts about the distinction
between sets and classes to be important, and his defense of
mathematical Platonism to be admirable.  We think that making these
papers available in translation is a service to scholarship.

We found the supporting materials prepared by the editors to be
unsatisfactory in most cases, with the notable exception of calling
attention to the relation to the set theory of Ackermann.  This
material called for the services of editors with a clearer
understanding of the mathematical issues involved.  The editors
attempt to defend the coherence of Finsler's axiom III against the
objections of Baer, Specker, and others; no such defense is possible.
The paper by Ziegler on paradoxes of self-reference presents a
``solution'' to paradoxes of self-reference which makes it impossible
for two letters in an algebraic expression to refer to the same
object!  (Finsler's approach is much more sensible.)  The editors
should have presented an explanation of the real relation of Finsler's
derivation of a ``formally undecidable proposition'' to the later work
of G\"odel; they do not.  (Finsler does not actually succeed in
presenting a formally undecidable proposition, though he is on the
right track; his ``proposition'' would need a truth predicate,
forbidden by Tarski's theorem, to actually be expressible in his
language).

The combination of the fascinating but ultimately untenable set
theoretical claims of Finsler and the unsatisfactory support provided
to the reader by the editors led the reviewer to write an extended
review of the book, which can be found on his Web page, {\tt
http://math.idbsu.edu/faculty/holmes.html}

Aczel, Peter, {\em Non-well-founded sets\/}, CSLI, Stanford, 1988.

Booth, David and Ziegler, Renatus, {\em Finsler Set Theory:  Platonism and Circularity\/}, Birk\"auser-Verlag, Basel, 1996.

Holmes, M. Randall, ``Review of ``Finsler Set Theory:
Platonism and Circularity'', David Booth and Renatus Ziegler, eds.'', unpublished, available at {\tt
http://math.idbsu.edu/faculty/holmes.html}

Levy, Azriel, ``The role of classes in set theory'', in M\"uller,
Gert, ed., {\em Sets and Classes\/}, North Holland, Amsterdam, 1976.
See pp. 207-212 on Ackermann's theory.



\end{document}







