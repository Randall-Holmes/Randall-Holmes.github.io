\documentclass[12pt]{article}

\title{Manual for CL Watson}
  
\author{M. Randall Holmes}

\begin{document}

\maketitle

\section{Introduction}

CL Watson is a version of the Watson theorem prover.  Watson is an
interactive equational higher-order prover whose higher-order logic is
based on the weakly extensional version {\em NFU\/} of Quine's set
theory ``New Foundations'', presented as a stratified lambda-calculus.

CL Watson is based on combinatory logic instead of lambda-calculus.
The intention is to implement stratified lambda-calculus without
actually explicitly implementing either stratification or
lambda-calculus.  Part of the intention is that all manipulations of
terms made by the computer represent logical moves which are directly,
locally justifiable without any appeal to such nonlocal conditions as
stratification of terms or position in terms defined by cases.

CL Watson uses variable binding notation of the usual kind, unlike
Watson.  It uses the metaphor of navigation in terms to handle
application of theorems (which are all understood as rewrite rules).
Navigation is extended to include navigation to formal values as well
as to subterms: this is how ``lambda-terms'' are implemented.  A term
$(\lambda x.T)$ has no subterm of the form $T$: it is a combinator
which evaluates to $T$ when applied to the variable $x$.  The display
functions of the prover recognize the term as an ``abstraction'', and
so display it as a $\lambda$-term, and the navigation functions allow
one to navigate to the ``subterm'' $T$, but what actually happens when
$T$ is rewritten involves abstraction from the rewritten form.

\section{The Logic}

The basic logical rules are those of equational logic which are
assumed to be familiar.  All theorems are equations, supposed
universally quantified over any variables present.

Primitives are constants {\tt Abst}, {\tt Id}, and {\tt Eq}.  If $T$
and $U$ are terms, then $T(U)$ is a term (the application of $T$ to
$U$) and $|T|$ is a term (the constant function with value $T$).

The combinatory logic axioms proper are

\begin{description}

\item[Identity:] ${\tt Id}(x)=x$

\item[Constant:] $|T|(U) = T$

\item[Distribution:] ${\tt Abst}(T)(U)(V) = T(|V|)(U(V))$

\item[Constant Expansion:]  $|T(U)| = {\tt Abst}(|T|)(|U|)$

\end{description}

These axioms support a meta-theorem.

\begin{description}

\item[Definition:] If $A$ is a set of integers, define $A+1$ as the
set of all successors of elements of $A$ and $A-1$ as the set of all
predecessors of elements of $A$.  Where $U$ is a term and $T$ is a
term, define ${\tt type}(U,T)$, a set of integers, as follows.  ${\tt
type}(T,T)=\{0\}$.  ${\tt type}(U,T) = {\cal Z}$ (the set of all
integers) if $T$ is atomic and $U$ is distinct from $T$.  ${\tt
type}(U,|T|) = {\tt type}(U,T)-1$.  ${\tt type}(V,T(U)) = ({\tt
type}(V,T)+1) \cap {\tt type}(V,U)$.  It should be clear that ${\tt
type}(,T)$ is always either $\cal Z$ (if $U$ does not occur in $T$),
a singleton set $\{n\}$ (in which case we say that $U$ has type $n$ in
$T$), or $\emptyset$ (in which case we say that $U$ is ill-typed in
$T$).

\item[Abstraction Theorem:] If $0 \in {\tt type}(x,T)$, then there is
a term $(\lambda x.T)$, not containing the variable $x$, such that
$(\lambda x.T)(x) = T$ is a theorem.

\item[Proof:] For any term $T$, we can construct a term $T'$ which
contains no subterm of the form $|T(U)|$ by repeated rewriting using
the Constant Expansion axiom.

For any terms $T$ and $U$, we define a term $(\lambda U.T)$ as
follows: if $T=U$, define $(\lambda U.T)$ as ${\tt Id}$.  If $T$ is an
atomic term or constant function term distinct from $U$ define
$(\lambda U.T)$ as $|T|$.  If $T$ is a composite term $V(W)$ define
$(\lambda U.T)$ as ${\tt Abst}(\lambda |U|.V)(\lambda U.W)$.

Induction on the structure of terms, along with the defining axioms
for the combinators, establishes that $(\lambda U.T)(U)=T$ is a
theorem for any $T$ and $U$.  Our further claim is that if $T'$ is
chosen (as it always can be) such that $T=T'$ is a theorem and $T'$
contains no subterm of the form $|V(W)|$ and if ${\tt type}(x.T)$
contains 0, and if $U$ is either atomic or an iterated constant
function of an a tomic term, then $(\lambda U.T')$ (which is already
seen to satisfy $(\lambda U.T')(x) = T' = T$) will further not have
$U$ as a subterm.

We run through the induction again.  If $T'=U$, then $(\lambda
U.T')={\tt Id}$ contains no occurrence of $U$.  If $T'$ is an atomic
term distinct from $U$, then $(\lambda U.T')=|T'|$ contains no
occurrence of $U$.  If $T'$ is a constant function, it must be an
iterated constant function of an atomic term.  If this term is $U$,
then ${\tt type}(U,T)$ is the singleton set of a negative integer, and
so does not contain 0.  If this term is not $U$, then $(\lambda
U.T')=|T'|$ does not contain an occurrence of $U$.  If $T'$ is of the
form $V(W)$ then $(\lambda U.T')={\tt Abst}(\lambda |U|.V)(\lambda
U.W)$ and we need to argue that neither $(\lambda |U|.V)$ nor
$(\lambda U.W)$ contain $x$ if ${\tt type}(U,T)$ contains 0.  We must
have 0 in ${\tt type}(U,W)$, so induction tells us that $(\lambda
U.W)$ contains no occurrence of $U$.  We must have $-1$ in ${\tt
type}(U,V)$.  Any occurrence of $U$ in $V$ must be in the context
$|U|$, and all these occurrences of $|U|$ must have type 0 in $V$.
From this it follows that $(\lambda |U|.V)$ contains no occurrence of
$|U|$ by induction, and so can contain no occurrence of $U$, since any
occurrence of $U$ in $(\lambda |U|.V)$ would need to be derived from
an occurrence of $U$ in $V$, which would appear as part of a subterm
$|U|$ eliminated by the abstraction algorithm.

\end{description}

We introduce a weak extensionality axiom.  The motivation is that we allow
rewrites inside abstraction terms $(\lambda x.T)$.

\begin{description}

\item[Extension Axiom:] If $T=U$ is a theorem, so is $(\lambda x.T) =
(\lambda x.U)$.  Further, $(\lambda x.T(x)) = T$, where $T$ is of one
of the forms ${\tt Abst}$, ${\tt Abst(U)}$.

\item[Refinement of the Abstraction Algorithm:] The abstraction
algorithm can be modified as follows: when it happens that $(\lambda
|U|.V) = |V|$ and $(\lambda U.W) = |W|$, define $(\lambda U.V(W))$ as
$|V(W)|$ (this is equal to the abstraction obtained under the original
algorithm by the axiom of Constant Expansion); define $(\lambda
x.U(x))$ as $U$ in case $U$ is of one of the forms ${\tt Id}$, $|V|$,
${\tt Abst}$, ${\tt Abst}(V)$ or ${\tt Abst}(V)(W)$ (justified by the
Extension Axiom), and otherwise define it as usual.  Further, replace
$(\lambda x.T(x)$ with $T$ in the contexts $U$, $V$ in terms ${\tt
Abst}(U)$, ${\tt Abst}(U)(V)$ (also justified by the Extension Axiom).

[Internally to CL Watson, an ``inertia operator'' is applied to $U$
when it is chosen as $(\lambda x.U(x))$: this helps make the reduction
algorithm inverse to the abstraction algorithm in case $U$ itself is a
``lambda-term''.]

The CL Watson logic of equality will be discussed as it is
implemented.  The intention is that ${\tt Eq}(|||T|||)(||U||)(|V|)(W)$ (which actually can be written {\tt V \{T = U\} W} in CL Watson infix notation, though it will initially be written {\tt V \{||T|| = ||U||\} W} )
means ``if $T=U$ then $V$ else $W$''.  The axioms are

\begin{description}

\item[True Equation:] $${\tt Eq}(|||X|||)(||X||)(|Y|)(Z) = Y$$

\item[False Equation:] $${\tt Eq}(|||{\tt Id}|||)(|||{\tt Id}|||)(|Y|)(Z) = Z$$

The nesting of constant functions obscures the fact that this axiom
asserts that {\tt Id} and {\tt |Id|} are distinct.  This is a generic
choice for drawing a distinction because these are in effect the
projection operators: ${\tt Id}(|X|)(Y) = X$ and ${\tt
|Id|}(|X|)(Y)=Y$.

\item[Application Distribution:] $${\tt Eq}(|||T|||)(||U||)(|A(B)|)(C(D))$$ =
$${\tt Eq}(|||T|||)(||U||)(|A|)(C)({\tt Eq}(|||T|||)(||U||)(|B|)(D))$$

\item[Constant Function Distribution:] $${\tt Eq}(|||T|||)(||U||)(||A||)(|B|)$$ =
$$|{\tt Eq}(|||T|||)(||U||)(A)(B)|$$

\item[Substitution under Hypotheses:]  $${\tt Eq}(|||T|||)(||U||)(|T|)(B)$$=
$${\tt Eq}(|||T|||)(||U||)(|U|)(B)$$

\item[Weak Extensionality/Choice:] $${\tt Eq}(|||(\lambda x.T(x))|||)(||(\lambda x.U(x))||)(|V|)(W)$$=
$${\tt Eq}(|||T({\tt Diff}(|T|)(U)(X))|||)(||U({\tt Diff}(|T|)(U)(X))||)(|V|)(W)$$

\end{description}

\end{description} 

\section{The Program}

\subsection{The Term Language}

Constant atomic terms of the language of CL Watson are either strings
of letters with the initial capitalized and any others lower case
({\tt A} is a term) or strings of special characters (precise definition is
the function {\tt isspecial} in the source).

Variable atomic terms of the language of CL Watson have the letter {\tt x}
followed by a numeral.

If {\tt T} is a term and {\tt U} is a term then {\tt |T|} and {\tt T(U)}
are terms.  These constructions correspond to the application and constant
function constructions of the logic.

CL Watson supports infix notation.  {\tt T U V} is syntactic sugar for
{\tt U(|T|)(V)} (this resembles the usual ``currying'' used to
implement functions of two variables in combinatory logic, but notice
the adjustment of the type of {\tt T} so that it is the same as the
type of {\tt U}).  If {\tt T} is an infix term, it will be enclosed in
parentheses.  If {\tt U} is an infix term, it will be enclosed in
braces.  If {\tt T} is an infix term in the context {\tt T(U)}, it
will be enclosed in braces.  Redundant braces and parentheses are all
right.  This all has the effect of enforcing APL operator precedence:
we expect to eventually implement some sort of user-defined operator
precedence.  CL Watson draws an internal distinction between constant
function constructors explicitly supplied by the user and those which
appear in the internals of infix notation, so as to avoid terms
acquiring unexpected infix forms.

CL Watson supports lambda-notation.  If {\tt x1} represents $x$ and
{\tt T} represents {\tt T}, then {\tt [x1=>T]} represents $(\lambda x.T)$.
Note that the internal representation of {\tt [x1=>T]} contains no variable
analogous to {\tt x1}, and the bound variable displayed by the system when
displaying a term may be different from the bound variable the user enters.

CL Watson displays {\tt F([x1=>T])} as {\tt [F(x1)=>T]} where F is an atomic
constant (in order to support standard notation for quantifiers and
other binders).  This sometimes has odd effects where atoms not
normally thought of as binders are applied to abstractions.  CL Watson
will sometimes also display terms in a way involving infix notation
``accidentally''.

One must be careful with whitespace: spaces actually have meaning in
the syntax of CL Watson (they separate infixes from neighboring
terms).  In certain cases CL Watson may supply these spaces when they
are omitted (next to special characters; never before an opening
parenthesis).  It is safer to write them and they are always
displayed.  Redundant spaces in locations not next to infixes should
always be avoided.

A very new feature is the presence of ``quoted'' terms {\tt "T"},
which represent interpretations of terms in a model of the logic.
These are intended for use in supporting unstratified quantification.

\subsection{The Proof Model}

The model of prover use is this: the user enters a term, applies
theorems to it as rewrite rules until a new term is obtained, then
records the equation of the original term and the final term as a new
theorem usable as a rewrite rule.

At any point, one views the right side of the theorem under construction,
and a selected subterm where rewrite rules will be applied.

The command

{\tt Start "T"}

sets up a new proof environment with the term {\tt T} as both left and
right term in the ``theorem under construction''.  {\tt s} is an
abbreviation for {\tt Start}.

The command 

{\tt Look()}

allows one to see the right side of the current term.  Variants {\tt
Look1()} and {\tt Look2()} support different term views.  View 1
supports neither abstraction nor infix notation.  View 2 supports
abstraction but not infix notation.  Commands {\tt view1()} and {\tt view2()}
will reset the default view to view 1 or 2; they also change the effect
of the navigation commands.  {\tt view3()} restores default behavior.

The commands

{\tt workback()}

interchanges the left and right sides of the theorem under construction.

The command

{\tt lookback()} 

allows one to look at the left side of the theorem
under construction and return to the right side.

The command

{\tt startover()}

sets both left and right sides of the theorem to the current left side.

Now we consider navigation commands, which reset the current subterm.

{\tt right()}

{\tt left()}

These commands are multipurpose.  They will go to the right or left
subterm of a function application term, to the right or left subterm
of an infix term (leaving aside the infix).  They will go to the body
of an abstraction (that is,to a formal value of the abstraction term).

{\tt middle()}

{\tt function()}

These commands are dedicated to infix terms.  {\tt middle()} goes to
the infix subterm of an infix term.  {\tt function()} goes to the left
subterm of an infix term when it is considered as a function
application term (the subterm {\tt U(|T|)} which disappears from view
when {\tt U(|T|)(V)} is presented as {\tt T U V}).  The use of the
{\tt function()} command can be avoided by using {\tt toggleinfix()}
to change the way the term is viewed.

{\tt value()}

This goes to the formal value of any ``abstract'' term:  it works where
an ``abstract'' is not actually presented in abstract form (this is true
of terms {\tt Id}, {\tt |T|}, and {\tt Abst}, and may be true of more terms
in later forms of CL Watson).  

Terms dedicated to infix terms will not work in view 2; terms dedicated to
abstraction terms will not work in view 1.  In view 1 or 2, {\tt right()}
and {\tt left()} will not do the same thing as in view 3.  Any movement
command at all takes one to the interior of the quotes in a quoted term.

{\tt up()}

{\tt top()}

The {\tt up()} command reverses the effect of the most recent movement command;
the {\tt top()} command takes one to the top of the term.

We now consider the application of theorems.

If {\tt T} and {\tt U} are terms, then {\tt T :>: U} and {\tt T :<: U}
are terms, with the same reference as {\tt U}.  The effect is to
signal the intent to apply a rewrite represented by the term {\tt T};
the alternative form applies the theorem in the opposite direction.
The head of the term {\tt T} will be the name of a theorem (or
theorem-like internal function of the prover); the rest of the term can
supply parameters to the theorem.

The command {\tt ri "T"} introduces an embedded theorem {\tt T} at the
current selected subterm.  The command {\tt rri "T"} introduces an embedded
theorem using the alternative infix {\tt :<:}.

The command {\tt Execute()} carries out all rewrites signalled by
embedded theorems in the selected subterm.  Rewrites are carried out in
a depth-first manner, and new rewrites introduced by executed rewrites are
carried out in their turn until all are eliminated.  {\tt ex()} abbreviates
{\tt Execute()}.

The command {\tt Onestep()} carries out a single step of the reduction
process which is carried out aggressively by {\tt Execute()}.

The command {\tt assign1 n "T"} implements a global substitution of
{\tt T} for the variable {\tt xn} (n is a numeral) in both the left
and right sides of the current theorem under construction.  The
command {\tt assigninto n "T"} substitutes the left and right sides of
the current term for {\tt xn} in {\tt T} to get new left and right
sides.

The command {\tt Extract "T"} converts the selected term {\tt U} to
{\tt [T=>U](U)}, where possible.  The command {\tt Red()} is the
reverse operation of beta-reduction.

The command {\tt makeinert()} makes the selected term ``inert'' (this
is a technicality in the reduction algorithm).  The command {\tt
toggleinfix\/} will toggle the display of a potential infix term
between the forms {\tt T(|U|)(V)} and {\tt U T V} (it resets the
hidden bit in the representation of the constant function between
user-supplied and implicit).

The command {\tt prove "T"} proves a new theorem, with left side the left side
of the current theorem under construction and right side the right side of the
current theorem under construction.  The head of the term {\tt T} will be
the name of the new theorem:  it may have arguments which can be used to pass
information to the theorem.  In examples, we will see that theorems may have
quite complex execution behavior, using built-in higher-order matching
facilities, information passed to them in parameters, and introducing new
embedded theorems which will in their turn be executed.

The command {\tt axiom "name" "T" "U"} introduces an axiom named {\tt name}
asserting {\tt T = U}.

The command {\tt reflect "name1" "name2"} converts a theorem called
name1 of the form {\tt "T" = "U"} to a theorem called name2 (if this name
is free) of the form {\tt T = U}:  it allows theorems proved about the object
model to be exported to the meta-model.  This is logically strong:  we'll
see if it is useful.

The command {\tt thmdisplay "T"} displays the theorem with name {\tt T}.

Further commands analogous to those in the Watson user manual may be expected
to appear.

\section{Examples}

\end{document}
