\documentclass[12pt]{article}

\title{The Strictly False Programming Language}

\author{M. Randall Holmes}

\begin{document}

\maketitle

\section{Introduction}

\subsection{Version}

This is dated September 12, 2005, correcting some errors (notably,
{\tt m} and {\tt M} were swapped) and adding
some features (notably {\tt Z}) introduced in that version.

\subsection{Introduction}

This document is the reference for the Strictly False programming
language, which is an extension of the elegant False language proposed
by Wouter van Oortmerssen (see {\tt http://wouter.fov120.com/false/}
for an account of this language), which is itself a variant of Forth.

Van Oortmerssen says ``I designed this language with two particular
objectives: confusing everyone with an obfuscated syntax, and
designing an as powerful language as possible with a tiny
implementation: in this case a compiler executable of just 1024 bytes
(!), written in pure 68000 assembler.''

In our opinion the syntax, while certainly compact, is hardly
obfuscated (though certainly confusing to the uninitiated): it is
simply reverse Polish notation, as found in Forth or on an HP
calculator, and compactness is enforced by the fact that every command
is a single character (with the caveat that there are really some
two-character commands, at least in our opinion).  In False there are also
some brackets.  The language is extremely easy to parse (for a computer).

This said, it must be admitted that the language is rather hard to
read for the uninitiated.  We think that the extreme compactness has
at least one sensible application: we plan to implement this language
on our handheld, where keystrokes are annoying.

Forth (and False, and Strictly False) are based on the use of a stack.
The + command of any of these languages, for example, has the
specification ``take two numbers off the stack, add them, and put the
result on the stack''.  A data item like 2 also has execution
behavior: ``put 2 on the stack''.  The program {\tt 2 3 +} then reads
``put 2 on the stack (stack is now $[2,\ldots]$); put 3 on the stack
(stack is now $[3,\dots]$); take the top two items off the stack and
add them, then put the result back on the stack (stack is now
$[5,\ldots]$).

The program itself can be thought of as a stack: execution consists of
popping commands one by one off the program stack and making the
appropriate changes to the data stack.  False takes this viewpoint and
introduces a new twist.  Consider the state of the machine with [3,+]
in the program stack: this is a function with one argument, which may
briefly be summarized as ``add 3''.  The refinement which produces
False (on top of a set of Forth-style stack commands with
one-character opcodes) is the ability to introduce objects like {\tt
[3+]} (program stack states) as data objects which can be put on the
stack!  A new opcode {\tt !} allows these functions to be executed:
for example, the program {\tt 2[3+]!} works as follows: ``put 2 on the
stack (stack is now $[2,\ldots]$); put {\tt [3+]} on the stack (stack
is now $[2,[3,+],\ldots]$); apply ({\tt 3+} is added to the beginning of
the program stack; stack is now $[2,\ldots]$); put 3 on the
stack(stack is now $[3,2,\ldots]]$; add (stack is now $[5,\ldots]$)''.

Of course what is actually popped onto the stack when {\tt [3+]} is
put there is the address of a list data structure.

Our reaction to the False language was of course first to marvel at
its obfuscated nature just like anyone else.  But the compact
representation of higher-order objects (functions of any number of
arguments) allowed by False is of considerable philosophical interest
to us.  One point about it is that it allows a representation of
abstractions which does not necessarily involve variables, in the same
way that combinatory logic does this.

At this point, we originally wrote an extensive discussion of why one
cannot implement combinatory logic directly with False functions.  On
examining this text, I decided that if I included it at this point no
one would finish reading the document, so I deferred it to the last
section.  Briefly, one cannot abstract into False functions because
there is no way to operate on the insides of them; but False functions
are lists of commands (some of which are confused with data), so if
one adds basic operations on lists one gets the ability to
implement synthetic combinatory logic -- and to do lots more
interesting programming :-)

Strictly False adds list processing to the capabilities of False and
we see that we gain the full (theoretical) power of synthetic
combinatory logic, which is a complete functional programming language
(False is certainly a complete programming language in some senses,
but it is probably not a complete functional programming language).

Another way in which Strictly False differs from False is that it is
(dynamically) typed.  Each object is either an integer, a character, a
truth value, or a list/function, and operators expect to see arguments
of particular types.  In the original False, various cool hacks could be
obtained by type-casting:  these are forbidden here.

Strictly False has file I/O: it can read and write characters from
files and also can execute a file as a function and save a function to
a file.  It has the same function definition capability as False
(using the : and ; opcodes) but these are explicitly restricted to
functions; the capability is added of converting such defined
functions to new opcodes.  It also has the ability to address a large
supply of numerically addressed data cells (the current interpreter
has a very generous memory model, probably because I haven't yet
considered implementing it with real memory allocation).

Strictly False supports continuation passing (?!?).

The current implementation is an interpreter running under Standard ML;
we hope to implement it in C (for efficiency) and then in some language
implement it for our handheld device.

\section{The Implementation in ML}

We assume familiarity with ML.  If you have compiled, loaded, and opened
{\tt purefalse.sml}, then to execute a Strictly False program {\tt <program>}
you should type

{\tt execute "<program>"}

at the ML prompt.

It is useful to be aware that

{\tt execute "T"}

toggles the trace feature and that

{\tt execute "nd"}

clears the stack, which persists.

{\tt executelines();}

allows you to write many lines of literal text to be fed to the
interpreter, stopping when you enter an empty line.  This allows you
to avoid ML escape sequences for \begin{verb} \ \end{verb}, 
\begin{verb} " \end{verb}, and perhaps other characters, and it also allows one to enter
carriage returns inside messages, a peculiar feature of False which is
inherited by Strictly False.

All these commands need to be typed at the ML prompt:  if the trace
feature is on, you need to hit return many times before you get back to the
ML prompt!  The system may also stop for input (and SF does not have
an input prompt:  you are encouraged to use a message for this purpose!)

\section{Command Reference}

\subsection{Parsing the Language}

Most commands in Strictly False are single characters.  There
are two special two-character forms:

\begin{description}

\item[Characters:] \begin{verb} 'x \end{verb} represents the literal
character following the single quote.  The effect of executing this
is to push the character onto the stack.

\item[Atomic Programs:] \begin{verb} `x \end{verb} stands for the
atomic program \begin{verb} [x] \end{verb} (where x is a single
character).  It's really only present so that the bracket notation is
eliminable in principle.

\end{description}

There are four multi-character notations.

\begin{description}

\item[Comments]  \begin{verb} {...} \end{verb} is a comment.  The
only restriction on the text represented by the ellipsis is that curly
brackets are balanced: comments can be nested.  Comments are
completely ignored by the interpreter.

Curly brackets have another function: if you see an expression in curly
brackets in the display of the data stack it indicates that this is a
literal piece of program text embedded in the term.  Be aware that
program text dropped onto the data stack is executed immediately.
Functions dropped onto the data stack have most program text
eliminated and replaced with the corresponding data (there are some
exceptions).  A single character opcode is always a piece of program
text, and this is not indicated in the display.  Messages in displayed
functions are always enclosed in curly braces: do not do this in your
programs!  The P command (see below) loads the current continuation as
an unanalyzed chunk of program text.

\item[Messages]  \begin{verb} "..." \end{verb} is a message.  The effect
of executing this is to send the text represented by the ellipsis to
standard output.  There is no effect on the stack.  The only restriction
on the text is that it may not contain double quotes (it may contain
carriage returns, though this is difficult to exploit in this interpreter).
Messages embedded in functions on the data stack are always program
text, and so are enclosed in braces as well.

\item[Functions/Lists]  \begin{verb} [...] \end{verb} is a function
or list.  The text must be a valid Strictly False program (the
interpreter does not check for the validity of opcodes, just for the
syntax of multi-character forms).  The effect of executing the
bracketed form is simply to put it on the stack (like any other data).
If the !  opcode (application) is run, the function on top of the
stack is executed as a Strictly False program.  The language also
supports list operations; one must be careful, though, as one may pop
out a ``data item'' with execution behavior!

\item[Numerals:] In some theoretical sense numerals can be interpreted
as made of single character opcodes and the execution of the
interpreter reflects this (it actually proceeds single character by
single character, with one-character lookahead where digits are
involved), but the internal parser views them as single chunks.  A
string of one or more digits followed by an underscore \begin{verb} _ \end{verb} is a numeral representing a negative integer.  The
underscore is also a general unary additive inverse operator.  The
only place where whitespace has a function is where it is used to mark
breaks between numerals.

\end{description}

\subsection{Opcodes (in functional categories)}

\subsubsection{Pure Stack Operations}

The ``pick'' operation of False, which was represented by a
nonstandard character, is not provided but can be implemented.

\begin{description}

\item[drop:]  \begin{verb} % \end{verb}  When this command is executed,
the top element of the data stack is dropped.  (False).

\item[swap:]  \begin{verb} \ \end{verb}  When this command is executed,
the top two elements of the stack are interchanged.  (False).

\item[duplicate:] \begin{verb} $ \end{verb}  When this command is executed,
the top element of the stack is duplicated (an extra copy is put on).  (False).

\item[rotate:] \begin{verb} @ \end{verb}  When this command is executed,
the third element of the stack is brought to the top:  if the original
order were 123 (from the top down) the new order would be 312. (False).

\end{description}


\subsubsection{Simple Data}

There are three types of simple data in Strictly False, booleans, integers,
and characters.  There is one other type of data:  programs = lists.  The data typing is a difference from False.

\begin{description}

\item[true:]  {\tt t}  When this command is executed, a copy of the value ``true'' is placed on the stack. (new in Strictly False).

\item[false:] {\tt f}  When this command is executed, a copy of the value ``false'' is placed on the stack. (new in Strictly False).

\item[digits:]  {\tt 0-9}  These execute in two different ways, depending
on whether they are followed by another digit or not. In any case,
the result is standard decimal notation for integers. (False).

\item[numerals:] A block of digits is interpreted as an integer as
usual (in base 10).  A block of digits followed (not preceded) by
\begin{verb} _  \end{verb} is interpreted as a negative integer. (False).

\item[Characters:]  As noted above, {\tt 'x} is read as the character {\tt x}
(for each possible choice of character.) The effect of executing this is
to put a copy of the character on the stack. (False).

\item[Atomic Programs:]  For any character x, the notation `x is an alternative
notation for the program [x].  (new in Strictly False; ` means something quite different in False).

\end{description}
\subsubsection{Simple Input and Output}

\begin{description}

\item[Messages:] {\tt "..."  q r } As noted above, any string of
characters enclosed in double quotes (excluding double quotes but
allowing carriage returns) has the effect when executed of sending the
enclosed character string to standard output.  {\tt q} sends a double
quote to standard output.  {\tt r} sends a carriage return to standard
output.  (quoted strings as in False; q and r are new in Strictly
False).

Messages embedded in functions on the data stack are program text,
and so are displayed in braces.

\item[Character input:] \begin{verb} ^ \end{verb}  When executed, this
command accepts character input.  Input is buffered, so you can enter
many characters and the program will take them as it needs them.  No
prompt is provided:  you should use a message for this purpose.  (False).

\item[Character output:] comma({\tt ,})  When executed, a character
is removed from the top of the stack and printed.  Type errors stop
the program.  (False).

\item[Integer output:] period({\tt .})  When executed, an integer is
removed from the top of the stack and printed.  Type errors stop
the program.  (False).

\item[Flush buffers:]  The close quote {\tt )} flushes the interpreter's
input buffer and ML's output buffer (in this implementation).  (This
is different from False in using a standard character.)

\end{description}
\subsubsection{Operations on Simple Data}
\begin{description}

\item[Arithmetic Operations:] {\tt + - * / } Each of these takes two
integers off the stack and puts the result of applying the operation
(with the second item on the stack being the first argument) back on
the stack.  All calculations are mod a large number (200000000 in this
implementation) with the larger half construed as negative numbers (subtract
200000000 from numbers greater than 100000000); there are no overflow
crashes, though results may be unexpected.  Division by zero stops
the program.  Type errors stop the program.  (False:  details of our
arithmetic that avoid overflows are different, I assume).

\item[Comparison:] {\tt = < > }  Each of these takes two items from
the stack (both integers or both characters) and compares them in the
indicated way, with the second item on the stack being the first
argument.  A truth value is placed on the stack.  Type errors stop
the program.  {\tt =} is also supported for lists/programs, but
it behaves differently; see the entry below.  (False:  I don't know that
equality of lists is supported).

\item[Type Casting:] {\tt c C} The {\tt c} command takes a character
off the stack and replaces it with an integer or vice versa.  An
integer will be processed mod 256 (no crashes).  Type errors crash the
program.  (new in Strictly False: not needed in False).

The {\tt C} command similarly converts characters to atomic programs
({\tt 'x} to {\tt `x}) and vice versa.  (new in Strictly False).

\item[Propositional Logic:] \begin{verb} ~ \end{verb} \begin{verb} & \end{verb} \begin{verb} | \end{verb}  These commands take one
or two truth values off the stack and replace them with the result of
the appropriate operation: negation, and, (inclusive) or respectively. (False)
\end{description}


\subsubsection{Operations on Lists/Programs}

All commands except ! (function application) are new in Strictly False.
I do not know whether False supports equality of lists.

\begin{description}

\item[Brackets:] {\tt [...] }  Entering a bracketed Strictly False
program simply places a copy of this program/list on the stack. {\tt []}
has the same effect as {\tt n}.

\item[Empty list/program:]  {\tt n x }  The {\tt n} opcode puts
an empty list {\tt []} on the stack.  The {\tt x} opcode puts {\em t}
on the stack if the list on top of the stack is empty and {\tt f}
if it does not; it does {\bf not} remove the tested list!
Type errors stop the program.

\item[Cons:]  {\tt p} When executed, removes the first item (which
must be may be of any type) and the second item (which must be a list) from
the stack and replaces them with the list with the first item
as head and the second as tail.

\item[Composition:]  {\tt o} When executed, removes two lists from the
stack and appends the second item to the first item (this is list
concatenation).

\item[Head:] {\tt i} When executed, removes a list from the top of the
stack, pushes the head of that list onto the stack (or executes the
head of the list if it is an opcode or program text) then pushes the
tail of the list back onto the stack.  Repeated execution of i traces
the execution of a function on top of the stack step by step.

\item[Inert Head:]  {\tt j} When executed, removes a list from the top of
the stack, puts the one-item list containing its head on the stack
then puts its tail on the stack.  This averts possible execution
behavior!

\item[Function application:]  {\tt !}  When executed, causes the
list on top of the stack to be executed as a program.

\item[List equality:]  {\tt =} This is supported with some misgivings.
It does {\em not} remove the two lists compared from the stack; it just
adds an appropriate truth value to the top of the stack.

\end{description}

\subsubsection{Control Structures}

Both of these are as in False.

The ! operator above and the P and D operators below are also
control operators.
\begin{description}
\item[Conditional:]  {\tt ?} Pops two items off the stack, a function (top) and
a truth value.  Executes the function just in case the truth value is {\tt t}.

\item[Loop:]{\tt \# }Pops two functions off the stack:  the top item (second argument)
is the body of the loop and the second item (first argument) is the
test.  Executes the test: there must be a truth value on top of the
stack at the end of each execution of the test (or the program stops
with a type error).  As long as the truth value is true, the body
continues to be applied.

\end{description}

\subsubsection{Function Definitions}

\begin{description}
\item[Function definition:]  {\tt :} pops a character (top)
and a function (second item, first argument) off the stack and binds
that function to that argument (only functions may be defined).  Use
{\tt ;} to exploit this binding.  Note that typically this will look
like {\tt 'x:} not {\tt x:} as in False.  (False, except only functions
may be bound to a character and the argument really is a character).

\item[Use of function definition:] {\tt ;} pops a character off
the stack and looks up the function bound to that character: this
function is immediately executed (not as in False, where any kind of
data can be bound using {\tt :}).  Note that typically this will
look like {\tt 'x;}. (False, except that the argument is a character
and it incorporates the effect of !).

\item[Viewing a defined function:] {\tt E} pops a character off the
stack and puts the function bound to it on top of the stack (as {\tt ;}
except inert).  This is actually the False version of {\tt ;}, I
believe.

\item[Defining Opcodes:] {\tt B} (not really intended to be used
inside a program): takes a character off the stack, and defines the
character (if there is no conflict) as a new opcode with the effect of
the defined function.  {\tt 'x;} will be replaced with {\tt x} as
appropriate inside the text of the function to enable recursion to
work.  The evil function x which makes essential use of {\tt 'x:}
inside its text will not be satisfactorily simulated.  There is no
user command to remove such a binding!  (New in Strictly False).

A hacking opportunity:  at the moment conflicts with built-in
opcodes are not recognized, so one could define a function to extend
the behavior of an existing function (into cases where the original
function stops with a type error).  I may fix it so this can't be done.
\end{description}
\subsubsection{Machine Access}

These are dangerous commands which can produce dazzling special
effects.  They are all new in Strictly False.

\begin{description}
\item[Data Stack Access:] {\tt s S d }  The relatively harmless
{\tt s} tests whether the data stack is empty.  The more exciting
{\tt S} loads the data stack onto the data stack as the top element.
The aggressive {\tt d} replaces the data stack altogether with the
list which is the top item of the stack (so {\tt nd} erases the data
stack completely).  Creation of a data stack with executable program
text or opcodes appearing after data causes special effects:  when
data is popped off, executable text revealed will be executed.
Although I haven't tried it yet, it ought to be possible to create
a data stack which is effectively an infinite lazy list.

\item[Program Stack Access:] {\tt P D} The {\tt P} command loads the
remaining portion of the current program yet to be executed (the
continuation) onto the top of the data stack.  This is loaded as a
chunk of unanalyzed program text, and so it is displayed in braces.
The {\tt D} command replaces the program being executed with the
function on top of the data stack.  So {\tt nD} is {\bf stop}, and the
combined use of {\tt P} and {\tt D} should allow continuation passing!
\end{description}
\subsubsection{The Memory Array}

These commands are new in Strictly False.

There is a memory array indexed by all available integers.  Cells must
be initialized before being used, and in fact local declarations
are supported (after a local declaration is undone, earlier values
become available).  Each memory cell is actually a stack, whose
top element is the current value in that cell.
\begin{description}
\item[Initialize:]  {\tt I} Takes two items off the data stack, the second of which
is an integer index of a cell.  Pop the top item from the data stack
onto the indexed memory cell stack:  this is declare and initialize a variable.

\item[Deallocate:] {\tt e} Takes one item off the data stack, the integer
index of a memory cell.  Pop off the top item in the stack associated
with that memory cell.  The effect is to free a variable (and perhaps
to make a variable with the same index declared earlier visible
again). If the memory cell stack is empty, the interpreter stops with
an error.

\item[Assignment:] {\tt A} Takes two items off the data stack, the first
being a value to be assigned and the second the integer index of a
memory cell.  Pop off the first item in the memory cell's stack and
push the ``value to be assigned'' taken from the data stack onto the
memory cell stack.  The effect is to assign the value to the indexed
variable.

\item[Reference:] {\tt a} Takes one item off the data stack, an
integer indexing a memory cell, and puts the top item in the stack in
the indexed memory cell on top of the data stack.  The effect is to
get the current value of the indexed variable onto the stack.
\end{description}
\subsubsection{File Input and Output}

These commands are new in Strictly False.  Note that in all file commands
the fileID (a character bound to the file as an identifier) is always the
argument next to the opcode.

\begin{description}
\item[Open a file:]  {\tt O} removes all characters from the stack
and reads the first one as a file designator and the rest as a filename,
with the second character on the data stack being the last character in the
file name.  This file is opened for input and output and bound to the
top character as an identifier (fileID).

\item[Open a file for read only:]  {\tt Z} is as {\tt O}, except that
the file is opened to read only.  The interpreter will prevent writing
operations to files opened with {\tt Z} (by fatal error).

\item[Close a file:] {\tt F} Takes a single character off the stack and closes
the file for which it is fileID for both input and output.  If no file
is found, fatal error.

\item[Read a character from a file:]  {\tt R} Takes a character (fileID)
off the stack and attempts to read a character from that file.
It puts a truth value (true if it read the character) on the stack
then puts the character read on the stack if there is one.  If the
fileID does not refer, the interpreter stops with an error.

\item[Write a character to a file:] {\tt W} Takes two characters off the stack:
writes the second item (first argument) to the file indicated by the
first item (second argument) (error if fileID does not refer).

\item[Write a function to a file:] {\tt m} takes a character off the
stack (fileID) then a function, and writes the function to a file (the
contents of the file will not include the brackets around the
function).

\item[Execute a function from a file:]  {\tt M} takes a character off the stack
(fileID) and executes the function written there.
\end{description}


\subsubsection{Wish List}

I'd like to be able to save file information in a memory cell; this would
need to be another data type (file handles).  The command to retrieve a
file handle would supply the fileID to which to bind it.

I'd like to be able to edit defined functions: I think that all I need
to do is be able to bring it to the top of the stack without executing
it (now implemented by {\tt E}), and then a Strictly False line editor
can be written :-)

In an implementation where memory is really being manipulated, the {\tt I}
command should probably have a parameter to indicate the length of
a block of memory cells to be allocated; when I write such an implementation,
I will probably change the spec of the language (and change this
interpreter) to do this:  for the moment, cells have to be declared one
by one anyway, because the implementation of the array is stupid :-)

\subsubsection{Debugger Commands}

These are provided with a mind to making the interpreter its
own development environment.
\begin{description}
\item[Trace feature:]  {\tt T} toggles the trace feature on and off.

\item[Display data stack:]  
{\tt U} sends an image of the data stack to standard output.

\item[Display continuation:]  {\tt V} sends the current continuation
(rest of the program to be executed) to the standard output.
\end{description}

\section{Implementation of Combinatory Logic}

Let a function {\em f} of one variable be considered as a False
program which pops a value $x$ off the stack and replaces it with
$f(x)$.

In combinatory logic, we construct for any expression $T$ in variables
and application, a function $(\lambda x.T)$ such that $(\lambda
x.T)(x) = T$ for all $x$.  If $T=x$, we define $(\lambda x.T)$ as I,
where $I(x) = x$ for all $x$.  If $T=a\neq x$, we define $(\lambda
x.T)$ as $K[a]$, where $K[a](x)=a$ for any $x$.  If $T=U(V)$, we
define $(\lambda x.T)$ as $S[(\lambda x.U),(\lambda x.V](x)$ (where we
may suppose that $(\lambda x.U)$ and $(\lambda x.V)$ have already been
defined, by structural induction on the language of terms built up
from variables by application).

$I$ is implemented by a False command which does nothing ({\tt \$\%},
``duplicate then drop'', will do the trick), so {\tt x[\$\%]!} will
execute as desired, leaving just $x$ on the stack.

$K[a](x) = a$ tells us that a program $P$ implementing $K[a]$ should
have the effect of removing the top item $x$ from the stack and
replacing it with $a$: the program {\tt [\%a]} has this effect:
{\tt x[\%a]!} puts $a$ on the stack.

$S[a,b](x) = a(x)(b(x))$ is the behavior required for $S[a,b]$
to work as advertised. \begin{verb} [$b!\a!!] \end{verb} is the program with the desired
effect, where \begin{verb} $ \end{verb} is the program which duplicates the top of the stack,
\begin{verb} \ \end{verb} is the program which swaps the top two elements of the stack,
({\tt \%} (seen above in K) is the program which drops the top element
of the stack and {\tt !} is the function application program).

This means that any expression written in the language of function
application and concrete functions can be abstracted from in False.  But notice
that we cannot abstract into our representations of $K[a]$ and
$S[a,b]$ themselves, since they involve the further construction of
False functions (lists of commands).  

We observed that $K[a]$ is represented by \begin{verb} [%a] \end{verb}
and considered that it is obvious how to construct this: the function
is a list: add \begin{verb} a \end{verb} to it at the front, then add
\begin{verb} % \end{verb} to it.  In other words, since we have
``functions'' in False which are clearly list data structures, add
the operations appropriate to lists to the language.

There is a technical problem which immediately presents itself: we
can't add \begin{verb} % \end{verb} onto the list at all, because it
has execution behavior: we are presumably pushing one data item on the
stack into another, and x (presumably data) can be there without
difficulty, but we can't put \begin{verb} % \end{verb} on the data
stack except in the packaged form \begin{verb} [%] \end{verb}.  So two different flavors of
push are provided: the program p adds the data item which is second on
the stack to the front of stack (list or function) on top of the stack, while
the program o composes the two stacks on top of the stack: to prepend
\begin{verb} % \end{verb} to the stack, push the data item
\begin{verb} `% \end{verb} (an abbreviation for
\begin{verb} [%] \end{verb}; the \begin{verb} ` \end{verb} notation
makes it so that the bracket notation is in principle eliminable in
favor of one-character opcodes).  To start constructing any list,
we need the null list (opcode {\tt n}).  So K is implementable in
our extended dialect as \begin{verb} [n\p`%o] \end{verb}:  pop the null
list onto the stack, then push our single argument
\begin{verb} x \end{verb}  into it (\begin {verb} \ \end{verb} puts 3 and the null list in the correct order for {\tt p})  creating
\begin{verb} [x] \end{verb} at the top of the stack, then push
\begin{verb} [%] \end{verb} onto the stack (this will
safely sit there rather than do anything) and compose the two (obtaining
\begin{verb} [%x] \end{verb}).

Let's look at S.  \begin{verb} [$b!\a!!] \end{verb} is our implementation
of $S[a,b]$, and we want to be able to read it as $S(a)(b)$.

\begin{verb} [n[!!]oap[!\]o\p[$]o] \end{verb} is $S(a)$ (this just builds the
string of commands representing $S[a,b]$ if b is on top of the stack;
it is easier to follow if you realize that it has to be built back to
front.  And, similarly, 
S is \begin{verb} [n[p[!\]o\p[$]o]o\p[n[!!]o]o] \end{verb}.  Wasn't that easy? [I really
have tested this; it runs (and it definitely required debugging)!]


\end{document}











